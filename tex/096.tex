
\chapter{寇員外喜待高僧 唐長老不貪富貴}

\begin{quote}
色色原無色,空空亦非空。靜喧語默本來同,夢裡何勞說夢。有用用中無用,無功功裡施功。還如果熟自然紅,莫問如何修種。
\end{quote}

話表唐僧師眾,使法力阻住那布金寺僧。僧見黑風過處,不見他師徒,以為活佛臨凡,磕頭而回不題。

他師徒們西行,正是春盡夏初時節:
\begin{quote}
清和天氣爽,池沼芰荷生。
梅逐雨餘熟,麥隨風裡成。
草香花落處,鶯老柳枝輕。
江燕攜雛習,山雞哺子鳴。
斗南當日永,萬物顯光明。
\end{quote}

說不盡那朝餐暮宿,轉澗尋波。在那平安路上,行經半月。前邊又見一城垣相近。三藏問道:「徒弟,此又是甚麼去處?」行者道:「不知,不知。」八戒笑道:「這路是你行過的,怎說不知?卻是又有些兒蹺蹊,故意推不認得,捉弄我們哩。」行者道:「這獃子全不察理。這路雖是走過幾遍,那時只在九霄空裡,駕雲而來,駕雲而去,何曾落在此地?事不關心,查他做甚?此所以不知。卻有甚蹺蹊,又捉弄你也?」

說話間,不覺已至邊前。三藏下馬,過吊橋,徑入門裡長街上,只見廊下坐著兩個老兒敘話。三藏叫:「徒弟,你們在那街心裡站住,低著頭,不要放肆。等我去那廊下,問個地方。」行者果依言立住。長老近前合掌,叫聲:「老施主,貧僧問訊了。」那二老正在那裡閑講閑論,說甚麼興衰得失,誰聖誰賢,當時的英雄事業,而今安在,誠可謂大嘆息。忽聽得道聲問訊,隨答禮道:「長老有何話說?」三藏道:「貧僧乃遠方來拜佛祖的,適到寶方,不知是甚地名。哪裡有向善的人家,化齋一頓?」老者道:「我敝處是銅臺府。府後有一縣,叫做地靈縣。長老若要吃齋,不須募化,過此牌坊,南北街坐西向東的,有一個虎坐門樓,乃是寇員外家,他門前有個『萬僧不阻』之牌。似你這遠方僧,盡著受用。去!去!去!莫打斷我們的話頭。」三藏謝了。轉身對行者道:「此處乃銅臺府地靈縣。那二老道:『過此牌坊南北街,向東虎坐門樓,有個寇員外家,他門前有個「萬僧不阻」之牌。』教我到他家去吃齋哩。」沙僧道:「西方乃佛家之地,真個有齋僧的。此間既是府縣,不必照驗關文,我們去化些齋吃了,就好走路。」

長老與三人緩步長街,又惹得那市口裡人都驚驚恐恐,猜猜疑疑的,圍繞爭看他們相貌。長老吩咐閉口,只教:「莫放肆,莫放肆。」三人果低著頭,不敢仰視。轉過拐角,果見一條南北大街。

正行時,見一個虎坐門樓,門裡邊影壁上掛著一面大牌,書著「萬僧不阻」四字。三藏道:「西方佛地,賢者愚者,俱無詐偽。那二老說時,我猶不信,至此果如其言。」八戒村野,就要進去。行者道:「獃子且住,待有人出來,問及何如,方好進去。」沙僧道:「大哥說得有理,恐一時不分內外,惹施主煩惱。」在門口歇下馬匹、行李。

須臾間,有個蒼頭出來,提著一把秤、一隻籃兒,猛然看見,慌的丟了,倒跑進去報道:「主公,外面有四個異樣僧家來也!」那員外拄著拐,正在天井中閑走,口裡不住的念佛,一聞報道,就丟了拐,出來迎接。見他四眾,也不怕醜惡,只叫:「請進!請進!」三藏謙謙遜遜,一同都入。轉過一條巷子,員外引路,至一座房裡,說道:「此上手房宇,乃管待老爺們的佛堂、經堂、齋堂。下手的,是我弟子老小居住。」三藏稱讚不已。隨取袈裟穿了拜佛,舉步登堂觀看,但見那:
\begin{quote}
香雲靉靆,燭焰光輝。滿堂中錦簇花攢,四下裡金鋪彩絢。朱紅架高掛紫金鐘,彩漆檠對設花腔鼓。幾對旛繡成八寶,千尊佛盡戧黃金。古銅爐,古銅瓶;雕漆桌,雕漆盒。古銅爐內,常常不斷沉檀;古銅瓶中,每每蓮花現彩。雕漆桌上五雲鮮,雕漆盒中香瓣積。玻璃盞,淨水澄清;琉璃燈,香油明亮。一聲金磬,響韻虛徐。真個是紅塵不到賽珍樓,家奉佛堂欺上剎。
\end{quote}

長老淨了手,拈了香,叩頭拜畢,卻轉回與員外行禮。員外道:「且住,請到經堂中相見。」又見那:
\begin{quote}
方臺豎櫃,玉匣金函。方臺豎櫃,堆積著無數經文;玉匣金函,收貯著許多簡札。彩漆桌上,有紙墨筆硯,都是些精精緻緻的文房;椒粉屏前,有書畫琴棋,盡是些妙妙玄玄的真趣。放一口輕玉浮金之仙磬,掛一柄披風披月之龍髯。清氣令人神氣爽,齋心自覺道心閑。
\end{quote}

長老到此,正欲行禮,那員外又攙住道:「請寬佛衣。」三藏脫了袈裟。才與長老見了。又請行者三人見了。又叫把馬喂了,行李安在廊下,方問起居。三藏道:「貧僧是東土大唐欽差詣寶方謁靈山見佛祖求真經者。聞知尊府敬僧,故此拜見,求一齋就行。」員外面生喜色,笑吟吟的道:「弟子賤名寇洪,字大寬,虛度六十四歲。自四十歲上,許齋萬僧,才做圓滿。今已齋了二十四年,有一簿齋僧的帳目。連日無事,把齋過的僧名算一算,已齋過九千九百九十六員,止少四眾,不得圓滿。今日可可的天降老師四位,圓滿萬僧之數。請留尊諱,好歹寬住月餘,待做了圓滿,弟子著轎馬送老師上山。此間到靈山只有八百里路,苦不遠也。」三藏聞言,十分歡喜,都就權且應承不題。

他那幾個大小家僮,往宅裡搬柴打水,取米麵蔬菜,整治齋供,忽驚動員外媽媽,問道:「是哪裡來的僧,這等上緊?」僮僕道:「才有四位高僧,爹爹問他起居,他說是東土大唐皇帝差來的,往靈山拜佛爺爺。到我們這裡,不知有多少路程。爹爹說是天降的,吩咐我們快整齋,供養他也。」那老嫗聽說也喜,叫丫鬟:「取衣服來我穿,我也去看看。」僮僕道:「奶奶,只一位看得,那三位看不得,形容醜得很哩。」老嫗道:「汝等不知,但形容醜陋,古怪清奇,必是天人下界。快先去報你爹爹知道。」那僮僕跑至經堂,對員外道:「奶奶來了,要拜見東土老爺哩。」三藏聽見,即起身下座。說不了,老嫗已至堂前。舉目見唐僧相貌軒昂,丰姿英偉。轉面見行者三人模樣非凡,雖知他是天人下界,卻也有幾分悚懼,朝上跪拜。三藏急急還禮道:「有勞菩薩錯敬。」老嫗問員外道:「四位師父,怎不並坐?」八戒掬著嘴道:「我三個是徒弟。」噫!他這一聲,就如深山虎嘯,那媽媽一發害怕。

正說處,又見一個家僮來報道:「兩個叔叔也來了。」三藏急轉身看時,原來是兩個少年秀才。那秀才走上經堂,對長老倒身下拜。慌得三藏急便還禮。員外上前扯住道:「這是我兩個小兒,喚名寇梁、寇棟,在書房裡讀書方回,來吃午飯,知老師下降,故來拜也。」三藏喜道:「賢哉,賢哉!正是:欲高門第須為善,要好兒孫在讀書。」二秀才啟上父親道:「這老爺是哪裡來的?」員外笑道:「來路遠哩,南贍部洲東土大唐皇帝欽差到靈山拜佛祖爺爺取經的。」秀才道:「我看《事林廣記》上,蓋天下只有四大部洲。我們這裡叫做西牛賀洲,還有個東勝神洲。想南贍部洲至此,不知走了多少年代?」三藏笑道:「貧僧在路,耽閣的日子多,行的日子少。常遭毒魔狠怪,萬苦千辛,甚虧我三個徒弟保護。共計一十四遍寒暑,方得至寶方。」秀才聞言,稱獎不盡道:「真是神僧!真是神僧!」

說未畢,又有個小的來請道:「齋筵已擺,請老爺進齋。」員外著媽媽與兒子轉宅,他卻陪四眾進齋堂吃齋。那裡鋪設的齊整,但見:
\begin{quote}
金漆桌案,黑漆交椅。前面是五色高果,俱巧匠新裝成的時樣;第二行五盤小菜;第三行五碟水果;第四行五大盤閑食。般般甜美,件件馨香。素湯米飯,蒸饅頭,辣辣爨爨熱騰騰,盡皆可口,真足充腸。七八個僮僕往來奔奉,四五個庖丁不住手。
\end{quote}

你看那上湯的上湯,添飯的添飯,一往一來,真如流星趕月。這豬八戒一口一碗,就是風捲殘雲。師徒們盡受用了一頓。長老起身,對員外謝了齋,就欲走路。那員外攔住道:「老師,放心住幾日兒。常言道:『起頭容易結梢難。』只等我做過了圓滿,方敢送程。」三藏見他心誠意懇,沒奈何住了。

早經過五七遍朝夕,那員外才請了本處應佛僧二十四員,辦做圓滿道場。眾僧們寫作有三四日,選定良辰,開啟佛事。他那裡與大唐的世情一般,卻倒也:
\begin{quote}
大揚旛,鋪設金容;齊秉燭,燒香供養。擂鼓敲鐃,吹笙捻管。雲鑼兒,橫笛音清,也都是尺工字樣。打一回,吹一趟,朗言齊語開經藏。先安土地,次請神將。發了文書,拜了佛像。談一部《孔雀經》,句句消災障;點一架藥師燈,焰焰輝光亮。拜水懺,解冤愆;諷《華嚴》,除誹謗。三乘妙法甚精勤,一二沙門皆一樣。
\end{quote}

如此做了三晝夜,道場已畢。唐僧想著雷音,一心要去,又相辭謝。員外道:「老師辭別甚急,想是連日佛事冗忙,多致簡慢,有見怪之意?」三藏道:「深擾尊府,不知何以為報,怎敢言怪?但只當時聖君送我出關,問幾時可回,我就誤答三年可回。不期在路耽閣,今已十四年矣。取經未知有無,及回又得十二三年,豈不違背聖旨?罪何可當?望老員外讓貧僧前去,待取得經回,再造府久住些時,有何不可?」

八戒忍不住,高叫道:「師父忒也不從人願,不近人情。老員外大家巨富,許下這等齋僧之願,今已圓滿,又況留得至誠,須住年把,也不妨事,只管要去怎的?放了這等現成好齋不吃,卻往人家化募。前頭有你甚老爺、老娘家哩?」長老咄的喝了一聲道:「你這夯貨,只知好吃,更不管回向之因,正是那槽裡吃食,胃裡擦癢的畜生。汝等既要貪此嗔痴,明日等我自家去罷。」行者見師父變了臉,即揪住八戒,著頭打一頓拳,罵道:「獃子不知好歹,惹得師父連我們都怪了。」沙僧笑道:「打得好,打得好。只這等不說話還惹人嫌,且又插嘴。」那獃子氣呼呼的立在傍邊,再不敢言。員外見他師徒們生惱,只得滿面陪笑道:「老師莫焦燥,今日且少寬容。待明日我辦些旗鼓,請幾個鄰里親戚,送你們起程。」

正講處,那老嫗又出來道:「老師父,既蒙到舍,不必苦辭。今到幾日了?」三藏道:「已半月矣。」老嫗道:「這半月算我員外的功德。老身也有些針線錢兒,也願齋老師父半月。」說不了,寇棟兄弟又出來道:「四位老爺,家父齋僧二十餘年,更不曾遇著好人。今幸圓滿,四位下降,誠然是蓬蓽生輝。學生年幼,不知因果,常聞得有云:『公修公得,婆修婆得,不修不得。』我家父、家母各欲獻芹者,正是各求得些因果,何必苦辭?就是愚兄弟,也省得有些束修錢兒,也指望供養老爺半月,方才送行。」三藏道:「令堂老菩薩盛情,已不敢領,怎麼又承賢昆玉厚愛?決不敢領。今朝定要起身,萬勿見罪。不然,久違欽限,罪不容誅矣。」那老嫗與二子見他執一不住,便生起惱來道:「好意留他,他這等固執要去。要去便就去了罷,只管勞叨甚麼?」母子遂抽身進去。

八戒忍不住口,又對唐僧道:「師父,不要拿過了班兒。常言道:『留得在,落得怪。』我們且住一個月兒,了了他母子的願心也罷了,只管忙怎的?」唐僧又咄了一聲,那獃子就自家把嘴打了兩下道:「啐啐啐!說道莫多話,又做聲了!」行者與沙僧赥赥的笑在一邊。唐僧又怪行者道:「你笑甚麼?」即捻訣,要念緊箍兒咒。慌得個行者跪下道:「師父,我不曾笑,我不曾笑。千萬莫念,莫念。」

員外又見他師徒們漸生煩惱,再也不敢苦留,只叫:「老師不必吵鬧,準於明早送行。」遂此出了經堂,吩咐書辦,寫了百十個簡帖兒,邀請鄰里親戚,明早奉送唐朝老師西行。一壁廂又叫庖人安排餞行的筵宴;一壁廂又叫管辦的做二十對彩旗,覓一班吹鼓手樂人,南來寺裡請一班和尚,東岳觀裡請一班道士,限明日巳時俱要整齊。眾執事領命去訖。

不多時,天又晚了。吃了晚齋,各歸寢處。但見:
\begin{quote}
幾點歸鴉過別村,樓頭鐘鼓遠相聞。
六街三市人煙靜,萬戶千門燈火昏。
月皎風清花弄影,銀河慘淡映星辰。
子規啼處更深矣,天籟無聲大地鈞。
\end{quote}

當夜三四更天氣,各管事的家僮盡皆早起,買辦各項物件。你看那辦筵席的,廚上慌忙;置彩旗的,堂前吵鬧;請僧道的,兩腳奔波;叫鼓樂的,一身急縱;送簡帖的,東走西跑;備轎馬的,上呼下應。這半夜,直嚷至天明,將巳時前後,各項俱完,也只是有錢不過。

卻表唐僧師徒們早起,又有那一班人供奉。長老吩咐收拾行李,扣備馬匹。獃子聽說要走,又努嘴胖唇,唧唧噥噥,只得將衣缽收拾,找出高肩擔子。沙僧刷洗馬匹,套起鞍轡伺候。行者將九環杖遞在師父手裡,他將通關文牒的引袋兒掛在胸前,只是一齊要走。員外又都請至後面大廠廳內,那裡面又鋪設了筵宴,比齋堂中相待的更是不同。但見那:
\begin{quote}
簾幕高掛,屏圍四繞。正中間掛一幅壽山福海之圖,兩壁廂列四軸春夏秋冬之景。龍文鼎內香飄靄,鵲尾爐中瑞氣生。看盤簇彩,寶妝花色色鮮明;排桌堆金,獅仙糖齊齊擺列。階前鼓舞按宮商,堂上果餚鋪錦繡。素湯素飯甚清奇,香酒香茶多美艷。雖然是百姓之家,卻不亞王侯之宅。只聽得一片歡聲,真個也驚天動地。
\end{quote}

長老正與員外作禮,只見家僮來報:「客俱到了。」卻是那請來的左鄰右舍、妻弟姨兄、姐夫妹丈,又有那些同道的齋公,念佛的善友,一齊都向長老禮拜。拜畢,各各敘坐。只見堂下面鼓瑟吹笙,堂上邊絃歌酒讌。這一席盛宴,八戒留心,對沙僧道:「兄弟,放懷放量吃些兒,離了寇家,再沒這好豐盛的東西了。」沙僧笑道:「二哥說哪裡話,常言道:『珍饈百味,一飽便休。』只有私房路,哪有私房肚?」八戒道:「你也忒不濟,不濟。我這一頓盡飽吃了,就是三日也急忙不餓。」行者聽見道:「獃子,莫脹破了肚子,如今要走路哩。」

說不了,日將中矣。長老在上舉箸,念《謁齋經》。八戒慌了,拿過添飯來,一口一碗,又丟夠有五六碗,把那饅頭、兒、餅子、燒果,沒好沒歹的滿滿籠了兩袖,才跟師父起身。長老謝了員外,又謝了眾人,一同出門。你看那門外擺著彩旗寶蓋、鼓手樂人,又見那兩班僧道方來。員外笑道:「列位來遲,老師去急,不及奉齋,俟回來謝罷。」眾等讓出道路,擡轎的擡轎,騎馬的騎馬,步行的步行,都讓長老四眾前行。只聞得鼓樂諠天,旗旛蔽日,人煙湊集,車馬駢填,都來看寇員外迎送唐僧。這一場富貴,真賽過珠圍翠繞,誠不亞錦帳藏春。那一班僧,打一套佛曲;那一班道,吹一道玄音。俱送出府城之外,行至十里長亭,又設著簞食壺漿,擎杯把盞,相飲而別。那員外猶不忍捨,噙著淚道:「老師取經回來,是必到舍再住幾日,以了我寇洪之心。」三藏感之不盡,謝之無已道:「我若到靈山,得見佛祖,首表員外之大德。回時定踵門叩謝叩謝。」說說話兒,不覺的又有二三里路,長老懇切拜辭。那員外又放聲大哭而轉。這正是:
\begin{quote}
有願齋僧歸妙覺,無緣得見佛如來。
\end{quote}

且不說寇員外送至十里長亭,同眾回家。卻說他師徒四眾,行有四五十里之地,天色將晚。長老道:「天晚了,何方借宿?」八戒挑著擔,努著嘴道:「放了現成茶飯不吃,清涼瓦屋不住,卻要走甚麼路,像搶喪撞魂的。如今天晚,倘下起雨來,卻如之何?」三藏罵道:「潑孽畜,又來報怨了。常言道:『長安雖好,不是久戀之家。』待我們有緣拜了佛祖,取得真經,那時回轉大唐,奏過主公,將那御廚裡飯,憑你吃上幾年,脹死你這孽畜,教你做個飽鬼。」那獃子哈哈的暗笑,不敢復言。

行者舉目遙觀,只見大路傍有幾間房宇,急請師父道:「那裡安歇,那裡安歇。」長老至前,見是一座倒塌的牌坊,坊上有一舊匾,匾上有落顏色積塵的四個大字,乃「華光行院」。長老下了馬道:「華光菩薩是火焰五光佛的徒弟,因剿除毒火鬼王,降了職,化做五顯靈官。此間必有廟祝。」遂一齊進去,但見廊房俱倒,不見人影。欲抽身而出,不期天上黑雲蓋頂,大雨淋漓。沒奈何,卻在那破房之下,揀遮得風雨處,將身躲避。密密寂寂,不敢高聲,恐有妖邪知覺。坐的坐,站的站,苦捱了一夜未睡。咦!真個是:
\begin{quote}
泰極還生否,樂處又逢悲。
\end{quote}

畢竟不知天曉向前去還是如何,且聽下回分解。
