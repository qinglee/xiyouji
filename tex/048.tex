
\chapter{魔弄寒風飄大雪 僧思拜佛履層冰}

話說陳家莊眾信人等,將豬羊牲醴與行者、八戒,喧喧嚷嚷,直擡至靈感廟裡排下;將童男女設在上首。行者回頭看見那供桌上香花蠟燭,正面一個金字牌位,上寫「靈感大王之神」,更無別的神像。眾信擺列停當,一齊朝上叩頭道:「大王爺爺,今年今月今日今時,陳家莊祭主陳澄等眾信,年甲不齊,謹遵年例,供獻童男一名陳關保、童女一名陳一秤金,豬羊牲醴如數,奉上大王享用。保佑風調雨順,五穀豐登。」祝罷,燒了紙馬,各回本宅不題。

那八戒見人散了,對行者道:「我們家去罷。」行者道:「你家在那裡?」八戒道:「往老陳家睡覺去。」行者道:「獃子又亂談了。既允了他,須與他了這願心才是哩。」八戒道:「你倒不是獃子,反說我是獃子。只哄他耍耍便罷,怎麼就與他祭賽,當起真來?」行者道:「為人為徹。一定等那大王來吃了,才是個全始全終;不然,又教他降災貽害,反為不美。」

正說間,只聽得呼呼風響。八戒道:「不好了,風響是那話兒來了。」行者只叫:「莫言語,等我答應。」頃刻間,廟門外來了一個妖邪。你看他怎生模樣:
\begin{quote}
金甲金盔燦爛新,腰纏寶帶繞紅雲。
眼如晚出明星皎,牙似重排鋸齒分。
足下煙霞飄蕩蕩,身邊霧靄暖薰薰。
行時陣陣陰風冷,立處層層煞氣溫。
卻似捲簾扶駕將,猶如鎮寺大門神。
\end{quote}

那怪物攔住廟門問道:「今年祭祀的是那家?」行者笑吟吟的答道:「承下問,莊頭是陳澄、陳清家。」那怪聞答,心中疑似道:「這童男膽大,言談伶俐。常來供養受用的,問一聲不言語;再問聲,諕了魂;用手去捉,已是死人。怎麼今日這童男善能應對?」怪物不敢來拿,又問:「童男女叫甚名字?」行者笑道:「童男陳關保,童女一秤金。」怪物道:「這祭賽乃上年舊規,如今供獻我,當吃你。」行者道:「不敢抗拒,請自在受用。」怪物聽說,又不敢動手,攔住門喝道:「你莫頂嘴。我常年先吃童男,今年倒要先吃童女。」八戒慌了道:「大王還照舊罷,不要吃壞例子。」

那怪不容分說,放開手,就捉八戒。獃子撲的跳下來,現了本相,掣釘鈀,劈手一築,那怪物縮了手,往前就走,只聽得噹的一聲響。八戒道:「築破甲了。」行者也現本相看處,原來是冰盤大小兩個魚鱗。喝聲:「趕上。」二人跳到空中。那怪物因來赴會,不曾帶得兵器,空手在雲端裡問道:「你是那方和尚,到此欺人,破了我的香火,壞了我的名聲?」行者道:「這潑物原來不知。我等乃東土大唐聖僧三藏奉欽差西天取經之徒弟。昨因夜寓陳家,聞有邪魔,假號靈感,年年要童男女祭賽。是我等慈悲,拯救生靈,捉你這潑物。趁早實實供來:一年吃兩個童男女,你在這裡稱了幾年大王?吃了多少男女?一個個算還我,饒你死罪。」那怪聞言就走,被八戒又一釘鈀,未曾打著,他化一陣狂風,鑽入通天河內。

行者道:「不消趕他了,這怪想是河中之物。且待明日設法拿他,送我師父過河。」八戒依言,徑回廟裡,把那豬羊祭醴,連桌面一齊搬到陳家。此時唐長老、沙和尚,共陳家兄弟,正在廳中候信,忽見他二人將豬羊等物都丟在天井裡。三藏迎來問道:「悟空,祭賽之事何如?」行者將那稱名趕怪鑽入河中之事,說了一遍。二老十分歡喜,即命打掃廂房,安排床鋪,請他師徒就寢不題。

卻說那怪得命,回歸水內,坐在宮中,默默無言。水中大小眷族問道:「大王每年享祭,回來歡喜,怎麼今日煩惱?」那怪道:「常年享畢,還帶些餘物與汝等受用,今日連我也不曾吃得。造化低,撞著一個對頭,幾乎傷了性命。」眾水族問:「大王,是那個?」那怪道:「是一個東土大唐聖僧的徒弟,往西天拜佛求經者,假變男女,坐在廟裡。我被他現出本相,險些兒傷了性命。一向聞得人講:唐三藏乃十世修行好人,但得吃他一塊肉,延壽長生。不期他手下有這般徒弟。我被他壞了名聲,破了香火,有心要捉唐僧,只怕不得能夠。」

那水族中閃上一個斑衣鱖婆,對怪物跬跬拜拜,笑道:「大王要捉唐僧,有何難處?但不知捉住他,可賞我些酒肉?」那怪道:「你若有謀,合同用力,捉了唐僧,與你拜為兄妹,共席享之。」鱖婆拜謝了道:「久知大王有呼風喚雨之神通,攪海翻江之勢力,不知可會降雪?」那怪道:「會降。」又道:「既會降雪,不知可會作冷結冰?」那怪道:「更會。」鱖婆鼓掌笑道:「如此,極易,極易。」那怪道:「你且將極易之功,講來我聽。」鱖婆道:「今夜有三更天氣,大王不必遲疑,趁早作法,起一陣寒風,下一陣大雪,把通天河盡皆凍結。著我等善變化者,變作幾個人形,在於路口,背包持傘,擔擔推車,不住的在冰上行走。那唐僧取經之心甚急,看見如此人行,斷然踏冰而渡。大王悄坐河心,待他腳蹤響處,迸裂寒冰,連他那徒弟們一齊墜落水中,一鼓可得也。」那怪聞言,滿心歡喜道:「甚妙,甚妙。」即出水府,踏長空,興風作雪,結冷凝凍成冰不題。

卻說唐長老師徒四人歇在陳家,將近天曉,師徒們衾寒枕冷。八戒咳歌打戰睡不得,叫道:「師兄,冷啊。」行者道:「你這獃子,忒不長俊。出家人寒暑不侵,怎麼怕冷?」三藏道:「徒弟,果然冷。你看,就是那:
\begin{quote}
重衾無暖氣,袖手似揣冰。此時敗葉垂霜蕊,蒼松掛凍鈴。地裂因寒甚,池平為水凝。漁舟不見叟,山寺怎逢僧。樵子愁柴少,王孫喜炭增。征人鬚似鐵,詩客筆如菱。皮襖猶嫌薄,貂裘尚恨輕。蒲團僵老衲,紙帳旅魂驚。繡被重裀褥,渾身戰抖鈴。」
\end{quote}

師徒們都睡不得,爬起來穿了衣服。開門看處,呀!外面白茫茫的,原來下雪哩。行者道:「怪道你們害冷哩,卻是這般大雪。」四人眼同觀看,好雪!但見那:
\begin{quote}
彤雲密佈,慘霧重浸。彤雲密佈,朔風凜凜號空;慘霧重浸,大雪紛紛蓋地。真個是:六出花,片片飛瓊;千林樹,株株帶玉。須臾積粉,頃刻成鹽。白鸚歌失素,皓鶴羽毛同。平添吳楚千江水,壓倒東南幾樹梅。卻便似戰退玉龍三百萬,果然如敗鱗殘甲滿天飛。那裡得東郭履,袁安臥,孫康映讀;更不見子猷舟,王恭幣,蘇武餐氈。但只是幾家村舍如銀砌,萬里江山似玉團。好雪,柳絮漫橋,梨花蓋舍。柳絮漫橋,橋邊漁叟掛蓑衣;梨花蓋舍,舍下野翁煨骨柮。客子難沽酒,蒼頭苦覓梅。灑灑瀟瀟裁蝶翹,飄飄蕩蕩剪鵝衣。團團滾滾隨風勢,疊疊層層道路迷。陣陣寒威穿小幙,颼颼冷氣透幽幃。豐年祥瑞從天降,堪賀人間好事宜。
\end{quote}

那場雪,紛紛灑灑,果如剪玉飛綿。

師徒們嘆玩多時,只見陳家老者,著兩個僮僕掃開道路,又兩個送出熱湯洗面。須臾,又送滾茶、乳餅,又擡出炭火。俱到廂房,與師徒們敘坐。長老問道:「老施主,貴處時令,不知可分春夏秋冬?」陳老笑道:「此間雖是僻地,但只風俗人物,與上國不同;至於諸凡穀苗牲畜,都是同天共日,豈有不分四時之理?」三藏道:「既分四時,怎麼如今就有這般大雪,這般寒冷?」陳老道:「此時雖是七月,昨日已交白露,就是八月節了。我這裡常年八月間就有霜雪。」三藏道:「甚比我東土不同,我那裡交冬節方有之。」

正話間,又見僮僕來安桌子,請吃粥。粥罷之後,雪比早間又大,須臾,平地有二尺來深。三藏心焦垂淚。陳老道:「老爺放心,莫見雪深憂慮。我舍下頗有幾石糧食,供養得老爺們半生。」三藏道:「老施主不知貧僧之苦。我當年蒙聖恩賜了旨意,擺大駕親送出關,唐王御手擎杯奉餞,問道:『幾時可回?』貧僧不知有山川之險,順口回奏:『只消三年,可取經回國。』自別後,今已七八個年頭,還未見佛面,恐違了欽限;又怕的是妖魔兇狠:所以焦慮。今日有緣得寓潭府,昨夜愚徒們略施小惠報答,實指望求一船隻渡河。不期天降大雪,道路迷漫,不知幾時才得功成回故土也。」陳老道:「老爺放心,正是多的日子過了,那裡在這幾日?且待天晴,化了冰,老拙傾家費產,必處置送老爺過河。」

只見一僮又請進早齋。到廳上吃畢。敘不多時,又午齋相繼而進。三藏見品物豐盛,再四不安道:「既蒙見留,只可以家常相待。」陳老道:「老爺,感蒙替祭救命之恩,雖逐日設筵奉款,也難酬難謝。」

此後大雪方住就有人行走。陳老見三藏不快,又打掃花園,大盆架火,請去雪洞裡閑耍散悶。八戒笑道:「那老兒忒沒算計。春二三月好賞花園,這等大雪,又冷,賞玩何物?」行者道:「獃子不知事。雪景自然幽靜:一則遊賞,二來與師父寬懷。」陳老道:「正是,正是。」遂此邀請到園。但見:
\begin{quote}
景值三秋,風光如臘。蒼松結玉蕊,衰柳掛銀花。階下玉苔堆粉屑,窗前翠竹吐瓊芽。巧石山頭,養魚池內。巧石山頭,削削尖峰排玉筍;養魚池內,清清活水作冰盤。臨岸芙蓉嬌色淺,傍崖木槿嫩枝垂。秋海棠,全然壓倒;臘梅樹,聊發新枝。牡丹亭、海榴亭、丹桂亭,亭亭盡鵝毛堆積;放懷處、款客處、遣興處,處處皆蝶翅鋪漫。兩籬黃菊玉綃金,幾樹丹楓紅間白。無數閑庭冷難到,且觀雪洞冷如冰。那裡邊放一個獸面象足銅火盆,熱烘烘炭火才生;上下有幾張虎皮搭苫漆交椅,軟溫溫紙窗鋪設。
\end{quote}

四壁上掛幾軸名公古畫,卻是那:
\begin{quote}
七賢過關,寒江獨釣,疊嶂層巒團雪景;蘇武餐氈,折梅逢使,瓊林玉樹寫寒文。說不盡那:家近水亭魚易買,雪迷山徑酒難沽。真個可堪容膝處,算來何用訪蓬壺?
\end{quote}

眾人觀玩良久,就於雪洞裡坐下,對鄰叟道取經之事,又捧香茶飲畢。陳老問:「列位老爺,可飲酒麼?」三藏道:「貧僧不飲,小徒略飲幾杯素酒。」陳老大喜,即命:「取素果品,燉暖酒,與列位湯寒。」那僮僕即擡桌圍爐,與兩個鄰叟,各飲了幾杯,收了家火。

不覺天色將晚,又仍請到廳上晚齋。只聽得街上行人都說:「好冷天啊,把通天河凍住了。」三藏聞言道:「悟空,凍住河,我們怎生是好?」陳老道:「乍寒乍冷,想是近河邊淺水處凍結。」那行人道:「把八百里都凍的似鏡面一般,路口上有人走哩。」三藏聽說有人走,就要去看。陳老道:「老爺莫忙,今日晚了,明日去看。」遂此別卻鄰叟。又晚齋畢,依然歇在廂房。

及次日天曉,八戒起來道:「師兄,今夜更冷,想必河凍住也。」三藏迎著門,朝天禮拜道:「眾位護教大神,弟子一向西來,虔心拜佛,苦歷山川,更無一聲報怨。今至於此,感得皇天佑助,結凍河水。弟子空心權謝,待得經回,奏上唐皇,竭誠酬答。」禮拜畢,遂教悟淨背馬,趁冰過河。陳老又道:「莫忙,待幾日雪融冰解,老拙這裡辦船相送。」沙僧道:「就行也不是話,再住也不是話。口說無憑,耳聞不如眼見。我背了馬,且請師父親去看看。」陳老道:「言之有理。」教:「小的們,快去背我們六匹馬來。且莫背唐僧老爺馬。」

就有六個小价跟隨。一行人徑往河邊來看,真個是:
\begin{quote}
雪積如山聳,雲收破曉晴。寒凝楚塞千峰瘦,冰結江湖一片平。朔風凜凜,滑凍棱棱。池魚偎密藻,野鳥戀枯槎。塞外征夫俱墜指,江頭梢子亂敲牙。裂蛇腹,斷鳥足,果然冰山千百尺。萬壑冷浮銀,一川寒浸玉。東方自信出僵蠶,北地果然有鼠窟。王祥臥,光武渡,一夜溪橋連底固。曲沼結凌層,深淵重疊沍。通天闊水更無波,皎潔冰漫如陸路。
\end{quote}

三藏與一行人到了河邊,勒馬觀看,真個那路口上有人行走。三藏問道:「施主,那些人上冰往那裡去?」陳老道:「河那邊乃西梁女國。這起人都是做買賣的。我這邊百錢之物,到那邊可值萬錢;那邊百錢之物,到這邊亦可值萬錢。利重本輕,所以人不顧死生而去。常年家有五七人一船,或十數人一船,飄洋而過。見如今河道凍住,故捨命而步行也。」三藏道:「世間事惟名利最重。似他為利的,捨死忘生;我弟子奉旨全忠,也只是為名,與他能差幾何?」教:「悟空,快回施主家收拾行囊,叩背馬匹,趁此層冰,早奔西方去也。」行者笑吟吟答應。

沙僧道:「師父啊,常言道:『千日吃了千升米。』今已託賴陳府上,且再住幾日,待天晴化凍,辦船而過,忙中恐有錯也。」三藏道:「悟淨,怎麼這等愚見?若是正二月,一日暖似一日,可以待得凍解;此時乃八月,一日冷似一日,如何可便望解凍?卻不又誤了半載行程?」

八戒跳下馬來:「你們且休講閑口,等老豬試看有多少厚薄。」行者道:「獃子,前夜試水,能去拋石;如今冰凍重漫,怎生試得?」八戒道:「師兄不知。等我舉釘鈀鈀他一下。假若築破,就是冰薄,且不敢行;若築不動,便是冰厚,如何不行?」三藏道:「正是,說得有理。」那獃子撩衣拽步,走上河邊,雙手舉鈀,盡力一築,只聽撲的一聲,築了九個白跡,手也振得生疼。獃子笑道:「去得,去得,連底都錮住了。」

三藏聞言,十分歡喜,與眾同回陳家,只教收拾走路。那兩個老者苦留不住,只得安排些乾糧、烘炒,做些燒餅、饝饝相送。一家子磕頭禮拜,又捧出一盤子散碎金銀,跪在面前道:「多蒙老爺活子之恩,聊表途中一飯之敬。」三藏擺手搖頭,只是不受道:「貧僧出家人,財帛何用?就途中也不敢取出,只是以化齋度日為正事。收了乾糧足矣。」二老又再三央求,行者用指尖兒捻了一小塊,約有四五錢重,遞與唐僧道:「師父,也只當些襯錢,莫教空負二老之意。」

遂此相向而別,徑至河邊冰上,那馬蹄滑了一滑,險些兒把三藏跌下馬來。沙僧道:「師父,難行。」八戒道:「且住,問陳老官討個稻草來我用。」行者道:「要稻草何用?」八戒道:「你那裡得知?要稻草包著馬蹄方才不滑,免教跌下師父來也。」陳老在岸上聽言,急命人家中取一束稻草,卻請唐僧上岸下馬。八戒將草包裹馬足,然後踏冰而行。

別陳老離河邊,行有三四里遠近,八戒把九環錫杖遞與唐僧道:「師父,你橫此在馬上。」行者道:「這獃子奸詐。錫杖原是你挑的,如何又叫師父拿著?」八戒道:「你不曾走過冰凌,不曉得。凡是冰凍之上,必有凌眼;倘或屣著凌眼,脫將下去,若沒橫擔之物,骨都的落水,就如一個大鍋蓋蓋住,如何鑽得上來?須是如此架住方可。」行者暗笑道:「這獃子倒是個積年走冰的。」果然都依了他:長老橫擔著錫杖,行者橫擔著鐵棒,沙僧橫擔著降妖寶杖,八戒肩挑著行李,腰橫著釘鈀,師徒們放心前進。這一直行到天晚,吃了些乾糧。卻又不敢久停,對著星月光華,觀的冰凍上亮灼灼,白茫茫,只情奔走,果然是馬不停蹄。師徒們莫能合眼,走了一夜。天明又吃些乾糧,望西又進。

正行時,只聽得冰底下撲喇喇一聲響喨,險些兒諕倒了白馬。三藏大驚道:「徒弟啞!怎麼這般響喨?」八戒道:「這河忒也凍得結實,地凌響了。或者這半中間連底通錮住了也。」三藏聞言,又驚又喜,策馬前進,趲行不題。

卻說那妖邪自從回歸水府,引眾精在於冰下。等候多時,只聽得馬蹄響處,他在底下弄個神通,滑喇的迸開冰凍。慌得孫大聖跳上空中;早把那白馬落於水內,三人盡皆脫下。

那妖邪將三藏捉住,引群精徑回水府,厲聲高叫:「鱖妹何在?」老鱖婆迎門施禮道:「大王,不敢,不敢。」妖邪道:「賢妹何出此言?『一言既出,駟馬難追。』原說聽從汝計,捉了唐僧,與你拜為兄妹。今日果成妙計,捉了唐僧,就好昧了前言?」教:「小的們,擡過案桌,磨快刀來,把這和尚剖腹剜心,剝皮剮肉;一壁廂響動樂器:與賢妹共而食之,延壽長生也。」鱖婆道:「大王,且休吃他,恐他徒弟們尋來吵鬧。且寧耐兩日,讓那廝不來尋,然後剖開,請大王上坐,眾眷族環列,吹彈歌舞,奉上大王,從容自在享用,卻不好也?」那怪依言,把唐僧藏於宮後,使一個六尺長的石匣蓋在中間不題。

卻說八戒、沙僧在水裡撈著行囊,放在白馬身上馱了,分開水路,湧浪翻波,負水而出。只見行者在半空中看見,問道:「師父何在?」八戒道:「師父姓陳,名到底了。如今沒處找尋,且上岸再作區處。」原來八戒本是天蓬元帥臨凡,他當年掌管天河八萬水兵大眾;沙和尚是流沙河內出身;白馬本是西海龍孫:故此能知水性。大聖在空中指引,須臾,回轉東崖,晒刷了馬匹,紾掠了衣裳。大聖雲頭按落,一同到於陳家莊上。早有人報與二老道:「四個取經的老爺,如今只剩了三個來也。」兄弟即忙接出門外,果見衣裳還濕,道:「老爺們,我等那般苦留,卻不肯住,只要這樣方休。怎麼不見三藏老爺?」八戒道:「不叫做三藏了,改名叫做陳到底也。」二老垂淚道:「可憐!可憐!我說等雪融備船相送,堅執不從,致令喪了性命。」行者道:「老兒,莫替古人耽憂。我師父管他不死長命。老孫知道,決然是那靈感大王弄法算計去了。你且放心,與我們漿漿衣服,晒晒關文,取草料喂著白馬。等我弟兄尋著那廝,救出師父,索性剪草除根,替你一莊人除了後患,庶幾永遠得安生也。」陳老聞言,滿心歡喜,即命安排齋供。

兄弟三人飽餐一頓,將馬匹、行囊交與陳家看守。各整兵器,徑赴河邊尋師擒怪。正是:
\begin{quote}
誤踏層冰傷本性,大丹脫漏怎周全?
\end{quote}

畢竟不知怎麼救得唐僧,且聽下回分解。
