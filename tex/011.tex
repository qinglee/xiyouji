
\chapter{遊地府太宗還魂 進瓜果劉全續配}

詩曰:
\begin{quote}
百歲光陰似水流,一生事業等浮漚。
昨朝面上桃花色,今日頭邊雪片浮。
白蟻陣殘方是幻,子規聲切早回頭。
古來陰騭能延壽,善不求憐天自周。
\end{quote}

卻說太宗渺渺茫茫,魂靈徑出五鳳樓前,只見那御林軍馬,請大駕出朝採獵。太宗忻然從之,縹渺而去。行了多時,人馬俱無。獨自一個,散步荒郊草野之間。正驚惶難尋道路,只見那一邊,有一人高聲大叫道:「大唐皇帝,往這裡來,往這裡來。」太宗聞言,擡頭觀看,只見那人:
\begin{quote}
頭頂烏紗,腰圍犀角。頭頂烏紗飄軟帶,腰圍犀角顯金廂。手擎牙笏凝祥靄,身著羅袍隱瑞光。腳踏一雙粉底靴,登雲促霧;懷揣一本生死簿,注定存亡。鬢髮蓬鬆飄耳上,鬍鬚飛舞繞腮旁。昔日曾為唐國相,如今掌案侍閻王。
\end{quote}

太宗行到那邊,只見他跪拜路旁,口稱:「陛下,赦臣失誤遠迎之罪。」太宗問曰:「你是何人?因甚事前來接拜?」那人道:「微臣半月前在森羅殿上,見涇河鬼龍告陛下許救反誅之故,第一殿秦廣大王即差鬼使催請陛下,要三曹對案。臣已知之,故來此間候接。不期今日來遲,望乞恕罪,恕罪。」太宗道:「你姓甚名誰?是何官職?」那人道:「微臣存日,在陽曹侍先君駕前,為茲州令,後拜禮部侍郎,姓崔名珏。今在陰司,得受酆都掌案判官。」太宗大喜,即近前,御手忙攙道:「先生遠勞。朕駕前魏徵有書一封,正寄與先生,卻好相遇。」判官謝恩,問書在何處。太宗即向袖中取出遞與。崔珏拜接了,拆封而看。其書曰:
\begin{quote}
辱愛弟魏徵頓首書拜大都案契兄崔老先生臺下:憶昔交遊,音容如在。倏爾數載,不聞清教。常只是遇節令,設蔬品奉祭,未卜享否?又承不棄,夢中臨示,始知我兄長大人高遷。奈何陰陽兩隔,天各一方,不能面覿。今因我太宗文皇帝倏然而故,料是對案三曹,必然得與兄長相會。萬祈俯念生日交情,方便一二,放我陛下回陽,殊為愛也。容再修謝。不盡。
\end{quote}

那判官看了書,滿心歡喜道:「魏人曹前日夢斬老龍一事,臣已早知,甚是誇獎不盡。又蒙他早晚看顧臣的子孫,今日既有書來,陛下寬心,微臣管送陛下還陽,重登玉闕。」太宗稱謝了。

二人正說間,只見那邊有一對青衣童子執幢幡、寶蓋,高叫道:「閻王有請,有請。」太宗遂與崔判官並二童子舉步前進。忽見一座城,城門上掛著一面大牌,上寫著「幽冥地府鬼門關」七個大金字。那青衣將幢幡搖動,引太宗徑入城中,順街而走。只見那街傍邊有先主李淵、先兄建成、故弟元吉,上前道:「世民來了,世民來了。」那建成、元吉就來揪打索命。太宗躲閃不及,被他扯住。幸有崔判官喚一青面獠牙鬼使,喝退了建成、元吉,太宗方得脫身而去。行不數里,見一座碧瓦樓臺,真個壯麗。但見:
\begin{quote}
飄飄萬疊彩霞堆,隱隱千條紅霧現。
耿耿簷飛怪獸頭,輝輝五疊鴛鴦片。
門鑽幾路赤金釘,檻設一橫白玉段。
牖近光放曉煙,簾櫳幌亮穿紅電。
樓臺高聳接青霄,廊廡平排連寶院。
獸鼎香雲襲御衣,絳紗燈火明宮扇。
左邊猛烈擺牛頭,右下崢嶸羅馬面。
接亡送鬼轉金牌,引魄招魂垂素練。
喚作陰司總會門,下方閻老森羅殿。
\end{quote}

太宗正在外面觀看,只見那壁廂環珮叮噹,仙香奇異,外有兩對提燭,後面卻是十代閻王降階而至,是那十代閻君:秦廣王、初江王、宋帝王、仵官王、閻羅王、平等王、泰山王、都市王、卞城王、轉輪王。十王出在森羅寶殿,控背躬身,迎迓太宗。太宗謙下,不敢前行。十王道:「陛下是陽間人王,我等是陰間鬼王,分所當然,何須過讓?」太宗道:「朕得罪麾下,豈敢論陰陽人鬼之道?」遜之不已。太宗前行,徑入森羅殿上,與十王禮畢,分賓主坐定。

約有片時,秦廣王拱手而進言曰:「涇河鬼龍告陛下許救而反殺之,何也?」太宗道:「朕曾夜夢老龍求救,實是允他無事。不期他犯罪當刑,該我那人曹官魏徵處斬。朕宣魏徵在殿著棋,不知他一夢而斬。這是那人曹官出沒神機,又是那龍王犯罪當死,豈是朕之過也?」十王聞言,伏禮道:「自那龍未生之前,南斗星死簿上已註定該遭殺於人曹之手,我等早已知之。但只是他在此折辨,定要陛下來此,三曹對案。是我等將他送入輪藏,轉生去了。今又有勞陛下降臨,望乞恕我催促之罪。」言畢,命掌生死簿判官急取簿子來,看陛下陽壽天祿該有幾何。崔判官急轉司房,將天下萬國國王天祿總簿,先逐一檢閱,只見南贍部洲大唐太宗皇帝註定貞觀一十三年。崔判官吃了一驚,急取濃墨大筆,將「一」字上添了兩畫,卻將簿子呈上。十王從頭一看,見太宗名下註定三十三年,閻王驚問:「陛下登基多少年了?」太宗道:「朕即位,今一十三年了。」閻王道:「陛下寬心勿慮,還有二十年陽壽。此一來已是對案明白,請返本還陽。」太宗聞言,躬身稱謝。十閻王差崔判官、朱太尉二人,送太宗還魂。太宗出森羅殿,又起手問十王道:「朕宮中老少安否如何?」十王道:「俱安,但恐御妹壽似不永。」太宗又再拜啟謝:「朕回陽世,無物可酬謝,惟答瓜果而已。」十王喜曰:「我處頗有東瓜、西瓜,只少南瓜。」太宗道:「朕回去即送來,即送來。」從此遂相揖而別。

那太尉執一首引魂旛,在前引路。崔判官隨後保著太宗,徑出幽司。太宗舉目而看,不是舊路,問判官曰:「此路差矣?」判官道:「不差。陰司裡是這般,有去路,無來路。如今送陛下自轉輪藏出身,一則請陛下遊觀地府,一則教陛下轉托超生。」太宗只得隨他兩個引路前來。

徑行數里,忽見一座高山,陰雲垂地,黑霧迷空。太宗道:「崔先生,那廂是甚麼山?」判官道:「乃幽冥背陰山。」太宗悚懼道:「朕如何去得?」判官道:「陛下寬心,有臣等引領。」太宗戰戰兢兢,相隨二人,上得山岩,擡頭觀看,只見:
\begin{quote}
形多凸凹,勢更崎嶇。峻如蜀嶺,高似廬巖。非陽世之名山,實陰司之險地。荊棘叢叢藏鬼怪,石崖磷磷隱邪魔。耳畔不聞獸鳥噪,眼前惟見鬼妖行。陰風颯颯,黑霧漫漫。陰風颯颯,是神兵口內哨來煙;黑霧漫漫,是鬼祟暗中噴出氣。一望高低無景色,相看左右盡猖亡。那裡山也有,峰也有,嶺也有,洞也有,澗也有;只是山不生草,峰不插天,嶺不行客,洞不納雲,澗不流水。岸前皆魍魎,嶺下盡神魔,洞中收野鬼,澗底隱邪魂。山前山後,牛頭馬面亂喧呼;半掩半藏,餓鬼窮魂時對泣。催命的判官,急急忙忙傳信票;追魂的太尉,吆吆喝喝趲公文。急腳子,旋風滾滾;勾司人,黑霧紛紛。
\end{quote}

太宗全靠著那判官保護,過了陰山。

前進又歷了許多衙門,一處處俱是悲聲振耳,惡怪驚心。太宗又道:「此是何處?」判官道:「此是陰山背後一十八層地獄。」太宗道:「是那十八層?」判官道:「你聽我說:
\begin{quote}
吊筋獄、幽枉獄、火坑獄,寂寂寥寥,煩煩惱惱,盡皆是生前作下千般業,死後通來受罪名。酆都獄、拔舌獄、剝皮獄,哭哭啼啼,悽悽慘慘,只因不忠不孝傷天理,佛口蛇心墮此門。磨捱獄、碓搗獄、車崩獄,皮開肉綻,抹嘴咨牙,乃是瞞心昧己不公道,巧語花言暗損人。寒冰獄、脫殼獄、抽腸獄,垢面蓬頭,愁眉皺眼,都是大斗小秤欺痴蠢,致使災屯累自身。油鍋獄、黑暗獄、刀山獄,戰戰兢兢,悲悲切切,皆因強暴欺良善,藏頭縮頸苦伶仃。血池獄、阿鼻獄、秤杆獄,脫皮露骨,折臂斷筋,也只為謀財害命,宰畜屠生,墮落千年難解釋,沉淪永世不翻身。一個個緊縛牢拴,繩纏索綁。差些赤髮鬼、黑臉鬼,長槍短劍;牛頭鬼、馬面鬼,鐵簡銅鎚:只打得皺眉苦面血淋淋,叫地叫天無救應。
\end{quote}
正是:
\begin{quote}
人生卻莫把心欺,神鬼昭彰放過誰?善惡到頭終有報,只爭來早與來遲。」
\end{quote}

太宗聽說,心中驚慘。

進前又走不多時,見一夥鬼卒各執幢幡,路傍跪下道:「橋梁使者來接。」判官喝令起去,上前引著太宗,從金橋而過。太宗又見那一邊有一座銀橋,橋上行幾個忠孝賢良之輩,公平正大之人,亦有幢幡接引;那壁廂又有一橋,寒風滾滾,血浪滔滔,號泣之聲不絕。太宗問道:「那座橋是何名色?」判官道:「陛下,那叫做奈河橋。若到陽間,切須傳記。那橋下都是些:
\begin{quote}
奔流浩浩之水,險峻窄窄之路。儼如疋練搭長江,卻似火坑浮上界。陰氣逼人寒透骨,腥風撲鼻味鑽心。波翻浪滾,往來並沒渡人船;赤腳蓬頭,出入盡皆作業鬼。橋長數里,闊只三㪥,高有百尺,深卻千重。上無扶手欄杆,下有搶人惡怪。枷杻纏身,打上奈河險路。你看那橋邊神將甚兇頑,河內孽魂真苦惱。枒杈樹上,掛的是青紅黃紫色絲衣;壁斗崖前,蹲的是毀罵公婆淫潑婦。銅蛇鐵狗任爭餐,永墮奈河無出路。」
\end{quote}

詩曰:
\begin{quote}
時聞鬼哭與神號,血水渾波萬丈高。
無數牛頭並馬面,猙獰把守奈河橋。
\end{quote}

正說間,那幾個橋梁使者早已回去了。太宗心又驚惶,點頭暗嘆,默默悲傷。相隨著判官、太尉,早過了奈河惡水,血盆苦界。前又到枉死城,只聽哄哄人嚷,分明說:「李世民來了,李世民來了。」太宗聽叫,心驚膽戰。見一夥拖腰折臂、有足無頭的鬼魅,上前攔住;都叫道:「還我命來!還我命來!」慌得那太宗藏藏躲躲,只叫:「崔先生救我!崔先生救我!」判官道:「陛下,那些人都是那六十四處煙塵、七十二處草寇眾王子、眾頭目的鬼魂,盡是枉死的冤業,無收無管,不得超生,又無錢鈔盤纏,都是孤寒餓鬼。陛下得些錢鈔與他,我才救得哩。」太宗道:「寡人空身到此,卻那裡得有錢鈔?」判官道:「陛下,陽間有一人,金銀若干,在我這陰司裡寄放。陛下可出名立一約,小判可作保,且借他一庫,給散這些餓鬼,方得過去。」太宗問曰:「此人是誰?」判官道:「他是河南開封府人氏,姓相名良,他有十三庫金銀在此。陛下若借用過他的,到陽間還他便了。」太宗甚喜,情願出名借用。遂立了文書與判官,借他金銀一庫,著太尉盡行給散。判官復吩咐道:「這些金銀,汝等可均分用度,放你大唐爺爺過去,他的陽壽還早哩。我領了十王鈞語,送他還魂,教他到陽間做一個水陸大會,度汝等超生,再休生事。」眾鬼聞言,得了金銀,俱唯唯而退。判官令太尉搖動引魂幡,領太宗出離了枉死城中,奔上平陽大路,飄飄蕩蕩而去。

前進多時,卻來到六道輪迴之所。又見那騰雲的,身披霞帔;受籙的,腰掛金魚。僧尼道俗,走獸飛禽,魑魅魍魎,滔滔都奔走那輪迴之下,各進其道。唐王問曰:「此意何如?」判官道:「陛下明心見性,是必記了,傳與陽間人知。這喚做六道輪迴:那行善的,昇化仙道;盡忠的,超生貴道;行孝的,再生福道;公平的,還生人道;積德的,轉生富道;惡毒的,沉淪鬼道。」唐王聽說,點頭嘆曰:
\begin{quote}
「善哉真善哉,作善果無災。
善心常切切,善道大開開。
莫教興惡念,是必少刁乖。
休言不報應,神鬼有安排。」
\end{quote}

判官送唐王直至那超生貴道門,拜呼唐王道:「陛下啊,此間乃出頭之處,小判告回,著朱太尉再送一程。」唐王謝道:「有勞先生遠踄。」判官道:「陛下到陽間,千萬做個水陸大會,超度那無主的冤魂,切勿忘了。若是陰司裡無報怨之聲,陽世間方得享太平之慶。凡百不善之處,俱可一一改過。普諭世人為善,管教你後代綿長,江山永固。」

唐王一一准奏,辭了崔判官,隨著朱太尉,同入門來。那太尉見門裡有一匹海騮馬,鞍韂齊備,急請唐王上馬,太尉左右扶持。馬行如箭,早到了渭水河邊。只見那水面上有一對金色鯉魚,在河裡翻波跳鬥。唐王見了心喜,兜馬貪看不捨。太尉道:「陛下,趲動些,趁早趕時辰進城去也。」那唐王只管貪看,不肯前行。被太尉撮著腳,高呼道:「還不走,等甚?」撲的一聲,望那渭河推下馬去。卻就脫了陰司,徑回陽世。

卻說那唐朝駕下有徐茂公、秦叔寶、胡敬德、段志玄、馬三寶、程咬金、高士廉、李世勣、房玄齡、杜如晦、蕭瑀、傅奕、張道源、張士衡、王珪等兩班文武,俱保著那東宮太子與皇后、嬪妃、宮娥、侍長,都在那白虎殿上舉哀。一壁廂議傳哀詔,要曉諭天下,欲扶太子登基。時有魏徵在傍道:「列位且住,不可,不可。假若驚動州縣,恐生不測。且再按候一日,我主必還魂也。」下邊閃上許敬宗道:「魏丞相言之甚謬。自古云:『潑水難收,人逝不返。』你怎麼還說這等虛言,惑亂人心,是何道理?」魏徵道:「不瞞許先生說,下官自幼得授仙術,推算最明,管取陛下不死。」

正講處,只聽得棺中連聲大叫道:「渰殺我耶!渰殺我耶!」諕得個文官武將心慌,皇后嬪妃膽戰。一個個:
\begin{quote}
面如秋後黃桑葉,腰似春前嫩柳條。儲君腳軟,難扶喪杖盡哀儀;侍長魂飛,怎戴梁冠遵孝禮。嬪妃打跌,綵女欹斜。嬪妃打跌,卻如狂風吹倒敗芙蓉;綵女欹斜,好似驟雨沖歪嬌菡萏。眾臣悚懼,骨軟筋麻。戰戰兢兢,痴痴啞啞。把一座白虎殿,卻像斷梁橋;鬧喪臺,就如倒塌寺。
\end{quote}

此時眾宮人走得精光,那個敢近靈扶柩。多虧了正直的徐茂公、理烈的魏丞相、有膽量的秦瓊、忒猛撞的敬德,上前來扶著棺材,叫道:「陛下有甚麼放不下心處,說與我等,不要弄鬼,驚駭了眷族。」魏徵道:「不是弄鬼,此乃陛下還魂也。快取器械來。」打開棺蓋,果見太宗坐在裡面,還叫:「渰死我了!是誰救撈?」茂公等上前扶起道:「陛下甦醒,莫怕,臣等都在此護駕哩。」唐王方才開眼道:「朕適才好苦:躲過陰司惡鬼難,又遭水面喪身災。」眾臣道:「陛下寬心勿懼,有甚水災來?」唐王道:「朕騎著馬,正行至渭水河邊,見雙頭魚戲。被朱太尉欺心,將朕推下馬來,跌落河中,幾乎渰死。」魏徵道:「陛下鬼氣尚未解。」急著太醫院進安神定魄湯藥,又安排粥膳。連服一二次,方才反本還原,知得人事。一計唐王死去,已三晝夜,復回陽間為君。有詩為證:
\begin{quote}
萬古江山幾變更,歷來數代敗和成。
周秦漢晉多奇事,誰似唐王死復生?
\end{quote}

當日天色已晚,眾臣請王歸寢,各各散訖。

次早,脫卻孝衣,換了綵服,一個個紅袍烏帽,一個個紫綬金章,在那朝門外等候宣召。

卻說太宗自服了安神定魄之劑,連進了數次粥湯,被眾臣扶入寢室,一夜穩睡,保養精神,直至天明方起,抖擻威儀,你看他怎生打扮:
\begin{quote}
戴一頂沖天冠,穿一領赭黃袍。繫一條藍田碧玉帶,踏一對創業無憂履。貌堂堂,賽過當朝;威烈烈,重興今日。好一個清平有道的大唐王,起死回生的李陛下。
\end{quote}

唐王上金鑾寶殿,聚集兩班文武,山呼已畢,依品分班。只聽得傳旨道:「有事出班來奏,無事退朝。」那東廂閃過徐茂公、魏徵、王珪、杜如晦、房玄齡、袁天罡、李淳風、許敬宗等;西廂閃過殷開山、劉洪基、馬三寶、段志玄、程咬金、秦叔寶、胡敬德、薛仁貴等,一齊上前,在白玉階前俯伏啟奏道:「陛下前朝一夢,如何許久方覺?」太宗道:「日前接得魏徵書,朕覺神魂出殿,只見羽林軍請朕出獵。正行時,人馬無蹤,又見那先君父王與先兄弟爭嚷。正難解處,見一人烏帽皂袍,乃是判官崔珏,喝退先兄弟。朕將魏徵書傳遞與他。正看時,又見青衣者執幢幡,引朕入內,到森羅殿上,與十代閻王敘坐。他說那涇河龍誣告我許救轉殺之事,是朕將前言陳具一遍。他說已三曹對過案了,急命取生死文簿,檢看我的陽壽。時有崔判官傳上簿子,閻王看了,道寡人有三十三年天祿,才過得一十三年,還該我二十年陽壽,即著朱太尉、崔判官送朕回來。朕與十王作別,允了送他瓜果謝恩。自出了森羅殿,見那陰司裡不忠不孝、非禮非義、作踐五穀、明欺暗騙、大斗小秤、姦盜詐偽、淫邪欺罔之徒,受那些磨燒舂剉之苦,煎熬弔剝之刑,有千千萬萬,看之不足。又過著枉死城中,有無數的冤魂,盡都是六十四處煙塵的草寇、七十二處叛賊的魂靈,擋住了朕之來路。幸虧崔判官作保,借得河南相老兒的金銀一庫,買轉鬼魂,方得前行。崔判官教朕回陽世,千萬作一場水陸大會,超度那無主的孤魂,將此言叮嚀。分別出了那六道輪迴之下,有朱太尉請朕上馬,飛也相似,行到渭水河邊,我看見那水面上有雙頭魚戲。正歡喜處,他將我撮著腳,推下水中,朕方得還魂也。」眾臣聞此言,無不稱賀。遂此編行傳報天下,各府縣官員上表稱慶不題。

卻說太宗又傳旨赦天下罪人。又查獄中重犯。時有審官將刑部絞斬罪人,查有四百餘名呈上。太宗放赦回家,拜辭父母兄弟,託產與親戚子姪,明年今日赴曹,仍領應得之罪。眾犯謝恩而退。又出恤孤榜文。又查宮中老幼綵女共有三千人,出旨配軍。自此,內外俱善。有詩為證。詩曰:
\begin{quote}
大國唐王恩德洪,道過堯舜萬民豐。
死囚四百皆離獄,怨女三千放出宮。
天下多官稱上壽,朝中眾宰賀元龍。
善心一念天應佑,福蔭應傳十七宗。
\end{quote}

太宗既放宮女,出死囚已畢,又出御製榜文,遍傳天下。榜曰:
\begin{quote}
乾坤浩大,日月照鑒分明;宇宙寬洪,天地不容姦黨。使心用術,果報只在今生;善布淺求,獲福休言後世。千般巧計,不如本分為人;萬種強徒,怎似隨緣節儉。心行慈善,何須努力看經;意欲損人,空讀如來一藏!
\end{quote}

自此時,蓋天下無一人不行善者。一壁廂又出招賢榜,招人進瓜果到陰司裡去;一壁廂將寶藏庫金銀一庫,差尉遲恭、胡敬德上河南開封府,訪相良還債。

榜張數日,有一赴命進瓜果的賢者,本是均州人,姓劉名全,家有萬貫之資。只因妻李翠蓮在門首拔金釵齋僧,劉全罵了他幾句,說他不遵婦道,擅出閨門。李氏忍氣不過,自縊而死。撇下一雙兒女年幼,晝夜悲啼。劉全又不忍見,無奈,遂捨了性命,棄了家緣,撇了兒女,情願以死進瓜,將皇榜揭了,來見唐王。王傳旨意,教他去金亭館裡,頭頂一對南瓜,袖帶黃錢,口噙藥物。

那劉全果服毒而死,一點魂靈,頂著瓜果,早到鬼門關上。把門的鬼使喝道:「你是甚人,敢來此處?」劉全道:「我奉大唐太宗皇帝欽差,特進瓜果與十代閻王受用的。」那鬼使欣然接引。劉全徑至森羅寶殿,見了閻王,將瓜果進上道:「奉唐王旨意,遠進瓜果,以謝十王寬宥之恩。」閻王大喜道:「好一個有信有德的太宗皇帝!」遂此收了瓜果。便問那進瓜的人姓名,那方人氏。劉全道:「小人是均州城民籍。姓劉名全。因妻李氏縊死,撇下兒女,無人看管,小人情願捨家棄子,捐軀報國,特與我王進貢瓜果,謝眾大王厚恩。」十王聞言,即命查勘劉全妻李氏。那鬼使速取來在森羅殿下,與劉全夫妻相會。訴罷前言,回謝十王恩宥。那閻王卻檢生死簿子看時,他夫妻們都有登仙之壽,急差鬼使送回。鬼使啟上道:「李翠蓮歸陰日久,屍首無存,魂將何附?」閻王道:「唐御妹李玉英今該促死,你可借他屍首,教他還魂去也。」那鬼使領命,即將劉全夫妻二人還魂,同出陰司而去。

畢竟不知夫妻二人如何還魂,且聽下回分解。
