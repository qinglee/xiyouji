
\chapter{難滅伽持圓大覺 法王成正體天然}

話說唐三藏固住元陽,出離了煙花苦套。隨行者投西前進,不覺夏時。正值那薰風初動,梅雨絲絲,好光景:
\begin{quote}
冉冉綠陰密,風輕燕引雛。
新荷翻沼面,修竹漸扶蘇。
芳草連天碧,山花遍地鋪。
溪邊蒲插劍,榴火壯行圖。
\end{quote}

師徒四眾,耽炎受熱,正行處,忽見那路傍有兩行高柳,柳陰中走出一個老母,右手下攙著一個小孩兒,對唐僧高叫道:「和尚,不要走了,快早兒撥馬東回,進西去都是死路。」諕得個三藏跳下馬來,打個問訊道:「老菩薩,古人云:『海闊從魚躍,天空任鳥飛。』怎麼西進便沒路了?」那老母用手朝西指道:「那裡去有五六里遠近,乃是滅法國。那國王前生那世裡結下冤仇,今世裡無端造罪。二年前許下一個羅天大願,要殺一萬個和尚。這兩年陸陸續續,殺夠了九千九百九十六個無名和尚,只要等四個有名的和尚,湊成一萬,好做圓滿哩。你們去,若到城中,都是送命王菩薩。」三藏聞言,心中害怕,戰兢兢的道:「老菩薩,深感盛情,感謝不盡。但請問可有不進城的方便路兒?我貧僧轉過去罷。」那老母笑道:「轉不過去,轉不過去。只除是會飛的,就過去了。」八戒在傍邊賣嘴道:「媽媽兒莫說黑話,我們都是會飛的。」

行者火眼金睛,其實認得好歹:那老母攙著孩兒,原是觀音菩薩與善財童子。慌得倒身下拜,叫道:「菩薩,弟子失迎,失迎。」那菩薩一朵祥雲,輕輕駕起。嚇得個唐長老立身無地,只情跪著磕頭;八戒、沙僧也慌跪下,朝天禮拜。一時間,祥雲縹緲,徑回南海而去。

行者起來,扶著師父道:「請起來,菩薩已回寶山也。」三藏起來道:「悟空,你既認得是菩薩,何不早說?」行者笑道:「你還問話不了,我即下拜,怎麼還是不早哩?」八戒、沙僧對行者道:「感蒙菩薩指示,前邊必是滅法國,要殺和尚,我等怎生奈何?」行者道:「獃子休怕。我們曾遭著那毒魔狠怪,虎穴龍潭,更不曾傷損;此間乃是一國凡人,有何懼哉?只奈這裡不是住處,天色將晚,且有鄉村人家,上城買賣回來的,看見我們是和尚,嚷出名去,不當穩便。且引師父找下大路,尋個僻靜之處,卻好商議。」真個三藏依言,一行都閃下路來,到一個坑坎之下坐定。行者道:「兄弟,你兩個好生保守師父,待老孫變化了,去那城中看看,尋一條僻路,連夜去也。」三藏叮囑道:「徒弟啊,莫當小可,王法不容,你須仔細。」行者笑道:「放心,放心。老孫自有道理。」

好大聖,話畢,將身一縱,唿哨的跳在空中。怪哉:
\begin{quote}
上面無繩扯,下頭沒棍撐。
一般同父母,他便骨頭輕。
\end{quote}

佇立在雲端裡,往下觀看。只見那城中喜氣沖融,祥光蕩漾。行者道:「好個去處,為何滅法?」看一會,漸漸天昏,又見那:
\begin{quote}
十字街燈光燦爛,九重殿香藹鐘鳴。七點皎星照碧漢,八方客旅卸行蹤。六軍營,隱隱的畫角才吹;五鼓樓,點點的銅壺初滴。四邊宿霧昏昏,三市寒煙藹藹。兩兩夫妻歸繡幙,一輪明月上東方。
\end{quote}

他想著:「我要下去,到街坊打看路逕,這般個嘴臉,撞見人,必定說是和尚。等我變一變了。」捻著訣,念動真言,搖身一變,變做個撲燈蛾兒:
\begin{quote}
形細翼磽輕巧,滅燈撲燭投明。
本來面目化生成。腐草中間靈應。
每愛炎光觸燄,忙忙飛繞無停。
紫衣香翅趕流螢。最喜夜深風靜。
\end{quote}

但見他翩翩翻翻,飛向六街三市,傍房簷,近屋角。正行時,忽見那隅頭拐角上一彎子人家,家家門首掛著個燈籠兒。他道:「這人家過元宵哩,怎麼挨排兒都點燈籠?」他硬硬翅,飛近前來,仔細觀看,正當中一家子,方燈籠上寫著「安歇往來商賈」六字,下面又寫著「王小二店」四字。行者才知是開飯店的。又伸頭打一看,看見有八九個人,都吃了晚飯,寬了衣服,卸了頭巾,洗了腳手,各各上床睡了。行者暗喜道:「師父過得去了。」你道他怎麼就知過得去?他要起個不良之心,等那些人睡著,要偷他的衣服、頭巾,裝做俗人進城。

噫!有這般不遂意的事:正思忖處,只見那小二走向前,吩咐:「列位官人,仔細些,我這裡君子小人不同,各人的衣物、行李都要小心著。」你想那在外做買賣的人,那樣不仔細?又聽得店家吩咐,越發謹慎。他都爬起來道:「主人家說得有理,我們走路的人辛苦,只怕睡著,急忙不醒,一時失所,奈何?你將這衣服、頭巾、搭聯都收進去,待天將明,交付與我們起身。」那王小二真個把些衣物之類,盡情都搬進他屋裡去了。行者性急,展開翅,就飛入裡面,丁在一個頭巾架上。又見王小二去門首摘了燈籠,放下吊搭,關了門窗,卻才進房,脫衣睡下。那王小二有個婆子,帶了兩個孩子,哇哇聒噪,急忙不睡。那婆子又拿了一件破衣,補補納納,也不見睡。

行者暗想道:「若等這婆子睡了下手,卻不誤了師父?」又恐更深,城門閉了,他就忍不住,飛下去,望燈上一撲。真是:捨身投火焰,焦額探殘生。那盞燈早已息了。他又搖身一變,變作個老鼠,嗔嗔哇哇的叫了兩聲,跳下來,拿著衣服、頭巾,往外就走。那婆子慌慌張張的道:「老頭子,不好了,夜耗子成精也。」行者聞言,又弄手段,攔著門,厲聲高叫道:「王小二,莫聽你婆子胡說。我不是夜耗子成精。明人不做暗事,吾乃齊天大聖臨凡,保唐僧往西天取經。你這國王無道,特來借此衣冠,裝扮我師父。一時過了城去,就便送還。」那王小二聽言,一轂轆爬起來,黑天摸地,又是著忙的人,撈著褲子當衫子,左穿也穿不上,右套也套不上。

那大聖使個攝法,早已駕雲出去。復翻身,徑至路下坑坎邊前。三藏見星光月皎,探身凝望,見是行者來至近前,即開口叫道:「徒弟,可過得滅法國麼?」行者上前放下衣物道:「師父,要過滅法國,和尚做不成。」八戒道:「哥,你勒掯那個哩?不做和尚也容易,只消半年不剃頭,就長出毛來也。」行者道:「那裡等得半年?眼下就都要做俗人哩。」那獃子慌了道:「但你說話,通不察理。我們如今都是和尚,眼下要做俗人,卻怎麼戴得頭巾?就是邊兒勒住,也沒收頂繩處。」三藏喝道:「不要打花,且幹正事。端的何如?」行者道:「師父,他這城池,我已看了,雖是國王無道殺僧,卻倒是個真天子,城上有祥光喜氣。城中的街道,我也認得。這裡的鄉談,我也省得,會說。卻才在飯店內借了這幾件衣服、頭巾,我們且扮作俗人,進城去借了宿,至四更天就起來,教店家安排了齋吃。捱到五更時候,挨城門而去,奔大路西行。就有人撞見扯住,也好折辨:只說是上邦欽差的,滅法王不敢阻滯,放我們來的。」沙僧道:「師兄處的最當,且依他行。」

真個長老無奈,脫了褊衫,去了僧帽,穿了俗人的衣服,戴了頭巾。沙僧也換了。八戒的頭大,戴不得巾兒,被行者取了些針線,把頭巾扯開兩頂,縫做一頂,與他搭在頭上;揀件寬大的衣服,與他穿了。然後自家也換上一套道:「列位,這一去,把『師父』、『徒弟』四個字兒且收起。」八戒道:「除了此四字,怎的稱呼?」行者道:「都要做弟兄稱呼:師父叫做唐大官兒,你叫做朱三官兒,沙僧叫做沙四官兒,我叫做孫二官兒。但到店中,你們切休言語,只讓我一個開口答話。等他問甚麼買賣,只說是販馬的客人,把這白馬做個樣子。說我們是十弟兄,我四個先來賃店房賣馬。那店家必然款待我們,我們受用了,臨行時,等我拾塊瓦查兒,變塊銀子謝他,卻就走路。」長老無奈,只得曲從。

四眾忙忙的牽馬挑擔,跑過那邊。此處是個太平境界,入更時分,尚未關門,徑直進去。行到王小二店門首,只聽得裡邊叫哩。有的說:「我不見了頭巾。」有的說:「我不見了衣服。」行者只推不知,引著他們,往斜對門一家安歇。那家子還未收燈籠,即近門叫道:「店家,可有閑房兒,我們安歇?」那裡邊有個婦人答應道:「有有有,請官人們上樓。」說不了,就有一個漢子來牽馬,行者把馬兒遞與牽進去。他引著師父,從燈影兒後面,徑上樓門,那樓上有方便的桌椅。推開窗格,映月光齊齊坐下。只見有人點上燈來,行者攔門,一口吹息道:「這般月亮不用燈。」

那人才下去,又一個丫鬟拿四碗清茶,行者接住。樓下又走上一個婦人來,約有五十七八歲的模樣,一直上樓,站在傍邊。問道:「列位客官,那裡來的?有甚寶貨?」行者道:「我們是北方來的,有幾匹粗馬販賣。」那婦人道:「販馬的客人尚還小。」行者道:「這一位是唐大官,這一位是朱三官,這一位是沙四官,我學生是孫二官。」婦人笑道:「異姓。」行者道:「正是異姓同居:我們共有十個弟兄,我四個先來賃店房打火;還有六個在城外借歇,領著一群馬,因天晚不好進城。待我們賃了房子,明早都進來。只等賣了馬才回。」那婦人道:「一群有多少馬?」行者道:「大小有百十匹兒,都像我這個馬的身子,卻只是毛片不一。」婦人笑道:「孫二官人誠然是個客綱客紀。早是來到舍下,第二個人家也不敢留你。我舍下院落寬闊,槽劄齊備,草料又有,憑你幾百匹馬都養得下。卻一件:我舍下在此開店多年,也有個賤名。先夫姓趙,不幸去世久矣。我喚做趙寡婦店。我店裡三樣兒待客。如今先小人,後君子,先把房錢講定後,好算帳。」行者道:「說得是。你府上是那三樣待客?常言道:『貨有高低三等價,客無遠近一般看。』你怎麼說三樣待客?你可試說說我聽。」趙寡婦道:「我這裡是上、中、下三樣。上樣者,五果五菜的筵席,獅仙斗糖桌面,二位一張,請小娘兒來陪唱陪歇,每位該銀五錢,連房錢在內。」行者笑道:「相應啊,我那裡五錢銀子還不夠請小娘兒哩。」寡婦又道:「中樣者,合盤桌兒,只是水果、熱酒,篩來憑自家猜枚行令,不用小娘兒,每位只該二錢銀子。」行者道:「一發相應。下樣兒怎麼?」婦人道:「不敢在尊客面前說。」行者道:「也說說無妨,我們好揀相應的幹。」婦人道:「下樣者,沒人伏侍,鍋裡有方便的飯,憑他怎麼吃;吃飽了,拿個草兒,打個地鋪,方便處睡覺。天光時,憑賜幾文飯錢,決不爭競。」八戒聽說道:「造化,造化!老朱的買賣到了!等我看著鍋吃飽了飯,灶門前睡他娘!」行者道:「兄弟,說那裡話?你我在江湖上,那裡不賺幾兩銀子?把上樣的安排將來。」

那婦人滿心歡喜,即叫:「看好茶來,廚下快整治東西。」遂下樓去,忙叫:「宰雞宰鵝,煮醃下飯。」又叫:「殺豬殺羊,今日用不了,明日也可用。看好酒,拿白米做飯,白麵捍餅。」三藏在樓上聽見道:「孫二官,怎好?他去宰雞鵝,殺豬羊,倘送將來,我們都是長齋,那個敢吃?」行者道:「我有主張。」去那樓門邊跌跌腳道:「趙媽媽,你上來。」那媽媽上來道:「二官人有甚吩咐?」行者道:「今日且莫殺生,我們今日齋戒。」寡婦驚訝道:「官人們是長齋,是月齋?」行者道:「俱不是,我們喚做庚申齋。今朝乃是庚申日,當齋。只過三更後,就是辛酉,便開齋了。你明日殺生罷。如今且去安排些素的來,定照上樣價錢奉上。」

那婦人越發歡喜,跑下去教:「莫宰,莫宰。取些木耳、閩筍、豆腐、麵筋,園裡拔些青菜,做粉湯,發麵蒸捲子,再煮白米飯,燒香茶。」咦!那些當廚的庖丁都是每日家做慣的手段,霎時間就安排停當,擺在樓上,又有現成的獅仙糖果,四眾任情受用。又問:「可吃素酒?」行者道:「止唐大官不用,我們也吃幾杯。」寡婦又取了一壺暖酒。他三個方才斟上,忽聽得乒乓板響。行者道:「媽媽,底下倒了甚麼家火了?」寡婦道:「不是,是我小莊上幾個客子送租米來晚了,教他在底下睡。因客官到,沒人使用,教他們擡轎子去院中請小娘兒陪你們,想是轎杠撞得樓板響。」行者道:「早是說哩,快不要去請:一則齋戒日期,二則兄弟們未到。索性明日進來,一家請個表子,在府上耍耍時,待賣了馬起身。」寡婦道:「好人,好人,又不失了和氣,又養了精神。」教:「擡進轎子來,不要去請。」四眾吃了酒飯,收了家火,都散訖。

三藏在行者耳根邊悄悄的道:「那裡睡?」行者道:「就在樓上睡。」三藏道:「不穩便。我們都辛辛苦苦的,倘或睡著,這家子一時再有人來收拾,見我們或滾了帽子,露出光頭,認得是和尚,嚷將起來,卻怎麼好?」行者道:「是啊!」又去樓前跌跌腳,寡婦又上來道:「孫官人又有甚吩咐?」行者道:「我們在那裡睡?」婦人道:「樓上好睡,又沒蚊子,又是南風,大開著窗子,忒好睡覺。」行者道:「睡不得。我這朱三官兒有些寒濕氣,沙四官兒有些漏肩風,唐大哥只要在黑處睡,我也有些兒羞明,此間不是睡處。」

那媽媽走下去,倚著櫃欄嘆氣。他有個女兒,抱著個孩子近前道:「母親,常言道:『十日灘頭坐,一日行九灘。』如今炎天,雖沒甚買賣,到交秋時,還做不了的生意哩,你嗟嘆怎麼?」婦人道:「兒啊,不是愁沒買賣。今日晚間,已是將收鋪子,入更時分,有這四個馬販子來賃店房,他要上樣管待。實指望賺他幾錢銀子,他卻吃齋,又賺不得他錢,故此嗟嘆。」那女兒道:「他既吃了飯,不好往別人家去,明日還好安排葷酒,如何賺不得他錢?」婦人又道:「他都有病、怕風、羞亮,都要在黑處睡。你想家中都是些單浪瓦兒的房子,那裡去尋黑暗處?不若捨一頓飯與他吃了,教他往別家去罷。」女兒道:「母親,我家有個黑處,又無風色,甚好,甚好。」婦人道:「是那裡?」女兒道:「父親在日曾做了一張大櫃,那櫃有四尺寬,七尺長,三尺高下,裡面可睡六七個人。教他們往櫃裡睡去罷。」婦人道:「不知可好?等我問他一聲。——孫官人,舍下蝸居,更無黑處,止有一張大櫃,不透風,又不透亮,往櫃裡睡去如何?」行者道:「好好好。」即著幾個客子把櫃擡出,打開蓋兒,請他們下樓。

行者引著師父,沙僧拿擔,順燈影後徑到櫃邊。八戒不管好歹,就先進櫃去。沙僧把行李遞入,攙著唐僧進去,沙僧也到裡邊。行者道:「我的馬在那裡?」旁有伏侍的道:「馬在後屋拴著吃草料哩。」行者道:「牽來,把糟擡來,緊挨著櫃兒拴住。」方才進去,叫:「趙媽媽,蓋上蓋兒,插上鎖釘,鎖上鎖子;還替我們看看,那裡透亮,使些紙兒糊糊。明日早些兒來開。」寡婦道:「忒小心了。」遂此各各關門去睡不題。

卻說他四個到了櫃裡,可憐啊!一則乍戴個頭巾,二來天氣炎熱,又悶住了氣,略不透風。他都摘了頭巾,脫了衣服,又沒把扇子,只將僧帽撲撲搧搧。你挨著我,我挨著你,直到有二更時分,卻都睡著。惟行者有心闖禍,偏他睡不著,伸過手,將八戒腿上一捻。那獃子縮了腳,口裡哼哼的道:「睡了罷,辛辛苦苦的,有甚麼心腸還捻手捻腳的耍子?」行者搗鬼道:「我們原來的本身是五千兩,前者馬賣了三千兩,如今兩搭聯裡現有四千兩,這一群馬還賣他三千兩,也有一本一利。夠了,夠了。」八戒要睡的人,那裡答對。

豈知他這店裡走堂的、挑水的、燒火的素與強盜一夥,聽見行者說有許多銀子,他就著幾個溜出去,夥了二十多個賊,明火執杖的來打劫馬販子,沖開門進來。諕得那趙寡婦娘女們戰戰兢兢的關了房門,盡他外邊收拾。原來那賊不要店中家火,只尋客人。到樓上不見形跡,打著火把,四下照看,只見天井中一張大櫃,櫃腳上拴著一匹白馬,櫃蓋緊鎖,掀翻不動。眾賊道:「走江湖的人都有手眼。看這櫃勢重,必是行囊財帛鎖在裡面。我們偷了馬,擡櫃出城,打開分用,卻不是好?」那些賊果找起繩扛,把櫃擡著就走,幌啊幌的。八戒醒了道:「哥哥,睡罷,搖甚麼?」行者道:「莫言語,沒人搖。」三藏與沙僧忽地也醒了,道:「是甚人擡著我們哩?」行者道:「莫嚷,莫嚷。等他擡,擡到西天,也省得走路。」

那賊得了手,不往西去,倒擡向城東,殺了守門的軍,打開城門出去。當時就驚動六街三市各鋪上火甲人夫,都報與巡城總兵、東城兵馬司。那總兵、兵馬事當干己,即點人馬弓兵,出城趕賊。那賊見官軍勢大,不敢抵敵,放下大櫃,丟了白馬,各自落草逃走。眾官軍不曾拿得半個強盜,只是奪下櫃,捉住馬,得勝而回。總兵在燈光下見那馬,好馬:
\begin{quote}
鬃分銀線,尾𢷑玉條。說甚麼八駿龍駒,賽過了驌驦款段。千金市骨,萬里追風。登山每與青雲合,嘯月渾如白雪勻。真是蛟龍離海島,人間喜有玉麒麟。
\end{quote}

總兵官把自家馬兒不騎,就騎上這個白馬,帥軍兵進城,把櫃子擡在總府,同兵馬寫個封皮封了,令人巡守到天明啟奏,請旨定奪。官軍散訖不題。

卻說唐長老在櫃裡埋怨行者道:「你這個猴頭,害殺我也。若在外邊,被人拿住,送與滅法國王,還好折辨;如今鎖在櫃裡,被賊劫去,又被官軍奪來,明日見了國王,現現成成的開刀請殺,卻不湊了他一萬之數?」行者道:「外面有人打開櫃,拿出來,不是綑著,便是吊著。且忍耐些兒,免了綑吊。明日見那昏君,老孫自有登答,管你一毫兒也不傷。且放心睡睡。」

挨到三更時分,行者弄個手段,順出棒來,吹口仙氣,叫:「變!」即變做三尖頭的鑽兒,挨櫃腳兩三鑽,鑽了一個眼子。收了鑽,搖身一變,變做個螻蟻兒,將出去。現原身,踏起雲頭,徑入皇宮門外。那國王正在睡濃之際。他使個「大分身普會神法」,將左臂上毫毛都拔下來,吹口仙氣,叫:「變!」都變做小行者。右臂上毛也都拔下來,吹口仙氣,叫:「變!」都變做瞌睡蟲。念一聲「唵」字真言,教當方土地領眾佈散皇宮內院、五府六部、各衙門大小官員宅內,但有品職者,都與他一個瞌睡蟲,人人穩睡,不許翻身。又將金箍棒取在手中,掂一掂,幌一幌,叫聲:「寶貝,變!」即變做千百口剃頭刀兒。他拿一把,吩咐小行者各拿一把,都去皇宮內院、五府六部、各衙門裡剃頭。咦!這才是:
\begin{quote}
法王滅法法無窮,法貫乾坤大道通。
萬法原因歸一體,三乘妙相本來同。
鑽開玉櫃明消息,佈散金毫破蔽蒙。
管取法王成正果,不生不滅去來空。
\end{quote}

這半夜剃削成功。念動咒語,喝退土地神祇。將身一抖,兩臂上毫毛歸伏。將剃頭刀總捻成真,依然認了本性,還是一條金箍棒,收來些小之形,藏於耳內。復翻身還做螻蟻,鑽入櫃內,現了本相,與唐僧守困不題。

卻說那皇宮內院,宮娥綵女天不亮起來梳洗,一個個都沒了頭髮;穿宮的大小太監也都沒了頭髮。一擁齊來,到於寢宮外,奏樂驚寢,個個噙淚,不敢傳言。少時,那三宮皇后醒來,也沒了頭髮。忙移燈到龍床下看處,錦被窩中,睡著一個和尚,皇后忍不住言語出來,驚醒國王。那國王急睜睛,見皇后的頭光,他連忙爬起來道:「梓童,你如何這等?」皇后道:「主公亦如此也。」那皇帝摸摸頭,諕得三尸呻咋,七魄飛空,道:「朕當怎的來耶?」正慌忙處,只見那六院嬪妃、宮娥綵女、大小太監,皆光著頭跪下道:「主公,我們做了和尚耶。」國王見了,眼中流淚道:「想是寡人殺害和尚」即傳旨吩咐:「汝等不得說出落髮之事,恐文武群臣褒貶國家不正。且都上殿設朝。」

卻說那五府六部,合衙門大小官員,天不明都要去朝王拜闕。原來這半夜一個個也沒了頭髮。各人都寫表啟奏此事。只聽那:
\begin{quote}
靜鞭三響朝皇帝,表奏當今剃髮因。
\end{quote}

畢竟不知那總兵官奪下櫃裡賊臟如何,與唐僧四眾的性命如何,且聽下回分解。
