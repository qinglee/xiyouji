
\chapter{金平府元夜觀燈 玄英洞唐僧供狀}

\begin{quote}
修禪何處用工夫,馬劣猿顛速剪除。
牢捉牢拴生五彩,暫停暫住墮三途。
若教自在神丹漏,才放從容玉性枯。
喜怒憂思須掃淨,得玄得妙恰如無。
\end{quote}

話表唐僧師徒四眾離了玉華城,一路平穩,誠所謂極樂之鄉。去有五六日程途,又見一座城池。唐僧問行者道:「此又是甚麼處所?」行者道:「是座城池。但城上有杆無旗,不知地方,俟近前再問。」及至東關廂,見那兩邊茶坊酒肆喧嘩,米市油房熱鬧。街衢中有幾個無事閑遊的浪子,見豬八戒嘴長,沙和尚臉黑,孫行者眼紅,都擁擁簇簇的爭看,只是不敢近前而問。唐僧捏著一把汗,惟恐他們惹禍。又走過幾條巷口,還不到城。忽見有一座山門,門上有「慈雲寺」三字。唐僧道:「此處略進去歇歇馬,打一個齋如何?」行者道:「好,好。」四眾遂一齊而入。但見那裡邊:
\begin{quote}
珍樓壯麗,寶座崢嶸。佛閣高雲外,僧房靜月中。丹霞縹緲浮屠挺,碧樹陰森輪藏清。真淨土,假龍宮,大雄殿上紫雲籠。兩廊不絕閑人戲,一塔常開有客登。爐中香火時時爇,臺上燈花夜夜熒。忽聞方丈金鐘韻,應佛僧人朗誦經。
\end{quote}

四眾正看時,又見廊下走出一個和尚,對唐僧作禮道:「老師何來?」唐僧道:「弟子中華唐朝來者。」那和尚倒身下拜。慌得唐僧攙起道:「院主何為行此大禮?」那和尚合掌道:「我這裡向善的人,看經念佛,都指望修到你中華地托生。才見老師丰采衣冠,果然是前生修到的,方得此受用,故當下拜。」唐僧笑道:「惶恐,惶恐。我弟子乃行腳僧,有何受用?若院主在此閑養自在,才是享福哩。」那和尚領唐僧入正殿,拜了佛像,唐僧方才招呼徒弟進來。原來行者三人自見那和尚與師父講話,他都背著臉,牽著馬,守著擔,立在一處,和尚不曾在心。忽的聞唐僧叫「徒弟」,他三人方才轉面。那和尚見了,慌得叫:「爺爺呀!你高徒如何恁般醜樣?」唐僧道:「醜則雖醜,倒頗有些法力,我一路甚虧他們保護。」

正說處,裡面又走出幾個和尚作禮。先見的那和尚對後的說道:「這老師是中華大唐來的人物,那三位是他高徒。」眾僧且喜且懼道:「老師中華大國,到此何為?」唐僧言:「我奉唐王聖旨,向靈山拜佛求經。適過寶方,特奔上剎,一則求問地方,二則打頓齋食就行。」那僧人個個歡喜,又邀入方丈。方丈裡又有幾個與人家做齋的和尚。這先進去的又叫道:「你們都來看看中華人物。原來中華有俊的,有醜的。俊的真個難描難畫,醜的卻十分古怪。」那許多僧同齋主都來相見。見畢,各坐下。茶罷,唐僧問道:「貴處是何地名?」眾僧道:「我這裡乃天竺國外郡,金平府是也。」唐僧道:「貴府至靈山還有許多遠近?」眾僧道:「此間到都下有二千里。這是我等走過的。西去到靈山,我們未走,不知還有多少路,不敢妄對。」唐僧謝了。

少時,擺上齋來。齋罷,唐僧要行,卻被眾僧並齋主款留道:「老師寬住一二日,過了元宵,耍耍去不妨。」唐僧驚問道:「弟子在路,只知有山有水,怕的是逢怪逢魔,把光陰都錯過了,不知幾時是元宵佳節?」眾僧笑道:「老師拜佛與悟禪心重,故不以此為念。今日乃正月十三,到晚就試燈。後日十五上元。直至十八九,方才謝燈。我這裡人家好事,本府太守老爺愛民,各地方俱高張燈火,徹夜笙簫。還有個金燈橋,乃上古傳留,至今豐盛。老爺們寬住數日,我荒山頗管待得起。」唐僧無奈,遂俱住下。當晚只聽得佛殿上鐘鼓喧天,乃是街坊眾信人等送燈來獻佛。唐僧等都出方丈來看了燈,各自歸寢。

次日,寺僧又獻齋。吃罷,同步後園閑耍。果然好個去處,正是:
\begin{quote}
時維正月,歲屆新春。園林幽雅,景物妍森。四時花木爭奇,一派峰巒疊翠。芳草階前萌動,老梅枝上生馨。紅入桃花嫩,青歸柳色新。金谷園富麗休誇,《輞川圖》流風慢說。水流一道,野鳧出沒無常;竹種千竿,墨客推敲未定。芍藥花、牡丹花、紫薇花、含笑花,天機方醒;山茶花、紅梅花、迎春花、瑞香花,艷質先開。陰崖積雪猶含凍,遠樹浮煙已帶春。又見那鹿向池邊照影,鶴來松下聽琴。東幾廈,西幾亭,客來留宿;南幾堂,北幾塔,僧靜安禪。花卉中,有一兩座養性樓,重簷高拱;山水內,有三四處煉魔室,靜几明窗。真個是天然堪隱逸,又何須他處覓蓬瀛。
\end{quote}

師徒們玩賞一日,殿上看了燈,又都去看燈遊戲。但見那:
\begin{quote}
瑪瑙花城,琉璃仙洞,水晶雲母諸宮:似重重錦繡,疊疊玲瓏。星橋影晃乾坤動,看數株火樹搖紅。六街簫鼓,千門璧月,萬戶香風。幾處鰲峰高聳,有魚龍出海,鸞鳳騰空。羨燈光月色,和氣融融。綺羅隊裡,人人喜聽笙歌。車馬轟轟:看不盡花容玉貌,風流豪俠,佳景無窮。
\end{quote}

三藏與眾僧在本寺裡看了燈,又到東關廂各街上遊戲。到二更時,方才回轉安置。

次日,唐僧對眾僧道:「弟子原有掃塔之願,趁今日上元佳節,請院主開了塔門,讓弟子了此願心。」眾僧隨開了門。沙僧取了袈裟,隨從唐僧;到了一層,就披了袈裟,拜佛禱祝畢,即將笤帚掃了一層,卸了袈裟,付與沙僧。又掃二層,一層層直掃上絕頂。那塔上層層有佛,處處開窗,掃一層,賞玩讚羨一層。掃畢下來,天色已晚,又都點上燈火。

此夜正是十五元宵。眾僧道:「老師父,我們前晚只在荒山與關廂看燈,今晚正節,進城看看金燈如何?」唐僧欣然從之,同行者三人及本寺多僧進城看燈。正是:
\begin{quote}
三五良宵節,上元春色和。花燈懸鬧市,齊唱太平歌。又見那六街三市燈亮,半空一鑑初升。那月如馮夷推上爛銀盤,這燈似仙女織成鋪地錦。燈映月,增一倍光輝;月照燈,添十分燦爛。觀不盡鐵鎖星橋,看不了燈花火樹。雪花燈、梅花燈,春冰剪碎;繡屏燈、畫屏燈,五彩攢成。核桃燈、荷花燈,燈樓高掛;青獅燈、白象燈,燈架高檠。蝦兒燈、鱉兒燈,棚前高弄;羊兒燈、兔兒燈,簷下精神。鷹兒燈、鳳兒燈,相連相併;虎兒燈、馬兒燈,同走同行。仙鶴燈、白鹿燈,壽星騎坐;金魚燈、長鯨燈,李白高乘。鰲山燈,神仙聚會;走馬燈,武將交鋒。萬千家燈火樓臺,十數里雲煙世界。那壁廂,索琅琅玉韂飛來;這壁廂,轂轆轆香車輦過。看那紅妝樓上,倚著欄,隔著簾,並著肩,攜著手,雙雙美女貪歡;綠水橋邊,鬧吵吵,錦簇簇,醉醺醺,笑呵呵,對對遊人戲彩。滿城中簫鼓諠譁,徹夜裡笙歌不斷。
\end{quote}

有詩為證。詩曰:
\begin{quote}
錦繡場中唱彩蓮,太平境內簇人煙。
燈明月皎元宵夜,雨順風調大有年。
\end{quote}

此時正是金吾不禁,亂烘烘的,無數人煙。有那跳舞的,屣蹺的,裝鬼的,騎象的,東一攢,西一簇,看之不盡。

卻才到金燈橋上,唐僧與眾僧近前看處,原來是三盞金燈。那燈有缸來大,上照著玲瓏剔透的兩層樓閣。都是細金絲兒編成,內托著琉璃薄片,其光晃月,其油噴香。唐僧回問眾僧道:「此燈是甚油?怎麼這等異香撲鼻?」眾僧道:「老師不知。我這府後有一縣,名喚旻天縣,縣有二百四十里。每年審造差徭,共有二百四十家燈油大戶。府縣的各項差徭猶可,惟有此大戶甚是吃累:每家當一年,要使二百多兩銀子。此油不是尋常之油,乃是酥合香油。這油每一兩值價銀二兩,每一斤值三十二兩銀子。三盞燈,每缸有五百斤,三缸共一千五百斤,共該銀四萬八千兩。還有雜項繳纏使用,將有五萬餘兩,只點得三夜。」行者道:「這許多油,三夜何以就點得盡?」眾僧道:「這缸裡每缸有四十九個大燈馬,都是燈草札的把,裹了絲綿,有雞子粗細。只點過今夜,見佛爺現了身,明夜油也沒了,燈就昏了。」八戒在傍笑道:「想是佛爺連油都收去了。」眾僧道:「正是此說,滿城裡人家,自古及今,皆是這等傳說。但油乾了,人俱說是佛祖收了燈,自然五穀豐登;若有一年不乾,卻就年成荒旱,風雨不調。所以人家都要這供獻。」

正說處,只聽得半空中呼呼風響,諕得些看燈的人盡皆四散。那些和尚也立不住腳道:「老師父,回去罷。風來了,是佛爺降祥,到此看燈也。」唐僧道:「怎見得是佛來看燈?」眾僧道:「年年如此,不上三更,就有風來。知道是諸佛降祥,所以人皆迴避。」唐僧道:「我弟子原是思佛念佛拜佛的人,今逢佳景,果有諸佛降臨,就此拜拜,多少是好。」眾僧連請不回。少時,風中果現出三位佛身,近燈來了。慌得那唐僧跑上橋頂,倒身下拜。行者急忙扯起道:「師父,不是好人,必定是妖邪也。」說不了,見燈光昏暗,呼的一聲,把唐僧抱起,駕風而去。噫!不知是那山那洞真妖怪,積年假佛看金燈。諕得那八戒兩邊尋找,沙僧左右招呼。行者叫道:「兄弟,不須在此叫喚。師父樂極生悲,已被妖精攝去了。」那幾個和尚害怕道:「爺爺,怎見得是妖精攝去?」行者笑道:「原來你這夥凡人累年不識,故被妖邪惑了,只說是真佛降祥,受此燈供。剛才風到處,現佛身者,就是三個妖精。我師父亦不能識,上橋頂就拜,卻被他侮暗燈光,將器皿盛了油,連我師父都攝去。我略走遲了些兒,所以他三個化風而遁。」沙僧道:「師兄,這般卻如之何?」行者道:「不必遲疑,你兩個同眾回寺,看守馬匹、行李,等老孫趁此風追趕去也。」

好大聖,急縱觔斗雲,起在半空,聞著那腥風之氣,往東北上徑趕。趕至天曉,倏爾風息。見有一座大山,十分險峻,著實嵯峨,好山:
\begin{quote}
重重丘壑,曲曲源泉。藤蘿懸削壁,松柏挺虛巖。鶴鳴晨霧裡,鴈唳曉雲間。峨峨矗矗峰排戟,突突磷磷石砌磐。頂巔高萬仞,峻嶺疊千灣。野花佳木知春發,杜宇黃鶯應景妍。能巍奕,實巉巖,古怪崎嶇險又艱。停玩多時人不語,只聽虎豹有聲鼾。香獐白鹿隨來往,玉兔青狼去復還。深澗水流千萬里,回湍激石響潺潺。
\end{quote}

大聖在山崖上正自找尋路徑,只見四個人趕著三隻羊,從西坡下,齊吆喝:「開泰。」大聖閃火眼金睛,仔細觀看,認得是年、月、日、時四值功曹使者,隱像化形而來。大聖即掣出鐵棒,幌一幌,碗來粗細,有丈二長短。跳下崖來,喝道:「你都藏頭縮頸的那裡走?」四值功曹見他說出風息,慌得喝散三羊,現了本相,閃下路傍施禮道:「大聖恕罪,恕罪。」行者道:「這一向也不曾用著你們,你們見老孫寬慢,都一個個弄懈怠了,見也不來見我一見,是怎麼說?你們怎麼不暗中保祐吾師,都往那裡去?」功曹道:「你師父寬了禪性,在於金平府慈雲寺貪歡,所以泰極生否,樂盛成悲,今被妖邪捕獲。他身邊有護法伽藍保著哩。吾等知大聖連夜追尋,恐大聖不識山林,特來傳報。」行者道:「你既傳報,怎麼隱姓埋名,趕著三個羊兒,吆吆喝喝作甚?」功曹道:「設此三羊,以應開泰之言,喚做『三陽開泰』,破解你師之否塞也。」行者恨恨的要打,見有此意,卻就免之,收了棒,回嗔作喜道:「這座山可是妖精之處?」功曹道:「正是,正是。此山名青龍山,內有洞,名玄英洞。洞中有三個妖精:大的個名辟寒大王,第二個號辟暑大王,第三個號辟塵大王。這妖精在此有千年了。他自幼兒愛食酥合香油,當年成精,到此假裝佛像,哄了金平府官員人等,設立金燈,燈油用酥合香油。他年年到正月半,變佛像收油。今年見你師父,他認得是聖僧之身,連你師父都攝在洞內,不日要割剮你師之肉,使酥合香油煎吃哩。你快用工夫,救援去也。」

行者聞言,喝退四功曹,轉過山崖,找尋洞府。行未數里,只見那澗邊有一石崖。崖下是座石屋,屋有兩扇石門,半開半掩。門傍立有石碣,上有六字,卻是「青龍山玄英洞」。行者不敢擅入,立定步,叫聲:「妖怪!快送我師父出來。」那裡唿喇一聲,大開了門,跑出一陣牛頭精,鄧鄧呆呆的問道:「你是誰,敢在這裡呼喚!」行者道:「我本是東土大唐取經的聖僧唐三藏之大徒弟。路過金平府觀燈,我師被你家魔頭攝來。快早送還,免汝等性命;如或不然,掀翻你窩巢,教你群精都化為膿血。」

那些小妖聽言,急入裡邊報道:「大王,禍事了,禍事了。」三個老妖正把唐僧拿在那洞中深遠處,那裡問甚麼青紅皂白,教小的選剝了衣裳,汲湍中清水洗淨,算計要細切細剉,著酥合香油煎吃。忽聞得報聲「禍事」,老大著驚,問是何故。小妖道:「大門前有一個毛臉雷公嘴的和尚嚷道:大王攝了他師父來,教快送出去,免吾等性命;不然,就要掀翻窩巢,教我們都化為膿血哩。」那老妖聽說,個個心驚道:「才拿了這廝,還不曾問他個姓名來歷。小的們,且把衣服與他穿了,帶過來審他一審,端是何人,何自而來也。」

眾妖一擁上前,把唐僧解了索,穿了衣服,推至座前。諕得唐僧戰兢兢的跪在下面,只叫:「大王饒命,饒命。」三個妖精異口同聲道:「你是那方來的和尚?怎麼見佛像不躲,卻衝撞我的雲路?」唐僧磕頭道:「貧僧是東土大唐駕下差來的,前往天竺國大雷音寺拜佛祖取經的。因到金平府慈雲寺打齋,蒙那寺僧留過元宵看燈。正在金燈橋上,見大王顯現佛像,貧僧乃肉眼凡胎,見佛就拜,故此衝撞大王雲路。」那妖精道:「你那東土到此,路程甚遠。一行共有幾眾?都叫甚名字?快實實供來,我饒你性命。」唐僧道:「貧僧俗名陳玄奘,自幼在金山寺為僧。後蒙唐皇敕賜在長安洪福寺為僧官。又因魏徵丞相夢斬涇河老龍,唐王遊地府,回生陽世,開設水陸大會,超度陰魂,蒙唐王又選賜貧僧為壇主,大闡都綱。幸觀世音菩薩出現,指化貧僧,說西天大雷音寺有三藏真經,可以超度亡者昇天,差貧僧來取,因賜號三藏,即倚唐為姓,所以人都呼我為唐三藏。我有三個徒弟。第一個姓孫,名悟空行者,乃齊天大聖歸正。」群妖聞得此名,著了一驚道:「這個齊天大聖,可是五百年前大鬧天宮的?」唐僧道:「正是,正是。第二個姓豬,名悟能八戒,乃天蓬大元帥轉世。第三個姓沙,名悟淨和尚,乃捲簾大將臨凡。」

三個妖王聽說,個個心驚道:「早是不曾吃他。小的們,且把唐僧將鐵鏈鎖在後面,待拿他三個徒弟來湊吃。」遂點了一群山牛精、水牛精、黃牛精,各持兵器,走出門,掌了號頭,搖旗擂鼓。三個妖披掛整齊,都到門外喝道:「是誰人敢在我這裡吆喝?」行者閃在石崖上,仔細觀看,那妖精生得:
\begin{quote}
彩面環睛,二角崢嶸。尖尖四隻耳,靈竅閃光明。一體花紋如彩畫,滿身錦繡若蜚英。第一個,頭頂狐裘花帽暖,一臉昂毛熱氣騰;第二個,身掛輕紗飛烈焰,四蹄花瑩玉玲玲;第三個,威雄聲吼如雷振,獠牙尖利賽銀針。個個勇而猛,手持三樣兵:一個使鉞斧,一個大刀能;但看第三個,肩上橫擔扢撻藤。
\end{quote}

又見那七長八短、七肥八瘦的大大小小妖精,都是些牛頭鬼怪,各執槍棒。有三面大旗,旗上明明書著「辟寒大王」、「辟暑大王」、「辟塵大王」。

孫行者看了一會,忍耐不得,上前高叫道:「潑賊怪!認得老孫麼?」那妖喝道:「你是那鬧天宮的孫悟空?真個是聞名不曾見面,見面羞殺天神。你原來是這等個猢猻兒。」行者大怒,罵道:「我把你這個偷燈油的賊,油嘴妖怪,不要胡談,快還我師父來。」趕近前,掄鐵棒就打;那三個老妖舉三般兵器,急架相迎。這一場在山凹中好殺:
\begin{quote}
鉞斧鋼刀扢撻藤,猴王一棒敢來迎。辟寒辟暑辟塵怪,認得齊天大聖名。棒起致令神鬼怕,斧來刀砍亂飛騰。好一個混元有法真空像,抵住三妖假佛形。那三個偷油潤鼻今年犯,務捉欽差駕下僧。這個因師不懼山程遠,那個為嘴常年設獻燈。乒乓只聽刀斧響,劈朴惟聞棒有聲。衝衝撞撞三攢一,架架遮遮各顯能。一朝鬥至天將晚,不知那個虧輸那個贏。
\end{quote}

孫行者一條棒與那三個妖魔鬥經百五十合,天色將晚,勝負未分。只見那辟塵大王把扢撻藤閃一閃,跳過陣前,將旗搖了一搖。那夥牛頭怪簇擁上前,把行者圍在垓心,各掄兵器,亂打將來。行者見事不諧,唿喇的縱起觔斗雲,敗陣而走。那妖更不來趕,招回群妖,安排些晚食,眾各吃了。也叫小妖送一碗與唐僧,只待拿住孫行者等才要整治。那師父一則長齋,二則愁苦,哭啼啼的未敢沾唇不題。

卻說行者駕雲回至慈雲寺內,叫聲:「師弟。」那八戒、沙僧正自盼望商量,聽得叫時,一齊出接道:「哥哥,如何去這一日方回?端的師父下落何如?」行者笑道:「昨夜聞風而趕,至天曉,到一山,不見。幸四值功曹傳信道:那山叫做青龍山,山中有一玄英洞。洞中有三個妖精,喚做辟寒大王、辟暑大王、辟塵大王。原來積年在此偷油,假變佛像,哄了金平府官員人等。今年遇見我們,他不知好歹,反連師父都攝去。老孫審得此情,吩咐功曹等眾暗中保護師父,我尋近門前叫罵。那三怪齊出,都像牛頭鬼形。第一個使鉞斧,第二個使大刀,第三個使藤棍。後引一窩子牛頭鬼怪,搖旗擂鼓,與老孫鬥了一日,殺個手平。那妖王搖動旗,小妖都來。我見天晚,恐不能取勝,所以駕觔斗回來也。」八戒道:「那裡想是酆都城鬼王弄喧?」沙僧道:「你怎麼就猜道是酆都城鬼王弄喧?」八戒笑道:「哥哥說是牛頭鬼怪,故知之耳。」行者道:「不是,不是。若論老孫看那怪,是三隻犀牛成的精。」八戒道:「若是犀牛,且拿住他,鋸下角來,倒值好幾兩銀子哩。」

正說處,眾僧道:「孫老爺可吃晚齋?」行者道:「方便吃些兒,不吃也罷。」眾僧道:「老爺征戰這一日,豈不饑了?」行者笑道:「這日把兒那裡便得饑?老孫曾五百年不吃飲食哩。」眾僧不知是實,只以為說笑。須臾拿來,行者也吃了。道:「且收拾睡覺,待明日我等都去相持,拿住妖王,庶可救師父也。」沙僧在傍道:「哥哥說那裡話!常言道:『停留長智。』那妖精倘或今晚不睡,把師父害了,卻如之何?不若如今就去,嚷得他措手不及,方才好救師父。少遲,恐有失也。」八戒聞言,抖擻神威道:「沙兄弟說得是,我們都趁此月光去降魔耶。」行者依言,即吩咐寺僧:「看守行李、馬匹,待我等把妖精捉來,對本府刺史證其假佛,免卻燈油,以蘇概縣小民之困,卻不是好?」眾僧遵命。他三個遂縱起祥雲,出城而去。正是那:
\begin{quote}
懶散無拘禪性亂,災危有分道心蒙。
\end{quote}

畢竟不知此去勝敗何如,且聽下回分解。
