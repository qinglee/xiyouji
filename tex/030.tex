
\chapter{邪魔侵正法 意馬憶心猿}

卻說那怪把沙僧綑住,也不來殺他,也不曾打他,罵也不曾罵他一句。綽起鋼刀,心中暗想道:「唐僧乃上邦人物,必知禮義,終不然我饒了他性命,又著他徒弟拿我不成?噫!這多是我渾家有甚麼書信到他那國裡,走了風汛。等我去問他一問。」那怪陡起兇性,要殺公主。

卻說那公主不知,梳妝方畢,移步前來。只見那怪怒目攢眉,咬牙切齒。那公主還陪笑臉迎道:「郎君有何事這等煩惱?」那怪咄的一聲罵道:「你這狗心賤婦,全沒人倫。我當初帶你到此,更無半點兒說話。你穿的錦,戴的金,缺少東西我去尋。四時受用,每日情深。你怎麼只想你父母,更無半點夫婦心?」那公主聞說,嚇得跪倒在地道:「郎君啊,你怎麼今日說起這分離的話?」那怪道:「不知是我分離,是你分離哩。我把那唐僧拿來,算計要他受用,你怎麼不先告過我,就放了他?原來是你暗地裡修了書信,教他替你傳寄;不然,怎麼這兩個和尚又來打上我門,教還你回去?這不是你幹的事?」公主道:「郎君,你差怪我了,我何嘗有甚書去?」老怪道:「你還強嘴哩,現拿住一個對頭在此,卻不是證見?」公主道:「是誰?」老妖道:「是唐僧第二個徒弟沙和尚。」原來人到了死處,誰肯認死,只得與他放賴。公主道:「郎君且息怒,我和你去問他一聲。果然有書,就打死了,我也甘心;假若無書,卻不枉殺了奴奴也?」

那怪聞言,不容分說,掄開一隻簸箕大小的藍靛手,抓住那金枝玉葉的髮萬根,把公主揪上前,捽在地下。執著鋼刀,卻來審沙僧,咄的一聲道:「沙和尚,你兩個輒敢擅打上我們門來,可是這女子有書到他那國,國王教你們來的?」沙僧已綑在那裡,見妖精兇惡之甚,把公主摜倒在地,持刀要殺,他心中暗想道:「分明是他有書去,救了我師父,此是莫大之恩。我若一口說出,他就把公主殺了,此卻不是恩將仇報?罷罷罷,想老沙跟我師父一場,也沒寸功報效,今日已此被縛,就將此性命與師父報了恩罷。」遂喝道:「那妖怪不要無禮,他有甚麼書來,你這等枉他,要害他性命?我們來此問你要公主,有個緣故。只因你把我師父捉在洞中,我師父曾看見公主的模樣動靜。及至寶象國,倒換關文,那皇帝將公主畫影圖形,前後訪問,因將公主的形影,問我師父沿途可曾看見,我師父遂將公主說起。他故知是他兒女,賜了我等御酒,教我們來拿你,要他公主還宮。此情是實,何嘗有甚書信?你要殺就殺了我老沙,不可枉害平人,大虧天理。」

那妖見沙僧說得雄壯,遂丟了刀,雙手抱起公主道:「是我一時粗鹵,多有衝撞,莫怪莫怪。」遂與他挽了青絲,扶上寶髻,軟款溫柔,怡顏悅色,撮哄著他進去了,又請上坐陪禮。那公主是婦人家水性,見他錯敬,遂回心轉意道:「郎君啊,你若念夫婦的恩愛,可把那沙僧的繩子略放鬆些兒。」老妖聞言,即命小的們把沙僧解了繩子,鎖在那裡。沙僧見解縛鎖住,立起來,心中暗喜道:「古人云:『與人方便,自己方便。』我若不方便了他,他怎肯教把我鬆放鬆放?」

那老妖又教安排酒席,與公主陪禮壓驚。吃酒到半酣,老妖忽的又換了一件鮮明的衣服,取了一口寶刀,佩在腰裡,轉過手,摸著公主道:「渾家,你且在家吃酒,看著兩個孩兒,不要放了沙和尚。趁那唐僧在那國裡,我也趕早兒去認認親也。」公主道:「你認甚親?」老妖道:「認你父王。我是他駙馬,他是我丈人,怎麼不去認認?」公主道:「你去不得。』老妖道:「怎麼去不得?」公主道:「我父王不是馬掙力戰的江山,他本是祖宗遺留的社稷。自幼兒是太子登基,城門也不曾遠出,沒有見你這等兇漢。你這嘴臉相貌,生得這等醜陋,若見了他,恐怕嚇了他,反為不美。卻不如不去認的還好。」老妖道:「既如此說,我變個俊的兒去便罷。」公主道:「你試變來我看看。」

好怪物,他在那酒席間搖身一變,就變做一個俊俏之人。真個生得:
\begin{quote}
形容典雅,體段崢嶸。言語多官樣,行藏正妙齡。才如子建成詩易,貌似潘安擲果輕。頭上戴一頂鵲尾冠,烏雲斂伏;身上穿一件玉羅褶,廣袖飄迎。足下烏靴花摺,腰間鸞帶光明。丰神真是奇男子,聳壑軒昂美俊英。
\end{quote}

公主見了,十分歡喜。那妖笑道:「渾家,可是變得好麼?」公主道:「變得好,變得好。你這一進朝啊,我父王是親不滅,一定著文武多官留你飲宴。倘吃酒中間,千千仔細,萬萬個小心;卻莫要現出原嘴臉來,露出馬腳,走了風汛,就不斯文了。」老妖道:「不消吩咐,自有道理。」

你看他縱雲頭,早到了寶象國。按落雲頭,行至朝門之外,對閣門大使道:「三駙馬特來見駕,乞為轉奏轉奏。」那黃門奏事官來至白玉階前奏道:「萬歲,有三駙馬來見駕,現在朝門外聽宣。」那國王正與唐僧敘話,忽聽得三駙馬,便問多官道:「寡人只有兩個駙馬,怎麼又有個三駙馬?」多官道:「三駙馬必定是妖怪來了。」國王道:「可好宣他進來?」那長老心驚道:「陛下,妖精啊,不精者不靈。他能知過去未來,他能騰雲駕霧。宣他也進來,不宣他也進來,倒不如宣他進來,還省些口面。」

國王准奏,叫宣,把妖宣至金階。他一般的也舞蹈山呼的行禮。多官見他生得俊麗,也不敢認他是妖精。他都是些肉眼凡胎,卻當做好人。那國王見他聳壑昂霄,以為濟世之梁棟,便問他:「駙馬,你家在那裡居住?是何方人氏?幾時得我公主配合?怎麼今日才來認親?」那老妖叩頭道:「主公,臣是城東碗子山波月洞人家。」國王道:「你那山離此處多遠?」老妖道:「不遠,只有三百里。」國王道:「三百里路,我公主如何得到那裡,與你匹配?」那妖精巧語花言,虛情假意的答道:「主公,微臣自幼兒好習弓馬,採獵為生。那十三年前,帶領家童數十,放鷹逐犬,忽見一隻斑斕猛虎,身馱著一個女子,往山坡下走。是微臣兜弓一箭,射倒猛虎,將女子帶上本莊,把溫水溫湯灌醒,救了他性命。因問他是那裡人家,他更不曾題『公主』二字。早說是萬歲的三公主,怎敢欺心,擅自配合?當得進上金殿,大小討一個官職榮身。只因他說是民家之女,才被微臣留在莊所。女貌郎才,兩相情願,故配合至此多年。當時配合之後,欲將那虎宰了,邀請諸親,卻是公主娘娘教且莫殺。其不殺之故,有幾句言詞,道得甚好,說道:
\begin{quote}
托天托地成夫婦,無媒無證配婚姻。
前世赤繩曾繫足,今將老虎做媒人。
\end{quote}

臣因此言,故將虎解了索子,饒了他性命。那虎帶著箭傷,跑蹄剪尾而去。不知他得了性命,在那山中,修了這幾年,煉體成精,專一迷人害人。臣聞得昔年也有幾次取經的,都說是大唐來的唐僧。想是這虎害了唐僧,得了他文引,變作那取經的模樣,今在朝中哄騙主公。主公啊,那繡墩上坐的,正是那十三年前馱公主的猛虎,不是真正取經之人。」

你看那水性的君王,愚迷肉眼,不識妖精,轉把他一片虛詞,當了真實。道:「賢駙馬,你怎的認得這和尚是馱公主的老虎?」那妖道:「主公,臣在山中,吃的是老虎,穿的也是老虎,與他同眠同起,怎麼不認得?」國王道:「你既認得,可教他現出本相來看。」怪物道:「借半盞淨水,臣就教他現了本相。」國王命官取水,遞與駙馬。

那怪接水在手,縱起身來,走上前,使個「黑眼定身法」。念了咒語,將一口水望唐僧噴去,叫聲:「變!」那長老的真身,隱在殿上,真個變作一隻斑斕猛虎。此時君臣肉眼觀看,那隻虎生得:
\begin{quote}
白額圓頭,花身電目。四隻蹄,挺直崢嶸;二十爪,鉤彎鋒利。鋸牙包口,尖耳連眉。獰猙壯若大貓形,猛烈雄如黃犢樣。剛鬚直直插銀條,刺舌騂騂噴惡氣。果然是隻猛斑斕,陣陣威風吹寶殿。
\end{quote}

國王一見,魄散魂飛。諕得那多官盡皆躲避。有幾個大膽的武將,領著將軍、校尉一擁上前,使各項兵器亂砍。這一番,不是唐僧該有命不死,就是二十個僧人也打為肉醬。此時幸有丁甲、揭諦、功曹、護教諸神暗在半空中護佑,所以那些人兵器皆不能打傷。眾臣嚷到天晚,才把那虎活活的捉了,用鐵繩鎖了,放在鐵籠裡,收於朝房之內。

那國王卻傳旨,教光祿寺大排筵宴,謝駙馬救拔之恩;不然,險被那和尚害了。當晚眾臣朝散,那妖魔進了銀安殿。又選十八個宮娥綵女,吹彈歌舞,勸妖魔飲酒作樂。那怪物獨坐上席,左右排列的都是那艷質嬌姿。你看他受用飲酒,至二更時分,醉將上來,忍不住胡為:跳起身,大笑一聲,現了本相,陡發兇心,伸開簸箕大手,把一個彈琵琶的女子抓將過來,扢咋的把頭咬了一口。嚇得那十七個宮娥,沒命的前後亂跑亂藏。你看那:
\begin{quote}
宮娥悚懼,綵女忙驚。宮娥悚懼,一似雨打芙蓉籠夜雨;綵女忙驚,就如風吹芍藥舞春風。捽碎琵琶顧命,跌傷琴瑟逃生。出門那分南北,離殿不管西東。磕損玉面,撞破嬌容。人人逃命走,各各奔殘生。
\end{quote}

那些人出去,又不敢吆喝。夜深了,又不敢驚駕。都躲在那短牆檐下,戰戰兢兢不題。

卻說那怪物坐在上面,自斟自酌。喝一盞,扳過人來,血淋淋的啃上兩口。他在裡面受用,外面人盡傳道:「唐僧是個虎精。」亂傳亂嚷,嚷到金亭館驛。此時驛裡無人,止有白馬在槽上吃草吃料。他本是西海小龍王,因犯天條,鋸角退鱗,變白馬,馱唐僧往西方取經。忽聞人講唐僧是個虎精,他也心中暗想道:「我師父分明是個好人,必然被怪把他變做虎精,害了師父。怎的好?怎的好?大師兄去得久了,八戒、沙僧又無音信。」他只捱到二更時分,卻才跳將起來道:「我今若不救唐僧,這功果休矣,休矣!」他忍不住頓絕韁繩,抖鬆鞍轡,急縱身,忙顯化,依然化作龍。駕起烏雲,直上九霄空裡觀看。有詩為證,詩曰:
\begin{quote}
三藏西來拜世尊,途中偏有惡妖氛。
今宵化虎災難脫,白馬垂韁救主人。
\end{quote}

小龍王在半空裡,只見銀安殿內燈燭輝煌。原來那八個滿堂紅上點著八根蠟燭。低下雲頭,仔細看處,那妖魔獨自個在上面逼法的飲酒吃人肉哩。小龍笑道:「這廝不濟,走了馬腳,識破風汛,屣匾秤鉈了。吃人可是個長進的?卻不知我師父下落何如,倒遇著這個潑怪。且等我去戲他一戲,若得手,拿住妖精,再救師父不遲。」

好龍王,他就搖身一變,也變做個宮娥,真個身體輕盈,儀容嬌媚。忙移步走入裡面,對妖魔道聲萬福:「駙馬啊,你莫傷我性命,我來替你把盞。」那妖道:「斟酒來。」小龍接過壺來,將酒斟在他盞中,酒比鍾高出三五分來,更不漫出。這是小龍使的「逼水法」。那怪見了不識,心中喜道:「你有這般手段?」小龍道:「還斟得有幾分高哩。」那怪道:「再斟上,再斟上。」他舉著壺,只情斟,那酒只情高,就如十三層寶塔一般,尖尖滿滿,更不漫出些須。那怪物伸過嘴來,吃了一鍾;扳著死人,吃了一口。道:「會唱麼?」小龍道:「也略曉得些兒。」依腔韻唱了一個小曲,又奉了一鍾。那怪道:「你會舞麼?」小龍道:「也略曉得些兒,但只是素手,舞得不好看。」那怪揭起衣服,解下腰間所佩寶劍,掣出鞘來,遞與小龍。

小龍接了刀,就留心,在那酒席前上三下四,左五右六,丟開了花刀法。怪看得眼咤。小龍丟了花字,望妖精劈一刀來。好怪物,側身躲過,慌了手腳,舉起一根滿堂紅,架住寶刀。那滿堂紅原是熟鐵打造的,連柄有八九十斤。兩個出了銀安殿,小龍現了本相,駕起雲頭,與那妖魔在那半空中相殺。這一場,黑地裡好殺。怎見得:
\begin{quote}
那一個是碗子山生成的怪物,這一個是西洋海罰下的真龍。一個放毫光,如噴白電;一個生銳氣,如迸紅雲。一個好似白牙老象走人間,一個就如金爪狸貓飛下界。一個是擎天玉柱,一個是架海金梁。銀龍飛舞,黃鬼翻騰。左右寶刀無怠慢,往來不歇滿堂紅。
\end{quote}

他兩個在雲端裡戰夠八九回合,小龍的手軟觔麻,老魔的身強力壯。小龍抵敵不住,飛起刀去,砍那妖怪。妖怪有接刀之法,一隻手接了寶刀,一隻手拋下滿堂紅便打。小龍措手不及,被他把後腿上著了一下。急慌慌按落雲頭,多虧了御水河救了性命,小龍一頭鑽下水去。那妖魔趕來尋他不見,執了寶刀,拿了滿堂紅,回上銀安殿,照舊吃酒睡覺不題。

卻說那小龍潛於水底,半個時辰聽不見聲息,方才咬著牙,忍著腿疼跳將起去。踏著烏雲,徑轉館驛,還變作依舊馬匹,伏於槽下。可憐渾身是水,腿有傷痕。那時節:
\begin{quote}
意馬心猿都失散,金公木母盡凋零。
黃婆傷損通分別,道義消疏怎得成!
\end{quote}

且不言三藏逢災,小龍敗戰。卻說那豬八戒從離了沙僧,一頭藏在草科裡,拱了一個豬渾塘。這一覺,直睡到半夜時候才醒。醒來時,又不知是甚麼去處。摸摸眼,定了神思,側耳才聽。噫!正是那山深無犬吠,野曠少雞鳴。他見那星移斗轉,約莫有三更時分,心中想道:「我要回救沙僧,誠然是『單絲不線,孤掌難鳴』。罷罷罷,我且進城去見了師父,奏准當今,再選些驍勇人馬,助著老豬明日來救沙僧罷。」

那獃子急縱雲頭,徑回城裡。半霎時,到了館驛。此時人靜月明。兩廊下尋不見師父,只見白馬睡在那廂,渾身水濕,後腿有盤子大小一點青痕。八戒失驚道:「雙晦氣了。這亡人又不曾走路,怎麼身上有汗,腿有青痕?想是歹人打劫師父,把馬打壞了。」那白馬認得是八戒,忽然口吐人言,叫聲:「師兄。」這獃子嚇了一跌,扒起來,往外要走。被白馬探探身,一口咬住皂衣,道:「哥啊,你莫怕我。」八戒戰兢兢的道:「兄弟,你怎麼今日說起話來了?你但說話,必有大不祥之事。」小龍道:「你知師父有難麼?」八戒道:「我不知。」小龍道:「你是不知。你與沙僧在皇帝面前弄了本事,思量拿倒妖魔,請功求賞。不想妖魔本領大,你們手段不濟,禁他不過。好道著一個回來,說個信息是,卻更不聞音。那妖精變做一個俊俏文人,撞入朝中,與皇帝認了親眷。把我師父變作一個斑斕猛虎,見被眾臣捉住,鎖在朝房鐵籠裡面。我聽得這般苦惱,心如刀割。你兩日又不在不知,恐一時傷了性命,只得化龍身去救,不期到朝裡,又尋不見師父。及到銀安殿外,遇見妖精,我又變做個宮娥模樣,哄那怪物。那怪叫我舞刀他看,遂爾留心,砍他一刀。早被他閃過,雙手舉個滿堂紅,把我戰敗。我又飛刀砍去,他又把刀接了,捽下滿堂紅,把我後腿上著了一下。故此鑽在御水河,逃得性命。腿上青是他滿堂紅打的。」

八戒聞言道:「真個有這樣事?」小龍道:「莫成我哄你了?」八戒道:「怎的好?怎的好?你可掙得動麼?」小龍道:「我掙得動便怎的?」八戒道:「你掙得動,便掙下海去罷。把行李等老豬挑去高老莊上,回爐做女婿去呀。」小龍聞說,一口咬住他直裰子,那裡肯放,止不住眼中滴淚道:「師兄啊,你千萬休生懶惰。」八戒道:「不懶惰便怎麼?沙兄弟已被他拿住,我是戰不過他,不趁此散火,還等甚麼?」

小龍沉吟半晌,又滴淚道:「師兄啊,莫說散火的話。若要救得師父,你只去請個人來。」八戒道:「教我請誰麼?」小龍道:「你趁早兒駕雲回上花果山,請大師兄孫行者來。他還有降妖的大法力,管教救了師父,也與你我報得這敗陣之仇。」八戒道:「兄弟,另請一個兒便罷了。那猴子與我有些不睦。前者在白虎嶺上,打殺了那白骨夫人,他怪我攛掇師父念緊箍兒咒。我也只當耍子,不想那老和尚當真的念起來,就把他趕逐回去。他不知怎麼樣的惱我,他也決不肯來。倘或言語上略不相對,他那哭喪棒又重,假若不知高低,撈上幾下,我怎的活得成麼?」小龍道:「他決不打你。他是個有仁有義的猴王。你見了他,且莫說師父有難,只說:『師父想你哩。』把他哄將來。到此處,見這樣個情節,他必然不忿,斷乎要與那妖精比併,管情拿得那妖精,救得我師父。」八戒道:「也罷,也罷。你倒這等盡心,我若不去,顯得我不盡心了。我這一去,果然行者肯來,我就與他一路來了;他若不來,你卻也不要望我,我也不來了。」小龍道:「你去,你去,管情他來也。」

真個獃子收拾了釘鈀,整束了直裰,跳將起去,踏著雲,徑往東來。這一回,也是唐僧有命。那獃子正遇順風,撐起兩個耳朵,好便似風篷一般,早過了東洋大海,按落雲頭。不覺的太陽星上,他卻入山尋路。

正行之際,忽聞得有人言語。八戒仔細看時,看來是行者在山凹裡,聚集群妖。他坐在一塊石頭崖上,面前有一千二百多猴子,分序排班,口稱:「萬歲,大聖爺爺。」八戒道:「且是好受用,且是好受用,怪道他不肯做和尚,只要來家哩,原來有這些好處,許大的家業,又有這多的小猴伏侍。若是老豬有這一座山場,也不做甚麼和尚了。如今既到這裡,卻怎麼好?必定要見他一見是。」那獃子有些怕他,又不敢明明的見他,卻往草崖邊溜阿溜的,溜在那一千二三百猴子當中擠著,也跟那些猴子磕頭。

不知孫大聖坐得高,眼又乖滑,看得他明白,便問:「那班部中亂拜的是個夷人,是那裡來的?拿上來。」說不了,那些小猴一窩蜂,把個八戒推將上來,按倒在地。行者道:「你是那裡來的夷人?」八戒低著頭道:「不敢,承問了。不是夷人,是熟人,熟人。」行者道:「我這大聖部下的群猴,都是一般模樣。你這嘴臉生得各樣,相貌有些雷堆,定是別處來的妖魔。既是別處來的,若要投我部下,先來遞個腳色手本,報了名字,我好留你在這隨班點扎。若不留你,你敢在這裡亂拜?」八戒低著頭,拱著嘴道:「不羞,就拿出這副嘴臉來了。我和你兄弟也做了幾年,又推認不得,說是甚麼夷人。」行者笑道:「擡起頭來我看。」那獃子把嘴往上一伸道:「你看麼,你認不得我,好道認得嘴耶。」行者忍不住笑道:「豬八戒。」他聽見一聲叫,就一轂轆跳將起來道:「正是,正是,我是豬八戒。」他又思量道:「認得就好說話了。」

行者道:「你不跟唐僧取經去,卻來這裡怎的?想是你衝撞了師父,師父也貶你回來了。有甚貶書,拿來我看。」八戒道:「不曾衝撞他,他也沒甚麼貶書,也不曾趕我。」行者道:「既無貶書,又不曾趕你,你來我這裡怎的?」八戒道:「師父想你,著我來請你的。」行者道:「他也不請我,他也不想我。他那日對天發誓,親筆寫了貶書,怎麼又肯想我,又肯著你遠來請我?我斷然也是不好去的。」八戒就地扯個謊,忙道:「委是想你,委是想你。」行者道:「他怎的想我來?」八戒道:「師父在馬上正行,叫聲『徒弟』,我不曾聽見,沙僧又推耳聾。師父就想起你來,說我們不濟,說你還是個聰明伶俐之人,常時聲叫聲應,問一答十。因這般想你,專專教我來請你的。萬望你去走走,一則不孤他仰望之心,二來也不負我遠來之意。」行者聞言,跳下崖來,用手攙住八戒道:「賢弟,累你遠來,且和我耍耍兒去。」八戒道:「哥啊,這個所在路遠,恐師父盼望去遲,我不耍子了。」行者道:「你也是到此一場,看看我的山景何如?」那獃子不敢苦辭,只得隨他走走。

二人攜手相攙,概眾小妖隨後,上那花果山極巔之處。好山,自是那大聖回家,這幾日,收拾得復舊如新。但見那:
\begin{quote}
青如削翠,高似摩雲。週圍有虎踞龍蟠,四面多猿啼鶴唳。朝出雲封山頂,暮觀日掛林間。流水潺潺鳴玉珮,澗泉滴滴奏瑤琴。山前有崖峰峭壁,山後有花木穠華。上連玉女洗頭盆,下接天河分派水。乾坤結秀賽蓬萊,清濁育成真洞府。丹青妙筆畫時難,仙子天機描不就。玲瓏怪石石玲瓏,玲瓏結綵嶺頭峰。日影動,千條紫艷;瑞氣搖,萬道紅霞。洞天福地人間有,遍山新樹與新花。
\end{quote}

八戒觀之不盡,滿心歡喜道:「哥啊,好去處,果然是天下第一名山。」行者道:「賢弟,可過得日子麼?」八戒笑道:「你看師兄說的話,寶山乃洞天福地之處,怎麼說度日之言也?」

二人談笑多時,下了山。只見路傍有幾個小猴,捧著紫巍巍的葡萄,香噴噴的梨棗,黃森森的枇杷,紅艷艷的楊梅,跪在路傍,叫道:「大聖爺爺,請進早膳。」行者笑道:「我豬弟食腸大,卻不是以果子作膳的。也罷,也罷,莫嫌菲薄,將就吃個兒當點心罷。」八戒道:「我雖食腸大,卻也隨鄉入鄉是。拿來,拿來,我也吃幾個兒嘗新。」

二人吃了果子,漸漸日高。那獃子恐怕誤了救唐僧,只管催促道:「哥哥,師父在那裡盼望我和你哩。望你和我早早兒去罷。」行者道:「賢弟,請你往水簾洞裡去耍耍。」八戒堅辭道:「多感老兄盛意,奈何師父久等,不勞進洞罷。」行者道:「既如此,不敢久留,請就此處奉別。」八戒道:「哥哥,你不去了?」行者道:「我往哪裡去?我這裡天不收,地不管,自由自在,不耍子兒,做甚麼和尚?我是不去,你自去罷。但上覆唐僧:既趕退了,再莫想我。」獃子聞言,不敢苦逼,只恐逼發他性子,一時打上兩棍。無奈,只得喏喏告辭,找路而去。

行者見他去了,即差兩個溜撒的小猴跟著八戒,聽他說些甚麼。真個那獃子下了山,不上三四里路,回頭指著行者,口裡罵道:「這個猴子,不做和尚,倒做妖怪。這個猢猻,我好意來請他,他卻不去。你不去便罷。」走幾步,又罵幾聲。那兩個小猴急跑回來報道:「大聖爺爺,那豬八戒不大老實,他走走兒,罵幾聲。」行者大怒,叫:「拿將來!」那眾猴滿地飛來趕上,把個八戒扛翻倒了,抓鬃扯耳,拉尾揪毛,捉將回去。

畢竟不知怎麼處治,性命死活若何,且聽下回分解。
