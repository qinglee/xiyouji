
\chapter{姹女求陽 元神護道}

卻說八戒跳下山,尋著一條小路。依路前行,有五六里遠近,忽見二個女怪,在那井上打水。他怎麼認得是兩個女怪,見他頭上戴一頂一尺二三寸高的篾絲䯼髻,甚不時興。獃子走近前,叫聲:「妖怪。」那怪聞言大怒,兩人互相說道:「這和尚憊𪬯,我們又不與他相識,平時又沒有調得嘴慣,他怎麼叫我們做妖怪?」那怪惱了,掄起擡水的杠子,劈頭就打。

這獃子手無兵器,遮架不得,被他撈了幾下,侮著頭跑上山來道:「哥啊,回去罷,妖怪兇。」行者道:「怎麼兇?」八戒道:「山凹內兩個女妖精在井上打水,我只叫了他一聲,就被他打了我三四杠子。」行者道:「你叫他做甚麼的?」八戒道:「我叫他做妖怪。」行者笑道:「打得還少。」八戒道:「謝你照顧。頭都打腫了,還說少哩。」行者道:「溫柔天下去得,剛強寸步難移。他們是此地之怪,我們是遠來之僧,你一身都是手,也要略溫存。你就去叫他做妖怪,他不打你,打我?人將禮樂為先。」八戒道:「一發不曉得。」行者道:「你自幼在山中吃人,你曉得有兩樣木麼?」八戒道:「不知。是甚麼木?」行者道:「一樣是楊木,一樣是檀木。楊木性格甚軟,巧匠取來,或雕聖像,或刻如來,裝金立粉,嵌玉裝花,萬人燒香禮拜,受了多少無量之福。那檀木性格剛硬,油房裡取了去,做柞撒,使鐵箍箍了頭,又使鐵鎚往下打,只因剛強,所以受此苦楚。」八戒道:「哥啊,你這好話兒,早與我說說也好,卻不受他打了。」行者道:「你還去問他個端的。」八戒道:「這去他認得我了。」行者道:「你變化了去。」八戒道:「哥啊,且如我變了,卻怎麼問麼?」行者道:「你變了去,到他跟前,行個禮兒,看他多大年紀。若與我們差不多,叫他聲『姑娘』;若比我們老些兒,叫他聲『奶奶』。」八戒笑道:「可是蹭蹬,這般許遠的田地,認得是甚麼親?」行者道:「不是認親,要套他的話哩。若是他拿了師父,就好下手;若不是他,卻不誤了我們別處幹事?」八戒道:「說得有理,等我再去。」

好獃子,把釘鈀撒在腰裡,下山凹,搖身一變,變做個黑胖和尚。搖搖擺擺,走近怪前,深深唱個大喏道:「奶奶,貧僧稽首了。」那兩個喜道:「這個和尚卻好,會唱個喏兒,又會稱道一聲兒。」問道:「長老,那裡來的?」八戒道:「那裡來的。」又問:「那裡去的?」又道:「那裡去的。」又問:「你叫做甚麼名字?」又答道:「我叫做甚麼名字。」那怪笑道:「這和尚好便好,只是沒來歷,會說順口話兒。」八戒道:「奶奶,你們打水怎的?」那怪道:「和尚,你不知道,我家老夫人今夜裡攝了一個唐僧在洞內,要管待他,我洞中水不乾淨,差我兩個來此打這陰陽交媾的好水,安排素果素菜的筵席,與唐僧吃了,晚間要成親哩。」

那獃子聞此言,急抽身跑上山,叫:「沙和尚,快拿將行李來,我們分了罷。」沙僧道:「二哥,又分怎的?」八戒道:「分了,便你還去流沙河吃人,我去高老莊探親,哥哥去花果山稱聖,白龍馬歸大海成龍。師父已在這妖精洞內成親哩,我們都各安生理去也。」行者道:「這獃子又胡說了。」八戒道:「你的兒子胡說。才那兩個擡水的妖精說,安排素筵席與唐僧吃了成親哩。」行者道:「那妖精把師父困在洞裡,師父眼巴巴的望我們去救,你卻在此說這樣話。」八戒道:「怎麼救?」行者道:「你兩個牽著馬,挑著擔,我們跟著那兩個女怪,做個引子,引到那門前,一齊下手。」

真個獃子只得隨行。行者遠遠的標著那兩怪,漸入深山,有一二十里遠近,忽然不見。八戒驚道:「師父是日裡鬼拿去了。」行者道:「你好眼力,怎麼就看出他本相來?」八戒道:「那兩個怪正擡著水走,忽然不見,卻不是個日裡鬼?」行者道:「想是鑽進洞去了。等我去看。」

好大聖,急睜火眼金睛,漫山看處,果然不見動靜。只見那陡崖前,有一座玲瓏剔透細妝花、堆五采、三簷四簇的牌樓。他與八戒、沙僧近前觀看,上有六個大字,乃「陷空山無底洞」。行者道:「兄弟呀,這妖精把個架子支在這裡,還不知門向那裡開哩。」沙僧說:「不遠,不遠,好生尋。」都轉身看時,牌樓下,山腳下有一塊大石,約有十餘里方圓,正中間有缸口大的一個洞兒,爬得光溜溜的。八戒道:「哥啊,這就是妖精出入洞也。」行者看了道:「怪哉!我老孫自保唐僧,瞞不得你兩個,妖精也拿了些,卻不見這樣洞府。八戒,你先下去試試,看有多少淺深,我好進去救師父。」八戒搖頭道:「這個難,這個難。我老豬身子夯夯的,若塌了腳吊下去,不知二三年可得到底哩。」行者道:「就有多深麼?」八戒道:「你看。」大聖伏在洞邊上,仔細往下看處,咦!深啊,周圍足有三百餘里。回頭道:「兄弟,果然深得緊。」八戒道:「你便回去罷,師父救不得耶。」行者道:「你說那裡話?莫生懶惰意,休起怠荒心。且將行李歇下,把馬拴在牌樓柱上。你使釘鈀,沙僧使杖,攔住洞門,讓我進去打聽打聽。若師父果在裡面,我將鐵棒把妖精從裡打出,跑至門口,你兩個卻在外面擋住:這是裡應外合。打死精靈,才救得師父。」二人遵命。

行者卻將身一縱,跳入洞中,足下彩雲生萬道,身邊瑞氣護千層。不多時,到於深遠之間,那裡邊明明朗朗,一般的有日色,有風聲,又有花草果木。行者喜道:「好去處啊!想老孫出世,天賜與水簾洞,這裡也是個洞天福地。」正看時,又有一座二滴水的門樓,團團都是松竹,內有許多房舍。又想道:「此必是妖精的住處了。我且到裡邊去打聽打聽。且住。若是這般去啊,他認得我了。且變化了去。」搖身捻訣,就變做個蒼蠅兒,輕輕的飛在門樓上聽聽。只見那怪高坐在草亭內,他那模樣,比在松林裡救他,寺裡拿他,便是不同,越發打扮得俊了:
\begin{quote}
髮盤雲髻似堆鴉,身著綠絨花比甲。
一對金蓮剛半折,十指如同春筍發。
團團粉面若銀盆,朱唇一似櫻桃滑。
端端正正美人姿,月裡嫦娥還喜恰。
今朝拿住取經僧,便要歡娛同枕榻。
\end{quote}

行者且不言語,聽他說甚話。少時,綻破櫻桃,喜孜孜的叫道:「小的們,快排素筵席來,我與唐僧哥哥吃了成親。」行者暗笑道:「真個有這話!我只道八戒作耍子亂說哩。等我且飛進去尋尋,看師父在那裡。不知他的心性如何?假若被他摩弄動了啊,留他在這裡也罷。」即展翅,飛到裡邊看處,那東廊下上明下暗的紅紙格子裡面,坐著唐僧哩。

行者一頭撞破格子眼,飛在唐僧光頭上丁著,叫聲:「師父。」三藏認得聲音,叫道:「徒弟,救我命啊!」行者道:「師父不濟呀,那妖精安排筵宴,與你吃了成親哩。或生下一男半女,也是你和尚之後代,你愁怎的?」長老聞言,咬牙切齒道:「徒弟,我自出了長安,到兩界山中收你,一向西來,那個時辰動葷?那一日子有甚歪意?今被這妖精拿住,要求配偶,我若把真陽喪了,我就身墮輪迴,打在那陰山背後,永世不得翻身。」行者笑道:「莫發誓。既有真心往西天取經,老孫帶你去罷。」三藏道:「進來的路兒,我通忘了。」行者道:「莫說你忘了。他這洞,不比走進來走出去的,是打上頭往下鑽。如今救了你,要打底下往上鑽。若是造化高,鑽著洞口兒,就出去了;若是造化低,鑽不著,還有個悶殺的日子了。」三藏滿眼垂淚道:「似此艱難,怎生是好?」行者道:「沒事,沒事。那妖精整治酒與你吃,沒奈何,也吃他一鍾。只要斟得急些兒,斟起一個喜花兒來,等我變作個蟭蟟蟲兒,飛在酒泡之下。他把我一口吞下肚去,我就捻破他的心肝,扯斷他的腸肚,弄死那妖精,你才得脫身出去。」三藏道:「徒弟,這等說,只是不當人子。」行者道:「只管行起善來,你命休矣。妖精乃害人之根,你惜他怎的?」三藏道:「也罷,也罷。你只是要跟著我。」正是:
\begin{quote}
那孫大聖護定唐三藏,取經僧全靠著美猴王。
\end{quote}

他師徒兩個商量未了,早是那妖精安排停當,走近東廊外,開了門鎖,叫聲:「長老。」唐僧不敢答應。又叫一聲,又不敢答應。他不敢答應者何意?想著:「口開神氣散,舌動是非生。」卻又一條心兒想著:「若死住法兒不開口,只怕他心狠,頃刻間就害了性命。」正是那進退兩難心問口,三思忍耐口問心。正自狐疑,那怪又叫一聲:「長老。」唐僧沒奈何,應他一聲道:「娘子,有。」那長老應出這一句言來,真是肉落千斤。人都說唐僧是個真心的和尚,往西天拜佛求經,怎麼與這女妖精答話?不知此時正是危急存亡之際,萬分出於無奈,雖是外有所答,其實內無所慾。妖精見長老應了一聲,他推開門,把唐僧攙起來,和他攜手挨背,交頭接耳。你看他做出那千般嬌態,萬種風情。豈知三藏一腔子煩惱。行者暗中笑道:「我師父被他這般哄誘,只怕一時動心。」正是:
\begin{quote}
真僧魔苦遇嬌娃,妖怪娉婷實可誇。
淡淡翠眉分柳葉,盈盈丹臉襯桃花。
繡鞋微露雙鉤鳳,雲髻高盤兩鬢鴉。
含笑與師攜手處,香飄蘭麝滿袈裟。
\end{quote}

妖精挽著三藏,行近草亭道:「長老,我辦了一杯酒,和你酌酌。」唐僧道:「娘子,貧僧自不用葷。」妖精道:「我知你不吃葷,因洞中水不潔淨,特命山頭上取陰陽交媾的淨水,做些素果素菜筵席,和你耍子。」唐僧跟他進去觀看,果然見那:
\begin{quote}
盈門下,繡纏綵結;滿庭中,香噴金猊。擺列著黑油壘鈿桌,烏漆篾絲盤。壘鈿桌上,有異樣珍饈;篾絲盤中,盛稀奇素物。林檎、橄欖、蓮肉、葡萄、榧、柰、榛、松、荔枝、龍眼、山栗、風菱、棗兒、柿子、胡桃、銀杏、金橘、香橙,果子隨山有;蔬菜更時新:豆腐、麵觔、木耳、鮮筍、蘑菇、香蕈、山藥、黃精。石花菜、黃花菜,青油煎炒;扁豆角、江豆角,熟醬調成。王瓜、瓠子,白果、蔓菁。鏇皮茄子鵪鶉做,剔種冬瓜方旦名。爛煨芋頭糖拌著,白煮蘿蔔醋澆烹。椒薑辛辣般般美,鹹淡調和色色平。
\end{quote}

那妖精露尖尖之玉指,捧晃晃之金杯,滿斟美酒,遞與唐僧,口裡叫道:「長老哥哥,妙人,請一杯交歡酒兒。」三藏羞答答的接了酒,望空澆奠,心中暗祝道:「護法諸天、五方揭諦、四值功曹:弟子陳玄奘,自離東土,蒙觀世音菩薩差遣列位眾神暗中保護,拜雷音,見佛求經,今在途中,被妖精拿住,強逼成親,將這一杯酒遞與我吃。此酒果是素酒,弟子勉強吃了,還得見佛成功;若是葷酒,破了弟子之戒,永墮輪迴苦。」孫大聖,他卻變得輕巧,在耳根後,若像一個耳報;但他說話,惟三藏聽見,別人不聞。他知師父平日好吃葡萄做的素酒,教吃他一鍾。那師父沒奈何吃了,急將酒滿斟一鍾,回與妖怪。果然斟起有一個喜花兒。行者變作個蟭蟟蟲兒,輕輕的飛入喜花之下。那妖精接在手,且不吃,把杯兒放住,與唐僧拜了兩拜,口裡嬌嬌怯怯,敘了幾句情話。卻才舉杯,那花兒已散,就露出蟲來。妖精也認不得是行者變的,只以為蟲兒,用小指挑起,往下一彈。

行者見事不諧,料難入他腹,即變做個餓老鷹。真個是:
\begin{quote}
玉爪金睛鐵翮,雄姿猛氣摶雲。妖狐狡兔見他忙,千里山河時遁。饑處迎風逐雀,飽來高貼天門。老拳鋼硬最傷人,得志凌霄嫌近。
\end{quote}

飛起來,掄開玉爪,響一聲,掀翻桌席,把些素果素菜,盤碟壺爵,盡皆捽碎,撇卻唐僧,飛將出去。諕得妖精心膽皆裂,唐僧亦骨肉通酥。妖精戰戰兢兢,摟住唐僧道:「長老哥哥,此物是那裡來的?」三藏道:「貧僧不知。」妖怪道:「我費了許多心,安排這個素宴與你耍耍,卻不知這個扁毛畜生從那裡飛來,把我的家火打碎。」眾小妖道:「夫人,打碎家火猶可,將些素品都潑散在地,穢了怎用?」三藏分明曉得是行者弄法,他那裡敢說。那妖精道:「小的們,我知道了:想必是我把唐僧困住,天地不容,故降此物。你們將碎家火拾出去,另安排些酒餚,不拘葷素,我指天為媒,指地作訂,然後再與唐僧成親。」依然把長老送在東廊裡坐下不題。

卻說行者飛出去,現了本相,到於洞口,叫聲:「開門!」八戒笑道:「沙僧,哥哥來了。」他二人撒開兵器。行者跳出,八戒上前扯住道:「可有妖精?可有師父?」行者道:「有有有。」八戒道:「師父在裡邊受罪哩,綁著是綑著?要蒸是要煮?」行者道:「這個事倒沒有,只是安排素宴,要與他幹那個事哩。」八戒道:「你造化,你造化,你吃了陪親酒來了?」行者道:「獃子啊,師父的性命也難保,吃甚麼陪親酒?」八戒道:「你怎的就來了?」行者把見唐僧施變化的上項事說了一遍。道:「兄弟們,再休胡思亂想。師父已在此間,老孫這一去,一定救他出來。」

復翻身入裡面,還變做個蒼蠅兒,丁在門樓上聽之。只聞得這妖怪氣呼呼的,在亭子上吩咐:「小的們,不論葷素,拿來燒紙。借煩天地為媒訂,務要與他成親。」行者聽見,暗笑道:「這妖精全沒一些兒廉恥,青天白日的,把個和尚關在家裡擺佈。且不要忙,等老孫再進去看看。」嚶的一聲,飛在東廊之下,只見那師父坐在裡邊,清滴滴腮邊淚淌。行者鑽將進去,丁在他頭上,又叫聲:「師父。」長老認得聲音,跳起來,咬牙恨道:「猢猻啊,別人膽大,還是身包膽;你的膽大,就是膽包身。你弄變化神通,打破家火,能值幾何?鬥得那妖精淫興發了,那裡不分葷素安排,定要與我交媾,此事怎了?」行者暗中陪笑道:「師父莫怪,有救你處。」唐僧道:「那裡救得我?」行者道:「我才一翅飛起去時,見他後邊有個花園。你哄他往園裡去耍子,我救了你罷。」唐僧道:「園裡怎麼樣救?」行者道:「你與他到園裡,走到桃樹邊,就莫走了。等我飛上桃枝,變作個紅桃子。你要吃果子,先揀紅的兒摘下來。紅的是我。他必然也要摘一個,你把紅的定要讓他。他若一口吃了,我卻在他肚裡,等我搗破他的皮袋,扯斷他的肝腸,弄死他,你就脫身了。」三藏道:「你若有手段,就與他賭鬥便了,只要鑽在他肚裡怎麼?」行者道:「師父,你不知趣。他這個洞,若好出入,便可與他賭鬥;只為出入不便,曲道難行,若就動手,他這一窩子,老老小小,連我都扯住,卻怎麼了?須是這般捽手幹,大家才得乾淨。」三藏點頭聽信,只叫:「你跟定我。」行者道:「曉得,曉得,我在你頭上。」

師徒們商量定了,三藏才欠起身來,雙手扶著那格子,叫道:「娘子,娘子。」那妖精聽見,笑唏唏的跑近跟前道:「妙人哥哥,有甚話說?」三藏道:「娘子,我出了長安,一路西來,無日不山,無日不水。昨在鎮海寺投宿,偶得傷風重疾,今日出了汗,略才好些。又蒙娘子盛情,攜入仙府。只是坐了這一日,又覺心神不爽。你帶我往那裡略散散心,耍耍兒去麼。」那妖精十分歡喜道:「妙人哥哥倒有些興趣,我和你去花園裡耍耍。」叫:「小的們,拿鑰匙來開了園門,打掃路逕。」眾妖都跑去開門收拾。

這妖精開了格子,攙出唐僧。你看他許多小妖,都是油頭粉面,嬝娜娉婷,簇簇擁擁,與唐僧徑上花園而去。好和尚,他在這綺羅隊裡無他故,錦繡叢中作啞聾。若不是這鐵打的心腸朝佛去,第二個酒色凡夫也取不得經。一行都到了花園之外,那妖精俏語低聲,叫道:「妙人哥哥,這裡耍耍,真可散心釋悶。」唐僧與他攜手相攙,同入園內,擡頭觀看,其實好個去處。但見那:
\begin{quote}
縈迴曲逕,紛紛盡點蒼苔;窈窕綺窗,處處暗籠繡箔。微風初動,輕飄飄展開蜀錦吳綾;細雨才收,嬌滴滴露出冰肌玉質。日灼鮮杏,紅如仙子曬霓裳;月映芭蕉,青似太真搖羽扇。粉牆四面,萬株楊柳囀黃鸝;閑館周圍,滿院海棠飛粉蝶。更看那凝香閣、青蛾閣、解酲閣、相思閣,層層捲映,朱簾上鉤控鬚,又見那養酸亭、披素亭、畫眉亭、四雨亭,個個崢嶸,華扁上字書鳥篆。看那浴鶴池、洗觴池、怡月池、濯纓池,青萍綠藻耀金鱗;又有墨花軒、異箱軒、適趣軒、慕雲軒,玉斗瓊卮浮綠蟻。池亭上下,有太湖石、紫英石、鸚落石、錦川石,青青栽著虎鬚蒲;軒閣東西,有木假山、翠屏山、嘯風山、玉芝山,處處叢生鳳尾竹。荼架、薔薇架,近著鞦韆架,渾如錦帳羅幃;松柏亭、辛夷亭,對著木香亭,卻似碧城繡幙。芍藥欄,牡丹叢,朱朱紫紫鬥穠華;夜合臺,茉藜檻,歲歲年年生嫵媚。涓涓滴露紫含笑,堪畫堪描;豔豔燒空紅佛桑,宜題宜賦。論景致,休誇閬苑蓬萊;較芳菲,不數姚黃魏紫。若到三春閑鬥草,園中只少玉瓊花。
\end{quote}

長老攜著那怪,步賞花園,看不盡的奇葩異卉。行過了許多亭閣,真個是漸入佳境。忽擡頭,到了桃樹林邊。行者把師父頭上一掐,那長老就知。行者飛在桃樹枝兒上,搖身一變,變作個紅桃兒,其實紅得可愛。長老對妖精道:「娘子,你這苑內花香,枝頭果熟。苑內花香蜂競採,枝頭果熟鳥爭啣。怎麼這桃樹上果子青紅不一,何也?」妖精笑道:「天無陰陽,日月不明;地無陰陽,草木不生;人無陰陽,不分男女。這桃樹上果子,向陽處,有日色相烘者先熟,故紅;背陰處無日者還生,故青:此陰陽之道理也。」三藏道:「謝娘子指教,其實貧僧不知。」即向前伸手摘了個紅桃。妖精也去摘了一個青桃。三藏躬身將紅桃捧與妖怪道:「娘子,你愛色,請吃這個紅桃,拿青的來我吃。」妖精真個換了,且暗喜道:「好和尚啊,果是個真人,一日夫妻未做,卻就有這般恩愛也。」那妖精喜喜歡歡的把唐僧親敬。這唐僧把青桃拿過來就吃。那妖精喜相陪,把紅桃兒張口便咬。啟朱唇,露銀牙,未曾下口,原來孫行者十分性急,轂轆一個跟頭,翻入他咽喉之下,徑到肚腹之中。妖精害怕,對三藏道:「長老啊,這個果子利害:怎麼不容咬破,就滾下去了?」三藏道:「娘子,新開園的果子愛吃,所以去得快了。」妖精道:「未曾吐出核子,他就攛下去了。」三藏道:「娘子意美情佳,喜吃之甚,所以不及吐核,就下去了。」

行者在他肚裡,復了本相,叫聲:「師父,不要與他答嘴,老孫已得了手也。」三藏道:「徒弟方便著些。」妖精聽見道:「你和那個說話哩?」三藏道:「和我徒弟孫悟空說話哩。」妖精道:「孫悟空在那裡?」三藏道:「在你肚裡哩,卻才吃的那個紅桃子不是?」妖精慌了道:「罷了,罷了。這猴頭鑽在我肚裡,我是死也。孫行者,你千方百計的鑽在我肚裡怎的?」行者在裡邊恨道:「也不怎的,只是吃了你的六葉連肝肺,三毛七孔心,五臟都淘淨,弄做個梆子精。」妖精聽說,諕得魂飛魄散,戰戰兢兢的把唐僧抱住道:「長老啊,我只道:
\begin{quote}
夙世前緣繫赤繩,魚水相和兩意濃。
不料鴛鴦今拆散,何期鸞鳳又西東。
藍橋水漲難成事,佛廟煙沉嘉會空。
著意一場今又別,何年與你再相逢!」
\end{quote}

行者在他肚裡聽見說時,只怕長老慈心,又被他哄了,便就掄拳跳腳,支架子,理四平,幾乎把個皮袋兒搗破了。那妖精忍不得疼痛,倒在塵埃,半晌家不敢言語。行者見不言語,想是死了,卻把手略鬆一鬆。他又回過氣來,叫:「小的們在那裡?」原來那些小妖自進園門來,各人知趣,都不在一處,各自去採花鬥草,任意隨心耍子,讓那妖精與唐僧兩個自在敘情兒。忽聽得叫,卻才都跑將來。又見妖精倒在地上,面容改色,口裡哼哼的爬不動,連忙攙起,圍在一處道:「夫人,怎的不好?想是急心疼了?」妖精道:「不是,不是。你莫要問,我肚裡已有了人也。快把這和尚送出去,留我性命。」那些小妖真個都來扛擡。行者在肚裡叫道:「那個敢擡?要便是你自家獻我師父出去,出到外邊,我饒你命。」那怪精沒及奈何,只是惜命之心。急掙起來,把唐僧背在身上,拽開步,往外就走。小妖跟隨道:「老夫人,往那裡去?」妖精道:「留得五湖明月在,何愁沒處下金鉤?把這廝送出去,等我別尋一個頭兒罷。」

好妖精,一縱雲光,直到洞口。又聞得叮叮噹噹,兵刃亂響。三藏道:「徒弟,外面兵器響哩。」行者道:「是八戒揉鈀哩。你叫他一聲。」三藏便叫:「八戒。」八戒聽見道:「沙和尚,師父出來也。」二人掣開鈀、杖,妖精把唐僧馱出。咦!正是:
\begin{quote}
心猿裡應降邪怪,土木司門接聖僧。
\end{quote}

畢竟不知那妖精性命如何,且聽下回分解。
