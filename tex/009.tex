
\chapter{陳光蕊赴任逢災 江流僧復讎報本}

話表陝西大國長安城,乃歷代帝王建都之地。自周、秦、漢以來,三州花似錦,八水繞城流,真個是名勝之邦。彼時是大唐太宗皇帝登基,改元貞觀,已登極十三年,歲在己巳,天下太平,八方進貢,四海稱臣。

忽一日,太宗登位,聚集文武眾官,朝拜禮畢,有魏徵丞相出班奏道:「方今天下太平,八方寧靜,應依古法,開立選場,招取賢士,擢用人材,以資化理。」太宗道:「賢卿所奏有理。」就傳招賢文榜,頒布天下:各府州縣,不拘軍民人等,但有讀書儒流,文義明暢,三場精通者,前赴長安應試。

此榜行至海州地方,有一人,姓陳名萼,表字光蕊,見了此榜,即時回家,對母張氏道:「朝廷頒下黃榜,詔開南省,考取賢才,孩兒意欲前去應試。倘得一官半職,顯親揚名,封妻蔭子,光耀門閭,乃兒之志也。特此稟告母親前去。」張氏道:「我兒讀書人,『幼而學,壯而行』,正該如此。但去赴舉,路上須要小心,得了官,早早回來。」

光蕊便吩咐家僮收拾行李,即拜辭母親,趲程前進。到了長安,正值大開選場,光蕊就進場。考畢,中選。及廷試三策,唐王御筆親賜狀元,跨馬遊街三日。

不期遊到丞相殷開山門首,有丞相所生一女,名喚溫嬌,又名滿堂嬌,未曾婚配,正高結綵樓,拋打繡毬卜婿。適值陳光蕊在樓下經過。小姐一見光蕊人材出眾,知是新科狀元,心內十分歡喜,就將繡毬拋下,恰打著光蕊的烏紗帽。猛聽得一派笙簫細樂,十數個婢妾走下樓來,把光蕊馬頭挽住,迎狀元入相府成婚。那丞相和夫人即時出堂,喚賓人贊禮,將小姐配與光蕊。拜了天地,夫妻交拜畢,又拜了岳丈、岳母。丞相吩咐安排酒席,歡飲一宵。二人同攜素手,共入蘭房。

次日五更三點,太宗駕坐金鑾寶殿,文武眾臣趨朝。太宗問道:「新科狀元陳光蕊應授何官?」魏徵丞相奏道:「臣查所屬州郡,有江州缺官,乞我主授他此職。」太宗就命為江州州主,即令收拾起身,勿誤限期。光蕊謝恩出朝,回到相府,與妻商議,拜辭岳丈、岳母,同妻前赴江州之任。離了長安登途。

正是暮春天氣,和風吹柳綠,細雨點花紅。光蕊便道回家,同妻交拜母親張氏。張氏道:「恭喜我兒,且又娶親回來。」光蕊道:「孩兒叨賴母親福庇,忝中狀元,欽賜遊街,經過丞相殷府門前,遇拋打繡毬適中,蒙丞相即將小姐招孩兒為婿。朝廷除孩兒為江州州主,今來接取母親,同去赴任。」張氏大喜,收拾行程。

在路數日,前至萬花店劉小二家安下。張氏身體忽然染病,與光蕊道:「我身上不安,且在店中調養兩日再去。」光蕊遵命。至次日早晨,見店門前有一人提著個金色鯉魚叫賣,光蕊即將一貫錢買了。欲待烹與母親吃,只見鯉魚閃閃䁪眼。光蕊驚異道:「聞說魚蛇䁪眼,必不是等閑之物。」遂問漁人道:「這魚那裡打來的?」漁人道:「離府十五里洪江內打來的。」光蕊就把魚送在洪江裡去放了生,回店對母親道知此事。張氏道:「放生好事,我心甚喜。」光蕊道:「此店已住三日了,欽限緊急,孩兒意欲明日起身,不知母親身體好否?」張氏道:「我身子不快,此時路上炎熱,恐添疾病。你可這裡賃間房屋,與我暫住,付些盤纏在此。你兩口兒先上任去,候秋涼卻來接我。」光蕊與妻商議,就租了屋宇,付了盤纏與母親,同妻拜辭前去。

途路艱苦,曉行夜宿,不覺已到洪江渡口。只見梢子劉洪、李彪二人,撐船到岸迎接。也是光蕊前生合當有此災難,撞著這冤家。光蕊令家僮將行李搬上船去,夫妻正齊齊上船,那劉洪睜眼看見殷小姐面如滿月,眼似秋波,櫻桃小口,綠柳蠻腰,真個有沉魚落雁之容,閉月羞花之貌,陡起狼心。遂與李彪設計,將船撐至沒人煙處。候至夜靜三更,先將家僮殺死,次將光蕊打死,把尸首都推在水裡去了。小姐見他打死了丈夫,也便將身赴水。劉洪一把抱住道:「你若從我,萬事皆休;若不從時,一刀兩斷。」那小姐尋思無計,只得權時應承,順了劉洪。那賊把船渡到南岸,將船付與李彪自管,他就穿了光蕊衣冠,帶了官憑,同小姐往江州上任去了。

卻說劉洪殺死的家僮屍首,順水流去。惟有陳光蕊的屍首,沉在水底不動。有洪江口巡海夜叉見了,星飛報入龍宮,正值龍王升殿,夜叉報道:「今洪江口不知甚人把一個讀書士子打死,將屍撇在水底。」龍王叫將屍擡來,放在面前,仔細一看道:「此人正是救我的恩人,如何被人謀死?常言道:『恩將恩報。』我今日須索救他性命,以報日前之恩。」即寫下牒文一道,差夜叉徑往洪州城隍、土地處投下,要取秀才魂魄來,救他的性命。城隍、土地遂喚小鬼把陳光蕊的魂魄交付與夜叉去。夜叉帶了魂魄到水晶宮,稟見了龍王。

龍王問道:「你這秀才姓甚名誰?何方人氏?因甚到此,被人打死?」光蕊施禮道:「小生陳萼,表字光蕊,係海州弘農縣人。忝中新科狀元,叨授江州州主,同妻赴任。行至江邊上船,不料梢子劉洪貪謀我妻,將我打死拋屍。乞大王救我一救。」龍王聞言道:「原來如此。先生,你前者所放金色鯉魚,即我也。你是救我的恩人,你今有難,我豈有不救你之理?」就把光蕊屍身安置一壁,口內含一顆定顏珠,休教損壞了,日後好還魂報仇。又道:「汝今真魂,權且在我水府中做個都領。」光蕊叩頭拜謝,龍王設宴相待不題。

卻說殷小姐痛恨劉賊,恨不食肉寢皮。只因身懷有孕,未知男女,萬不得已,權且勉強相從。轉盼之間,不覺已到江州。吏書門皂,俱來迎接。所屬官員,公堂設宴相敘。劉洪道:「學生到此,全賴諸公大力匡持。」屬官答道:「堂尊大魁高才,自然視民如子,訟簡刑清。我等合屬有賴,何必過謙?」公宴已罷,眾人各散。

光陰迅速。一日,劉洪公事遠出。小姐在衙思念婆婆、丈夫,在花亭上感嘆。忽然身體困倦,腹內疼痛,暈悶在地,不覺生下一子。耳邊有人囑曰:「滿堂嬌,聽吾叮囑:吾乃南極星君,奉觀音菩薩法旨,特送此子與你。異日聲名遠大,非比等閑。劉賊若回,必害此子,汝可用心保護。汝夫已得龍王相救,日後夫妻相會,子母團圓,雪冤報仇有日也。謹記吾言。快醒,快醒。」言訖而去。

小姐醒來,句句記得,將子抱定,無計可施。忽然劉洪回來,一見此子,便要淹殺。小姐道:「今日天色已晚,容待明日拋去江中。」幸喜次早劉洪忽有緊急公事遠出。小姐暗思:「此子若待賊人回來,性命休矣。不如及早拋棄江中,聽其生死。倘或皇天見憐,有人救得,收養此子,他日還得相逢。」但恐難以識認,即咬破手指,寫下血書一紙,將父母姓名、跟腳緣由,備細開載;又將此子左腳上一個小指,用口咬下,以為記驗。取貼身汗衫一件,包裹此子,乘空抱出衙門。幸喜官衙離江不遠。小姐到了江邊,大哭一場。正欲拋棄,忽見江岸岸側飄起一片木板,小姐即朝天拜禱,將此子安在板上,用帶縛住,血書繫在胸前,推放江中,聽其所之。小姐含淚回衙不題。

卻說此子在木板上順水流去,一直流到金山寺腳下停住。那金山寺長老叫做法明和尚,修真悟道,已得無生妙訣。正當打坐參禪,忽聞得小兒啼哭之聲,一時心動,急到江邊觀看,只見涯邊一片木板上,睡著一個嬰兒。長老慌忙救起,見了懷中血書,方知來歷。取個乳名,叫做江流,託人撫養。血書緊緊收藏。

光陰似箭,日月如梭。不覺江流年長一十八歲。長老就叫他削髮修行,取法名為玄奘,摩頂受戒,堅心修道。

一日,暮春天氣,眾人同在松陰之下講經參禪,談說奧妙,那酒肉和尚恰被玄奘難倒。和尚大怒,罵道:「你這業畜,姓名也不知,父母也不識,還在此搗甚麼鬼?」玄奘被他罵出這般言語,入寺跪告師父,眼淚雙流道:「人生於天地之間,稟陰陽而資五行,盡由父生母養,豈有為人在世而無父母者乎?」再三哀告,求問父母姓名。長老道:「你真個要尋父母,可隨我到方丈裡來。」玄奘就跟到方丈。長老到重梁之上,取下一個小匣兒,打開來,取出血書一紙、汗衫一件,付與玄奘。玄奘將血書拆開讀之,才備細曉得父母姓名,並冤仇事跡。

玄奘讀罷,不覺哭倒在地道:「父母之仇,不能報復,何以為人?十八年來,不識生身父母,至今日方知有母親。此身若非師父撈救撫養,安有今日?容弟子去尋見母親,然後頭頂香盆,重建殿宇,報答師父之深恩也。」師父道:「你要去尋母,可帶這血書與汗衫前去。只做化緣,徑往江州私衙,才得你母親相見。」

玄奘領了師父言語,就做化緣的和尚,徑至江州。適值劉洪有事出外,也是天叫他母子相會,玄奘就直至私衙門口抄化。那殷小姐原來夜間得了一夢,夢見月缺再圓,暗想道:「我婆婆不知音信;我丈夫被這賊謀殺;我的兒子拋在江中,倘若有人收養,算來有十八歲矣,或今日天教相會,亦未可知。」正沉吟間,忽聽私衙前有人念經,連叫「抄化」,小姐又乘便出來問道:「你是何處來的?」玄奘答道:「貧僧乃是金山寺法明長老的徒弟。」小姐道:「你既是金山寺長老的徒弟」叫進衙來,將齋飯與玄奘吃。仔細看他舉止言談,好似與丈夫一般。

小姐將從婢打發開去,問道:「你這小師父,還是自幼出家的,還是中年出家的?姓甚名誰?可有父母否?」玄奘答道:「我也不是自幼出家,我也不是中年出家,我說起來,冤有天來大,仇有海樣深:我父被人謀死,我母卻被賊人占了。我師父法明長老教我在江州衙內尋取母親。」小姐問道:「你母姓甚?」玄奘道:「我母姓殷,名喚溫嬌。我父姓陳,名光蕊。我小名叫做江流,法名取為玄奘。」小姐道:「溫嬌就是我。但你今有何憑據?」玄奘聽說是他母親,雙膝跪下,哀哀大哭:「我娘若不信,見有血書、汗衫為證。」溫嬌取過一看,果然是真,母子相抱而哭。就叫:「我兒快去。」玄奘道:「十八年不識生身父母,今朝才見母親,教孩兒如何割捨?」小姐道:「我兒,你火速抽身前去。劉賊若回,他必害你性命。我明日假裝一病,只說先年曾許捨百雙僧鞋,來你寺中還願。那時節,我有話與你說。」玄奘依言拜別。

卻說小姐自見兒子之後,心內一憂一喜。忽一日推病,茶飯不吃,臥於床上。劉洪歸衙,問其原故。小姐道:「我幼時曾許下一願,許捨僧鞋一百雙。昨五日之前,夢見個和尚手執利刃,要索僧鞋,便覺身子不快。」劉洪道:「這些小事,何不早說?」隨升堂,吩咐王左衙、李右衙:江州城內百姓,每家要辦僧鞋一雙,限五日內完納。百姓俱依派完納訖。小姐對劉洪道:「僧鞋做完,這裡有甚麼寺院,好去還願?」劉洪道:「這江州有個金山寺、焦山寺,聽你在那個寺裡去。」小姐道:「久聞金山寺好個寺院,我就往金山寺去。」劉洪即喚王、李二衙辦下船隻。小姐帶了心腹人,同上了船,梢子將船撐開,就投金山寺去。

卻說玄奘回寺,見法明長老,把前項說了一遍。長老甚喜。次日,只見一個丫鬟先到,說夫人來寺還願。眾僧都出寺迎接。小姐徑進寺門,參了菩薩,大設齋襯。喚丫鬟將僧鞋暑襪托於盤內,來到法堂,小姐復拈心香禮拜,就教法明長老分俵與眾僧去訖。玄奘見眾僧散了,法堂上更無一人,他卻近前跪下。小姐叫他脫了鞋襪看時,那左腳上果然少了一個小指頭。當時兩個又抱住而哭,拜謝長老養育之恩。法明道:「汝今母子相會,恐奸賊知之,可速速抽身回去,庶免其禍。」小姐道:「我兒,我與你一隻香環,你徑到洪州西北地方,約有一千五百里之程,那裡有個萬花店,當時留下婆婆張氏在那裡,是你父親生身之母。我再寫一封書與你,徑到唐王皇城之內,金殿左邊,殷開山丞相家,是你母生身之父母。你將我的書遞與外公,叫外公奏上唐王,統領人馬,擒殺此賊,與父報仇,那時才救得老娘的身子出來。我今不敢久停,誠恐賊漢怪我歸遲。」便出寺登舟而去。

玄奘哭回寺中,告過師父,即時拜別,徑往洪州。來到萬花店,問那店主劉小二道:「昔年江州陳客官有一母親住在你店中,如今好麼?」劉小二道:「他原在我店中。後來昏了眼,三四年並無店租還我。如今在南門頭一個破瓦窰裡,每日上街叫化度日。那客官一去許久,到如今杳無信息,不知為何。」玄奘聽罷,即時問到南門頭破瓦窰,尋著婆婆。婆婆道:「你聲音好似我兒陳光蕊。」玄奘道:「我不是陳光蕊,我是陳光蕊的兒子。溫嬌小姐是我的娘。」婆婆道:「你爹娘怎麼不來?」玄奘道:「我爹爹被強盜打死了,我娘被強盜霸占為妻。」婆婆道:「你怎麼曉得來尋我?」玄奘道:「是我娘著我來尋婆婆。我娘有書在此,又有香環一隻。」那婆婆接了書並香環,放聲痛哭道:「我兒為功名到此,我只道他背義忘恩,那知他被人謀死。且喜得皇天憐念,不絕我兒之後,今日還有孫子來尋我。」玄奘問:「婆婆的眼,如何都昏了?」婆婆道:「我因思量你父親,終日懸望,不見他來,因此上哭得兩眼都昏了。」

玄奘便跪倒向天禱告道:「今玄奘一十八歲,父母之仇不能報復。今日領母命來尋婆婆,天若憐鑒弟子誠意,保我婆婆雙眼復明。」祝罷,就將舌尖與婆婆舔眼。須臾之間,雙眼舔開,仍復如初。婆婆覷了小和尚道:「你果是我的孫子,恰和我兒子光蕊形容無二。」婆婆又喜又悲。玄奘就領婆婆出了窰門,還到劉小二店內。將些房錢賃屋一間,與婆婆棲身。又將盤纏與婆婆道:「我此去,只月餘就回。」

隨即辭了婆婆,徑往京城。尋到皇城東街殷丞相府上,與門上人道:「小僧是親戚,來探相公。」門上人稟知丞相,丞相道:「我與和尚並無親眷。」夫人道:「我昨夜夢見我女兒滿堂嬌來家,莫不是女婿有書信回來也?」丞相便教請小和尚來到廳上。小和尚見了丞相與夫人,哭拜在地,就懷中取出一封書來,遞與丞相。丞相拆開,從頭讀罷,放聲痛哭。夫人問道:「相公,有何事故?」丞相道:「這和尚是我與你的外孫。女婿陳光蕊被賊謀死,滿堂嬌被賊強占為妻。」夫人聽罷,亦痛哭不止。丞相道:「夫人休得煩惱,來朝奏知主上,親自統兵,定要與女婿報仇。」

次日,丞相入朝,啟奏唐王曰:「今有臣婿狀元陳光蕊,帶領家小江州赴任,被梢子劉洪打死,占女為妻;假冒臣婿,為官多年。事屬異變,乞陛下立發人馬,剿除賊寇。」唐王見奏大怒,就發御林軍六萬,著殷丞相督兵前去。丞相領旨出朝,即往教場內點了兵,徑往江州進發。曉行夜宿,星落鳥飛,不覺已到江州,殷丞相兵馬俱在北岸下了營寨。星夜令金牌下戶喚到江州同知、州判二人,丞相對他說知此事,叫他提兵相助,一同過江而去。天尚未明,就把劉洪衙門圍了。劉洪正在夢中,聽得火炮一響,金鼓齊鳴,眾兵殺進私衙,劉洪措手不及,早被擒住。丞相傳下軍令,將劉洪一干人犯綁赴法場,令眾軍俱在城外安營去了。

丞相直入衙內正廳坐下,請小姐出來相見。小姐欲待要出,羞見父親,就要自縊。玄奘聞知,急急將母解救,雙膝跪下,對母道:「兒與外公統兵至此,與父報仇。今日賊已擒捉,母親何故反要尋死?母親若死,孩兒豈能存乎?」丞相亦進衙勸解。小姐道:「吾聞『婦人從一而終』。痛夫已被賊人所殺,豈可靦顏從賊?止因遺腹在身,只得忍恥偷生。今幸兒已長大,又見老父提兵報仇,為女兒者,有何面目相見?惟有一死以報丈夫耳。」丞相道:「此非我兒以盛衰改節,皆因出乎不得已,何得為恥?」父子相抱而哭,玄奘亦哀哀不止。丞相拭淚道:「你二人且休煩惱;我今已擒捉仇賊,且去發落去來。」即起身到法場。恰好江州同知亦差哨兵拿獲水賊李彪解到。丞相大喜,就令軍牢押過劉洪、李彪,每人痛打一百大棍,取了供狀,招了先年不合謀死陳光蕊情由,先將李彪釘在木驢上,推去市曹,剮了千刀,梟首示眾訖。把劉洪拿到洪江渡口,先年打死陳光蕊處。丞相與小姐、玄奘三人親到江邊,望空祭奠,活剜取劉洪心肝,祭了光蕊,燒了祭文一道。

三人望江痛哭,早已驚動水府,有巡海夜叉將祭文呈與龍王。龍王看罷,就差鱉元帥去請光蕊來到,道:「先生,恭喜,恭喜。今有先生夫人、公子同岳丈俱在江邊祭你。我今送你還魂去也。再有如意珠一顆、走盤珠二顆、絞綃十端、明珠玉帶一條奉送。你今日便可夫妻子母相會也。」光蕊再三拜謝。龍王就令夜叉將光蕊身屍送出江口還魂。夜叉領命而去。

卻說殷小姐哭奠丈夫一番,又欲將身赴水而死,慌得玄奘拚命扯住。正在倉皇之際,忽見水面上一個死屍浮來,靠近江岸之傍。小姐忙向前認看,認得是丈夫的屍首,一發嚎啕大哭不已。眾人俱來觀看,只見光蕊舒拳伸腳,身子漸漸展動,忽地爬將起來坐下。眾人不勝驚駭。光蕊睜開眼,早見殷小姐與丈人殷丞相同著小和尚俱在身邊啼哭。光蕊道:「你們為何在此?」小姐道:「因汝被賊人打死,後來妾身生下此子,幸遇金山寺長老撫養長大,尋我相會,我教他去尋外公。父親得知,奏聞朝廷,統兵到此,拿住賊人,適才生取心肝,望空祭奠我夫。不知我夫怎生又得還魂?」光蕊道:「皆因我與你昔年在萬花店時,買放了那尾金色鯉魚,誰知那鯉魚就是此處龍王。後來逆賊把我推在水中,全虧得他救我。方才又賜我還魂,送我寶物,俱在身上。更不想你生下這兒子,又得岳丈為我報仇。真是苦盡甘來,莫大之喜。」

眾官聞知,都來賀喜。丞相就令安排酒席,答謝所屬官員。即日軍馬回程。來到萬花店,那丞相傳令安營。光蕊便同玄奘到劉家店尋婆婆。那婆婆當夜得了一夢,夢見枯木開花,屋後喜鵲頻頻喧噪,想道:「莫不是我孫兒來也?」說猶未了,只見店門外,光蕊父子齊到。小和尚指道:「這不是俺婆婆?」光蕊見了老母,連忙拜倒。母子抱頭痛哭一場,把上項事說了一遍。算還了小二店錢,起程回到京城。進了相府,光蕊同小姐與婆婆、玄奘都來見了夫人。夫人不勝之喜,吩咐家僮,大排筵宴慶賀。丞相道:「今日此宴,可取名為團圓會。」真正合家歡樂。

次日早朝,唐王登殿。殷丞相出班,將前後事情備細啟奏,並薦光蕊才可大用。唐王准奏,即命陞陳萼為學士之職,隨朝理政。玄奘立意安禪,送在洪福寺內修行。後來,殷小姐畢竟從容自盡。玄奘自到金山寺中報答法明長老。

不知後來事體若何,且聽下回分解。
