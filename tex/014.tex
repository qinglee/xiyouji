
\chapter{心猿歸正 六賊無蹤}

詩曰:
\begin{quote}
佛即心兮心即佛,心佛從來皆要物。
若知無物又無心,便是真心法身佛。
法身佛,沒模樣,一顆圓光涵萬象。
無體之體即真體,無相之相即實相。
非色非空非不空,不來不向不回向。
無異無同無有無,難捨難取難聽望。
內外靈光到處同,一佛國在一沙中。
一粒沙含大千界,一個身心萬法同。
知之須會無心訣,不染不滯為淨業。
善惡千端無所為,便是南無釋迦葉。
\end{quote}

卻說那劉伯欽與唐三藏驚驚慌慌,又聞得叫聲「師父來也」。眾家僮道:「這叫的必是那山腳下石匣中老猿。」太保道:「是他,是他。」三藏問:「是甚麼老猿?」太保道:「這山舊名五行山,因我大唐王征西定國,改名兩界山。先年間曾聞得老人家說:王莽篡漢之時,天降此山,下壓著一個神猴,不怕寒暑,不吃飲食,自有土神監押,教他饑餐鐵丸,渴飲銅汁。自昔到今,凍餓不死。這叫必定是他。長老莫怕,我們下山去看來。」三藏只得依從,牽馬下山。行不數里,只見那石匣之間果有一猴,露著頭,伸著手,亂招手道:「師父,你怎麼此時才來?來得好,來得好。救我出來,我保你上西天去也。」這長老近前細看,你道他是怎生模樣:
\begin{quote}
尖嘴縮腮,金睛火眼。頭上堆苔蘚,耳中生薜蘿。鬢邊少髮多青草,頷下無鬚有綠莎。眉間土,鼻凹泥,十分狼狽;指頭粗,手掌厚,塵垢餘多。還喜得眼睛轉動,喉舌聲和。語言雖利便,身體莫能那。正是五百年前孫大聖,今朝難滿脫天羅。
\end{quote}

劉太保誠然膽大,走上前來,與他拔去了鬢邊草,頷下莎,問道:「你有甚麼說話?」那猴道:「我沒話說,教那個師父上來,我問他一問。」三藏道:「你問我甚麼?」那猴道:「你可是東土大王差往西天取經去的麼?」三藏道:「我正是,你問怎麼?」那猴道:「我是五百年前大鬧天宮的齊天大聖,只因犯了誑上之罪,被佛祖壓於此處。前者有個觀音菩薩,領佛旨意,上東土尋取經人。我教他救我一救,他勸我再莫行兇,歸依佛法,盡慇懃保護取經人,往西方拜佛,功成後自有好處。故此晝夜提心,晨昏弔膽,只等師父來救我脫身。我願保你取經,與你做個徒弟。」三藏聞言,滿心歡喜道:「你雖有此善心,又蒙菩薩教誨,願入沙門,只是我又沒斧鑿,如何救得你出?」那猴道:「不用斧鑿,你但肯救我,我自出來也。」三藏道:「我自救你,你怎得出來?」那猴道:「這山頂上有我佛如來的金字壓帖,你只上山去將帖兒揭起,我就出來了。」三藏依言,回頭央浼劉伯欽道:「太保啊,我與你上山走一遭。」伯欽道:「不知真假何如?」那猴高叫道:「是真,決不敢虛謬。」

伯欽只得呼喚家僮,牽了馬匹。他卻扶著三藏,復上高山。攀籐附葛,只行到那極巔之處,果然見金光萬道,瑞氣千條,有塊四方大石,石上貼著一封皮,卻是「唵嘛呢叭吽」六個金字。三藏近前跪下,朝石頭看著金字,拜了幾拜,望西禱祝道:「弟子陳玄奘,特奉旨意求經。果有徒弟之分,揭得金字,救出神猴,同證靈山;若無徒弟之分,此輩是個兇頑怪物,哄賺弟子,不成吉慶,便揭不得起。」祝罷又拜。拜畢,上前將六個金字輕輕揭下。只聞得一陣香風,劈手把壓帖兒刮在空中,叫道:「吾乃監押大聖者。今日他的難滿,吾等回見如來,繳此封皮去也。」嚇得個三藏與伯欽一行人望空禮拜。徑下高山,又至石匣邊,對那猴道:「揭了壓帖矣,你出來罷。」那猴歡喜,叫道:「師父,你請走開些,我好出來,莫驚了你。」

伯欽聽說,領著三藏,一行人回東即走。走了五七里遠近,又聽得那猴高叫道:「再走,再走。」三藏又行了許遠,下了山,只聞得一聲響喨,真個是地裂山崩。眾人盡皆悚懼。只見那猴早到了三藏的馬前,赤淋淋跪下,道聲:「師父,我出來也。」對三藏拜了四拜,急起身,與伯欽唱個大喏道:「有勞大哥送我師父,又承大哥替我臉上薅草。」謝畢,就去收拾行李,扣背馬匹。那馬見了他,腰軟蹄矬,戰兢兢的立站不住。蓋因那猴原是弼馬溫,在天上看養龍馬的,有些法則,故此凡馬見他害怕。

三藏見他意思,實有好心,真個像沙門中的人物,便叫:「徒弟啊,你姓甚麼?」猴王道:「我姓孫。」三藏道:「我與你起個法名,卻好呼喚。」猴王道:「不勞師父盛意,我原有個法名,叫做孫悟空。」三藏歡喜道:「也正合我們的宗派。你這個模樣,就像那小頭陀一般,我與你起個混名,稱為行者,好麼?」悟空道:「好,好,好。」自此時又稱為孫行者。

那伯欽見孫行者一心收拾要行,卻轉身對三藏唱個喏道:「長老,你幸此間收得個好徒,甚喜,甚喜。此人果然去得。我卻告回。」三藏躬身作禮相謝道:「多有拖步,感激不勝。回府多多致意令堂老夫人、令荊夫人,貧僧在府多擾,容回時踵謝。」伯欽回禮,遂此兩下分別。

卻說那孫行者請三藏上馬,他在前邊背著行李,赤條條,拐步而行。不多時,過了兩界山,忽然見一隻猛虎,咆哮剪尾而來。三藏在馬上驚心。行者在路傍歡喜道:「師父莫怕他,他是送衣服與我的。」放下行李,耳朵裡拔出一個針兒,迎著風,幌一幌,原來是個碗來粗細一條鐵棒。他拿在手中,笑道:「這寶貝,五百餘年不曾用著他,今日拿出來掙件衣服兒穿穿。」你看他拽開步,迎著猛虎,道聲:「業畜!那裡去!」那隻虎蹲著身,伏在塵埃,動也不敢動動。卻被他照頭一棒,就打的腦漿迸萬點桃紅,牙齒噴幾珠玉塊。諕得那陳玄奘滾鞍落馬,咬指道聲:「天那!天那!劉太保前日打的斑斕虎,還與他鬥了半日;今日孫悟空不用爭持,把這虎一棒打得稀爛。正是強中更有強中手!」

行者拖將虎來道:「師父略坐一坐,等我脫下他的衣服來,穿了走路。」三藏道:「他那裡有甚衣服?」行者道:「師父莫管我,我自有處置。」好猴王,把毫毛拔下一根,吹口仙氣,叫:「變!」變作一把牛耳尖刀,從那虎腹上挑開皮,往下一剝,剝下個囫圇皮來。剁去了爪甲,割下頭來,割個四四方方一塊虎皮。提起來,量了一量道:「闊了些兒,一幅可作兩幅。」拿過刀來,又裁為兩幅。收起一幅,把一幅圍在腰間。路傍揪了一條葛籐,緊緊束定,遮了下體道:「師父,且去,且去。到了人家,借些針線,再縫不遲。」他把條鐵棒捻一捻,依舊像個針兒,收在耳裡。背著行李,請師父上馬。

兩個前進,長老在馬上問道:「悟空,你才打虎的鐵棒,如何不見?」行者笑道:「師父,你不曉得。我這棍,本是東洋大海龍宮裡得來的,喚做天河鎮底神珍鐵,又喚做如意金箍棒。當年大反天宮,甚是虧他。隨身變化,要大就大,要小就小。剛才變做一個繡花針兒模樣,收在耳內矣。但用時,方可取出。」三藏聞言暗喜。又問道:「方才那虎見了你,怎麼就不動動?讓你自在打他,何說?」悟空道:「不瞞師父說,莫道是隻虎,就是一條龍,見了我也不敢無禮。我老孫頗有降龍伏虎的手段,翻江攪海的神通;見貌辨色,聆音察理;大之則量於宇宙,小之則攝於毫毛;變化無端,隱顯莫測。剝這個虎皮,何為稀罕?若到那疑難處,看展本事麼。」三藏聞得此言,愈加放懷無慮,策馬前行。

師徒兩個走著路,說著話,不覺得太陽西墜。但見:
\begin{quote}
燄燄斜暉返照,天涯海角歸雲。
千山鳥雀噪聲頻,覓宿投林成陣。
野獸雙雙對對,回窩族族群群。
一鉤新月破黃昏。萬點明星光暈。
\end{quote}

行者道:「師父走動些,天色晚了。那壁廂樹木森森,想必是人家莊院,我們趕早投宿去來。」三藏果策馬而行,徑奔人家,到了莊院前下馬。行者撇了行李,走上前,叫聲:「開門!開門!」那裡面有一老者扶筇而出,唿喇的開了門。看見行者這般惡相,腰繫著一塊虎皮,好似個雷公模樣,諕得腳軟身麻,口出譫語道:「鬼來了!鬼來了!」三藏近前攙住,叫道:「老施主休怕。他是我貧僧的徒弟,不是鬼怪。」老者擡頭,見了三藏的面貌清奇,方才立定,問道:「你是那寺裡來的和尚,帶這惡人上我門來?」三藏道:「我貧僧是唐朝來的,往西天拜佛求經。適路過此間,天晚,特造檀府借宿一宵,明早不犯天光就行。萬望方便一二。」老者道:「你雖是個唐人,那個惡的卻非唐人。」悟空厲聲高呼道:「你這個老兒全沒眼色!唐人是我師父,我是他徒弟。我也不是甚糖人,蜜人,我是齊天大聖!你們這裡人家,也有認得我的。我也曾見你來。」那老者道:「你在那裡見我?」悟空道:「你小時不曾在我面前扒柴?不曾在我臉上挑菜?」老者道:「這廝胡說!你在那裡住?我在那裡住?我來你面前扒柴、挑菜?」悟空道:「我兒子便胡說。你是認不得我了,我本是這兩界山石匣中的大聖,你再認認看。」老者方才省悟道:「你倒有些像他。但你是怎麼得出來的?」悟空將菩薩勸善,令他等待唐僧揭帖脫身之事,對那老者細說了一遍。

老者卻才下拜,將唐僧請到裡面,即喚老妻與兒女都來相見,具言前事,個個忻喜。又命看茶。茶罷,問悟空道:「大聖啊,你也有年紀了?」悟空道:「你今年幾歲了?」老者道:「我痴長一百三十歲了。」行者道:「還是我重子重孫哩。我那生身的年紀,我不記得是幾時;但只在這山腳下,已五百餘年了。」老者道:「是有,是有。我曾記得祖公公說,此山乃從天降下,就壓了一個神猴。只到如今,你才脫體。我那小時見你時,你頭上有草,臉上有泥,還不怕你;如今臉上無了泥,頭上無了草,卻像瘦了些,腰間又苫了一塊大虎皮,與鬼怪能差多少?」一家兒聽得這般話說,都呵呵大笑。

這老兒頗賢,即令安排齋飯。飯後,悟空道:「你家姓甚?」老者道:「舍下姓陳。」三藏聞言,即下來起手道:「老施主與貧僧是華宗。」行者道:「師父,你是唐姓,怎的和他是華宗?」三藏道:「我俗家也姓陳,乃是唐朝海州弘農郡聚賢莊人氏。我的法名叫做陳玄奘。只因我大唐太宗皇帝賜我做御弟三藏,指唐為姓,故名唐僧也。」那老者見說同姓,又十分歡喜。

行者道:「老陳,左右打攪你家,我有五百多年不洗澡了,你可去燒些湯來,與我師徒們洗浴洗浴,一發臨行謝你。」那老兒即令燒湯拿盆,掌上燈火。師徒浴罷,坐在燈前。行者道:「老陳,還有一事累你:有針線借我用用。」那老兒道:「有,有,有。」即教媽媽取針線來,遞與行者。行者又有眼色,見師父洗浴,脫下一件白布短小直裰未穿,他即扯過來披在身上。卻將那虎皮脫下,聯接一處。打一個馬面樣的摺子,圍在腰間,勒了籐條,走到師父面前道:「老孫今日這等打扮,比昨日如何?」三藏道:「好,好,好。這等樣,才像個行者。」三藏道:「徒弟,你不嫌殘舊,那件直裰兒,你就穿了罷。」悟空唱個喏道:「承賜,承賜。」他又去尋些草料喂了馬。此時各各事畢,師徒與那老兒亦各歸寢。

次早,悟空起來,請師父走路。三藏著衣,教行者收拾鋪蓋行李。正欲告辭,只見那老兒早具臉湯,又具齋飯。齋罷,方才起身。三藏上馬,行者引路。不覺饑餐渴飲,夜宿曉行。又值初冬時候,但見那:
\begin{quote}
霜凋紅葉千林瘦,嶺上幾株松柏秀。未開梅蕊散香幽,暖短晝,小春候,菊殘荷盡山茶茂,寒橋古樹爭枝鬥。曲澗涓涓泉水溜,淡雲欲雪滿天浮。朔風驟,牽衣袖,向晚寒威人怎受?
\end{quote}

師徒們正走多時,忽見路傍唿哨一聲,闖出六個人來,各執長槍短劍,利刃強弓,大咤一聲道:「那和尚那裡走!趕早留下馬匹,放下行李,饒你性命過去。」諕得那三藏魂飛魄散,跌下馬來,不能言語。行者用手扶起道:「師父放心,沒些兒事,這都是送衣服送盤纏與我們的。」三藏道:「悟空,你想有些耳閉。他說教我們留馬匹、行李,你倒問他要甚麼衣服、盤纏。」行者道:「你管守著衣服、行李、馬匹,待老孫與他爭持一場,看是何如。」三藏道:「好手不敵雙拳,雙拳不如四手。他那裡六條大漢,你這般小小的一個人兒,怎麼敢與他爭持?」

行者的膽量原大,那容分說,走上前來,叉手當胸,對那六個人施禮道:「列位有甚麼緣故,阻我貧僧的去路?」那人道:「我等是剪徑的大王,行好心的山主。大名久播,你量不知。早早的留下東西,放你過去;若道半個『不』字,教你碎屍粉骨。」行者道:「我也是祖傳的大王,積年的山主,卻不曾聞得列位有甚大名。」那人道:「你是不知,我說與你聽:一個喚做眼看喜,一個喚做耳聽怒,一個喚做鼻嗅愛,一個喚作舌嘗思,一個喚作意見慾,一個喚作身本憂。」悟空笑道:「原來是六個毛賊。你卻不認得我這出家人是你的主人公,你倒來擋路。把那打劫的珍寶拿出來,我與你作七分兒均分,饒了你罷。」那賊聞言,喜的喜,怒的怒,愛的愛,思的思,慾的慾,憂的憂,一齊上前亂嚷道:「這和尚無禮。你的東西全然沒有,轉來和我等要分東西。」他掄槍舞劍,一擁前來,照行者劈頭亂砍,乒乒乓乓,砍有七八十下。悟空停立中間,只當不知。那賊道:「好和尚,真個的頭硬。」行者笑道:「將就看得過罷了。你們也打得手困了,卻該老孫取出個針兒來耍耍。」那賊道:「這和尚是一個行針灸的郎中變的。我們又無病症,說甚麼動針的話?」

行者伸手去耳朵裡拔出一根繡花針兒,迎風一幌,卻是一條鐵棒,足有碗來粗細。拿在手中道:「不要走,也讓老孫打一棍兒試試手。」諕得這六個賊四散逃走。被他拽開步,團團趕上,一個個盡皆打死。剝了他的衣服,奪了他的盤纏,笑吟吟走將來道:「師父請行,那賊已被老孫剿了。」三藏道:「你十分撞禍。他雖是剪徑的強徒,就是拿到官司,也不該死罪;你縱有手段,只可退他去便了,怎麼就都打死?這卻是無故傷人的性命,如何做得和尚?出家人掃地恐傷螻蟻命,愛惜飛蛾紗罩燈。你怎麼不分皂白,一頓打死?全無一點慈悲好善之心。早還是山野中無人查考;若到城市,倘有人一時衝撞了你,你也行兇,執著棍子亂打傷人,我可做得白客,怎能脫身?」悟空道:「師父,我若不打死他,他卻要打死你哩。」三藏道:「我這出家人寧死,決不敢行兇。我就死,也只是一身,你卻殺了他六人,如何理說?此事若告到官,就是你老子做官,也說不過去。」行者道:「不瞞師父說,我老孫五百年前,據花果山稱王為怪的時節,也不知打死多少人。假似你說這般到官,倒也得些狀告,我就做不到『齊天大聖』了。」三藏道:「只因你沒收沒管,暴橫人間,欺天誑上,才受這五百年前之難。今既入了沙門,若是還像當時行兇,一味傷生,去不得西天,做不得和尚。忒惡!忒惡!」

原來這猴子一生受不得人氣。他見三藏只管緒緒叨叨,按不住心頭火發道:「你既是這等,說我做不得和尚,上不得西天,不必恁般絮聒惡我,我回去便了。」那三藏卻不曾答應。他就使一個性子,將身一縱,說一聲:「老孫去也!」三藏急擡頭,早已不見。只聞得呼的一聲,回東而去。撇得那長老孤孤零零,點頭自歎,悲怨不已道:「這廝這等不受教誨,我但說他幾句,他怎麼就無形無影的徑回去了?罷罷罷,也是我命裡不該招徒弟,進人口。如今欲尋他無處尋,欲叫他叫不應,去來,去來。」正是:捨身拚命歸西去,莫倚傍人自主張。

那長老只得收拾行李,捎在馬上,也不騎馬,一隻手拄著錫杖,一隻手揪著韁繩,淒淒涼涼,往西前進。行不多時,只見山路前面有一個年高的老母,捧一件綿衣,綿衣上有一頂花帽。三藏見他來得至近,慌忙牽馬,立於右側讓行。那老母問道:「你是那裡來的長老,孤孤恓恓獨行於此?」三藏道:「弟子乃東土大唐奉差往西天拜活佛求真經者。」老母道:「西方佛乃大雷音寺天竺國界,此去有十萬八千里路。你這等單人獨馬,又無個伴侶,又無個徒弟,你如何去得?」三藏道:「弟子日前收得一個徒弟,他性潑兇頑,是我說了他幾句,他不受教,遂渺然而去也。」老母道:「我有這一領綿布直裰、一頂嵌金花帽,原是我兒子用的,他只做了三日和尚,不幸命短身亡。我才去他寺裡哭了一場,辭了他師父,將這兩件衣、帽拿來,做個憶念。長老啊,你既有徒弟,我把這衣帽送了你罷。」三藏道:「承老母盛賜,但只是我徒弟已走了,不敢領受。」老母道:「他那廂去了?」三藏道:「我聽得呼的一聲,他回東去了。」老母道:「東邊不遠,就是我家,想必往我家去了。我那裡還有一篇咒兒,喚做『定心真言』,又名做『緊箍兒咒』。你可暗暗的念熟,牢記心頭,再莫洩漏一人知道。我去趕上他,叫他還來跟你,你卻將此衣帽與他穿戴。他若不服你使喚,你就默念此咒,他再不敢行兇,也再不敢去了。」三藏聞言,低頭拜謝。那老母化一道金光,回東而去。三藏情知是觀音菩薩授此真言,急忙撮土焚香,望東懇懇禮拜。拜罷,收了衣帽,藏在包袱中間。卻坐於路傍,誦習那定心真言,來回念了幾遍,念得爛熟,牢記心胸不題。

卻說那悟空別了師父,一觔斗雲,徑轉東洋大海。按住雲頭,分開水道,徑至水晶宮前。早驚動龍王出來迎接,接至宮裡坐下。禮畢,龍王道:「近聞得大聖難滿,失賀!想必是重整仙山,復歸古洞矣?」悟空道:「我也有此心性,只是又做了和尚了。」龍王道:「做甚和尚?」行者道:「我虧了南海菩薩勸善,教我正果,隨東土唐僧上西方拜佛,皈依沙門,又喚為行者了。」龍王道:「這等真是可賀,可賀。這才叫做改邪歸正,懲創善心。既如此,怎麼不西去,復東回何也?」行者笑道:「那是唐僧不識人性。有幾個毛賊剪徑,是我將他打死,唐僧就緒緒叨叨,說了我若干的不是。你想老孫可是受得悶氣的?是我撇了他,欲回本山,故此先來望你一望,求鍾茶吃。」龍王道:「承降,承降。」當時龍子、龍孫即捧香茶來獻。

茶畢,行者回頭一看,見後壁上掛著一幅「圯橋進履」的畫兒。行者道:「這是甚麼景致?」龍王道:「大聖在先,此事在後,故你不認得。這叫做『圯橋三進履』。」行者道:「怎的是『三進履』?」龍王道:「此仙乃是黃石公,此子乃是漢世張良,石公坐在圯橋上,忽然失履於橋下,遂喚張良取來。此子即忙取來,跪獻於前。如此三度,張良略無一毫倨傲怠慢之心,石公遂愛他勤謹,夜授天書,著他扶漢。後果然運籌帷幄之中,決勝千里之外。太平後,棄職歸山,從赤松子遊,悟成仙道。大聖,你若不保唐僧,不盡勤勞,不受教誨,到底是個妖仙,休想得成正果。」悟空聞言,沉吟半晌不語。龍王道:「大聖自當裁處,不可圖自在,誤了前程。」悟空道:「莫多話,老孫還去保他便了。」龍王忻喜道:「既如此,不敢久留,請大聖早發慈悲,莫要疏久了你師父。」

行者見他催促請行,急縱身,出離海藏,駕著雲,別了龍王。正走,卻遇著南海菩薩。菩薩道:「孫悟空,你怎麼不受教誨,不保唐僧,來此處何幹?」慌得個行者在雲端裡施禮道:「向蒙菩薩善言,果有唐朝僧到,揭了壓帖,救了我命,跟他做了徒弟。他卻怪我兇頑,我才閃他一閃,如今就去保他也。」菩薩道:「趕早去,莫錯過了念頭。」言畢各回。

這行者須臾間看見唐僧在路傍悶坐。他上前道:「師父,怎麼不走路?還在此做甚?」三藏擡頭道:「你往那裡去來?教我行又不敢行,動又不敢動,只管在此等你。」行者道:「我往東洋大海老龍王家討茶吃吃。」三藏道:「徒弟啊,出家人不要說謊。你離了我沒多一個時辰,就說到龍王家吃茶?」行者笑道:「不瞞師父說,我會駕觔斗雲,一個觔斗有十萬八千里路,故此得即去即來。」三藏道:「我略略的言語重了些兒,你就怪我,使個性子丟了我去。像你這有本事的,討得茶吃;像我這去不得的,只管在此忍餓。你也過意不去呀。」行者道:「師父,你若餓了,我便去與你化些齋吃。」三藏道:「不用化齋,我那包袱裡還有些乾糧,是劉太保母親送的。你去拿缽盂尋些水來,等我吃些兒走路罷。」

行者去解開包袱,在那包裹中間見有幾個粗麵燒餅,拿出來遞與師父。又見那光豔豔的一領綿布直裰、一頂嵌金花帽。行者道:「這衣帽是東土帶來的?」三藏就順口兒答應道:「是我小時穿戴的。這帽子若戴了,不用教經,就會念經;這衣服若穿了,不用演禮,就會行禮。」行者道:「好師父,把與我穿戴了罷。」三藏道:「只怕長短不一,你若穿得,就穿了罷。」行者遂脫下舊白布直裰,將綿布直裰穿上,也就是比量著身體裁的一般。把帽兒戴上。三藏見他戴上帽子,就不吃乾糧,卻默默的念那緊箍咒一遍。行者叫道:「頭痛,頭痛。」那師父不住的又念了幾遍,把個行者痛得打滾,抓破了嵌金的花帽。三藏又恐怕扯斷金箍,住了口不念。不念時,他就不痛了。伸手去頭上摸摸,似一條金線兒模樣,緊緊的勒在上面,取不下,揪不斷,已此生了根了。他就耳裡取出針兒來,插入箍裡,往外亂捎。三藏又恐怕他捎斷了,口中又念起來。他依舊生痛,痛得豎蜻蜓,翻觔斗,耳紅面赤,眼脹身麻。那師父見他這等,又不忍不捨,復住了口。他的頭又不痛了。行者道:「我這頭,原來是師父咒我的?」三藏道:「我念得是緊箍經,何曾咒你?」行者道:「你再念念看。」三藏真個又念。行者真個又痛,只教:「莫念,莫念。念動我就痛了。這是怎麼說?」三藏道:「你今番可聽我教誨了?」行者道:「聽教了。」「你再可無禮了?」行者道:「不敢了。」

他口裡雖然答應,心上還懷不善,把那針兒幌一幌,碗來粗細,望唐僧就欲下手。慌得長老口中又念了兩三遍。這猴子跌倒在地,丟了鐵棒,不能舉手,只教:「師父,我曉得了。再莫念,再莫念。」三藏道:「你怎麼欺心,就敢打我?」行者道:「我不曾敢打。我問師父,你這法兒是誰教你的?」三藏道:「是適間一個老母傳授我的。」行者大怒道:「不消講了,這個老母,坐定是那個觀世音。他怎麼那等害我?等我上南海打他去。」三藏道:「此法既是他授與我,他必然先曉得了。你若尋他,他念起來,你卻不是死了?」行者見說得有理,真個不敢動身,只得回心,跪下哀告道:「師父,這是他奈何我的法兒,教我隨你西去。我也不去惹他,你也莫當常言,只管念誦。我願保你,再無退悔之意了。」三藏道:「既如此,伏侍我上馬去也。」那行者才死心塌地,抖擻精神,束一束綿布直裰,扣背馬匹,收拾行李,奔西而進。

畢竟這一去,後面又有甚話說,且聽下回分解。
