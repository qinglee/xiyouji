
\chapter{色邪淫戲唐三藏 性正修持不壞身}

卻說孫大聖與豬八戒正要使法定那些婦女,忽聞得風響處,沙僧嚷鬧,急回頭時,不見了唐僧。行者道:「是甚人來搶師父去了?」沙僧道:「是一個女子,弄陣旋風,把師父攝去也。」行者聞言,唿哨跳在雲端裡,用手搭涼篷,四下裡觀看。只見一陣灰塵,風滾滾,往西北上去了。急回頭叫道:「兄弟們,快駕雲同我趕師父去來。」八戒與沙僧即把行囊捎在馬上,響一聲,都跳在半空裡去。

慌得那西梁國君臣女輩,跪在塵埃,都道:「是白日飛昇的羅漢,我主不必驚疑。唐御弟也是個有道的禪僧,我們都有眼無珠,錯認了中華男子,枉費了這場神思。請主公上輦回朝也。」女王自覺慚愧,多官都一齊回國不題。

卻說孫大聖兄弟三人騰空踏霧,望著那陣旋風,一直趕來。前至一座高山,只見灰塵息靜,風頭散了,更不知妖向何方。兄弟們按落雲霧,找路尋訪,忽見一壁廂青石光明,卻似個屏風模樣。三人牽著馬轉過石屏,石屏後有兩扇石門,門上有六個大字,乃是「毒敵山琵琶洞」。八戒無知,上前就使釘鈀築門。行者急止住道:「兄弟莫忙。我們隨旋風趕便趕到這裡,尋了這會,方遇此門,又不知深淺如何。倘不是這個門兒,卻不惹他見怪?你兩個且牽了馬,還轉石屏前立等片時,待老孫進去打聽打聽,察個有無虛實,卻好行事。」沙僧聽說,大喜道:「好好好,正是粗中有細,果然急處從寬。」他二人牽馬回頭。

孫大聖顯個神通,捻著訣,念個咒語,搖身一變,變作蜜蜂兒,真個輕巧。你看他:
\begin{quote}
翅薄隨風軟,腰輕映日纖。
嘴甜曾覓蕊,尾利善降蟾。
釀蜜功何淺,投衙禮自謙。
如今施巧計,飛舞入門簷。
\end{quote}

行者自門瑕處鑽將進去,飛過二層門裡。只見正當中花亭子上端坐著一個女妖,左右列幾個彩衣繡服、丫髻兩揫的女童,都歡天喜地,正不知講論甚麼。這行者輕輕的飛上去,釘在那花亭格子上,側耳才聽,又見兩個總角蓬頭女子,捧兩盤熱騰騰的麵食,上亭來道:「奶奶,一盤是人肉餡的葷饝饝,一盤是鄧沙餡的素饝饝。」那女怪笑道:「小的們,攙出唐御弟來。」幾個彩衣繡服的女童走向後房,把唐僧扶出。那師父面黃唇白,眼紅淚滴。行者在暗中嗟嘆道:「師父中毒了。」

那怪走下亭,露春蔥十指纖纖,扯住長老道:「御弟寬心。我這裡雖不是西梁女國的宮殿,不比富貴奢華,其實卻也清閑自在,正好念佛看經。我與你做個道伴兒,真個是百歲和諧也。」三藏不語。那怪道:「且休煩惱。我知你在女國中赴宴之時,不曾進得飲食。這裡葷素麵飯兩盤,憑你受用些兒壓驚。」三藏沉思默想道:「我待不說話,不吃東西,此怪比那女王不同:女王還是人身,行動以禮;此怪乃是妖神,恐為加害,奈何?我三個徒弟不知我困陷在於這裡,倘或加害,卻不枉丟性命?」以心問心,無計所奈,只得強打精神,開口道:「葷的何如?素的何如?」女怪道:「葷的是人肉餡饝饝,素的是鄧沙餡饝饝。」三藏道:「貧僧吃素。」那怪笑道:「女童,看熱茶來,與你家長爺爺吃素饝饝。」一女童果捧著香茶一盞,放在長老面前。那怪將一個素饝饝劈破,遞與三藏。三藏將個葷饝饝囫圇遞與女怪。女怪笑道:「御弟,你怎麼不劈破與我?」三藏合掌道:「我出家人,不敢破葷。」那女怪道:「你出家人不敢破葷,怎麼前日在子母河邊吃水高,今日又好吃鄧沙餡?」三藏道:「水高船去急,沙陷馬行遲。」

行者在格子眼聽著兩個言語相攀,恐怕師父亂了真性,忍不住,現了本相,掣鐵棒喝道:「孽畜無禮!」那女怪見了,口噴一道煙光,把花亭子罩住,教:「小的們,收了御弟。」他卻拿一柄三股鋼叉,跳出亭門,罵道:「潑猴憊𪬯!怎麼敢私入吾家,偷窺我容貌?不要走,吃老娘一叉。」這大聖使鐵棒架住,且戰且退,二人打出洞外。那八戒、沙僧正在石屏前等候,忽見他兩人爭持,慌得八戒將白馬牽過道:「沙僧,你只管看守行李、馬匹,等老豬去幫打幫打。」好獃子,雙手舉鈀,趕上前叫道:「師兄靠後,讓我打這潑賤。」那怪見八戒來,他又使個手段,呼了一聲,鼻中出火,口內生煙,把身子抖了一抖,三股叉飛舞沖迎。那女怪也不知有幾隻手,沒頭沒臉的滾將來。這行者與八戒兩邊攻住。那怪道:「孫悟空,你好不識進退。我便認得你,你是不認得我。你那雷音寺裡佛如來,也還怕我哩。量你這兩個毛人,到得那裡?都上來,一個個仔細看打。」這一場怎見得好戰:
\begin{quote}
女怪威風長,猴王氣概興。天蓬元帥爭功績,亂舉釘鈀要顯能。那一個手多叉緊煙光繞,這兩個性急兵強霧氣騰。女怪只因求配偶,男僧怎肯泄元精。陰陽不對相持鬥,各逞雄才恨苦爭。陰靜養榮思動動,陽收息衛愛清清。致令兩處無和睦,叉鈀鐵棒賭輸贏。這個棒有力,鈀更能,女怪鋼叉丁對丁。毒敵山前三不讓,琵琶洞外兩無情。那一個喜得唐僧諧鳳侶,這兩個必隨長老取真經。驚天動地來相戰,只殺得日月無光星斗更。
\end{quote}

三個戰鬥多時,不分勝負。那女怪將身一縱,使出個倒馬毒樁,不覺的把大聖頭皮上扎了一下。行者叫聲:「苦啊!」忍耐不得,負痛敗陣而走。八戒見事不諧,拖著鈀撤身而退。那怪得了勝,收了鋼叉。

行者抱頭,皺眉苦面,叫聲:「利害!利害!」八戒到跟前問道:「哥哥,你怎麼正戰到好處,卻就叫苦連天的走了?」行者抱著頭,只叫:「疼疼疼。」沙僧道:「想是你頭風發了?」行者跳道:「不是,不是。」八戒道:「哥哥,我不曾見你受傷,卻頭疼,何也?」行者哼哼的道:「了不得,了不得。我與他正然打處,他見我破了他的叉勢,他就把身子一縱,不知是件甚麼兵器,著我頭上扎了一下,就這般頭疼難禁,故此敗了陣來。」八戒笑道:「只這等靜處常誇口,說你的頭是修煉過的。卻怎麼就不禁這一下扎?」行者道:「正是。我這頭,自從修煉成真,盜食了蟠桃仙酒、老子金丹,大鬧天宮時,又被玉帝差大力鬼王、二十八宿,押赴斗牛宮處處斬,那些神將使刀斧鎚劍,雷打火燒;及老子把我安於八卦爐,煅煉四十九日:俱未傷損。今日不知這婦人用的是甚麼兵器,把老孫頭弄傷也。」沙僧道:「你放了手,等我看看,莫破了?」行者道:「不破,不破。」八戒道:「我去西梁國討個膏藥你貼貼。」行者道:「又不瘇不破,怎麼貼得膏藥?」八戒笑道:「哥啊,我的胎前產後病倒不曾有,你倒弄了個腦門癰了。」

沙僧道:「二哥且休取笑。如今天色晚矣,大哥傷了頭,師父又不知死活,怎的是好?」行者哼道:「師父沒事。我進去時,變作蜜蜂兒,飛入裡面,見那婦人坐在花亭子上。少頃,兩個丫鬟捧兩盤饝饝:一盤是人肉餡,葷的;一盤是鄧沙餡,素的。又著兩個女童扶師父出來吃一個壓驚,又要與師父做甚麼道伴兒。師父始初不與那婦人答話,也不吃饝饝。後見他甜言美語,不知怎麼,就開口說話,卻說吃素的。那婦人就將一個素的劈開,遞與師父。師父將個囫圇葷的遞與那婦人。婦人道:『怎不劈破?』師父道:『出家人不敢破葷。』那婦人道:『既不破葷,前日怎麼在子母河邊飲水高,今日又好吃鄧沙餡?』師父不解其意,答他兩句道:『水高船去急,沙陷馬行遲。』我在格子上聽見,恐怕師父亂性,便就現了原身,掣棒就打。他也使神通,噴出煙霧,叫『收了御弟』,就掄鋼叉,與老孫打出洞來也。」沙僧聽說,咬指道:「這潑賤也不知從那裡就隨將我們來,把上項事情都知道了。」

八戒道:「這等說,便我們安歇不成。莫管甚麼黃昏半夜,且去他門上索戰,嚷嚷鬧鬧,攪他個不睡,莫教他捉弄了我師父。」行者道:「頭疼,去不得。」沙僧道:「不須索戰:一則師兄頭痛;二來我師父是個真僧,決不以色空亂性。且就在山坡下,閉風處坐這一夜,養養精神,待天明再作理會。」遂此三個弟兄拴牢白馬,守護行囊,就在坡下安歇不題。

卻說那女怪放下兇惡之心,重整歡愉之色,叫:「小的們,把前後門都關緊了。」又使兩個支更,防守行者,但聽門響,即時通報。卻又教:「女童,將臥房收拾齊整,掌燭焚香,請唐御弟來,我與他交歡。」遂把長老從後邊攙出。那女怪弄出十分嬌媚之態,攜定唐僧道:「常言:『黃金未為貴,安樂值錢多。』且和你做會夫妻兒耍子去也。」這長老咬定牙關,聲也不透。欲待不去,恐他生心害命,只得戰兢兢,跟著他步入香房。卻如痴如啞,那裡擡頭舉目,更不曾看他房裡是甚床鋪幔帳,也不知有甚箱籠梳妝。那女怪說出的雨意雲情,亦漠然無聽。好和尚,真是:
\begin{quote}
目不視惡色,耳不聽淫聲。他把這錦繡嬌容如糞土,金珠美貌若灰塵。一生只愛參禪,半步不離佛地。那裡會惜玉憐香,只曉得修真養性。那女怪活潑潑,春意無邊;這長老死丁丁,禪機有在。一個似軟玉溫香,一個如死灰槁木。那一個展鴛衾,淫興濃濃;這一個束褊衫,丹心耿耿。那個要貼胸交股和鸞鳳,這個要面壁歸山訪達摩。女怪解衣,賣弄他肌香膚膩;唐僧斂衽,緊藏了糙肉粗皮。女怪道:「我枕剩衾閑何不睡?」唐僧道:「我頭光服異怎相陪?」那個道:「我願作前朝柳翠翠。」這個道:「貧僧不是月闍黎。」女怪道:「我美若西施還嬝娜。」唐僧道:「我越王因此久埋屍。」女怪道:「御弟,你記得『寧教花下死,做鬼也風流?』」唐僧道:「我的真陽為至寶,怎肯輕與你這粉骷髏」
\end{quote}

他兩個散言碎語的,直鬥到更深,唐長老全不動念。那女怪扯扯拉拉的不放,這師父只是老老成成的不肯。直纏到有半夜時候,把那怪弄得惱了,叫:「小的們,拿繩來。」可憐將一個心愛的人兒,一條繩,綑的像個猱獅模樣。又教拖在房廊下去,卻吹滅銀燈,各歸寢處。一夜無詞。

不覺的雞聲三唱。那山坡下孫大聖欠身道:「我這頭疼了一會,到如今也不疼不麻,只是有些作癢。」八戒笑道:「癢便再教他扎一下,何如?」行者啐了一口道:「放放放。」八戒又笑道:「放放放,我師父這一夜倒浪浪浪。」沙僧道:「且莫鬥口。天亮了,快趕早兒捉妖怪去。」行者道:「兄弟,你只管在此守馬,休得動身。豬八戒跟我去。」

那獃子抖擻精神,束一束皂錦直裰,相隨行者,各帶了兵器,跳上山崖,徑至石屏之下。行者道:「你且立住。只怕這怪物夜裡傷了師父,先等我進去打聽打聽。倘若被他哄了,喪了元陽,真個虧了德行,卻就大家散火;若不亂性情,禪心未動,卻好努力相持,打死精怪,救師西去。」八戒道:「你好痴啞。常言道:『乾魚可好與貓兒作枕頭?』就不如此,就不如此,也要抓你幾把是。」行者道:「莫胡疑亂說,待我看去。」

好大聖,轉石屏,別了八戒,搖身還變個蜜蜂兒,飛入門裡。見那門裡有兩個丫鬟,頭枕著梆鈴,正然睡哩。卻到花亭子觀看,那妖精原來弄了半夜,都辛苦了,一個個都不知天曉,還睡著哩。行者飛來後面,影影的只聽見唐僧聲喚。忽擡頭,見那房廊下四馬攢蹄綑著師父。行者輕輕的釘在唐僧頭上,叫:「師父。」唐僧認得聲音,道:「悟空來了?快救我命。」行者道:「夜來好事如何?」三藏咬牙道:「我寧死也不肯如此。」行者道:「昨日我見他有相憐相愛之意,卻怎麼今日把你這般挫折?」三藏道:「他把我纏了半夜,我衣不解帶,身未沾床。他見我不肯相從,才綑我在此。你千萬救我取經去也。」

他師徒們正然問答,早驚醒了那個妖精。妖精雖是下狠,卻還有流連不捨之意。一覺翻身,只聽見「取經去也」一句,他就滾下床來,厲聲高叫道:「好夫妻不做,卻取甚麼經去?」

行者慌了,撇卻師父,急展翅,飛將出去,現了本相,叫聲:「八戒。」那獃子轉過石屏道:「那話兒成了否?」行者笑道:「不曾,不曾。老師父被他摩弄不從,惱了綑在那裡。正與我訴說前情,那怪驚醒了,我慌得出來也。」八戒道:「師父曾說甚來?」行者道:「他只說衣不解帶,身未沾床。」八戒笑道:「好好好,還是個真和尚!我們救他去。」

獃子粗鹵,不容分說,舉釘鈀,望他那石頭門上盡力氣一鈀,唿喇喇築做幾塊。諕得那幾個枕梆鈴睡的丫鬟跑至二層門外,叫聲:「開門,前門被昨日那兩個醜男人打破了!」那女怪正出房門,只見四五個丫鬟跑進去報道:「奶奶,昨日那兩個醜男人又來把前門已打碎矣。」那怪聞言,即忙叫:「小的們,燒湯洗面梳妝。」叫:「把御弟連繩擡在後房收了。等我打他去。」

好妖精,走出來,舉著三股叉,罵道:「潑猴!野彘!老大無知。你怎敢打破我門?」八戒罵道:「濫淫賤貨!你倒困陷我師父,返敢硬嘴。我師父是你哄將來做老公的,快快送出饒你;敢再說半個『不』字,老豬一頓鈀,連山也築倒你的。」那妖精那容分說,抖擻身軀,依前弄法,鼻口內噴煙冒火,舉鋼叉就刺八戒。八戒側身躲過,著鈀就築;孫大聖使鐵棒並力相幫。那怪又弄神通,也不知是幾隻手,左右遮攔。交鋒三五個回合,不知是甚兵器,把八戒嘴唇上也扎了一下。那獃子拖著鈀,侮著嘴,負痛逃生。行者卻也有些醋他,虛丟一棒,敗陣而走。那怪得勝而回,叫小的們搬石塊壘疊了前門不題。

卻說那沙和尚正在坡前放馬,只聽得那裡豬哼。忽擡頭,見八戒侮著嘴,哼將來。沙僧道:「怎的說?」獃子哼道:「了不得,了不得。疼疼疼。」說不了,行者也到跟前,笑道:「好獃子啊,昨日咒我是腦門癰,今日卻也弄做個瘇嘴瘟了。」八戒哼道:「難忍難忍,疼得緊,利害利害。」

三人正然難處,只見一個老媽媽兒,左手提著一個青竹籃兒,自南山路上挑菜而來。沙僧道:「大哥,那媽媽來得近了,等我問他個信兒,看這個是甚妖精,是甚兵器,這般傷人?」行者道:「你且住,等老孫問他去來。」行者急睜睛看,只見頭直上有祥雲蓋頂,左右有香霧籠身。行者認得,即叫:「兄弟們,還不來叩頭,那媽媽是菩薩來也。」慌得豬八戒忍疼下拜,沙和尚牽馬躬身,孫大聖合掌跪下,叫聲:「南無大慈大悲救苦救難靈感觀世音菩薩。」

那菩薩見他們認得元光,即踏祥雲,起在半空,現了真像,原來是魚籃之像。行者趕到空中,拜告道:「菩薩,恕弟子失迎之罪。我等努力救師,不知菩薩下降。今遇魔難難收,萬望菩薩搭救搭救。」菩薩道:「這妖精十分利害。他那三股叉是生成的兩隻鉗腳。扎人痛者,是尾上一個鉤子,喚做倒馬毒。本身是個蝎子精。他前者在雷音寺聽佛談經,如來見了,不合用手推他一把,他就轉過鉤子,把如來左手中拇指上扎了一下。如來也疼難禁,即著金剛拿他。他卻在這裡。若要救得唐僧,除是別告一位方好,我也是近他不得。」行者再拜道:「望菩薩指示指示,別告那位去好?弟子即去請他也。」菩薩道:「你去東天門裡光明宮告求昴日星官,方能降伏。」言罷,遂化作一道金光,徑回南海。

孫大聖才按雲頭,對八戒、沙僧道:「兄弟放心,師父有救星了。」沙僧道:「是那裡救星?」行者道:「才然菩薩指示,教我告請昴日星官。老孫去來。」八戒侮著嘴哼道:「哥啊,就問星官討些止疼的藥餌來。」行者笑道:「不須用藥,只似我昨日疼過夜就好了。」沙僧道:「不必煩絮,快早去罷。」

好行者,急忙駕觔斗雲,須臾到東天門外。忽見增長天王當面作禮道:「大聖何往?」行者道:「因保唐僧西方取經,路遇魔障纏身,要到光明宮見昴日星官走走。」忽又見陶、張、辛、鄧四大元帥,也問何往。行者道:「要尋昴日星官去降妖救師。」四元帥道:「星官今早奉玉帝旨意,上觀星臺巡察去了。」行者道:「可有這話?」辛天君道:「小將等與他同下斗牛宮,豈敢說假?」陶天君道:「今已許久,或將回矣。大聖還先去光明宮,如未回,再去觀星臺可也。」

大聖遂喜,即別他們。至光明宮門首,果是無人,復抽身就走,只見那壁廂有一行兵士擺列,後面星官來了。那星官還穿的是拜駕朝衣,一身金縷。但見他:
\begin{quote}
冠簪五岳金光彩,笏執山河玉色瓊。
袍掛七星雲靉靆,腰圍八極寶環明。
叮噹珮響如敲韻,迅速風聲似擺鈴。
翠羽扇開來昴宿,天香飄襲滿門庭。
\end{quote}

前行的兵士看見行者立於光明宮外,急轉身報道:「主公,孫大聖在這裡也。」那星官斂雲霧整束朝衣,停執事分開左右,上前作禮道:「大聖何來?」行者道:「專來拜煩救師父一難。」星官道:「何難?在何地方?」行者道:「在西梁國毒敵山琵琶洞。」星官道:「那山洞有甚妖怪,卻來呼喚小神?」行者道:「觀音菩薩適才顯化,說是一個蝎子精,特舉先生方能治得,因此來請。」星官道:「本欲回奏玉帝,奈大聖至此,又感菩薩舉薦,恐遲誤事,小神不敢請獻茶,且和你去降妖精,卻再來回旨罷。」

大聖聞言,即同出東天門,直至西梁國,望見毒敵山不遠,行者指道:「此山便是。」星官按下雲頭,同行者至石屏前山坡之下。沙僧見了道:「二哥起來,大哥請得星官來了。」那獃子還侮著嘴道:「恕罪,恕罪。有病在身,不能行禮。」星官道:「你是個修行之人,何病之有?」八戒道:「早間與那妖精交戰,被他著我唇上扎了一下,至今還疼哩。」星官道:「你上來,我與你醫治醫治。」獃子才放了手,口裡哼哼唧唧道:「千萬治治,待好了謝你。」那星官用手把嘴唇上摸了一摸,吹一口氣,就不疼了。獃子歡喜下拜道:「妙啊!妙啊!」行者笑道:「煩星官也把我頭上摸摸。」星官道:「你未遭毒,摸他何為?」行者道:「昨日也曾遭過,只是過了夜,才不疼。如今還有些麻癢,只恐發天陰,也煩治治。」星官真個也把頭上摸了一摸,吹口氣,也就解了餘毒,不麻不癢了。八戒發狠道:「哥哥,去打那潑賤去。」星官道:「正是,正是。你兩個叫他出來,等我好降他。」

行者與八戒跳上山坡,又至石屏之後。獃子口裡亂罵「手似撈鉤」,一頓釘鈀,把那洞門外壘疊的石塊爬開。闖至一層門,又一釘鈀,將二門築得粉碎。慌得那門裡小妖飛報:「奶奶,那兩個醜男人又把二層門也打破了。」那怪正教解放唐僧,討素茶飯與他吃哩。聽見打破二門,即便跳出花亭子,掄叉來刺八戒;八戒使釘鈀迎架;行者在傍,又使鐵棒來打。那怪趕至身邊,要下毒手;行者與八戒識得方法,回頭就走。

那妖怪趕過石屏之後,行者叫聲:「昴宿何在?」只見那星官立於山坡之上,現出本相,原來是一隻雙冠子大公雞,昂起頭來,約有六七尺高,對著妖怪叫了一聲。那怪即時就現了本相,原來是個琵琶來大小的一個蝎子精。這星官再叫一聲,那怪渾身酥軟,死在坡前。有詩為證。詩曰:
\begin{quote}
花冠繡頸若團纓,爪硬距長目怒睛。
踴躍雄威全五德,崢嶸壯勢羨三鳴。
豈如凡鳥啼茅屋,本是天星顯聖名。
毒蝎枉修人道行,還原反本見真形。
\end{quote}

八戒上前,一隻腳屣住那怪的胸背道:「孽畜!今番使不得倒馬毒了。」那怪動也不動,被獃子一頓釘鈀,搗作一團爛醬。那星官復聚金光,駕雲而去。行者與八戒、沙僧朝天拱謝道:「有累,有累。改日赴宮拜酬。」

三人謝畢卻才收拾行李、馬匹,都進洞裡。見那大小丫鬟兩邊跪下,拜道:「爺爺,我們不是妖邪,都是西梁國女人,前者被這妖精攝來的。你師父在後邊香房裡坐著哭哩。」行者聞言,仔細觀看,果然不見妖氣。遂入後邊叫道:「師父。」那唐僧見眾齊來,十分歡喜道:「賢徒,累及你們了。那婦人何如也?」八戒道:「那廝原是個大母蝎子。幸得觀音菩薩指示,大哥去天宮裡請得那昴日星官下降,把那廝收伏。才被老豬築做個泥了,方敢深入於此,得看師父之面。」唐僧謝之不盡。又尋些素米、素麵,安排了飲食,吃了一頓。把那些攝將來的女子趕下山,指與回家之路。點上一把火,把幾間房宇燒毀罄盡。請唐僧上馬,找尋大路西行。正是:
\begin{quote}
割斷塵緣離色相,推乾金海悟禪心。
\end{quote}

畢竟不知幾年上才得成真,且聽下回分解。
