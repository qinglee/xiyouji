
\chapter{心猿空用千般計 水火無功難煉魔}

話說齊天大聖空著手敗了陣,來坐於金山後,撲梭梭兩眼滴淚,叫道:「師父啊,指望和你:
\begin{quote}
佛恩有德有和融,同幼同生意莫窮。
同住同修同解脫,同慈同念顯靈功。
同緣同相心真契,同見同知道轉通。
豈料如今無主杖,空拳赤腳怎興隆!
\end{quote}

大聖悽慘多時,心中暗想道:「那妖精認得我,我記得他在陣上誇獎道:『真個是鬧天宮之類!』這等啊,決不是凡間怪物,定然是天上兇星,想因思凡下界。又不知是那裡降下來魔頭,且須上界去查勘查勘。」

行者這才是以心問心,自張自主,急翻身,縱起祥雲,直至南天門外。忽擡頭見廣目天王,當面迎著長揖道:「大聖何往?」行者道:「有事要見玉帝。你在此何幹?」廣目道:「今日輪該巡視南天門。」說未了,又見那馬、趙、溫、關四大元帥作禮道:「大聖,失迎。請待茶。」行者道:「有事哩。」遂辭了廣目並四元帥,徑入南天門裡,直至靈霄殿外,果又見張道陵、葛仙翁、許旌陽、丘弘濟四天師,並南斗六司、北斗七元,都在殿前迎著行者,一齊起手道:「大聖如何到此?」又問:「保唐僧之功完否?」行者道:「早哩,早哩,路遙魔廣,才有一半之功。見如今阻住在金山金洞。有一個兕怪,把唐師父拿於洞裡,是老孫尋上門與他交戰一場,那廝神通廣大,把老孫的金箍棒搶去了,因此難縛魔王。疑是上界那個兇星思凡下界,又不知是那裡降來的魔頭,老孫因此來尋尋玉帝,問他個鉗束不嚴。」許旌陽笑道:「這猴頭還是如此放刁。」行者道:「不是放刁,我老孫一生是這口兒緊些,才尋的著個頭兒。」張道陵道:「不消多說,只與他傳報便了。」行者道:「多謝,多謝。」

當時四天師傳奏靈霄,引見玉陛。行者朝上唱個大喏道:「老官兒,累你,累你。我老孫保護唐僧往西天取經,一路凶多吉少,也不消說。於今來在金山,金洞,有一兕怪,把唐僧拿在洞裡,不知是要蒸,要煮,要晒。是老孫尋上他門,與他交戰,那怪卻就有些認得老孫,卓是神通廣大,把老孫的金箍棒搶去,因此難縛妖魔。疑是上天兇星,思凡下界,為此老孫特來啟奏。伏乞天尊垂慈洞鑒,降旨查勘兇星,發兵收剿妖魔,老孫不勝戰慄屏營之至。」卻又打個深躬道:「以聞。」傍有葛仙翁笑道:「猴子是何前倨後恭?」行者道:「不敢,不敢。不是甚前倨後恭,老孫於今是沒棒弄了。」

彼時玉皇天尊聞奏,即忙降旨可韓司知道:「既如悟空所奏,可隨查諸天星斗、各宿神王,有無思凡下界,隨即覆奏施行,以聞。」可韓丈人真君領旨,當時即同大聖去查。先查了四天門門上神王官吏。次查了三微垣垣中大小群真。又查了雷霆官將陶、張、辛、鄧,苟、畢、龐、劉。最後才查三十三天,天天自在。又查二十八宿:東七宿:角、亢、氐、房、參、尾、箕;西七宿:斗、牛、女、虛、危、室、壁;南七宿、北七宿:宿宿安寧。又查了太陽、太陰、水、火、木、金、土七政;羅睺、計都、炁孛四餘。滿天星斗,並無思凡下界。行者道:「既是如此,我老孫也不消上那靈霄寶殿。打攪玉皇大帝,深為不便。你自回旨去罷,我只在此等你回話便了。」那可韓丈人真君依命。孫行者等候良久,作詩紀興曰:
\begin{quote}
風清雲霽樂昇平,神靜星明顯瑞禎。
河漢安寧天地泰,五方八極偃戈旌。
\end{quote}

那可韓司丈人真君歷歷查勘,回奏玉帝道:「滿天星宿不少,各方神將皆存,並無思凡下界者。」玉帝聞奏:「著孫悟空挑選幾員天將,下界擒魔去也。」

四大天師奉旨意,即出靈霄寶殿,對行者道:「大聖啊,玉帝寬恩,言天宮無神思凡,著你挑選幾員天將,擒魔去哩。」行者低頭暗想道:「天上將不如老孫者多,勝似老孫者少。想我鬧天宮時,玉帝遣十萬天兵,佈天羅地網,更不曾有一將敢與我比手。向後來,調了小聖二郎,方是我的對手。如今那怪物手段又強似老孫,卻怎麼得能夠取勝?」許旌陽道:「此一時,彼一時,大不同也。常言道:『一物降一物』哩。你好違了旨意?但憑高見,選用天將,勿得遲疑誤事。」行者道:「既然如此,深感上恩,果是不好違旨;一則老孫又不可空走這遭。煩旌陽轉奏玉帝,只教托塔李天王與哪吒太子去罷,他還有幾件降妖兵器,且下界與那怪見一仗,以看如何。果若能擒得他,是老孫之幸;若不能,那時再作區處。」

真個那天師啟奏了玉帝,玉帝即令李天王父子率領眾部天兵,與行者助力。那天王即奉旨來會行者。行者又對天師道:「蒙玉帝遣差天王,謝謝不盡。還有一事,再煩轉達:但得兩個雷公使用,等天王戰鬥之時,教雷公在雲端裡下個雷㨝,照頂門上錠死那妖魔,深為良計也。」天師笑道:「好好好。」天師又奏玉帝,傳旨教九天府下點鄧化、張蕃二雷公,與天王合力縛妖救難。遂與天王、孫大聖徑下南天門外。

頃刻而到。行者道:「此山便是金山。山中間乃是金洞。列位商議,卻教那個先去索戰?」天王停下雲頭,扎住天兵在於山南坡下道:「大聖素知小兒哪吒,曾降九十六洞妖魔,善能變化,隨身有降妖兵器,須教他先去出陣。」行者道:「既如此,等老孫引太子去來。」

那太子抖擻雄威,與大聖跳在高山,徑至洞口,但見那洞門緊閉,崖下無精。行者上前高叫:「潑魔!快開門,還我師父來也。」那洞裡把門的小妖看見,急報道:「大王,孫行者領著一個小童男,在門前叫戰哩。」那魔王道:「這猴子鐵棒被我奪了,空手難爭,想是請得救兵來也。」叫:「取兵器。」魔王綽槍在手,走到門外觀看,那小童男生得相貌清奇,十分精壯。真個是:
\begin{quote}
玉面嬌容如滿月,朱脣方口露銀牙。
眼光掣電睛珠暴,額闊凝霞髮髻髽。
繡帶舞風飛彩焰,錦袍映日放金花。
環絛灼灼攀心鏡,寶甲輝輝襯戰靴。
身小聲洪多壯麗,三天護教惡哪吒。
\end{quote}

魔王笑道:「你是李天王第三個孩兒,名喚做哪吒太子,卻如何到我這門前呼喝?」太子道:「因你這潑魔作亂,困害東土聖僧,奉玉帝金旨,特來拿你。」魔王大怒道:「你想是孫悟空請來的。我就是那聖僧的魔頭哩。量你這小兒曹有何武藝,敢出朗言?不要走,吃吾一槍。」

這太子使斬妖劍,劈手相迎。他兩個搭上手,卻才賭鬥,那大聖急轉山坡,叫:「雷公何在?快早去,著妖魔下個雷㨝,助太子降伏來也。」鄧、張二公即踏雲光,正欲下手,只見那太子使出法來,將身一變,變作三頭六臂,手持六般兵器,望妖魔砍來;那魔王也變作三頭六臂,三柄長槍抵住。這太子又弄出降妖法力,將六般兵器拋將起去。是那六般兵器?卻是砍妖劍、斬妖刀、縛妖索、降魔杵、繡球、火輪兒。大叫一聲:「變!」一變十,十變百,百變千,千變萬,都是一般兵器,如驟雨冰雹,紛紛密密,望妖魔打將去。那魔王公然不懼,一隻手取出那白森森的圈子來,望空拋起,叫聲:「著!」唿喇的一下,把六般兵器套將下來。慌得那哪吒太子赤手逃生。魔王得勝而回。

鄧、張二雷公在空中暗笑道:「早是我先看頭勢,不曾放了雷㨝。假若被他套將去,卻怎麼回見天尊?」二公按落雲頭,與太子來山南坡下,對李天王道:「妖魔果神通廣大。」悟空在傍笑道:「那廝神通也只如此,爭奈那個圈子利害。不知是甚麼寶貝,丟起來善套諸物。」哪吒恨道:「這大聖甚不成人。我等折兵敗陣,十分煩惱,都只為你,你反喜笑何也?」行者道:「你說煩惱,終然我老孫不煩惱?我如今沒計奈何,哭不得,所以只得笑也。」

天王道:「似此怎生結果?」行者道:「憑你等再怎計較;只是圈子套不去的,就可拿住他了。」天王道:「套不去者,惟水火最利。常言道:『水火無情。』」行者聞言道:「說得有理。你且穩坐在此,待老孫再上天走走來。」鄧、張二公道:「又去做甚的?」行者道:「老孫這去,不消啟奏玉帝,只到南天門裡,上彤華宮,請熒惑火德星君來此放火,燒那怪物一場,或者連那圈子燒做灰燼,捉住妖魔。一則取兵器還汝等歸天,二則可解脫吾師之難。」太子聞言甚喜道:「不必遲疑,請大聖早去早來,我等只在此拱候。」

行者縱起祥光,又至南天門外。那廣目與四將迎道:「大聖如何又來?」行者道:「李天王著太子出師,只一陣,被那魔王把六件兵器撈了去了。我如今要到彤華宮請火德星君助陣哩。」四將不敢久留,讓他進去。至彤華宮,只見那火部眾神,即入報道:「孫悟空欲見主公。」那南方三炁火德星君整衣出門迎進道:「昨日可韓司查點小宮,更無一人思凡。」行者道:「已知。但李天王與太子敗陣,失了兵器,特來請你救援救援。」星君道:「那哪吒乃三壇海會大神,他出身時,曾降九十六洞妖魔,神通廣大,若他不能,小神又怎敢望也?」行者道:「因與李天王計議,天地間至利者,惟水火也。那怪物有一個圈子,善能套人的物件,不知是甚麼寶貝,故此說火能滅諸物,特請星君領火部到下方縱火燒那妖魔,救我師父一難。」

火德星君聞言,即點本部神兵,同行者到金山南坡下,與天王、雷公等相見了。天王道:「孫大聖,你還去叫那廝出來,等我與他交戰,待他拿動圈子,我卻閃過,教火德帥眾燒他。」行者笑道:「正是,我和你去來。」火德共太子、鄧、張二公立於高峰之上,與他眺戰。

這大聖到了金洞口,叫聲:「開門!快早還我師父。」那小怪又急通報道:「孫悟空又來了。」那魔帥眾出洞,見了行者道:「你這潑猴,又請了甚麼兵來耶?」這壁廂轉上托塔天王喝道:「潑魔頭!認得我麼?」魔王笑道:「李天王,想是要與你令郎報仇,欲討兵器麼?」天王道:「一則報仇要兵器,二來是拿你救唐僧。不要走,吃吾一刀。」那怪物側身躲過,挺長槍,隨手相迎。他兩個在洞前這場好殺。你看那:
\begin{quote}
天王刀砍,妖怪槍迎。刀砍霜光噴烈火,槍迎銳氣迸愁雲。一個是金山生成的惡怪,一個是靈霄殿差下的天神。那一個因欺禪性施威武,這一個為救師災展大倫。天王使法飛沙石,魔怪爭強播土塵。播土能教天地暗,飛沙善著海江渾。兩家努力爭功績,皆為唐僧拜世尊。
\end{quote}

那孫大聖見他兩個交戰,即轉身跳上高峰,對火德星君道:「三炁用心者。」你看那個妖魔與天王正鬥到好處,卻又取出圈子來。天王看見,即撥祥光,敗陣而走。這高峰上火德星君忙傳號令,教眾部火神一齊放火。這一場真個利害,好火:
\begin{quote}
經云:「南方者火之精也。」雖星星之火,能燒萬頃之田;乃三炁之威,能變百端之火。今有火槍、火刀、火弓、火箭,各部神祇,所用不一。但見那半空中火鴉飛噪,滿山頭火馬奔騰。雙雙赤鼠,對對火龍。雙雙赤鼠噴烈焰,萬里通紅;對對火龍吐濃煙,千方共黑。火車兒推出,火葫蘆撒開。火旗搖動一天霞,火棒攪行盈地燎。說甚麼甯戚鞭牛,勝強似周郎赤壁。這個是天火非凡真利害,烘烘火風紅。
\end{quote}

那妖魔見火來時,全無恐懼。將圈子望空拋起,唿喇一聲,把這火龍、火馬、火鴉、火鼠、火槍、火刀、火弓、火箭,一圈子又套將下去,轉回本洞,得勝收兵。

這火德星君手執著一桿空旗,招回眾將,會合天王等,坐於山南坡下,對行者道:「大聖啊,這個兇魔真是罕見。我今折了火具,怎生是好?」行者笑道:「不須報怨。列位且請寬坐坐,老孫再去去來。」天王道:「你又往那裡去?」行者道:「那怪物既不怕火,斷然怕水。常言道:『水能剋火。』等老孫去北天門裡,請水德星君施佈水勢,往他洞裡一灌,把魔王渰死,取物件還你們。」天王道:「此計雖妙,但恐連你師父都淹死也。」行者道:「沒事。渰死我師,我自有個法兒教他活來。如今稽遲列位,甚是不當。」火德道:「既如此,且請行,請行。」

好大聖,又駕觔斗雲,徑到北天門外。忽擡頭,見多聞天王向前施禮道:「孫大聖何往?」行者道:「有一事要入烏浩宮見水德星君。你在此作甚?」多聞道:「今日輪該巡視。」正說處,又見那龐、劉、苟、畢四大天將進禮邀茶。行者道:「不勞,不勞,我事急矣。」遂別卻門神,直至烏浩宮,著水部眾神即時通報。眾神報道:「齊天大聖孫悟空來了。」水德星君聞言,即將查點四海、五湖、八河、四瀆、三江、九派並各處龍王俱遣退。整冠束帶,接出宮門,迎進宮內道:「昨日可韓司查勘小宮,恐有本部之神思凡作怪,正在此點查江海河瀆之神,尚未完也。」行者道:「那魔王不是江河之神,此乃廣大之精。先蒙玉帝差李天王父子並兩個雷公下界擒拿,被他弄個圈子,將六件神兵套去。老孫無奈,又上彤華宮請火德星君帥火部眾神放火,又將火龍、火馬等物一圈子套去。我想此物既不怕火,必然怕水,特來告請星君施水勢,與我捉那妖精,取兵器歸還天將,吾師之難,亦可救也。」水德聞言,即令黃河水伯神王:「隨大聖去助功。」水伯自衣袖中取出一個白玉盂兒道:「我有此物盛水。」行者道:「看這盂兒能盛幾何?妖魔如何渰得?」水伯道:「不瞞大聖說。我這一盂乃是黃河之水,半盂就是半河,一盂就是一河。」行者喜道:「只消半盂足矣。」遂辭別水德,與黃河神躲離天闕。

那水伯將盂兒望黃河舀了半盂,跟大聖至金山,向南坡下見了天王、太子、雷公、火德,具言前事。行者道:「不必細講,且教水伯跟我去。待我叫開他門,不要等他出來,就將水往門裡一倒,那怪物一窩子可都渰死。我卻去撈師父的屍首,再救活不遲。」

那水伯依命,緊隨行者,轉山坡,徑至洞口,叫聲:「妖怪開門!」那把門的小妖,聽得是孫大聖的聲音,急又去報道:「孫悟空又來矣!」那魔聞說,帶了寶貝,綽槍就走,響一聲,開了石門。這水伯將白玉盂向裡一傾。那妖見是水來,撒了長槍,即忙取出圈子,撐住二門。只見那股水骨都都的只往外泛將出來。慌得孫大聖急縱觔斗,與水伯跳在高峰。那天王同眾都駕雲停於高峰之前,觀看那水,波濤泛漲,著實狂瀾,好水!真個是:
\begin{quote}
一勺之多,果然不測。蓋唯神功運化,利萬物而流漲百川。只聽得那潺潺聲振谷,又見那滔滔勢漫天。雄威響若雷奔走,猛湧波如雪捲顛。千丈波高漫路道,萬層濤激泛山巖。冷冷如漱玉,滾滾似鳴絃。觸石滄滄噴碎玉,回湍渺渺漩窩圓。低低凹凹隨流蕩,滿澗平溝上下連。
\end{quote}

行者見了,心慌道:「不好啊!水漫四野,渰了民田,未曾灌在他的洞裡,曾奈之何?」喚水伯急忙收水。水伯道:「小神只會放水,卻不會收水。常言道:『潑水難收。』」咦!那座山卻也高峻,這場水只奔低流。須臾間,四散而歸澗壑。

又只見那洞外跳出幾個小妖,在外邊吆吆喝喝,伸拳邏袖,弄棒拈槍,依舊喜喜歡歡耍子。天王道:「這水原來不曾灌入洞內,枉費一場之功也。」行者忍不住心中怒發,雙手掄拳,闖至妖魔門首,喝道:「那裡走!看打!」諕得那幾個小妖丟了槍棒,跑入洞裡,戰兢兢的報道:「大王,不好了,打將來了!」那魔王挺長槍,迎出門前道:「這潑猴老大憊𪬯!你幾番家敵不過我,縱水火亦不能近,怎麼又踵將來送命?」行者道:「這兒子反說了哩。不知是我送命,是你送命?走過來,吃老外公一拳。」那妖魔笑道:「這猴兒強勉纏帳!我倒使槍,他卻使拳。那般一個筋䯞子拳頭,只好有個核桃兒大小,怎麼稱得個鎚子起也?罷罷罷,我且把槍放下,與你走一路拳看看。」行者笑道:「說得是,走上來。」

那妖撩衣進步,丟了個架子,舉起兩個拳來,真似打油的鐵鎚模樣。這大聖展足挪身,擺開解數,在那洞門前,與那魔王遞走拳勢。這一場好打。咦!
拽開大四平,踢起雙飛腳。韜脅劈胸墩,剜心摘膽著。仙人指路,老子騎鶴。餓虎撲食最傷人,蛟龍戲水能兇惡。魔王使個蟒翻身,大聖卻施鹿解角。翹跟淬地龍,扭腕拿天橐。青獅張口來,鯉魚跌脊躍。蓋頂撒花,遶腰貫索。迎風貼扇兒,急雨催花落。妖精便使觀音掌,行者就對羅漢腳。長掌開闊自然鬆,怎比短拳多緊削。兩個相持數十回,一般本事無強弱。

他兩個在那洞門前廝打,只見這高峰頭喜得個李天王厲聲喝采,火德星鼓掌誇稱。那兩個雷公與哪吒太子,帥眾神跳到跟前,都要來相助;這壁廂群妖搖旗擂鼓,舞劍掄刀一齊護。孫大聖見事不諧,將毫毛拔下一把,望空撒起,叫:「變!」即變做三五十個小猴,一擁上前,把那妖纏住,抱腿的抱腿,扯腰的扯腰,抓眼的抓眼,撏毛的撏毛。那怪物慌了,急把圈子拿將出來。大聖與天王等見他弄出圈套,撥轉雲頭,走上高峰逃陣。那妖把圈子往上拋起,唿喇的一聲,把那三五十個毫毛變的小猴,收為本相,套入洞中,得了勝,領兵閉門,賀喜而去。

這太子道:「孫大聖還是個好漢。這一路拳,走得似錦上添花;使分身法,正是人前顯貴。」行者笑道:「列位在此遠觀,那怪的本事,比老孫如何?」李天王道:「他拳鬆腳慢,不如大聖的緊疾。他見我們去時,也就著忙;又見你使出分身法來,他就急了。所以大弄個圈套。」行者道:「魔王好治,只是套子難降。」火德與水伯道:「若還取勝,除非得了他那寶貝,然後可擒。」行者道:「他那寶貝如何可得?只除是偷去來。」鄧、張二公笑道:「若要行偷禮,除大聖再無能者。想當年大鬧天宮時,偷御酒,偷蟠桃,偷龍肝、鳳髓及老君之丹,那是何等手段!今日正該拿此處用也。」行者道:「好說,好說。既如此,你們且坐,等老孫打聽去來。」

好大聖,跳下峰頭,私至洞口,搖身一變,變做個麻蒼蠅兒,真個秀溜。你看他:
\begin{quote}
翎翅薄如竹膜,身軀小似花心。手足比毛更奘,星星眼窟明明。善自聞香逐氣,飛時迅速乘風。稱來剛壓定盤星,可愛些些有用。
\end{quote}

輕輕的飛在門上,爬到門縫邊,鑽進去。只見那大小群妖舞的舞,唱的唱,排列兩傍。老魔王高坐臺上,面前擺著些蛇肉、鹿脯、熊掌、駝峰、山蔬果品。有一把青磁酒壺,香噴噴的羊酪椰醪,大碗家寬懷暢飲。行者落於小妖叢裡,又變做一個獾頭精,慢慢的演近臺邊。看夠多時,全不見寶貝放在何方。急抽身轉至臺後,又見那後廳上高吊著火龍吟嘯,火馬號嘶。忽擡頭,見他的那金箍棒靠在東壁,喜得他心癢難撾,忘記了更容變像,走上前拿了鐵棒,現原身丟開解數,一路棒打將出去。慌得那群妖膽戰心驚,老魔王措手不及,卻被他推倒三個,放倒兩個,打開一條血路,徑自出了洞門。這才是:
\begin{quote}
魔頭驕傲無防備,主杖還歸與本人。
\end{quote}

畢竟不知吉凶如何,且聽下回分解。
