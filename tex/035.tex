
\chapter{外道施威欺正性 心猿獲寶伏邪魔}

\begin{quote}
本性圓明道自通,翻身跳出網羅中。
修成變化非容易,煉就長生豈俗同?
清濁幾番隨運轉,闢開數劫任西東。
逍遙萬億年無計,一點神光永注空。
\end{quote}

此詩暗合孫大聖的道妙。他自得了那魔真寶,籠在袖中,喜道:「潑魔苦苦用心拿我,誠所謂水中撈月;老孫若要擒你,就好似火上弄冰。」藏著葫蘆,密密的溜出門外,現了本相,厲聲高叫道:「精怪開門!」傍有小妖道:「你又是甚人,敢來吆喝?」行者道:「快報與你那老潑魔,吾乃行者孫來也。」

那小妖急入裡報道:「大王,門外有個甚麼行者孫來了。」老魔大驚道:「賢弟,不好了,惹動他一窩風了。幌金繩現拴著孫行者,葫蘆裡現裝著者行孫,怎麼又有個甚麼行者孫?想是他幾個兄弟都來了。」二魔道:「兄長放心。我這葫蘆裝下一千人哩,我才裝了者行孫一個,又怕那甚麼行者孫?等我出去看看,一發裝來。」老魔道:「兄弟仔細。」

你看那二魔拿著個假葫蘆,還像前番,雄糾糾,氣昂昂,走出門高呼道:「你是那裡人氏,敢在此間吆喝?」行者道:「你認不得我:
\begin{quote}
家居花果山,祖貫水簾洞。
只為鬧天宮,多時罷爭競。
如今幸脫災,棄道從僧用。
秉教上雷音,求經歸覺正。
相逢野潑魔,卻把神通弄。
還我大唐僧,上西參佛聖。
兩家罷戰爭,各守平安境。
休惹老孫焦,傷殘老性命。」
\end{quote}

那魔道:「你且過來,我不與你相打,但我叫你一聲,你敢應麼?」行者笑道:「你叫我,我就應了;我若叫你,你可應麼?」那魔道:「我叫你,是我有個寶貝葫蘆,可以裝人;你叫我,卻有何物?」行者道:「我也有個葫蘆兒。」那魔道:「既有,拿出來我看。」行者就於袖中取出葫蘆道:「潑魔,你看。」幌一幌,復藏在袖中,恐他來搶。

那魔見了,大驚道:「他葫蘆是那裡來的?怎麼就與我的一般?縱是一根藤上結的,也有個大小不同,偏正不一,卻怎麼一般無二?」他便正色叫道:「行者孫,你那葫蘆是那裡來的?」行者委的不知來歷,接過口來,就問他一句道:「你那葫蘆是那裡來的?」那魔不知是個見識,只道是句老實言語,就將根本從頭說出道:「我這葫蘆是混沌初分,天開地闢,有一位太上老祖,解化女媧之名,煉石補天,普救閻浮世界。補到乾宮缺地,見一座崑崙山腳下,有一縷仙藤,上結著這個紫金紅葫蘆,卻便是老君留下到如今。」大聖聞言,就綽了他口氣道:「我的葫蘆,也是那裡來的。」魔頭道:「怎見得?」大聖道:「自清濁初開,天不滿西北,地不滿東南,太上道祖解化女媧,補完天缺,行至崑崙山下,有根仙藤,藤結有兩個葫蘆。我得一個是雄的,你那個卻是雌的。」那怪道:「莫說雌雄,但只裝得人的,就是好寶貝。」大聖道:「你也說得是,我就讓你先裝。」

那怪甚喜,急縱身跳將起去,到空中,執著葫蘆,叫一聲:「行者孫。」大聖聽得,卻就不歇氣,連應了八九聲,只是不能裝去。那魔墜將下來,跌腳搥胸道:「天那!只說世情不改變哩,這樣個寶貝,也怕老公,雌見了雄,就不敢裝了。」行者笑道:「你且收起,輪到老孫該叫你哩。」急縱觔斗,跳起去,將葫蘆底兒朝天,口兒朝地,照定妖魔,叫聲:「銀角大王。」那怪不敢閉口,只得應了一聲。倏的裝在裡面,被行者貼上「太上老君急急如律令奉敕」的帖子。心中暗喜道:「我的兒,你今日也來試試新了。」

他就按落雲頭,拿著葫蘆,心心念念,只是要救師父,又往蓮花洞口而來。那山上都是些窪踏不平之路,況他又是個圈盤腿,拐呀拐的走著,搖的那葫蘆裡漷漷索索,響聲不絕。你道他怎麼便有響聲?原來孫大聖是熬煉過的身體,急切化他不得;那怪雖也能騰雲駕霧,不過是些法術,大端是凡胎未脫,到於寶貝裡就化了。行者還不當他就化了,笑道:「我兒子啊,不知是撒尿耶,不知是漱口哩?這是老孫幹過的買賣。不等到七八日,化成稀汁,我也不揭蓋來看。忙怎的?有甚要緊?想著我出來的容易,就該千年不看才好。」他拿著葫蘆,說著話,不覺的到了洞口,把那葫蘆搖搖,一發響了。他道:「這個像發課的筩子響,倒好發課。等老孫發一課,看師父甚麼時才得出門。」你看他手裡不住的搖,口裡不住的念道:「周易文王、孔子聖人、桃花女先生、鬼谷子先生。」

那洞裡小妖看見道:「大王,禍事了,行者孫把二大王爺爺裝在葫蘆裡發課哩。」那老魔聞得此言,諕得魂飛魄散,骨軟觔麻,撲的跌倒在地,放聲大哭道:「賢弟呀!我和你私離上界,轉託塵凡,指望同享榮華,永為山洞之主。怎知為這和尚,傷了你的性命,斷吾手足之情。」滿洞群妖,一齊痛哭。

豬八戒吊在梁上,聽得他一家子齊哭,忍不住叫道:「妖精,你且莫哭,等老豬講與你聽。先來的孫行者,次來的者行孫,後來的行者孫,返復三字,都是我師兄一人。他有七十二變化,騰那進來,盜了寶貝,裝了令弟。令弟已是死了,不必這等扛喪。快些兒刷淨鍋灶,辦些香蕈、蘑菇、茶芽、竹筍、豆腐、麵觔、木耳、蔬菜,請我師徒們下來,與你令弟念卷《受生經》。」那老魔聞言,心中大怒道:「只說豬八戒老實,原來甚不老實!他倒作笑話兒打覷我。」叫:「小妖,且休舉哀,把豬八戒解下來,蒸得稀爛,等我吃飽了,再去拿孫行者報仇。」沙僧埋怨八戒道:「好麼,我說教你莫多話,多話的要先蒸吃哩。」那獃子也盡有幾分悚懼。傍有一小妖道:「大王,豬八戒不好蒸。」八戒道:「阿彌陀佛!是那位哥哥積陰德的?果是不好蒸。」又有一個妖道:「將他皮剝了,就好蒸。」八戒慌了道:「好蒸,好蒸,皮骨雖然粗糙,湯滾就爛,戶戶。」

正嚷處,只見前門外一個小妖報道:「行者孫又罵上門來了。」那老魔又大驚道:「這廝輕我無人。」叫:「小的們,且把豬八戒照舊吊起,查一查還有幾件寶貝。」管家的小妖道:「洞中還有三件寶貝哩。」老魔問:「是那三件?」管家的道:「還有七星劍、芭蕉扇與淨瓶。」老魔道:「那瓶子不中用:原是叫人,人應了就裝得,轉把個口訣兒教了那孫行者,倒把自家兄弟裝去了。不用他,放在家裡。快將劍與扇子拿來。」那管家的即將兩件寶貝獻與老魔。老魔將芭蕉扇插在後項衣領,把七星劍提在手中,又點起大小群妖有三百多名,都教一個個拈槍弄棒,理索掄刀。這老魔卻頂盔貫甲,罩一領赤焰焰的絲袍。群妖擺出陣去,要拿孫大聖。

那孫大聖早已知二魔化在葫蘆裡面,卻將他緊緊拴扣停當,撒在腰間,手持著金箍棒,準備廝殺。只見那老妖紅旗招展,跳出門來。卻怎生打扮:
\begin{quote}
頭上盔纓光燄燄,腰間帶束彩霞鮮。
身穿鎧甲龍鱗砌,上罩紅袍烈火然。
圓眼睜開光掣電,鋼鬚飄起亂飛煙。
七星寶劍輕提手,芭蕉扇子半遮肩。
行似流雲離海岳,聲如霹靂震山川。
威風凜凜欺天將,怒帥群妖出洞前。
\end{quote}

那老魔急令小妖擺開陣勢,罵道:「你這猴子,十分無禮。害我兄弟,傷我手足,著然可恨!」行者罵道:「你這討死的怪物,你一個妖精的性命捨不得。似我師父、師弟,連馬四個生靈,平白的吊在洞裡,我心何忍?情理何甘?快快的送將出來還我,多多貼些盤費,喜喜歡歡打發老孫起身,還饒了你這個老妖的狗命。」那怪那容分說,舉寶劍劈頭就砍;這大聖使鐵棒舉手相迎。這一場在洞門外好殺。咦!
\begin{quote}
金箍棒與七星劍,對撞霞光如閃電。
悠悠冷氣逼人寒,蕩蕩昏雲遮嶺堰。
那個皆因手足情,些兒不放善;
這個只為取經僧,毫釐不容緩。
兩家各恨一般仇,二處每懷生怒怨。
只殺得天昏地暗鬼神驚,日淡煙濃龍虎戰。
這個咬牙剉玉釘,那個怒目飛金焰。
一來一往逞英雄,不住翻騰棒與劍。
\end{quote}

這老魔與大聖戰經二十回合,不分勝負。他把那劍梢一指,叫聲:「小妖齊來。」那三百餘精一齊擁上,把行者圍在垓心。好大聖,公然無懼,使一條棒,左衝右撞,後抵前遮。那小妖都有手段,越打越上,一似綿絮纏身,摟腰扯腿,莫肯退後。大聖慌了,即使個身外身法,將左脅下毫毛拔了一把,嚼碎噴去,喝聲叫:「變!」一根根都變做行者。你看他長的使棒,短的掄拳,再小的沒處下手,抱著孤拐啃觔,把那小妖都打得星落雲散,齊聲喊道:「大王啊,事不諧矣,難矣乎哉!滿地盈山,皆是孫行者了。」被這身外法把群妖打退,止撇得老魔圍困中間,趕得東奔西走,出路無門。

那魔慌了,將左手擎著寶劍,右手伸於項後,取出芭蕉扇子,望東南丙丁火,正對離宮,唿喇的一扇子搧將下來只見那就地上,火光焰焰。原來這般寶貝,平白地搧出火來。那怪物著實無情,一連搧了七八扇子,熯天熾地,烈火飛騰。好火:
\begin{quote}
那火不是天上火,不是爐中火,也不是山頭火,也不是灶底火,乃是五行中自然取出的一點靈光火。這扇也不是凡間常有之物,也不是人工造就之物,乃是自開闢混沌以來產成的真寶之物。用此扇,搧此火,煌煌燁燁,就如電掣紅綃;灼灼輝輝,卻似霞飛絳綺。更無一縷青煙,盡是滿山赤焰。只燒得嶺上松翻成火樹,崖前柏變作燈籠。那窩中走獸貪性命,西撞東奔;這林內飛禽惜羽毛,高飛遠舉。這場神火飄空燎,只燒得石爛溪乾遍地紅。
\end{quote}

大聖見此惡火,卻也心驚膽顫,道聲:「不好了,我本身可處,毫毛不濟,一落這火中,豈不真如燎毛之易?」將身一抖,遂將毫毛收上身來。只將一根變作假身子,避火逃災。他的真身,捻著避火訣,縱觔斗,跳將起去,脫離了大火之中,徑奔他蓮花洞裡,想著要救師父。急到門前,把雲頭按落,又見那洞門外有百十個小妖,都破頭折腳,肉綻皮開。原來都是他分身法打傷了的,都在這裡聲聲喚喚,忍疼而立。大聖見了,按不住惡性兇頑,掄起鐵棒,一路打將進去。可憐把那苦煉人身的功果息,依然是塊舊皮毛。

那大聖打絕了小妖,撞入洞裡,要解師父。又見那內面有火光焰焰,諕得他手慌腳忙道:「罷了,罷了,這火從後門口燒起來,老孫卻難救師父也。」正悚懼處,仔細看時,呀!原來不是火光,卻是一道金光。他正了性,往裡視之,乃羊脂玉淨瓶放光,卻自心中歡喜道:「好寶貝耶!那瓶子曾是那小妖拿在山上放光,老孫得了,不想那怪又復搜去。今日藏在這裡,原來也放光。」你看他竊了這瓶子,喜喜歡歡,且不救師父,急抽身往洞外而走。才出門,只見那妖魔提著寶劍,拿著扇子,從南而來。孫大聖迴避不及,被那老魔舉劍劈頭就砍。大聖急縱觔斗雲跳將起去,無影無蹤的逃了不題。

卻說那怪到得門口,但見屍橫滿地,就是他手下的群精。慌得仰天長嘆,止不住放聲大哭道:「苦哉!痛哉!」有詩為證。詩曰:
\begin{quote}
可恨猿乖馬劣頑,靈胎轉託降塵凡。
只因錯念離天闕,致使忘形落此山。
鴻雁失群情切切,妖兵絕族淚潺潺。
何時孽滿開愆鎖,返本還原上御關?
\end{quote}

那老魔慚惶不已,一步一聲,哭入洞內。只見那什物傢火俱在,只落得靜悄悄,沒個人形,悲切切,愈加悽慘。獨自個坐在洞中,蹋伏在那石案之上,將寶劍斜倚案邊,把扇子插於肩後,昏昏默默睡著了,這正是:人逢喜事精神爽,悶上心來瞌睡多。

話說孫大聖撥轉觔斗雲,佇立山前,想著要救師父,把那淨瓶兒牢扣腰間,徑來洞口打探。見那門開兩扇,靜悄悄的不聞消耗。隨即輕輕移步,潛入裡邊。只見那魔斜倚石案,呼呼睡著。芭蕉扇褪出肩衣,半蓋著腦後;七星劍還斜倚案邊。卻被他輕輕的走上前拔了扇子,急回頭,呼的一聲,跑將出去。原來這扇柄兒刮著那怪的頭髮,早驚醒他。擡頭看時,是孫行者偷了,急慌忙執劍來趕。那大聖早已跳出門前,將扇子撒在腰間,雙手掄開鐵棒,與那魔抵敵。這一場好殺:
\begin{quote}
惱壞潑妖王,怒發沖冠志。恨不過撾來囫圇吞,難解心頭氣。惡口罵猢猻:「你老大將人戲,傷我若干生,還來偷寶貝。這場決不容,定見存亡計。」大聖喝妖魔:「你好不知趣,徒弟要與老孫爭,累卵焉能擊石碎?」寶劍來,鐵棒去,兩家更不留仁義。一翻二復賭輸贏,三轉四回施武藝。蓋為取經僧,靈山參佛位。致令金火不相投,五行撥亂傷和氣;揚威耀武顯神通,走石飛沙弄本事。交鋒漸漸日將晡,魔頭力怯先迴避。
\end{quote}

那老魔與大聖戰經三四十合,天將晚矣,抵敵不住,敗下陣來;徑往西南上,投奔壓龍洞去不題。

這大聖才按落雲頭,闖入蓮花洞裡,解下唐僧與八戒、沙和尚來。他三人脫得災危,謝了行者,卻問:「妖魔那裡去了?」行者道:「二魔已裝在葫蘆裡,想是這會子已化了。大魔才然一陣戰敗,往西南壓龍山去訖。概洞小妖,被老孫分身法打死一半;還有些敗殘回的,又被老孫殺絕。方才得入此處,解放你們。」唐僧謝之不盡道:「徒弟啊,多虧你受了勞苦。」行者笑道:「誠然勞苦。你們還只是吊著受疼,我老孫再不曾住腳,比急遞鋪的鋪兵還甚,反復裡外,奔波無已。因是偷了他的寶貝,方能平退妖魔。」豬八戒道:「師兄,你把那葫蘆兒拿出來與我們看看。只怕那二魔已化了也。」大聖先將淨瓶解下,又將金繩與扇子取出,然後把葫蘆兒拿在手道:「莫看,莫看。他先曾裝了老孫,被老孫漱口,哄得他揭開蓋子,老孫方得走了。我等切莫揭蓋,只怕他也會弄喧走了。」師徒們喜喜歡歡,將他那洞中的米麵菜蔬尋出,燒刷了鍋灶,安排些素齋吃了。飽餐一頓,安寢洞中,一夜無詞,早又天曉。

卻說那老魔徑投壓龍山,會聚了大小女怪,備言打殺母親,裝了兄弟,絕滅妖兵,偷騙寶貝之事。眾女怪一齊大哭,哀痛多時道:「你等且休悽慘。我身邊還有這口七星劍,欲會汝等女兵,都去壓龍山後,會借外家親戚,斷要拿住那孫行者報仇。」說不了,有門外小妖報道:「大王,山後老舅爺帥領若干兵卒來也。」老魔聞言,急換了縞素孝服,躬身迎接。原來那老舅爺是他母親之弟,名喚狐阿七大王。因聞得哨山的妖兵報道,他姐姐被孫行者打死,假變姐形,盜了外甥寶貝,連日在平頂山拒敵,他即帥本洞妖兵二百餘名,特來助陣,故此先攏姐家問信。才進門,見老魔掛了孝服,二人大哭。哭久,老魔拜下,備言前事。那阿七大怒,即命老魔換了孝服,提了寶劍,盡點女妖,合同一處,縱風雲,徑投東北而來。

這大聖卻教沙僧整頓早齋,吃了走路。忽聽得風聲,走出門看,乃是一夥妖兵,自西南上來。行者大驚,急抽身,忙呼八戒道:「兄弟,妖精又請救兵來也。」三藏聞言,驚恐失色道:「徒弟,似此如何?」行者笑道:「放心,放心。把他這寶貝都拿來與我。」大聖將葫蘆、淨瓶繫在腰間,金繩籠於袖內,芭蕉扇插在肩後,雙手掄著鐵棒。教沙僧保守師父,穩坐洞中。著八戒執釘鈀,同出洞外迎敵。

那怪物擺開陣勢,只見當頭的是阿七大王。他生的玉面長髯,鋼眉刀耳;頭戴金煉盔,身穿鎖子甲,手執方天戟。高聲罵道:「我把你個大膽的潑猴!怎敢這等欺人?偷了寶貝,傷了眷族,殺了妖兵,又敢久占洞府。趕早兒一個個引頸受死,雪我姐家之仇。」行者罵道:「你這夥作死的毛團,不識你孫外公的手段。不要走,領吾一棒。」那怪物側身躲過,使方天戟劈面相迎。兩個在山頭上一來一往,戰經三四回合,那怪力軟。敗陣回走。行者趕來,卻被老魔接住。又鬥了三合,只見那狐阿七復轉來攻。這壁廂八戒見了,急掣九齒鈀擋住。一個抵一個,戰經多時,不分勝敗,那老魔喝了一聲,眾妖兵一齊圍上。

卻說那三藏坐在蓮花洞裡,聽得喊聲振地,便叫:「沙和尚,你出去看你師兄勝負如何?」沙僧果舉降妖杖出來,喝一聲,撞將出去,打退群妖。阿七見事勢不利,回頭就走;被八戒趕上,照背後一鈀,就築得九點鮮紅往外冒,可憐一靈真性赴前程。急拖來剝了衣服看處,原來也是個狐狸精。

那老魔見傷了他老舅,丟了行者,提起寶劍,就劈八戒;八戒使鈀架住。正賭鬥間,沙僧撞近前來,舉杖便打。那妖抵敵不住,縱風雲,往南逃走。八戒、沙僧緊緊趕來。大聖見了,急縱雲跳在空中,解下淨瓶,罩定老魔,叫聲:「金角大王。」那怪只道是自家敗殘的小妖呼叫,就回頭應了一聲。颼的裝將進去,被行者貼上「太上老君急急如律令奉敕」的帖子。只見那七星劍墜落塵埃,也歸了行者。八戒迎著道:「哥哥,寶劍你得了,精怪何在?」行者笑道:「了了,已裝在我這瓶兒裡也。」沙僧聽說,與八戒十分歡喜。

當時通掃淨諸邪,回至洞裡,與三藏報喜道:「山已淨,妖已無矣,請師父上馬走路。」三藏喜不自勝。師徒們吃了早齋,收拾了行李、馬匹,奔西找路。

正行處,猛見路傍閃出一個瞽者,走上前,扯住三藏馬道:「和尚,那裡去?還我寶貝來。」八戒大驚道:「罷了,這是老妖來討寶貝了。」行者仔細觀看,原來是太上李老君,慌得近前施禮道:「老官兒,那裡去?」那老祖急昇玉局寶座,在九霄空裡佇立,叫:「孫行者,還我寶貝。」大聖起到空中道:「甚麼寶貝?」老君道:「葫蘆是我盛丹的,淨瓶是我盛水的,寶劍是我煉魔的,扇子是我搧火的,繩子是我一根勒袍的帶。那兩個怪:一個是我看金爐的童子,一個是我看銀爐的童子。只因他偷了我的寶貝,走下界來,正無覓處,卻是你今拿住,得了功績。」大聖道:「你這老官兒,著實無禮。縱放家屬為邪,該問個鈐束不嚴的罪名。」老君道:「不干我事,不可錯怪了人。此乃海上菩薩問我借了三次,送他在此,託化妖魔,試你師徒可有真心往西去也。」大聖聞言,心中作念道:「這菩薩也老大憊𪬯。當時解逃老孫,教保唐僧西去取經,我說路途艱澀難行,他曾許我到急難處,親來相救;如今反使精邪掯害。語言不的,該他一世無夫。若不是老官兒親來,我決不與他。既是你這等說,拿去罷。」

那老君收得五件寶貝,揭開葫蘆與淨瓶蓋口,倒出兩股仙氣。用手一指,仍化為金、銀二童子,相隨左右。只見那霞光萬道,咦!
\begin{quote}
縹緲同歸兜率院,逍遙直上大羅天。
\end{quote}

畢竟不知此後又有甚事,孫大聖怎生保護唐僧,幾時得到西天,且聽下回分解。
