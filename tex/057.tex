
\chapter{真行者落伽山訴苦 假猴王水簾洞謄文}

卻說孫大聖惱惱悶悶,起在空中,欲待回花果山水簾洞,恐本洞小妖見笑,笑我出乎爾反乎爾,不是個大丈夫之器;欲待要投奔天宮,又恐天宮內不容久住;欲待要投海島,卻又羞見那三島諸仙;欲待要奔龍宮,又不伏氣求告龍王。真個是無依無倚。苦自忖量道:「罷罷罷,我還去見我師父,還是正果。」

遂按下雲頭,徑至三藏馬前侍立道:「師父,恕弟子這遭!向後再不敢行兇,一一受師父教誨。千萬還得我保你西天去也。」唐僧見了,更不答應,兜住馬,即念緊箍兒咒。顛來倒去,又念有二十餘遍。把大聖咒倒在地,箍兒陷在肉裡有一寸來深淺,方才住口道:「你不回去,又來纏我怎的?」行者只教:「莫念,莫念。我是有處過日子的,只怕你無我去不得西天。」三藏發怒道:「你這猢猻殺生害命,連累了我多少,如今實不要你了。我去得去不得,不干你事。快走,快走,遲了些兒,我又念真言,這番決不住口,把你腦漿都勒出來哩。」大聖疼痛難忍,見師父更不回心,沒奈何,只得又駕觔斗雲,起在空中。忽然省悟道:「這和尚負了我心,我且向普陀崖告訴觀音菩薩去來。」

好大聖,撥回觔斗,那消一個時辰,早至南洋大海。住下祥光,直至落伽山上,撞入紫竹林中,忽見木叉行者迎面作禮道:「大聖何往?」行者道:「要見菩薩。」木叉即引行者至潮音洞口,又見善財童子作禮道:「大聖何來?」行者道:「有事要告菩薩。」善財聽見一個「告」字,笑道:「好刁嘴猴兒,還像當時我拿住唐僧被你欺哩。我菩薩是個大慈大悲、大願大乘、救苦救難、無邊無量的聖善菩薩,有甚不是處,你要告他?」行者滿懷悶氣,一聞此言,心中怒發,咄的一聲,把善財童子喝了個倒退,道:「這個背義忘恩的小畜生,著實愚魯!你那時節作怪成精,我請菩薩收了你,皈正迦持,如今得這等極樂長生,自在逍遙,與天同壽,還不拜謝老孫,轉倒這般侮慢。我是有事來告求菩薩,卻怎麼說我刁嘴要告菩薩?」善財陪笑道:「還是個急猴子。我與你作笑耍子,你怎麼就變臉了?」

正講處,只見白鸚哥飛來飛去,知是菩薩呼喚,木叉與善財遂向前引導,至寶蓮臺下。行者望見菩薩,倒身下拜,止不住淚如泉湧,放聲大哭。菩薩教木叉與善財扶起道:「悟空,有甚傷感之事,明明說來。莫哭,莫哭,我與你救苦消災也。」行者垂淚再拜道:「當年弟子為人,曾受那個氣來?自蒙菩薩解脫天災,秉教沙門,保護唐僧往西天拜佛求經,我弟子捨身拚命,救解他的魔障,就如老虎口裡奪脆骨,蛟龍背上揭生鱗。只指望歸真正果,洗孽除邪。怎知那長老背義忘恩,直迷了一片善緣,更不察皂白之苦。」菩薩道:「且說那皂白原因來我聽。」行者即將那打殺草寇前後始終,細陳了一遍。卻說唐僧因他打死多人,心生怨恨,不分皂白,遂念緊箍兒咒,趕他幾次。上天無路,入地無門,特來告訴菩薩。菩薩道:「唐三藏奉旨投西,一心要秉善為僧,決不輕傷性命。似你有無量神通,何苦打殺許多草寇?草寇雖是不良,到底是個人身,不該打死。比那妖禽怪獸、鬼魅精魔不同。那個打死,是你的功績;這人身打死,還是你的不仁。但祛退散,自然救了你師父。據我公論,還是你的不善。」

行者噙淚叩頭道:「縱是弟子不善,也當將功折罪,不該這般逐我。萬望菩薩捨大慈悲,將鬆箍兒咒念念,褪下金箍,交還與你,放我仍往水簾洞逃生去罷」菩薩笑道:「緊箍兒咒,本是如來傳我的。當年差我上東土尋取經人,賜我三件寶貝,乃是錦襴袈裟、九環錫杖、金緊禁三個箍兒。秘授與咒語三篇,卻無甚麼鬆箍兒咒。」行者道:「既如此,我告辭菩薩去也。」菩薩道:「你辭我往那裡去?」行者道:「我上西天,拜告如來,求念鬆箍兒咒去也。」菩薩道:「你且住,我與你看看祥晦如何。」行者道:「不消看,只這樣不祥也夠了。」菩薩道:「我不看你,看唐僧的祥晦。」

好菩薩,端坐蓮臺,運心三界,慧眼遙觀,遍周宇宙,霎時間開口道:「悟空,你那師父頃刻之際,就有傷身之難,不久便來尋你。你只在此處,待我與唐僧說,教他還同你去取經,了成正果。」孫大聖只得皈依,不敢造次,侍立於寶蓮臺下不題。

卻說唐長老自趕回行者,教八戒引馬,沙僧挑擔,連馬四口,奔西走不上五十里遠近,三藏勒馬道:「徒弟,自五更時出了村舍,又被那弼馬溫著了氣惱,這半日饑又饑,渴又渴,那個去化些齋來我吃?」八戒道:「師父且請下馬,等我看可有鄰近的莊村,化齋去也。」三藏聞言,滾下馬來。獃子縱起雲頭,半空中仔細觀看,一望盡是山嶺,莫想有個人家。八戒按下雲來,對三藏道:「卻是沒處化齋,一望之間,全無莊舍。」三藏道:「既無化齋之處,且得些水來解渴也可。」八戒道:「等我去南山澗下取些水來。」沙僧即取缽盂,遞與八戒。八戒托著缽盂,駕起雲霧而去。那長老坐在路傍,等夠多時,不見回來,可憐口乾舌苦難熬。有詩為證。詩曰:
\begin{quote}
保神養氣謂之精,情性原來一稟形。
心亂神昏諸病作,形衰精敗道元傾。
三花不就空勞碌,四大蕭條枉費爭。
土木無功金水絕,法身疏懶幾時成!
\end{quote}

沙僧在傍,見三藏饑渴難忍,八戒又取水不來,只得穩了行囊,拴牢了白馬道:「師父,你自在坐著,等我去催水來。」長老含淚無言,但點頭相答。沙僧急駕雲光,也向南山而去。

那師父獨鍊自熬,困苦太甚。正在愴惶之際,忽聽得一聲響亮,諕得長老欠身看處,原來是孫行者跪在路傍,雙手捧著一個磁杯道:「師父,沒有老孫,你連水也不能夠哩。這一杯好涼水,你且吃口水解渴,待我再去化齋。」長老道:「我不吃你的水,立地渴死,我當任命。不要你了,你去罷。」行者道:「無我你去不得西天也。」三藏道:「去得去不得,不干你事。潑猢猻,只管來纏我做甚?」那行者變了臉,發怒生嗔,喝罵長老道:「你這個狠心的潑禿!十分賤我。」掄鐵棒,丟了磁杯,望長老脊背上砑了一下。那長老昏暈在地,不能言語,被他把兩個青氈包袱提在手中,駕觔斗雲,不知去向。

卻說八戒托著缽盂,只奔山南坡下,忽見山凹之間有一座草舍人家。原來在先看時,被山高遮住,未曾見得;今來到邊前,方知是個人家。獃子暗想道:「我若是這等醜嘴臉,決然怕我,枉勞神思,斷然化不得齋飯。須是變好,須是變好。」

好獃子,捻著訣,念個咒,把身搖了七八搖,變作一個食癆病黃胖和尚,口裡哼哼唧唧的挨近門前,叫道:「施主,廚中有剩飯,路上有饑人。貧僧是東土來,往西天取經的。我師父在路饑渴了,家中有鍋巴冷飯,千萬化些兒救口。」原來那家子男人不在,都去插秧種穀去了。只有兩個女人在家,正才煮了午飯,盛起兩盆,卻收拾送下田去,鍋裡還有些飯與鍋巴,未曾盛了。那女人見他這等病容,卻又說東土往西天去的話,只恐他是病昏了胡說,又怕跌倒,死在門首。只得哄哄翕翕,將些剩飯鍋巴,滿滿的與了一缽。獃子拿轉來,現了本像,徑回舊路。

正走間,聽得有人叫「八戒」。八戒擡頭看時,卻是沙僧站在山崖上喊道:「這裡來,這裡來。」及下崖,迎至面前道:「這澗裡好清水不舀,你往那裡去的?」八戒笑道:「我到這裡,見山凹子有個人家,我去化了這一缽乾飯來了。」沙僧道:「飯也用著,只是師父渴得緊了,怎得水去?」八戒道:「要水也容易,你將衣襟來兜著這飯,等我使缽盂去舀水。」

二人歡歡喜喜回至路上,只見三藏面磕地,倒在塵埃;白馬撒韁,在路傍長嘶跑跳;行李擔不見蹤影。慌得八戒跌腳搥胸,大呼小叫道:「不消講,不消講,這還是孫行者趕走的餘黨,來此打殺師父,搶了行李去了。」沙僧道:「且去把馬拴住!」只叫:「怎麼好?怎麼好?這誠所謂半途而廢,中道而止也。」叫一聲:「師父。」滿眼拋珠,傷心痛哭。八戒道:「兄弟,且休哭。如今事已到此,取經之事,且莫說了。你看著師父的屍靈,等我把馬騎到那個府州縣鄉村店集賣幾兩銀子,買口棺木,把師父埋了,我兩個各尋道路散夥。」

沙僧實不忍捨,將唐僧扳轉身體,以臉溫臉,哭一聲:「苦命的師父!」只見那長老口鼻中吐出熱氣,胸前溫暖,連叫:「八戒,你來,師父未傷命哩。」那獃子才近前扶起長老。甦醒呻吟一會,罵道:「好潑猢猻,打殺我也。」沙僧、八戒問道:「是那個猢猻?」長老不言,只是嘆息。卻討水吃了幾口,才說:「徒弟,你們剛去,那悟空更來纏我。是我堅執不收,他遂將我打了一棒,青氈包袱都搶去了。」八戒聽說,咬響口中牙,發起心頭火道:「叵耐這潑猴子!怎敢這般無禮?」教:「沙僧,你伏侍師父,等我到他家討包袱去。」沙僧道:「你且休發怒。我們扶師父到那山凹人家化些熱茶湯,將先化的飯熱熱,調理師父,再去尋他。」

八戒依言,把師父扶上馬,拿著缽盂,兜著冷飯,直至那家門首。只見那家止有個老婆子在家,忽見他們,慌忙躲過。沙僧合掌道:「老母親,我等是東土唐朝差往西天去者。師父有些不快,特拜府上,化口熱茶湯,與他吃飯。」那媽媽道:「適才有個食癆病和尚,說是東土差來的,已化齋去了,又有個甚麼東土的?我沒人在家,請別處轉轉。」長老聞言,扶著八戒,下馬躬身道:「老婆婆,我弟子有三個徒弟,合意同心,保護我上天竺國大雷音拜佛求經。只因我大徒弟(喚孫悟空)一生兇惡,不遵善道,是我逐回。不期他暗暗走來,著我背上打了一棒,將我行囊衣缽搶去。如今要著一個徒弟尋他取討,因在那空路上不是坐處,特來老婆婆府上權安息一時。待討將行李來就行,決不敢久住。」那媽媽道:「剛才一個食癆病黃胖和尚,他化齋去了,也說是東土往西天去的,怎麼又有一起?」八戒忍不住笑道:「就是我。因我生得嘴長耳大,恐你家害怕,不肯與齋,故變作那等模樣。你不信,我兄弟衣兜裡不是你家鍋巴飯?」

那媽媽認得果是他與的飯,遂不拒他,留他們坐了。卻燒了一罐熱茶,遞與沙僧泡飯。沙僧即將冷飯泡了,遞與師父。師父吃了幾口,定性多時道:「那個去討行李?」八戒道:「我前年因師父趕他回去,我曾尋他一次,認得他花果山水簾洞。等我去,等我去。」長老道:「你去不得。那猢猻原與你不和,你又說話粗魯,或一言兩句之間,有些差池,他就要打你。著悟淨去罷。」沙僧應承道:「我去,我去。」長老又吩咐沙僧道:「你到那裡,須看個頭勢。他若肯與你包袱,你就假謝謝拿來;若不肯,切莫與他爭競,徑至南海菩薩處,將此情告訴,請菩薩去問他要。」沙僧一一聽從。向八戒道:「我今尋他去,你千萬莫僝僽,好生供養師父。這人家亦不可撒潑,恐他不肯供飯。我去就回。」八戒點頭道:「我理會得。但你去,討得討不得,趁早回來,不要弄做尖擔擔柴兩頭脫也。」沙僧遂捻了訣,駕起雲光,直奔東勝神洲而去。真個是:
\begin{quote}
身在神飛不守舍,有爐無火怎燒丹。
黃婆別主求金老,木母延師奈病顏。
此去不知何日返,這回難量幾時還。
五行生剋情無順,只待心猿復進關。
\end{quote}

那沙僧在半空裡,行經三晝夜,方到了東洋大海,忽聞波浪之聲。低頭觀看,真個是黑霧漲天陰氣盛,滄溟銜日曉光寒。他也無心觀玩,望仙山渡過瀛洲,向東方直抵花果山界。乘海風,踏水勢,又多時,卻望見高峰排戟,峻壁懸屏。即至峰頭,按雲找路下山,尋水簾洞。步近前,只聽得那山中無數猴精,滔滔亂嚷。沙僧又近前仔細再看,原來是孫行者高坐石臺之上,雙手扯著一張紙,朗朗的念道:
\begin{quote}
東土大唐王皇帝李,駕前敕命御弟聖僧陳玄奘法師,上西方天竺國娑婆靈山大雷音寺,專拜如來佛祖求經。朕因促病侵身,魂遊地府,幸有陽數臻長,感冥君放送回生,廣陳善會,修建度亡道場。盛蒙救苦救難觀世音菩薩金身出現,指示西方有佛有經,可度幽亡超脫,特著法師玄奘,遠歷千山,詢求經偈。倘過西邦諸國,不滅善緣,照牒施行。
\end{quote}

大唐貞觀一十三年秋吉日御前文牒。
自別大國以來,經度諸邦,中途收得大徒弟孫悟空行者、二徒弟豬悟能八戒、三徒弟沙悟淨和尚。」

念了從頭又念。

沙僧聽得是通關文牒,止不住近前厲聲高叫:「師兄,師父的關文你念他怎的?」那行者聞言,急擡頭,不認得是沙僧,叫:「拿來,拿來。」眾猴一齊圍繞,把沙僧拖拖扯扯,拿近前來,喝道:「你是何人,擅敢近吾仙洞?」沙僧見他變了臉,不肯相認,只得朝上行禮道:「上告師兄:前者實是師父性暴,錯怪了師兄,把師兄咒了幾遍,逐趕回家。一則弟等未曾勸解,二來又為師父饑渴去尋水化齋。不意師兄好意復來,又怪師父執法不留,遂把師父打倒,昏暈在地。將行李搶去。後我等救轉師父,特來拜兄。若不恨師父,還念昔日解脫之恩,同小弟將行李回見師父,共上西天,了此正果;倘怨恨之深,不肯同去,千萬把包袱賜弟,兄在深山,樂桑榆晚景,亦誠兩全其美也。」

行者聞言,呵呵冷笑道:「賢弟,此論甚不合我意。我打唐僧,搶行李,不因我不上西方,亦不因我愛居此地。我今熟讀了牒文,我自己上西方拜佛求經,送上東土,我獨成功,教那南贍部洲人立我為祖,萬代傳名也。」沙僧笑道:「師兄言之欠當。自來沒個孫行者取經之說。我佛如來造下三藏真經,原著觀音菩薩向東土尋取經人求經,要我們苦歷千山,詢求諸國,保護那取經人。菩薩曾言:取經人乃如來門生,號曰金蟬長老。只因他不聽佛祖談經,貶下靈山,轉生東土,教他果正西方,復修大道。遇路上該有這般魔障,解脫我等三人,與他做護法。兄若不得唐僧去,那個佛祖肯傳經與你?卻不是空勞一場神思也?」那行者道:「賢弟,你原來懞懂,但知其一,不知其二。諒你說你有唐僧,同我保護,我就沒有唐僧?我這裡另選個有道的真僧在此,自去取經,老孫獨力扶持,有何不可?已選明日大早起身去矣。你不信,待我請來你看。」叫:「小的們,快請老師父出來。」果跑進去,牽出一匹白馬,請出一個唐三藏;跟著一個八戒,挑著行李;一個沙僧,拿著錫杖。

這沙僧見了,大怒道:「我老沙行不更名,坐不改姓,那裡又有一個沙和尚?不要無禮,吃我一杖!」好沙僧,雙手舉降妖杖,把一個假沙僧劈頭一下打死,原來這是一個猴精。那行者惱了,掄金箍棒,帥眾猴,把沙僧圍了。沙僧東沖西撞,打出路口,縱雲霧逃生道:「這潑猴如此憊𪬯,我告菩薩去來。」那行者見沙僧打死一個猴精,把沙和尚逼得走了,他也不來追趕。回洞教小的們把打死的妖屍拖在一邊,剝了皮,取肉煎炒,將椰子酒、葡萄酒,同群猴都吃了。另選一個會變化的妖猴,還變一個沙和尚,從新教導,要上西方不題。

沙僧一駕雲離了東海,行經一晝夜,到了南海。正行時,早見落伽山不遠。急至前,低停雲霧觀看,好去處!果然是:
\begin{quote}
包乾之奧,括坤之區。會百川而浴日滔星,歸眾流而生風漾月。潮發騰凌大鯤化,波翻浩蕩巨鰲遊。水通西北海,浪合正東洋。四海相連同地脈,仙方洲島各仙宮。休言滿地蓬萊,且看普陀雲洞。好景致!山頭霞彩壯元精,巖下祥風漾月晶。紫竹林中飛孔雀,綠楊枝上語靈鸚。琪花瑤草年年秀,寶樹金蓮歲歲生。白鶴幾番朝頂上,素鸞數次到山亭。遊魚也解修真性,躍浪穿波聽講經。
\end{quote}

沙僧徐步落伽山,玩看仙境。只見木叉行者當面相迎道:「沙悟淨,你不保唐僧取經,卻來此何幹?」沙僧作禮畢,道:「有一事特來朝見菩薩,煩為引見引見。」木叉情知是尋行者,更不題起,即先進去對菩薩道:「外有唐僧的小徒弟沙悟淨朝拜。」孫行者在臺下聽見,笑道:「這定是唐僧有難,沙僧來請菩薩的。」菩薩即命木叉門外叫進。這沙僧倒身下拜,拜罷,擡頭正欲告訴前事,忽見孫行者站在傍邊,等不得說話,就掣降妖杖望行者劈臉便打。這行者更不回手,撤身躲過。沙僧口裡亂罵道:「我把你個犯十惡造反的潑猴!你又來影瞞菩薩哩。」菩薩喝道:「悟淨不要動手,有甚事先與我說。」

沙僧收了寶杖,再拜臺下,氣沖沖的對菩薩道:「這猴一路行兇,不可數計。前日在山坡下打殺兩個剪路的強人,師父怪他。不期晚間就宿在賊窩主家裡,又把一夥賊人盡情打死,又血淋淋提一個人頭來與師父看。師父諕得跌下馬來,罵了他幾句,趕他回來。分別之後,師父饑渴太甚,教八戒去尋水。久等不來,又著我去尋他。不期孫行者見我二人不在,復回來把師父打一鐵棍,將兩個青氈包袱搶去。我等回來,將師父救醒,特來他水簾洞尋他討包袱,不想他變了臉,不肯認我,將師父關文念了又念。我問他念了做甚,他說不保唐僧,他要自上西天取經,送上東土,算他的功果,立他為祖,萬古傳揚。我又說:『沒唐僧,那肯傳經與你?』他說他選了一個有道的真僧。及請出,果是一匹白馬,一個唐僧,後跟著八戒、沙僧。我道:『我便是沙和尚,那裡又有個沙和尚?』是我趕上前,打了他一寶杖,原來是個猴精。他就帥眾拿我,是我特來告請菩薩。不知他會使觔斗雲,預先到此處。又不知他將甚巧語花言,影瞞菩薩也。」菩薩道:「悟淨,不要賴人。悟空到此,今已四日,我更不曾放他回去,他那裡有另請唐僧,自去取經之意?」沙僧道:「見如今水簾洞有一個孫行者,怎敢欺誑?」菩薩道:「既如此,你休發急,教悟空與你同去花果山看看,是真難滅,是假易除,到那裡自見分曉。」

這大聖聞言,即與沙僧辭了菩薩。這一去,到那:
\begin{quote}
花果山前分皂白,水簾洞口辨真邪。
\end{quote}

畢竟不知如何分辨,且聽下回分解。
