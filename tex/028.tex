
\chapter{花果山群妖聚義 黑松林三藏逢魔}

卻說那大聖雖被唐僧逐趕,然猶思念感嘆不已,早望見東洋大海,道:「我不走此路者,已五百年矣!」只見那海水:
\begin{quote}
煙波蕩蕩,巨浪悠悠。煙波蕩蕩接天河,巨浪悠悠通地脈。潮來洶湧,水浸灣環。潮來洶湧,猶如霹靂吼三春;水浸灣環,卻似狂風吹九夏。乘龍福老,往來必定皺眉行;跨鶴仙童,反覆果然憂慮過。近岸無村社,傍水少漁舟。浪捲千年雪,風生六月秋。野禽憑出沒,沙鳥任沉浮。眼前無釣客,耳畔只聞鷗。海底遊魚樂,天邊過雁愁。
\end{quote}

那行者將身一縱,跳過了東洋大海,早至花果山。按落雲頭,睜睛觀看,那山上花草俱無,煙霞盡絕;峰巖倒塌,林樹焦枯。你道怎麼這等?只因他鬧了天宮,拿上界去,此山被顯聖二郎神率領那梅山七弟兄,放火燒壞了。這大聖倍加悽慘。有一篇敗山頹景的古風為證。古風云:
\begin{quote}
回顧仙山兩淚垂,對山悽慘更傷悲。
當時只道山無損,今日方知地有虧。
可恨二郎將我滅,堪嗔小聖把人欺。
行兇掘你先靈墓,無干破爾祖墳基。
滿天霞霧皆消蕩,遍地風雲盡散稀。
東嶺不聞斑虎嘯,西山那見白猿啼。
北谿狐兔無蹤跡,南谷獐沒影遺。
青石燒成千塊土,碧砂化作一堆泥。
洞外喬松皆倚倒,崖前翠柏盡稀少。
椿杉槐檜栗檀焦,桃杏李梅梨棗了。
柘絕桑無怎養蠶?柳稀竹少難棲鳥。
峰頭巧石化為塵,澗底泉乾都是草。
崖前土黑沒芝蘭,路畔泥紅藤薜攀。
往日飛禽飛那處?當時走獸走何山?
豹嫌蟒惡傾頹所,鶴避蛇回敗壞間。
想是日前行惡念,致令目下受艱難。
\end{quote}

那大聖正當悲切,只聽得那芳草坡前,曼荊凹內,響一聲,跳出七八個小猴,一擁上前,圍住叩頭。高叫道:「大聖爺爺,今日來家了?」美猴王道:「你們因何不耍不頑,一個個都潛蹤隱跡?我來多時了,不見你們形影,何也?」群猴聽說,一個個垂淚告道:「自大聖擒拿上界,我們被獵人之苦,著實難捱。怎禁他硬弩強弓,黃鷹劣犬,網扣槍鉤,故此各惜性命,不敢出頭頑耍,只是深潛洞府,遠避窩巢。饑去坡前偷草食,渴來澗下吸清泉。卻才聽得大聖爺爺聲音,特來接見,伏望扶持。」那大聖聞得此言,愈加悽慘。便問:「你們還有多少在此山上?」群猴道:「老者小者,只有千把。」大聖道:「我當時共有四萬七千群妖,如今都往那裡去了?」群猴道:「自從爺爺去後,這山被二郎菩薩點上火,燒殺了大半。我們蹲在井裡,鑽在澗內,藏於鐵板橋下,得了性命。及至火滅煙消出來時,又沒花果養贍,難以存活,別處又去了一半。我們這一半,捱苦的住在山中。這兩年,又被些打獵的搶了一半去也。」行者道:「他搶你去何幹?」群猴道:「說起這獵戶,可恨!他把我們中箭著槍的,中毒打死的,拿了去剝皮剔骨,醬煮醋蒸,油煎鹽炒,當做下飯食用。或有那遭網的,遇扣的,夾活兒拿去了,教他跳圈做戲,翻觔斗,豎蜻蜓,當街上篩鑼擂鼓,無所不為的頑耍。」

大聖聞此言,更十分惱怒道:「洞中有甚麼人執事?」群妖道:「還有馬、流二元帥,崩、芭二將軍管著哩。」大聖道:「你們去報他知道,說我來了。」那些小妖,撞入門內報道:「大聖爺爺來家了。」那馬、流、奔、芭聞報,忙出門叩頭,迎接進洞。大聖坐在中間,群怪羅拜於前,啟道:「大聖爺爺,近聞得你得了性命,保唐僧往西天取經,如何不走西方,卻回本山?」大聖道:「小的們,你不知道,那唐三藏不識賢愚:我為他一路上捉怪擒魔,使盡了平生的手段,幾番家打殺妖精;他說我行兇作惡,不要我做徒弟,把我逐趕回來,寫立貶書為照,永不聽用了。」

眾猴鼓掌大笑道:「造化,造化。做甚麼和尚,且家來,帶攜我們耍子幾年罷。」叫:「快安排椰子酒來,與爺爺接風。」大聖道:「且莫飲酒,我問你那打獵的人,幾時來我山上一度?」馬、流道:「大聖,不論甚麼時度,他逐日家在這裡纏擾。」大聖道:「他怎麼今日不來?」馬、流道:「看待來耶。」大聖吩咐:「小的們,都出去把那山上燒酥了的碎石頭與我搬將起來堆著。或二三十個一推,或五六十個一堆堆著,我有用處。」那些小猴都是一窩峰,一個個亂搬了許多堆集。大聖看了,教:「小的們,都往洞內藏躲,讓老孫作法。」

那大聖上了山巔看處,只見那南半邊鼕鼕鼓響,噹噹鑼鳴,閃上有千餘人馬,都架著鷹犬,持著刀槍。猴王仔細看那些人來得兇險,好男子,真個驍勇。但見:
\begin{quote}
狐皮蓋肩頂,錦綺裹腰胸。
袋插狼牙箭,胯掛寶雕弓。
人似搜山虎,馬如跳澗龍。
成群引著犬,滿膀架其鷹。
荊筐擡火炮,帶定海東青。
粘竿百十擔,兔叉有千根。
牛頭攔路網,閻王扣子繩。
一齊亂吆喝,散撒滿天星。
\end{quote}

大聖見那些人佈上他的山來,心中大怒,手裡捻訣,口內念念有詞,往那巽地上吸了一口氣,呼的吹將去,便是一陣狂風。好風!但見:
\begin{quote}
揚塵播土,倒樹摧林。海浪如山聳,渾波萬疊侵。乾坤昏蕩蕩,日月暗沉沉。一陣搖松如虎嘯,忽然入竹似龍吟。萬竅怒號天噫氣,飛砂走石亂傷人。
\end{quote}

大聖作起這大風,將那碎石,乘風亂飛亂舞。可憐把那些千餘人馬,一個個:
\begin{quote}
石打烏頭粉碎,沙飛海馬俱傷。人參官桂嶺前忙,血染朱砂地上。附子難歸故里,檳榔怎得還鄉。屍骸輕粉臥山場,紅娘子家中盼望。
\end{quote}

詩曰:
\begin{quote}
人亡馬死怎歸家,野鬼孤魂亂似麻。
可憐抖擻英雄將,不辨賢愚血染沙。
\end{quote}

大聖按落雲頭,鼓掌大笑道:「造化,造化。自從歸順唐僧,做了和尚,他每每勸我話道:『千日行善,善猶不足;一日行惡,惡自有餘。』真有此話。我跟著他,打殺幾個妖精,他就怪我行兇。今日來家,卻結果了這許多獵戶。」叫:「小的們,出來!」那群猴狂風過去,聽得大聖呼喚,一個個跳將出來。大聖道:「你們去南山下,把那打死的獵戶衣服剝得來家,洗淨血跡,穿了遮寒;把死人的屍首都推在那萬丈深潭內;把死倒的馬拖將來,剝了皮,做靴穿,將肉醃著,慢慢的食用;把那些弓箭槍刀,與你們操演武藝;將那雜色旗號,收來我用。」群猴一個個領諾。

那大聖把旗拆洗,總鬥做一面雜彩花旗,上寫著「重修花果山,復整水簾洞,齊天大聖」十四字。豎起杆子,將旗掛於洞外。逐日招魔聚獸,積草屯糧,不題「和尚」二字。他的人情又大,手段又高,便去四海龍王借些甘霖仙水,把山洗青了。前栽榆柳,後種松柟,桃李棗梅,無所不備。逍遙自在,樂業安居不題。

卻說唐僧聽信狡性,縱放心猿,攀鞍上馬。八戒前邊開路,沙僧挑著行李西行。過了白虎嶺,忽見一帶林坵,真個是藤攀葛繞,柏翠松青。三藏叫道:「徒弟呀,山路崎嶇,甚是難走,卻又松林叢簇,樹木森羅,切須仔細,恐有妖邪妖獸。」你看那獃子抖擻精神,叫沙僧帶著馬,他使釘鈀開路,領唐僧徑入松林之內。正行處,那長老兜住馬道:「八戒,我這一日其實饑了,那裡尋些齋飯我吃?」八戒道:「師父請下馬,在此等老豬去尋。」長老下了馬,沙僧歇了擔,取出缽盂,遞與八戒。八戒道:「我去也。」長老問:「那裡去?」八戒道:「莫管,我這一去,鑽冰取火尋齋至,壓雪求油化飯來。」

你看他出了松林,往西行經十餘里,更不曾撞著一個人家,真是有狼虎無人煙的去處。那獃子走得辛苦,心內沉吟道:「當年行者在日,老和尚要的就有;今日輪到我的身上,誠所謂『當家才知柴米價,養子方曉父娘恩』。公道沒去化處。」卻又走得瞌睡上來,思道:「我若就回去,對老和尚說沒處化齋,他也不信我走了這許多路。須是再多幌個時辰,才好去回話。也罷,也罷,且往這草科裡睡睡。」獃子就把頭拱在草裡睡下。當時也只說朦朧朦朧就起來,豈知走路辛苦的人,丟倒頭,只管齁齁睡起。

且不言八戒在此睡覺。卻說長老在那林間耳熱眼跳,身心不安。急回叫沙僧道:「悟能去化齋,怎麼這早晚還不回?」沙僧道:「師父,你還不曉得哩。他見這西方上人家齋僧的多,他肚子又大,他管你?直等他吃飽了才來哩。」三藏道:「正是呀,倘或他在那裡貪著吃齋,我們那裡會他?天色晚了,此間不是個住處,須要尋個下處方好哩。」沙僧道:「不打緊,師父,你且坐在這裡,等我去尋他來。」三藏道:「正是,正是。有齋沒齋罷了,只是尋下處要緊。」沙僧綽了寶杖,徑出松林來找八戒。

長老獨坐林中,十分悶倦,只得強打精神,跳將起來,把行李攢在一處,將馬拴在樹上。取下戴的斗笠,插定了錫杖,整一整緇衣,徐步幽林,權為散悶。那長老看遍了野草山花,聽不得歸巢鳥噪。原來那林子內都是些草深路小的去處,只因他情思紊亂,卻走錯了。他一來也是要散散悶,二來也是要尋八戒、沙僧。不期他兩個走的是直西路,長老轉了一會,卻走向南邊去了。出得松林,忽擡頭,見那壁廂金光閃爍,彩氣騰騰。仔細看處,原來是一座寶塔,金頂放光。這是那西落的日色,映著那金頂放亮。他道:「我弟子卻沒緣法哩。自離東土,發願逢廟燒香,見佛拜佛,遇塔掃塔。那放光的不是一座黃金寶塔?怎麼就不曾走那條路?塔下必有寺院,院內必有僧家,且等我走走。這行李、白馬,料此處無人行走,卻也無事。那裡若有方便處,待徒弟們來,一同借歇。」

噫!長老一時晦氣到了。你看他拽開步,竟至塔邊。但見那:
\begin{quote}
石崖高萬丈,山大接青霄。根連地厚,峰插天高。兩邊雜樹數千棵,前後藤纏百餘里。花映草梢風有影,水流雲竇月無根。倒木橫擔深澗,枯藤結掛光峰。石橋下,流滾滾清泉;臺座上,長明明白粉。遠觀一似三島天堂,近看有如蓬萊勝境。香松紫竹遶山溪,鴉鵲猿猴穿峻嶺。洞門外,有一來一往的走獸成行;樹林裡,有或出或入的飛禽作隊。青青香草秀,艷艷野花開。這所在分明是惡境,那長老晦氣撞將來。
\end{quote}

那長老舉步進前,才來到塔門之下,只見一個斑竹簾兒掛在裡面。他破步入門,揭起來,往裡就進猛擡頭,見那石床上,側睡著一個妖魔。你道他怎生模樣:
\begin{quote}
青靛臉,白獠牙,一張大口呀呀。兩邊亂蓬蓬的鬢毛,卻都是些胭脂染色;三四紫巍巍的髭髯,恍疑是那荔枝排芽。鸚嘴般的鼻兒拱拱,曙星樣的眼兒巴巴。兩個拳頭,和尚缽盂模樣;二隻藍腳,懸崖榾柮枒槎。斜披著淡黃袍帳,賽過那織錦袈裟。拿的一口刀,精光耀映;眠的一塊石,細潤無瑕。他也曾小妖排蟻陣,他也曾老怪坐蜂衙。你看他威風凜凜,大家吆喝,叫一聲爺。他也曾月作三人壺酌酒,他也曾風生兩腋盞傾茶。你看他神通浩浩,霎著下眼,遊遍天涯。荒林喧鳥雀,深莽宿龍蛇。仙子種田生白玉,道人伏火養丹砂。小小洞門,雖到不得那阿鼻地獄;楞楞妖怪,卻就是一個牛頭夜叉。
\end{quote}

那長老看見他這般模樣,諕得打了一個倒退,遍體酥麻,兩腿酸軟,即忙的抽身便走。剛剛轉了一個身,那妖魔他的靈性著實是強,大撐開著一雙金睛鬼眼,叫聲:「小的們,你看門外是甚麼人?」一個小妖就伸頭望門外一看,看見是個光頭的長老,連忙跑將進去報道:「大王,外面是個和尚哩。團頭大面,兩耳垂肩;嫩刮刮的一身肉,細嬌嬌的一張皮:且是好個和尚。」那妖聞言,啊聲笑道:「這叫做個『蛇頭上蒼蠅,自來的衣食』。你眾小的們,疾忙趕上去,與我拿將來,我這裡重重有賞。」那些小妖就是一窩蜂,齊齊擁上。三藏見了,雖則是一心忙似箭,兩腳走如飛,終是心驚膽顫,腿軟腳麻;況且是山路崎嶇,林深日暮,步兒那裡移得動:被那些小妖平擡將去。正是:
\begin{quote}
龍遊淺水遭蝦戲,虎落平原被犬欺。
縱然好事多磨障,誰像唐僧西向時?
\end{quote}

你看那眾小妖擡得長老,放在那竹簾兒外,歡歡喜喜報聲道:「大王,拿得和尚進來了。」那老妖他也偷眼瞧一瞧,只見三藏頭直上,貌堂堂,果然好一個和尚。他便心中想道:「這等好和尚,必是上方人物,不當小可的。若不做個威風,他怎肯服降哩?」陡然間,就狐假虎威,紅鬚倒豎,血髮朝天,眼睛迸裂,大喝一聲道:「帶那和尚進來!」眾妖們大家響響的答應了一聲:「是!」就把三藏望裡面只是一推。這是「既在矮檐下,怎敢不低頭。」三藏只得雙手合著,與他見個禮。那妖道:「你是那裡和尚?從那裡來?到那裡去?快快說明!」三藏道:「我本是唐朝僧人,奉大唐皇帝敕命,前往西方訪求經偈,經過貴山,特來塔下謁聖,不期驚動威嚴,望乞恕罪。待往西方取得經回東土,永註高名也。」那妖聞言,呵呵大笑道:「我說是上邦人物,果然是你。正要吃你哩,卻來的甚好,甚好,不然,卻不錯放過了?你該是我口內的食,自然要撞將來,就放也放不去,就走也走不脫!」叫小妖:「把那和尚拿去綁了。」果然那些小妖一擁上前,把個長老繩纏索綁,縛在那定魂樁上。

老妖持刀又問道:「和尚,你一行有幾個?終不然一人敢上西天?」三藏見他持刀,又老實說道:「大王,我有兩個徒弟,叫做豬八戒、沙和尚,都出松林化齋去了。還有一擔行李,一匹白馬,都在松林裡放著哩。」老妖道:「又造化了。兩個徒弟,連你三個,連馬四個,夠吃一頓了。」小妖道:「我們去捉他來。」老妖道:「不要出去,把前門關了。他兩個化齋來,一定尋師父吃;尋不著,一定尋著我門上。常言道:『上門的買賣好做。』且等慢慢的捉他。」眾小妖把前門閉了。

且不言三藏逢災。卻說那沙僧出林找八戒,直有十餘里遠近,不曾見個莊村。他卻站在高埠上正然觀看,只聽得草中有人言語,急使杖撥開深草看時,原來是獃子在裡面說夢話哩。被沙僧揪著耳朵,方叫醒了。道:「好獃子啊!師父教你化齋,許你在此睡覺的?」那獃子冒冒失失的醒來道:「兄弟,有甚時候了?」沙僧道:「快起來,師父說有齋沒齋也罷,教你我那裡尋下住處哩。」

獃子懵懵懂懂的托著缽盂,拑著釘鈀,與沙僧徑直回來。到林中看時,不見了師父。沙僧埋怨道:「都是你這獃子化齋不來,必有妖精拿師父也。」八戒笑道:「兄弟,莫要胡說。那林子裡是個清雅的去處,決然沒有妖精。想是老和尚坐不住,往那裡觀風去了。我們尋他去來。」二人只得牽馬挑擔,收拾了斗篷、錫杖,出松林尋找師父。

這一回,也是唐僧不該死。他兩個尋一會不見,忽見那正南下有金光閃灼,八戒道:「兄弟啊,有福的只是有福,你看師父往他家去了。那放光的是座寶塔,誰敢怠慢?一定要安排齋飯,留他在那裡受用。我們還不走動些,也趕上去吃些齋兒。」沙僧道:「哥啊,定不得吉凶哩,我們且去看來。」

二人雄糾糾的到了門前:「呀!閉著門哩。」只見那門上橫安了一塊白玉石板,上鐫著六個大字:「碗子山波月洞」。沙僧道:「哥啊,這不是甚麼寺院,是一座妖精洞府也。我師父在這裡,也見不得哩。」八戒道:「兄弟莫怕。你且拴下馬匹,守著行李,待我問他的信看。」那獃子舉著鈀,上前高叫:「開門!開門!」那洞內有把門的小妖開了門忽見他兩個的模樣,急抽身,跑入裡面報道:「大王,買賣來了。」老妖道:「那裡買賣?」小妖道:「洞門外有一個長嘴大耳的和尚,與一個晦氣色的和尚,來叫門了。」老妖大喜道:「是豬八戒與沙僧尋將來也。噫,他也會尋哩,怎麼就尋到我這門上?既然嘴臉兇頑,卻莫要怠慢了他。」叫:「取披掛來。」小妖擡來,就結束了,綽刀在手,徑出門來。

卻說那八戒、沙僧在門前正等,只見妖魔來得兇險。你道他怎生打扮:
\begin{quote}
青臉紅鬚赤髮飄,黃金鎧甲亮光饒。
裹肚襯腰𥓼石帶,攀胸勒甲步雲絛。
閑立山前風吼吼,悶遊海外浪滔滔。
一雙藍靛焦觔手,執定追魂取命刀。
要知此物名和姓,聲揚二字喚黃袍。
\end{quote}

那黃袍老怪出得門來,便問:「你是那方和尚,在我門首吆喝?」八戒道:「我兒子,你不認得?我是你老爺。我是大唐差往西天去的。我師父是那御弟三藏。若在你家內,趁早送出來,省了我釘鈀築進去。」那怪笑道:「是是是,有一個唐僧在我家,我也不曾怠慢他,安排些人肉包兒與他吃哩。你們也進去吃一個兒,何如?」

這獃子認真就要進去。沙僧一把扯住道:「哥啊,他哄你哩,你幾時又吃人肉哩?」獃子卻才省悟,掣釘鈀,望妖怪劈臉就築;那怪物側身躲過,使鋼刀急架相迎。兩個都顯神通,縱雲頭,跳在空中廝殺。沙僧撇了行李、白馬,舉寶杖,急急幫攻。此時兩個狠和尚,一個潑妖魔,在雲端裡,這一場好殺。正是那:
\begin{quote}
杖起刀迎,鈀來刀架。一員魔將施威,兩個神僧顯化。九齒鈀真個英雄,降妖杖誠然兇咤。沒前後左右齊來,那黃袍公然不怕。你看他蘸鋼刀晃亮如銀,其實的那神通也為廣大。只殺得滿空中霧遶雲迷,半山裡崖崩嶺咋。一個為聲名,怎肯干休;一個為師父,斷然不怕。
\end{quote}

他三個在半空中往往來來,戰經數十回合,不分勝負。各因性命要緊,其實難解難分。

畢竟不知怎救唐僧,且聽下回分解。
