
\chapter{心主夜間修藥物 君王筵上論妖邪}

話表孫大聖同近侍宦官到於皇宮內院,直至寢宮門外立定。將三條金線與宦官拿入裡面,吩咐:「教內宮妃后,或近侍太監,先繫在聖躬左手腕下,按寸、關、尺三部上,卻將線頭從窗櫺兒穿出與我。」真個那宦官依此言,請國王坐在龍床,按寸、關、尺,以金線一頭繫了,一頭理出窗外。行者接了線頭,以自己右手大指先托著食指,看了寸脈;次將中指按大指,看了關脈;又將大指托定無名指,看了尺脈。調停自家呼吸,分定四氣、五鬱、七表、八裡、九候、浮中沉、沉中浮,辨明了虛實之端。又教解下左手,依前繫在右手腕下部位。行者即以左手指,一一從頭診視畢,卻將身抖了一抖,把金線收上身來。厲聲高呼道:「陛下左手寸脈強而緊,關脈濇而緩,尺脈芤且沉;右手寸脈浮而滑,關脈遲而結,尺脈數而牢。夫左寸強而緊者,中虛心痛也;關濇而緩者,汗出肌麻也;尺芤而沉者,小便赤而大便帶血也。右手寸脈浮而滑者,內結經閉也;關遲而結者,宿食留飲也;尺數而牢者,煩滿虛寒相持也。診此貴恙,是一個驚恐憂思,號為『雙鳥失群』之症。」那國王在內聞言,滿心歡喜,打起精神,高聲應道:「指下明白,指下明白,果是此疾。請出外面用藥來也。」

大聖卻才緩步出宮。早有在傍聽見的太監,已先對眾報知。須臾,行者出來,唐僧即問如何。行者道:「診了脈,如今對症製藥哩。」眾官上前道:「神僧長老適才說『雙鳥失群』之症,何也?」行者笑道:「有雌雄二鳥,原在一處同飛,忽被暴風驟雨驚散,雌不能見雄,雄不能見雌,雌乃想雄,雄亦想雌:這不是『雙鳥失群』也?」眾官聞說,齊聲喝采道:「真是神僧!真是神醫!」稱讚不已。當有太醫官問道:「病勢已看出矣,但不知用何藥治之?」行者道:「不必執方,見藥就要。」醫官道:「經云:『藥有八百八味,人有四百四病。』病不在一人之身,藥豈有全用之理?如何見藥就要?」行者道:「古人云:『藥不執方,合宜而用。』故此全徵藥品,而隨便加減也。」

那醫官不復再言,即出朝門之外,差本衙當值之人,遍曉滿城生熟藥鋪,即將藥品。每味各辦三斤,送與行者。行者道:「此間不是製藥處,可將諸藥之數並製藥一應器皿,都送入會同館,交與我師弟二人收下。」醫官聽命,即將八百八味每味三斤及藥碾、藥磨、藥羅、藥乳並乳缽、乳槌之類都送至館中,一一交付收訖。

行者往殿上請師父同至館中製藥。那長老正自起身,忽見內宮傳旨,教閣下留住法師,同宿文華殿。待明朝服藥之後,病痊酬謝,倒換關文送行。三藏大驚道:「徒弟啊,此意是留我做當頭哩。若醫得好,歡喜起送;若醫不好,我命休矣。你須仔細上心,精虔制度也。」行者笑道:「師父放心在此受用,老孫自有醫國之手。」

好大聖,別了三藏,辭了眾臣,徑至館中。八戒迎著笑道:「師兄,我知道你了。」行者道:「你知甚麼?」八戒道:「知你取經之事不果,欲作生涯無本,今日見此處富庶,設法要開藥鋪哩。」行者喝道:「莫胡說,醫好國王,得意處辭朝走路,開甚麼藥鋪?」八戒道:「終不然,這八百八味藥,每味三斤,共計二千四百二十四斤,只醫一人,能用多少?不知多少年代方吃得了哩。」行者道:「那裡用得許多?他那太醫院官都是些愚盲之輩,所以取這許多藥品,教他沒處捉摸,不知我用的是那幾味,難識我神妙之方也。」

正說處,只見兩個館使當面跪下道:「請神僧老爺進晚齋。」行者道:「早間那般待我,如今卻跪而請之,何也?」館使叩頭道:「老爺來時,下官有眼無珠,不識尊顏。今聞老爺大展三折之肱,治我一國之主,若主上病愈,老爺江山有分,我輩皆臣子也,禮當拜請。」行者見說,欣然登堂上坐;八戒、沙僧分坐左右。擺上齋來,沙僧便問道:「師兄,師父在那裡哩?」行者笑道:「師父被國王留住作當頭哩。只待醫好了病,方才酬謝送行。」沙僧又問:「可有些受用麼?」行者道:「國王豈無受用?我來時,他已有三個閣老陪侍左右,請入文華殿去也。」八戒道:「這等說,還是師父大哩:他倒有閣老陪侍,我們只得兩個館使奉承。且莫管他,讓老豬吃頓飽飯也。」兄弟們遂自在受用一番。

天色已晚。行者叫館使:「收了家火,多辦些油蠟,我等到夜靜時,方好製藥。」館使果送若干油蠟,各命散訖。

至半夜,天街人靜,萬籟無聲。八戒道:「哥哥,製何藥?趕早幹事,我瞌睡了。」行者道:「你將大黃取一兩來,碾為細末。」沙僧乃道:「大黃味苦,性寒無毒。其性沉而不浮,其用走而不守。奪諸鬱而無壅滯,定禍亂而致太平。名之曰『將軍』。此行藥耳,但恐久病虛弱,不可用此。」行者笑道:「賢弟不知。此藥利痰順氣,蕩肚中凝滯之寒熱。你莫管我。你去取一兩巴豆,去殼去膜,搥去油毒,碾為細末來。」八戒道:「巴豆味辛,性熱有毒。削堅積,蕩肺腑之沉寒;通閉塞,利水穀之道路。乃斬關奪門之將,不可輕用。」行者道:「賢弟,你也不知。此藥破結宣腸,能理心膨水脹。快製來,我還有佐使之味輔之也。」

他二人即時將二藥碾細道:「師兄,還用那幾十味?」行者道:「不用了。」八戒道:「八百八味,每味三斤,只用此二兩,誠為起奪人了。」行者將一個花磁盞子,道:「賢弟莫講,你拿這個盞兒,將鍋臍灰刮半盞過來。」八戒道:「要怎的?」行者道:「藥內要用。」沙僧道:「小弟不曾見藥內用鍋灰。」行者道:「鍋灰名為『百草霜』,能調百病,你不知道。」那獃子真個刮了半盞,又碾細了。

行者又將盞子遞與他道:「你再去把我們的馬尿等半盞來。」八戒道:「要他怎的?」行者道:「要丸藥。」沙僧又笑道:「哥哥,這事不是耍子。馬尿腥臊,如何入得藥品?我只見醋糊為丸,陳米糊為丸,煉蜜為丸,或只是清水為丸,那曾見馬尿為丸?那東西腥腥臊臊,脾虛的人,一聞就吐;再服巴豆、大黃,弄得人上吐下瀉,可是耍子?」行者道:「你不知就裡。我那馬不是凡馬,他本是西海龍身。若得他肯去便溺,憑你何疾,服之即愈。但急不可得耳。」八戒聞言,真個去到邊前,那馬斜伏地下睡哩。獃子一頓腳踢起,襯在肚下,等了半會,全不見撒尿。他跑將來,對行者說:「哥啊,且莫去醫皇帝,且快去醫醫馬來。那亡人乾結了,莫想尿得出一點兒。」行者笑道:「我和你去。」沙僧道:「我也去看看。」

三人都到馬邊,那馬跳將起來,口吐人言,厲聲高叫道:「師兄,你豈不知?我本是西海飛龍,因為犯了天條,觀音菩薩救了我,將我鋸了角,退了鱗,變作馬,馱師父往西天取經,將功折罪。我若過水撒尿,水中遊魚食了成龍;過山撒尿,山中草頭得味變作靈芝,仙僮採去長壽。我怎肯在此塵俗之處輕拋卻也?」行者道:「兄弟謹言。此間乃西方國王,非塵俗也,亦非輕拋棄也。常言道:『眾毛攢裘。』要與本國之王治病哩。醫得好時,大家光輝;不然,恐俱不得善離此地也。」那馬才叫聲:「等著。」你看他往前撲了一撲,往後蹲了一蹲,咬得那滿口牙齕支支的響喨,僅努出幾點兒,將身立起。八戒道:「這個亡人,就是金汁子,再撒些兒也罷。」那行者見有少半盞,道:「夠了,夠了。拿去罷。」沙僧方才歡喜。

三人回至廳上,把前項藥餌攪和一處,搓了三個大丸子。行者道:「兄弟,忒大了。」八戒道:「只有核桃大,若論我吃,還不夠一口哩。」遂此收在一個小盒兒裡,兄弟們連衣睡下。一夜無詞,早是天曉。

卻說那國王耽病設朝,請唐僧見了,即命眾官快往會同館參拜神僧孫長老取藥去。

多官隨至館中,對行者拜伏於地道:「我王特命臣等拜領妙劑。」行者叫八戒取盒兒,揭開蓋子,遞與多官。多官啟問:「此藥何名?好見王回話。」行者道:「此名烏金丹。」八戒二人暗中作笑道:「鍋灰拌的,怎麼不是烏金?」多官又問道:「用何引子?」行者道:「藥引兒兩般都下得。有一般易取者,乃六物煎湯送下。」多官問:「是何六物?」行者道:「半空飛的老鴉屁,緊水負的鯉魚尿,王母娘娘搽臉粉,老君爐裡煉丹灰,玉皇戴破的頭巾要三塊,還要五根困龍鬚。六物煎湯送此藥,你王憂病等時除。」多官聞言道:「此物乃世間所無者。請問那一般引子是何?」行者道:「用無根水送下。」眾官笑道:「這個易取。」行者道:「怎見得易取?」多官道:「我這裡人家俗論:若用無根水,將一個碗盞,到井邊或河下,舀了水,急轉步,更不落地,亦不回頭,到家與病人吃藥,便是。」行者道:「井中河內之水,俱是有根的。我這無根水,非此之論,乃是天上落下者,不沾地就吃,才叫做無根水。」多官又道:「這也容易。等到天陰下雨時,再吃藥便罷了。」

遂拜謝了行者,將藥持回獻上。國王大喜,即命近侍接上來,看了道:「此是甚麼丸子?」多官道:「神僧說是『烏金丹』,用無根水送下。」國王便教宮人取無根水。眾官道:「神僧說,無根水不是井、河中者,乃是天上落下不沾地的才是。」國王即喚當駕官傳旨,教請法官求雨。眾官遵依出榜不題。

卻說行者在會同館廳上,叫豬八戒道:「適間允他天落之水,才可用藥,此時急忙,怎麼得個雨水?我看這王倒也是個大賢大德之君,我與你助他些兒雨下藥,如何?」八戒道:「怎麼樣助?」行者道:「你在我左邊立下,做個輔星。」又叫沙僧:「你在我右邊立下,做個弼宿。等老孫助他些無根水兒。」好大聖,步了罡訣,念聲咒語,早見那正東上一朵烏雲,漸近於頭頂上。叫道:「大聖,東海龍王敖廣來見。」行者道:「無事不敢捻煩,請你來助些無根水與國王下藥。」龍王道:「大聖呼喚時,不曾說用水,小龍隻身來了,不曾帶得雨器,亦未有風雲雷電,怎生降雨?」行者道:「如今用不著風雲雷電,亦不須多雨,只要些須引藥之水便了。」龍王道:「既如此,待我打兩個噴涕,吐些涎津溢,與他吃藥罷。」行者大喜道:「最好,最好。不必遲疑,趁早行事。」

那老龍在空中漸漸低下烏雲,直至皇宮之上,隱身潛像,噀一口津唾,遂化作甘霖。那滿朝官齊聲喝采道:「我主萬千之喜,天公降下甘雨來也。」國王即傳旨,教:「取器皿盛著,不拘宮內外及官大小,都要等貯仙水,拯救寡人。」你看那文武多官並三宮六院妃嬪與三千綵女、八百嬌娥,一個個擎杯托盞,舉碗持盤,等接甘雨。那老龍在半空運化津涎,不離了王宮前後。將有一個時辰,龍王辭了大聖回海。眾臣將杯盂碗盞收來,也有等著一點兩點者,也有等著三點五點者,也有一點不曾等著者,共合一處,約有三盞之多,總獻至御案。真個是異香滿襲金鑾殿,佳味熏飄天子庭。

那國王辭了法師,將著烏金丹並甘雨至宮中,先吞了一丸,吃了一盞甘雨;再吞了一丸,又飲了一盞甘雨;三次,三丸俱吞了,三盞甘雨俱送下。不多時,腹中作響,如轆轤之聲不絕。即取淨桶,連行了三五次。服了些米飲,攲倒在龍床之上。有兩個妃子將淨桶檢看,說不盡那穢污痰涎,內有糯米飯塊一團。妃子近龍床前來報:「病根都行下來也。」國王聞此言,甚喜,又進一次米飯。

少頃,漸覺胸心寬泰,氣血調和,就精神抖擻,腳力強健。下了龍床,穿上朝服,即登寶殿,見了唐僧,輒倒身下拜。那長老忙忙還禮。拜畢,以御手攙著,便教閣下:「快具簡帖,帖上寫朕『再拜頓首』字樣,差官奉請法師高徒三位。一壁廂大開東閣,光祿寺排宴酬謝。」多官領旨,具簡的具簡,排宴的排宴,正是:國家有倒山之力,霎時俱完。

卻說八戒見官投簡,喜不自勝道:「哥啊,果是好妙藥。今來酬謝,乃兄長之功。」沙僧道:「二哥說那裡話,常言道:『一人有福,帶挈一屋。』我們在此合藥,俱是有功之人。只管受用去,再休多話。」咦!你看他弟兄們俱歡歡喜喜,徑入朝來。

眾官接引,上了東閣,早見唐僧、國王、閣老,已都在那裡安排筵宴哩。這行者與八戒、沙僧對師父唱了個喏。隨後眾官都至。只見那上面有四張素桌面,都是吃一看十的筵席。前面有一張葷桌面,也是吃一看十的珍饈。左右有四五百張單桌面,真個排得齊整:
\begin{quote}
古云:「珍饈百味,美祿千鍾。瓊膏酥酪,錦縷肥紅。」寶妝花彩豔,果品味香濃。斗糖龍纏列獅仙,餅錠拖爐擺鳳侶。葷有豬羊雞鵝魚鴨般般肉,素有蔬餚筍芽木耳並蘑菇。幾樣香湯餅,數次透糖酥。滑軟黃粱飯,清新菰米糊。色色粉湯香又辣,般般添換美還甜。君臣舉盞方安席,名分品級慢傳壺。
\end{quote}

那國王御手擎杯,先與唐僧安坐。三藏道:「貧僧不會飲酒。」國王道:「素酒,法師飲此一杯何如?」三藏道:「酒乃僧家第一戒。」國王甚不過意道:「法師戒飲,卻以何物為敬?」三藏道:「頑徒三眾代飲罷。」國王卻才歡喜,轉金卮,遞與行者。行者接了酒,對眾禮畢,吃了一杯。國王見他吃得爽利,又奉一杯。行者不辭,又吃了。國王笑道:「吃個三寶鍾兒。」行者不辭,又吃了。國王又叫斟上,吃個四季杯兒。

八戒在旁,見酒不到他,忍得他嘓嘓嚥唾。又見那國王苦勸行者,他就叫將起來道:「陛下,吃的藥也虧了我,那藥裡有馬」這行者聽說,恐怕獃子走了消息,卻將手中酒遞與八戒。八戒接著就吃,卻不言語。國王問道:「神僧說藥裡有馬,是甚麼馬?」行者接過口來道:「我這兄弟是這般口敞,他有個經驗的好方兒,他就要說與人。陛下早間吃藥,內有馬兜鈴。」國王問眾官道:「馬兜鈴是何品味?能醫何症?」時有太醫院官在旁道:「主公:
\begin{quote}
兜鈴味苦寒無毒,定喘消痰大有功。
通氣最能除血蠱,補虛寧嗽又寬中。」
\end{quote}

國王笑道:「用得當,用得當。豬長老再飲一杯。」獃子亦不言語,卻也吃了個三寶鍾。國王又遞了沙僧酒,也吃了三杯,卻俱敘坐。

飲宴多時,國王又擎大爵,奉與行者。行者道:「陛下請坐。老孫依巡痛飲,決不敢推辭。」國王道:「神僧恩重如山,寡人酬謝不盡。好歹進此一巨觥,朕有話說。」行者道:「有甚話說了,老孫好飲。」國王道:「寡人有數載憂疑病,被神僧一貼靈丹打通,所以就好了。」行者笑道:「昨日老孫看了陛下,已知是憂疑之疾,但不知憂疑何事?」國王道:「古人云:『家醜不可外談。』奈神僧是朕恩主,惟不笑,方可告之。」行者道:「怎敢笑話?請說無妨。」國王道:「神僧東來,不知經過幾個邦國?」行者道:「經有五六處。」又問:「他國之后,不知是何稱呼。」行者道:「國王之后,都稱為正宮、東宮、西宮。」國王道:「寡人不是這等稱呼:將正宮稱為金聖宮,東宮稱為玉聖宮,西宮稱為銀聖宮。現今只有銀、玉二后在宮。」行者道:「金聖宮因何不在宮中?」國王滴淚道:「不在已三年矣。」行者道:「向那廂去了?」國王道:「三年前,正值端陽之節,朕與嬪后都在御花園海榴亭下解粽插艾,飲菖蒲雄黃酒,看鬥龍舟。忽然一陣風至,半空中現出一個妖精,自稱賽太歲,說他在麒麟山獬豸洞居住,洞中少個夫人,訪得我金聖宮生得貌美嬌姿,要做個夫人,教朕快早送出;如若三聲不獻出來,就要先吃寡人,後吃眾臣,將滿城黎民盡皆吃絕。那時節,朕卻憂國憂民,無奈,將金聖宮推出海榴亭外,被那妖響一聲攝將去了。寡人為此著了驚恐,吃那粽子,凝滯在內;況又晝夜憂思不息:所以成此苦疾三年。今得神僧靈丹服後,行了數次,盡是那三年前積滯之物,所以這會體健身輕,精神如舊。今日之命,皆是神僧所賜,豈但如泰山之重而已乎!」

行者聞得此言,滿心喜悅,將那巨觥之酒,兩口吞之,笑問國王曰:「陛下原來是這般驚憂。今遇老孫,幸而獲愈。但不知可要金聖宮回國?」那國王滴淚道:「朕切切思思,無晝無夜,但只是沒一個能獲得妖精的,豈有不要他回國之理?」行者道:「我老孫與你去伏妖邪,何如?」國王跪下道:「若救得朕后,朕願領三宮九嬪,出城為民,將一國江山,盡付神僧,讓你為帝。」八戒在旁,見出此言,行此禮,忍不住呵呵大笑道:「這皇帝失了體統,怎麼為老婆就不要江山,跪著和尚?」行者急上前,將國王攙起道:「陛下,那妖精自得金聖宮去後,這一向可曾再來?」國王道:「他前年五月節攝了金聖宮,至十月間,來要兩個宮娥,說是伏侍娘娘,朕即獻出兩個;至舊年三月間,又來要兩個宮娥;七月間,又要去兩個;今年二月裡,又要去兩個。不知到幾時又要來也。」行者道:「似他這等頻來,你們可怕他麼?」國王道:「寡人見他來得多遭,一則懼怕,二來又恐有傷害之意。舊年四月內,是朕命工起了一座避妖樓,但聞風響,知是他來,即與二后、九嬪入樓躲避。」行者道:「陛下不棄,可攜老孫去看那避妖樓一番,何如?」那國王即將左手攜著行者出席。眾官一齊起身。豬八戒道:「哥哥,你不達禮。這般御酒不吃,搖席破坐的,且去看甚麼哩?」國王聞說,情知八戒是為嘴,即命當駕官擡兩張素桌面看酒,在避妖樓外伺候。獃子卻才不嚷,同師父、沙僧笑道:「翻席去也。」

一行文武官引導,那國王並行者相攙,穿過皇宮,到了御花園後,更不見樓臺殿閣。行者道:「避妖樓何在?」說不了,只見兩個太監拿兩根紅漆扛子,往那空地上掬起一塊四方石板。國王道:「此間便是。這底下有三丈多深,穵成的九間朝殿。內有四個大缸,缸內滿注清油,點著燈火,晝夜不息。寡人聽得風響,就入裡邊躲避,外面著人蓋上石板。」行者笑道:「那妖精還是不害你;若要害你,這裡如何躲得?」正說間,只見那正南上,呼呼的吹得風響,播土揚塵。諕得那多官齊聲報怨道:「這和尚鹽醬口,講甚麼妖精,妖精就來了。」慌得那國王丟了行者,即鑽入地穴;唐僧也就跟入;眾官亦躲個乾淨。

八戒、沙僧也都要躲,被行者左右手扯住他兩個道:「兄弟們不要怕得,我和你認他一認,看是個甚麼妖精?」八戒道:「可是扯淡,認他怎的?眾官躲了,師父藏了,國王避了,我們不去了罷,衒的是那家世?」那獃子左掙右掙,掙不得脫手。被行者拿定多時,只見那半空裡閃出一個妖精。你看他怎生模樣:
\begin{quote}
九尺長身多惡獰,一雙環眼閃金燈。
兩輪查耳如撐扇,四個鋼牙似插釘。
鬢繞紅毛眉豎焰,鼻垂糟準孔開明。
髭髯幾縷朱砂線,顴骨崚嶒滿面青。
兩臂紅觔藍靛手,十條尖爪把槍擎。
豹皮裙子腰間繫,赤腳蓬頭若鬼形。
\end{quote}

行者見了道:「沙僧,你可認得他?」沙僧道:「我又不曾與他相識,那裡認得?」又問:「八戒,你可認得他?」八戒道:「我又不曾與他會茶會酒,又不是賓朋鄰里,我怎麼認得他?」行者道:「他卻像東嶽天齊手下把門的那個醮面金睛鬼。」八戒道:「不是,不是。」行者道:「你怎知他不是?」八戒道:「鬼乃陰靈也,一日至晚,交申酉戌亥時方出。今日還在巳時,那裡有鬼敢出來?就是鬼,也不會駕雲。縱會弄風,也只是一陣旋風耳,有這等狂風?或者他就是賽太歲也。」行者笑道:「好獃子,倒也有些論頭。既如此說,你兩個護持在此,等老孫去問他個名號,好與國王救取金聖宮來朝。」八戒道:「你去自去,切莫供出我們來。」行者昂然不答,急縱祥光,跳將上去。咦!正是:
\begin{quote}
安邦先卻君王病,守道須除愛惡心。
\end{quote}

畢竟不知此去到於空中,勝敗如何,怎麼擒得妖怪,救得金聖宮,且聽下回分解。
