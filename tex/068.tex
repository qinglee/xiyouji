
\chapter{朱紫國唐僧論前世 孫行者施為三折肱}

\begin{quote}
善正萬緣收,名譽傳揚四部洲。智慧光明登彼岸,颼颼,靉靉雲生天際頭。諸佛共相酬,永住瑤臺萬萬秋。打破人間蝴蝶夢,休休,滌淨塵氛不惹愁。
\end{quote}

話表三藏師徒洗污穢之衚衕,上逍遙之道路,光陰迅速,又值炎天。正是:
\begin{quote}
海榴舒錦彈,荷葉綻青盤。兩路綠楊藏乳燕,行人避暑扇搖紈。
\end{quote}

進前行處,忽見有一城池相近。三藏勒馬叫:「徒弟們,你看那是甚麼去處?」行者道:「師父原來不識字,虧你怎麼領唐王旨意離朝也?」三藏道:「我自幼為僧,千經萬典皆通,怎麼說我不識字?」行者道:「既識字,怎麼那城頭上杏黃旗,明書三個大字,就不認得,卻問是甚去處何也?」三藏喝道:「這潑猴胡說。那旗被風吹得亂擺,縱有字也看不明白。」行者道:「老孫偏怎看見?」八戒、沙僧道:「師父,莫聽師兄搗鬼。這般遙望,城池尚不明白,如何就見是甚字號?」行者道:「卻不是『朱紫國』三字?」三藏道:「朱紫國必是西邦王位,卻要倒換關文。」行者道:「不消講了。」

不多時,至城門下馬,過橋,入進三層門裡,真個好個皇州,但見:
\begin{quote}
門樓高聳,垛疊齊排。周圍活水通流,南北高山相對。六街三市貨資多,萬戶千家生意盛。果然是個帝王都會處,天府大京城。絕域梯航至,遐方玉帛盈。形勝連山遠,宮垣接漢清。三關嚴鎖鑰,萬古樂昇平。
\end{quote}

師徒們在那大街市上行時,但見人物軒昂,衣冠齊整,言語清朗,真不亞大唐世界。那兩邊做買做賣的,忽見豬八戒相貌醜陋,沙和尚面黑身長,孫行者臉毛額廓,丟了買賣,都來爭看。三藏只叫:「不要撞禍,低著頭走。」八戒遵依,把個蓮蓬嘴揣在懷裡;沙僧不敢仰視;惟行者東張西望,緊隨唐僧左右。那些人有知事的,看看兒就回去了。有那遊手好閑的,並那頑童們,烘烘笑笑,都上前拋瓦丟磚,與八戒作戲。唐僧捏著一把汗,只教:「莫要生事。」那獃子不敢擡頭。

不多時,轉過隅頭,忽見一座門牆,上有「會同館」三字。唐僧道:「徒弟,我們進這衙門去也。」行者道:「進去怎的?」唐僧道:「會同館乃天下通會通同之所,我們也打攪得。且到裡面歇下,待我見駕,倒換了關文,再趕出城走路。」八戒聞言,掣出嘴來,把那些隨看的人諕倒了數十個。他上前道:「師父說的是,我們且到裡邊藏下,免得這夥鳥人噪嚷。」遂進館去。那些人方漸漸而退。

卻說那館中有兩個大使,乃是一正一副,都在廳上查點人夫,要往那裡接官。忽見唐僧來到,個個心驚,齊道:「是甚麼人?是甚麼人?往那裡走?」三藏合掌道:「貧僧乃東土大唐駕下差往西天取經者。今到寶方,不敢私過,有關文欲倒驗放行,權借高衙暫歇。」那兩個館使聽言,屏退左右,一個個整冠束帶,下廳迎上相見。即命打掃客房安歇,教辦清素支應。三藏謝了。二官帶領人夫,出廳而去。手下人請老爺客房安歇,三藏便走。行者恨道:「這廝憊𪬯,怎麼不讓老孫在正廳?」三藏道:「他這裡不服我大唐管屬,又不與我國相連,況不時又有上司過客來往,所以不好留此相待。」行者道:「這等說,我偏要他相待。」

正說處,有管事的送支應來,乃是一盤白米、一盤白麵、兩把青菜、四塊豆腐、兩個麵觔、一盤乾筍、一盤木耳。三藏教徒弟收了,謝了管事的。管事的道:「西房裡有乾淨鍋灶,柴火方便,請自去做飯。」三藏道:「我問你一聲:國王可在殿上麼?」管事的道:「我萬歲爺爺久不上朝,今日乃黃道良辰,正與文武多官議出黃榜。你若要倒換關文,趁此急去,還趕上;到明日,就不能夠了,不知還有多少時伺候哩。」三藏道:「悟空,你們在此安排齋飯,等我急急去驗了關文回來,吃了走路。」八戒急取出袈裟關文。三藏整束了進朝,只是吩咐徒弟,不可出外去生事。

不一時,已到五鳳樓前。說不盡那殿閣崢嶸,樓臺壯麗。直至端門外,煩奏事官轉達天廷,欲倒驗關文。那黃門官果至玉階前啟奏道:「朝門外有東土大唐欽差一員僧,前往西天雷音寺拜佛求經,欲倒換通關文牒,聽宣。」國王聞言,喜道:「寡人久病,不曾登基。今上殿出榜招醫,就有高僧來國。」即傳旨宣至階下。三藏即禮拜俯伏。國王又宣上金殿賜坐,命光祿寺辦齋。三藏謝了恩,將關文獻上。

國王看畢,十分歡喜道:「法師,你那大唐,幾朝君正?幾輩臣賢?至於唐王,因甚作疾回生,著你遠涉山川求經?」這長老因問,即欠身合掌道:「貧僧那裡:
\begin{quote}
三皇治世,五帝分倫。堯舜正位,禹湯安民。成周子眾,各立乾坤。倚強欺弱,分國稱君。邦君十八,分野邊塵。後成十二,宇宙安淳。因無車馬,卻又相吞。七雄爭勝,六國歸秦。天生魯沛,各懷不仁。江山屬漢,約法欽遵。漢歸司馬,晉又紛紜。南北十二,宋齊梁陳。列祖相繼,大隋紹真。賞花無道,塗炭多民。我王李氏,國號唐君。高祖晏駕,當今世民。河清海晏,大德寬仁。茲因長安城北,有個怪水龍神,刻減甘雨,應該損身。夜間託夢,告王救迍。王言准赦,早召賢臣。款留殿內,慢把棋輪。時當日午,那賢臣夢斬龍身。」
\end{quote}

國王聞言,忽作呻吟之聲,問道:「法師,那賢臣是那邦來者?」三藏道:「就是我王駕前丞相,姓魏名徵。他識天文,知地理,辨陰陽,乃安邦立國之大宰輔也。因他夢斬了涇河龍王,那龍王告到陰司,說我王許救又殺之,故我王遂得促病,漸覺身危。魏徵又寫書一封,與我王帶至陰司,寄與酆都城判官崔珏。少時,唐王身死,至三日復得回生。虧了魏徵,感崔判官改了文書,加王二十年壽。今要做水陸大會,故遣貧僧遠涉道途,詢求諸國,拜佛祖,取《大乘經》三藏,超度孽苦昇天也。」那國王又呻吟嘆道:「誠乃是天朝大國,君正臣賢。似我寡人久病多時,並無一臣拯救。」長老聽說,偷睛觀看,見那皇帝面黃肌瘦,形脫神衰。長老正欲啟問,有光祿寺官奏請唐僧奉齋。王傳旨,教「在披香殿,連朕之膳擺下,與法師同享。」三藏謝了恩,與王同進膳進齋不題。

卻說行者在會同館中,著沙僧安排茶飯,並整治素菜。沙僧道:「茶飯易煮,蔬菜不好安排。」行者問道:「如何?」沙僧道:「油、鹽、醬、醋俱無也。」行者道:「我這裡有幾文襯錢,教八戒上街買去。」那獃子躲懶道:「我不敢去,嘴臉欠俊,恐惹下禍來,師父怪我。」行者道:「公平交易,又不化他,又不搶他,何禍之有?」八戒道:「你才不曾看見獐智?在這門前扯出嘴來,把人諕倒了十來個;若到鬧市叢中,也不知諕殺多少人哩。」行者道:「你只知鬧市叢中,你可曾看見那市上賣的是甚麼東西?」八戒道:「師父只教我低著頭,莫撞禍,實是不曾看見。」行者道:「酒店、米鋪、磨坊並綾羅雜貨不消說,著然又好茶房、麵店、大燒餅、大饝饝,飯店又有好湯飯、好椒料、好蔬菜,與那異品的糖糕、蒸酥、點心、𩜇子、油食、蜜食,無數好東西,我去買些兒請你如何?」那獃子聞說,口內流涎,喉嚨裡嘓嘓的嚥唾,跳起來道:「哥哥,這遭我擾你,待下次趲錢,我也請你回席。」行者暗笑道:「沙僧,好生煮飯,等我們去買調和來。」沙僧也知是耍獃子,只得順口應承道:「你們去,須是多買些,吃飽了來。」那獃子撈個碗盞拿了,就跟行者出門。有兩個在官人問道:「長老那裡去?」行者道:「買調和。」那人道:「這條街往西去,轉過拐角鼓樓,那鄭家雜貨店,憑你買多少,油、鹽、醬、醋、薑、椒、茶葉俱全。」

他二人攜手相攙,徑上街西而去。行者過了幾處茶房,幾家飯店,當買的不買,當吃的不吃。八戒叫道:「師兄,這裡將就買些用罷。」那行者原是耍他,那裡肯買,道:「賢弟,你好不經紀,再走走,揀大的買吃。」兩個人說說話兒,又領了許多人跟隨爭看。不時到了鼓樓邊,只見那樓下無數人喧嚷,擠擠挨挨,填街塞路。八戒見了道:「哥哥,我不去了。那裡人嚷得緊,只怕是拿和尚的,又況是面生可疑之人,拿了去,怎的了?」行者道:「胡談!和尚又不犯法,拿我怎的?我們走過去,到鄭家店買些調和來。」八戒道:「罷罷罷,我不撞禍。這一擠到人叢裡,把耳朵捽了兩拄,諕得他跌跌爬爬,跌死幾個,我倒償命哩!」行者道:「既然如此,你在這壁根下站定,等我過去買了回來,與你買素麵、燒餅吃罷。」那獃子將碗盞遞與行者,把嘴拄著牆根,背著臉,死也不動。

這行者走至樓邊,果然擠塞。直挨入人叢裡聽時,原來是那皇榜張掛樓下,故多人爭看。行者擠到近處,閃開火眼金睛,仔細看時,那榜上卻云:
\begin{quote}
朕西牛賀洲朱紫國王,自立業以來,四方平服,百姓清安。近因國事不祥,沉痾伏枕,淹延日久難痊。本國太醫院屢選良方,未能調治。今出此榜文,普招天下賢士。不拘北往東來,中華外國,若有精醫藥者,請登寶殿,療理朕躬。稍得病愈,願將社稷平分,決不虛示。為此出給張掛。須至榜者。
\end{quote}

覽畢,滿心歡喜道:「古人云:『行動有三分財氣。』早是不在館中獃坐。即此不必買甚調和,且把取經事寧耐一日,等老孫做個醫生耍耍。」

好大聖,彎倒腰,丟了碗盞,拈一撮土,往上灑去,念聲咒語,使個隱身法,輕輕的上前揭了榜。朝著巽地上吸口仙氣吹來,立起一陣旋風,將人都吹散。他卻回身,徑到八戒站處,只見那獃子嘴拄著牆根,卻是睡著了一般。行者更不驚他,將榜文摺了,輕輕揣在他懷裡,拽轉步,先往會同館去了不題。

卻說那樓下眾人見風起時,各各蒙頭閉眼。不覺風過時,沒了皇榜,眾皆悚懼。那榜原有十二個太監、十二個校尉早朝領出,才掛不上三個時辰,被風吹去,戰兢兢左右追尋,忽見豬八戒懷中露出個紙邊兒來。眾人近前道:「你揭了榜來耶?」那獃子猛擡頭,把嘴一𢵮。諕得那幾個校尉踉踉蹡蹡,跌倒在地。他卻轉身要走,又被面前幾個膽大的扯住道:「你揭了招醫的皇榜,還不進朝醫治我萬歲去,卻待何往?」那獃子慌慌張張道:「你兒子便揭了皇榜,你孫子便會醫治。」校尉道:「你懷中揣的是甚?」獃子卻才低頭看時,真個有一張字紙。展開一看,咬著牙罵道:「那猢猻害殺我也。」恨一聲,便要扯破。早被眾人架住道:「你是死了。此乃當今國王出的榜文,誰敢扯壞?你既揭在懷中,必有醫國之手,快同我去。」八戒喝道:「汝等不知。這榜不是我揭的,是我師兄孫悟空揭的。他暗暗揣在我懷中,他卻丟下我去了。若得此事明白,我與你尋他去。」眾人道:「說甚麼亂話?現鐘不打打鑄鐘?你現揭了榜文,教我們尋誰?不管你,扯了去見主上。」那夥人不分清白,將獃子推推扯扯。這獃子立定腳,就如生了根一般,十來個人也弄他不動。八戒道:「汝等不知高低,再扯一會,扯得我獃性子發了,你卻休怪。」

不多時,鬧動了街坊,將他圍繞。內有兩個年老的太監道:「你這相貌稀奇,聲音不對,是那裡來的,這般村強?」八戒道:「我們是東土差往西天取經的。我師父乃唐王御弟法師,卻才入朝,倒換關文去了。我與師兄來此買辦調和,我見樓下人多,未曾敢去,是我師兄教我在此等候。他原來見有榜文,弄陣旋風揭了,暗揣我懷內,先去了。」那太監道:「我先前見個白面胖和尚,徑奔朝門而去,想就是你師父?」八戒道:「正是,正是。」太監道:「你師兄往那裡去了?」八戒道:「我們一行四眾,師父去倒換關文,我三眾並行囊、馬匹俱歇在會同館。師兄弄了我,他先回館中去了。」太監道:「校尉不要扯他,我等同到館中,便知端的。」八戒道:「你這兩個奶奶知事。」眾校尉道:「這和尚委不識貨,怎麼趕著公公叫起奶奶來耶?」八戒笑道:「不羞,你這反了陰陽的。他二位老媽媽兒,不叫他做婆婆、奶奶,倒叫他做公公?」眾人道:「莫弄嘴,快尋你師兄去。」

那街上人吵吵鬧鬧,何止三五百,共扛到館門首。八戒道:「列位住了。我師兄卻不比我任你們作戲。他卻是個猛烈認真之士。汝等見了,須要行個大禮,叫他聲孫老爺,他就招架了;不然啊,他就變了嘴臉,這事卻弄不成也。」眾太監、校尉俱道:「你師兄果有手段,醫好國王,他也該有一半江山,我等合該下拜。」

那些閑雜人都在門外諠譁。八戒領著一行太監、校尉,徑入館中。只聽得行者與沙僧在客房裡正說那揭榜之事耍笑哩。八戒上前扯住,亂嚷道:「你可成個人?哄我去買素麵、燒餅、饝饝我吃,原來都是空頭。又弄旋風,揭了甚麼皇榜,暗暗的揣在我懷裡,拿我裝胖。這可成個弟兄?」行者笑道:「你這獃子,想是錯了路,走向別處去。我過鼓樓,買了調和,急回來尋你不見,我先來了。在那裡揭甚皇榜?」八戒道:「現在看榜的官員在此。」說不了,只見那幾個太監、校尉朝上禮拜道:「孫老爺,今日我王有緣,天遣老爺下降。是必大展經綸手,微施三折肱,治得我王病愈,江山有分,社稷平分也。」行者聞言,正了聲色,接了八戒的榜文,對眾道:「你們想是看榜的官麼?」太監叩頭道:「奴婢乃司禮監內臣。這幾個是錦衣校尉。」行者道:「這招醫榜,委是我揭的,故遣我師弟引見。既然你主有病,常言道:『藥不輕賣,病不討醫。』你去教那國王親來請我,我有手到病除之功。」太監聞言,無不驚駭。校尉道:「口出大言,必有度量。我等著一半在此啞請,著一半入朝啟奏。」

當分了四個太監、六個校尉,更不待宣召,徑入朝,當階奏道:「主公萬千之喜。」那國王正與三藏膳畢清談,忽聞此奏,問道:「喜自何來?」太監奏道:「奴婢等早領出招醫皇榜,鼓樓下張掛。有東土大唐遠來取經的一個聖僧孫長老揭了,現在會同館內,要王親自去請他,他有手到病除之功。故此特來啟奏。」國王聞言,滿心歡喜,就問唐僧道:「法師有幾位高徒?」三藏合掌答曰:「貧僧有三個頑徒。」國王問:「那一位高徒善醫?」三藏道:「實不瞞陛下說,我那頑徒,俱是山野庸才,只會挑包背馬,轉澗尋波,帶領貧僧登山涉嶺,或者到峻險之處,可以伏魔擒怪,捉虎降龍而已,更無一個能知藥性者。」國王道:「法師何必太謙?朕當今日登殿,幸遇法師來朝,誠天緣也。高徒既不知醫,他怎肯揭我榜文,教寡人親迎?斷然有醫國之能也。」叫:「文武眾卿,寡人身虛力怯,不敢乘輦。汝等可替寡人,俱到朝外,敦請孫長老,看朕之病。汝等見他,切不可輕慢,稱他做『神僧孫長老』,皆以君臣之禮相見。」

那眾臣領旨,與看榜的太監、校尉徑至會同館,排班參拜。諕得那八戒躲在廂房,沙僧閃於壁下。那大聖看他坐在當中,端然不動。八戒暗地裡怨惡道:「這猢猻活活的折殺也。怎麼這許多官員禮拜,更不還禮,也不站將起來?」不多時,禮拜畢,分班啟奏道:「上告神僧孫長老:我等俱朱紫國王之臣,今奉王旨,敬以潔禮參請神僧,入朝看病。」行者方才立起身來,對眾道:「你王如何不來?」眾臣道:「我王身虛力怯,不敢乘輦,特令臣等行代君之禮,拜請神僧也。」行者道:「既如此說,列位請前行,我當隨至。」眾臣各依品從,作隊而走。行者整衣而起。八戒道:「哥哥,切莫攀出我們來。」行者道:「我不攀你,只要你兩個與我收藥。」沙僧道:「收甚麼藥?」行者道:「凡有人送藥來與我,照數收下,待我回來取用。」二人領諾不題。

這行者即同多官頃間便到。眾臣先走,奏知那國王,高捲珠簾,閃龍睛鳳目,開金口御言,便問:「那一位是神僧孫長老?」行者進前一步,厲聲道:「老孫便是。」那國王聽得聲音兇狠,又見相貌刁鑽,諕得戰兢兢,跌在龍床之上。慌得那女官內宦,急扶入宮中。道:「諕殺寡人也!」眾官都嗔怨行者道:「這和尚怎麼這等粗魯村疏?怎敢就擅揭榜?」

行者聞言,笑道:「列位錯怪了我也。若像這等慢人,你國王之病,就是一千年也不得好。」眾臣道:「人生能有幾多陽壽?就一千年也還不好?」行者道:「他如今是個病君,死了是個病鬼,再轉世也還是個病人,卻不是一千年也還不好?」眾臣怒曰:「你這和尚甚不知禮,怎麼敢這等滿口胡柴?」行者笑道:「不是胡柴,你都聽我道來:
\begin{quote}
醫門理法至微玄,大要心中有轉旋。
望聞問切四般事,缺一之時不備全:
第一望他神氣色,潤枯肥瘦起和眠;
第二聞聲清與濁,聽他真語及狂言;
三問病原經幾日,如何飲食怎生便;
四才切脈明經絡,浮沉表裡是何般。
我不望聞並問切,今生莫想得安然。」
\end{quote}

那兩班文武叢中,有太醫院官,一聞此言,對眾稱揚道:「這和尚也說得有理。就是神仙看病,也須望、聞、問、切,謹合著神聖功巧也。」

眾官依此言,著近侍傳奏道:「長老要用望、聞、問、切之理,方可認病用藥。」那國王睡在龍床上,聲聲喚道:「叫他去罷,寡人見不得生人面了。」近侍的出宮來道:「那和尚,我王旨意,教你去罷,見不得生人面哩。」行者道:「若見不得生人面啊,我會懸絲診脈。」眾官暗喜道:「懸絲診脈,我等耳聞,不曾眼見。再奏去來。」那近侍的又入宮奏道:「主公,那孫長老不見主公之面,他會懸絲診脈。」國王心中暗想道:「寡人病了三年,未曾試此,宣他進來。」近侍的即忙傳出道:「主公已許他懸絲診脈,快宣孫長老進宮診視。」

行者卻就上了寶殿。唐僧迎著罵道:「你這潑猴,害了我也。」行者笑道:「好師父,我倒與你壯觀,你返說我害你?」三藏喝道:「你跟我這幾年,那曾見你醫好誰來?你連藥性也不知,醫書也未讀,怎麼大膽撞這個大禍?」行者笑道:「師父,你原來不曉得。我有幾個草頭方兒,能治大病,管情醫得他好便了。就是醫殺了,也只問得個庸醫殺人罪名,也不該死,你怕怎的!不打緊,不打緊,你且坐下,看我的脈理如何?」長老又道:「你那曾見《素問》、《難經》、《本草》、《脈訣》是甚般章句,怎生註解?就這等胡說亂道,會甚麼懸絲診脈?」行者笑道:「我有金線在身,你不曾見哩。」即伸手下去,尾上拔了三根毫毛,捻一把,叫聲:「變!」即變作三條絲線,每條各長二丈四尺,按二十四氣,托於手內,對唐僧道:「這不是我的金線?」近侍宦官在傍道:「長老且休講口,請入宮中診視去來。」行者別了唐僧,隨著近侍入宮看病。正是那:
\begin{quote}
心有秘方能治國,內藏妙訣註長生。
\end{quote}

畢竟這去不知看出甚麼病來,用甚麼藥品。欲知端的,且聽下回分解。
