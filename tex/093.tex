
\chapter{給孤園問古談因 天竺國朝王遇偶}

\begin{quote}
起念斷然有愛,留情必定生災。
靈明何事辨三臺。行滿自歸元海。
不論成仙成佛,須從個裡安排。
清清淨淨絕塵埃。果正飛昇上界。
\end{quote}

卻說寺僧天明不見了三藏師徒,都道:「不曾留得,不曾別得,不曾求告得,清清的把個活菩薩放得走了。」正說處,只見南關廂有幾個大戶來請。眾僧撲掌道:「昨晚不曾防禦,今夜都駕雲去了。」眾人齊望空拜謝。此言一講,滿城中官員人等盡皆知之。叫此大戶人家,俱治辦五牲花果,往生祠祭獻酬恩不題。

卻說唐僧四眾餐風宿水,一路平寧,行有半個多月。忽一日,見座高山。唐僧又悚懼道:「徒弟,那前面山嶺峻峭,是必小心。」行者笑道:「這邊路上將近佛地,斷乎無甚妖邪,師父放懷勿慮。」唐僧道:「徒弟,雖然佛地不遠,但前日那寺僧說,到天竺國都下有二千里,還不知是有多少路哩。」行者道:「師父,你好是又把烏巢禪師《心經》忘記了也。」三藏道:「《般若心經》是我隨身衣缽,自那烏巢禪師教後,那一日不念?那一時得忘?顛倒也念得來,怎會忘得?」行者道:「師父只是念得,不曾求那師父解得。」三藏說:「猴頭,怎又說我不曾解得?你解得麼?」行者道:「我解得,我解得。」自此,三藏、行者再不作聲。旁邊笑倒一個八戒,喜壞一個沙僧,說道:「嘴巴,替我一般的做妖精出身,又不是那裡禪和子聽過講經,那裡應佛僧也曾見過說法。弄虛頭,找架子,說甚麼『曉得』、『解得』。怎麼就不作聲?聽講,請解。」沙僧說:「二哥,你也信他?大哥扯長話,哄師父走路。他曉得弄棒罷了,他那裡曉得講經?」三藏道:「悟能、悟淨,休要亂說。悟空解得是無言語文字,乃是真解。」

他師徒們正說話間,卻倒也走過許多路程,離了幾個山岡,路旁早見一座大寺。三藏道:「悟空,前面是座寺啊。你看那寺,倒也:
\begin{quote}
不小不大,卻也是琉璃碧瓦;半新半舊,卻也是八字紅牆。隱隱見蒼松偃蓋,也不知是幾千百年間故物到於今;潺潺聽流水鳴絃,也不道是那朝代時分開山留得在。山門上,大書著『布金禪寺』;懸匾上,留題著『上古遺跡』。」
\end{quote}

行者看得是「布金禪寺」,八戒也道是「布金禪寺」。三藏在馬上沉思道:「『布金』『布金』這莫不是舍衛國界了麼?」八戒道:「師父,奇啊!我跟師父幾年,再不曾見識得路,今日也識得路了?」三藏說道:「不是。我常看經誦典,說是佛在舍衛城祗樹給孤園。這園說是給孤獨長者問太子買了,請佛講經。太子說:『我這園不賣,他若要買我的時,除非黃金滿布園地。』給孤獨長者聽說,隨以黃金為磚,布滿園地,才買得太子祗園,才請得世尊說法。我想這布金寺莫非就是這個故事?」八戒笑道:「造化,若是就是這個故事,我們也去摸他塊把磚兒送人。」大家又笑了一會,三藏才下得馬來。

進得山門,只見山門下挑擔的,背包的,推車的,整車坐下:也有睡的去睡,講的去講。忽見他們師徒四眾,俊的又俊,醜的又醜,大家有些害怕,卻也就讓開些路兒。三藏生怕惹事,口中不住只叫:「斯文,斯文。」這時節,卻也大家收斂。轉過金剛殿後,早有一位禪僧走出,卻也威儀不俗。真是:
\begin{quote}
面如滿月光,身似菩提樹。
擁錫袖飄風,芒鞋石頭路。
\end{quote}

三藏見了問訊。那僧即忙還禮道:「師從何來?」三藏道:「弟子陳玄奘,奉東土大唐皇帝之旨,差往西天拜佛求經。路過寶方,造次奉謁,便借一宿,明日就行。」那僧道:「荒山十方常住,都可隨喜;況長老東土神僧,但得供養,幸甚。」三藏謝了,隨即喚他三人同行。過了迴廊香積,徑入方丈。相見禮畢,分賓主坐定。行者三人,亦垂手坐了。

話說這時寺中聽說到了東土大唐取經僧人,寺中若大若小,不問長住、掛榻、長老、行童,一一都來參見。茶罷,擺上齋供。這時長老還正開齋念偈,八戒早是要緊,饅頭、素食、粉湯,一攪直下。這時方丈卻也人多,有知識的,讚說三藏威儀;好耍子的,都看八戒吃飯。卻說沙僧眼溜,看見頭底,暗把八戒捏了一把,說道:「斯文。」八戒著忙,急的叫將起來,說道:「斯文斯文,肚裡空空。」沙僧笑道:「二哥,你不曉的。天下多少斯文,若論起肚子裡來,正替你我一般哩。」八戒方才肯住。三藏念了結齋,左右徹了席面,三藏稱謝。

寺僧問起東土來因,三藏說到古跡,才問布金寺名之由。那僧答曰:「這寺原是舍衛國給孤獨園寺,又名祇園。因是給孤獨長者請佛講經,金磚布地,又易今名。我這寺一望之前,乃是舍衛國。那時給孤獨長者正在舍衛國居住,我荒山原是長者之祗園,因此遂名給孤布金寺。寺後邊還有祗園基址。近年間,若遇時雨滂沱,還淋出金銀珠兒。有造化的,每每拾著。」三藏道:「話不虛傳果是真。」又問道:「才進寶山,見門下兩廊有許多騾馬車擔的行商,為何在此歇宿?」眾僧道:「我這山喚做百腳山。先年且是太平,近因天氣循環,不知怎的,生幾個蜈蚣精,常在路下傷人;雖不至於傷命,其實人不敢走。山下有一座關,喚做雞鳴關。但到雞鳴之時,才敢過去。那些客人因到晚了,惟恐不便,權借荒山一宿,等雞鳴後便行。」三藏道:「我們也等雞鳴後去罷。」師徒們正說處,又見拿上齋來,卻與唐僧等吃畢。

此時上弦月皎。三藏與行者步月閑行,又見個道人來報道:「我們老師爺要見見中華人物。」三藏急轉身,見一個老和尚,手持竹杖,向前作禮道:「此位就是中華來的師父?」三藏答禮道:「不敢。」老僧稱讚不已,因問:「老師高壽?」三藏道:「虛度四十五年矣。敢問老院主尊壽?」老僧笑道:「比老師痴長一花甲也。」行者道:「今年是一百零五歲了。你看我有多少年紀?」老僧道:「師家貌古神清,況月夜眼花,急看不出來。」敘了一會,又向後廊看看。三藏道:「才說給孤園基址,果在何處?」老僧道:「後門外就是。——快教開門。」但見是一塊空地,還有些碎石疊的牆腳。三藏合掌嘆曰:
\begin{quote}
「憶昔檀那須達多,曾將金寶濟貧痾。
祗園千古留名在,長者何方伴覺羅?」
\end{quote}

他都玩著月,緩緩而行。行近後門外,至臺上,又坐了一坐,忽聞得有啼哭之聲。三藏靜心誠聽,哭的是爺娘不知苦痛之言。他就感觸心酸,不覺淚墮,回問眾僧道:「是甚人在何處悲切?」老僧見問,即命眾僧先回去煎茶。見無人,方才對唐僧、行者下拜。三藏攙起道:「老院主,為何行此禮?」老僧道:「弟子年歲百餘,略通人事,每於禪靜之間,也曾見過幾番景象。若老爺師徒,弟子聊知一二,與他人不同。若言悲切之事,非這位師家明辨不得。」行者道:「你且說,是甚事?」老僧道:「舊年今日,弟子正明性月之時,忽聞一陣風響,就有悲怨之聲。弟子下榻,到祗園基上看處,乃是一個美貌端正之女。我問他:『你是誰家女子?為甚到於此地?』那女子道:『我是天竺國國王的公主,因為月下觀花,被風刮來的。』我將他鎖在一間敝空房裡,將那房砌作個監房模樣,門上止留一小孔,僅遞得碗過。當日與眾僧傳道:『是個妖邪,被我綑了。』但我僧家乃慈悲之人,不肯傷他性命。每日與他兩頓粗茶粗飯,吃著度命。那女子也聰明,即解吾意。恐為眾僧點污,就裝風作怪,尿裡眠,屎裡臥。白日家說胡話,呆呆鄧鄧的;到夜靜處,卻思量父母啼哭。我幾番家進城來去打探公主之事,全然無損。故此堅收緊鎖,更不放出。今幸老師來國,萬望到了國中,廣施法力,辨明辨明:一則救拔良善,二則昭顯神通也。」三藏與行者聽罷,切切在心。

正說處,只見兩個小和尚請吃茶安置,遂而回去。八戒與沙僧在方丈中,突突噥噥的道:「明日要雞鳴走路,此時還不來睡。」行者道:「獃子又說甚麼?」八戒道:「睡了罷,這等夜深,還看甚麼景致?」因此,老僧散去,唐僧就寢。正是那:
\begin{quote}
人靜月沉花夢悄,暖風微透壁窗紗。
銅壺點點看三汲,銀漢明明照九華。
\end{quote}

當夜睡還未久,即聽雞鳴。那前邊行商烘烘皆起,引燈造飯。這長老也喚醒八戒、沙僧,扣馬收拾,行者叫點燈來。那寺僧已先起來,安排茶湯點心,在後候敬。八戒歡喜,吃了一盤饝饝,把行李、馬匹牽出。三藏、行者對眾辭謝。老僧又向行者道:「悲切之事,在心,在心。」行者笑道:「謹領,謹領。我到城中,自能聆音而察理,見貌而辨色也。」那夥行商哄哄嚷嚷的,也一同上了大路。將有寅時,過了雞鳴關。至巳時,方見城垣。真是鐵甕金城,神洲天府。那城:
\begin{quote}
虎踞龍蟠形勢高,鳳樓麟閣彩光搖。
御溝流水如環帶,福地依山插錦標。
曉日旌旗明輦路,春風簫鼓遍溪橋。
國王有道衣冠勝,五穀豐登顯俊豪。
\end{quote}

當日入於東市街,眾商各投旅店。他師徒們進城,正走處,有一個會同館驛,三藏等徑入驛內。那驛內管事的即報驛丞道:「外面有四個異樣的和尚,牽一匹白馬進來了。」驛丞聽說有馬,就知是官差的,出廳迎迓。三藏施禮道:「貧僧是東土唐朝欽差靈山大雷音見佛求經的,隨身有關文,入朝照驗。借大人高衙一歇,事畢就行。」驛丞答禮道:「此衙門原設待使客之處,理當款迓。請進,請進。」三藏喜悅,教徒弟們都來相見。那驛丞看見嘴臉醜陋,暗自心驚,不知是人是鬼,戰兢兢的,只得看茶擺齋。三藏見他驚怕,道:「大人勿驚,我等三個徒弟,相貌雖醜,心地俱良。俗謂『面惡人善』,何以懼為?」

驛丞聞言,方才定了心性,問道:「國師,唐朝在於何方?」三藏道:「在南贍部洲中華之地。」又問:「幾時離家?」三藏道:「貞觀十三年,今已歷過十四載,苦經了些萬水千山,方到此處。」驛丞道:「神僧,神僧!」三藏問道:「上國天年幾何?」驛丞道:「我敝處乃大天竺國,自太祖、太宗傳到今,已五百餘年。現在位的爺爺,愛山水花卉,號做怡宗皇帝,改元靖宴,今已二十八年了。」三藏道:「今日貧僧要去見駕倒換關文,不知可得遇朝?」驛丞道:「好,好,正好。近因國王的公主娘娘年登二十青春,正在十字街頭高結綵樓,拋打繡毬,撞天婚招駙馬。今日正當熱鬧之際,想我國王爺爺還未退朝,若欲倒換關文,趁此時好去。」三藏欣然要走,只見擺上齋來,遂與驛丞、行者等吃了。

時已過午。三藏道:「我好去了。」行者道:「我保師父去。」八戒道:「我去。」沙僧道:「二哥罷麼,你的嘴臉不見怎的,莫到朝門外裝胖。還教大哥去。」三藏道:「悟淨說得好,獃子粗夯,悟空還有些細膩。」那獃子掬著嘴道:「除了師父,我三個的嘴臉也差不多兒。」三藏卻穿了袈裟,行者拿了引袋同去。只見街坊上士農工商、文人墨客、愚夫俗子,齊咳咳都道:「看拋繡毬去也。」三藏立於道傍,對行者道:「他這裡人物衣冠、宮室器用、言語談吐,也與我大唐一般。我想著我俗家先母也是拋打繡毬,遇舊姻緣,結了夫婦。此處亦有此等風俗。」行者道:「我們也去看看,如何?」三藏道:「不可,不可。你我服色不便,恐有嫌疑。」行者道:「師父,你忘了那給孤布金寺老僧之言?一則去看彩樓,二則去辨真假。似這般忙忙的,那皇帝必聽公主之喜報,那裡視朝理事?且去去來。」三藏聽說,真與行者相隨,見各項人等俱在那裡看打繡毬。呀!那知此去卻是:
\begin{quote}
漁翁拋下鉤和線,從今釣出是非來。
\end{quote}

話表那個天竺國王,因愛山水花卉,前年帶后妃公主在御花園,月夜賞玩,惹動一個妖邪,把真公主攝去,他卻變做一個假公主。知得唐僧今年今月今日今時到此,他假借國家之富,搭起彩樓,欲招唐僧為偶,採取元陽真氣,以成太乙上仙。

正當午時三刻,三藏與行者雜入人叢,行近樓下,那公主才拈香焚起,祝告天地。左右有五七十胭嬌繡女,近侍的捧著繡毬。那樓八窗玲瓏,公主轉睛觀看,見唐僧來得至近,將繡毬取過來,親手拋在唐僧頭上。唐僧著了一驚,把個毘盧帽子打歪,雙手忙扶著那毬。那毬轂轆的滾在他衣袖之內。那樓上齊聲發喊道:「打著個和尚了,打著個和尚了。」噫!十字街頭,那些客商人等濟濟哄哄,都來奔搶繡毬。被行者喝一聲,把牙傞一傞,把腰躬一躬,長了有三丈高的個神威,弄出醜臉。諕得些人跌跌爬爬,不敢相近,霎時人散。行者還現了本像。

那樓上繡女宮娥並大小太監,都來對唐僧下拜道:「貴人,貴人,請入朝堂賀喜。」三藏急還禮,扶起眾人,回頭埋怨行者道:「你這猴頭,又是撮弄我也。」行者笑道:「繡毬兒打在你頭上,滾在你袖裡,干我何事?埋怨怎麼?」三藏道:「似此怎生區處?」行者道:「師父,你且放心,便入朝見駕,我回驛報與八戒、沙僧等候。若是公主不招你便罷,倒換了關文就行;如必欲招你,你對國王說:『召我徒弟來,我要吩咐他一聲。』那時召我三個入朝,我其間自能辨別真假。此是倚婚降怪之計。」唐僧無已從言,行者轉身回驛。

那長老被眾宮娥等撮擁至樓前。公主下樓,玉手相攙,同登寶輦,擺開儀從,回轉朝門。早有黃門官先奏道:「萬歲,公主娘娘攙著一個和尚,想是繡毬打著,現在午門外候旨。」那國王見說,心甚不喜,意欲趕退,又不知公主之意何如,只得含情宣入。公主與唐僧遂至金鑾殿下,正是:一對夫妻呼萬歲,兩門邪正拜千秋。禮畢,又宣至殿上,開言問道:「僧人何來,遇朕女拋毬得中?」唐僧俯伏奏道:「貧僧乃南贍部洲大唐皇帝差往西天大雷音寺拜佛求經的。因有長路關文,特來朝王倒換。路過十字街彩樓之下,不期公主娘娘拋繡毬,打在貧僧頭上。貧僧是出家異教之人,怎敢與玉葉金枝為偶?萬望赦貧僧死罪,倒換關文,打發早赴靈山,見佛求經,回我國土,永註陛下之天恩也。」國王道:「你乃東土聖僧,正是『千里姻緣使線牽』。寡人公主,今登二十歲未婚,因擇今日年月日時俱利,所以結綵樓拋毬,以求佳偶。可可的你來拋著,朕雖不喜,卻不知公主之意如何。」那公主叩頭道:「父王,常言『嫁雞逐雞,嫁犬逐犬』。女有誓願在先,結了這毬,告奏天地神明,撞天婚拋打。今日打著聖僧,即是前世之緣,遂得今生之遇,豈敢更移?願招他為駙馬。」國王方喜,即宣欽天監正臺官選擇日期。一壁廂收拾妝奩,又出旨曉諭天下。

三藏聞言,更不謝恩,只教:「放赦,放赦。」國王道:「這和尚甚不通理。朕以一國之富,招你做駙馬,為何不在此享用,念念只要取經?再若推辭,教錦衣官校推出斬了。」長老諕得魂不附體,只得戰兢兢叩頭啟奏道:「感蒙陛下天恩。但貧僧一行四眾,還有三個徒弟在外,今當領納,只是不曾吩咐得一言。萬望召他到此,倒換關文,教他早去,不誤了西來之意。」國王遂准奏道:「你徒弟在何處?」三藏道:「都在會同館驛。」隨即差官召聖僧徒弟領關文西去,留聖僧在此為駙馬。長老只得起身侍立。有詩為證:
\begin{quote}
大丹不漏要三全,苦行難成恨惡緣。
道在聖傳修在己,善由人積福由天。
休逞六根多貪欲,頓開一性本來原。
無愛無思自清淨,管教解脫得超然。
\end{quote}

當時差官至會同館驛,宣召唐僧徒弟不題。

卻說行者自彩樓下別了唐僧,走兩步,笑兩聲,喜喜歡歡的回驛。八戒、沙僧迎著道:「哥哥,你怎麼那般喜笑?師父如何不見?」行者道:「師父喜了。」八戒道:「還未到地頭,又不曾見佛取得經回,是何來之喜?」行者笑道:「我與師父只走至十字街彩樓之下,可可的被當朝公主拋繡毬打中了師父,師父被些宮娥、綵女、太監推擁至樓前,同公主坐輦入朝,招為駙馬,此非喜而何?」八戒聽說,跌腳搥胸道:「早知我去好來,都是那沙僧憊𪬯。你不阻我啊,我徑奔彩樓之下,一繡毬打著我老豬,那公主招了我,卻不美哉妙哉?俊刮標致,停當,大家造化耍子兒,何等有趣。」沙僧上前,把他臉上一抹道:「不羞,不羞,好個嘴巴骨子。三錢銀子買個老驢——自誇騎得。要是一繡毬打著你,就連夜燒退送紙也還道遲了,敢惹你這晦氣進門?」八戒道:「你這黑子不知趣。醜自醜,還有些風味。自古道:『皮肉粗糙,骨格堅強,各有一得可取。』」行者道:「獃子莫胡談,且收拾行李。但恐師父著了急,來叫我們,卻好進朝保護他。」八戒道:「哥哥又說差了。師父做了駙馬,到宮中與皇帝的女兒交歡,又不是爬山踵路,遇怪逢魔,要你保護他怎的?他那樣一把子年紀,豈不知被窩裡之事,要你去扶揝?」行者一把揪住耳朵,掄拳罵道:「你這個淫心不斷的夯貨!說那甚胡話?」

正吵鬧間,只見驛丞來報道:「聖上有旨,差官來請三位神僧。」八戒道:「端的請我們為何?」驛丞道:「老神僧幸遇公主娘娘打中繡毬,招為駙馬,故此差官來請。」行者道:「差官在那裡?教他進來。」那官看行者施禮,禮畢,不敢仰視,只管暗暗說道:「是鬼,是怪?是雷公,夜叉?」行者道:「那官兒,有話不說,為何沉吟?」那官兒慌得戰戰兢兢的雙手舉著聖旨,口裡亂道:「我公主有請會親,我主公會親有請。」八戒道:「我這裡沒刑具,不打你,你慢慢說,不要怕。」行者道:「莫成道怕你打?怕你那臉嘴。快收拾挑擔,牽馬進朝見師父,議事去也。」這正是:
\begin{quote}
路逢狹道難迴避,定教恩愛反為仇。
\end{quote}

畢竟不知見了國王有何話說,且聽下回分解。
