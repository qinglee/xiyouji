
\chapter{鎮海寺心猿知怪 黑松林三眾尋師}

話表三藏師徒到鎮海禪林寺,眾僧相見,安排齋供。四眾食畢,那女子也得些食力。漸漸天昏,方丈裡點起燈來。眾僧一則是問唐僧取經來歷,二則是貪看那女子,都攢攢簇簇,排列燈下。三藏對那初見的喇嘛僧道:「院主,明日離了寶山,西去的路途如何?」那僧雙膝跪下。慌得長老一把扯住道:「院主請起。我問你個路程,你為何行禮?」那僧道:「老師父明日西行,路途平正,不須費心。只是眼下有件事兒不尷尬,一進門就要說,恐怕冒犯洪威。卻才齋罷,方敢大膽奉告:老師東來,路遙辛苦,都在小和尚房中安歇甚好;只是這位女菩薩,不方便,不知請他那裡睡好。」三藏道:「院主,你不要生疑,說我師徒們有甚邪意。早間打黑松林過,撞見這個女子綁在樹上。小徒孫悟空不肯救他,是我發菩提心,將他救了。到此,隨院主送他那裡睡去。」那僧謝道:「既老師寬厚,請他到天王殿裡,就在天王爺爺身後,安排個草鋪,教他睡罷。」三藏道:「甚好,甚好。」遂此時眾小和尚引那女子往殿後睡去。長老就在方丈中,請眾院主自在,遂各散去。三藏吩咐悟空:「辛苦了,早睡早起。」遂一處都睡了,不敢離側,護著師父。漸入夜深,正是那:
\begin{quote}
玉兔高升萬籟寧,天街寂靜斷人行。
銀河耿耿星光燦,鼓發譙樓趲換更。
\end{quote}

一宵晚話不題。及天明了,行者起來,教八戒、沙僧收拾行囊、馬匹,卻請師父走路。此時長老還貪睡未醒。行者近前叫聲:「師父。」那師父把頭擡了一擡,又不曾答應得出。行者問:「師父怎麼說?」長老呻吟道:「我怎麼這般頭懸眼脹,渾身皮骨皆疼?」八戒聽說,伸手去摸摸身上,有些發熱。獃子笑道:「我曉得了,這是昨晚見沒錢的飯,多吃了幾碗,倒沁著頭睡,傷食了。」行者喝道:「胡說!等我問師父,端的何如。」三藏道:「我半夜之間起來解手,不曾戴得帽子,想是風吹了。」行者道:「這還說得是。如今可走得路麼?」三藏道:「我如今起坐不得,怎麼上馬?但只誤了路啊。」行者道:「師父說那裡話。常言道:『一日為師,終身為父。』我等與你做徒弟,就是兒子一般。又說道:『養兒不用阿金溺銀,只是見景生情便好。』你既身子不快,說甚麼誤了行程?便寧耐幾日何妨?」兄弟們都伏侍著師父,不覺的早盡午來昏又至,良宵才過又侵晨。

光陰迅速,早過了三日。那一日,師父欠身起來叫道:「悟空,這兩日病體沉痾,不曾問得你:那個脫命的女菩薩,可曾有人送些飯與他吃?」行者笑道:「你管他怎的?且顧了自家的病著。」三藏道:「正是,正是。你且扶我起來,取出我的紙、筆、墨,寺裡借個硯臺來使使。」行者道:「要怎的?」長老道:「我要修一封書,並關文封在一處,你替我送上長安駕下,見太宗皇帝一面。」行者道:「這個容易,我老孫別事無能,若說送書,人間第一。你把書收拾停當取與我,我一觔斗送到長安,遞與唐王,再一觔斗轉將回來,你的筆硯還不乾哩。但只是你寄書怎的?且把書意念念我聽,念了再寫不遲。」長老滴淚道:「我寫著:
\begin{quote}
臣僧稽首三頓首,萬歲山呼拜聖君;
文武兩班同入目,公卿四百共知聞:
當年奉旨離東土,指望靈山見世尊。
不料途中遭厄難,何期半路有災迍。
僧病沉痾難進步,佛門深遠接天門。
有經無命空勞碌,啟奏當今別遣人。」
\end{quote}

行者聽得此言,忍不住呵呵大笑道:「師父,你忒不濟,略有些些病兒,就起這個意念。你若是病重,要死要活,只消問我,我老孫自有個本事:問道那個閻王敢起心?那個判官敢出票?那個鬼使來勾取?若惱了我,我拿出那大鬧天宮之性子,又一路棍,打入幽冥,捉住十代閻王,一個個抽了他的筋,還不饒他哩。」三藏道:「徒弟呀,我病重了,切莫說這大話。」

八戒上前道:「師兄,師父說不好,你只管說好,十分不尷尬。我們趁早商量,先賣了馬,典了行囊,買棺木送終散火。」行者道:「獃子又胡說了,你不知道。師父是我佛如來第二個徒弟,原叫做金蟬長老,只因他輕慢佛法,該有這場大難。」八戒道:「哥啊,師父既是輕慢佛法,貶回東土,在是非海內,口舌場中,託化做人身,發願往西天拜佛求經,遇妖精就捆,逢魔頭就吊,受諸苦惱,也夠了,怎麼又叫他害病?」行者道:「你那裡曉得。老師父不曾聽佛講法,打了一個盹,往下一試,左腳屣了一粒米,下界來,該有這三日病。」八戒驚道:「像老豬吃東西潑潑撒撒的,也不知害多少年代病是。」行者道:「兄弟,佛不與你眾生為念,你又不知。人云:『鋤禾日當午,汗滴禾下土。誰知盤中餐,粒粒皆辛苦。』師父只今日一日,明日就好了。」三藏道:「我今日比昨日不同:咽喉裡十分作渴。你去那裡有涼水,尋些來我吃。」行者道:「好了,師父要水吃,便是好了。等我取水去。」

即時取了缽盂,往寺後面香積廚取水。忽見那些和尚一個個眼兒通紅,悲啼哽咽,只是不敢放聲大哭。行者道:「你們這些和尚忒小家子樣。我們住幾日,臨行謝你,柴火錢照日算還,怎麼這等膿包?」眾僧慌跪下道:「不敢,不敢。」行者道:「怎麼不敢?想是我那長嘴和尚食腸大,吃傷了你的本兒也?」眾僧道:「老爺,我這荒山,大大小小也有百十眾和尚,每一人養老爺一日,也養得起百十日。怎麼敢欺心,計較甚麼食用?」行者道:「既不計較,你卻為甚麼啼哭?」眾僧道:「老爺,不知是那山裡來的妖邪在這寺裡。我們晚夜間著兩個小和尚去撞鐘打鼓,只聽得鐘鼓響罷,再不見人回。至次日找尋,只見僧帽、僧鞋丟在後邊園裡,骸骨尚存,將人吃了。你們住了三日,我寺裡不見了六個和尚。故此,我兄弟們不由的不怕,不由的不傷。因見你老師父貴恙,不敢傳說,忍不住淚珠偷垂也。」行者聞言,又驚又喜道:「不消說了,必定是妖魔在此傷人也。等我與你剿除他。」眾僧道:「老爺,妖精不精者不靈。一定會騰雲駕霧,一定會出幽入冥。古人道得好:『莫信直中直,須防人不仁。』老爺,你莫怪我們說:你若拿得他住哩,便與我荒山除這條禍根,正是三生有幸了;若還拿他不住啊,卻有好些兒不便處。」行者道:「怎叫做好些不便處?」那眾僧道:「直不相瞞老爺說,我這荒山雖有百十眾和尚,卻都只是自小兒出家的。髮長尋刀削,衣單破衲縫。早晨起來洗著臉,叉手躬身皈依大道;夜來收拾燒著香,虔心叩齒念的彌陀。舉頭看見佛蓮九品,三乘慈航共法雲,願見祇園釋世尊低頭看見心,受五戒,度大千,生生萬法中,願悟頑空與色空。諸檀越來啊,老的小的、長的矮的、胖的瘦的,一個個敲木魚,擊金磬,挨挨拶拶,兩卷《法華經》,一策《梁王懺》;諸檀越不來啊,新的舊的、生的熟的、村的俏的,一個個合著掌,瞑著目,悄悄冥冥,入定蒲團上,牢關月下門。一任他鶯啼鳥語閑爭鬥,不上我方便慈悲大法乘。因此上,也不會伏虎,也不會降龍;也不識的怪,也不識的精。你老爺若還惹起那妖魔啊,我百十個和尚只夠他齋一飽。一則墮落我眾生輪迴;二則滅抹了這禪林古跡;三則如來會上全沒半點兒光輝。這卻是好些兒不便處。」

行者聞得眾和尚說出這一端的話語,他便怒從心上起,惡向膽邊生,高叫一聲:「你這眾和尚,好獃哩!只曉得那妖精,就不曉得我老孫的行止麼?」眾僧輕輕的答道:「實不曉得。」行者道:「我今日略節說說,你們聽著。我也曾花果山伏虎降龍,我也曾上天堂大鬧天宮。饑時把老君的丹,略略咬了兩三顆;渴時把玉帝的酒,輕輕呼了六七鍾。睜著一雙不白不黑的金睛眼,天慘淡,月朦朧;拿著一條不短不長的金箍棒,來無影,去無蹤。說甚麼大精小怪,那怕他憊𪬯膿。一趕趕上去,跑的跑,顫的顫,躲的躲,慌的慌;一捉捉將來,銼的銼,燒的燒,磨的磨,舂的舂。正是八仙同過海,獨自顯神通。眾和尚,我拿這妖精與你看看,你才認得我老孫。」

眾僧聽著,暗點頭道:「這賊禿開大口,說大話,想是有些來歷。」都一個個諾諾連聲。只有那喇嘛僧道:「且住。你老師父貴恙,你拿這妖精不至緊。俗語道:『公了登筵,不醉便飽;壯士臨陣,不死即傷。』你兩下裡角鬥之時,倘貽累你師父,不當穩便。」

行者道:「有理,有理。我且送涼水與師父吃了再來。」掇起缽盂,著上涼水,轉出香積廚,就到方丈,叫聲:「師父,吃涼水哩。」三藏正當煩渴之時,便擡起頭來,捧著水,只是一吸。真個:渴時一滴如甘露,藥到真方病即除。行者見長老精神漸爽,眉目舒開,就問道:「師父,可吃些湯飯麼?」三藏道:「這涼水就是靈丹一般,這病兒減了一半,有湯飯也吃得些。」行者連聲高高叫道:「我師父好了,要湯飯吃哩。」教那些和尚忙忙的安排:淘米煮飯、捍麵烙餅,蒸饝饝、做粉湯,擡了四五桌。唐僧只吃得半碗兒米湯,行者、沙僧止用了一席,其餘的都是八戒一肚餐之。家火收去,點起燈來,眾僧各散。

三藏道:「我們今住幾日了?」行者道:「三整日矣。明朝向晚,便就是四個日頭。」三藏道:「三日誤了許多路程。」行者道:「師父,也算不得路程,明日去罷。」三藏道:「正是,就帶幾分病兒,也沒奈何。」行者道:「既是明日要去,且讓我今晚捉了妖精看。」三藏驚道:「又捉甚麼妖精?」行者道:「有個妖精在這寺裡,等老孫替他捉捉。」唐僧道:「徒弟呀,我的病身未可,你怎麼又興此念?倘那怪有神通,你拿他不住啊,卻又不是害我?」行者道:「你好滅人威風。老孫到處降妖,你見我弱與誰的?只是不動手,動手就要贏。」三藏扯住道:「徒弟,常言說得好:『遇方便時行方便,得饒人處且饒人。』『操心怎似存心好,爭氣何如忍氣高?』」孫大聖見師父苦苦勸他,不許降妖,他說出老實話來道:「師父,實不瞞你說,那妖在此吃了人了。」唐僧大驚道:「吃了甚麼人?」行者說道:「我們住了三日,已是吃了這寺裡六個小和尚了。」長老道:「『兔死狐悲,物傷其類。』他既吃了寺內之僧,我亦僧也,我放你去,只但用心仔細些。」行者道:「不消說,老孫的手到就消除了。」

你看他燈光前吩咐八戒、沙僧看守師父,他喜孜孜跳出方丈,徑來佛殿看時,天上有星,月還未上,那殿裡黑暗暗的。他就吹出真火,點起琉璃,東邊打鼓,西邊撞鐘。響罷,搖身一變,變做個小和尚兒,年紀只有十二三歲,披著黃絹褊衫,白布直裰,手敲著木魚,口裡念經。等到一更時分,不見動靜。二更時分,殘月才升,只聽見呼呼的一陣風響。好風:
\begin{quote}
黑霧遮天暗,愁雲照地昏。四方如潑墨,一派靛妝渾。先刮時揚塵播土,次後來倒樹摧林。揚塵播土星光現,倒樹摧林月色昏。只刮得:嫦娥緊抱梭羅樹,玉兔團團找藥盆;九曜星官皆閉戶,四海龍王盡掩門;廟裡城隍覓小鬼,空中仙子怎騰雲;地府閻羅尋馬面,判官亂跑趕頭巾。刮動崑崙頂上石,捲得江湖波浪混。
\end{quote}

那風才然過處,猛聞得蘭麝香熏,環珮聲響。即欠身擡頭觀看,呀!卻是一個美貌佳人,徑上佛殿。行者口裡嗚哩嗚喇,只情念經。那女子走近前,一把摟住道:「小長老,念的甚麼經?」行者道:「許下的。」女子道:「別人都自在睡覺,你還念經怎麼?」行者道:「許下的,如何不念?」女子摟住,與他親個嘴道:「我與你到後面耍耍去。」行者故意的扭過頭去道:「你有些不曉事。」女子道:「你會相面?」行者道:「也曉得些兒。」女子道:「你相我怎的樣子?」行者道:「我相你有些兒偷生㧚熟,被公婆趕出來的。」女子道:「相不著,相不著。我
\begin{quote}
不是公婆趕逐,不因㧚熟偷生。
奈我前生命薄,投配男子年輕。
不會洞房花燭,避夫逃走之情。
\end{quote}

趁如今星光月皎,也是有緣千里來相會。我和你到後園中交歡配鸞儔去也。」行者聞言,暗點頭道:「那幾個愚僧,都被色慾引誘,所以傷了性命。他如今也來哄我。」就隨口答應道:「娘子,我出家人年紀尚幼,卻不知甚麼交歡之事。」女子道:「你跟我去,我教你。」行者暗笑道:「也罷,我跟他去,看他怎生擺佈。」

他兩個摟著肩,攜著手,出了佛殿,徑至後邊園裡。那怪把行者使個絆子腿,跌倒在地。口裡「心肝哥哥」的亂叫,將手就去掐他的臊根。行者道:「我的兒,真個要吃老孫哩。」卻被行者接住他手,使個小坐跌法,把那怪一轆轤掀翻在地上。那怪口裡還叫道:「心肝哥哥,你倒會跌你的娘哩。」行者暗算道:「不趁此時下手他,還到幾時?正是:『先下手為強,後下手遭殃。』」就把手一叉,腰一躬,一跳跳起來,現出原身法像,掄起金箍鐵棒,劈頭就打。那怪倒也吃了一驚。他心想道:「這個小和尚,這等利害。」打開眼一看,原來是那唐長老的徒弟姓孫的。他也不懼他。你說這精怪是甚麼精怪:
\begin{quote}
金作鼻,雪鋪毛。地道為門屋,安身處處牢。養成三百年前氣,曾向靈山走幾遭。一飽香花和蠟燭,如來吩咐下天曹。托塔天王恩愛女,哪吒太子認同胞。也不是個填海鳥,也不是個戴山鰲。也不怕的雷煥劍,也不怕的呂虔刀。往往來來,一任他水流江漢闊;上上下下,那論他山聳泰恆高。你看他月貌花容嬌滴滴,誰識得是個鼠老成精逞點豪。
\end{quote}

他自恃的神通廣大,便隨手架起雙股劍,玎玎璫璫的響,左遮右格,隨東倒西。行者雖強些,卻也撈他不倒。陰風四起,殘月無光。你看他兩人。後園中一場好殺:
\begin{quote}
陰風從地起,殘月蕩微光。闃靜梵王宇,闌珊小鬼廊,後園裡一片戰爭場。孫大士,天上聖;毛姹女,女中王:賭賽神通未肯降。一個兒扭轉芳心嗔黑禿,一個兒圓睜慧眼恨新妝。兩手劍飛,那認得女菩薩;一根棍打,狠似個活金剛。響處金箍如電掣,霎時鐵白耀星芒。玉樓抓翡翠,金殿碎鴛鴦。猿啼巴月小,雁叫楚天長。十八尊羅漢暗暗喝采,三十二諸天個個慌張。
\end{quote}

那孫大聖精神抖擻,棍兒沒半點差池。妖精自料敵他不住,猛可的眉頭一蹙,計上心來,抽身便走。行者喝道:「潑貨,那走?快快來降。」那妖精只是不理,直往後退。等行者趕到緊急之時,即將左腳上花鞋脫下來,吹口仙氣,念個咒語,叫一聲:「變!」就變做本身模樣,使兩口劍舞將來;真身一幌,化陣清風而去。這卻不是三藏的災星?他便徑撞到方丈裡,把唐三藏攝將去雲頭上,杳杳冥冥,霎霎眼,就到了陷空山,進了無底洞,叫小的們安排素筵席成親不題。

卻說行者鬥得心焦性燥,閃一個空,一棍把那妖精打落下來,乃是一隻花鞋。行者曉得中了他計,連忙轉身來看師父,那有個師父。只見那獃子和沙僧口裡嗚哩嗚哪說甚麼。行者怒氣填胸,也不管好歹,撈起棍來一片打,連聲叫道:「打死你們,打死你們。」那獃子慌得走也沒路。沙僧卻是個靈山大將,見得事多,就軟款溫柔,近前跪下道:「兄長,我知道了,想你要打殺我兩個,也不去救師父,徑自回家去哩。」行者道:「我打殺你兩個,我自去救他。」沙僧笑道:「兄長說那裡話?無我兩個,真是『單絲不線,孤掌難鳴』。兄啊,這行囊、馬匹,誰與看顧?寧學管鮑分金,休仿孫龐鬥智。自古道:『打虎還得親兄弟,上陣須教父子兵。』望兄長且饒打,待天明和你同心戮力,尋師去也。」行者雖是神通廣大,卻也明理識時。見沙僧苦苦哀告,便就回心道:「八戒、沙僧,你都起來。明日找尋師父。卻要用力。」那獃子聽見饒了,恨不得天也許下半邊,道:「哥啊,這個都在老豬身上。」兄弟們思思想想,那曾得睡,恨不得點頭喚出扶桑日,一口吹散滿天星。

三眾只坐到天曉,收拾要行,早有寺僧攔門來問:「老爺那裡去?」行者笑道:「不好說。昨日對眾誇口,說與他們拿妖精,妖精未曾拿得,倒把我個師父不見了。我們尋師父去哩。」眾僧害怕道:「老爺,小可的事,倒帶累老師;卻往那裡去尋?」行者道:「有處尋他。」眾僧忙道:「既去莫忙,且吃些早齋。」連忙的端了兩三盆湯飯。八戒盡力吃個乾淨,道:「好和尚,我們尋著師父,再到你這裡來耍子。」行者道:「還到這裡吃他飯哩?你去天王殿裡看看那女子在否。」眾僧道:「老爺,不在了,不在了。自是當晚宿了一夜,第二日就不見了。」

行者喜喜歡歡的辭了眾僧,著八戒、沙僧牽馬挑擔,徑回東走。八戒道:「哥哥差了,怎麼又往東行?」行者道:「你豈知道?前日在那黑松林綁的那個女子,老孫火眼金睛,把他認透了,你們都認做好人。今日吃和尚的也是他,攝師父的也是他。你們救得好女菩薩。今既攝了師父,還從舊路上找尋去也。」二人嘆服道:「好好好,真是粗中有細。去來,去來。」

三人急急到於林內,只見那:
\begin{quote}
雲藹藹,霧漫漫;石層層,路盤盤。狐蹤兔跡交加走,虎豹豺狼往復鑽。林內更無妖怪影,不知三藏在何端。
\end{quote}

行者心焦,掣出棒來,搖身一變,變作大鬧天宮的本相:三頭六臂,六隻手,理著三根棒,在林裡辟哩撥喇的亂打。八戒見了道:「沙僧,師兄著了惱,尋不著師父,弄做個氣心風了。」原來行者打了一路,打出兩個老頭兒來:一個是山神,一個是土地,上前跪下道:「大聖,山神、土地來見。」八戒道:「好靈根啊,打了一路,打出兩個山神、土地;若再打一路,連太歲都打出來也。」行者問道:「山神、土地,汝等這般無禮,在此處專一結夥強盜。強盜得了手,買些豬羊祭賽你,又與妖精結擄,打夥兒把我師父攝來。如今藏在何處?快快的從實供來,免打。」二神慌了道:「大聖錯怪了我耶。妖精不在小神山上,不伏小神管轄。但只夜間風響處,小神略知一二。」行者道:「既知,一一說來。」土地道:「那妖精攝你師父去,在那正南下,離此有千里之遙。那廂有座山,喚做陷空山。山中有個洞,叫做無底洞。是那山裡妖精,到此變化攝去也。」行者聽言,暗自驚心。喝退了山神、土地,收了法身,現出本相,與八戒、沙僧道:「師父去得遠了。」八戒道:「遠便騰雲趕去。」

好獃子,一縱狂風先起;隨後是沙僧駕雲;那白馬原是龍子出身,馱了行李,也踏了風霧;大聖即起觔斗:一直南來。不多時,早見一座大山,阻住雲腳。三人採住馬,都按定雲頭。見那山:
\begin{quote}
頂摩碧漢,峰接青霄。周圍雜樹萬萬千,來往飛禽喳喳噪。虎豹成陣走,獐鹿打叢行。向陽處,琪花瑤草馨香;背陰方,臘雪頑冰不化。崎嶇峻嶺,削壁懸崖。直立高峰,灣環深澗。松鬱鬱,石磷磷,行人見了悚其心。打柴樵子全無影,採藥仙童不見蹤。眼前虎豹能興霧,遍地狐狸亂弄風。
\end{quote}

八戒道:「哥啊,這山如此嶮峻,必有妖邪。」行者道:「不消說了,山高原有怪,嶺峻豈無精?」叫:「沙僧,我和你且在此,著八戒先下山凹裡打聽打聽,看那條路好走,端的可有洞府,再看是那裡開門,俱細細打探,我們好一齊去尋師父救他。」八戒道:「老豬晦氣,先拿我頂缸。」行者道:「你夜來說都在你身上,如何打仰?」八戒道:「不要嚷,等我去。」獃子放下鈀,抖抖衣裳,空著手,跳下高山,找尋路徑。

這一去,畢竟不知好歹如何,且聽下回分解。
