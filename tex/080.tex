
\chapter{姹女育陽求配偶 心猿護主識妖邪}

卻說比丘國君臣黎庶送唐僧四眾出城,有二十里之遠,還不肯捨。三藏勉強下輦,乘馬辭別而行,目送者直至望不見蹤影方回。四眾行夠多時,又過了冬殘春盡,看不了野花山樹,景物芳菲。前面又見一座高山峻嶺。三藏心驚,問道:「徒弟,前面高山有路無路?是必小心。」行者笑道:「師父這話,也不像走長路的,卻似個公子王孫,坐井觀天之類。自古道:『山不礙路,路自通山。』何以言有路無路?」三藏道:「雖然是山不礙路,但恐嶮峻之間生怪物,密叢深處出妖精。」八戒道:「放心,放心。這裡來相近極樂不遠,管取太平無事。」

師徒正說,不覺的到了山腳下。行者取出金箍棒,走上石崖,叫道:「師父,此間乃轉山的路兒,忒好走。快來,快來。」長老只得放懷策馬。沙僧教:「二哥,你把擔子挑一肩兒。」真個八戒接了擔子挑上,沙僧攏著韁繩,老師父穩坐雕鞍,隨行者都奔山崖上大路。但見那山:
\begin{quote}
雲霧籠峰頂,潺湲湧澗中。百花香滿路,萬樹密叢叢。梅青李白,柳綠桃紅。杜鵑啼處春將暮,紫燕呢喃社已終。嵯峨石,翠蓋松。崎嶇嶺道,突兀玲瓏。削壁懸崖峻,薜蘿草木穠。千巖競秀如排戟,萬壑爭流遠浪洪。
\end{quote}

老師父緩觀山景,忽聞啼鳥之聲,又起思鄉之念,兜馬叫道:「徒弟!
\begin{quote}
我自天牌傳旨意,錦屏風下領關文。
觀燈十五離東土,才與唐王天地分。
甫能龍虎風雲會,卻又師徒拗馬軍。
行盡巫山峰十二,何時對子見當今?」
\end{quote}

行者道:「師父,你常以思鄉為念,全不似個出家人。放心且走,莫要多憂。古人云:『欲求生富貴,須下死工夫。』」三藏道:「徒弟雖然說得有理,但不知西天路還在那裡哩。」八戒道:「師父,我佛如來捨不得那三藏經,知我們要取去,想是搬了;不然,如何只管不到?」沙僧道:「莫胡談,只管跟著大哥走。只把工夫捱他,終須有個到之之日。」

師徒正自閑敘,又見一派黑松大林。唐僧害怕,又叫道:「悟空,我們才過了那崎嶇山路,怎麼又遇這個深黑松林?是必在意。」行者道:「怕他怎的?」三藏道:「說那裡話?『不信直中直,須防仁不仁。』我也與你走過好幾處松林,不似這林深遠?」你看:
\begin{quote}
東西密擺,南北成行。東西密擺徹雲霄,南北成行侵碧漢。密查荊棘週圍結,蓼卻纏枝上下盤。藤來纏葛,葛去纏藤。藤來纏葛,東西客旅難行;葛去纏藤,南北經商怎進。這林中住半年,那分日月;行數里,不見斗星。你看那背陰之處千般景,向陽之所萬叢花。又有那千年槐,萬載檜,耐寒松,山桃果,野芍藥,旱芙蓉,一攢攢密砌重堆,亂紛紛神仙難畫。又聽得百鳥聲:鸚鵡哨,杜鵑啼;喜鵲穿枝,烏鴉反哺;黃鸝飛舞,百舌調音;鷓鴣鳴,紫燕語;八哥兒學人說話,畫眉郎也會看經。又見那大蟲擺尾,老虎磕牙;多年狐狢妝娘子,日久蒼狼吼振林。就是托塔天王來到此,縱會降妖也失魂。」
\end{quote}

孫大聖公然不懼,使鐵棒上前劈開大路,引唐僧徑入深林。逍逍遙遙,行經半日,未見出林之路。唐僧叫道:「徒弟,一向西來,無數的山林崎嶮,幸得此間清雅,一路太平。這林中奇花異卉,其實可人情意。我要在此坐坐:一則歇馬;二則腹中饑了,你去那裡化些齋來我吃。」行者道:「師父請下馬,老孫化齋去來。」那長老果然下了馬,八戒將馬拴在樹上。沙僧歇下行李,取了缽盂,遞與行者。行者道:「師父穩坐,莫要驚怕,我去了就來。」三藏端坐松陰之下,八戒、沙僧卻去尋花覓果閑耍。

卻說大聖縱觔斗,到了半空,佇定雲光,回頭觀看,只見松林中祥雲縹緲,瑞靄氤氳。他忽失聲叫道:「好啊!好啊!」你道他叫好做甚?原來誇獎唐僧,說他是金蟬長老轉世,十世修行的好人,所以有此祥瑞罩頭。「若我老孫,方五百年前大鬧天宮之時,雲遊海角,放蕩天涯;聚群精,自稱齊天大聖;降龍伏虎,消了死籍。頭戴著三額金冠,身穿著黃金鎧甲,手執著金箍棒,足踏著步雲履。手下有四萬七千群怪,都稱我做大聖爺爺,著實為人。如今脫卻天災,做小伏低,與你做了徒弟。想師父頭頂上有祥雲瑞靄罩定,徑回東土,必定有些好處,老孫也必定得個正果。」

正自家這等誇念中間,忽然見林南下有一股子黑氣,骨都都的冒將上來。行者大驚道:「那黑氣裡必定有邪了。我那八戒、沙僧卻不會放甚黑氣。」那大聖在半空中詳察不定。

卻說三藏坐在林中,明心見性,諷念那《摩訶般若波羅密多心經》,忽聽得嚶嚶的叫聲「救人」。三藏大驚道:「善哉,善哉!這等深林裡,有甚麼人叫?想是狼蟲虎豹諕倒的,待我看看。」那長老起身挪步,穿過千年柏,隔起萬年松,附葛攀藤,近前觀之。只見那大樹上綁著一個女子,上半截使葛滕綁在樹上,下半截埋在土裡。長老立定腳,問他一句道:「女菩薩,你有甚事,綁在此間?」咦!分明這廝是個妖怪,長老肉眼凡胎,卻不能認得。那怪見他來問,淚如泉湧。你看他桃腮垂淚,有沉魚落雁之容;星眼含悲,有閉月羞花之貌。長老實不敢近前,又開口問道:「女菩薩,你端的有何罪過?說與貧僧,卻好救你。」那妖精巧語花言,虛情假意,忙忙的答應道:「師父,我家住在貧婆國,離此有二百餘里。父母在堂,十分好善,一生的和親愛友。時遇清明,邀請諸親及本家老小拜掃先塋,一行轎馬,都到了荒郊野外。至塋前,擺開祭祀,剛燒化紙馬,只聞得鑼鳴鼓響,跑出一夥強人,持刀弄杖,喊殺前來,慌得我們魂飛魄散。父母諸親得馬得轎的各自逃了性命;奴奴年幼。跑不動,諕倒在地,被眾強人拐來山內,大大王要做夫人,二大王要做妻室,第三第四個都愛我美色,七八十家一齊爭吵,大家都不忿氣,所以把奴奴綁在林間,眾強人散盤而去。今已五日五夜,看看命盡,不久身亡。不知是那世裡祖宗積德,今日遇著老師父到此。千萬發大慈悲,救我一命,九泉之下,決不忘恩。」說罷,淚下如雨。

三藏真個慈心,也就忍不住吊下淚來,聲音哽咽,叫道:「徒弟。」那八戒、沙僧正在林中尋花覓果,猛聽得師父叫得悽愴,獃子道:「沙和尚,師父在此認了親耶。」沙僧笑道:「二哥胡纏,我們走了這些時,好人也不曾撞見一個,親從何來?」八戒道:「不是親,師父那裡與人哭麼?我和你去看來。」沙僧真個回轉舊處,牽了馬,挑了擔,至跟前叫:「師父,怎麼說?」唐僧用手指定那樹上,叫:「八戒,解下那女菩薩來,救他一命。」獃子不分好歹,就去動手。

卻說那大聖在半空中,又見那黑氣濃厚,把祥光盡情蓋了,道聲:「不好,不好!黑氣罩暗祥光,怕不是妖邪害俺師父?化齋還是小事,且去看我師父去。」卻返雲頭,按落林裡,只見八戒亂解繩兒。行者上前一把揪住耳朵,撲的捽了一跌。獃子擡頭看見,爬起來說道:「師父教我救人,你怎麼恃你有力,將我摜這一跌?」行者笑道:「兄弟,莫解他。他是個妖精,弄喧兒,騙我們哩。」三藏喝道:「你這潑猴,又來胡說了,怎麼這等一個女子,就認得他是個妖怪?」行者道:「師父原來不知,這都是老孫幹過的買賣,想人肉吃的法兒,你那裡認得?」八戒嗊著嘴道:「師父,莫信這弼馬溫哄你,這女子乃是此間人家。我們東土遠來,不與相較,又不是親眷,如何說他是妖精?他打發我們丟了前去,他卻翻觔斗,弄神法轉來和他幹巧事兒,倒踏門也。」行者喝道:「夯貨,莫亂談。我老孫一向西來,那裡有甚憊𪬯處?似你這個重色輕生、見利忘義的囔糟,不識好歹,替人家哄了招女婿,綁在樹上哩。」三藏道:「也罷,也罷。八戒啊,你師兄常時也看得不差,既這等說,不要管他,我們去罷。」行者大喜道:「好了,師父是有命的了。請上馬,出松林外,有人家化齋你吃。」四人果一路前進,把那怪撇了。

卻說那怪綁在樹上,咬牙恨齒道:「幾年家聞人說孫悟空神通廣大,今日見他,果然話不虛傳。那唐僧乃童身修行,一點元陽未泄,正欲拿他去配合,成太乙金仙,不知被此猴識破吾法,將他救去了。若是解了繩,放我下來,隨手捉將去,卻不是我的人兒也?今被他一篇散言碎語帶去,卻又不是勞而無功?等我再叫他兩聲,看是如何。」妖精不動繩索,把幾聲善言善語,用一陣順風,嚶嚶的吹在唐僧耳內。你道叫的甚麼?他叫道:「師父啊,你放著活人的性命還不救,昧心拜佛取何經?」

唐僧在馬上聽得又這般叫喚,即勒馬叫:「悟空,去救那女子下來罷。」行者道:「師父走路,怎麼又想起他來了?」唐僧道:「他又在那裡叫哩。」行者問:「八戒,你聽見麼?」八戒道:「耳大遮住了,不曾聽見。」又問:「沙僧,你聽見麼?」沙僧道:「我挑擔前走,不曾在心,也不曾聽見。」行者道:「老孫也不曾聽見。師父,他叫甚麼?偏你聽見?」唐僧道:「他叫得有理。說道:『活人性命還不救,昧心拜佛取何經?』『救人一命,勝造七級浮屠。』快去救他下來,強似取經拜佛。」行者笑道:「師父要善將起來,就沒藥醫。你想你離了東土,一路西來,卻也過了幾重山場,遇著許多妖怪,常把你拿將進洞。老孫來救你,使鐵棒,常打死千千萬萬。今日一個妖精的性命,捨不得,要去救他?」唐僧道:「徒弟呀,古人云:『勿以善小而不為,勿以惡小而為之。』還去救他救罷。」行者道:「師父既然如此,只是這個擔兒,老孫卻擔不起。你要救他,我也不敢苦勸:我勸一會,你又惱了。任你去救。」唐僧道:「猴頭莫多話,你坐著,等我和八戒救他去。」

唐僧回至林裡,教八戒解了上半截繩子,用鈀築出下半截身子。那怪跌跌鞋,束束裙,喜孜孜跟著唐僧出松林,見了行者。行者只是冷笑不止。唐僧罵道:「潑猴頭,你笑怎的?」行者道:「我笑你時來逢好友,運去遇佳人。」三藏又罵道:「潑猢猻胡說。我自出娘肚皮,就做和尚,如今奉旨西來,虔心禮佛求經,又不是利祿之輩,有甚運退時?」行者笑道:「師父,你雖是自幼為僧,卻只會看經念佛,不曾見王法條律。這女子生得年少標致,我和你乃出家人,同他一路行走,倘或遇著歹人,把我們拿送官司,不論甚麼取經拜拂,且都打做姦情;縱無此事,也要問個拐帶人口:師父追了度牒,打個小死;八戒該問充軍;沙僧也問擺站;我老孫也不得乾淨,饒我口能,怎麼折辯,也要問個不應。」三藏喝道:「莫胡說,終不然,我救他性命,有甚貽累不成?帶了他去,凡有事,都在我身上。」行者道:「師父雖說有事在你,卻不知你不是救他,反是害他。」三藏道:「我救他出林,得其活命,怎麼反是害他?」行者道:「他當時綁在林間,或三五日,十日半月,沒飯吃,餓死了,還得個完全身體歸陰。如今帶他出來,你坐得是個快馬,行路如風,我們只得隨你,那女子腳小,挪步艱難,怎麼跟得上走?一時把他丟下,若遇著狼蟲虎豹,一口吞之,卻不是反害其生也?」

三藏道:「正是呀,這件事卻虧你想,如何處置?」行者笑道:「抱他上來,和你同騎著馬走罷。」三藏沉吟道:「我那裡好與他同馬?」——「他怎生得去?」三藏道:「教八戒馱他走罷。」行者笑道:「獃子造化到了。」八戒道:「『遠路沒輕擔。』教我馱人,有甚造化?」行者道:「你那嘴長,馱著他,轉過嘴來,計較私情話兒,卻不便益?」八戒聞此言,搥胸爆跳道:「不好,不好。師父要打我幾下,寧可忍疼。背著他決不得乾淨,師兄一生會贓埋人。我馱,不成。」三藏道:「也罷,也罷。我也還走得幾步,等我下來,慢慢的同走,著八戒牽著空馬罷。」行者大笑道:「獃子倒有買賣,師父照顧你牽馬哩。」三藏道:「這猴頭又胡說了。古人云:『馬行千里,無人不能自往。』假如我在路上慢走,你好丟了我去?我若慢,你們也慢。大家一處同這女菩薩走下山去,或到庵觀寺院,有人家之處,留他在那裡,也是我們救他一場。」行者道:「師父說得有理,快請前進。」

三藏撩前走,沙僧挑擔,八戒牽著空馬,引著女子,行者拿鐵棒,一行前進。不上二三十里,天色將晚,又見一座樓臺殿閣。三藏道:「徒弟,那裡必定是座庵觀寺院,就此借宿了,明日早行。」行者道:「師父說得是,各各走動些。」霎時到了門首,吩咐道:「你們略站遠些,等我先去借宿,若有方便處,著人來叫你。」眾人俱立在柳蔭之下,惟行者拿鐵棒,轄著那女子。

長老拽步近前,只見那門東倒西歪,零零落落。推開看時,忍不住心中悽慘:長廊寂靜,古剎蕭疏;苔蘚盈庭,蒿蓁滿徑;惟螢火之飛燈,只蛙聲而代漏。長老忽然吊下淚來。真個是:
\begin{quote}
殿宇凋零倒塌,廊房寂寞傾頹。斷磚破瓦十餘堆,盡是些歪梁折柱。前後盡生青草,塵埋朽爛香廚。鐘樓崩壞鼓無皮,琉璃香燈破損。佛祖金身沒色,羅漢倒臥東西。觀音淋壞盡成泥,楊柳淨瓶墜地。日內並無僧人,夜間盡宿狐狸。只聽風響吼如雷,都是虎豹藏身之處。四下牆垣皆倒,亦無門扇關居。
\end{quote}

有詩為證。詩曰:
\begin{quote}
多年古剎沒人修,狼狽凋零倒更休。
猛風吹裂伽藍面,大雨澆殘佛像頭。
金剛跌損隨淋灑,土地無房夜不收。
更有兩般堪嘆處,銅鐘著地沒懸樓。
\end{quote}

三藏硬著膽,走進二層門。見那鐘鼓樓俱倒了,止有一口銅鐘,扎在地下,上半截如雪之白,下半截如靛之青。原來是日久年深,上邊被雨淋白,下邊是土氣上的銅青。三藏用手摸著鐘,高叫道:「鐘啊,你也曾懸掛高樓吼,也曾鳴遠彩梁聲。也曾雞啼就報曉,也曾天晚送黃昏。不知化銅的道人歸何處,鑄銅匠作那邊存。想他二命歸陰府,他無蹤跡你無聲。」

長老高聲讚嘆,不覺的驚動寺裡之人。那裡邊有一個侍奉香火的道人,他聽見人語,扒起來,拾一塊斷磚,照鐘上打將去,那鐘噹的響了一聲。把個長老諕了一跌,掙起身要走,又絆著樹根,撲的又是一跌。長老倒在地下,擡頭又叫道:「鐘啊,貧僧正然感嘆你,忽的叮噹響一聲。想是西天路上無人到,日久多年變作精。」

那道人趕上前,一把攙住道:「老爺請起。不干鐘成精之事,卻才是我打得鐘響。」三藏擡頭見他的模樣醜黑,道:「你莫是魍魎妖邪?我不是尋常之人,我是大唐來的,我手下有降龍伏虎的徒弟。你若撞著他,性命難存也。」道人跪下道:「老爺休怕。我不是妖邪,我是這寺裡侍奉香火的道人。卻才聽見老爺善言相讚,就欲出來迎接;恐怕是個邪鬼敲門,故此拾一塊斷磚,把鐘打一下壓驚,方敢出來。老爺請起。」那唐僧方然正性道:「住持,險些兒諕殺我也。你帶我進去。」

那道人引定唐僧,直至三層門裡看處,比外邊甚是不同。但見那:
\begin{quote}
青磚砌就彩雲牆,綠瓦蓋成琉璃殿。黃金裝聖像,白玉造階臺。大雄殿上舞青光,毘羅閣下生銳氣。文殊殿結采飛雲,輪藏堂描花堆翠。三簷頂上寶瓶尖,五福樓中平繡蓋。千株翠竹搖禪榻,萬種青松映佛門。碧雲宮裡放金光,紫霧叢中飄瑞靄。朝聞四野香風遠,暮聽山高畫鼓鳴。應有朝陽補破衲,豈無對月了殘經。又只見半壁燈光明後院,一行香霧照中庭。
\end{quote}

三藏見了,不敢進去,叫:「道人,你這前邊十分狼狽,後邊這等齊整,何也?」道人笑道:「老爺,這山中多有妖邪強寇,天色清明,沿山打劫,天陰就來寺裡藏身,被他把佛像推倒墊坐,木植搬來燒火。本寺僧人軟弱,不敢與他講論,因此把這前邊破房都捨與那些強人安歇,從新另化了些施主,蓋得那一所寺院。清混各一,這是西方的事情。」三藏道:「原來是如此。」

正行間,又見山門上有五個大字,乃「鎮海禪林寺」。才舉步,䟕入門裡,忽見一個和尚走來。你看他怎生模樣:
\begin{quote}
頭戴左笄絨錦帽,一對銅圈墜耳根。
身著頗羅毛線服,一雙白眼亮如銀。
手中搖著播郎鼓,口念番經聽不真。
三藏原來不認得,這是西方路上喇嘛僧。
\end{quote}

那喇嘛和尚走出門來,看見三藏眉清目秀,額闊頂平,耳垂肩,手過膝,好似羅漢臨凡,十分俊雅。他走上前扯住,滿面笑唏唏的與他捻手捻腳,摸他鼻子,揪他耳朵,以示親近之意。攜至方丈中,行禮畢,卻問:「老師父何來?」三藏道:「弟子乃東土大唐駕下欽差往西方天竺國大雷音寺拜佛取經者。適行至寶方天晚,特奔上剎借宿一宵,明日早行。望垂方便一二。」那和尚笑道:「不當人子,不當人子。我們不是好意要出家的,皆因父母生身,命犯華蓋,家裡養不住,才捨斷了出家。既做了佛門弟子,切莫說脫空之話。」三藏道:「我是老實話。」和尚道:「那東土到西天,有多少路程?路上有山,山中有洞,洞內有精。想你這個單身,又生得嬌嫩,那裡像個取經的?」三藏道:「院主也見得是。貧僧一人,豈能到此?我有三個徒弟,逢山開路,遇水疊橋,保我弟子,所以到得上剎。」那和尚道:「三位高徒何在?」三藏道:「現在山門外伺候。」那和尚慌了道:「師父,你不知我這裡有虎狼、妖賊、鬼怪傷人。白日裡不敢遠出,未經天晚就關了門戶。這早晚還把人放在外邊?」叫:「徒弟,快去請將進來。」

有兩個小喇嘛兒跑出外去,看見行者,諕了一跌;見了八戒,又是一跌。扒起來往後飛跑,道:「爺爺,造化低了,你的徒弟不見,只有三四個妖怪站在那門首也。」三藏問道:「怎麼模樣?」小和尚道:「一個雷公嘴,一個碓挺嘴,一個青臉獠牙。傍有一個女子,倒是個油頭粉面。」三藏笑道:「你不認得。那三個醜的,是我徒弟。那一個女子,是我打松林裡救命來的。」那喇嘛道:「爺爺呀!這們好俊師父,怎麼尋這般醜徒弟?」三藏道:「他醜自醜,卻俱有用。你快請他進來,若再遲了些兒,那雷公嘴的有些闖禍,不是個人生父母養的,他就打進來也。」

那小和尚即忙跑出,戰兢兢的跪下道:「列位老爺,唐老爺請哩。」八戒笑道:「哥啊,他請便罷了,卻這般戰兢兢的,何也?」行者道:「看見我們醜陋害怕。」八戒道:「可是扯淡。我們乃生成的,那個是好要醜哩?」行者道:「把那醜且略收拾收拾。」獃子真個把嘴揣在懷裡,低著頭,牽著馬;沙僧挑著擔;行者在後面拿著棒,轄著那女子:一行進去。穿過了那倒塌房廊,入三層門裡,拴了馬,歇了擔。進方丈中,與喇嘛僧相見,分了坐次。那和尚入裡邊,引出七八十個小喇嘛來,見禮畢,收拾辦齋管待。正是:
\begin{quote}
積功須在慈悲念,佛法興時僧讚僧。
\end{quote}

畢竟不知怎生離寺,且聽下回分解。
