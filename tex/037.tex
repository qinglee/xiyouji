
\chapter{鬼王夜謁唐三藏 悟空神化引嬰兒}

卻說三藏坐於寶林寺禪堂中燈下,念一會《梁皇水懺》,看一會《孔雀真經》,只坐到三更時候,卻才把經本包在囊裡。正欲起身去睡,只聽得門外撲剌剌一聲響喨,淅零零刮陣狂風。那長老恐吹滅了燈,慌忙將褊衫袖子遮住。又見那燈或明或暗,便覺有些心驚膽戰。此時又困倦上來,伏在經案上盹睡。雖是合眼朦朧,卻還心中明白。耳內嚶嚶,聽著那窗外陰風颯颯。好風!真個那:
\begin{quote}
淅淅瀟瀟,飄飄蕩蕩。淅淅瀟瀟飛落葉,飄飄蕩蕩捲浮雲。滿天星斗皆昏昧,遍地塵沙盡灑紛。一陣家猛,一陣家純。純時松竹敲清韻,猛處江湖波浪渾;刮得那山鳥難棲聲哽哽,海魚不定跳噴噴;東西館閣門窗脫,前後廊房神鬼瞋;佛殿花瓶吹墮地,琉璃搖落慧燈昏;香爐攲倒香灰迸,燭架歪斜燭焰煙。幢幡寶蓋都搖拆,鐘鼓樓臺撼動根。
\end{quote}

那長老昏夢中聽著風聲一時過處,又聞得禪堂外隱隱的叫一聲「師父」!忽擡頭夢中觀看,門外站著一條漢子,渾身上下水淋淋的,眼中垂淚,口裡不住的只叫「師父」。三藏欠身道:「你莫是魍魎妖魅、神怪邪魔,至夜深時,來此戲我?我卻不是那貪慾貪嗔之類。我本是個光明正大之僧,奉東土大唐旨意,上西天拜佛求經者。我手下有三個徒弟,都是降龍伏虎之英豪,掃怪除魔之壯士。他若見了你,碎屍粉骨,化作微塵。此是我大慈悲之意,方便之心。你趁早兒潛身遠遁,莫上我的禪門來。」那人倚定禪堂道:「師父,我不是妖魔鬼怪,亦不是魍魎邪神。」三藏道:「你既不是此類,卻深夜來此何為?」那人道:「師父,你舍眼看我一看。」長老果仔細定睛看處,呀!只見他:
\begin{quote}
頭戴一頂沖天冠,腰束一條碧玉帶,身穿一領飛龍舞鳳赭黃袍,足踏一雙雲頭繡口無憂履,手執一柄列斗羅星白玉珪。面如東岳長生帝,形似文昌開化君。
\end{quote}

三藏見了,大驚失色,急躬身厲聲高叫道:「是那一朝陛下?請坐。」用手忙攙,撲了個空虛。回身坐定,再看處,還是那個人。長老便問:「陛下,你是那裡皇帝?何邦帝王?想必是國土不寧,讒臣欺虐,半夜逃生至此。有何話說,說與我聽。」這人才淚滴腮邊談舊事,愁攢眉上訴前因。道:「師父啊,我家住在正西,離此只有四十里遠近。那廂有座城池,便是興基之處。」三藏道:「叫做甚麼地名?」那人道:「不瞞師父說,便是朕當時創立家邦,改號烏雞國。」三藏道:「陛下這等驚慌,卻因甚事至此?」那人道:「師父啊,我這裡五年前,天年乾旱,草子不生,民皆饑死,甚是傷情。」三藏聞言,點頭笑道:「陛下啊,古人云:『國正天心順。』想必是你不慈恤萬民,既遭荒歉,怎麼就躲離城郭?且去開了倉庫,賑濟黎民,悔過前非,重興今善,放赦了那枉法冤人,自然天心和合,雨順風調。」那人道:「我國中倉廩空虛,錢糧盡絕。文武兩班停俸祿,寡人膳食亦無葷。倣效禹王治水,與萬民同受甘苦,沐浴齋戒,晝夜焚香祈禱。如此三年,只乾得河枯井涸。正都在危急之處,忽然鍾南山來了一個全真,能呼風喚雨,點石成金。先見我文武多官,後來見朕,當即請他登壇祈禱,果然有應,只見令牌響處,頃刻間大雨滂沱。寡人只望三尺雨足矣,他說久旱不能潤澤,又多下了二寸。朕見他如此尚義,就與他八拜為交,以兄弟稱之。」三藏道:「此陛下萬千之喜也。」那人道:「喜自何來?」三藏道:「那全真既有這等本事,若要雨時,就教他下雨;若要金時,就教他點金。還有那些不足,卻離了城闕來此?」那人道:「朕與他同寢食者,只得二年。又遇著陽春天氣,紅杏夭桃,開花綻蕊。家家士女,處處王孫,俱去遊春賞玩。那時節,文武歸衙,嬪妃轉院。朕與那全真攜手緩步,至御花園裡,忽行到八角琉璃井邊,不知他拋下些甚麼物件,井中有萬道金光。哄朕到井邊看甚麼寶貝,他陡起兇心,撲通的把寡人推下井內,將石板蓋住井口,擁上泥土,移一株芭蕉栽在上面。可憐我啊,已死去三年,是一個落井傷生的冤屈之鬼也。」

唐僧見說是鬼,諕得觔力酥軟,毛骨聳然。沒奈何,只得將言又問他道:「陛下,你說的這話,全不在理。既死三年,那文武多官、三宮皇后,遇三朝見駕殿上,怎麼就不尋你?」那人道:「師父啊,說起他的本事,果然世間罕有。自從害了朕,他當時在花園內搖身一變,就變做朕的模樣,更無差別。現今占了我的江山,暗侵了我的國土。他把我兩班文武、四百朝官、三宮皇后、六院嬪妃,盡屬了他矣。」三藏道:「陛下,你忒也懦。」那人道:「何懦?」三藏道:「陛下,那怪倒有些神通,變作你的模樣,侵占你的乾坤,文武不能識,后妃不能曉,只有你死的明白,你何不在陰司閻王處具告,把你的屈情伸訴伸訴?」那人道:「他的神通廣大,官吏情熟:都城隍常與他會酒,海龍王盡與他有親,東嶽齊天是他的好朋友,十代閻羅是他的異兄弟。因此這般,我也無門投告。」

三藏道:「陛下,你陰司裡既沒本事告他,卻來我陽世間作甚?」那人道:「師父啊,我這一點冤魂,怎敢上你的門來?山門前有那護法諸天、六丁六甲、五方揭諦、四值功曹、一十八位護教伽藍,緊隨鞍馬。卻才被夜遊神一陣神風,把我送將進來。他說我三年水災該滿,著我來拜謁師父。他說你手下有一個大徒弟,是齊天大聖,極能斬怪降魔。今來志心拜懇,千乞到我國中,拿住妖魔,辨明邪正。朕當結草銜環,報酬師父恩也。」三藏道:「陛下,你此來是請我徒弟與你去除卻那妖怪麼?」那人道:「正是,正是。」三藏道:「我徒弟幹別的事不濟,但說降妖捉怪,正合他宜。陛下啊,雖是著他拿怪,但恐理上難行。」那人道:「怎麼難行?」三藏道:「那怪既神通廣大,變得與你相同;滿朝文武,一個個言和心順;三宮妃嬪,一個個意合情投。我徒弟縱有手段,決不敢輕動干戈。倘被多官拿住,說我們欺邦滅國,問一款大逆之罪,困陷城中,卻不是畫虎刻鵠也?」那人道:「我朝中還有人哩。」

三藏道:「卻好,卻好。想必是一代親王侍長,發付何處鎮守去了?」那人道:「不是。我本宮有個太子,是我親生的儲君。」三藏道:「那太子想必被妖魔貶了?」那人道:「不曾。他只在金鑾殿上,五鳳樓中,或與學士講書,或共全真登位。自此三年,禁太子不入皇宮,不能夠與娘娘相見。」三藏道:「此是何故?」那人道:「此是妖怪使下的計策。只恐他母子相見,閑中論出長短,怕走了消息。故此兩不會面,他得永住常存也。」三藏道:「你的災屯,想應天付,卻與我相類。當時我父曾被水賊傷生;我母被水賊欺占,經三個月,分娩了我。我在水中逃了性命,幸金山寺恩師救養成人。記得我幼年無父母,此間那太子失雙親,真個可憐!」

又問道:「你縱有太子在朝,我怎的與他相見?」那人道:「如何不得見?」三藏道:「他被妖魔拘轄,連一個生身之母尚不得見,我一個和尚,欲見何由?」那人道:「他明早出朝來也。」三藏問:「出朝作甚?」那人道:「明日早朝,領三千人馬,架鷹犬,出城採獵,師父斷得與他相見。見時肯將我的言語說與他,他便信了。」三藏道:「他本是肉眼凡胎,被妖魔哄在殿上,那一日不叫他幾聲父王?他怎肯信我的言語?」那人道:「既恐他不信,我留下一件表記與你罷。」三藏問:「是何物件?」那人把手中執的金廂白玉珪放下道:「此物可以為記。」三藏道:「此物何如?」那人道:「全真自從變作我的模樣,只是少變了這件寶貝。他到宮中,說那求雨的全真拐了此珪去了。自此三年,還沒此物。我太子若看見,他睹物思人,此仇必報。」三藏道:「也罷,等我留下,著徒弟與你處置。卻在那裡等麼?」那人道:「我也不敢等。我這去,還央求夜遊神,再使一陣神風,把我送進皇宮內院,託一夢與我那正宮皇后,教他母子們合意,你師徒們同心。」三藏點頭應承道:「你去罷。」

那冤魂叩頭拜別,舉步相送,不知怎麼踢了腳,跌了一個觔斗,把三藏驚醒,卻原來是南柯一夢。慌得對著那盞昏燈,連忙叫:「徒弟,徒弟。」八戒醒來道:「甚麼『土地土地』?當時我做好漢,專一吃人度日,受用腥羶,其實快活,偏你出家,教我們保護你跑路。原說只做和尚,如今拿做奴才,日間挑包袱、牽馬,夜間提尿瓶、務腳!這早晚不睡,又叫徒弟作甚?」三藏道:「徒弟,我剛才伏在案上打盹,做了一個怪夢。」行者跳將起來道:「師父,夢從想中來。你未曾上山,先怕怪物;又愁雷音路遠,不能得到;思念長安,不知何日回程:所以心多夢多。似老孫一點真心,專要西方見佛,更無一個夢兒到我。」三藏道:「徒弟,我這一夢,不是思鄉之夢。才然合眼,見一陣狂風過處,禪房門外有一朝皇帝,自言是烏雞國王。渾身水濕,滿眼垂淚。」這等這等,如此如此,將那夢中話一一的說與行者。行者笑道:「不消說了,他來託夢與你,分明是照顧老孫一場生意。必然是個妖怪在那裡篡位謀國,等我與他辨個真假。想那妖魔棍到處,立業成功。」三藏道:「徒弟,他說那怪神通廣大哩。」行者道:「怕他甚麼廣大?早知老孫到,教他即走無方。」三藏道:「我又記得留下一件寶貝做表記。」八戒答道:「師父莫要胡纏,做個夢便罷了,怎麼只管當真?」沙僧道:「『不信直中直,須防仁不仁。』我們打起火,開了門,看看如何便是。」

行者果然開門,一齊看處,只見星月光中,階簷上,真個放著一柄金廂白玉珪。八戒近前拿起道:「哥哥,這是甚麼東西?」行者道:「這是國王手中執的寶貝,名喚玉珪。師父啊,既有此物,想此事是真。明日拿妖,全都在老孫身上。只是要你三樁兒造化低哩。」八戒道:「好好好,做個夢罷了,又告訴他。他那些兒不會作弄人哩?就教你三樁兒造化低。」三藏回入裡面道:「是那三樁?」行者道:「明日要你頂缸、受氣、遭瘟。」八戒笑道:「一樁兒也是難的,三樁兒卻怎麼耽得?」唐僧是個聰明的長老,便問:「徒弟啊,此三事如何講?」行者道:「也不消講,等我先與你二件物。」

好大聖,拔了一根毫毛,吹口仙氣,叫聲:「變!」變做一個紅金漆匣兒,把白玉珪放在內盛著,道:「師父,你將此物捧在手中,到天曉時,穿上錦襴袈裟,去在正殿坐著念經,等我去看看他那城池。端的是個妖怪,就打殺他,也在此間立個功績;假若不是,且休撞禍。」三藏道:「正是,正是。」行者道:「那太子不出城便罷,若真個應夢出城來,我定引他來見你。」三藏道:「見了我如何迎答?」行者道:「來到時,我先報知。你把那匣蓋兒扯開些,等我變作二寸長的一個小和尚,放在匣兒裡,你連我捧在手中。那太子進了寺來,必然拜佛。你盡他怎的下拜,只是不睬他。他見你不動身,一定教拿你。你憑他拿下去,打也由他,綁也由他,殺也由他。」三藏道:「呀!他的軍令大,真個殺了我,怎麼好?」行者道:「沒事,有我哩,若到那緊關處,我自然護你。他若問時,你說是東土欽差上西天拜佛取經進寶的和尚。他道:『有甚寶貝?』你卻把錦襴袈裟對他說一遍,說道:『此是三等寶貝。還有頭一等、第二等的好物哩。』但問處,就說這匣內有一件寶貝,上知五百年,下知五百年,中知五百年,共一千五百年過去未來之事,俱盡曉得。卻把老孫放出來。我將那夢中話告訴那太子。他若是肯信,就去拿了那妖魔,一則與他父王報仇,二來我們立個名節;他若不信,再將白玉珪拿與他看。只恐他年幼,還不認得哩。」三藏聞言,大喜道:「徒弟啊,此計絕妙!但說這寶貝,一個叫做錦襴袈裟,一個叫做白玉珪;你變的寶貝卻叫做甚名?」行者道:「就叫做立帝貨罷。」三藏依言,記在心上。師徒們一夜那曾得睡,盼到天明,恨不得點頭喚出扶桑日,噴氣吹散滿天星。

不多時,東方發白。行者又吩咐了八戒、沙僧,教他兩個:「不可攪擾僧人,出來亂走。待我成功之後,共汝等同行。」才別了,唿哨,一觔斗,跳在空中。睜火眼平西看處,果見有一座城池。你道怎麼就看見了?當時說那城池離寺只有四十里,故此憑高就望見了。行者近前仔細看處,又見那怪霧愁雲漠漠,妖風怨氣紛紛。行者在空中讚嘆道:
\begin{quote}
「若是真王登寶座,自有祥光五色雲。
只因妖怪侵龍位,騰騰黑氣鎖金門。」
\end{quote}

行者正在感嘆,忽聽得炮聲響喨,又只見東門開處,閃出一路人馬,真個是採獵之軍,果然勢勇。但見:
\begin{quote}
曉出禁城東,分圍淺草中。彩旗開映日,白馬驟迎風。鼉鼓鼕鼕擂,標槍對對衝。架鷹軍猛烈,牽犬將驍雄。火炮連天振,粘竿映日紅。人人支弩箭,個個挎雕弓。張網山坡下,鋪繩小徑中。一聲驚霹靂,千騎擁貔熊。狡兔身難保,乖獐智亦窮。狐狸該命盡,麋鹿喪當中。山雉難飛脫,野雞怎避兇。他都揀占山場擒猛獸,摧殘林木射飛蟲。
\end{quote}

那些人出得城來,散步東郊。不多時,有二十里向高田地,又只見中軍營裡,有小小的一個將軍:頂著盔,貫著甲,裹肚花,十八札,手執青鋒寶劍,坐下黃驃馬,腰帶滿弦弓。真個是:
\begin{quote}
隱隱君王像,昂昂帝主容。
規模非小輩,行動顯真龍。
\end{quote}

行者在空暗喜道:「不須說,那個就是皇帝的太子了。等我戲他一戲。」好大聖,按落雲頭,撞入軍中太子馬前,搖身一變,變作一個白兔兒,只在太子馬前亂跑。太子看見,正合歡心,拈起箭,拽滿弓,一箭正中了那兔兒。

原來是那大聖故意教他中了,卻眼乖手疾,一把接住那箭頭,把箭翎花落在前邊,丟開腳步跑了。那太子見箭中了玉兔,兜開馬,獨自爭先來趕。不知馬行的快,行者如風;馬行的遲,行者慢走:只在他面前不遠。看他一程一里,將太子哄到寶林寺山門之下,行者現了本身。不見兔兒,只見一枝箭插在門檻上。徑撞進去,見唐僧道:「師父,來了,來了。」卻又一變,變做二寸長的小和尚兒,鑽在紅匣之內。

卻說那太子趕到山門前,不見了玉兔,只見門檻上插著一枝雕翎箭。太子大驚失色道:「怪哉!怪哉!分明我箭中了玉兔,玉兔怎麼不見,只見箭在此間?想是年多日久,成了精魅也。」拔了箭,擡頭看處,山門上有五個大字,寫著「敕建寶林寺」。太子道:「我知之矣。向年間曾記得我父王在金鑾殿上,差官賫些金帛,與這和尚修理佛殿佛象,不期今日到此。正是:因過道院逢僧話,又得浮生半日閑。我且進去走走。」

那太子跳下馬來,正要進去,只見那保駕的官將與三千人馬趕上,簇簇擁擁,都入山門裡面。慌得那本寺眾僧,都來叩頭拜接,接入正殿中間,參拜佛像。卻才舉目觀瞻,又欲遊廊玩景,忽見正當中坐著一個和尚,太子大怒道:「這個和尚無禮!我今半朝鑾駕進山,雖無旨意知會,不當遠接,此時軍馬臨門,也該起身,怎麼還坐著不動?」教:「拿下來!」說聲「拿」字,兩邊校尉一齊下手,把唐僧抓將下來,急理繩索便綑。行者在匣裡默默的念咒,教道:「護法諸天、六丁六甲,我今設法降妖,這太子不能知識,將繩要綑我師父,汝等即早護持;若真綑了,汝等都該有罪。」那大聖暗中吩咐,誰敢不遵,卻將三藏護持定了,那些人摸也摸不著他光頭,好似一壁牆擋住,難攏其身。

那太子道:「你是那方來的,使這般隱身法欺我?」三藏上前施禮道:「貧僧無隱身法,乃是東土唐僧,上雷音寺拜佛求經進寶的和尚。」太子道:「你那東土雖是中原,其窮無比,有甚寶貝,你說來我聽。」三藏道:「我身上穿的這袈裟,是第三樣寶貝。還有第一等、第二等更好的物哩。」太子道:「你那衣服,半邊苫身,半邊露臂,能值多少物,敢稱寶貝?」三藏道:「這袈裟雖不全體,有詩幾句。詩曰:
\begin{quote}
佛衣偏袒不須論,內隱真如脫世塵。
萬線千針成正果,九珠八寶合元神。
仙娥聖女恭修製,遺賜禪僧靜垢身。
見駕不迎猶自可,你的父冤未報枉為人。」
\end{quote}

太子聞言,心中大怒道:「這潑和尚胡說!你那半片衣,憑著你口能舌便,誇好誇強。我的父冤從何未報?你說來我聽。」三藏進前一步,合掌問道:「殿下,為人生在天地之間,能有幾恩?」太子道:「有四恩。」三藏道:「那四恩?」太子道:「感天地蓋載之恩,日月照臨之恩,國王水土之恩,父母養育之恩。」三藏笑曰:「殿下言之有失。人只有天地蓋載,日月照臨,國王水土,那得個父母養育來?」太子怒道:「和尚是那遊手遊食削髮逆君之徒。人不得父母養育,身從何來?」三藏道:「殿下,貧僧不知。但只這紅匣內有一件寶貝,叫做立帝貨,他上知五百年,中知五百年,下知五百年,共知一千五百年過去未來之事,便知無父母養育之恩,令貧僧在此久等多時矣。」

太子聞說,教:「拿來我看。」三藏扯開匣蓋兒,那行者跳將出來,呀的兩邊亂走。太子道:「這星星小人兒,能知甚事?」行者聞言嫌小,卻就使個神通,把腰伸一伸,就長了有三尺四五寸。眾軍士吃驚道:「若是這般快長,不消幾日,就撐破天也。」行者長到原身,就不長了。太子才問道:「立帝貨,這老和尚說你能知未來過去吉凶,你卻有龜作卜?有蓍作筮?憑書句斷人禍福?」行者道:「我一毫不用,只是全憑三寸舌,萬事盡皆知。」太子道:「這廝又是胡說。自古以來,《周易》之書,極其玄妙,斷盡天下吉凶,使人知所趨避,故龜所以卜,蓍所以筮。聽汝之言,憑據何理?妄言禍福,扇惑人心。」

行者道:「殿下且莫忙,等我說與你聽。你本是烏雞國王的太子。你那裡五年前,年程荒旱,萬民遭苦,你家皇帝共臣子秉心祈禱。正無點雨之時,鍾南山來了一個道士,他善呼風喚雨,點石為金。君王忒也愛小,就與他拜為兄弟。這樁事有麼?」太子道:「有有有。你再說說。」行者道:「後三年不見全真,稱孤的卻是誰?」太子道:「果是有個全真,父王與他拜為兄弟,食則同食,寢則同寢。三年前在御花園裡玩景,被他一陣神風,把父王手中金廂白玉珪攝回鍾南山去了。至今父王還思慕他。因不見他,遂無心賞玩,把花園緊閉了,已三年矣。做皇帝的,非我父王而何?」

行者聞言,哂笑不絕。太子再問不答,只是哂笑。太子怒道:「這廝當言不言,如何這等哂笑?」行者又道:「還有許多話哩,奈何左右人眾,不是說處。」太子見他言語有因,將袍袖一展,教軍士且退。那駕上官將急傳令,將三千人馬都出門外住扎。此時殿上無人,太子坐在上面,長老立在前邊,左手傍立著行者。本寺諸僧皆退。行者才正色上前道:「殿下,化風去的是你生身之父母見坐位的,是那祈雨之全真。」太子道:「胡說,胡說。我父自全真去後,風調雨順,國泰民安。照依你說,就不是我父王了。還是我年孺,容得你;若我父王聽見你這反話,拿了去,碎屍萬段。」把行者咄的喝下來。行者對唐僧道:「何如?我說他不信,果然,果然。如今卻拿那寶貝進與他,倒換關文,往西方去罷。」三藏即將紅匣子遞與行者。行者接過來,將身一抖,那匣兒卒不見了。原是他毫毛變的,被他收上身去。卻將白玉珪雙手捧上,獻與太子。

太子見了道:「好和尚,好和尚。你五年前本是個全真,來騙了我家的寶貝,如今又妝做和尚來進獻。」叫:「拿了!」一聲傳令,把長老諕得慌忙指著行者道:「你這弼馬溫,專撞空頭禍,帶累我哩。」行者近前一齊攔住道:「休嚷,莫走了風!我不叫做立帝貨,還有真名哩。」太子怒道:「你上來。我問你個真名字,好送法司定罪!」行者道:「我是那長老的大徒弟,名喚悟空孫行者。因與我師父上西天取經,昨宵到此覓宿。我師父夜讀經卷,至三更時分,得一夢。夢見你父王道,他被那全真欺害,推在御花園八角琉璃井內,全真變作他的模樣。滿朝官不能知。你年幼亦無分曉,禁你入宮,關了花園,大端怕漏了消息。你父王今夜特來請我降魔。我恐不是妖邪,自空中看了,果然是個妖精。正要動手拿他,不期你出城打獵。你箭中的玉兔,就是老孫。老孫把你引到寺裡,見師父,訴此衷腸,句句是實。你既然認得白玉珪,怎麼不念鞠養恩情,替親報仇?」

那太子聞言,心中慘慼,暗自傷愁道:「若不信此言語,他卻有三分兒真實;若信了,怎奈殿上見是我父王?」這才是進退兩難心問口,三思忍耐口問心。行者見他疑惑不定,又上前道:「殿下不必心疑,請殿下駕回本國,問你國母娘娘一聲,看他夫妻恩愛之情,比三年前如何。只此一問,便知真假矣。」

那太子回心道:「正是。且待我問我母親去來。」他跳起身,籠了白玉珪就走。行者扯住道:「你這些人馬都回,卻不走漏消息?我難成功。但要你單人獨馬進城,不可揚名賣弄。莫入正陽門,須從後宰門進去。到宮中見你母親,切休高聲大氣,須是悄語低言。恐那怪神通廣大,一時走了消息,你娘兒們性命俱難保也。」太子謹遵教命。出山門吩咐將官:「穩在此扎營,不得移動。我有一事,待我去了就來,一同進城。」看他:
\begin{quote}
指揮號令屯軍士,上馬如飛即轉城。
\end{quote}

這一去,不知見了娘娘,有何話說且聽下回分解。
