
\chapter{悟徹菩提真妙理 斷魔歸本合元神}

話表美猴王得了姓名,怡然踴躍,對菩提前作禮啟謝。那祖師即命大眾引孫悟空出二門外,教他灑掃應對、進退周旋之節。眾仙奉行而出。悟空到門外,又拜了大眾師兄,就於廊廡之間安排寢處。次早,與眾師兄學言語禮貌、講經論道、習字焚香。每日如此。閑時即掃地鋤園、養花修樹、尋柴燃火、挑水運漿。凡所用之物,無一不備。在洞中不覺倏六七年。

一日,祖師登壇高坐,喚集諸仙,開講大道。真個是:
\begin{quote}
天花亂墜,地湧金蓮。妙演三乘教,精微萬法全。慢搖麈尾噴珠玉,響振雷霆動九天。說一會道,講一會禪,三家配合本如然。開明一字皈誠理,指引無生了性玄。
\end{quote}

孫悟空在傍聞講,喜得他抓耳撓腮,眉花眼笑,忍不住手之舞之,足之蹈之。忽被祖師看見,叫孫悟空道:「你在班中,怎麼顛狂躍舞,不聽我講?」悟空道:「弟子誠心聽講,聽到老師父妙音處,喜不自勝,故不覺作此踴躍之狀。望師父恕罪。」祖師道:「你既識妙音,我且問你,你到洞中多少時了?」悟空道:「弟子本來懵懂,不知多少時節。只記得灶下無火,常去山後打柴,見一山好桃樹,我在那裡吃了七次飽桃矣。」祖師道:「那山喚名爛桃山。你既吃七次,想是七年了。你今要從我學些甚麼道?」悟空道:「但憑尊師教誨,只是有些道氣兒,弟子便就學了。」

祖師道:「『道』字門中有三百六十傍門,傍門皆有正果。不知你學那一門哩?」悟空道:「憑尊師意思,弟子傾心聽從。」祖師道:「我教你個『術』字門中之道,如何?」悟空道:「術門之道怎麼說?」祖師道:「術字門中,乃是些請仙、扶鸞、問卜、揲蓍,能知趨吉避凶之理。」悟空道:「似這般可得長生麼?」祖師道:「不能,不能。」悟空道:「不學,不學。」

祖師又道:「教你『流』字門中之道,如何?」悟空又問:「流字門中是甚義理?」祖師道:「流字門中,乃是儒家、釋家、道家、陰陽家、墨家、醫家,或看經,或念佛,並朝真降聖之類。」悟空道:「似這般可得長生麼?」祖師道:「若要長生,也似壁裡安柱。」悟空道:「師父,我是個老實人,不曉得打市語。怎麼謂之『壁裡安柱』?」祖師道:「人家蓋房,欲圖堅固,將牆壁之間立一頂柱,有日大廈將頹,他必朽矣。」悟空道:「據此說,也不長久。不學,不學。」

祖師道:「教你『靜』字門中之道,如何?」悟空道:「靜字門中是甚正果?」祖師道:「此是休糧守谷、清靜無為、參禪打坐、戒語持齋,或睡功,或立功,並入定、坐關之類。」悟空道:「這般也能長生麼?」祖師道:「也似窰頭土坯。」悟空笑道:「師父果有些滴澾。一行說我不會打市語。怎麼謂之『窰頭土坯』?」祖師道:「就如那窰頭上造成磚瓦之坯,雖已成形,尚未經水火鍛煉,一朝大雨滂沱,他必濫矣。」悟空道:「也不長遠。不學,不學。」

祖師道:「教你『動』字門中之道,如何?」悟空道:「動門之道卻又怎麼?」祖師道:「此是有為有作:採陰補陽、攀弓踏弩、摩臍過氣、用方炮製、燒茅打鼎、進紅鉛、煉秋石,並服婦乳之類。」悟空道:「似這等也得長生麼?」祖師道:「此欲長生,亦如水中撈月。」悟空道:「師父又來了。怎麼叫做『水中撈月』?」祖師道:「月在長空,水中有影,雖然看見,只是無撈摸處,到底只成空耳。」悟空道:「也不學,不學。」

祖師聞言,咄的一聲,跳下高臺,手持戒尺,指定悟空道:「你這猢猻,這般不學,那般不學,卻待怎麼?」走上前,將悟空頭上打了三下。倒背著手,走入裡面,將中門關了,撇下大眾而去。諕得那一班聽講的人人驚懼,皆怨悟空道:「你這潑猴,十分無狀。師父傳你道法,如何不學,卻與師父頂嘴?這番衝撞了他,不知幾時才出來啊!」此時俱甚報怨他,又鄙賤嫌惡他。悟空一些兒也不惱,只是滿臉陪笑。原來那猴王已打破盤中之謎,暗暗在心,所以不與眾人爭競,只是忍耐無言。祖師打他三下者,教他三更時分存心;倒背著手走入裡面,將中門關上者,教他從後門進步,秘處傳他道也。

當日悟空與眾等喜喜歡歡,在三星仙洞之前,盼望天色,急不能到晚。及黃昏時,卻與眾就寢,假合眼,定息存神。山中又沒打更傳箭,不知時分,只自家將鼻孔中出入之氣調定。約到子時前後,輕輕的起來,穿了衣服,偷開前門,躲離大眾,走出外,擡頭觀看,正是那:
\begin{quote}
月明清露冷,八極迥無塵。
深樹幽禽宿,源頭水溜汾。
飛螢光散影,過雁字排雲。
正直三更候,應該訪道真。
\end{quote}

你看他從舊路徑至後門外,只見那門兒半開半掩。悟空喜道:「老師父果然注意與我傳道,故此開著門也。」即曳步近前,側身進得門裡,只走到祖師寢榻之下。見祖師踡跼身軀,朝裡睡著了。悟空不敢驚動,即跪在榻前。那祖師不多時覺來,舒開兩足,口中自吟道:
\begin{quote}
「難!難!難!道最玄,莫把金丹作等閑。不遇至人傳妙訣,空言口困舌頭乾!」
\end{quote}

悟空應聲叫道:「師父,弟子在此跪候多時。」祖師聞得聲音是悟空,即起披衣,盤坐喝道:「這猢猻,你不在前邊去睡,卻來我這後邊作甚?」悟空道:「師父昨日壇前對眾相允,教弟子三更時候,從後門裡傳我道理,故此大膽徑拜老爺榻下。」祖師聽說,十分歡喜,暗自尋思道:「這廝果然是個天地生成的,不然,何就打破我盤中之暗謎也?」悟空道:「此間更無六耳,止只弟子一人,望師父大捨慈悲,傳與我長生之道罷,永不忘恩。」祖師道:「你今有緣,我亦喜說。既識得盤中暗謎,你近前來,仔細聽之,當傳與你長生之妙道也。」悟空叩頭謝了,洗耳用心,跪於榻下。祖師云:
\begin{quote}
顯密圓通真妙訣,惜修性命無他說。
都來總是精氣神,謹固牢藏休漏泄。
休漏泄,體中藏,汝受吾傳道自昌。
口訣記來多有益,屏除邪慾得清涼。
得清涼,光皎潔,好向丹臺賞明月。
月藏玉兔日藏烏,自有龜蛇相盤結。
相盤結,性命堅,卻能火裡種金蓮。
攢簇五行顛倒用,功完隨作佛和仙。
\end{quote}

此時說破根源,悟空心靈福至,切切記了口訣。對祖師拜謝深恩,即出後門觀看。但見東方天色微舒白,西路金光大顯明。依舊路,轉到前門,輕輕的推開進去,坐在原寢之處,故將床鋪搖響道:「天光了,天光了,起耶!」那大眾還正睡哩,不知悟空已得了好事。當日起來打混,暗暗維持,子前午後,自己調息。

卻早過了三年,祖師復登寶座,與眾說法。談的是公案比語,論的是外像包皮。忽問:「悟空何在?」悟空近前跪下:「弟子有。」祖師道:「你這一向修些甚麼道來?」悟空道:「弟子近來法性頗通,根源亦漸堅固矣。」祖師道:「你既通法性,會得根源,已注神體,卻只是防備著三災利害。」悟空聽說,沉吟良久道:「師父之言謬矣。我常聞道高德隆,與天同壽;水火既濟,百病不生。卻怎麼有個『三災利害』?」祖師道:「此乃非常之道:奪天地之造化,侵日月之玄機;丹成之後,鬼神難容。雖駐顏益壽,但到了五百年後,天降雷災打你,須要見性明心,預先躲避。躲得過,壽與天齊;躲不過,就此絕命。再五百年後,天降火災燒你。這火不是天火,亦不是凡火,喚做『陰火』。自本身湧泉穴下燒起,直透泥垣宮,五臟成灰,四肢皆朽,把千年苦行,俱為虛幻。再五百年,又降風災吹你。這風不是東南西北風,不是和薰金朔風,亦不是花柳松竹風,喚做『贔風』。自顖門中吹入六腑,過丹田,穿九竅,骨肉消疏,其身自解。所以都要躲過。」

悟空聞說,毛骨悚然,叩頭禮拜道:「萬望老爺垂憫,傳與躲避三災之法,到底不敢忘恩。」祖師道:「此亦無難,只是你比他人不同,故傳不得。」悟空道:「我也頭圓頂天,足方履地,一般有九竅四肢,五臟六腑,何以比人不同?」祖師道:「你雖然像人,卻比人少腮。」原來那猴子孤拐面,凹臉尖嘴。悟空伸手一摸,笑道:「師父沒成算。我雖少腮,卻比人多這個素袋,亦可准折過也。」祖師說:「也罷,你要學那一般?有一般天罡數,該三十六般變化;有一般地煞數,該七十二般變化。」悟空道:「弟子願多裡撈摸,學一個地煞變化罷。」祖師道:「既如此,上前來,傳與你口訣。」遂附耳低言,不知說了些甚麼妙法。這猴王也是他一竅通時百竅通,當時習了口訣,自修自煉,將七十二般變化都學成了。

忽一日,祖師與眾門人在三星洞前戲玩晚景。祖師道:「悟空,事成了未曾?」悟空道:「多蒙師父海恩,弟子功果完備,已能霞舉飛昇也。」祖師道:「你試飛舉我看。」悟空弄本事,將身一聳,打了個連扯跟頭,跳離地有五六丈,踏雲霞去勾有頓飯之時,返復不上三里遠近,落在面前,扠手道:「師父,這就是飛舉騰雲了。」祖師笑道:「這個算不得騰雲,只算得爬雲而已。自古道:『神仙朝遊北海暮蒼梧。』似你這半日,去不上三里,即爬雲也還算不得哩。」悟空道:「怎麼為『朝遊北海暮蒼梧』?」祖師道:「凡騰雲之輩,早辰起自北海,遊過東海、西海、南海,復轉蒼梧。蒼梧者,卻是北海零陵之語話也。將四海之外,一日都遊遍,方算得騰雲。」悟空道:「這個卻難,卻難。」祖師道:「世上無難事,只怕有心人。」悟空聞得此言,叩頭禮拜,啟道:「師父,為人須為徹,索性捨個大慈悲,將此騰雲之法,一發傳與我罷,決不敢忘恩。」祖師道:「凡諸仙騰雲,皆跌足而起,你卻不是這般。我才見你去,連扯方才跳上。我今只就你這個勢,傳你個觔斗雲罷。」悟空又禮拜懇求,祖師卻又傳個口訣道:「這朵雲,捻著訣,念動真言,攢緊了拳,將身一抖,跳將起來,一觔斗就有十萬八千里路哩。」大眾聽說,一個個嘻嘻笑道:「悟空造化,若會這個法兒,與人家當鋪兵、送文書、遞報單,不管那裡都尋了飯吃。」師徒們天昏各歸洞府。

這一夜,悟空即運神煉法,會了觔斗雲。逐日家無拘無束,自在逍遙,此亦長生之美。

一日,春歸夏至,大眾都在松樹下會講多時。大眾道:「悟空,你是那世修來的緣法?前日老師父附耳低言,傳與你的躲三災變化之法,可都會麼?」悟空笑道:「不瞞諸兄長說,一則是師父傳授,二來也是我晝夜慇懃,那幾般兒都會了。」大眾道:「趁此良時,你試演演,讓我等看看。」悟空聞說,抖擻精神,賣弄手段道:「眾師兄請出個題目。要我變化甚麼?」大眾道:「就變棵松樹罷。」悟空捻著訣,念動咒語,搖身一變,就變做一棵松樹。真個是:
\begin{quote}
鬱鬱含煙貫四時,凌雲直上秀貞姿。
全無一點妖猴像,盡是經霜耐雪枝。
\end{quote}

大眾見了鼓掌,呵呵大笑,都道:「好猴兒,好猴兒!」不覺的嚷鬧,驚動了祖師。祖師急拽杖出門來問道:「是何人在此喧嘩?」大眾聞呼,慌忙檢束,整衣向前。悟空也現了本相,雜在叢中道:「啟上尊師:我等在此會講,更無外姓喧嘩。」祖師怒喝道:「你等大呼小叫,全不像個修行的體段!修行的人,口開神氣散,舌動是非生,如何在此嚷笑?」大眾道:「不敢瞞師父,適才孫悟空演變化耍子。教他變棵松樹,果然是棵松樹,弟子們俱稱揚喝采,故高聲驚冒尊師,望乞恕罪。」

祖師道:「你等起去。」叫:「悟空過來!我問你弄甚麼精神,變甚麼松樹?這個工夫,可好在人前賣弄?假如你見別人有,不要求他?別人見你有,必然求你。你若畏禍,卻要傳他;若不傳他,必然加害:你之性命又不可保。」悟空叩頭道:「只望師父恕罪。」祖師道:「我也不罪你,但只是你去罷。」悟空聞此言,滿眼墮淚道:「師父,教我往那裡去?」祖師道:「你從那裡來,便從那裡去就是了。」悟空頓然醒悟道:「我自東勝神洲傲來國花果山水簾洞來的。」祖師道:「你快回去,全你性命;若在此間,斷然不可。」悟空領罪,上告尊師:「我也離家有二十年矣,雖是回顧舊日兒孫,但念師父厚恩未報,不敢去。」祖師道:「那裡甚麼恩義,你只是不惹禍,不牽帶我就罷了。」

悟空見沒奈何,只得拜辭,與眾相別。祖師道:「你這去,定生不良。憑你怎麼惹禍行兇,卻不許說是我的徒弟。你說出半個字來,我就知之,把你這猢猻剝皮剉骨,將神魂貶在九幽之處,教你萬劫不得翻身!」悟空道:「決不敢提起師父一字,只說是我自家會的便罷。」

悟空謝了,即抽身,捻著訣,丟個連扯,縱起觔斗雲,徑回東勝。那裡消一個時辰,早看見花果山水簾洞。美猴王自知快樂,暗暗的自稱道:
\begin{quote}
去時凡骨凡胎重,得道身輕體亦輕。
舉世無人肯立志,立志修玄玄自明。
當時過海波難進,今日回來甚易行。
別語叮嚀還在耳,何期頃刻見東溟。
\end{quote}

悟空按下雲頭,直至花果山,找路而走。忽聽得鶴唳猿啼:鶴唳聲沖霄漢外,猿啼悲切甚傷情。即開口叫道:「孩兒們,我來了也!」那崖下石坎邊,花草中,樹木裡,若大若小之猴,跳出千千萬萬,把個美猴王圍在當中,叩頭叫道:「大王,你好寬心,怎麼一去許久?把我們俱閃在這裡,望你誠如饑渴。近來被一妖魔在此欺虐,強要占我們水簾洞府,是我等捨死忘生,與他爭鬥。這些時,被那廝搶了我們家火,捉了許多子姪,教我們晝夜無眠,看守家業。幸得大王來了,大王若再年載不來,我等連山洞盡屬他人矣。」悟空聞說,心中大怒,道:「是甚麼妖魔,輒敢無狀?你且細細說來,待我尋他報仇。」眾猴叩頭:「告上大王:那廝自稱混世魔王,住居在直北下。」悟空道:「此間到他那裡,有多少路程?」眾猴道:「他來時雲,去時霧,或風或雨,或電或雷,我等不知有多少路。」悟空道:「既如此,你們休怕,且自頑耍,等我尋他去來。」

好猴王,將身一縱,跳起去,一路觔斗,直至北下觀看,見一座高山,真是十分險峻。好山:
\begin{quote}
筆峰挺立,曲澗深沉。筆峰挺立透空霄,曲澗深沉通地戶。兩崖花木爭奇,幾處松篁鬥翠。左邊龍,熟熟馴馴;右邊虎,平平伏伏。每見鐵牛耕,常有金錢種。幽禽睍睆聲,丹鳳朝陽立。石磷磷,波淨淨,古怪蹺蹊真惡獰。世上名山無數多,花開花謝蘩還眾。爭如此景永長存,八節四時渾不動。誠為三界坎源山,滋養五行水臟洞。
\end{quote}

美猴王正默觀看景致,只聽得有人言語,徑自下山尋覓。原來那陡崖之前,乃是那水臟洞。洞門外有幾個小妖跳舞,見了悟空就走。悟空道:「休走!借你口中言,傳我心內事。我乃正南方花果山水簾洞洞主。你家甚麼混世鳥魔,屢次欺我兒孫,我特尋來,要與他見個上下。」

那小妖聽說,疾忙跑入洞裡報道:「大王,禍事了!」魔王道:「有甚禍事?」小妖道:「洞外有猴頭稱為花果山水簾洞洞主,他說你屢次欺他兒孫,特來尋你,見個上下哩。」魔王笑道:「我常聞得那些猴精說他有個大王,出家修行去,想是今番來了。你們見他怎生打扮?有甚器械?」小妖道:「他也沒甚麼器械,光著個頭,穿一領紅色衣,勒一條黃絛,足下踏一對烏靴,不僧不俗,又不像道士、神仙,赤手空拳,在門外叫哩。」魔王聞說:「取我披掛、兵器來。」那小妖即時取出。

那魔王穿了甲冑,綽刀在手,與眾妖出得門來,即高聲叫道:「那個是水簾洞洞主?」悟空急睜睛觀看,只見那魔王:
\begin{quote}
頭戴烏金盔,映日光明;身掛皂羅袍,迎風飄蕩。下穿著黑鐵甲,緊勒皮條;足踏著花褶靴,雄如上將。腰廣十圍,身高三丈。手執一口刀,鋒刃多明亮。稱為混世魔,磊落兇模樣。
\end{quote}

猴王喝道:「這潑魔這般眼大,看不見老孫。」魔王見了,笑道:「你身不滿四尺,年不過三旬,手內又無兵器,怎麼大膽猖狂,要尋我見甚麼上下?」悟空罵道:「你這潑魔,原來沒眼。你量我小,要大卻也不難;你量我無兵器,我兩隻手勾著天邊月哩。你不要怕,只吃老孫一拳。」縱一縱,跳上去,劈臉就打。那魔王伸手架住道:「你這般矬矮,我這般高長;你要使拳,我要使刀。使刀就殺了你,也吃人笑。待我放下刀,與你使路拳看。」悟空道:「說得是。好漢子,走來。」那魔王丟開架子便打,這悟空鑽進去相撞相迎。他兩個拳搥腳踢,一衝一撞。原來長拳空大,短簇堅牢。那魔王被悟空掏短脅,撞了襠,幾下觔節,把他打重了。他閃過,拿起那板大的鋼刀,望悟空劈頭就砍。悟空急撤身,他砍了一個空。悟空見他兇猛,即使身外身法,拔一把毫毛,丟在口中嚼碎,望空噴去,叫一聲:「變!」即變做三、二百個小猴,週圍攢簇。

原來人得仙體,出神變化無方。不知這猴王自從了道之後,身上有八萬四千毛羽,根根能變,應物隨心。那些小猴眼乖會跳,刀來砍不著,槍去不能傷。你看他前踴後躍,鑽上去,把個魔王圍繞,抱的抱,扯的扯,鑽襠的鑽襠,扳腳的扳腳,踢打撏毛,摳眼睛,捻鼻子,擡鼓弄,直打做一個攢盤。這悟空才去奪得他的刀來,分開小猴,照頂門一下,砍為兩段。領眾殺進洞中,將那大小妖精盡皆剿滅。卻把毫毛一抖,收上身來,又見那收不上身者,卻是那魔王在水簾洞擒去的小猴。悟空道:「汝等何為到此?」約有三五十個,都含淚道:「我等因大王修仙去後,這兩年被他爭吵,把我們都攝將來。那不是我們洞中的家火?石盆、石碗都被這廝拿來也。」悟空道:「既是我們的家火,你們都搬出外去。」隨即洞裡放起火來,把那水臟洞燒得枯乾,盡歸了一體。對眾道:「汝等跟我回去。」眾猴道:「大王,我們來時,只聽得耳邊風響,虛飄飄到於此地,更不識路徑,今怎得回鄉?」悟空道:「這是他弄的個術法兒,有何難也?我如今一竅通,百竅通,我也會弄。你們都合了眼,休怕。」

好猴王,念聲咒語,駕陣狂風,雲頭落下。叫:「孩兒們睜眼。」眾猴腳屣實地,認得是家鄉,個個歡喜,都奔洞門舊路。那在洞眾猴,都一齊簇擁同入,分班序齒,禮拜猴王。安排酒果,接風賀喜,啟問降魔救子之事。悟空備細言了一遍,眾猴稱揚不盡道:「大王去到那方,不意學得這般手段。」悟空又道:「我當年別汝等,隨波逐流,飄過東洋大海,到西牛賀洲地界,徑至南贍部洲,學成人像,著此衣,穿此履,擺擺搖搖,雲遊了八九年餘,更不曾有道。又渡西洋大海,到西牛賀洲地界,訪問多時,幸遇一老祖,傳了我與天同壽的真功果,不死長生的大法門。」眾猴稱賀,都道:「萬劫難逢也!」悟空又笑道:「小的們,又喜我這一門皆有姓氏。」眾猴道:「大王姓甚?」悟空道:「我今姓孫,法名悟空。」眾猴聞說,鼓掌忻然道:「大王是老孫,我們都是二孫、三孫、細孫、小孫一家孫、一國孫、一窩孫矣!」都來奉承老孫,大盆小碗的椰子酒、葡萄酒、仙花、仙果,真個是合家歡樂。咦!
\begin{quote}
貫通一姓身歸本,只待榮遷仙籙名。
\end{quote}

畢竟不知怎生結果,居此界終始如何,且聽下回分解。
