
\chapter{行者假名降怪犼 觀音現像伏妖王}

\begin{quote}
色即空兮自古,空言是色如然。
人能悟徹色空禪。何用丹砂炮煉。
德行全修休懈,工夫苦用熬煎。
有時行滿始朝天。永駐仙顏不變。
\end{quote}

話說那賽太歲緊關了前後門戶,搜尋行者,直嚷到黃昏時分,不見蹤跡。坐在那剝皮亭上,點聚群妖,發號施令,都教各門上提鈴喝號,擊鼓敲梆;一個個弓上絃,刀出鞘,支更坐夜。

原來孫大聖變做個痴蒼蠅,釘在門旁。見前面防備甚緊,他即抖開翅,飛入後宮門首看處,見金聖娘娘伏在御案上,清清滴淚,隱隱聲悲。行者飛進門去,輕輕的落在他那烏雲散髻之上,聽他哭的甚麼。少頃間,那娘娘忽失聲道:「主公啊,我和你:
\begin{quote}
前生燒了斷頭香,今世遭逢潑怪王。
拆鳳三年何日會?分鴛兩處致悲傷。
差來長老才通信,驚散佳姻一命亡。
只為金鈴難解識,相思又比舊時狂。」
\end{quote}

行者聞言,即移身到他耳根後,悄悄的叫道:「聖宮娘娘,你休恐懼。我還是你國差來的神僧孫長老,未曾傷命。只因自家性急,近妝臺偷了金鈴,你與妖王吃酒之時,我卻脫身私出了前亭,忍不住打開看看。不期扯動那塞口的綿花,那鈴響一聲,迸出煙、火、黃沙。我就慌了手腳,把金鈴丟了,現出原身,使鐵棒,苦戰不出,恐遭毒手,故變作一個蒼蠅兒,釘在門樞上,躲到如今。那妖王愈加嚴緊,不肯開門。你可再以夫妻之禮,哄他進來安寢,我好脫身行事,別作區處救你也。」

娘娘一聞此言,戰兢兢,髮似神揪;虛怯怯,心如杵築。淚汪汪的道:「你如今是人是鬼?」行者道:「我也不是人,我也不是鬼,如今變作個蒼蠅兒在此。你休怕,快去請那妖王也。」娘娘不信,淚滴滴,悄語低聲道:「你莫魘寐我。」行者道:「我豈敢魘寐你?你若不信,展開手,等我跳下來你看。」那娘娘真個把左手張開,行者輕輕飛下。落在他玉掌之間,好便似:
\begin{quote}
菡萏蕊頭釘黑豆,牡丹花上歇遊蜂;
繡毬心裡葡萄落,百合枝邊黑點濃。
\end{quote}

金聖宮高擎玉掌,叫聲:「神僧。」行者嚶嚶的應道:「我是神僧變的。」那娘娘方才信了。悄悄的道:「我去請那妖王來時,你卻怎生行事?」行者道:「古人云:『斷送一生惟有酒。』又云:『破除萬事無過酒。』酒之為用多端,你只以飲酒為上。你將那貼身的侍婢喚一個進來,指與我看,我就變作他的模樣,在旁邊伏侍,卻好下手。」

那娘娘真個依言,即叫:「春嬌何在?」那屏風後轉出一個玉面狐狸來,跪下道:「娘娘喚春嬌有何使令?」娘娘道:「你去叫他們來點紗燈,焚腦麝,扶我上前庭,請大王安寢也。」那春嬌即轉前面,叫了七八個怪鹿妖狐,打著兩對燈籠、一對提爐,擺列左右。娘娘欠身叉手,那大聖早已飛去。好行者,展開翅,徑飛到那玉面狐狸頭上,拔下一根毫毛,吹口仙氣,叫:「變!」變作一個瞌睡蟲,輕輕的放在他臉上。原來瞌睡蟲到了人臉上,往鼻孔裡爬,爬進孔中,即瞌睡了。那春嬌果然漸覺困倦,立不住腳,搖樁打盹,即忙尋著原睡處,丟倒頭,只情呼呼的睡起。行者跳下來,搖身一變,變做那春嬌一般模樣,轉屏風,與眾排立不題。

卻說那金聖宮娘娘往前正走,有小妖看見,即報賽太歲道:「大王,娘娘來了。」那妖王急出剝皮亭外迎迓。娘娘道:「大王啊,煙火既息,賊已無蹤,深夜之際,特請大王安置。」那妖滿心歡喜道:「娘娘珍重。卻才那賊乃是孫悟空。他敗了我先鋒,打殺我小校,變化進來,哄了我們。我們這般搜檢,他卻渺無蹤跡,故此心上不安。」娘娘道:「那廝想是走脫了。大王放心勿慮,且自安寢去也。」妖精見娘娘侍立敬請,不敢堅辭,只得吩咐群妖,各要小心火燭,謹防盜賊,遂與娘娘徑往後宮。行者假變春嬌,從兩班侍婢引入。

娘娘叫:「安排酒來與大王解勞。」妖王笑道:「正是,正是。快將酒來,我與娘娘壓驚。」假春嬌即同眾怪鋪排了果品,整頓些腥肉,調開桌椅。那娘娘擎杯,這妖王也以一杯奉上,二人穿換了酒杯。假春嬌在旁,執著酒壺道:「大王、與娘娘今夜才遞交杯盞,請各飲乾,穿個雙喜杯兒。」真個又各斟上,又飲乾了。「假春嬌」又道:「大王娘娘喜會,眾侍婢會唱的供唱,善舞的起舞來耶。」說未畢,只聽得一派歌聲,齊調音律,唱的唱,舞的舞。他兩個又飲了許多,娘娘叫住了歌舞。眾侍婢分班,出屏風外擺列。惟有假春嬌執壺,上下奉酒。娘娘與那妖王專說得是夫妻之話。你看那娘娘一片雲情雨意,哄得那妖王骨軟觔麻。只是沒福,不得沾身。可憐!真是貓咬尿胞——空歡喜。

敘了一會,笑了一會,娘娘問道:「大王,寶貝不曾傷損麼?」妖王道:「這寶貝乃先天摶鑄之物,如何得損?只是被那賊扯開塞口之綿,燒了豹皮包袱也。」娘娘說:「怎生收拾?」妖王道:「不用收拾,我帶在腰間哩。」假春嬌聞得此言,即拔下毫毛一把,嚼得粉碎,輕輕挨近妖王,將那毫毛放在他身上,吹了三口仙氣,暗暗的叫:「變!」那些毫毛即變做三樣惡物,乃虱子、虼蚤、臭蟲,攻入妖王身內,挨著皮膚亂咬。那妖王燥癢難禁,伸手入懷揣摸揉癢,用指頭捏出幾個虱子來,拿近燈前觀看。娘娘見了,含忖道:「大王,想是襯衣髒了,久不曾漿洗,故生此物耳。」妖王慚愧道:「我從來不生此物,可可的今宵出醜。」娘娘笑道:「大王何為出醜?常言道:『皇帝身上也有三個御虱』哩。且脫下衣服來,等我替你捉捉。」妖王真個解帶脫衣。

假春嬌在傍,著意看著那妖王身上衣服,層層皆有虼蚤跳,件件皆排大臭蟲;子母虱密密濃濃,就如螻蟻出窩中。不覺的揭到第三層見肉之處,那金鈴上紛紛垓垓的,也不勝其數。假春嬌道:「大王,拿鈴子來,等我也與你捉捉虱子。」那妖王一則羞,二則慌,卻也不認得真假,將三個鈴兒遞與假春嬌。假春嬌接在手中,賣弄多時,見那妖王低著頭抖這衣服,他即將金鈴藏了,拔下一根毫毛,變作三個鈴兒,一般無二,拿向燈前翻檢。卻又把身子扭扭捏捏的抖了一抖,將那虱子、臭蟲、虼蚤,收了歸在身上,把假金鈴兒遞與那怪。那怪接在手中,一發朦朧無措,那裡認得甚麼真假,雙手托著那鈴兒,遞與娘娘道:「今番你卻收好了,卻要仔細仔細,不要像前一番。」那娘娘接過來,輕輕的揭開衣箱,把那假鈴收了,用黃金鎖鎖了。卻又與妖王飲了幾杯酒,教侍婢:「淨拂牙床,展開錦被,我與大王同寢。」那妖王諾諾連聲道:「沒福,沒福,不敢奉陪。我還帶個宮女往西宮裡睡去,娘娘請自安置。」遂此各歸寢處不題。

卻說假春嬌得了手,將他寶貝帶在腰間,現了本像,把身子抖一抖,收去那個瞌睡蟲兒,徑往前走。只聽得梆鈴齊響,緊打三更。好行者,捏著訣,念動真言,使個隱身法,直至門邊,又見那門上拴鎖甚密。卻就取出金箍棒,望門一指,使出那解鎖之法,那門就輕輕開了。急拽步出門站下,厲聲高叫道:「賽太歲,還我金聖娘娘來。」連叫兩三遍,驚動大小群妖,急急看處,前門開了。即忙掌燈尋鎖,把門兒依然鎖上。著幾個跑入裡邊去報道:「大王,有人在大門外呼喚大王尊號,要金聖娘娘哩。」那裡邊侍婢即出宮門,悄悄的傳言道:「莫吆喝,大王才睡著了。」行者又在門前高叫,那小妖又不敢去驚動。如此者三四遍,俱不敢去通報。那大聖在外嚷嚷鬧鬧的,直弄到天曉。忍不住,手掄著鐵棒,上前打門。慌得那大小群妖頂門的頂門,報信的報信。那妖王一覺方醒,只聞得亂攛攛的諠譁,起身穿了衣服,即出羅帳之外,問道:「嚷甚麼?」眾侍婢才跪下道:「爺爺,不知是甚人在洞外叫罵了半夜,如今卻又打門。」

妖王走出宮門,只見那幾個傳報的小妖慌張張的磕頭道:「外面有人叫罵,要金聖宮娘娘哩;若說半個『不』字,他就說出無數的歪話,甚不中聽。見天曉大王不出,逼得打門也。」那妖道:「且休開門。你去問他是那裡來的?姓甚名誰?快來回報。」小妖急出去,隔門問道:「打門的是誰?」行者道:「我是朱紫國拜請來的外公,來取聖宮娘娘回國哩。」那小妖聽得,即以此言回報。那妖隨往後宮,查問來歷。原來那娘娘才起來,還未梳洗,早見侍婢來報:「爺爺來了。」那娘娘急整衣,散挽黑雲,出宮迎迓。才坐下,還未及問,又聽得小妖來報:「那來的外公已將門打破矣。」那妖笑道:「娘娘,你朝中有多少將帥?」娘娘道:「在朝有四十八衛人馬,良將千員;各邊上元帥總兵,不計其數。」妖王道:「可有個姓外的麼?」娘娘道:「我在宮,只知內裡輔助君王,早晚教誨妃嬪,外事無邊,我怎記得名姓?」妖王道:「這來者稱為『外公』,我想著《百家姓》上,更無個姓外的。娘娘賦性聰明,出身高貴,居皇宮之中,必多覽書籍。記得那本書上有此姓也?」娘娘道:「止《千字文》上有句『外受傅訓』,想必就是此矣。」

妖王喜道:「定是,定是。」即起身辭了娘娘,到剝皮亭上,結束整齊,點出妖兵,開了門,直至外面,手持一柄宣花鉞斧,厲聲高叫道:「那個是朱紫國來的外公?」行者把金箍棒揝在右手,將左手指定道:「賢甥,叫我怎的?」那妖王見了,心中大怒道:「你這廝:
\begin{quote}
相貌若猴子,嘴臉似猢猻。
七分真是鬼,大膽敢欺人。」
\end{quote}

行者笑道:「你這個誑上欺君的潑怪,原來沒眼。想我五百年前大鬧天宮時,九天神將見了我,無一個『老』字,不敢稱呼;你叫我聲外公,那裡虧了你?」妖王喝道:「快早說出姓甚名誰,有些甚麼武藝,敢到我這裡猖獗!」行者道:「你若不問姓名猶可,若要我說出姓名,只怕你立身無地。你上來,站穩著,聽我道:
\begin{quote}
生身父母是天地,日月精華結聖胎。
仙石懷抱無歲數,靈根孕育甚奇哉。
當年產我三陽泰,今日歸真萬會諧。
曾聚眾妖稱帥首,能降眾怪拜丹崖。
玉皇大帝傳宣旨,太白金星捧詔來。
請我上天承職裔,官封弼馬不開懷。
初心造反謀山洞,大膽興兵鬧御階。
托塔天王並太子,交鋒一陣盡猥衰。
金星復奏玄穹帝,再降招安敕旨來。
封做齊天真大聖,那時方稱棟梁材。
又因攪亂蟠桃會,仗酒偷丹惹下災。
太上老君親奏駕,西池王母拜瑤臺。
情知是我欺王法,即點天兵發火牌。
十萬兇星並惡曜,干戈劍戟密排排。
天羅地網漫山布,齊舉刀兵大會垓。
惡鬥一場無勝敗,觀音推薦二郎來。
兩家對敵分高下,他有梅山兄弟儕。
各逞英雄施變化,天門三聖撥雲開。
老君丟了金剛套,眾神擒我到金階。
不須詳允書供狀,罪犯凌遲殺斬災。
斧剁鎚敲難損命,刀掄劍砍怎傷懷。
火燒雷打只如此,無計摧殘長壽胎。
押赴太清兜率院,爐中煅煉盡安排。
日期滿足才開鼎,我向當中跳出來。
手挺這條如意棒,翻身打上玉龍臺。
各星各象皆潛躲,大鬧天宮任我歪。
巡視靈官忙請佛,釋伽與我逞英才。
手心之內翻觔斗,遊遍周天去復來。
佛使先知賺哄法,被他壓住在天崖。
到今五百餘年矣,解脫微軀又弄乖。
特保唐僧西域去,悟空行者甚明白。
西方路上降妖怪,那個妖邪不懼哉!」
\end{quote}

那妖王聽他說出悟空行者,遂道:「你原來是大鬧天宮的那廝。你既脫身保唐僧西去,你走你的路去便罷了,怎麼羅織管事,替那朱紫國為奴,卻到我這裡尋死?」行者喝道:「賊潑怪!說話無知。我受朱紫國拜請之禮,又蒙他稱呼管待之恩,我老孫比那王位還高千倍,他敬之如父母,事之如神明,你怎麼說出『為奴』二字?我把你這誑上欺君之怪,不要走,吃外公一棒。」那妖慌了手腳,即閃身躲過,使宣花斧劈面相迎。這一場好殺!你看:
\begin{quote}
金箍如意棒,風刃宣花斧。一個咬牙發狠兇,一個切齒施威武。這個是齊天大聖降臨凡,那個是作怪妖王來下土。兩個噴雲噯霧照天宮,真是走石揚沙遮斗府。往往來來解數多,翻翻復復金光吐。齊將本事施,各把神通賭。這個要取娘娘轉帝都,那個喜同皇后居山塢。這場都是沒來由,捨死忘生因國主。
\end{quote}

他兩個戰經五十回合,不分勝負。那妖王見行者手段高強,料不能取勝,將斧架住他的鐵棒道:「孫行者,你且住了。我今日還未早膳,待我進了膳,再來與你定雌雄。」行者情知是要取鈴鐺,收了鐵棒道:「『好漢子不趕乏兔兒』。你去,你去,吃飽些,好來領死。」

那妖急轉身闖入裡邊,對娘娘道:「快將寶貝拿來。」娘娘道:「寶貝何幹?」妖王道:「今早叫戰者,乃是取經的和尚之徒,叫做孫悟空行者,假稱外公。我與他戰到此時,不分勝負。等我拿寶貝出去,放些煙火,燒這猴頭。」娘娘見說,心中怛突:欲不取出鈴兒,恐他見疑;欲取出鈴兒,又恐傷了孫行者性命。正自躊躇未定,那妖王又催逼道:「快拿出來。」這娘娘無奈,只得將鎖鑰開了,把三個鈴兒遞與妖王。妖王拿了,就走出洞。娘娘坐在宮中,淚如雨下,思量行者不知可能逃得性命?兩人卻俱不知是假鈴也。

那妖出了門,就占起上風,叫道:「孫行者休走,看我搖搖鈴兒。」行者笑道:「你有鈴,我就沒鈴?你會搖,我就不會搖?」妖王道:「你有甚麼鈴兒?拿出來我看。」行者將鐵棒捏做個繡花針兒,藏在耳內。卻去腰間解下三個真寶貝來,對妖王說:「這不是我的紫金鈴兒?」妖王見了,心驚道:「蹺蹊,蹺蹊!他的鈴兒怎麼與我的鈴兒就一般無二?縱然是一個模子鑄的,好道打磨不到,也有多個瘢兒,少個蒂兒,卻怎麼這等一毫不差?」又問:「你那鈴兒是那裡來的?」行者道:「賢甥,你那鈴兒卻是那裡來的?」妖王老實,便就說道:「我這鈴兒是:
\begin{quote}
太清仙君道源深,八卦爐中久煉金。
結就鈴兒稱至寶,老君留下到如今。
\end{quote}

行者笑道:「老孫的鈴兒,也是那時來的。」妖王道:「怎生出處?」行者道:「我這鈴兒是:
\begin{quote}
道祖燒丹兜率宮,金鈴摶煉在爐中。
二三如六循環寶,我的雌來你的雄。」
\end{quote}

妖王道:「鈴兒乃金丹之寶,又不是飛禽走獸,如何辨得雌雄?但只是搖出寶來,就是好的。」行者道:「口說無憑,做出便見。且讓你先搖。」

那妖王真個將頭一個鈴兒幌了三幌,不見火出;第二個幌了三幌,不見煙出;第三個幌了三幌,也不見沙出。妖王慌了手腳道:「怪哉,怪哉!世情變了,這鈴兒想是懼內,雄見了雌,所以不出來了。」行者道:「賢甥,住了手,等我也搖搖你看。」好猴子,一把揝了三個鈴兒,一齊搖起。你看那紅火、青煙、黃沙,一齊滾出,骨都都燎樹燒山。大聖口裡又念個咒語,望巽地上叫:「風來!」真個是風催火勢,火挾風威,紅焰焰,黑沉沉,滿天煙火,遍地黃沙。把那賽太歲唬得魄散魂飛,走頭無路,在那火當中,怎逃性命?

只聞得半空中厲聲高叫:「孫悟空,我來也。」行者急回頭上望,原來是觀音菩薩,左手托著淨瓶,右手拿著楊柳,灑下甘露救火哩。慌得行者把鈴兒藏在腰間,即合掌倒身下拜。那菩薩將柳枝連拂幾點甘露,霎時間,煙火俱無,黃沙絕跡。行者叩頭道:「不知大慈臨凡,有失迴避。敢問菩薩何往?」菩薩道:「我特來收尋這個妖怪。」行者道:「這怪是何來歷,敢勞金身下降收之?」菩薩道:「他是我跨的個金毛犼。因牧童盹睡,失於防守,這孽畜咬斷鐵索走來,卻與朱紫國王消災也。」行者聞言,急欠身道:「菩薩反說了,他在這裡欺君騙后,敗俗傷風,與那國王生災,卻說是消災,何也?」菩薩道:「你不知之。當時朱紫國先王在位之時,這個王還做東宮太子,未曾登基。他年幼間,極好射獵。他率領人馬,縱放鷹犬,正來到落鳳坡前,有西方佛母孔雀大明王菩薩所生二子,乃雌雄兩個雀雛,停翅在山坡之下,被此王弓開處,射傷了雄孔雀,那雌孔雀也帶箭歸西。佛母懺悔以後,吩咐教他拆鳳三年,身耽啾疾。那時節,我跨著這犼,同聽此言。不期這孽畜留心,故來騙了皇后,與王消災。至今三年,冤愆滿足,幸你來救治王患。我特來收妖邪也。」行者道:「菩薩,雖是這般故事,奈何他玷污了皇后,敗俗傷風,壞倫亂法,卻是該他死罪。今蒙菩薩親臨,饒得他死罪,卻饒不得他活罪。讓我打他二十棒,與你帶去罷。」菩薩道:「悟空,你既知我臨凡,就當看我分上,一發都饒了罷,也算你一番降妖之功;若是動了棍子,他也就是死了。」行者不敢違言,只得拜道:「菩薩既收他回海,再不可令他私降人間,貽害不淺。」

那菩薩才喝了一聲:「孽畜!還不還原,待何時也?」只見那怪打個滾,現了原身,將毛衣抖抖,菩薩騎上。菩薩又望項下一看,不見那三個金鈴。菩薩道:「悟空,還我鈴來。」行者道:「老孫不知。」菩薩喝道:「你這賊猴!若不是你偷了這鈴,莫說一個悟空,就是十個,也不敢近身。快拿出來。」行者笑道:「實不曾見。」菩薩道:「既不曾見,等我念念緊箍兒咒。」那行者慌了,只教:「莫念,莫念。鈴兒在這裡哩。」這正是:犼項金鈴何人解?解鈴人還問繫鈴人。菩薩將鈴兒套在犼項下,飛身高坐。你看他四足蓮花生焰焰,滿身金縷迸森森。大慈悲回南海不題。

卻說孫大聖整束了衣裙,掄鐵棒打進獬豸洞去,把群妖眾怪盡情打死,剿除乾淨。直至宮中,請聖宮娘娘回國。那娘娘頂禮不盡。行者將菩薩降妖並拆鳳原由備說了一遍。尋些軟草,扎了一條草龍,教:「娘娘跨上,合著眼,莫怕,我帶你回朝見主也。」那娘娘謹遵吩咐,行者使起神通,只聽得耳內風響。

半個時辰,帶進城,按落雲頭,叫:「娘娘開眼。」那皇后睜開眼看,認得是鳳閣龍樓,心中歡喜,撇了草龍,與行者同登寶殿。那國王見了,急下龍床,就來扯娘娘玉手,欲訴離情,猛然跌倒在地,只叫:「手疼,手疼。」八戒哈哈大笑道:「嘴臉,沒福消受。一見面就蜇殺了也。」行者道:「獃子,你敢扯他扯兒麼?」八戒道:「就扯他扯兒便怎的?」行者道:「娘娘身上生了毒刺,手上有蜇陽之毒。自到麒麟山,與那賽太歲三年,那妖更不曾沾身。但沾身就害身疼,但沾手就害手疼。」眾官聽說:「似此怎生奈何?」此時外面眾官憂疑,內裡妃嬪悚懼。傍有玉聖、銀聖二宮,將君王扶起。

俱正在倉皇之際,忽聽得那半空中有人叫道:「大聖,我來也。」行者擡頭觀看,只見那:
\begin{quote}
肅肅沖天鶴唳,飄飄徑至朝前。繚繞祥光道道,氤氳瑞氣翩翩。棕衣苫體放雲煙,足踏芒鞋罕見。手執龍鬚蠅帚,絲絛腰下圍纏。乾坤處處結人緣,大地逍遙遊遍。此乃是大羅天上紫雲仙,今日臨凡解魘。
\end{quote}

行者上前迎住道:「張紫陽何往?」紫陽真人直至殿前,躬身施禮道:「大聖,小仙張伯端起手。」行者答禮道:「你從何來?」真人道:「小仙三年前曾赴佛會,因打這裡經過,見朱紫國王有拆鳳之憂,我恐那妖將皇后玷辱,有壞人倫,後日難與國王復合,是我將一件舊棕衣變作一領新霞裳,光生五彩,進與妖王,教皇后穿了裝新。那皇后穿上身,即生一身毒刺。毒刺者,乃棕衣也。今知大聖成功,特來解魘。」行者道:「既如此,累你遠來,且快解脫。」真人走向前,對娘娘用手一指,即脫下那件棕衣。那娘娘遍體如舊。真人將衣抖一抖,披在身上,對行者道:「大聖勿罪,小仙告辭。」行者道:「且住,待君王謝謝。」真人笑道:「不勞,不勞。」遂長揖一聲,騰空而去。慌得那皇帝、皇后及大小眾臣,一個個望空禮拜。

拜畢,即命大開東閣,酬謝四僧。那君王領眾跪拜,夫妻才得重諧。正當歡宴時,行者叫:「師父,拿那戰書來。」長老袖中取出,遞與行者。行者遞與國王道:「此書乃那怪差小校送來者。那小校已先被我打死,送來報功。後復至山中,變作小校,進洞回覆,因得見娘娘,盜出金鈴,幾乎被他拿住。又變化,復偷出,與他對敵。幸遇觀音菩薩將他收去,又與我說拆鳳之故。」從頭至尾,細說了一遍。那舉國君臣內外,無一人不感謝稱讚。唐僧道:「一則是賢王之福,二來是小徒之功。今蒙盛宴,至矣,至矣。就此拜別,不要誤貧僧向西去也。」那國王懇留不得,遂換了關文,大排鑾駕,請唐僧穩坐龍車。那君王、妃后,俱捧轂推輪,相送而別。正是:
\begin{quote}
有緣洗盡憂疑病,絕念無思心自寧。
\end{quote}

畢竟這去,後面再有甚麼吉凶之事,且聽下回分解。
