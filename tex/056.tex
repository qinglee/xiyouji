
\chapter{神狂誅草寇 道迷放心猿}

詩曰:
\begin{quote}
靈臺無物謂之清,寂寂全無一念生。
猿馬牢收休放蕩,精神謹慎莫崢嶸。
除六賊,悟三乘,萬緣都罷自分明。
色邪永滅超真界,坐享西方極樂城。
\end{quote}

話說唐三藏咬釘嚼鐵,以死命留得一個不壞之身,感蒙行者等打死蝎子精,救出琵琶洞。一路無詞,又早是朱明時節。但見那:
\begin{quote}
熏風時送野蘭香,濯雨才晴新竹涼。
艾葉滿山無客採,蒲花盈澗自爭芳。
海榴嬌艷遊蜂喜,溪柳陰濃黃雀狂。
長路那能包角黍,龍舟應弔汨羅江。
\end{quote}

他師徒們行賞端陽之景,虛度中天之節,忽又見一座高山阻路。長老勒馬回頭叫道:「悟空,前面有山,恐又生妖怪,是必謹防。」行者等道:「師父放心。我等皈命投誠,怕甚妖怪?」長老聞言甚喜。加鞭催駿馬,放轡趲蛟龍。須臾,上了山崖,舉頭觀看,真個是:
\begin{quote}
頂巔松柏接雲青,石壁荊榛掛野藤。萬丈崔巍,千層懸削。萬丈崔巍峰嶺峻,千層懸削壑崖深。蒼苔碧蘚鋪陰石,古檜高槐結大林。林深處,聽幽禽,巧聲睍睆實堪吟。澗內水流如瀉玉,路傍花落似堆金。山勢惡,不堪行,十步全無半步平。狐狸糜鹿成雙遇,白鹿玄猿作對迎。忽聞虎嘯驚人膽,鶴鳴振耳透天庭。黃梅紅杏堪供食,野草閑花不識名。
\end{quote}

四眾進山,緩行良久,過了山頭。下西坡,乃是一段平陽之地。豬八戒賣弄精神,教沙和尚挑著擔子,他雙手舉鈀,上前趕馬。那馬更不懼他,憑那獃子嗒笞笞的,還只是緩行不緊。行者道:「兄弟,你趕他怎的?讓他慢慢走罷了。」八戒道:「天色將晚,自上山行了這一日,肚裡餓了,大家走動些,尋個人家,化些齋吃。」行者聞言道:「既如此,等我教他快走。」把金箍棒幌一幌,喝了一聲,那馬溜了韁,如飛似箭,順平路往前去了。

你說馬不怕八戒,只怕行者何也?行者五百年前曾受玉帝封在大羅天御馬監養馬,官名弼馬溫,故此傳留至今,是馬皆懼猴子。

那長老挽不住韁繩,只扳緊著鞍鞽,讓他放了一路轡頭,有二十里向開田地,方才緩步而行。

正走處,忽聽得一棒鑼聲,路兩邊閃出三十多人,一個個槍刀棍棒,攔住路口道:「和尚,那裡?。」諕得個唐僧戰兢兢,坐不穩,跌下馬來,蹲在路傍草科裡,只叫:「大王饒命,大王饒命。」那為頭的兩個大漢道:「不打你,只是有盤纏留下。」長老方才省悟,知他是一夥強人,卻欠身擡頭觀看。但見他:
\begin{quote}
一個青臉獠牙欺太歲,一個暴睛圜眼賽喪門。鬢邊紅髮如飄火,頷下黃鬚似插針。他兩個頭戴虎皮花磕腦,腰繫貂裘彩戰裙。一個手中執著狼牙棒,一個肩上橫擔扢撻藤。果然不亞巴山虎,真個猶如出水龍。
\end{quote}

三藏見他這般兇惡,只得走起來,合掌當胸道:「大王,貧僧是東土唐王差往西天取經者。自別了長安,年深日久,就有些盤纏也使盡了。出家人專以乞化為由,那得個財帛?萬望大王方便方便,讓貧僧過去罷。」那兩個賊帥眾向前道:「我們在這裡起一片虎心,截住要路,專要些財帛,甚麼方便方便?你果無財帛,快早脫下衣服,留下白馬,放你過去。」三藏道:「阿彌陀佛!貧僧這件衣服是東家化布,西家化針,零零碎碎化來的。你若剝去,可不害殺我也?只是這世裡做得好漢,那世裡變畜生哩。」那賊聞言大怒,掣大棍,上前就打。這長老口內不言,心中暗想道:「可憐!你只說你的棍子,還不知我徒弟的棍子哩。」那賊那容分說,舉著棒,沒頭沒臉的打來。長老一生不會說謊,遇著這急難處,沒奈何,只得打個誑語道:「二位大王且莫動手。我有個小徒弟,在後面就到,他身上有幾兩銀子,把與你罷。」那賊道:「這和尚是也吃不得虧,且綑起來。」眾嘍囉一齊下手,把一條繩綑了,高高吊在樹上。

卻說三個撞禍精隨後趕來。八戒呵呵大笑道:「師父去得好快,不知在那裡等我們哩。」忽見長老在樹上,他又說:「你看師父,等便罷了,卻又有這般心腸,爬上樹去,扯著藤兒打鞦韆耍子哩。」行者見了道:「獃子,莫亂談。師父吊在那裡不是?你兩個慢來,等我去看看。」

好大聖,急登高坡細看,認得是夥強人,心中暗喜道:「造化,造化,買賣上門了。」即轉步,搖身一變,變做個乾乾淨淨的小和尚,穿一領緇衣,年紀只有二八,肩上背著一個藍布包袱。拽開步,來到前邊,叫道:「師父,這是怎麼說話?這都是些甚麼歹人?」三藏道:「徒弟呀,還不救我一救,還問甚的?」行者道:「是幹甚勾當的?」三藏道:「這一夥攔路的把我攔住,要買路錢。因身邊無物,遂把我吊在這裡,只等你來計較計較。不然,把這匹馬送與他罷。」行者聞言,笑道:「師父不濟,天下也有和尚,似你這樣皮鬆的卻少。唐太宗差你往西天見佛,誰教你把這龍馬送人?」三藏道:「徒弟呀,似這等吊起來,打著要,怎生是好?」行者道:「你怎麼與他說來?」三藏道:「他打的我急了,沒奈何,把你供出來也。」行者道:「師父,你好沒搭撒。你供我怎的?」三藏道:「我說你身邊有些盤纏,且教他莫打我,是一時救難的話兒。」行者道:「好好好,承你擡舉,正是這樣供。若肯一個月供得七八十遭,老孫越有買賣。」

那夥賊見行者與他師父講話,撒開勢,圍將上來道:「小和尚,你師父說你腰裡有盤纏,趁早拿出來,饒你們性命;若道半個『不』字,就都送了你的殘生。」行者放下包袱道:「列位長官,不要嚷。盤纏有些在此包袱,不多,只有馬蹄金二十來錠,粉面銀二三十錠,散碎的未曾見數。要時就連包兒拿去,切莫打我師父。古書云:『德者,本也;財者,末也。』此是末事。我等出家人自有化處,若遇著個齋僧的長者,襯錢也有,衣服也有,能用幾何?只望放下我師父來,我就一並奉承。」那夥賊聞言,都甚歡喜道:「這老和尚慳吝,這小和尚倒還慷慨。」教:「放下來。」那長老得了性命,跳上馬,顧不得行者,操著鞭,一直跑回舊路。

行者忙叫道:「走錯路了。」提著包袱,就要追去。那夥賊攔住道:「那裡走?將盤纏留下,免得動刑。」行者笑道:「說開,盤纏須三分分之。」那賊頭道:「這小和尚忒乖,就要瞞著他師父留起些兒。也罷,拿出來看,若多時,也分些與你背地裡買果子吃。」行者道:「哥呀,不是這等說,我那裡有甚盤纏?說你兩個打劫別人的金銀,是必分些與我。」那賊聞言大怒,罵道:「這和尚不知死活,你倒不肯與我,反問我要。不要走,看打。」掄起一條扢撻藤棍,照行者光頭上打了七八下。行者只當不知,且滿面陪笑道:「哥呀,若是這等打,就打到來年打罷春也是不當真的。」那賊大驚道:「這和尚好硬頭!」行者笑道:「不敢,不敢,承過獎了,也將就看得過。」那賊那容分說,兩三個一齊亂打。行者道:「列位息怒,等我拿出來。」

好大聖,耳中摸一摸,拔出一個繡花針兒道:「列位,我出家人,果然不曾帶得盤纏,只這個針兒送你罷。」那賊道:「晦氣呀,把一個富貴和尚放了,卻拿住這個窮禿驢。你好道會做裁縫?我要針做甚的?」行者聽說不要,就拈在手中,幌了一幌,變作碗來粗細的一條棍子。那賊害怕道:「這和尚生得小,倒會弄術法兒。」行者將棍子插在地下道:「列位拿得動,就送你罷。」兩個賊上前搶奪,可憐就如蜻蜓撼石柱,莫想弄動半分毫。這條棍本是如意金箍棒,天秤稱的一萬三千五百斤重,那夥賊怎麼知得?大聖走上前,輕輕的拿起,丟一個蟒翻身拗步勢,指著強人道:「你都造化低,遇著我老孫了。」那賊上前來,又打了五六十下。

行者笑道:「你也打得手困了,且讓老孫打一棒兒,卻休當真。」你看他展開棍子幌一幌,有井欄粗細,七八丈長短,盪的一棍,把一個打倒在地,嘴唇土,再不做聲。那一個開言罵道:「這禿廝老大無禮!盤纏沒有,轉傷我一個人。」行者笑道:「且消停,且消停,待我一個個打來,一發教你斷了根罷。」盪的又一棍,把第二個又打死了。諕得那眾嘍囉撇槍棄棍,四路逃生而走。

卻說唐僧騎著馬,往東正跑,八戒、沙僧攔住道:「師父往那裡去?錯走路了。」長老兜馬道:「徒弟啊,趁早去與你師兄說,教他棍下留情,莫要打殺那些強盜。」八戒道:「師父住下,等我去來。」獃子一路跑到前邊,厲聲高叫道:「哥哥,師父教你莫打人哩。」行者道:「兄弟,那曾打人?」八戒道:「那強盜往那裡去了?」行者道:「別個都散了,只是兩個頭兒在這裡睡覺哩。」八戒笑道:「你兩個遭瘟的,好道是熬了夜,這般辛苦,不往別處睡,卻睡在此處。」獃子行到身邊,看看道:「倒與我是一起的,乾淨張著口睡,淌出些粘涎來了。」行者道:「是老孫一棍子打出豆腐來了。」八戒道:「人頭上又有豆腐?」行者道:「打出腦子來了。」

八戒聽說打出腦子來,慌忙跑轉去,對唐僧道:「散了夥也。」三藏道:「善哉!善哉!往那條路上去了?」八戒道:「打也打得直了腳,又會往那裡去走哩?」三藏道:「你怎麼說散夥?」八戒道:「打殺了,不是散夥是甚的?」三藏問:「打的怎麼模樣?」八戒道:「頭上打了兩個大窟窿。」三藏教:「解開包,取幾文襯錢,快去那裡討兩個膏藥,與他兩個貼貼。」八戒笑道:「師父好沒正經,膏藥只好貼得活人的瘡瘇,那裡好貼得死人的窟窿?」三藏道:「真打死了?」就惱起來,口裡不住的絮絮叨叨,猢猻長,猴子短。兜轉馬,與沙僧、八戒至死人前,見那血淋淋的倒臥山坡之下。

這長老甚不忍見,即著八戒:「快使釘鈀,築個坑子埋了,我與他念卷《倒頭經》。」八戒道:「師父左使了人也。行者打殺人,還該教他去燒埋,怎麼教老豬做土工?」行者被師父罵惱了,喝著八戒道:「潑懶夯貨!趁早兒去埋,遲了些兒,就是一棍。」獃子慌了,往山坡下築了有三尺深,下面都是石腳石根,掆住鈀齒。獃子丟了鈀,便把嘴拱。拱到軟處,一嘴有二尺五。兩嘴有五尺深,把兩個賊屍埋了,盤作一個墳堆。三藏叫:「悟空,取香燭來,待我禱祝,好念經。」行者努著嘴道:「好不知趣,這半山之中,前不巴村,後不著店,那討香燭?就有錢也無處去買。」三藏恨恨的道:「猴頭過去,等我撮土焚香禱告。」這是三藏離鞍悲野塚,聖僧善念祝荒墳。祝云:
\begin{quote}
拜惟好漢,聽禱原因:念我弟子,東土唐人。奉太宗皇帝旨意,上西方求取經文。適來此地,逢爾多人,不知是何府、何州、何縣,都在此山內結黨成群。我以好話,哀告慇懃。爾等不聽,反善生嗔。卻遭行者,棍下傷身。切念屍骸暴露,吾隨掩土盤墳。折青竹為香燭,無光彩,有心勤;取頑石作施食,無滋味,有誠真。你到森羅殿下興詞,倒樹尋根,他姓孫,我姓陳,各居異姓。冤有頭,債有主,切莫告我取經僧人。
\end{quote}

八戒笑道:「師父推了乾淨。他打時卻也沒有我們兩個。」三藏真個又撮土禱告道:「好漢告狀,只告行者,也不干八戒、沙僧之事。」

大聖聞言,忍不住笑道:「師父,你老人家忒沒情義。為你取經,我費了多少慇懃勞苦,如今打死這兩個毛賊,你倒教他去告老孫。雖是我動手打,卻也只是為你。你不往西天取經,我不與你做徒弟,怎麼會來這裡,會打殺人?索性等我祝他一祝。」揝著鐵棒,望那墳上搗了三下道:「遭瘟的強盜,你聽著:我被你前七八棍,後七八棍,打得我不疼不癢的,觸惱了性子,一差二誤,將你打死了。盡你到那裡去告,我老孫實是不怕:玉帝認得我,天王隨得我;二十八宿懼我,九曜星官怕我;府縣城隍跪我,東岳天齊怖我;十代閻君曾與我為僕從,五路猖神曾與我當後生。不論三界五司,十方諸宰,都與我情深面熟,隨你那裡去告。」三藏見說出這般惡話,卻又心驚道:「徒弟呀,我這禱祝是教你體好生之德,為良善之人,你怎麼就認真起來?」行者道:「師父,這不是好耍子的勾當。且和你趕早尋宿去。」那長老只得懷嗔上馬。

孫大聖有不睦之心,八戒、沙僧亦有嫉妒之意,師徒都面是背非。依大路向西正走,忽見路北下有一座莊院。三藏用鞭指定道:「我們到那裡借宿去。」八戒道:「正是。」遂行至莊舍邊下馬。看時,卻也好個住場。但見:
\begin{quote}
野花盈徑,雜樹遮扉。遠岸流山水,平畦種麥葵。蒹葭露潤輕鷗宿,楊柳風微倦鳥棲。青柏間松爭翠碧,紅蓬映蓼鬥芳菲。村犬吠,晚雞啼,牛羊食飽牧童歸。爨煙結霧黃粱熟,正是山家入暮時。
\end{quote}

長老向前,忽見那村舍門裡走出一個老者,即與相見,道了問訊。那老者問道:「僧家從那裡來?」三藏道:「貧僧乃東土大唐欽差往西天求經者。適路過寶方,天色將晚,特來檀府告宿一宵。」老者笑道:「你貴處到我這裡,程途迢遞,怎麼涉水登山,獨自到此?」三藏道:「貧僧還有三個徒弟同來。」老者問:「高徒何在?」三藏用手指道:「那大路傍立的便是。」老者猛擡頭,看見他們面貌醜陋,急回身往裡就走。被三藏扯住道:「老施主,千萬慈悲,告借一宿!」老者戰兢兢拑口難言,搖著頭,擺著手道:「不、不、不、不像人模樣!是、是、是幾個妖精。」三藏陪笑道:「施主切休恐懼。我徒弟生得是這等相貌,不是妖精。」老者道:「爺爺呀!一個夜叉,一個馬面,一個雷公。」行者聞言,厲聲高叫道:「雷公是我孫子,夜叉是我重孫,馬面是我玄孫哩。」那老者聽見,魄散魂飛,面容失色,只要進去。三藏攙住他,同到草堂,陪笑道:「老施主,不要怕他,他都是這等粗魯,不會說話。」

正勸解處,只見後面走出一個婆婆,攜著五六歲的一個小孩兒,道:「爺爺,為何這般驚恐?」老者才叫:「媽媽,看茶來。」那婆婆真個丟了孩兒,入裡面捧出二鍾茶來。茶罷,三藏卻轉下來,對婆婆作禮道:「貧僧是東土大唐差往西天取經的。才到貴處,拜求尊府借宿,因是我三個徒弟貌醜,老家長見了虛驚也。」婆婆道:「見貌醜的就這等虛驚,若見了老虎豺狼,卻怎麼好?」老者道:「媽媽呀,人面醜陋還可,只是言語一發嚇人。我說他像夜叉、馬面、雷公,他吆喝道,雷公是他孫子,夜叉是他重孫,馬面是他玄孫。我聽此言,故然悚懼。」唐僧道:「不是,不是。像雷公的,是我大徒孫悟空;像馬面的,是我二徒豬悟能;像夜叉的,是我三徒沙悟淨。他們雖是醜陋,卻也秉教沙門,皈依善果,不是甚麼惡魔毒怪,怕他怎麼?」

公婆兩個聞說他名號,皈正沙門之言,卻才定性回驚,教:「請來,請來。」長老出門叫來,又吩咐道:「適才這老者甚惡你等,今進去相見,切勿抗禮,各要尊重些。」八戒道:「我俊秀,我斯文,不比師兄撒潑。」行者笑道:「不是嘴長、耳大、臉醜,便也是一個好男子。」沙僧道:「莫爭講,這裡不是那抓乖弄俏之處。且進去,且進去。」遂此把行囊、馬匹都到草堂上,齊同唱了個喏,坐定。那媽媽兒賢慧,即便攜轉小兒,咐吩煮飯,安排一頓素齋,他師徒吃了。

漸漸晚了,又掌起燈來,都在草堂上閑敘。長老才問:「施主高姓?」老者道:「姓楊。」又問年紀。老者道:「七十四歲。」又問:「幾位令郎?」老者道:「止得一個。適才媽媽攜的是小孫。」長老請令郎相見拜揖。老者道:「那廝不中拜。老拙命苦,養不著他,如今不在家了。」三藏道:「何方生理?」老者點頭而嘆:「可憐,可憐!若肯何方生理,是吾之幸也。那廝專生惡念,不務本等,專好打家截道,殺人放火。相交的都是些狐群狗黨。自五日之前出去,至今未回。」三藏聞說,不敢言喘,心中暗想道:「或者悟空打殺的就是也。」長老神思不安,欠身道:「善哉!善哉!如此賢父母,何生惡逆兒?」

行者近前道:「老官兒,似這等不良不肖、奸盜邪淫之子,連累父母,要他何用?等我替你尋他來打殺了罷。」老者道:「我待也要送了他,奈何再無以次人丁,縱是不才,一定還留他與老漢掩土。」沙僧與八戒笑道:「師兄,莫管閑事,你我不是官府。他家不肯,與我何干?且告施主,見賜一束草兒,在那廂打鋪睡覺,天明走路。」老者即起身,著沙僧到後園裡拿兩個稻草,教他們在園中草團瓢內安歇。行者牽了馬,八戒挑了行李,同長老俱到團瓢內安歇不題。

卻說那夥賊內果有老楊的兒子。自天早在山前被行者打死兩個賊首,他們都四奔逃生。約摸到四更時候,又結成一夥,在門前打門。老者聽得門響,即披衣道:「媽媽,那廝們來也。」媽媽道:「既來,你去開門,放他來家。」老者方才開門,只見那一夥賊都嚷道:「餓了,餓了。」這老楊的兒子忙入裡面,叫起他妻來,打米煮飯。卻廚下無柴,往後園裡拿柴,到廚房裡問妻道:「後園裡白馬是那裡的?」其妻道:「是東土取經的和尚,昨晚至此借宿,公公、婆婆管待他一頓晚齋,教他在草團瓢內睡哩。」

那廝聞言,走出草堂,拍手打掌笑道:「兄弟們,造化,造化,冤家在我家裡也。」眾賊道:「那個冤家?」那廝道:「卻是打死我們頭兒的和尚,來我家借宿,現睡在草團瓢裡。」眾賊道:「卻好,卻好。拿住這些禿驢,一個個剁成肉醬:一則得那行囊、白馬,二來與我們頭兒報仇。」那廝道:「且莫忙,你們且去磨刀。等我煮飯熟了,大家吃飽些,一齊下手。」真個那些賊磨刀的磨刀,磨槍的磨槍。

那老兒聽得此言,悄悄的走到後園,叫起唐僧四位道:「那廝領眾來了,知得汝等在此,意欲圖害。我老拙念你遠來,不忍傷害。快早收拾行李,我送你往後門出去罷。」三藏聽說,戰兢兢的叩頭謝了老者,即喚八戒牽馬,沙僧挑擔,行者拿了九環錫杖。老者開後門,放他去了,依舊悄悄的來前睡下。

卻說那廝們磨快了刀槍,吃飽了飯食,時已五更天氣,一齊來到園中看處,卻不見了。即忙點燈著火,尋夠多時,四無蹤跡,但見後門開著。都道:「從後門走了,走了。」發一聲喊,趕上來。一個個如飛似箭,直趕到東方日出,卻才望見唐僧。那長老忽聽得喊聲,回頭觀看,後面有二三十人,槍刀簇簇而來。便叫:「徒弟啊,賊兵追至,怎生奈何?」行者道:「放心,放心,老孫了他去來。」三藏勒馬道:「悟空,切莫傷人,只嚇退他便罷。」行者那肯聽信,急掣棒回首相迎道:「列位那裡去?」眾賊罵道:「禿廝無禮!還我大王的命來。」那廝們圈子陣把行者圍在中間,舉槍刀亂砍亂搠。這大聖把金箍棒幌一幌,碗來粗細,把那夥賊打得星落雲散,搪著的就死,挽著的就亡;搕著的骨折,擦著的皮傷;乖些的跑脫幾個,痴些的都見閻王。

三藏在馬上見打倒許多人,慌的放馬奔西。豬八戒與沙和尚緊隨鞭鐙而去。行者問那不死帶傷的賊人道:「那個是那楊老兒的兒子?」那賊哼哼的告道:「爺爺,那穿黃的是。」行者上前,奪過刀來,把個穿黃的割下頭來,血淋淋提在手中,收了鐵棒,拽開雲步,趕到唐僧馬前,提著頭道:「師父,這是楊老兒的逆子,被老孫取將首級來也。」三藏見了,大驚失色,慌得跌下馬來,罵道:「這潑猢猻諕殺我也!快拿過,快拿過。」八戒上前,將人頭一腳踢下路傍,使釘鈀築些土蓋了。

沙僧放下擔子,攙著唐僧道:「師父請起。」那長老在地下正了性,口中念起緊箍兒咒來,把個行者勒得耳紅面赤,眼脹頭昏,在地下打滾,只教:「莫念,莫念。」那長老念夠有十餘遍,還不住口。行者翻觔斗,豎蜻蜓,疼痛難禁,只叫:「師父饒我罪罷,有話便說,莫念,莫念。」三藏卻才住口道:「沒話說,我不要你跟了,你回去罷。」行者忍疼磕頭道:「師父,怎的就趕我去耶?」三藏道:「你這潑猴兇惡太甚,不是個取經之人。昨日在山坡下打死那兩個賊頭,我已怪你不仁。及晚了到老者之家,蒙他賜齋借宿,又蒙他開後門放我等逃了性命。雖然他的兒子不肖,與我無干,也不該就梟他首;況又殺死多人,壞了多少生命,傷了天地多少和氣。屢次勸你,更無一毫善念,要你何為?快走,快走,免得又念真言。」行者害怕,只教:「莫念,莫念,我去也。」說聲去,一路觔斗雲,無影無蹤,遂不見了。咦!這正是:
\begin{quote}
心有兇狂丹不熟,神無定位道難成。
\end{quote}

畢竟不知那大聖投向何方,且聽下分解。
