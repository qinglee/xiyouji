
\chapter{法身元運逢車力 心正妖邪度脊關}

詩曰:
\begin{quote}
求經脫障向西遊,無數名山不盡休。
兔走烏飛催晝夜,鳥啼花落自春秋。
微塵眼底三千界,錫杖頭邊四百州。
宿水餐風登紫陌,未期何日是回頭。
\end{quote}

話說唐三藏幸虧龍子降妖,黑水河神開路,師徒們過了黑水河,找大路一直西來。真個是迎風冒雪,戴月披星。行夠多時,又值早春天氣。但見:
\begin{quote}
三陽轉運,萬物生輝。三陽轉運,滿天明媚開圖畫;萬物生輝,遍地芳菲設繡茵。梅殘數點雪,麥漲一川雲。漸開冰解山泉溜,盡放萌芽沒燒痕。正是那:太昊乘震,勾芒御辰;花香風氣暖,雲淡日光新;道傍楊柳舒青眼,膏雨滋生萬象春。
\end{quote}

師徒們在路上,遊觀景色,緩馬而行,忽聽得一聲吆喝,好便似千萬人吶喊之聲。唐三藏心中害怕,兜住馬不能前進,急回頭道:「悟空,是那裡這等響振?」八戒道:「好一似地裂山崩。」沙僧道:「也就如雷聲霹靂。」三藏道:「還是人喊馬嘶。」孫行者笑道:「你們都猜不著,且住,待老孫看是何如。」

好行者,將身一縱,踏雲光,起在空中,睜眼觀看,遠見一座城池。又近覷,倒也祥光隱隱,不見甚麼凶氣紛紛。行者暗自沉吟道:「好去處,如何有響聲振耳?那城中又無旌旗戈戟,又不是炮聲響振,何以若人馬諠譁?」正疑間,只見那城門外,有一塊沙灘空地,攢簇了許多和尚,在那裡扯車兒哩。原來是一齊著力打號,齊喊「大力王菩薩」,所以驚動唐僧。

行者漸漸按下雲頭來看處,呀!那車子裝的都是磚瓦、木植、土坯之類。灘頭上坡最高,又有一道夾脊小路,兩座大關。關下之路都是直立壁陡之崖,那車兒怎麼拽得上去?雖是天色和暖,那些人卻也衣衫藍縷,看此像十分窘迫。行者心疑道:「想是修蓋寺院,他這裡五穀豐登,尋不出雜工人來,所以這和尚親自努力。」正自猜疑未定,只見那城門裡搖搖擺擺,走出兩個少年道士來。你看他怎生打扮。但見他:
\begin{quote}
頭戴星冠,身披錦繡。頭戴星冠光耀耀,身披錦繡彩霞飄。足踏雲頭履,腰繫熟絲絛。面如滿月多聰俊,形似瑤天仙客嬌。
\end{quote}

那些和尚見道士來,一個個心驚膽戰,加倍著力,恨苦的拽那車子。行者就曉得了:「咦!想必這和尚們怕那道士;不然啊,怎麼這等著力拽扯?我曾聽得人言,西方路上有個敬道滅僧之處,斷乎此間是也。我待要回報師父,奈何事不明白,返惹他怪,道我這等一個伶俐之人,就不能探個實信。且等下去問得明白,好回師父話。」

你道他來問誰?好大聖,按落雲頭,去郡城腳下,搖身一變,變做個遊方的雲水全真,左臂上掛著一個水火籃兒,手敲著漁鼓,口唱著道情詞,近城門,迎著兩個道士,當面躬身道:「道長,貧道起手。」那道士還禮道:「先生那裡來的?」行者道:「我弟子雲遊於海角,浪蕩在天涯。今朝來此處,欲募善人家。動問二位道長:這城中那條街上好道?那個巷裡好賢?我貧道好去化些齋吃。」那道士笑道:「你這先生,怎麼說這等敗興的話?」行者道:「何為敗興?」道士道:「你要化些齋吃,卻不是敗興?」行者道:「出家人以乞化為由,卻不化齋吃,怎生有錢買?」道士笑道:「你是遠方來的,不知我這城中之事。我這城中,且休說文武官員好道,富民長者愛賢,大男小女見我等拜請奉齋,這般都不須掛齒,頭一等就是萬歲君王好道愛賢。」

行者道:「我貧道一則年幼,二則是遠方乍來,實是不知。煩二位道長將這裡地名,君王好道愛賢之事,細說一遍,足見同道之情。」道士說:「此城名喚車遲國。寶殿上君王與我們有親。」行者聞言,呵呵笑道:「想是道士做了皇帝?」他道:「不是。只因這二十年前,民遭亢旱,天無點雨,地絕穀苗,不論君臣黎庶,大小人家,家家沐浴焚香,戶戶拜天求雨。正都在倒懸捱命之處,忽然天降下三個仙長來,俯救生靈。」行者問道:「是那三個仙長?」道士說:「便是我家師父。」行者道:「尊師甚號?」道士云:「我大師父號做虎力大仙,二師父鹿力大仙,三師父羊力大仙。」行者問曰:「三位尊師有多少法力?」道士云:「我那師父呼風喚雨,只在翻掌之間;指水為油,點石成金,卻如轉身之易。所以有這般法力,能奪天地之造化,換星斗之玄微,君臣相敬,與我們結為親也。」

行者道:「這皇帝十分造化。常言道:『術動公卿。』老師父有這般手段,結了親,其實不虧他。噫!不知我貧道可有星星緣法,得見那老師父一面哩?」道士笑曰:「你要見我師父,有何難處?我兩個是他靠胸貼肉的徒弟,我師父卻又好道愛賢,只聽見說個『道』字,就也接出大門,若是我兩個引進你,乃吹灰之力。」行者深深的唱個大喏道:「多承舉薦,就此進去罷。」道士說:「且少待片時,你在這裡坐下,等我兩個把公事幹了來,和你進去。」行者道:「出家人無拘無束,自由自在,有甚公事?」道士用手指定那沙灘上僧人:「他做的是我家生活,恐他躲懶,我們去點他一卯就來。」行者笑道:「道長差了,僧道之輩都是出家人,為何他替我們做活,伏我們點卯?」道士云:「你不知道。因當年求雨之時,僧人在一邊拜佛,道士在一邊告斗,都請朝廷的糧食。誰知那和尚不中用,空念空經,不能濟事。後來我師父一到,喚雨呼風,拔濟了萬民塗炭。卻才發惱了朝廷,說那和尚無用,拆了他的山門,毀了他的佛像,追了他的度牒,不放他回鄉,御賜與我們家做活,就當小廝一般。我家裡燒火的也是他,掃地的也是他,頂門的也是他。因為後邊還有住房,未曾完備,著這和尚來拽磚瓦,拖木植,起蓋房宇。只恐他貪頑躲懶,不肯拽車,所以著我兩個去查點查點。」

行者聞言,扯住道士滴淚道:「我說我無緣,真個無緣,不得見老師父尊面。」道士云:「如何不得見面?」行者道:「我貧道在方上雲遊,一則是為性命,二則也為尋親。」道士問:「你有甚麼親?」行者道:「我有一個叔父,自幼出家,削髮為僧。向日年程饑饉,也來外面求乞。這幾年不見回家,我念祖上之恩,特來順便尋訪。想必是羈遲在此等地方,不能脫身,未可知也。我怎的尋著他,見一面,才可與你進城。」道士云:「這般卻是容易。我兩個且坐下,即煩你去沙灘上替我一查,只點頭目有五百名數目便罷,看內中那個是你令叔。果若有呀,我們看道中情分,放他去了,卻與你進城好麼?」

行者頂謝不盡,長揖一聲,別了道士,敲著漁鼓,徑往沙灘之上。過了雙關,轉下夾脊,那和尚一齊跪下磕頭道:「爺爺,我等不曾躲懶,五百名半個不少,都在此扯車哩。」行者看見,暗笑道:「這些和尚被道士打怕了,見我這假道士就這般悚懼。若是個真道士,好道也活不成了。」行者又搖手道:「不要跪,休怕。我不是監工的,我來此是尋親的。」眾僧們聽說認親,就把他圈子陣圍將上來,一個個出頭露面,咳嗽打響,巴不得要認出去。道:「不知那個是他親哩。」行者認了一會,呵呵笑將起來。眾僧道:「老爺不認親,如何發笑?」行者道:「你們知我笑甚麼?笑你這些和尚全不長俊。父母生下你來,皆因命犯華蓋,妨爺剋娘,或是不招姊妹,才把你捨斷了出家。你怎的不遵三寶,不敬佛法,不去看經拜懺,卻怎麼與道士傭工,作奴婢使喚?」眾僧道:「老爺,你來羞我們哩。你老人家想是個外邊來的,不知我這裡利害。」行者道:「果是外方來的,其實不知你這裡有甚利害。」

眾僧滴淚道:「我們這一國君王偏心無道,只喜得是老爺等輩,惱的是我們佛子。」行者道:「為何來?」眾僧道:「只因呼風喚雨,三個仙長來此處滅了我等,哄信君王,把我們寺拆了,度牒追了,不放歸鄉,亦不許補役當差,賜與那仙長家使用,苦楚難當。但有個遊方道者至此,即請拜王領賞;若是和尚來,不分遠近,就拿來與仙長家傭工。」行者道:「想必那道士還有甚麼巧法術,誘了君王;若只是呼風喚雨,也都是傍門小法術耳,安能動得君心?」眾僧道:「他會摶砂煉汞、打坐存神、指水為油、點石成金;如今興蓋三清觀宇,對天地晝夜看經懺悔,祈君王萬年不老:所以就把君心惑動了。」

行者道:「原來這般。你們都走了便罷。」眾僧道:「老爺,走不脫。那仙長奏准君王,把我們畫了影身圖,四下裡長川張掛。他這車遲國地界也寬,各府州縣鄉村店集之方,都有一張和尚圖,上面是御筆親題。若有官職的,拿得一個和尚,高陞三級;無官職的,拿得一個和尚,就賞白銀五十兩:所以走不脫。且莫說是和尚,就是剪鬃、禿子、毛稀的,都也難逃。四下裡快手又多,緝事的又廣,憑你怎麼也是難脫。我們沒奈何,只得在此苦捱。」

行者道:「既然如此,你們死了便罷。」眾僧道:「老爺,有死的。到處捉來與本處和尚,也共有二千餘眾。到此熬不得苦楚,受不得爊煎,忍不得寒冷,服不得水土,死了有六七百,自盡了有七八百。只有我這五百個不得死。」行者道:「怎麼不得死?」眾僧道:「懸梁繩斷,刀刎不疼;投河的飄起不沉,服藥的身安不損。」行者道:「你卻造化,天賜汝等長壽哩。」眾僧道:「老爺呀,你少了一個字兒,是『長受罪』哩。我等日食三餐,乃是糙米熬得稀粥。到晚就在沙灘上冒露安身。才合眼,就有神人擁護。」行者道:「想是累苦了,見鬼麼?」眾僧道:「不是鬼,乃是六丁六甲、護教伽藍。但至夜,就來保護。但有要死的,就保著,不教他死。」行者道:「這些神卻也沒理。只該教你們早死早生天,卻來保護怎的?」眾僧道:「他在夢寐中勸解我們,教不要尋死,且苦捱著,等那東土大唐聖僧往西天取經的羅漢。他手下有個徒弟,乃齊天大聖,神通廣大,專秉忠良之心,與人間報不平之事,濟困扶危,恤孤念寡。只等他來顯神通,滅了道士,還敬你們沙門禪教哩。」

行者聞得此言,心中暗笑道:「莫說老孫無手段,預先神聖早傳名。」他急抽身,敲著漁鼓,別了眾僧,徑來城門口,見了道士。那道士迎著道:「先生,那一位是令親?」行者道:「五百個都與我有親。」兩個道士笑道:「你怎麼就有許多親?」行者道:「一百個是我左鄰,一百個是我右舍,一百個是我父黨,一百個是我母黨,一百個是我交契。你若肯把這五百人都放了,我便與你進去;不放,我不去了。」道士云:「你想有些風病,一時間就胡說了。那些和尚乃國王御賜,若放一二名,還要在師父處遞了病狀,然後補個死狀,才了得哩,怎麼說都放了?此理不通,不通。且不要說我家沒人使喚,就是朝廷也要怪他。那里長要差官查勘,或時御駕也親來點劄,怎麼敢放?」行者道:「不放麼?」道士說:「不放!」行者連問三聲,就怒將起來,把耳朵裡鐵棒取出,迎風捻了一捻,就碗來粗細,幌了一幌,照道士臉上一刮。可憐就打得頭破血流身倒地,皮開頸折腦漿傾。

那灘上僧人遠遠望見他打殺了兩個道士,丟了車兒,跑將上來道:「不好了,不好了,打殺皇親了。」行者道:「那個是皇親?」眾僧把他簸箕陣圍了,道:「他師父上殿不參王,下殿不辭主,朝廷常稱做『國師兄長先生』。你怎麼到這裡闖禍?他徒弟出來監工,與你無干,你怎麼把他來打死?那仙長不說是你來打殺,只說是來此監工,我們害了他性命,我等怎了?且與你進城去,會了人命出來。」行者笑道:「列位休嚷。我不是雲水全真,我是來救你們的。」眾僧道:「你倒打殺人,害了我們,添了擔兒,如何是救我們的?」行者道:「我是大唐聖僧徒弟孫悟空行者,特特來此救你們性命。」眾僧道:「不是,不是,那老爺我們認得他。」行者道:「又不曾會他,如何認得?」眾僧道:「我們夢中嘗見一個老者,自言太白金星,常教誨我等,說那孫行者的模樣,莫教錯認了。」行者道:「他和你怎麼說來?」眾僧道:「他說那大聖:
\begin{quote}
磕額金睛晃亮,圓頭毛臉無腮。
咨牙尖嘴性情乖。貌比雷公古怪。
慣使金箍鐵棒,曾將天闕攻開。
如今皈正保僧來。專救人間災害。」
\end{quote}

行者聞言,又嗔又喜。喜道:「替老孫傳名!」嗔道:「那老賊憊𪬯,把我的元身都說與這夥凡人。」忽失聲道:「列位誠然認我不是孫行者,我是孫行者的門人,來此處學闖禍耍子的。那裡不是孫行者來了?」用手向東一指,哄得眾僧回頭,他卻現了本相。眾僧們方才認得,一個個倒身下拜道:「爺爺,我等凡胎肉眼,不知是爺爺顯化。望爺爺與我們雪恨消災,早進城降邪從正也。」行者道:「你們且跟我來。」眾僧緊隨左右。

那大聖徑至沙灘上,使個神通,將車兒拽過兩關,穿過夾脊,提起來,摔得粉碎。把那些磚瓦、木植,盡拋下坡坂。喝教眾僧:「散,莫在我手腳邊。等我明日見這皇帝,滅那道士。」眾僧道:「爺爺呀!我等不敢遠走,但恐在官人拿住解來,卻又吃打發贖,反又生災。」行者道:「既如此,我與你個護身法兒。」好大聖,把毫毛拔了一把,嚼得粉碎,每一個和尚與他一截。都教他:「捻在無名指甲裡,捻著拳頭,只情走路。無人敢拿你便罷;若有人拿你,攢緊了拳頭,叫一聲齊天大聖,我就來護你。」眾僧道:「爺爺,倘若去得遠了,看不見你,叫你不應,怎麼是好?」行者道:「你只管放心,就是萬里之遙,可保全無事。」眾僧有膽量大的,捻著拳頭,悄悄的叫聲:「齊天大聖!」只見一個雷公站在面前,手執鐵棒,就是千軍萬馬,也不能近身。此時有百十眾齊叫,足有百十個大聖護持。眾僧叩頭道:「爺爺,果然靈顯。」行者又吩咐:「叫聲『寂』字,還你收了。」真個是叫聲「寂」,依然還是毫毛在那指甲縫裡。眾和尚卻才歡喜逃生,一齊而散。行者道:「不可十分遠遁,聽我城中消息,但有招僧榜出,就進城還我毫毛也。」五百個和尚東的東,西的西,走的走,立的立,四散不題。

卻說那唐僧在路傍等不得行者回話,教豬八戒引馬投西,遇著些僧人奔走。將近城邊,見行者還與十數個未散的和尚在那裡。三藏勒馬道:「悟空,你怎麼來打聽個響聲,許久不回?」行者引了十數個和尚,對唐僧馬前施禮,將上項事說了一遍。三藏大驚道:「這般啊,我們怎了?」那十數個和尚道:「老爺放心,孫大聖爺爺乃天神降的,神通廣大,定保老爺無虞。我等是這城裡敕建智淵寺內僧人。因這寺是先王太祖御造的,現有先王太祖神像在內,未曾拆毀。城中寺院,大小盡皆拆了。我等請老爺趕早進城,到我荒山安下,待明日早朝,孫大聖必有處置。」行者道:「汝等說得是,也罷,趁早進城去來。」

那長老卻才下馬,行到城門之下。此時已太陽西墜。過吊橋,進了三層門裡,街上人見智淵寺的和尚牽馬挑包,盡皆迴避。正行時,卻到山門前。但見那門上高懸著一面金字大匾,乃「敕建智淵寺」。眾僧推開門,穿過金剛殿,把正殿門開了。唐僧取袈裟披起,拜畢金身,方入。眾僧叫:「看家的。」老和尚走出來,看見行者,就拜道:「爺爺,你來了?」行者道:「你認得我是那個爺爺,就是這等呼拜?」那和尚道:「我認得你是齊天大聖孫爺爺。我們夜夜夢中見你。太白金星常常來託夢,說道只等你來,我們才得性命。今日果見尊顏與夢中無異。爺爺呀!喜得早來;再遲一兩日,我等俱做鬼矣。」行者笑道:「請起,請起。明日就有分曉。」眾僧安排齋飯,他師徒們吃了。打掃乾淨方丈,安寢一宿。

二更時候,孫大聖心中有事,偏睡不著,只聽得那裡吹打。悄悄的爬起來,穿了衣服,跳在空中觀看,原來是正南上燈燭熒煌。低下雲頭仔細再看,卻是三清觀道士禳星哩。但見那:
\begin{quote}
靈區高殿,福地真堂。靈區高殿,巍巍壯似蓬壺景;福地真堂,隱隱清如化樂宮。兩邊道士奏笙簧,正面高公擎玉簡。宣理消災懺,開講《道德經》。揚塵幾度盡傳符,表白一番皆俯伏。咒水發檄,燭焰飄搖沖上界;查罡佈斗,香煙馥郁透清霄。案頭有供獻新鮮,桌上有齋筵豐盛。
\end{quote}

殿門前掛一聯黃綾織錦的對句,繡著二十二個大字云:「雨順風調,願祝天尊無量法;河清海晏,祈求萬歲有餘年。」行者見三個老道士披了法衣,想是那虎力、鹿力、羊力大仙。下面有七八百個散眾司鼓司鐘、侍香表白,盡都侍立兩邊。行者暗自喜道:「我欲下去與他混一混,奈何單絲不線,孤掌難鳴。且回去照顧八戒、沙僧,一同來耍耍。」

按落祥雲,徑至方丈中。原來八戒與沙僧通腳睡著。行者先叫悟淨,沙和尚醒來道:「哥哥,你還不曾睡哩?」行者道:「你且起來,我和你受用些來。」沙僧道:「半夜三更,口枯眼澀,有甚受用?」行者道:「這城裡果有一座三清觀,觀裡道士們修蘸,三清殿上有許多供養:饅頭足有斗大,燒果有五六十斤一個,襯飯無數,果品新鮮。和你受用去來。」那豬八戒睡夢裡聽見說吃好東西,就醒了,道:「哥哥,就不帶挈我些兒?」行者道:「兄弟,你要吃東西,不要大呼小叫,驚醒了師父,都跟我去。」

他兩個套上衣服,悄悄的走出門前,隨行者踏了雲頭,跳將起去。那獃子看見燈光,就要下手。行者扯住道:「且休忙,待他散了,方可下去。」八戒道:「他才念到興頭上,卻怎麼肯散?」行者道:「等我弄個法兒,他就散了。」好大聖,捻著訣,念個咒語,往巽地上吸一口氣,呼的吹去,便是一陣狂風,徑直捲進那三清殿上,把他些花瓶、燭臺,四壁上懸掛的功德,一齊刮倒,遂而燈火無光。眾道士心驚膽戰。虎力大仙道:「徒弟們且散。這陣神風所過,吹滅了燈燭香花。各人歸寢,明朝早起,多念幾卷經文補數。」眾道士果各退回。

這行者卻引八戒、沙僧,按落雲頭,闖上三清殿。獃子不論生熟,拿過燒果來,張口就啃。行者掣鐵棒,著手便打。八戒縮手躲過道:「還不曾嘗著甚麼滋味,就打。」行者道:「莫要小家子行,且敘禮坐下受用。」八戒道:「不羞,偷東西吃,還要敘禮。若是請將來,卻要如何?」行者道:「這上面坐的是甚麼菩薩?」八戒笑道:「三清也認不得,卻認做甚麼菩薩。」行者道:「那三清?」八戒道:「中間的是元始天尊,左邊的是靈寶道君,右邊的是太上老君。」行者道:「都要變得這般模樣,才吃得安穩哩。」那獃子急了,聞得那香噴噴供養,要吃,爬上高臺,把老君一嘴拱下去道:「老官兒,你也坐得夠了,讓我老豬坐坐。」八戒變做太上老君,行者變做元始天尊,沙僧變作靈寶道君。把原像都推下去。

及坐下時,八戒就搶大饅頭吃。行者道:「莫忙哩。」八戒道:「哥哥,變得如此,還不吃等甚?」行者道:「兄弟呀,吃東西事小,泄漏天機事大。這聖像都推在地下,倘有起早的道士來撞鐘掃地,或絆一個根頭,卻不走漏消息?你把他藏過一邊來。」八戒道:「此處路生,摸門不著,卻那裡藏他?」行者道:「我才進來時,那右手下有一重小門兒,那裡面穢氣畜人,想必是個五穀輪迴之所。你把他送在那裡去罷。」

這獃子有些夯力量,跳下來,把三個聖像拿在肩膊上,扛將出來。到那廂,用腳登開門看時,原來是個大東廁。笑道:「這個弼馬溫著然會弄嘴弄舌,把個毛坑也與他起個道號,叫做甚麼『五穀輪迴之所』。」那獃子扛在肩上且不丟了去,口裡嘓嘓噥噥的禱道:
\begin{quote}
「三清,三清,我說你聽:遠方到此,慣滅妖精。欲享供養,無處安寧。借你坐位,略略少停。你等坐久,也且暫下毛坑。你平日家受用無窮,做個清淨道士;今日裡不免享些穢物,也做個受臭氣的天尊!」
\end{quote}

祝罷,烹的望裡一捽,灒了半衣襟臭水,走上殿來。

行者道:「可藏得好麼?」八戒道:「藏便藏得好,只是灒起些水來,污了衣服,有些醃臟臭氣,你休惡心。」行者笑道:「也罷,你且來受用。但不知可得個乾淨身子出門哩。」那獃子還變做老君,三人坐下,盡情受用。先吃了大饅頭,後吃簇盤、襯飯、點心、拖爐、餅錠、油煠、蒸酥,那裡管甚麼冷熱,任情吃起。原來孫行者不大吃煙火食,只吃幾個果子,陪他兩個。那一頓如流星趕月,風捲殘雲,吃得罄盡,已此沒得吃了。還不走路,且在那裡閑講,消食耍子。

噫!有這般事。原來那東廊下有一個小道士才睡下,忽然起來道:「我的手鈴兒忘記在殿上,若失落了,明日師父見責。」與那同睡者道:「你睡著,等我尋去。」急忙中不穿底衣,止扯一領直裰,徑到正殿中尋鈴。摸來摸去,鈴兒摸著了。正欲回頭,只聽得有呼吸之聲。道士害怕,急拽步往外走時,不知怎的,屣著一個荔枝核子,撲的滑了一跌。噹的一聲,把個鈴兒跌得粉碎。豬八戒忍不住呵呵大笑出來。把個小道士諕走了三魂,驚回了七魄,一步一跌,撞到那方丈外,打著門叫:「師公,不好了,禍事了。」三個老道士還未曾睡,即開門問:「有甚禍事?」他戰戰兢兢道:「弟子忘失了手鈴兒,因去殿上尋鈴,只聽得有人呵呵大笑,險些兒諕殺我也。」老道士聞言,即叫:「掌燈來,看是甚麼邪物?」一聲傳令,驚動那兩廊的道士,大大小小,都爬起來點燈著火,往正殿上觀看。

不知端的何如,且聽下回分解。
