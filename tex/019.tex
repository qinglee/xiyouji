
\chapter{雲棧洞悟空收八戒 浮屠山玄奘受心經}

卻說那怪的火光前走,這大聖的彩霞隨後。正行處,忽見一座高山,那怪把紅光結聚,現了本相,撞入洞內,取出一柄九齒釘鈀來戰。行者喝一聲道:「潑怪!你是那裡來的邪魔?怎麼知道我老孫的名號?你有甚麼本事,實實供來,饒你性命。」那怪道:「是你也不知我的手段,上前來站穩著,我說與你聽。我:
\begin{quote}
自小生來心性拙,貪閑愛懶無休歇。
不曾養性與修真,混沌迷心熬日月。
忽然閑裡遇真仙,就把寒溫坐下說。
勸我回心莫墮凡,傷生造下無邊孽。
有朝大限命終時,八難三途悔不喋。
聽言意轉要修行,聞語心回求妙訣。
有緣立地拜為師,指示天關並地闕。
得傳九轉大還丹,工夫晝夜無時輟。
上至頂門泥丸宮,下至腳板湧泉穴。
周流腎水入華池,丹田補得溫溫熱。
嬰兒姹女配陰陽,鉛汞相投分日月。
離龍坎虎用調和,靈龜吸盡金烏血。
三花聚頂得歸根,五氣朝元通透徹。
功圓行滿卻飛昇,天仙對對來迎接。
朗然足下彩雲生,身輕體健朝金闕。
玉皇設宴會群仙,各分品級排班列。
敕封元帥管天河,總督水兵稱憲節。
只因王母會蟠桃,開宴瑤池邀眾客。
那時酒醉意昏沉,東倒西歪亂撒潑。
逞雄撞入廣寒宮,風流仙子來相接。
見他容貌挾人魂,舊日凡心難得滅。
全無上下失尊卑,扯住嫦娥要陪歇。
再三再四不依從,東躲西藏心不悅。
色膽如天叫似雷,險些震倒天關闕。
糾察靈官奏玉皇,那日吾當命運拙。
廣寒圍困不通風,進退無門難得脫。
卻被諸神拿住我,酒在心頭還不怯。
押赴靈霄見玉皇,依律問成該處決。
多虧太白李金星,出班俯顖親言說。
改刑重責二千鎚,肉綻皮開骨將折。
放生遭貶出天關,福陵山下圖家業。
我因有罪錯投胎,俗名喚做豬剛鬣。」
\end{quote}

行者聞言道:「你這廝原來是天蓬水神下界,怪道知我老孫名號。」那怪道聲:「哏!你這誑上的弼馬溫,當年撞那禍時,不知帶累我等多少,今日又來此欺人。不要無禮,吃我一鈀。」行者怎肯容情,舉起棒,當頭就打。他兩個在那半山之中,黑夜裡賭鬥。好殺:
\begin{quote}
行者金睛似閃電,妖魔環眼似銀花。這一個口噴彩霧,那一個氣吐紅霞。氣吐紅霞昏處亮,口噴彩霧夜光華。金箍棒,九齒鈀,兩個英雄實可誇:一個是大聖臨凡世,一個是元帥降天涯。那個因失威儀成怪物,這個幸逃苦難拜僧家。鈀去好似龍伸爪,棒迎渾若鳳穿花。那個道:「你破人親事如殺父!」這個道:「你強姦幼女正該拿!」閑言語,亂喧嘩,往往來來棒架鈀。看看戰到天將曉,那妖精兩膊覺酸麻。
\end{quote}

他兩個自二更時分,直戰到東方發白。那怪不能迎敵,敗陣而逃,依然又化狂風,徑回洞裡,把門緊閉,再不出頭。行者在這洞門外看有一座石碣,上書雲棧洞三字。見那怪不出,天又大明,心卻思量:「恐師父等候,且回去見他一見,再來捉此怪不遲。」隨踏雲點一點,早到高老莊。

卻說三藏與那諸老談今論古,一夜無眠。正想行者不來,只見天井裡忽然站下行者。行者收藏鐵棒,整衣上廳。叫道:「師父,我來了。」慌得那諸老一齊下拜,謝道:「多勞,多勞。」三藏問道:「悟空,你去這一夜,拿得妖精在那裡?」行者道:「師父,那妖不是凡間的邪祟,也不是山間的怪獸。他本是天蓬元帥臨凡,只因錯投了胎,嘴臉像一個野豬模樣,其實性靈尚存。他說以相為姓,喚名豬剛鬣。是老孫從後宅裡掣棒就打,他化一陣狂風走了。被老孫著風一棒,他就化道火光,徑轉他那本山洞裡,取出一柄九齒釘鈀,與老孫戰了一夜。適才天色將明,他怯戰而走,把洞門緊閉不出。老孫還要打開那門,與他見個好歹,恐師父在此疑慮盼望,故先來回個信息。」

說罷,那老高上前跪下道:「長老,沒及奈何,你雖趕得去了,他等你去後復來,卻怎區處?索性累你與我拿住,除了根,才無後患。我老夫不敢怠慢,自有重謝:將這家財田地,憑眾親友寫立文書,與長老平分。只是要剪草除根,莫教壞了我高門清德。」行者笑道:「你這老兒不知分限。那怪也曾對我說,他雖是食腸大,吃了你家些茶飯,也與你幹了許多好事,這幾年掙了許多家貲,皆是他之力量。他不曾白吃了你東西,問你祛他怎的?據他說,他是一個天神下界,替你巴家做活,又未曾害了你家女兒。想這等一個女婿,也門當戶對,不怎麼壞了家聲,辱了行止,當真的留他也罷。」老高道:「長老,雖是不傷風化,但名聲不甚好聽,動不動著人就說:『高家招了一個妖怪女婿。』這句話兒教人怎當?」三藏道:「悟空,你既是與他做了一場,一發與他做個結局,才見始終。」行者道:「我才試他一試耍子。此去一定拿來與你們看,且莫憂愁。」叫:「老高,你還好生管待我師父,我去也。」

說聲去,就無形無影的,跳到他那山上,來到洞口,一頓鐵棍,把兩扇門打得粉碎。口裡罵道:「那饢糠的夯貨,快出來與老孫打麼。」那怪正喘噓噓的睡在洞內,聽見打得門響,又聽見罵饢糠的夯貨,他卻惱怒難禁,只得拖著鈀,抖擻精神,跑將出來,厲聲罵道:「你這個弼馬溫,著實憊𪬯。與你有甚相干,你把我大門打破?你且去看看律條,打進大門而入,該個雜犯死罪哩。」行者笑道:「這個獃子!我就打了大門,還有個辨處。像你強占人家女子,又沒個三媒六證,又無些茶紅酒禮,該問個真犯斬罪哩。」那怪道:「且休閑講,看老豬這鈀。」行者使棒支住道:「你這鈀可是與高老家做長工築地種菜的?有何好處怕你?」那怪道:「你錯認了,這鈀豈是凡間之物?你且聽我道來:
\begin{quote}
此是鍛煉神冰鐵,磨琢成工光皎潔。
老君自己動鈐鎚,熒親身添炭屑。
五方五帝用心機,六丁六甲費周折。
造成九齒玉垂牙,鑄就雙環金墜葉。
身妝六曜排五星,體按四時依八節。
短長上下定乾坤,左右陰陽分日月。
六爻神將按天條,八卦星辰依斗列。
名為上寶沁金鈀,進與玉皇鎮丹闕。
因我修成大羅仙,為吾養就長生客。
敕封元帥號天蓬,欽賜釘鈀為御節。
舉起烈焰並毫光,落下猛風飄瑞雪。
天曹神將盡皆驚,地府閻羅心膽怯。
人間那有這般兵,世上更無此等鐵。
隨身變化可心懷,任意翻騰依口訣。
相攜數載未曾離,伴我幾年無日別。
日食三餐並不丟,夜眠一宿渾無撇。
也曾佩去赴蟠桃,也曾帶他朝帝闕。
皆因仗酒卻行兇,只為倚強便撒潑。
上天貶我降凡塵,下世儘我作罪孽。
石洞心邪曾吃人,高莊情喜婚姻結。
這鈀下海掀翻龍鼉窩,上山抓碎虎狼穴。
諸般兵刃且休題,惟有吾當鈀最切。
相持取勝有何難,賭鬥求功不用說。
何怕你銅頭鐵腦一身鋼,鈀到魂消神氣泄。」
\end{quote}

行者聞言,收了鐵棒道:「獃子不要說嘴,老孫把這頭伸在那裡,你且築一下兒,看可能魂消氣泄?」那怪真個舉起鈀,著氣力築將來,撲的一下,鑽起鈀的火光焰焰,更不曾築動一些兒頭皮。諕得他手麻腳軟,道聲:「好頭!好頭!」行者道:「你是也不知。老孫因為鬧天宮,偷了仙丹,盜了蟠桃,竊了御酒,被小聖二郎擒住,押在斗牛宮前,眾天神把老孫斧剁鎚敲,刀砍劍刺,火燒雷打,也不曾損動分毫。又被那太上老君拿了我去,放在八卦爐中,將神火鍛煉,煉做個火眼金睛,銅頭鐵臂。不信,你再築幾下,看看疼與不疼?」那怪道:「你這猴子,我記得你鬧天宮時,家住在東勝神洲傲來國花果山水簾洞裡,到如今久不聞名,你怎麼來到這裡,上門子欺我?莫敢是我丈人去那裡請你來的?」行者道:「你丈人不曾去請我。因是老孫改邪歸正,棄道從僧,保護一個東土大唐駕下御弟,叫做三藏法師,往西天拜佛求經,路過高莊借宿,那高老兒因話說起,就請我救他女兒,拿你這饢糠的夯貨。」

那怪一聞此言,丟了釘鈀,唱個大喏道:「那取經人在那裡?累煩你引見引見。」行者道:「你要見他怎的?」那怪道:「我本是觀世音菩薩勸善,受了他的戒行,這裡持齋把素,教我跟隨那取經人往西天拜佛求經,將功折罪,還得正果。教我等他這幾年,不聞消息。今日既是你與他做了徒弟,何不早說取經之事,只倚兇強,上門打我?」行者道:「你莫詭詐欺心軟我,欲為脫身之計。果然是要保護唐僧,略無虛假,你可朝天發誓,我才帶你去見我師父。」那怪撲的跪下,望空似搗碓的一般,只管磕頭道:「阿彌陀佛,南無佛,我若不是真心實意,還教我犯了天條,劈屍萬段。」行者見他賭咒發願,道:「既然如此,你點把火來燒了你這住處,我方帶你去。」那怪真個搬些蘆葦荊棘,點著一把火,將那雲棧洞燒得像個破瓦窰。對行者道:「我今已無罣礙了,你卻引我去罷。」行者道:「你把釘鈀與我拿著。」那怪就把鈀遞與行者。行者又拔了一根毫毛,吹口仙氣,叫:「變!」即變做一條三股麻繩,走過來,把手背綁剪了。那怪真個倒背著手,憑他怎麼綁縛。卻又揪著耳朵,拉著他,叫:「快走,快走。」那怪道:「輕著些兒,你的手重,揪得我耳根子疼。」行者道:「輕不成,顧你不得。常言道:『善豬惡拿。』只等見了我師父,果有真心,方才放你。」他兩個半雲半霧的,徑轉高家莊來。有詩為證:
\begin{quote}
金性剛強能剋木,心猿降得木龍歸。
金從木順皆為一,木戀金仁總發揮。
一主一賓無間隔,三交三合有玄微。
性情並喜貞元聚,同證西方話不違。
\end{quote}

頃刻間到了莊前。行者拑著他的鈀,揪著他的耳道:「你看那廳堂上端坐的是誰?乃吾師也。」那高氏諸親友與老高,忽見行者把那怪背綁揪耳而來,一個個忻然迎到天井中,道聲:「長老,長老,他正是我家的女婿。」那怪走上前,雙膝跪下,背著手,對三藏叩頭,高叫道:「師父,弟子失迎。早知是師父住在我丈人家,我就來拜接,怎麼又受到許多周折?」三藏道:「悟空,你怎麼降得他來拜我?」行者才放了手,拿釘鈀柄兒打著,喝道:「獃子,你說麼。」那怪把菩薩勸善事情,細陳了一遍。

三藏大喜,便叫:「高太公,取個香案用用。」老高即忙擡出香案。三藏淨了手焚香,望南禮拜道:「多蒙菩薩聖恩。」那幾個老兒也一齊添香禮拜。拜罷,三藏上廳高坐,教悟空放了他繩。行者才把身抖了一抖,收上身來,其縛自解。那怪從新禮拜三藏,願隨西去。又與行者拜了,以先進者為兄,遂稱行者為師兄。三藏道:「既從吾善果,要做徒弟,我與你起個法名,早晚好呼喚。」他道:「師父,我是菩薩已與我摩頂受戒,起了法名,叫做豬悟能也。」三藏笑道:「好,好。你師兄叫做悟空,你叫做悟能,其實是我法門中的宗派。」悟能道:「師父,我受了菩薩戒行,斷了五葷三厭,在我丈人家持齋把素,更不曾動葷。今日見了師父,我開了齋罷。」三藏道:「不可,不可。你既是不吃五葷三厭,我再與你起個別名,喚為八戒。」那獃子歡歡喜喜道:「謹遵師命。」因此又叫做豬八戒。

高老見這等去邪歸正,更十分喜悅,遂命家僮安排筵宴,酬謝唐僧。八戒上前扯住老高道:「爺,請我拙荊出來拜見公公、伯伯,如何?」行者笑道:「賢弟,你既入了沙門,做了和尚,從今後,再莫題起那『拙荊』的話說。世間只有個火居道士,那裡有個火居的和尚?我們且來敘了坐次,吃頓齋飯,趕早兒往西天走路。」高老兒擺了桌席,請三藏上坐;行者與八戒坐於左右兩傍;諸親下坐。高老把素酒開樽,滿斟一杯,奠了天地,然後奉與三藏。三藏道:「不瞞太公說,貧僧是胎裡素,自幼兒不吃葷。」老高道:「因知老師清素,不曾敢動葷。此酒也是素的,請一杯不妨。」三藏道:「也不敢用酒,酒是我僧家第一戒者。」悟能慌了道:「師父,我自持齋,卻不曾斷酒。」悟空道:「老孫雖量窄,吃不上罈把,卻也不曾斷酒。」三藏道:「既如此,你兄弟們吃些素酒也罷,只是不許醉飲誤事。」遂而他兩個接了頭鍾。各人俱照舊坐下,擺下素齋。說不盡那杯盤之盛,品物之豐。

師徒們宴罷,老高將一紅漆丹盤,拿出二百兩散碎金銀,奉三位長老為途中之費;又將三領綿布褊衫為上蓋之衣。三藏道:「我們是行腳僧,遇莊化飯,逢處求齋,怎敢受金銀財帛?」行者近前,掄開手抓了一把,叫:「高才,昨日累你引我師父,今日招了一個徒弟,無物謝你,把這些碎金碎銀,權作帶領錢,拿了去買草鞋穿。以後但有妖精,多作成我幾個,還有謝你處哩。」高才接了,叩頭謝賞。老高又道:「師父們既不受金銀,望將這粗衣笑納,聊表寸心。」三藏又道:「我出家人,若受了一絲之賄,千劫難修。只是把席上吃不了的餅果,帶些去做乾糧足矣。」

八戒在傍邊道:「師父、師兄,你們不要便罷,我與他家做了這幾年女婿,就是掛腳糧也該三石哩。——丈人啊,我的直裰,昨晚被師兄扯破了,與我一件青錦袈裟;鞋子綻了,與我一雙好新鞋子。」高老聞言,不敢不與,隨買一雙新鞋,將一領褊衫,換下舊時衣物。那八戒搖搖擺擺,對高老唱個喏道:「上覆丈母、大姨、二姨並姨夫、姑舅諸親:我今日去做和尚了,不及面辭,休怪。丈人啊,你還好生看待我渾家,只怕我們取不成經時,好來還俗,照舊與你做女婿過活。」行者喝道:「夯貨,卻莫胡說。」八戒道:「不是胡說,只恐一時間有些兒差池,卻不是和尚誤了做,老婆誤了娶,兩下裡都耽擱了?」

三藏道:「少題閑話,我們趕早兒去來。」遂此收拾了一擔行李,八戒擔著;背了白馬,三藏騎著;行者肩擔鐵棒,前面引路。一行三眾,辭別高老及眾親友,投西而去。有詩為證。詩曰:
\begin{quote}
滿地煙霞樹色高,唐朝佛子苦勞勞。
饑餐一缽千家飯,寒著千針一衲袍。
意馬胸頭休放蕩,心猿乖劣莫教嚎。
情和性定諸緣合,月滿金華是伐毛。
\end{quote}

三眾進西路途,有個月平穩。行過了烏斯藏界,猛擡頭見一座高山。三藏停鞭勒馬道:「悟空、悟能,前面山高,須索仔細仔細。」八戒道:「沒事。這山喚做浮屠山,山中有一個烏巢禪師,在此修行,老豬也曾會他。」三藏道:「他有些甚麼勾當?」八戒道:「他倒也有些道行。他曾勸我跟他修行,我不曾去罷了。」師徒們說著話,不多時,到了山上。好山!但見那:
\begin{quote}
山南有青松碧檜,山北有綠柳紅桃。鬧聒聒,山禽對語;舞翩翩,仙鶴齊飛。香馥馥,諸花千樣色;青冉冉,雜草萬般奇。澗下有滔滔綠水,崖前有朵朵祥雲。真個是景致非常幽雅處,寂然不見往來人。
\end{quote}

那師父在馬上遙觀,見香檜樹前有一柴草窩,左邊有麋鹿啣花,右邊有山猴獻果,樹梢頭有青鸞、彩鳳齊鳴,玄鶴、錦雞咸集。八戒指道:「那不是烏巢禪師?」三藏縱馬加鞭,直至樹下。

卻說那禪師見他三眾前來,即便離了巢穴,跳下樹來。三藏下馬奉拜,那禪師用手攙道:「聖僧請起。失迎,失迎。」八戒道:「老禪師,作揖了。」禪師驚問道:「你是福陵山豬剛鬣,怎麼有此大緣,得與聖僧同行?」八戒道:「前年蒙觀音菩薩勸善,願隨他做個徒弟。」禪師大喜道:「好,好,好!」又指定行者,問道:「此位是誰?」行者笑道:「這老禪怎麼認得他,倒不認得我?」禪師道:「因少識耳。」三藏道:「他是我的大徒弟孫悟空。」禪師陪笑道:「欠禮,欠禮。」

三藏再拜:「請問西天大雷音寺還在那裡?」禪師道:「遠哩,遠哩。只是路多虎豹,難行。」三藏慇懃致意,再問:「路途果有多遠?」禪師道:「路途雖遠,終須有到之日,卻只是魔瘴難消。我有《多心經》一卷,凡五十四句,共計二百七十字。若遇魔瘴之處,但念此經,自無傷害。」三藏拜伏於地懇求,那禪師遂口誦傳之。經云:
\begin{quote}
《摩訶般若波羅蜜多心經》:觀自在菩薩,行深般若波羅蜜多,時照見五蘊皆空,度一切苦厄。舍利子,色不異空,空不異色;色即是空,空即是色。受想行識,亦復如是。舍利子,是諸法空相,不生不滅,不垢不淨,不增不減。是故空中無色,無受想行識,無眼耳鼻舌身意,無色聲香味觸法,無眼界,乃至無意識界,無無明,亦無無明盡。乃至無老死,亦無老死盡。無苦寂滅道,無智亦無得。以無所得故,菩提薩埵。依般若波羅蜜多故,心無罣礙;無罣礙故,無有恐怖。遠離顛倒夢想,究竟涅槃。三世諸佛,依般若波羅蜜多故,得阿耨多羅三藐三菩提。故知般若波羅蜜多是大神咒,是大明咒,是無上咒,是無等等咒,能除一切苦,真實不虛。故說般若波羅蜜多咒,即說咒曰:「揭諦揭諦,波羅揭諦,波羅僧揭諦,菩提薩婆訶!」
\end{quote}

此時唐朝法師本有根源,耳聞一遍《多心經》,即能記憶,至今傳世。此乃修真之總經,作佛之會門也。

那禪師傳了經文,踏雲光,要上烏巢而去。被三藏又扯住奉告,定要問個西去的路程端的。那禪師笑云:
\begin{quote}
道路不難行,試聽我吩咐。
千山千水深,多瘴多魔處。
若遇接天崖,放心休恐怖。
行來摩耳巖,側著腳蹤步。
仔細黑松林,妖狐多截路。
精靈滿國城,魔主盈山住。
老虎坐琴堂,蒼狼為主簿。
獅象盡稱王,虎豹皆作御。
野豬挑擔子,水怪前頭遇。
多年老石猴,那裡懷嗔怒。
你問那相識,他知西去路。
\end{quote}

行者聞言,冷笑道:「我們去,不必問他,問我便了。」三藏還不解其意。那禪師化作金光,徑上烏巢而去。長老往上拜謝,行者心中大怒,舉鐵棒望上亂搗,只見蓮花生萬朵,祥霧護千層。行者縱有攪海翻江力,莫想挽著烏巢一縷籐。三藏見了,扯住行者道:「悟空,這樣一個菩薩,你搗他窩巢怎的?」行者道:「他罵了我兄弟兩個一場去了。」三藏道:「他講的西天路徑,何嘗罵你?」行者道:「你那裡曉得?他說『野豬挑擔子』是罵的八戒;『多年老石猴』是罵的老孫。你怎麼解得此意?」八戒道:「師兄息怒。這禪師也曉得過去未來之事,但看他『水怪前頭遇』這句話,不知驗否?饒他去罷。」行者見蓮花祥霧,近那巢邊,只得請師父上馬,下山往西而去。那一去:
\begin{quote}
管教清福人間少,致使災魔山裡多。
\end{quote}

畢竟不知前程端的如何,且聽下回分解。
