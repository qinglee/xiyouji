
\chapter{我佛造經傳極樂 觀音奉旨上長安}

\begin{quote}
試問禪關,參求無數,往往到頭虛老。磨磚作鏡,積雪為糧,迷了幾多年少。毛吞大海,芥納須彌,金色頭陀微笑。悟時超十地三乘,凝滯了四生六道。誰聽得,絕想崖前,無陰樹下,杜宇一聲春曉。曹溪路險,鷲嶺雲深,此處故人音杳。千丈冰崖,五葉蓮開,古殿簾垂香裊。那時節,識破源流,便見龍王三寶。
\end{quote}

這一篇詞,名《蘇武慢》。話表我佛如來辭別了玉帝,回至雷音寶剎。但見那三千諸佛、五百阿羅、八大金剛、無邊菩薩,一個個都執著幢幡寶蓋、異寶仙花,擺列在靈山仙境娑羅雙林之下接迎。如來駕住祥雲,對眾道:「我以甚深般若,遍觀三界。根本性原,畢竟寂滅。同虛空相,一無所有。殄伏乖猴,是事莫識。名生死始,法相如是。」說罷,放舍利之光,滿空有白虹四十二道,南北通連。大眾見了,皈身禮拜。少頃間,聚慶雲彩霧,登上品蓮臺,端然坐下。那三千諸佛、五百羅漢、八金剛、四菩薩,合掌近前禮畢,問曰:「鬧天宮攪亂蟠桃者,何也?」如來道:「那廝乃花果山產的一妖猴,罪惡滔天,不可名狀。概天神將,俱莫能降伏;雖二郎捉獲,老君用火鍛煉,亦莫能傷損。我去時,正在雷將中間揚威耀武,賣弄精神。被我止住兵戈,問他來歷。他言有神通,會變化,又駕觔斗雲,一去十萬八千里。我與他打了個賭賽,他出不得我手,卻將他一把抓住,指化五行山,封壓他在那裡。玉帝大開金闕瑤宮,請我坐了首席,立安天大會謝我,卻方辭駕而回。」大眾聽言喜悅,極口稱揚。

謝罷,各分班而退,各執乃事,共樂天真。果然是:
\begin{quote}
瑞靄漫天竺,虹光擁世尊。西方稱第一,無相法王門。常見玄猿獻果,麋鹿啣花;青鸞舞,彩鳳鳴;靈龜捧壽,仙鶴噙芝。安享淨土祗園,受用龍宮法界。日日花開,時時果熟。習靜歸真,參禪果正。不滅不生,不增不減。煙霞縹緲隨來往,寒暑無侵不記年。
\end{quote}

詩曰:
\begin{quote}
去來自在任優游,也無恐怖也無愁。
極樂場中俱坦蕩,大千之處沒春秋。
\end{quote}

佛祖居於靈山大雷音寶剎之間。一日,喚聚諸佛、阿羅、揭諦、菩薩、金剛、比丘僧尼等眾曰:「自伏乖猿安天之後,我處不知年月,料凡間有半千年矣。今值孟秋望日,我有一寶盆,盆中具設百樣奇花、千般異果等物,與汝等享此盂蘭盆會,如何?」概眾一個個合掌,禮佛三匝領會。如來卻將寶盆中花果品物,著阿儺捧定,著迦葉佈散。大眾感激,各獻詩伸謝。

福詩曰:
\begin{quote}
福星光耀世尊前,福納彌深遠更綿。
福德無疆同地久,福緣有慶與天連。
福田廣種年年盛,福海洪深歲歲堅。
福滿乾坤多福蔭,福增無量永周全。
\end{quote}

祿詩曰:
\begin{quote}
祿重如山彩鳳鳴,祿隨時泰祝長庚。
祿添萬斛身康健,祿享千鍾世太平。
祿俸齊天還永固,祿名似海更澄清。
祿恩遠繼多瞻仰,祿爵無邊萬國榮。
\end{quote}

壽詩曰:
\begin{quote}
壽星獻彩對如來,壽域光華自此開。
壽果滿盤生瑞靄,壽花新採插蓮臺。
壽詩清雅多奇妙,壽曲調音按美才。
壽命延長同日月,壽如山海更悠哉。
\end{quote}

眾菩薩獻畢,因請如來明示根本,指解源流。那如來微開善口,敷演大法,宣揚正果,講的是三乘妙典,五蘊楞嚴。但見那天龍圍繞,花雨繽紛。正是:
\begin{quote}
禪心朗照千江月,真性清涵萬里天。
\end{quote}

如來講罷,對眾言曰:「我觀四大部洲,眾生善惡,各方不一:東勝神洲者,敬天禮地,心爽氣平;北俱盧洲者,雖好殺生,只因糊口,性拙情疏,無多作踐;我西牛賀洲者,不貪不殺,養氣潛靈,雖無上真,人人固壽;但那南贍部洲者,貪淫樂禍,多殺多爭,正所謂口舌兇場,是非惡海。我今有三藏真經,可以勸人為善。」諸菩薩聞言,合掌皈依,向佛前問曰:「如來有那三藏真經?」如來曰:「我有法一藏,談天;論一藏,說地;經一藏,度鬼。三藏共計三十五部,該一萬五千一百四十四卷,乃是修真之經,正善之門。我待要送上東土,叵耐那方眾生愚蠢,毀謗真言,不識我法門之旨要,怠慢了瑜迦之正宗。怎麼得一個有法力的,去東土尋一個善信,教他苦歷千山,詢經萬水,到我處求取真經,永傳東土,勸化眾生,卻乃是個山大的福緣,海深的善慶。誰肯去走一遭來?」當有觀音菩薩行近蓮臺,禮佛三匝道:「弟子不才,願上東土尋一個取經人來也。」諸眾擡頭觀看,那菩薩:
\begin{quote}
理圓四德,智滿金身。纓絡垂珠翠,香環結寶明。烏雲巧疊盤龍髻,繡帶輕飄彩鳳翎。碧玉紐,素羅袍,祥光籠罩;錦絨裙,金落索,瑞氣遮迎。眉如小月,眼似雙星。玉面天生喜,朱脣一點紅。淨瓶甘露年年盛,斜插垂楊歲歲青。解八難,度群生,大慈憫:故鎮太山,居南海,救苦尋聲,萬稱萬應,千聖千靈。蘭心欣紫竹,蕙性愛香藤。他是落伽山上慈悲主,潮音洞裡活觀音。
\end{quote}

如來見了,心中大喜道:「別個是也去不得。須是觀音尊者,神通廣大,方可去得。」菩薩道:「弟子此去東土,有甚言語吩咐?」如來道:「這一去,要踏看路道,不許在霄漢中行。須是要半雲半霧,目過山水,謹記程途遠近之數,叮嚀那取經人。但恐善信難行,我與你五件寶貝。」即命阿儺、迦葉取出錦襴袈裟一領。九環錫杖一根,對菩薩言曰:「這袈裟、錫杖,可與那取經人親用。若肯堅心來此,穿我的袈裟,免墮輪迴;持我的錫杖,不遭毒害。」這菩薩皈依拜領。如來又取出三個箍兒,遞與菩薩道:「此寶喚做緊箍兒,雖是一樣三個,但只是用各不同。我有金緊禁的咒語三篇。假若路上撞見神通廣大的妖魔,你須是勸他學好,跟那取經人做個徒弟。他若不伏使喚,可將此箍兒與他戴在頭上,自然見肉生根。各依所用的咒語念一念,眼脹頭痛,腦門皆裂,管教他入我門來。」

那菩薩聞言,踴躍作禮而退。即喚惠岸行者隨行。那惠岸使一條渾鐵棍,重有千斤,只在菩薩左右作一個降魔的大力士。菩薩遂將錦襴袈裟,作一個包裹,令他背了。菩薩將金箍藏了,執了錫杖,徑下靈山。這一去,有分教:
\begin{quote}
佛子還來歸本願,金蟬長老裹栴檀。
\end{quote}

那菩薩到山腳下,有玉真觀金頂大仙在觀門首接住,請菩薩獻茶。菩薩不敢久停,曰:「今領如來法旨,上東土尋取經人去。」大仙道:「取經人幾時方到?」菩薩道:「未定,約摸二三年間,或可至此。」遂辭了大仙,半雲半霧,約記程途。有詩為證。詩曰:
\begin{quote}
萬里相尋自不言,卻云誰得意難全。
求人忽若渾如此,是我平生豈偶然。
傳道有方成妄說,說明無信也虛傳。
願傾肝膽尋相識,料想前頭必有緣。
\end{quote}

師徒二人正走間,忽然見弱水三千,乃是流沙河界。菩薩道:「徒弟呀,此處卻是難行。取經人濁骨凡胎,如何得渡?」惠岸道:「師父,你看河有多遠?」那菩薩停立雲步看時,只見:
\begin{quote}
東連沙磧,西抵諸番,南達烏戈,北通韃靼。徑過有八百里遙,上下有千萬里遠。水流一似地翻身,浪滾卻如山聳背。洋洋浩浩,漠漠茫茫,十里遙聞萬丈洪。仙槎難到此,蓮葉莫能浮。衰草斜陽流曲浦,黃雲影日暗長堤。那裡得客商來往?何曾有漁叟依棲?平沙無雁落,遠岸有猿啼。只是紅蓼花蘩知景色,白蘋香細任依依。
\end{quote}

菩薩正然點看,只見那河中潑剌一聲響喨,水波裡跳出一個妖魔來,十分醜惡。他生得:
\begin{quote}
青不青,黑不黑,晦氣色臉;長不長,短不短,赤腳筋軀。眼光閃爍,好似灶底雙燈;口角丫叉,就如屠家火缽。獠牙撐劍刃,紅髮亂蓬鬆。一聲叱咤如雷吼,兩腳奔波似滾風。
\end{quote}

那怪物手執一根寶杖,走上岸就捉菩薩,卻被惠岸掣渾鐵棒擋住,喝聲:「休走!」那怪物就持寶杖來迎。兩個在流沙河邊這一場惡殺,真個驚人:
\begin{quote}
木叉渾鐵棒,護法顯神通;怪物降妖杖,努力逞英雄。雙條銀蟒河邊舞,一對神僧岸上沖。那一個威鎮流沙施本事,這一個力保觀音建大功。那一個翻波躍浪,這一個吐霧噴風。翻波躍浪乾坤暗,吐霧噴風日月昏。那個降妖杖,好便似出山的白虎;這個渾鐵棒,卻就如臥道的黃龍。那個使將來,尋蛇撥草;這個丟開去,撲鷂分松。只殺得昏漠漠,星辰燦爛;霧騰騰,天地朦朧。那個久住弱水惟他狠,這個初出靈山第一功。
\end{quote}

他兩個來來往往,戰上數十合,不分勝負。那怪物架住了鐵棒道:「你是那裡和尚,敢來與我抵敵?」木叉道:「我是托塔天王二太子木叉惠岸行者,今保我師父往東土尋取經人去。你是何怪,敢大膽阻路?」那怪方才醒悟道:「我記得你跟南海觀音在紫竹林中修行,你為何來此?」木叉道:「那岸上不是我師父?」

怪物聞言,連聲喏喏,收了寶杖。讓木叉揪了去見觀音,納頭下拜,告道:「菩薩,恕我之罪,待我訴告:我不是妖邪,我是靈霄殿下侍鑾輿的捲簾大將。只因在蟠桃會上失手打碎了玻璃盞,玉帝把我打了八百,貶下界來,變得這般模樣。又叫七日一次,將飛劍來穿我胸脅百餘下方回。故此這般苦惱。沒奈何,饑寒難忍,三二日間,出波濤尋一個行人食用。不期今日無知,衝撞了大慈菩薩。」菩薩道:「你在天有罪,既貶下來,今又這等傷生,正所謂罪上加罪。我今領了佛旨,上東土尋取經人。你何不入我門來,皈依善果,跟那取經人做個徒弟,上西天拜佛求經?我叫飛劍不來穿你。那時節功成免罪,復你本職,心下如何?」那怪道:「我願皈正果。」又向前道:「菩薩,我在此間吃人無數,向來有幾次取經人來,都被我吃了。凡吃的人頭,拋落流沙,竟沉水底。這個水,鵝毛也不能浮。惟有九個取經人的骷髏浮在水面,再不能沉。我以為異物,將索兒穿在一處,閑時拿來頑耍。這去,但恐取經人不得到此,卻不是反誤了我的前程也?」菩薩曰:「豈有不到之理?你可將骷髏兒掛在頭項下,等候取經人,自有用處。」怪物道:「既然如此,願領教誨。」菩薩方與他摩頂受戒,指沙為姓,就姓了沙;起個法名,叫做個沙悟淨。當時入了沙門,送菩薩過了河,他洗心滌慮,再不傷生,專等取經人。

菩薩與他別了,同木叉徑奔東土。行了多時,又見一座高山,山上有惡氣遮漫,不能步上。正欲駕雲過山,不覺狂風起處,又閃上一個妖魔。他生得又甚兇險,但見他:
\begin{quote}
捲臟蓮蓬吊搭嘴,耳如蒲扇顯金睛。
獠牙鋒利如鋼剉,長嘴張開似火盆。
金盔緊繫腮邊帶,勒甲絲絛蟒退鱗。
手執釘鈀龍探爪,腰挎彎弓月半輪。
糾糾威風欺太歲,昂昂志氣壓天神。
\end{quote}

他撞上來,不分好歹,望菩薩舉釘鈀就築。被木叉行者擋住,大喝一聲道:「那潑怪,休得無禮,看棒。」妖魔道:「這和尚不知死活。看鈀。」兩個在山底下一衝一撞,賭鬥輸贏,真個好殺:
\begin{quote}
妖魔兇猛,惠岸威能。鐵棒分心搗,釘鈀劈面迎。播土揚塵天地暗,飛砂走石鬼神驚。九齒鈀,光耀耀,雙環響喨;一條棒,黑悠悠,兩手飛騰。這個是天王太子,那個是元帥精靈。一個在普陀為護法,一個在山洞作妖精。這場相遇爭高下,不知那個虧輸那個贏。
\end{quote}

他兩個正殺到好處,觀世音在半空中拋下蓮花,隔開鈀、杖。怪物見了心驚,便問:「你是那裡和尚,敢弄甚麼眼前花兒哄我?」木叉道:「我把你個肉眼凡胎的潑物!我是南海菩薩的徒弟。這是我師父拋來的蓮花,你也不認得哩!」那怪道:「南海菩薩,可是掃三災救八難的觀世音麼?」木叉道:「不是他是誰?」怪物撇了釘鈀,納頭下禮道:「老兄,菩薩在那裡?累煩你引見一引見。」木叉仰面指道:「那不是?」怪物朝上磕頭,厲聲高叫道:「菩薩,恕罪,恕罪。」

觀音按下雲頭,前來問道:「你是那裡成精的野豕,何方作怪的老彘,敢在此間擋我?」那怪道:「我不是野豕,亦不是老彘,我本是天河裡天蓬元帥。只因帶酒戲弄嫦娥,玉帝把我打了二千鎚,貶下塵凡。一靈真性,徑來奪舍投胎,不期錯了道路,投在個母豬胎裡,變得這般模樣。是我咬殺母豬,打死群彘,在此處占了山場,吃人度日。不期撞著菩薩,萬望拔救拔救。」菩薩道:「此山叫做甚麼山?」怪物道:「叫做福陵山。山中有一洞,叫做雲棧洞。洞裡原有個卵二姐,他見我有些武藝,招我做了家長,又喚做倒蹅門。不上一年,他死了,將一洞的家當,盡歸我受用。在此日久年深,沒有贍身的勾當,只是依本等吃人度日。萬望菩薩恕罪。」菩薩道:「古人云,『若要有前程,莫做沒前程。』你既上界違法,今又不改兇心,傷生造孽,卻不是二罪俱罰?」那怪道:「前程,前程,若依你,教我喝風?常言道:『依著官法打殺,依著佛法餓殺。』去也,去也,還不如捉個行人,肥膩膩的吃他家娘,管甚麼二罪三罪,千罪萬罪!」菩薩道:「『人有善願,天必從之。』汝若肯歸依正果,自有養身之處。世有五穀,可以濟饑,為何吃人度日?」

怪物聞言,似夢方覺,向菩薩道:「我欲從正,奈何『獲罪於天,無所禱也』。」菩薩道:「我領了佛旨,上東土尋取經人。你可跟他做個徒弟,往西天走一遭來,將功折罪,管教你脫離災瘴。」那怪滿口道:「願隨,願隨。」菩薩才與他摩頂受戒,指身為姓,就姓了豬;替他起了法名,就叫做豬悟能。遂此領命歸真,持齋把素,斷絕了五葷三厭,專候那取經人。

菩薩卻與木叉辭了悟能,半興雲霧前來。正走處,只見空中有一條玉龍叫喚。菩薩近前問曰:「你是何龍,在此受罪?」那龍道:「我是西海龍王敖閏之子,因縱火燒了殿上明珠,我父王表奏天庭,告了忤逆。玉帝把我吊在空中,打了三百,不日遭誅。望菩薩搭救搭救。」

觀音聞言,即與木叉撞上南天門裡,早有丘、張二天師接著,問道:「何往?」菩薩道:「貧僧要見玉帝一面。」二天師即忙上奏。玉帝遂下殿迎接。菩薩上前禮畢道:「貧僧領佛旨上東土尋取經人,路遇孽龍懸吊,特來啟奏,饒他性命,賜與貧僧,教他與取經人做個腳力。」玉帝聞言,即傳旨赦宥,差天將解放,送與菩薩。菩薩謝恩而出。這小龍叩頭謝活命之恩,聽從菩薩使喚。菩薩把他送在深澗之中,只等取經人來,變做白馬,上西方立功。小龍領命潛身不題。

菩薩帶引木叉行者過了此山,又奔東土。行不多時,忽見金光萬道,瑞氣千條。木叉道:「師父,那放光之處,乃是五行山了,見有如來的壓帖在那裡。」菩薩道:「此卻是那攪亂蟠桃會、大鬧天宮的齊天大聖,今乃壓在此也。」木叉道:「正是,正是。」師徒俱上山來,觀看帖子,乃是「唵嘛呢叭吽」六字真言。菩薩看罷,嘆惜不已,作詩一首。詩曰:
\begin{quote}
堪嘆妖猴不奉公,當年狂妄逞英雄。
欺心攪亂蟠桃會,大膽私行兜率宮。
十萬軍中無敵手,九重天上有威風。
自遭我佛如來困,何日舒伸再顯功?
\end{quote}

師徒們正說話處,早驚動了那大聖。大聖在山根下高叫道:「是那個在山上吟詩,揭我的短哩?」菩薩聞言,徑下山來尋看。只見那石崖之下,有土地、山神、監押大聖的天將,都來拜接了菩薩,引至那大聖面前。看時,他原來壓於石匣之中,口能言,身不能動。菩薩道:「姓孫的,你認得我麼?」大聖睜開火眼金睛,點著頭兒高叫道:「我怎麼不認得你,你好的是那南海普陀落伽山救苦救難大慈大悲南無觀世音菩薩。承看顧,承看顧。我在此度日如年,更無一個相知的來看我一看。你從那裡來也?」菩薩道:「我奉佛旨,上東土尋取經人去,從此經過,特留殘步看你。」大聖道:「如來哄了我,把我壓在此山,五百餘年了,不能展掙。萬望菩薩方便一二,救我老孫一救。」菩薩道:「你這廝罪業彌深,救你出來,恐你又生禍害,反為不美。」大聖道:「我已知悔了,但願大慈悲指條門路,情願修行。」這才是:
\begin{quote}
人心生一念,天地盡皆知。
善惡若無報,乾坤必有私。
\end{quote}

那菩薩聞得此言,滿心歡喜,對大聖道:「聖經云:『出其言善,則千里之外應之;出其言不善,則千里之外違之。』你既有此心,待我到了東土大唐國尋一個取經的人來,教他救你。你可跟他做個徒弟,秉教迦持,入我佛門,再修正果,如何?」大聖聲聲道:「願去,願去。」菩薩道:「既有善果,我與你起個法名。」大聖道:「我已有名了,叫做孫悟空。」菩薩又喜道:「我前面也有二人歸降,正是『悟』字排行,你今也是『悟』字,卻與他相合,甚好,甚好。這等也不消叮囑,我去也。」那大聖見性明心歸佛教,這菩薩留情在意訪神僧。

他與木叉離了此處,一直東來,不一日就到了長安大唐國。斂霧收雲,師徒們變作兩個疥癩遊僧,入長安城裡,早不覺天晚。行至大市街傍,見一座土地廟祠,二人徑入。諕得那土地心慌,鬼兵膽戰,知是菩薩,叩頭接入。那土地又急跑報與城隍、社令,及滿長安各廟神祇,都知是菩薩,參見告道:「菩薩,恕眾神接遲之罪。」菩薩道:「汝等切不可走漏一毫消息。我奉佛旨,特來此處尋訪取經人。借你廟宇,權住幾日,待訪著真僧即回。」眾神各歸本處,把個土地趕在城隍廟裡暫住,他師徒們隱遁真形。

畢竟不知尋出那個取經人來,且聽下回分解。
