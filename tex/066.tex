
\chapter{諸神遭毒手 彌勒縛妖魔}

話表孫大聖無計可施,縱一朵祥雲,駕觔斗,徑轉南贍部洲去拜武當山,參請蕩魔天尊,解釋三藏、八戒、沙僧、天兵等眾之災。他在半空裡無停止,不一日,早望見祖師仙境,輕輕按落雲頭,定睛觀看,好去處:
\begin{quote}
巨鎮東南,中天神岳。芙蓉峰竦傑,紫蓋嶺巍峨。九江水盡荊揚遠,百越山連翼軫多。上有太虛之寶洞,朱陸之靈臺。三十六宮金磬響,百千萬客進香來。舜巡禹禱,玉簡金書。樓閣飛青鳥,幢幡擺赤裾。地設名山雄宇宙,天開仙境透空虛。幾樹榔梅花正放,滿山瑤草色皆舒。龍潛澗底,虎伏崖中。幽含如訴語,馴鹿近人行。白鶴伴雲棲老檜,青鸞丹鳳向陽鳴。玉虛師相真仙地,金闕仁慈治世門。
\end{quote}

上帝祖師乃淨樂國王與善勝皇后夢吞日光,覺而有孕,懷胎一十四個月,於開皇元年甲辰之歲三月初一日午時降誕於王宮。那爺爺:
\begin{quote}
幼而勇猛,長而神靈。
不統王位,惟務修行。
父母難禁,棄舍皇宮。
參玄入定,在此山中。
功完行滿,白日飛昇。
玉皇敕號,真武之名。
玄虛上應,龜蛇合形。
周天六合,皆稱萬靈。
無幽不察,無顯不成。
劫終劫始,剪伐魔精。
\end{quote}

孫大聖玩著仙境景致,早來到一天門、二天門、三天門。卻至太和宮外,忽見那祥光瑞氣之間,簇擁著五百靈官。那靈官上前迎著道:「那來的是誰?」大聖道:「我乃齊天大聖孫悟空,要見師相。」眾靈官聽說,隨報。祖師即下殿,迎到太和宮。行者作禮道:「我有一事奉勞。」問:「何事?」行者道:「保唐僧西天取經,路遭險難。至西牛賀洲,有座山喚小西天,小雷音寺有一妖魔。我師父進得山門,見有阿羅、揭諦、比丘、聖僧排列,以為真佛,倒身才拜,忽被他拿住綁了。我又失於防閑,被他拋一副金鐃,將我罩在裡面,無纖毫之縫,口合如鉗。甚虧金頭揭諦請奏玉帝,欽差二十八宿,當夜下界,掀揭不起。幸得亢金龍將角透入鐃內,將我度出,被我打碎金鐃,驚醒怪物。趕戰之間,又被撒一個白布搭包兒,將我與二十八宿並五方揭諦,盡皆裝去,復用繩綑了。是我當夜脫逃,救了星辰等眾與我唐僧等。後為找尋衣缽,又驚醒那怪,與天兵趕戰。那怪又拿出搭包兒,理弄之時,我卻知道前音,遂走了。眾等被他依然裝去。我無計可施,特來拜求師相一助力也。」祖師道:「我當年威鎮北方,統攝真武之位,剪伐天下妖邪,乃奉玉帝敕旨。後又披髮跣足,踏騰蛇神龜,領五雷神將、巨虯獅子、猛獸毒龍,收降東北方黑氣妖氛,乃奉元始天尊符召。今日靜享武當山,安逸太和殿,一向海岳平寧,乾坤清泰。奈何我南贍部洲並北俱蘆洲之地,妖魔剪伐,邪鬼潛蹤,今蒙大聖下降,不得不行。只是上界無有旨意,不敢擅動干戈。假若法遣眾神,又恐玉帝見罪;十分卻了大聖,又是我逆了人情。我諒著那西路上縱有妖邪,也不為大害。我今著龜、蛇二將並五大神龍與你助力,管教擒妖精,救你師之難。」

行者拜謝了祖師,即同龜、蛇、龍神各帶精銳之兵,復轉西洲之界。不一日,到了小雷音寺,按下雲頭,徑至山門外叫戰。

卻說那黃眉大王聚眾怪在寶閣下說:「孫行者這兩日不來,又不知往何方去借兵也。」說不了,只見前門上小妖報道:「行者引幾個龍、蛇、龜相,在門外叫戰。」妖魔道:「這猴兒怎麼得個龍、蛇、龜相?此等之類,卻是何方來者?」隨即披掛,走出山門高叫:「汝等是那路龍神,敢來造吾仙境?」五龍、二將相貌崢嶸,精神抖擻,喝道:「那潑怪!我乃武當山太和宮混元教主蕩魔天尊之前五位龍神、龜蛇二將。今蒙齊天大聖相邀,我天尊符召,到此捕你。你這妖精,快送唐僧與天星等出來,免你一死;不然,將這一山之怪碎劈其屍,幾間之房燒為灰燼。」那怪聞言,心中大怒道:「這畜生,有何法力,敢出大言?不要走,吃吾一棒。」這五條龍翻雲使雨,那兩員將播土揚沙,各執槍刀劍戟,一擁而攻;孫大聖又使鐵棒隨後。這一場好殺:
\begin{quote}
兇魔施武,行者求兵。兇魔施武,擅據珍樓施佛像;行者求兵,遠參寶境借龍神。龜蛇生水火,妖怪動刀兵。五龍奉旨來西路,行者因師在後收。劍戟光明搖彩電,槍刀晃亮閃霓虹。這個狼牙棒,強能短軟;那個金箍棒,隨意如心。只聽得扢撲響聲如爆竹,叮噹音韻似敲金。水火齊來征怪物,刀兵共簇繞精靈。喊殺驚狼虎,諠譁振鬼神。渾戰正當無勝處,妖魔又取寶和珍。
\end{quote}

行者帥五龍、二將,與妖魔戰經半個時辰,那妖精即解下搭包在手。行者見了心驚,叫道:「列位仔細。」那龍神、蛇、龜不知甚麼仔細,一個個都停住兵,近前抵擋。那妖精幌的一聲,把搭包兒撇將起去。孫大聖顧不得五龍、二將,駕觔斗,跳在九霄逃脫。他把個龍神、龜、蛇一搭包子又裝將去了。妖精得勝回寺,也將繩綑了,擡在地窖子裡蓋住不題。

你看那大聖落下雲頭,斜欹在山巔之上,沒精沒采,懊恨道:「這怪物十分利害。」不覺的合著眼,似睡一般。猛聽得有人叫道:「大聖,休推睡,快早上緊求救,你師父性命只在須臾間矣。」行者急睜睛跳起來看,原來是日值功曹。行者喝道:「你這毛神,一向在那方貪圖血食,不來點卯,今日卻來驚我。伸過孤拐來,讓老孫打兩棒解悶。」功曹慌忙施禮道:「大聖,你是人間之喜仙,何悶之有?我等早奉菩薩旨令,教我等暗中護佑唐僧,乃同土地等神,不敢暫離左右,是以不得常來參見,怎麼反見責也?」行者道:「你既是保護,如今那眾星、揭諦、伽藍並我師等,被妖精困在何方?受甚罪苦?」功曹道:「你師父、師弟都吊在寶殿廊下,星辰等眾都收在地窖之間受罪。這兩日不聞大聖消息,卻才見妖精又拿了神龍、龜、蛇,又送在地窖裡去了,方知是大聖請來的兵,小神特來尋大聖。大聖莫辭勞倦,千萬再急急去求救援。」

行者聞言及此,不覺對功曹滴淚道:「我如今愧上天宮,羞臨海藏;怕問菩薩之原由,愁見如來之玉像。才拿去者,乃真武師相之龜、蛇、五龍聖眾。教我再無方求救,奈何?」功曹笑道:「大聖寬懷,小神想起一處精兵,請來斷然可降。適才大聖至武當,是南贍部洲之地。這枝兵也在南贍部洲盱眙山蠙城,即今泗洲是也。那裡有個大聖國師王菩薩,神通廣大;他手下有一個徒弟,喚名小張太子,還有四大神將:昔年曾降伏水母娘娘。你今親去請他,他來施恩相助,準可捉怪救師也。」行者心喜道:「你且去保護我師父,勿令傷他,待老孫去請也。」

行者縱起觔斗雲,躲離怪處,直奔盱眙山,不一日早到。細觀,真好去處:
\begin{quote}
南近江津,北臨淮水,東通海嶠,西接封浮。山頂上有樓觀崢嶸,山凹裡有澗泉浩湧。嵯峨怪石,槃秀喬松。百般果品應時新,千樣花枝迎日放。人如蟻陣往來多,船似雁行歸去廣。上邊有瑞巖觀、東岳宮、五顯祠、龜山寺,鍾韻香煙沖碧漢;又有玻璃泉、五塔峪、八仙臺、杏花園,山光樹色映蠙城。白雲橫不度,幽鳥倦還鳴。說甚泰嵩衡華秀,此間仙景若蓬瀛。
\end{quote}

大聖觀玩不盡,徑過了淮河,入蠙城之內,到大聖禪寺山門外。又見那殿宇軒昂,長廊彩麗,有一座寶塔崢嶸。真是:
\begin{quote}
插雲倚漢高千丈,仰視金瓶透碧空。
上下有光凝宇宙,東西無影映簾櫳。
風吹寶鐸聞天樂,日映冰虯對梵宮。
飛宿靈禽時訴語,遙瞻淮水渺無窮。
\end{quote}

行者且觀且走,直至二層門下。那國師王菩薩早已知之,即與小張太子出門迎迓。相見敘禮畢,行者道:「我保唐僧西天取經,路上有個小雷音寺,那裡有個黃眉怪,假充佛祖。我師父不辨真偽,就下拜,被他拿了。又將金鐃把我罩住,幸虧天降星辰救出。是我打碎金鐃,與他賭鬥,又將一個布搭包兒,把天神、揭諦、伽藍與我師父、師弟盡皆裝了進去。我前去武當山請玄天上帝救援,他差五龍、龜、蛇拿怪,又被他一搭包子裝去。弟子無依無倚,故來拜請菩薩,大展威力,將那收水母之神通,拯生民之妙用,同弟子去救師父一難。取得經回,永傳中國,揚我佛之智慧,興般若之波羅也。」國師王道:「你今日之事,誠我佛教之興隆,理當親去。奈時值初夏,正淮水泛漲之時。新收了水猿大聖,那廝遇水即興,恐我去後,他乘空生頑,無神可治。今著小徒領四將和你去助力,煉魔收伏罷。」

行者稱謝,即同四將並小張太子,又駕雲回小西天,直至小雷音寺。小張太子使一條楮白槍,四大將掄四把錕鋘劍,和孫大聖上前罵戰。小妖又去報知,那妖王復帥群妖鼓噪而出道:「猢猻,你今又請得何人來也?」說不了,小張太子指揮四將,上前喝道:「潑妖精!你面上無肉,不認得我等在此?」妖王道:「是那方小將,敢來與他助力?」太子道:「吾乃泗州大聖國師王菩薩弟子,帥領四大神將,奉令擒你。」妖王笑道:「你這孩兒有甚武藝,擅敢到此輕薄?」太子道:「你要知我武藝,等我道來:
\begin{quote}
祖居西土流沙國,我父原為沙國王。
自幼一身多疾苦,命干華蓋惡星妨。
因師遠慕長生訣,有分相逢捨藥方。
半粒丹砂祛病退,願從修行不為王。
學成不老同天壽,容顏永似少年郎。
也曾趕赴龍華會,也曾騰雲到佛堂。
捉霧拿風收水怪,擒龍伏虎鎮山場。
撫民高立浮屠塔,靜海深明舍利光。
楮白槍尖能縛怪,淡緇衣袖把妖降。
如今靜樂蠙城內,大地揚名說小張!」
\end{quote}

妖王聽說,微微冷笑道:「那太子,你捨了國家,從那國師王菩薩,修的是甚麼長生不老之術?只好收捕淮河水怪,卻怎麼聽信孫行者誑謬之言,千山萬水,來此納命?看你可長生可不老也?」

小張聞言,心中大怒,纏槍當面便刺;四大將一擁齊攻;孫大聖使鐵棒上前又打。好妖精,公然不懼,掄著他那短軟狼牙棒,左遮右架,直挺橫衝。這場好殺:
\begin{quote}
小太子,楮白槍,四柄錕鋘劍更強。悟空又使金箍棒,齊心圍繞殺妖王。妖王其實神通大,不懼分毫左右搪。狼牙棒是佛中寶,劍砍槍掄莫可傷。只聽狂風聲吼吼,又觀惡氣混茫茫。那個有意思凡弄本事,這個專心拜佛取經章。幾番馳騁,數次張狂。噴雲霧,閉三光,奮怒懷嗔各不良。多時三乘無上法,致令百藝苦相將。
\end{quote}

概眾爭戰多時,不分勝負。那妖精又解搭包兒。行者又叫:「列位仔細。」太子並眾等不知「仔細」之意。那怪滑的一聲,把四大將與太子,一搭包又裝將進去。只是行者預先知覺走了。那妖王得勝回寺,又教取繩綑了,送在地窖,牢封固鎖不題。

這行者縱觔斗雲,起在空中,見那怪回兵閉門,才按下祥光,立於西山坡上,悵望悲啼道:「師父啊,我
\begin{quote}
自從秉教入禪林,感荷菩薩脫難深。
保你西來求大道,相同輔助上雷音。
只言平坦羊腸路,豈料崔巍怪物侵。
百計千方難救你,東求西告枉勞心。」
\end{quote}

大聖正當悽慘之時,忽見那西南上一朵彩雲墜地,滿山頭大雨繽紛,有人叫道:「悟空,認得我麼?」行者急走前看處,那個人:
\begin{quote}
大耳橫頤方面相,肩查腹滿身軀胖。
一腔春意喜盈盈,兩眼秋波光蕩蕩。
敞袖飄然福氣多,芒鞋灑落精神壯。
極樂場中第一尊,南無彌勒笑和尚。
\end{quote}

行者見了,連忙下拜道:「東來佛祖,那裡去?弟子失迴避了,萬罪,萬罪。」佛祖道:「我此來,專為這小雷音妖怪也。」行者道:「多蒙老爺盛德大恩。敢問那妖是那方怪物,何處精魔?不知他那搭包兒是件甚麼寶貝?煩老爺指示指示。」佛祖道:「他是我面前司磬的一個黃眉童兒。三月三日,我因赴元始會去,留他在宮看守,他把我這幾件寶貝拐出,假佛成精。那搭包兒是我的後天袋子,俗名喚做『人種袋』。那條狼牙棒是個敲磬的槌兒。」行者聽說,高叫一聲道:「好個笑和尚,你走了這童兒,教他誑稱佛祖,陷害老孫,未免有個家法不謹之過。」彌勒道:「一則是我不謹,走失人口;二則是你師徒們魔障未完,故此百靈下界,應該受難。我今來與你收他去也。」

行者道:「這妖精神通廣大,你又無些兵器,何以收之?」彌勒笑道:「我在這山坡下設一草庵,種一田瓜果在此。你去與他索戰,交戰之時許敗不許勝,引他到我這瓜田裡。我別的瓜都是生的,你卻變做一個大熟瓜。他來定要瓜吃,我卻將你與他吃。吃下肚中,任你怎麼在內擺佈他。那時等我取了他的搭包兒,裝他回去。」行者道:「此計雖妙,你卻怎麼認得變的熟瓜?他怎麼就肯跟我來此?」彌勒笑道:「我為治世之尊,慧眼高明,豈不認得你?憑你變作甚物,我皆知之。但恐那怪不肯跟來耳,我卻教你一個法術。」行者道:「他斷然是以搭包兒裝我,怎肯跟來?有何法術可來也?」彌勒笑道:「你伸手來。」行者即舒左手,遞將過去。彌勒將右手食指蘸著口中神水,在行者掌上寫了一個「禁」字,教他捏著拳頭,見妖精當面放手,他就跟來。

行者揝拳,欣然領教。一隻手掄著鐵棒,直至山門外,高叫道:「妖魔,你孫爺爺又來了,可快出來,與你見個上下。」小妖又忙忙奔告。妖王問道:「他又領多少兵來叫戰?」小妖道:「別無甚兵,止他一個。」妖王笑道:「那猴兒計窮力竭,無處求人,斷然是送命來也。」隨又結束整齊,帶了寶貝,舉著那輕軟狼牙棒,走出門來,叫道:「孫悟空,今番掙挫不得了。」行者罵道:「潑怪物,我怎麼掙挫不得?」妖王道:「我見你計窮力竭,無處求人,獨自個強來支持,如今拿住,再沒個甚麼神兵救拔,此所以說你掙挫不得也。」行者道:「這怪不知死活。莫說嘴,吃我一棒。」那妖王見他一隻手掄棒,忍不住笑道:「這猴兒,你看他弄巧,怎麼一隻手使棒支吾?」行者道:「兒子,你禁不得我兩隻手打;若是不使搭包子,再著三五個,也打不過老孫這一隻手。」妖王聞言,道:「也罷,也罷,我如今不使寶貝,只與你實打,比個雌雄。」即舉狼牙棒,上前來鬥。孫行者迎著面,把拳頭一放,雙手掄棒。那妖精著了禁,不思退步,果然不弄搭包,只顧使棒來趕。行者虛幌一下,敗陣就走。那妖精直趕到西山坡下。

行者見有瓜田,打個滾,鑽入裡面,即變做一個大熟瓜,又熟又甜。那妖精停身四望,不知行者那方去了。他卻趕至庵邊叫道:「瓜是誰人種的?」彌勒變作一個種瓜叟,出草庵答道:「大王,瓜是小人種的。」妖王道:「可有熟瓜麼?」彌勒道:「有熟的。」妖王叫:「摘個熟的來,我解渴。」彌勒即把行者變的那瓜,雙手遞與妖王。妖王更不察情,到此接過手,張口便啃。那行者乘此機會,一轂轆鑽入咽喉之下,等不得好歹,就弄手腳:抓腸蒯腹,翻根頭,豎蜻蜓,任他在裡面擺佈。那妖精疼得傞牙徠嘴,眼淚汪汪,把一塊種瓜之地,滾得似個打麥之場。口中只叫:「罷了,罷了,誰人救我一救?」彌勒卻現了本像,嘻嘻笑笑,叫道:「孽畜,認得我麼?」那妖擡頭看見,慌忙跪倒在地,雙手揉著肚子,磕頭撞腦,只叫:「主人公,饒我命罷,饒我命罷,再不敢了。」彌勒上前,一把揪住,解了他的後天袋兒,奪了他的敲磬槌兒。叫:「孫悟空,看我面上,饒他命罷。」

行者十分恨苦,卻又左一拳,右一腳,在裡面亂掏亂搗。那怪萬分疼痛難忍,倒在地下。彌勒又道:「悟空,他也夠了,你饒他罷。」行者才叫:「你張大口,等老孫出來。」那怪雖是肚腹絞痛,還未傷心。俗語云:「人未傷心不得死,花殘葉落是根枯。」他聽見叫張口,即便忍著疼,把口大張。行者方才跳出,現了本像,急掣棒還要打時,早被佛祖把妖精裝在袋裡,斜跨在腰間。手執著磬槌,罵道:「孽畜!金鐃偷了,那裡去了?」那怪卻只要憐生,在後天袋內哼哼嗔嗔的道:「金鐃是孫悟空打破了。」佛祖道:「鐃破,還我金來。」那怪道:「碎金堆在殿蓮臺上哩。」

那佛祖提著袋子,執著磬槌,嘻嘻笑笑,叫道:「悟空,我和你去尋金還我。」行者見此法力,怎敢違誤,只得引佛上山,回至寺內,收取碎金。只見那山門緊閉,佛祖使槌一指門開。入裡看時,那些小妖已得知老妖被擒,各自收拾囊底,都要逃生四散。被行者見一個打一個,見兩個打兩個,把五七百個小妖盡皆打死。各現原身,都是些山精樹怪,獸孽禽魔。佛祖將金收攢一處,吹口仙氣,念聲咒語,即時返本還原,復得金鐃一副。別了行者,駕祥雲,徑轉極樂世界。

這大聖卻才解下唐僧、八戒、沙僧。那獃子吊了幾日,餓得慌了,且不謝大聖,卻就蝦著腰,跑到廚房尋飯吃。原來那怪正安排了午飯,因行者索戰,還未得吃。這獃子看見,即吃了半鍋。卻拿出兩缽頭叫師父、師弟們各吃了兩碗。然後才謝了行者。問及妖怪原由,行者把先請祖師、龜、蛇,後請大聖借太子,並彌勒收降之事,細陳了一遍。三藏聞言,謝之不盡,頂禮了諸天,道:「徒弟,這些神聖,困於何所?」行者道:「昨日日值功曹對老孫說,都在地窖之內。」叫:「八戒,我與你去解脫他等。」

那獃子得食力壯,抖擻精神,尋著他的釘鈀,即同大聖到後面,打開地窖,將眾等解了繩,請出珍樓之下。三藏披了袈裟,朝上一一拜謝。這大聖才送五龍、二將回武當,送小張太子與四將回蠙城,後送二十八宿歸天府,發放揭諦、伽藍各回境。

師徒們卻寬住了半日。喂飽了白馬,收拾行囊,至次早登程。臨行時,放上一把火,將那些珍樓、寶座、高閣、講堂,俱盡燒為灰燼。這裡才:
\begin{quote}
無罣無牽逃難去,消災消障脫身行。
\end{quote}

畢竟不知幾時才到大雷音,且聽下回分解。
