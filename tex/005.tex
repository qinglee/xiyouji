
\chapter{亂蟠桃大聖偷丹 反天宮諸神捉怪}

話表齊天大聖到底是個妖猴,更不知官銜品從,也不較俸祿高低,但只註名便了。那齊天府下二司仙吏,早晚伏侍,只知日食三餐,夜眠一榻,無事牽縈,自由自在。閑時節會友遊宮,交朋結義。見三清稱個「老」字,逢四帝道個「陛下」。與那九曜星、五方將、二十八宿、四大天王、十二元辰、五方五老、普天星相、河漢群神,俱只以弟兄相待,彼此稱呼。今日東遊,明日西蕩,雲去雲來,行蹤不定。

一日,玉帝早朝,班部中閃出許旌陽真人,頫顖啟奏道:「今有齊天大聖無事閑遊,結交天上眾星宿,不論高低,俱稱朋友,恐後閑中生事。不若與他一件事管,庶免別生事端。」玉帝聞言,即時宣詔。那猴王欣欣然而至,道:「陛下,詔老孫有何陞賞?」玉帝道:「朕見你身閑無事,與你件執事:你且權管那蟠桃園,早晚好生在意。」大聖歡喜謝恩,朝上唱喏而退。

他等不得窮忙,即入蟠桃園內查勘。本園中有個土地攔住問道:「大聖何往?」大聖道:「吾奉玉帝點差,代管蟠桃園,今來查勘也。」那土地連忙施禮,即呼那一班鋤樹力士、運水力士、修桃力士、打掃力士都來見大聖磕頭,引他進去。但見那:
\begin{quote}
夭夭灼灼,顆顆株株。夭夭灼灼花盈樹,顆顆株株果壓枝。果壓枝頭垂錦彈;花盈樹上簇胭脂。時開時結千年熟,無夏無冬萬載遲。先熟的,酡顏醉臉;還生的,帶蒂青皮。凝煙肌帶綠,映日顯丹姿。樹下奇葩並異卉,四時不謝色齊齊;左右樓臺並館舍,盈空常見罩雲霓。不是玄都凡俗種,瑤池王母自栽培。
\end{quote}

大聖看玩多時,問土地道:「此樹有多少株數?」土地道:「有三千六百株:前面一千二百株,花微果小,三千年一熟,人吃了成仙了道,體健身輕;中間一千二百株,層花甘實,六千年一熟,人吃了霞舉飛昇,長生不老;後面一千二百株,紫紋緗核,九千年一熟,人吃了與天地齊壽,日月同庚。」大聖聞言,歡喜無任。當日查明了株樹,點看了亭閣,回府。自此後,三五日一次賞玩,也不交友,也不他遊。

一日,見那老樹枝頭,桃熟大半。他心裡要吃個嘗新,奈何本園土地、力士並齊天府仙吏緊隨不便。忽設一計道:「汝等且出門外伺候,讓我在這亭上少憩片時。」那眾仙果退。只見那猴王脫了冠服,爬上大樹,揀那熟透的大桃,摘了許多,就在樹枝上自在受用。吃了一飽,卻才跳下樹來,簪冠著服,喚眾等儀從回府。遲三二日,又去設法偷桃,儘他享用。

一朝,王母娘娘設宴,大開寶閣,瑤池中做蟠桃勝會。即著那紅衣仙女、青衣仙女、素衣仙女、皂衣仙女、紫衣仙女、黃衣仙女、綠衣仙女各頂花籃,去蟠桃園摘桃建會。七衣仙女直至園門首,只見蟠桃園土地、力士同齊天府二司仙吏,都在那裡把門。仙女近前道:「我等奉王母懿旨,到此摘桃設宴。」土地道:「仙娥且住。今歲不比往年了,玉帝點差齊天大聖在此督理,須是報大聖得知,方敢開園。」仙女道:「大聖何在?」土地道:「大聖在園內,因困倦,自家在亭子上睡哩。」仙女道:「既如此,尋他去來,不可遲誤。」土地即與同進。尋至花亭不見,只有衣冠在亭,不知何往,四下裡都沒尋處。原來大聖耍了一會,吃了幾個桃子,變做二寸長的個人兒,在那大樹梢頭濃葉之下睡著了。七衣仙女道:「我等奉旨前來,尋不見大聖,怎敢空回?」傍有仙使道:「仙娥既奉旨來,不必遲疑。我大聖閑遊慣了,想是出園會友去了。汝等且去摘桃,我們替你回話便是。」

那仙女依言,入樹林之下摘桃:先在前樹摘了二籃,又在中樹摘了三籃,到後樹上摘取,只見那樹上花果稀疏,止有幾個毛蒂青皮的。原來熟的都是猴王吃了。七仙女張望東西,只見向南枝上止有一個半紅半白的桃子。青衣女用手扯下枝來,紅衣女摘了,卻將枝子望上一放。

原來那大聖變化了,正睡在此枝,被他驚醒。大聖即現本相,耳朵內掣出金箍棒,幌一幌,碗來粗細,咄的一聲道:「你是那方怪物,敢大膽偷摘我桃。」慌得那七仙女一齊跪下道:「大聖息怒。我等不是妖怪,乃王母娘娘差來的七衣仙女,摘取仙桃,大開寶閣,做蟠桃勝會。適至此間,先見了本園土地等神,尋大聖不見。我等恐遲了王母懿旨,是以等不得大聖,故先在此摘桃。萬望恕罪。」大聖聞言,回嗔作喜道:「仙娥請起。王母開閣設宴,請的是誰?」仙女道:「上會自有舊規,請的是西天佛老、菩薩、聖僧、羅漢,南方南極觀音,東方崇恩聖帝、十洲三島仙翁,北方北極玄靈,中央黃極黃角大仙,這個是五方五老。還有五斗星君,上八洞三清、四帝、太乙天仙等眾,中八洞玉皇、九壘、海嶽神仙,下八洞幽冥教主、注世地仙,各宮各殿大小尊神,俱一齊赴蟠桃嘉會。」大聖笑道:「可請我麼?」仙女道:「不曾聽得說。」大聖道:「我乃齊天大聖,就請我老孫做個席尊,有何不可?」仙女道:「此是上會舊規,今會不知如何。」大聖道:「此言也是,難怪汝等。你且立下,待老孫先去打聽個消息,看可請老孫不請。」

好大聖,捻著訣,念聲咒語,對眾仙女道:「住!住!住!」這原來是個定身法,把那七衣仙女,一個個睖睖睜睜,白著眼,都站在桃樹之下。大聖縱朵祥雲,跳出園內,竟奔瑤池路上而去。正行時,只見那壁廂:
\begin{quote}
一天瑞靄光搖曳,五色祥雲飛不絕。白鶴聲鳴振九皋,紫芝色秀分千葉。中間現出一尊仙,相貌天然丰采別。神舞虹霓幌漢霄,腰懸寶籙無生滅。名稱赤腳大羅仙,特赴蟠桃添壽節。
\end{quote}

那赤腳大仙覿面撞見大聖,大聖低頭定計,賺哄真仙,他要暗去赴會,卻問:「老道何往?」大仙道:「蒙王母見招,去赴蟠桃嘉會。」大聖道:「老道不知。玉帝因老孫觔斗雲疾,著老孫五路邀請列位,先至通明殿下演禮,後方去赴宴。」大仙是個光明正大之人,就以他的誑語作真,道:「常年就在瑤池演禮謝恩,如何先去通明殿演禮,方去瑤池赴會?」無奈,只得撥轉祥雲,徑往通明殿去了。

大聖駕著雲,念聲咒語,搖身一變,就變做赤腳大仙模樣,前奔瑤池。不多時,直至寶閣,按住雲頭,輕輕移步,走入裡面。只見那裡:
\begin{quote}
瓊香繚繞,瑞靄繽紛。瑤臺鋪彩結,寶閣散氤氳。鳳翥鸞翔形縹緲,金花玉萼影浮沉。上排著九鳳丹霞扆,八寶紫霓墩,五彩描金桌,千花碧玉盆。桌上有龍肝和鳳髓,熊掌與猩唇。珍饈百味般般美,異果嘉殽色色新。
\end{quote}

那裡鋪設得齊齊整整,卻還未有仙來。

這大聖點看不盡,忽聞得一陣酒香撲鼻。忽轉頭,見右壁廂長廊之下,有幾個造酒的仙官、盤糟的力士,領幾個運水的道人、燒火的童子,在那裡洗缸刷甕,已造成了玉液瓊漿,香醪佳釀。大聖止不住口角流涎,就要去吃,奈何那些人都在這裡。他就弄個神通,把毫毛拔下幾根,丟入口中嚼碎,噴將出去,念聲咒語,叫:「變!」即變做幾個瞌睡蟲,奔在眾人臉上。你看那夥人,手軟頭低,閉眉合眼,丟了執事,都去盹睡。大聖卻拿了些百味八珍,佳殽異品,走入長廊裡面,就著缸,挨著甕,放開量,痛飲一番。吃勾了多時,酕醄醉了。自揣自摸道:「不好,不好!再過會,請的客來,卻不怪我?一時拿住,怎生是好?不如早回府中睡去也。」

好大聖,搖搖擺擺,仗著酒,任情亂撞。一會把路差了,不是齊天府,卻是兜率天宮。一見了,頓然醒悟道:「兜率宮是三十三天之上,乃離恨天太上老君之處,如何錯到此間?也罷,也罷,一向要來望此老,不曾得來,今趁此殘步,就望他一望也好。」即整衣撞進去,那裡不見老君,四無人跡。原來那老君與燃燈古佛在三層高閣朱陵丹臺上講道,眾仙童、仙將、仙官、仙吏都侍立左右聽講。這大聖直至丹房裡面,尋訪不遇。但見丹灶之傍,爐中有火。爐左右安放著五個葫蘆,葫蘆裡都是煉就的金丹。大聖喜道:「此物乃仙家之至寶。老孫自了道以來,識破了內外相同之理,也要煉些金丹濟人,不期到家無暇。今日有緣,卻又撞著此物。趁老子不在,等我吃他幾丸嘗新。」他就把那葫蘆都傾出來,就都吃了,如吃炒豆相似。

一時間,丹滿酒醒。又自己揣度道:「不好,不好!這場禍比天還大,若驚動玉帝,性命難存。走,走,走,不如下界為王去也。」他就跑出兜率宮,不行舊路,從西天門,使個隱身法逃去。即按雲頭,回至花果山界。但見那旌旗閃灼,戈戟光輝,原來是四健將與七十二洞妖王,在那裡演習武藝。大聖高叫道:「小的們,我來也!」眾怪丟了器械,跪倒道:「大聖好寬心,丟下我等許久,不來相顧。」大聖道:「沒多時,沒多時。」

且說且行,徑入洞天深處。四健將打掃安歇,叩頭禮拜畢,俱道:「大聖在天這百十年,實受何職?」大聖笑道:「我記得才半年光景,怎麼就說百十年話?」健將道:「在天一日,即在下方一年也。」大聖道:「且喜這番玉帝相愛,果封做齊天大聖,起一座齊天府,又設安靜、寧神二司,司設仙吏侍衛。向後見我無事,著我看管蟠桃園。近因王母娘娘設蟠桃大會,未曾請我,是我不待他請,先赴瑤池,把他那仙品、仙酒,都是我偷吃了。走出瑤池,踉踉蹡蹡誤入老君宮闕,又把他五個葫蘆金丹也偷吃了。但恐玉帝見罪,方才走出天門來也。」

眾怪聞言大喜。即安排酒果接風,將椰酒滿斟一石碗奉上。大聖喝了一口,即咨牙徠嘴道:「不好吃,不好吃。」崩、芭二將道:「大聖在天宮吃了仙酒、仙殽,是以椰酒不甚美口。常言道:『美不美,鄉中水。』」大聖道:「你們就是『親不親,故鄉人。』我今早在瑤池中受用時,見那長廊之下有許多瓶罐,都是那玉液瓊漿。你們都不曾嘗著,待我再去偷他幾瓶回來,你們各飲半杯,一個個也長生不老。」眾猴歡喜不勝。

大聖即出洞門,又翻一觔斗,使個隱身法,徑至蟠桃會上,進瑤池宮闕,只見那幾個造酒、盤糟、運水、燒火的還鼾睡未醒。他將大的從左右脅下挾了兩個,兩手提了兩個,即撥轉雲頭回來,會眾猴在於洞中,就做個仙酒會,各飲了幾杯,快樂不題。

卻說那七衣仙女自受了大聖的定身法術,一周天方能解脫。各提花籃,回奏王母,說道:「齊天大聖使術法困住我等,故此來遲。」王母問道:「汝等摘了多少蟠桃?」仙女道:「只有兩籃小桃,三籃中桃。至後面,大桃半個也無,想都是大聖偷吃了。及正尋間,不期大聖走將出來,行兇拷打,又問設宴請誰。我等把上會事說了一遍,他就定住我等,不知去向。直到如今,才得醒解回來。」

王母聞言,即去見玉帝,備陳前事。說不了,又見那造酒的一班人,同仙官等來奏:「不知甚麼人,攪亂了蟠桃大會,偷吃了玉液瓊漿;其八珍百味,亦俱偷吃了。」又有四個大天師來奏上:「太上道祖來了。」玉帝即同王母出迎。老君朝禮畢,道:「老道宮中煉了些九轉金丹,伺候陛下做丹元大會,不期被賊偷去,特啟陛下知之。」玉帝見奏悚懼。少時,又有齊天府仙吏叩頭道:「孫大聖不守執事,自昨日出遊,至今未轉,更不知去向。」玉帝又添疑思。只見那赤腳大仙又頫顖上奏道:「臣蒙王母詔,昨日赴會,偶遇齊天大聖,對臣言萬歲有旨,著他邀臣等先赴通明殿演禮,方去赴會。臣依他言語,即返至通明殿外,不見萬歲龍車鳳輦,又急來此俟候。」玉帝越發大驚道:「這廝假傳旨意,賺哄賢卿。快著糾察靈官緝訪這廝蹤跡。」

靈官領旨,即出殿遍訪,盡得其詳細,回奏道:「攪亂天宮者,乃齊天大聖也。」又將前事盡訴一番。玉帝大惱,即差四大天王,協同李天王並哪吒太子,點二十八宿、九曜星官、十二元辰、五方揭諦、四值功曹、東西星斗、南北二神、五岳四瀆、普天星相,共十萬天兵,佈一十八架天羅地網,下界去花果山圍困,定捉獲那廝處治。

眾神即時興師,離了天宮。這一去,但見那:
\begin{quote}
黃風滾滾遮天暗,紫霧騰騰罩地昏。只為妖猴欺上帝,致令眾聖降凡塵。四大天王,五方揭諦:四大天王權總制,五方揭諦調多兵。李托塔中軍掌號,惡哪吒前部先鋒。羅猴星為頭檢點,計都星隨後崢嶸。太陰星精神抖擻,太陽星照耀分明。五行星偏能豪傑,九曜星最喜相爭。元辰星子午卯酉,一個個都是大力天丁。五瘟五岳東西擺,六丁六甲左右行。四瀆龍神分上下,二十八宿密層層。角亢氐房為總領,奎婁胃昴慣翻騰。斗牛女虛危室壁,心尾箕星個個能。井鬼柳星張翼軫,掄槍舞劍顯威靈。停雲降霧臨凡世,花果山前扎下營。
\end{quote}

詩曰:
\begin{quote}
天產猴王變化多,偷丹偷酒樂山窩。
只因攪亂蟠桃會,十萬天兵佈網羅。
\end{quote}

當時李天王傳了令,著眾天兵扎了營,把那花果山圍得水泄不通,上下佈了十八架天羅地網,先差九曜惡星出戰。九曜即提兵徑至洞外,只見那洞外大小群猴跳躍頑耍。星官厲聲高叫道:「那小妖,你那大聖在那裡?我等乃上界差調的天神,到此降你這造反的大聖。教他快快來歸降;若道半個不字,教汝等一概遭誅。」那小妖慌忙傳入道:「大聖,禍事了!禍事了!外面有九個兇神,口稱上界差來的天神,收降大聖。」

那大聖正與七十二洞妖王並四健將分飲仙酒,一聞此報,公然不理道:「今朝有酒今朝醉,莫管門前是與非。」說不了,一起小妖又跳來道:「那九個兇神惡言潑語,在門前罵戰哩。」大聖笑道:「莫採他。詩酒且圖今日樂,功名休問幾時成。」說猶未了,又一起小妖來報:「爺爺!那九個兇神已把門打破,殺進來也。」大聖怒道:「這潑毛神,老大無禮。本待不與他計較,如何上門來欺我?」即命獨角鬼王:「領帥七十二洞妖王出陣。老孫領四健將隨後。」那鬼王疾帥妖兵出門迎敵,卻被九曜惡星一齊掩殺,抵住在鐵板橋頭,莫能得出。

正嚷間,大聖到了,叫一聲:「開路!」掣開鐵棒,幌一幌,碗來粗細,丈二長短,丟開架子,打將出來。九曜星那個敢抵,一時打退。那九曜星立住陣勢道:「你這不知死活的弼馬溫,你犯了十惡之罪:先偷桃,後偷酒,攪亂了蟠桃大會,又竊了老君仙丹,又將御酒偷來此處享樂。你罪上加罪,豈不知之?」大聖笑道:「這幾樁事,實有,實有。但如今你怎麼?」九曜星道:「吾奉玉帝金旨,帥眾到此收降你。快早皈依,免教這些生靈納命,不然,就屣平了此山,掀翻了此洞也。」大聖大怒道:「量你這些毛神,有何法力,敢出浪言。不要走,請吃老孫一棒。」這九曜星一齊踴躍;那美猴王不懼分毫,掄起金箍棒,左遮右擋。把那九曜星戰得筋疲力軟,一個個倒拖器械,敗陣而走,急入中軍帳下,對托塔天王道:「那猴王果十分驍勇,我等戰他不過,敗陣來了。」

李天王即調四大天王與二十八宿,一路出師來鬥。大聖也公然不懼,調出獨角鬼王、七十二洞妖王與四個健將,就於洞門外列成陣勢。你看這場混戰,好驚人也:
\begin{quote}
寒風颯颯,怪霧陰陰。那壁廂旌旗飛彩,這壁廂戈戟生輝。滾滾盔明,層層甲亮。滾滾盔明映太陽,如撞天的銀磬;層層甲亮砌岩崖,似壓地的冰山。大桿刀,飛雲掣電;楮白槍,度霧穿雲。方天戟,虎眼鞭,麻林擺列;青銅劍,四明鏟,密樹排陣。彎弓硬弩雕翎箭,短棍蛇矛挾了魂。大聖一條如意棒,翻來覆去戰天神。殺得那空中無鳥過,山內虎狼奔;揚砂走石乾坤黑,播土飛塵宇宙昏。只聽兵兵撲撲驚天地,煞煞威威振鬼神。
\end{quote}

這一場自辰時佈陣,混殺到日落西山。那獨角鬼王與七十二洞妖怪,盡被眾天神捉拿去了。止走了四健將與那群猴,深藏在水簾洞底。

這大聖一條棒,抵住了四大天神與李托塔、哪吒太子,俱在半空中,殺勾多時,大聖見天色將晚,即拔毫毛一把,丟在口中,嚼碎了,噴將出去,叫聲:「變!」就變了千百個大聖,都使的是金箍棒,打退了哪吒太子,戰敗了五個天王。

大聖得勝,收了毫毛,急轉身回洞,早又見鐵板橋頭,四個健將領眾叩迎,那大眾,哽哽咽咽大哭三聲,又唏唏哈哈大笑三聲。大聖道:「汝等見了我,又哭又笑,何也?」四健將道:「今早帥眾將與天王交戰,把七十二洞妖王與獨角鬼王盡被眾神捉了,我等逃生,故此該哭。這見大聖得勝回來,未曾傷損,故此該笑。」大聖道:「勝負乃兵家之常。古人云:『殺人一萬,自損三千。』況捉了去的頭目乃是虎豹狼蟲、獾獐狐狢之類,我同類者未傷一個,何須煩惱?他雖被我使個分身法殺退,他還要安營在我山腳下。我等且緊緊防守,飽食一頓,安心睡覺,養養精神。天明看我使個大神通,拿這些天將,與眾報仇。」四將與眾猴將椰酒吃了幾碗,安心睡覺不題。

那四大天王收兵罷戰,眾各報功:有拿住虎豹的,有拿住獅象的,有拿住狼蟲狐狢的。更不曾捉著一個猴精。當時果又安轅營,下大寨,賞𤛮了得功之將,吩咐了天羅地網之兵,各各提鈴喝號,圍困了花果山,專待明早大戰。各人得令,一處處謹守。此正是:
\begin{quote}
妖猴作亂驚天地,佈網張羅晝夜看。
\end{quote}

畢竟天曉後如何處治,且聽下回分解。
