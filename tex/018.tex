
\chapter{觀音院唐僧脫難 高老莊大聖除魔}

行者辭了菩薩,按落雲頭,將袈裟掛在香柟樹上,掣出棒來,打入黑風洞裡,那洞裡那得一個小妖。原來是他見菩薩出現,降得那老怪就地打滾,急急都散走了。行者一發行兇,將他那幾層門上都積了乾柴,前前後後,一齊發火,把個黑風洞燒做個紅風洞,卻拿了袈裟,駕祥光,轉回直北。

話說那三藏望行者急忙不來,心甚疑惑:不知是請菩薩不至,不知是行者託故而逃。正在那胡猜亂想之中,只見半空中彩霧燦燦,行者忽墜階前跪道:「師父,袈裟來了。」三藏大喜。眾僧亦無不歡悅道:「好了,好了,我等性命今日方才得全了。」三藏接了袈裟道:「悟空,你早間去時,原約到飯罷晌午,如何此時日西方回?」行者將那請菩薩施變化降妖的事情,備陳了一遍。三藏聞言,遂設香案,朝南禮拜罷,道:「徒弟啊,既然有了佛衣,可快收拾包裹去也。」行者道:「莫忙,莫忙。今日將晚,不是走路的時候,且待明日早行。」眾僧們一齊跪下道:「孫老爺說得是。一則天晚,二來我等有些願心兒,今幸平安,有了寶貝,待我還了願,請老爺散了福,明早再送西行。」行者道:「正是,正是。」你看那些和尚都傾囊倒底,把那火裡搶出的餘資,各出所有,整頓了些齋供,燒了些平安無事的紙,念了幾卷消災解厄的經。當晚事畢。

次早,方刷扮了馬匹,包裹了行囊出門,眾僧遠送方回。行者引路而去,正是那春融時節,但見那:
\begin{quote}
草襯玉驄蹄跡軟,柳搖金線露華新。
桃杏滿林爭豔麗,薜蘿遶徑放精神。
沙堤日暖鴛鴦睡,山澗花香蛺蝶馴。
這般秋去冬殘春過半,不知何年行滿得真文。
\end{quote}

師徒們行了五七日荒路,忽一日天色將晚,遠遠的望見一村人家。三藏道:「悟空,你看那壁廂有座山莊相近,我們去告宿一宵,明日再行何如?」行者道:「且等老孫去看看吉凶,再作區處。」那師父挽住絲韁,這行者定睛觀看,真個是:
\begin{quote}
竹籬密密,茅屋重重。參天野樹迎門,曲水溪橋映戶。道傍楊柳綠依依,園內花開香馥馥。此時那夕照沉西,處處山林喧鳥雀;晚煙出爨,條條道徑轉牛羊。又見那食飽雞豚眠屋角,醉酣鄰叟唱歌來。
\end{quote}

行者看罷道:「師父請行,定是一村好人家,正可借宿。」那長老催動白馬,早到街衢之口。又見一個少年,頭裹綿布,身穿藍襖,持傘背包,斂褌劄褲,腳踏著一雙三耳草鞋,雄糾糾的,出街忙走。行者順手一把扯住道:「那裡去?我問你一個信兒:此間是甚麼地方?」那個人只管苦掙,口裡嚷道:「我莊上沒人,只是我好問信?」行者陪著笑道:「施主莫惱。『與人方便,自己方便。』你就與我說說地名何害?我也可解得你的煩惱。」那人掙不脫手,氣得亂跳道:「蹭蹬,蹭蹬。家長的屈氣受不了,又撞著這個光頭,受他的清氣。」行者道:「你有本事,劈開我的手,你便就去了也罷。」那人左扭右扭,那裡扭得動,卻似一把鐵鈐拑住一般。氣得他丟了包袱,撇了傘,兩隻手雨點似來抓行者。行者把一隻手扶著行李,一隻手抵住那人,憑他怎麼支吾,只是不能抓著。行者愈加不放,急得爆燥如雷。三藏道:「悟空,那裡不有人來了?你再問那人就是,只管扯住他怎的?放他去罷。」行者笑道:「師父不知,若是問了別人沒趣,須是問他,才有買賣。」那人被行者扯住不放,只得說出道:「此處乃是烏斯藏國界之地,喚做高老莊。一莊人家有大半姓高,故此喚做高老莊。你放了我去罷。」行者又道:「你這樣行裝,不是個走近路的。你實與我說,你要往那裡去,端的所幹何事,我才放你。」

這人無奈,只得以實情告訴道:「我是高太公的家人,名叫高才。我那太公有個老女兒,年方二十歲,更不曾配人。三年前被一個妖精占了,那妖整做了這三年女婿。我太公不悅,說道:『女兒招了妖精,不是長法:一則敗壞家門,二則沒個親家來往。』一向要退這妖精。那妖精那裡肯退,轉把女兒關在他後宅,將有半年,再不放出與家內人相見。我太公與了我幾兩銀子,教我尋訪法師,拿那妖怪。我這些時不曾住腳,前前後後,請了有三四個人,都是不濟的和尚,膿包的道士,降不得那妖精。剛才罵了我一場,說我不會幹事。又與了我五錢銀子做盤纏,教我再去請好法師降他。不期撞著你這個紇刺星扯住,誤了我走路,故此裡外受氣,我無奈,才與你叫喊。不想你又有些拿法,我掙不過你,所以說此實情。你放我走罷。」行者道:「你的造化,我有營生,這才是湊四合六的勾當。你也不須遠行,莫要花費了銀子。我們不是那不濟的和尚,膿包的道士,其實有些手段,慣會拿妖。這正是:『一來照顧郎中,二來又醫得眼好。』煩你回去上覆你那家主,說我們是東土駕下差來的御弟聖僧,往西天拜佛求經者,善能降妖縛怪。」高才道:「你莫誤了我。我是一肚子氣的人,你錯哄了我,沒甚手段,拿不住那妖精,卻不又帶累我來受氣?」行者道:「管教不誤了你,你引我到你家門首去來。」那人也無計奈何,真個提著包袱,拿了傘,轉步回身,領他師徒到於門首道:「二位長老,你且在馬臺上略坐坐,等我進去報主人知道。」行者才放了手,落擔牽馬,師徒們坐立門傍等候。

那高才入了大門,徑往中堂上走,可可的撞見高太公。太公罵道:「你那個蠻皮畜生!怎麼不去尋人,又回來做甚?」高才放下包、傘道:「上告主人公得知:小人才行出街口,忽撞見兩個和尚:一個騎馬,一個挑擔。他扯住我不放,問我那裡去。我再三不曾與他說及,他纏得沒奈何,不得脫手,遂將主人公的事情,一一說與他知。他卻十分歡喜,要與我們拿那妖怪哩。」高老道:「是那裡來的?」高才道:「他說是東土駕下差來的御弟聖僧,前往西天拜佛求經的。」太公道:「既是遠來的和尚,怕不真有些手段。他如今在那裡?」高才道:「現在門外等候。」

那太公即忙換了衣服,與高才出來迎接,叫聲:「長老。」三藏聽見,急轉身,早已到了面前。那老者戴一頂烏綾巾,穿一領蔥白蜀錦衣,踏一雙糙米皮的犢子靴,繫一條黑綠絛子,出來笑語相迎,便叫:「二位長老,作揖了。」三藏還了禮,行者站著不動。那老者見他相貌兇醜,便就不敢與他作揖。行者道:「怎麼不唱老孫喏?」那老兒有幾分害怕,叫高才道:「你這小廝卻不弄殺我也?家裡現有一個醜頭怪腦的女婿打發不開,怎麼又引這個雷公來害我?」行者道:「老高,你空長了許大年紀,還不省事。若專以相貌取人,乾淨錯了。我老孫醜自醜,卻有些本事。替你家擒得妖精,捉得鬼魅,拿住你那女婿,還了你女兒,便是好事,何必諄諄以相貌為言?」太公見說,戰兢兢的,只得強打精神,叫聲:「請進。」這行者見請,才牽了白馬,教高才挑著行李,與三藏進去。他也不管好歹,就把馬拴在敞廳柱上,扯過一張退光漆交椅,叫師父坐下。他又扯過一張椅子,坐在傍邊。那高老道:「這個小長老,倒也家懷。」行者道:「你若肯留我住得半年,還家懷哩。」

坐定,高老問道:「適間小价說,二位長老是東土來的?」三藏道:「便是。貧僧奉朝命往西天拜佛求經,因過寶莊,特借一宿,明日早行。」高老道:「二位原是借宿的,怎麼說會拿怪?」行者道:「因是借宿,順便拿幾個妖怪兒耍耍的。動問府上有多少妖怪?」高老道:「天哪!還吃得有多少哩,只這一個妖怪女婿,已被他磨慌了。」行者道:「你把那妖怪的始末,有多大手段,從頭兒說說我聽,我好替你拿他。」高老道:「我們這莊上,自古至今,也不曉得有甚麼鬼祟魍魎,邪魔作耗。只是老拙不幸,不曾有子,止生三個女兒:大的喚名香蘭,第二的名玉蘭,第三的名翠蘭。那兩個從小兒配與本莊人家。止有小的個要招個女婿,指望他與我同家過活,做個養老女婿,撐門抵戶,做活當差。不期三年前,有一個漢子,模樣兒倒也精緻。他說是福陵山上人家,姓豬,上無父母,下無兄弟,願與人家做個女婿。我老拙見是這般一個無根無絆的人,就招了他。一進門時,倒也勤謹:耕田耙地,不用牛具;收割田禾,不用刀杖;昏去明來,其實也好。只是一件,有些會變嘴臉。」行者道:「怎麼樣變?」高老道:「初來時是一條黑胖漢,後來就變做一個長嘴大耳朵的獃子,腦後又有一溜鬃毛,身體粗糙怕人,頭臉就像個豬的模樣。食腸卻又甚大:一頓要吃三五斗米飯,早間點心也得百十個燒餅才夠。喜得還吃齋素;若再吃葷酒,便是老拙這些家業田產之類,不上半年,就吃個罄淨。」三藏道:「只因他做得,所以吃得。」高老道:「吃還是件小事。他如今又會弄風,雲來霧去,走石飛砂,諕得我一家並左鄰右舍,俱不得安生。又把那翠蘭小女關在後宅子裡,一發半年也不曾見面,更不知死活如何。因此知他是個妖怪,要請個法師與他去退去退。」

行者道:「這個何難?老兒你管放心,今夜管情與你拿住,教他寫了退親文書,還你女兒如何?」高老大喜道:「我為招了他不打緊,壞了我多少清名,疏了我多少親眷。但得拿住他,要甚麼文書?就煩與我除了根罷。」行者道:「容易,容易。入夜之時,就見好歹。」

老兒十分歡喜,才教展抹桌椅,擺列齋供。齋罷將晚,老兒問道:「要甚兵器?要多少人隨?趁早好備。」行者道:「兵器我自有。」老兒道:「二位只是那根錫杖,錫杖怎麼打得那個妖精?」行者隨於耳內取出一個繡花針來,捻在手中,迎風幌了一幌,就是碗來粗細的一根金箍鐵棒,對著高老道:「你看這條棍子,比你家兵器如何?可打得這怪否?」高老又道:「既有兵器,可要人跟?」行者道:「我不用人,只是要幾個年高有德的老兒,陪我師父清坐閑敘,我好撇他而去。等我把那妖精拿來,對眾取供,替你除了根罷。」那老兒即喚家僮,請了幾個親故朋友。一時都到,相見已畢,行者道:「師父,你放心穩坐,老孫去也。」

你看他揝著鐵棒,扯著高老道:「你引我去後宅子裡妖精的住處看看。」高老遂引他到後宅門首。行者道:「你去取鑰匙來。」高老道:「你且看看,若是用得鑰匙,卻不請你了。」行者笑道:「你那老兒年紀雖大,卻不識耍。我把這話兒哄你一哄,你就當真。」走上前,摸了一摸,原來是銅汁灌的鎖子。狠得他將金箍棒一搗,搗開門扇,裡面卻黑洞洞的。行者道:「老高,你去叫你女兒一聲,看他可在裡面?」那老兒硬著膽叫道:「三姐姐!」那女兒認得是他父親的聲音,才少氣無力的應了一聲道:「爹爹,我在這裡哩。」行者閃金睛,向黑影裡仔細看時,你道他怎生模樣?但見那:
\begin{quote}
雲鬢亂堆無掠,玉容未洗塵淄。一片蘭心依舊,十分嬌態傾頹。櫻唇全無氣血,腰肢屈屈偎偎。愁蹙蹙,蛾眉淡;瘦怯怯,語聲低。
\end{quote}

他走來看見高老,一把扯住,抱頭大哭。行者道:「且莫哭,且莫哭。我問你,妖怪往那裡去了?」女子道:「不知往那裡去。這些時,天明就去,入夜方來。雲雲霧霧,往回不知何所。因是曉得父親要祛退他,他也常常防備,故此昏來朝去。」行者道:「不消說了。老兒,你帶令愛往前邊宅裡,慢慢的敘闊,讓老孫在此等他。他若不來,你卻莫怪;他若來了,定與你剪草除根。」那老高歡歡喜喜的把女兒帶將前去。

行者卻弄神通,搖身一變,變得就如那女子一般,獨自個坐在房裡等那妖精。不多時,一陣風來,真個是走石飛砂。好風:
\begin{quote}
起初時微微蕩蕩,向後來渺渺茫茫。
微微蕩蕩乾坤大,渺渺茫茫無阻礙。
凋花折柳勝揌麻,倒樹摧林如拔菜。
翻江攪海鬼神愁,裂石崩山天地怪。
啣花糜鹿失來蹤,摘果猿猴迷在外。
七層鐵塔侵佛頭,八面幢幡傷寶蓋。
金梁玉柱起根搖,房上瓦飛如燕塊。
舉棹梢公許願心,開船忙把豬羊賽。
當坊土地棄祠堂,四海龍王朝上拜。
海邊撞損夜叉船,長城刮倒半邊塞。
\end{quote}

那陣狂風過處,只見半空裡來了一個妖精,果然生得醜陋:黑臉短毛,長喙大耳;穿一領青不青、藍不藍的梭布直裰,繫一條花布手巾。行者暗笑道:「原來是這個買賣。」好行者,卻不迎他,也不問他,且睡在床上推病,口裡哼哼嘖嘖的不絕。那怪不識真假,走進房,一把摟住,就要親嘴。行者暗笑道:「真個要來弄老孫哩。」即使個拿法,托著那怪的長嘴,叫做個小跌。漫頭一抖,撲的摜下床來。那怪爬起來,扶著床邊道:「姐姐,你怎麼今日有些怪我?想是我來得遲了?」行者道:「不怪,不怪。」那妖道:「既不怪我,怎麼就丟我這一跌?」行者道:「你怎麼就這等樣小家子,就摟我親嘴?我因今日有些不自在;若每常好時,便起來開門等你了。你可脫了衣服睡是。」那怪不解其意,真個就去脫衣。行者跳起來,坐在淨桶上。那怪依舊復來床上摸一把,摸不著人,叫道:「姐姐,你往那裡去了?請脫衣服睡罷。」行者道:「你先睡,等我出個恭來。」那怪果先解衣上床。

行者忽然嘆口氣,道聲:「造化低了。」那怪道:「你惱怎的?造化怎麼得低的?我得到了你家,雖是吃了些茶飯,卻也不曾白吃你的:我也曾替你家掃地通溝、搬磚運瓦、築土打牆、耕田耙地、種麥插秧、創家立業。如今你身上穿的錦,戴的金,四時有花果享用,八節有蔬菜烹煎,你還有那些兒不趁心處,這般短嘆長吁,說甚麼造化低了?」行者道:「不是這等說。今日我的父母隔著牆,丟磚料瓦的,甚是打我罵我哩。」那怪道:「他打罵你怎的?」行者道:「他說我和你做了夫妻,你是他門下一個女婿,全沒些兒禮體。這樣個醜嘴臉的人,又會不得姨夫,又見不得親戚,又不知你雲來霧去,端的是那裡人家,姓甚名誰,敗壞他清德,玷辱他門風,故此這般打罵,所以煩惱。」那怪道:「我雖是有些兒醜陋,若要俊,卻也不難。我一來時,曾與他講過,他願意方才招我。今日怎麼又說起這話?我家住在福陵山雲棧洞。我以相貌為姓,故姓豬,官名叫做豬剛鬣。他若再來問你,你就以此話與他說便了。」

行者暗喜道:「那怪卻也老實,不用動刑,就供得這等明白。既有了地方、姓名,不管怎的也拿住他。」行者道:「他要請法師來拿你哩。」那怪笑道:「睡著,睡著,莫睬他。我有天罡數的變化,九齒的釘鈀,怕甚麼法師、和尚、道士?就是你老子有虔心,請下九天蕩魔祖師下界,我也曾與他做過相識,他也不敢怎的我。」行者道:「他說請一個五百年前大鬧天宮姓孫的齊天大聖,要來拿你哩。」那怪聞得這個名頭,就有三分害怕道:「既是這等說,我去了罷,兩口子做不成了。」行者道:「你怎的就去?」那怪道:「你不知道,那鬧天宮的弼馬溫有些本事,只恐我弄他不過,低了名頭,不像模樣。」

說罷,套上衣服,開了門,往外就走。被行者一把扯住,將自己臉上抹了一抹,現出原身,喝道:「好妖怪,那裡走!你擡頭看看我是那個?」那怪轉過眼來,看見行者咨牙徠嘴,火眼金睛,磕頭毛臉,就是個活雷公相似。慌得他手麻腳軟,劃剌的一聲,掙破了衣服,化狂風脫身而去。行者急上前,掣鐵棒,望風打了一下。那怪化萬道火光,徑轉本山而去。行者駕雲,隨後趕來,叫聲:「那裡走!你若上天,我就趕到斗牛宮;你若入地,我就追至枉死獄。」

咦!畢竟不知這一去趕至何方,有何勝敗,且聽下回分解。
