
\chapter{四僧宴樂御花園 一怪空懷情慾喜}

話表孫行者三人隨著宣召官至午門外,黃門官即時傳奏宣進。他三個齊齊站定,更不下拜。國王問道:「那三位是聖僧駙馬之高徒?姓甚名誰?何方居住?因甚事出家?取何經卷?」行者即近前,意欲上殿。傍有護駕的喝道:「不要走,有甚話,立下奏來。」行者笑道:「我們出家人,得一步就進一步。」隨後八戒、沙僧亦俱近前。長老恐他村魯驚駕,便起身叫道:「徒弟啊,陛下問你來因,你即奏上。」行者見他那師父在傍侍立,忍不住大叫一聲道:「陛下輕人輕己。既招我師為駙馬,如何教他侍立?世間稱女夫謂之貴人,豈有貴人不坐之理。」國王聽說,大驚失色,欲退殿,恐失了觀瞻。只得硬著膽,教近侍的取繡墩來,請唐僧坐了。行者才奏道:
\begin{quote}
「老孫祖居東勝神洲傲來國花果山水簾洞。父天母地,石裂吾生。曾拜至人,學成大道。復轉仙鄉,嘯聚在洞天福地。下海降龍,登山擒獸。消死名,上生籍,官拜齊天大聖。玩賞瓊樓,喜遊寶閣。會天仙,日日歌歡;居聖境,朝朝快樂。只因亂卻蟠桃宴,大反天宮,被佛擒伏。困壓在五行山下,饑餐鐵彈,渴飲銅汁,五百年未嘗茶飯。幸我師出東土,拜西方,觀音教令脫天災,離大難,皈正在瑜伽門下。舊諱悟空,稱名行者。」
\end{quote}

國王聞得這般名重,慌得下了龍床,走將來,以御手挽定長老道:「駙馬,也是朕之天緣,得遇你這仙姻仙眷。」三藏滿口謝恩,請國王登位。

復問:「那位是第二高徒?」八戒掬嘴揚威道:
\begin{quote}
「老豬先世為人,貪歡愛懶。一生混沌,亂性迷心。未識天高地厚,難明海闊山遙。正在幽閑之際,忽然遇一真人。半句話,解開孽網;兩三言,劈破災門。當時省悟,立地投師,謹修二八之工夫,敬煉三三之前後。行滿飛昇,得超天府。荷蒙玉帝厚恩,官賜天蓬元帥,管押河兵,逍遙漢闕。只因蟠桃酒醉,戲弄嫦娥,謫官銜,遭貶臨凡;錯投胎,托生豬像。住福陵山,造惡無邊。遇觀音,指明善道。皈依佛教,保護唐僧。徑往西天,拜求妙典。法諱悟能,稱為八戒。」
\end{quote}

國王聽言,膽戰心驚,不敢觀覷。這獃子越弄精神,搖著頭,掬著嘴,撐起耳朵,呵呵大笑。三藏又怕驚駕,即叱道:「八戒收斂!」方才叉手拱立,假扭斯文。

又問:「第三位高徒,因甚皈依?」沙和尚合掌道:
\begin{quote}
「老沙原係凡夫,因怕輪迴訪道。雲遊海角,浪蕩天涯。常得衣缽隨身,每煉心神在舍。因此虔誠,得逢仙侶。養就孩兒,配緣姹女。工滿三千,合和四相。超天界,拜玄穹,官授捲簾大將,侍御鳳輦龍車,封號將軍。也為蟠桃會上,失手打破玻璃盞,貶在流沙河,改頭換面,造孽傷生。幸喜菩薩遠遊東土,勸我皈依,等候唐朝佛子,往西天求經果正。從立自新,復修大覺。指河為姓,法諱悟淨,稱名和尚。」
\end{quote}

國王見說,多驚多喜:喜的是女兒招了活佛,驚的是三個實乃妖神。正在驚喜之間,忽有正臺陰陽官奏道:「婚期已定本年本月十二日壬子辰良,周堂通利,宜配婚姻。」國王道:「今日是何日辰?」陰陽官奏:「今日初八,乃戊申之日,猿猴獻果,正宜進賢納事。」國王大喜,即著當駕官打掃御花園館閣樓亭,且請駙馬同三位高徒安歇,待後安排合巹佳筵,著公主匹配。眾等欽遵,國王退朝,多官皆散不題。

卻說三藏師徒們都到御花園,天色漸晚,擺了素膳。八戒喜道:「這一日也該吃飯了。」管辦人即將素米飯、麵飯等物,整擔挑來。那八戒吃了又添,添了又吃,直吃得撐腸拄腹,方才住手。少頃,又點上燈,設鋪蓋,各自歸寢。

長老見左右無人,卻恨責行者,怒聲叫道:「悟空,你這猢猻,番番害我。我說只去倒換關文,莫向彩樓前去,你怎麼直要引我去看看?如今看得好麼,卻惹出這般事來,怎生是好?」行者陪笑道:「師父說:『先母也是拋打繡毬,遇舊緣,成其夫婦。』似有慕古之意,老孫才引你去。又想著那個給孤布金寺長老之言,就此檢視真假。適見那皇帝之面,略有些晦暗之色,但只未見公主何如耳。」長老道:「你見公主便怎的?」行者道:「老孫的火眼金睛,但見面,就認得真假善惡,富貴貧窮,卻好施為,辨明邪正。」沙僧與八戒笑道:「哥哥近日又學得會相面了。」行者道:「相面之士,當我孫子罷了。」三藏喝道:「且休調嘴。只是他如今定要招我,果何以處之?」行者道:「且到十二日會喜之時,必定那公主出來參拜父母,等老孫在傍觀看。若還是個真女人,你就做了駙馬,享用國內之榮華也罷。」三藏聞言,越生嗔怒,罵道:「好猢猻!你還害我哩。卻是悟能說的,我們十節兒已上了九節七八分了,你還把熱舌頭鐸我。快早夾著,你休開那臭口;再若無禮,我就念起咒來,教你了當不得。」行者聽說念咒,慌得跪在面前道:「莫念,莫念。若是真女人,待拜堂時,我們一齊大鬧皇宮,領你去也。」

師徒說話,不覺早已入更。正是:
\begin{quote}
沉沉宮漏,廕廕花香。繡戶垂珠箔,閑庭絕火光。鞦韆索冷空留影,羌笛聲殘靜四方。繞屋有花籠月燦,隔空無樹顯星芒。杜鵑啼歇,蝴蝶夢長。銀漢橫天宇,白雲歸故鄉。正是離人情切處,風搖嫩柳更淒涼。
\end{quote}

八戒道:「師父,夜深了,有事明早再議,且睡,且睡。」師徒們果然安歇一宵。

早又金雞唱曉。國王即登殿設朝。但見:
\begin{quote}
宮殿開軒紫氣高,風吹御樂透青霄。
雲移豹尾旌旗動,日射螭頭玉珮搖。
香霧細添宮柳綠,露珠微潤苑花嬌。
山呼舞蹈千官列,海晏河清一統朝。
\end{quote}

眾文武百官朝罷,又宣:「光祿寺安排十二日會喜佳筵。今日且整春罍,請駙馬在御花園中款玩。」吩咐儀制司領三位賢親去會同館少坐,著光祿寺安排三席素宴去彼奉陪。兩處俱著教坊司奏樂,伏侍賞春景,消遲日也。八戒聞得,應聲道:「陛下,我師徒自相會,更無一刻相離。今日既在御花園飲宴,帶我們去耍兩日,好教師父替你家做駙馬;不然,這個買賣生意弄不成。」那國王見他醜陋,說話粗俗,又見他扭頭捏頸,掬嘴巴,搖耳朵,即像有些風氣,猶恐攪破親事,只得依從。便教:「在永鎮華夷閣裡安排二席,我與駙馬同坐。留春亭上安排三席,請三位別坐,恐他師徒們坐次不便。」那獃子才朝上唱個喏,叫聲:「多謝。」各各而退。又傳旨教內宮官排宴,著三宮六院后妃與公主上頭,就為添妝餪子,以待十二日佳配。

將有巳時前後,那國王排駕,請唐僧都到御花園內觀看。好去處:
\begin{quote}
徑鋪彩石,檻鑿雕欄。徑鋪彩石,徑邊石畔長奇葩;檻鑿雕欄,檻外欄中生異卉。夭桃迷翡翠,嫩柳閃黃鸝。步覺幽香來袖滿,行沾清味上衣多。鳳臺龍沼,竹閣松軒。鳳臺之上,吹簫引鳳來儀;龍沼之間,養魚化龍而去。竹閣有詩,費盡推敲裁白雪;松軒文集,考成珠玉註青編。假山拳石翠,曲水碧波深。牡丹亭,薔薇架,疊錦鋪絨;茉藜檻,海棠畦,堆霞砌玉。芍藥異香,蜀葵奇豔。白梨紅杏鬥芳菲,紫蕙金萱爭爛熳。麗春花、木筆花、杜鵑花,夭夭灼灼;含笑花、鳳仙花、玉簪花,戰戰巍巍。一處處紅透胭脂潤,一叢叢芳濃錦繡圍。更喜東風回煖日,滿園嬌媚逞光輝。
\end{quote}

一行君王幾位,觀之良久。早有儀制司官邀請行者三人入留春亭,國王攜唐僧上華夷閣,各自飲宴。那歌舞吹彈,鋪張陳設,真是:
\begin{quote}
崢嶸閶闔曙光生,鳳閣龍樓瑞靄橫。
春色細鋪花草繡,天光遙射錦袍明。
笙歌繚繞如仙宴,杯斝飛傳玉液清。
君悅臣歡同玩賞,華夷永鎮世康寧。
\end{quote}

此時長老見那國王敬重,無計可奈,只得勉強隨喜,誠是外喜而內憂也。坐間見壁上掛著四面金屏,屏上畫著春夏秋冬四景,皆有題詠,皆是翰林名士之詩:

《春景詩》曰:
\begin{quote}
周天一氣轉洪鈞,大地熙熙萬象新。
桃李爭妍花爛熳,燕來畫棟疊香塵。
\end{quote}

《夏景詩》曰:
\begin{quote}
薰風拂拂思遲遲,宮院榴葵映日輝。
玉笛音調驚午夢,芰荷香散到庭幃。
\end{quote}

《秋景詩》曰:
\begin{quote}
金井梧桐一葉黃,珠簾不捲夜來霜。
燕知社日辭巢去,鴈折蘆花過別鄉。
\end{quote}

《冬景詩》曰:
\begin{quote}
天雨飛雲暗淡寒,朔風吹雪積千山。
深宮自有紅爐暖,報道梅開玉滿欄。
\end{quote}

那國王見唐僧恣意看詩,便道:「駙馬喜玩詩中之味,必定善於吟哦。如不吝珠玉,請依韻各和一首如何?」長老是個對景忘情,明心見性之意。見國王欽重,命和前韻,他不覺忽詠一句道:「日暖冰消大地鈞。」國王大喜,即召侍衛官:「取文房四寶,請駙馬和完錄下,俟朕緩緩味之。」長老欣然不辭,舉筆而和:

和《春景詩》曰:
\begin{quote}
日暖冰消大地鈞,御園花卉又更新。
和風膏雨民沾澤,海晏河清絕俗塵。
\end{quote}

和《夏景詩》曰:
\begin{quote}
斗指南方白晝遲,槐雲榴火鬥光輝。
黃鸝紫燕啼宮柳,巧轉雙聲入絳幃。
\end{quote}

和《秋景詩》曰:
\begin{quote}
香飄橘綠與橙黃,松柏青青喜降霜。
籬菊半開攢錦繡,笙歌韻徹水雲鄉。
\end{quote}

和《冬景詩》曰:
\begin{quote}
瑞雪初晴氣味寒,奇峰巧石玉團山。
爐燒獸炭煨酥酪,袖手高歌倚翠欄。
\end{quote}

國王見和大喜,稱唱道:「好個『袖手高歌倚翠欄』!」遂命教坊司以新詩奏樂,盡日而散。

行者三人在留春亭亦盡受用,各飲了幾杯,也都有些酣意。正欲去尋長老,只見長老已同國王在一閣。八戒獃性發作,應聲叫道:「好快活,好自在,今日也受用這一下了。卻該趁飽兒睡覺去也。」沙僧笑道:「二哥忒沒修養。這氣飽飫,如何睡覺?」八戒道:「你那裡知道,俗語云:『吃了飯兒不挺屍,肚裡沒板脂』哩。」

唐僧與國王相別,至亭內,嗔責八戒道:「這夯貨,越發村了。這是甚麼去處,只管大呼小叫?倘或惱著國王,卻不被他傷害性命?」八戒道:「沒事,沒事。我們與他親家禮道的,他便不好生怪。常言道:『打不斷的親,罵不斷的鄰。』大家耍子,怕他怎的?」長老叱道,教拿過獃子來,打他二十禪杖。行者果一把揪翻,長老舉杖就打。獃子喊叫道:「駙馬爺爺,饒罪,饒罪。」傍有陪宴官勸住。獃子爬將起來,突突囔囔的道:「好貴人,好駙馬,親還未成,就行起王法來了。」行者侮著他嘴道:「莫胡說,莫胡說,快早睡去。」

他們又在留春亭住了一宿。到明早,依舊宴樂。

不覺樂了三四日,正值十二日佳辰。有光祿寺三部各官回奏道:「臣等自八日奉旨,駙馬府已修完,專等妝奩鋪設。合巹宴亦已完備,葷素共五百餘席。」國王心喜,欲請駙馬赴席,忽有內宮官對御前啟奏道:「萬歲,正宮娘娘有請。」國王遂退入內宮,只見那三宮皇后、六院嬪妃,引領著公主,都在昭陽宮談笑。真個是花團錦簇,那一片富麗妖嬈,真勝似天堂月殿,不亞於仙府瑤宮。有喜會佳姻新詞四首為證。

《喜詞》云:
\begin{quote}
喜喜喜,欣然樂矣。結婚姻,恩愛美。巧樣宮妝,嫦娥怎比。龍釵與鳳釵,豔豔飛金縷。櫻唇皓齒朱顏,嬝娜如花輕體。錦重重,五彩叢中;香拂佛,千金隊裡。
\end{quote}

《會詞》云:
\begin{quote}
會會會,妖嬈嬌媚。賽毛嬙,欺楚妹。傾國傾城,比花比玉。妝飾更鮮妍,釵環多豔麗。蘭心蕙性清高,粉臉冰肌榮貴。黛眉一線遠山微,窈窕嫣姌攢錦隊。
\end{quote}

《佳詞》云:
\begin{quote}
佳佳佳,玉女仙娃。深可愛,實堪誇。異香馥郁,脂粉交加。天臺福地遠,怎似國王家。笑語紛然嬌態,笙歌繚繞喧嘩。花堆錦砌千般美,看遍人間怎若他。
\end{quote}

《姻詞》云:
\begin{quote}
姻姻姻,蘭麝香噴。仙子陣,美人群。嬪妃換綵,宮主妝新。雲鬢堆鴉髻,霓裳壓鳳裙。一派仙音嘹喨,兩行朱紫繽紛。當年曾結乘鸞信,今朝幸喜會佳姻。
\end{quote}

卻說國王駕到,那后妃引著公主,並綵女、宮娥,都來迎接。國王喜孜孜,進了昭陽宮坐下。后妃等朝拜畢,國王道:「公主賢女,自初八日結綵拋毬,幸遇聖僧,想是心願已足。各衙門官又能體朕心,各項事俱已完備。今日正是佳期,可早赴合巹之宴,不要錯過時辰。」那公主走近前,倒身下拜,奏道:「父王,乞赦小女萬千之罪,有一言啟奏:這幾日聞得宮官傳說,唐聖僧有三個徒弟,他生得十分醜惡。小女不敢見他,恐見時必生恐懼。萬望父王將他發放出城方好,不致驚傷弱體,反為禍害也。」國王道:「孩兒不說,朕幾乎忘了。果然生得有些醜惡,連日教他在御花園裡留春亭管待。趁今日就上殿,打發他關文,教他出城,卻好會宴。」公主叩頭謝了恩。國王即出駕上殿,傳旨請駙馬共他三位。

原來那唐僧捏指頭兒算日子,熬至十二日,天未明,就與他三人計較道:「今日卻是十二了,這事如何區處?」行者道:「那國王我已識得他有些晦氣,還未沾身,不為大害。但只不得公主見面,若得出來,老孫一覷,就知真假,方才動作。你只管放心。他如今一定來請,打發我等出城。你自應承莫怕,我閃閃身兒就來,緊緊隨護你也。」

師徒們正講,果見當駕官同儀制司來請。行者笑道:「去來,去來。必定是與我們送行,好留師父會合。」八戒道:「送行必定有千百兩黃金白銀,我們也好買些人事回去。到我那丈人家,也再會親耍子兒去耶。」沙僧道:「二哥箝著口,休亂說,只憑大哥主張。」

遂此將行李、馬匹,俱隨那些官到於丹墀下。國王見了,教請行者三位近前道:「汝等將關文拿上來,朕當用寶、花押,交付汝等,外多備盤纏,送你三位早去靈山見佛。若取經回來,還有重謝。留駙馬在此,勿得懸念。」行者稱謝。遂教沙僧取出關文遞上。國王看了,即用了印,押了花字,又取黃金十錠、白金二十錠,聊達親禮。八戒原來財色心重,即去接了。行者朝上唱個喏道:「聒噪,聒噪。」便轉身要走。慌得個三藏一轂轆爬起,扯住行者,咬響牙根道:「你們都不顧我就去了?」行者把手捏著三藏手掌,丟個眼色道:「你在這裡寬懷歡會,我等取了經,回來看你。」那長老似信不信的,不肯放手。多官都看見,以為實是相別而去。早見國王又請駙馬上殿,著多官送三位出城。長老只得放了手上殿。

行者三人同眾出了朝門,各自相別。八戒道:「我們當真的走哩?」行者不言語,只管走至驛中。驛丞接入,看茶,擺飯。行者對八戒、沙僧道:「你兩個只在此,切莫出頭。但驛丞問甚麼事情,且含糊答應,莫與我說話。我保師父去也。」

好大聖,拔一根毫毛,吹口仙氣,叫:「變!」即變作本身模樣,與八戒、沙僧同在驛內。真身卻幌的跳在半空,變作一個蜜蜂兒。但見:
\begin{quote}
翅黃口甜尾利,隨風飄舞顛狂。
最能摘蕊與偷香。度柳穿花搖蕩。
辛苦幾番淘染,飛來飛去空忙。
釀成濃美自何嘗。只好留存名狀。
\end{quote}

你看他輕輕的飛入朝中,遠見那唐僧在國王左邊繡墩上坐著,愁眉不展,心存焦燥。徑飛至他毘盧帽上,悄悄的爬及耳邊,叫道:「師父,我來了,切莫憂慮。」這句話只有唐僧聽見,那夥凡人莫想知覺。唐僧聽見,始覺心寬。不一時,宮官來請道:「萬歲,合巹嘉筵已排設在鳷鵲宮中,娘娘與公主俱在宮伺候,專請萬歲同貴人會親也。」國王喜之不盡,即同駙馬進宮而去。正是那:
\begin{quote}
邪主愛花花作禍,禪心動念念生愁。
\end{quote}

畢竟不知唐僧在內宮怎生解脫,且聽下回分解。
