
\chapter{情因舊恨生災毒 心主遭魔幸破光}

話說孫大聖扶持著唐僧,與八戒、沙僧奔上大路,一直西來。不半晌,忽見一處樓閣重重,宮殿巍巍。唐僧勒馬道:「徒弟,你看那是個甚麼去處?」行者舉頭觀看,忽然見:
\begin{quote}
山環樓閣,溪遶亭臺。門前雜樹密森森,宅外野花香豔豔。柳間棲白鷺,渾如煙裡玉無瑕;桃內囀黃鶯,卻是火中金有色。雙雙野鹿,忘情閑踏綠莎茵;對對山禽,飛語高鳴紅樹杪。真如劉阮天臺洞,不亞神仙閬苑家。
\end{quote}

行者報道:「師父,那所在也不是王侯第宅,也不是豪富人家,卻像一個庵觀寺院。到那裡方知端的。」三藏聞言,加鞭促馬。師徒們來至門前觀看,門上嵌著一塊石板,上有「黃花觀」三字。三藏下馬。八戒道:「黃花觀乃道士之家,我們進去會他一會也好,他與我們衣冠雖別,修行一般。」沙僧道:「說得是。一則進去看看景致,二來也當撒貨頭口。看方便處,安排些齋飯,與師父吃。」

長老依言,四眾共入。但見二門上有一對春聯:「黃芽白雪神仙府;瑤草琪花羽士家。」行者笑道:「這個是燒茅煉藥,弄爐火,提罐子的道士。」三藏捻他一把道:「謹言,謹言。我們不與他相識,又不認親,左右暫時一會,管他怎的?」說不了,進了二門,只見那正殿謹閉,東廊下坐著一個道士,在那裡丸藥。你看他怎生打扮:
\begin{quote}
戴一頂紅豔豔戧金冠,穿一領黑淄淄烏皂服,踏一雙綠陣陣雲頭履,繫一條黃拂拂呂公絛。面如瓜鐵,目若朗星。準頭高大類回回,唇口翻張如達達。道心一片隱轟雷,伏虎降龍真羽士。
\end{quote}

三藏見了,厲聲高叫道:「老神仙,貧僧問訊了。」那道士猛擡頭,一見心驚,丟了手中之藥,按簪兒,整衣服,降階迎接道:「老師父,失迎了。請裡面坐。」長老歡喜上殿。推開門,見有三清聖像,供桌有爐有香。即拈香注爐,禮拜三匝,方與道士行禮。遂至客位中,同徒弟們坐下。急喚仙童看茶。當有兩個小童,即入裡邊,尋茶盤,洗茶盞,擦茶匙,辦茶果,忙忙的亂走,早驚動那幾個冤家。

原來那盤絲洞七個女怪與這道士同堂學藝。自從穿了舊衣,喚出兒子,徑來此處。正在後面裁剪衣服,忽見那童子看茶,便問道:「童兒,有甚客來了,這般忙冗?」仙童道:「適間有四個和尚進來,師父教來看茶。」女怪道:「可有個白胖和尚?」道:「有。」又問:「可有個長嘴大耳朵的?」道:「有。」女怪道:「你快去遞了茶,對你師父丟個眼色,著他進來,我有要緊的話說。」果然那仙童將五杯茶拿出去,道士斂衣,雙手拿一杯遞與三藏,然後與八戒、沙僧、行者。茶罷,收鍾。小童丟個眼色,那道士就欠身道:「列位請坐。」教:「童兒,放了茶盤陪侍。等我去去就來。」此時長老與徒弟們並一個小童,出殿上觀玩不題。

卻說道士走進方丈中,只見七個女子齊齊跪倒,叫:「師兄,師兄,聽小妹子一言。」道士用手攙起道:「你們早間來時,要與我說甚麼話,可可的今日丸藥,這枝藥忌見陰人,所以不曾答你。如今又有客在外面,有話且慢慢說罷。」眾怪道:「告稟師兄:這樁事,專為客來,方敢告訴;若客去了,縱說也沒用了。」道士笑道:「你看賢妹說話,怎麼專為客來才說?卻不瘋了?且莫說我是個清靜修仙之輩,就是個俗人家,有妻子老小家務事,也等客去了再處。怎麼這等不賢,替我裝幌子哩?且讓我出去。」眾怪又一齊扯住道:「師兄息怒。我問你,前邊那客是那方來的?」道士唾著臉,不答應。眾怪道:「方才小童進來取茶,我聞得他說,是四個和尚。」道士作怒道:「和尚便怎麼?」眾怪道:「四個和尚,內有一個白面胖的,有一個長嘴大耳的,師兄可曾問他是那裡來的?」道士道:「內中是有這兩個,你怎麼知道?想是在那裡見他來?」

女子道:「師兄原不知這個委曲。那和尚乃唐朝差往西天取經去的。今早到我洞裡化齋,委是妹子們聞得唐僧之名,將他拿了。」道士道:「你拿他怎的?」女子道:「我們久聞人說,唐僧乃十世修行的真體,有人吃他一塊肉,延壽長生,故此拿了他。後被那個長嘴大耳朵的和尚把我們攔在濯垢泉裡,先搶了衣服,後弄本事,強要同我等洗浴,也止他不住。他就跳下水,變作一個鮎魚,在我們腿襠裡鑽來鑽去,欲行姦騙之事,果有十分憊𪬯。他又跳出水去,現了本相。見我們不肯相從,他就使一柄九齒釘鈀,要傷我們性命。若不是我們有些見識,幾乎遭他毒手,故此戰兢兢逃生。又著你愚外甥與他敵鬥,不知存亡如何。我們特來投兄長,望兄長念昔日同窗之雅,與我今日做個報冤之人。」那道士聞此言,卻就惱恨,遂變了聲色道:「這和尚原來這等無禮,這等憊𪬯。你們都放心,等我擺佈他。」眾女子謝道:「師兄如若動手,等我們都來相幫打他。」道士道:「不用打,不用打。常言道:『一打三分低。』你們都跟我來。」

眾女子相隨左右。他入房內,取了梯子,轉過床後,爬上屋梁,拿下一個小皮箱兒。那箱兒有八寸高下,一尺長短,四寸寬窄,上有一把小銅鎖兒鎖住。即於袖中拿出一方鵝黃綾汗巾兒來,汗巾鬚上繫著一把小鑰匙兒。開了鎖,取出一包兒藥來。此藥乃是:
\begin{quote}
山中百鳥糞,掃積上千斤。
是用銅鍋煮,煎熬火候勻。
千斤熬一杓,一杓煉三分。
三分還要炒,再煅再重熏。
製成此毒藥,貴似寶和珍。
如若嘗他味,入口見閻君。
\end{quote}

道士對七個女子道:「妹妹,我這寶貝,若與凡人吃,只消一釐,入腹就死;若與神仙吃,也只消三釐就絕;這些和尚,只怕也有些道行,須得三釐。快取等子來。」內一女子急拿了一把等子道:「稱出一分二釐,分作四分。」卻拿了十二個紅棗兒,將棗掐破些兒,揌上一釐,分在四隻茶鍾內;又將兩個黑棗兒做一個茶鍾,著一個托盤安了對眾女說:「等我去問他,不是唐朝的便罷;若是唐朝來的,就教換茶,你卻將此茶令童兒拿出。但吃了,個個身亡,就與你報了此讎,解了煩惱也。」七女感激不盡。

那道士換了一件衣服,虛禮謙恭,走將出去,請唐僧等又至客位坐下,道:「老師父莫怪。適間去後面吩咐小徒,教他們挑些青菜、蘿蔔,安排一頓素齋供養,所以失陪。」三藏道:「貧僧素手進拜,怎麼敢勞賜齋?」道士笑云:「你我都是出家人,見山門就有三升俸糧,何言素手?敢問老師父,在何寶山?到此何幹?」三藏道:「貧僧乃東土大唐駕下差往西天大雷音寺取經者。卻才路過仙宮,竭誠進拜。」道士聞言,滿面生春道:「老師乃忠誠大德之佛,小道不知,失於遠候,恕罪,恕罪。」叫:「童兒,快去換茶來,一廂作速辦齋。」那小童走將進去,眾女子招呼他來道:「這裡有現成好茶,拿出去。」那童子果然將五鍾茶拿出。道士連忙雙手拿一個紅棗兒茶鍾奉與唐僧。他見八戒身軀大,就認做大徒弟;沙僧認做二徒弟;見行者身量小,認做三徒弟。所以第四鍾才奉與行者。

行者眼乖,接了茶鍾,早已見盤子裡那茶鍾是兩個黑棗兒。他道:「先生,我與你穿換一杯。」道士笑道:「不瞞長老說,山野中貧道士,茶果一時不備,才然在後面親自尋果子,止有這十二個紅棗,做四鍾茶奉敬。小道又不可空陪,所以將兩個下色棗兒作一杯奉陪。此乃貧道恭敬之意也。」行者笑道:「說那裡話?古人云:『在家不是貧?路貧貧殺人。』你是住家兒的,何以言貧!像我們這行腳僧,才是真貧哩。我和你換換。我和你換換。」三藏聞言道:「悟空,這仙長實乃愛客之意,你吃了罷,換怎的?」行者無奈,將左手接了,右手蓋住,看著他們。

卻說那八戒一則饑,二則渴,原來是食腸大大的,見那鍾子裡有三個紅棗兒,拿起來嘓的都咽在肚裡。師父也吃了,沙僧也吃了。一霎時,只見八戒臉上變色,沙僧滿眼流淚,唐僧口中吐沫。他們都坐不住,暈倒在地。

這大聖情知是毒,將茶鍾手舉起來,望道士劈臉一摜。道士將袍袖隔起,噹的一聲,把個鍾子跌得粉碎。道士怒道:「你這和尚,十分村魯!怎麼把我鍾子捽了?」行者罵道:「你這畜生!你看我那三個人是怎麼說?我與你有甚相干,你卻將毒藥茶藥倒我的人?」道士道:「你這個村畜生闖下禍來,你豈不知?」行者道:「我們才進你門,方敘了坐次,道及鄉貫,又不曾有個高言,那裡闖下甚禍?」道士道:「你可曾在盤絲洞化齋麼?你可曾在濯垢泉洗澡麼?」行者道:「濯垢泉乃七個女怪,你既說出這話,必定與他苟合,必定也是妖精。不要走,吃我一棒。」好大聖,去耳朵裡摸出金箍棒,幌一幌,碗來粗細,望道士劈臉打來;那道士急轉身躲過,取一口寶劍來迎。

他兩個廝罵廝打,早驚動那裡邊的女怪。他七個一擁出來,叫道:「師兄且莫勞心,待小妹子拿他。」行者見了,越生嗔怒,雙手掄鐵棒,丟開解數,滾將進去亂打。只見那七個敞開懷,腆著雪白肚子,臍孔中作出法來:骨都都絲繩亂冒,搭起一個天篷,把行者蓋在底下。行者見事不諧,即翻身念聲咒語,打個觔斗,撲的撞破天篷走了。忍著性氣,淤淤的立在空中看處,見那怪絲繩晃亮,穿穿道道,卻是穿梭的經緯,頃刻間,把黃花觀的樓臺殿閣都遮得無影無形。行者道:「利害,利害。早是不曾著他手。怪道豬八戒跌了若干。似這般怎生是好?我師父與師弟卻又中了毒藥。這夥怪合意同心,卻不知是個甚來歷,待我還去問那土地神也。」

好大聖,按落雲頭,捻著訣,念聲「唵」字真言,把個土地老兒又拘來了。戰兢兢跪下路旁,叩頭道:「大聖,你去救你師父的,為何又轉來也?」行者道:「早間救了師父,前去不遠,遇一座黃花觀,我與師父等進去看看,那觀主迎接。才敘話間,被他把毒藥茶藥倒我師父等。我幸不曾吃茶,使棒就打。他卻說出盤絲洞化齋,濯垢泉洗澡之事,我就知那廝是怪。才舉手相敵,只見那七個女子跑出,吐放絲繩,老孫虧有見識走了。我想你在此間為神,定知他的來歷,是個甚麼妖精?老實說來,免打。」土地叩頭道:「那妖精到此,住不上十年。小神自三年前檢點之後,方見他的本相,乃是七個蜘蛛精。他吐那些絲繩,乃是蛛絲。」行者聞言,十分歡喜道:「據你說,卻是小可。既這般,你回去,等我作法降他也。」那土地叩頭而去。

行者卻到黃花觀外,將尾巴上毛捋下七十根,吹口仙氣,叫:「變!」即變做七十個小行者;又將金箍棒吹口仙氣,叫:「變!」即變做七十一條雙角叉兒棒。每一個小行者與他一根,他自家使一根,站在外邊,將叉兒攪那絲繩,一齊著力,打個號子,把那絲繩都攪斷,各攪了有十餘斤。裡面拖出七個蜘蛛,足有巴斗大小的身軀。一個個攢著手腳,索著頭,只叫:「饒命,饒命。」此時七十個小行者,按住七個蜘蛛,那裡肯放。行者道:「且不要打他,只教還我師父、師弟來。」那怪厲聲高叫道:「師兄,還他唐僧,救我命也。」那道士從裡邊跑出道:「妹妹,我要吃唐僧哩,救不得你了。」行者聞言,大怒道:「你既不還我師父,且看你妹妹的樣子。」好大聖,把叉兒棒幌一幌,復了一根鐵棒,雙手舉起,把七個蜘蛛精盡情打爛。

卻又將尾巴搖了兩搖,收了毫毛,單身掄棒,趕入裡邊來打道士。那道士見他打死了師妹,心甚不忍,即發狠舉劍來迎。這一場各懷忿怒,一個個大展神通。這一場好殺:
\begin{quote}
妖精掄寶劍,大聖舉金箍。都為唐朝三藏,先教七女嗚呼。如今大展經綸手,施威弄法逞金吾。大聖神光壯,妖仙膽氣粗。渾身解數如花錦,雙手騰那似轆轤。乒乓劍棒響。慘淡野雲浮。劖言語,使機謀,一來一往如畫圖。殺得風響沙飛狼虎怕,天昏地暗斗星無。
\end{quote}

那道士與大聖戰經五六十合,漸覺手軟。一時間鬆了筋節,便解開衣帶,忽辣的響一聲,脫了皂袍。行者笑道:「我兒子,打不過人,就脫剝了也是不能夠的。」

原來這道士剝了衣裳,把手一齊擡起,只見那兩脅下有一千隻眼,眼中迸放金光,十分利害:
\begin{quote}
森森黃霧,艷艷金光。森森黃霧,兩邊脅下似噴雲;艷艷金光,千隻眼中如放火。左右卻如金桶,東西猶似銅鐘。此乃妖仙施法力,道士顯神通:幌眼迷天遮日月,罩人爆燥氣朦朧;把個齊天孫大聖,困在金光黃霧中。
\end{quote}

行者慌了手腳,只在那金光影裡亂轉,向前不能舉步,退後不能動腳,卻便似在個桶裡轉的一般。無奈又爆燥不過,他急了,往上著實一跳,卻撞破金光,撲的跌了一個倒栽蔥,覺道撞的頭疼。急伸頭摸摸,把頂梁皮都撞軟了。自家心焦道:「晦氣,晦氣,這顆頭今日也不濟了。常時刀砍斧剁,莫能傷損,卻怎麼被這金光撞軟了皮肉?久以後定要貢膿。縱然好了,也是個破傷風。」一會家爆燥難禁,卻又自家計較道:「前去不得,後退不得,左行不得,右行不得,往上又撞不得,卻怎麼好?往下走他娘罷。」

好大聖,念個咒語,搖身一變,變做個穿山甲,又名鯪鯉鱗。真個是:
\begin{quote}
四隻鐵爪,鑽山碎石如撾粉;滿身鱗甲,破嶺穿巖似切蔥。兩眼光明,好便似雙星晃亮;一嘴尖利,勝強如鋼鑽金錐。藥中有性穿山甲,俗語呼為鯪鯉鱗。
\end{quote}

你看他硬著頭,往地下一鑽,就鑽了有二十餘里,方才出頭。原來那金光只罩得十餘里。出來現了本相,力軟觔麻,渾身疼痛,止不住眼中流淚。忽失聲叫道:「師父啊,
\begin{quote}
當年秉教出山中,共往西來苦用工。
大海洪波無恐懼,陽溝之內卻遭風。」
\end{quote}

美猴王正當悲切,忽聽得山背後有人啼哭,即欠身揩了眼淚,回頭觀看。但見一個婦人,身穿重孝,左手托一盞涼漿水飯,右手執幾張燒紙黃錢,從那廂一步一聲,哭著走來。行者點頭嗟嘆道:「正是:『流淚眼逢流淚眼,斷腸人遇斷腸人。』這一個婦人,不知所哭何事?待我問他一問。」那婦人不一時走上前來,迎著行者。行者躬身問道:「女菩薩,你哭的是甚人?」婦人噙淚道:「我丈夫因與黃花觀觀主買竹竿爭講,被他將毒藥茶藥死,我將這陌紙錢燒化,以報夫婦之情。」行者聽言,眼中流淚。那女子見了,作怒道:「你甚無知,我為丈夫煩惱生悲,你怎麼淚眼愁眉,欺心戲我?」

行者躬身道:「女菩薩息怒。我本是東土大唐欽差御弟唐三藏大徒弟孫悟空行者。因往西天,行過黃花觀歇馬。那觀中道士,不知是個甚麼妖精,他與七個蜘蛛精結為兄妹。蜘蛛精在盤絲洞要害我師父,是我與師弟八戒、沙僧救解得脫。那蜘蛛精走到他這裡,背了是非,說我等有欺騙之意。道士將毒藥茶藥倒我師父、師弟共三人,連馬四口,陷在他觀裡。惟我不曾吃他茶,將茶鍾摜碎,他就與我相打。正嚷時,那七個蜘蛛精跑出來吐放絲繩,將我網住,是我使法力走脫。問及土地,說他本相。我卻又使分身法攪絕絲繩,拖出妖來,一頓棒打死。這道士即與他報仇,舉寶劍與我相鬥。鬥經六十回合,他敗了陣,隨脫了衣裳,兩脅下放出千隻眼,有萬道金光,把我罩定。所以進退兩難,才變做一個鯪鯉鱗,從地下鑽出來。正自悲切,忽聽得你哭,故此相問。因見你為丈夫有此紙錢報答,我師父喪身,更無一物相酬,所以自怨生悲,豈敢相戲。」

那婦女放下水飯、紙錢,對行者陪禮道:「莫怪,莫怪,我不知你是被難者。才據你說將起來,你不認得那道士。他本是個百眼魔君,又喚做多目怪。你既然有此變化,脫得金光,戰得許久,必定有大神通,卻只是還近不得那廝。我教你去請一位聖賢,他能破得金光,降得道士。」行者聞言,連忙唱喏道:「女菩薩知此來歷,煩為指教指教。果是那位聖賢,我去請求,救我師父之難,就報你丈夫之仇。」婦人道:「我就說出來,你去請他,降了道士,只可報仇而已,恐不能救你師父。」行者道:「怎不能救?」婦人道:「那廝毒藥最狠:藥倒人,三日之間,骨髓俱爛。你此往回恐遲了,故不能救。」行者道:「我會走路,憑他多遠,只消半日。」女子道:「你既會走路,聽我說:此處到那裡有千里之遙。那廂有一座山,名喚紫雲山。山中有個千花洞,洞中有位聖賢,喚做毘藍婆,他能降得此怪。」行者道:「那山坐落何方?卻從何方去?」女子用手指定道:「那直南上便是。」行者回頭看時,那女子早不見了。行者慌忙禮拜道:「是那位菩薩?我弟子鑽昏了,不能相識,千乞留名,好謝。」只見那半空中叫道:「大聖,是我。」行者急擡頭看處,原是黎山老姆。趕至空中謝道:「老姆從何來指教我也?」老姆道:「我才自龍華會上回來,見你師父有難,假做孝婦,借夫喪之名,免他一死。你快去請他,但不可說出是我指教,那聖賢有些多怪人。」

行者謝了,辭別,把觔斗雲一縱,隨到紫雲山上。按定雲頭,就見那千花洞。那洞外:
\begin{quote}
青松遮勝境,翠柏繞仙居。
綠柳盈山道,奇花滿澗渠。
香蘭圍石屋,芳草映巖嵎。
流水連溪碧,雲封古樹虛。
野禽聲聒聒,幽鹿步徐徐。
修竹枝枝秀,紅梅葉葉舒。
寒鴉棲古樹,春鳥噪高樗。
夏麥盈田廣,秋禾遍地餘。
四時無葉落,八節有花如。
每生瑞藹連霄漢,常放祥雲接太虛。
\end{quote}

這大聖喜喜歡歡走將進去,一程一節,看不盡無邊的景致。直入裡面,更沒個人兒,靜靜悄悄的,雞犬之聲也無。心中暗道:「這聖賢想是不在家了。」又進數里看時,見一個女道姑坐在榻上。你看他怎生模樣:
\begin{quote}
頭戴五花納錦帽,身穿一領織金袍。
腳踏雲尖鳳頭履,腰繫攢絲雙穗絛。
面似秋容霜後老,聲如春燕社前嬌。
腹中久諳三乘法,心上常修四諦饒。
悟出空空真正果,煉成了了自逍遙。
正是千花洞裡佛,毘藍菩薩姓名高。
\end{quote}

行者止不住腳,近前叫道:「毘藍婆菩薩,問訊了。」那菩薩即下榻,合掌回禮道:「大聖,失迎了。你從那裡來的?」行者道:「你怎麼就認得我是大聖?」毘藍婆道:「你當年大鬧天宮時,普地裡傳了你的形像,誰人不知,那個不識?」行者道:「正是:『好事不出門,惡事傳千里。』像我如今皈正佛門,你就不曉的了?」毘藍道:「幾時皈正?恭喜,恭喜。」行者道:「近能脫命,保師父唐僧上西天取經,師父遇黃花觀道士,將毒藥茶藥倒。我與那廝賭鬥,他就放金光罩住我,是我使神通走脫了。聞菩薩能滅他的金光,特來拜請。」菩薩道:「是誰與你說的?我自赴了盂蘭會,到今三百餘年,不曾出門。我隱姓埋名,更無一人得知,你卻怎麼知道?」行者道:「我是個地裡鬼,不管那裡,自家都會訪著。」毘藍道:「也罷,也罷。我本當不去,奈蒙大聖下臨,不可滅了求經之善,我和你去來。」

行者稱謝了,道:「我忒無知,擅自催促。但不知曾帶甚麼兵器?」菩薩道:「我有個繡花針兒,能破那廝。」行者忍不住道:「老姆誤了我,早知是繡花針,不須勞你,就問老孫要一擔也是有的。」毘藍道:「你那繡花針,無非是鋼鐵金針,用不得。我這寶貝,非鋼非鐵非金,乃我小兒日眼裡煉成的。」行者道:「令郎是誰?」毘藍道:「小兒乃昴日星官。」行者驚駭不已。早望見金光艷艷,即回向毘藍道:「金光處便是黃花觀也。」毘藍隨於衣領裡取出一個繡花針,似眉毛粗細,有五六分長短,拈在手,望空拋去。少時間,響一聲,破了金光。行者喜道:「菩薩,妙哉,妙哉!尋針,尋針。」毘藍托在手掌內道:「這不是?」行者卻同按下雲頭,走入觀裡,只見那道士合了眼,不能舉步。行者罵道:「你這潑怪裝瞎子哩。」耳朵裡取出棒來就打。毘藍扯住道:「大聖莫打,且看你師父去。」

行者徑至後面客位裡看時,他三人都睡在地上吐痰吐沫哩。行者垂淚道:「卻怎麼好?卻怎麼好?」毘藍道:「大聖莫悲。也是我今日出門一場,索性積個陰德。我這裡有解毒丹,送你三丸。」行者轉身拜求。那菩薩袖中取出一個破紙包兒,內將三粒紅丸子遞與行者,教放入口裡。行者把藥扳開他們牙關,每人揌了一丸。須臾,藥味入腹,便就一齊嘔噦,遂吐出毒味,得了性命。那八戒先爬起道:「悶殺我也。」三藏、沙僧俱醒了道:「好暈也。」行者道:「你們那茶裡中了毒了。虧這毘藍菩薩搭救,快都來拜謝。」三藏欠身整衣謝了。

八戒道:「師兄,那道士在那裡?等我問他一問,為何這般害我?」行者把蜘蛛精上項事說了一遍。八戒發狠道:「這廝既與蜘蛛為姊妹,定是妖精。」行者指道:「他在那殿外立定裝瞎子哩。」八戒拿鈀就築,又被毘藍止住道:「天蓬息怒。大聖知我洞裡無人,待我收他去看守門戶也。」行者道:「感蒙大德,豈不奉承。但只是教他現本像,我們看看。」毘藍道:「容易。」即上前用手一指,那道士撲的倒在塵埃,現了原身,乃是一條七尺長短的大蜈蚣精。毘藍使小指頭挑起,駕祥雲,徑轉千花洞去。

八戒打仰道:「這媽媽兒卻也利害,怎麼就降這般惡物?」行者笑道:「我問他有甚兵器破他金光,他道有個繡花針兒,是他兒子在日眼裡煉的。及問他令郎是誰,他道是昴日星官。我想昴日星是隻公雞,這老媽媽必定是個母雞。雞最能降蜈蚣,所以能收伏也。」

三藏聞言,頂禮不盡。教:「徒弟們,收拾去罷。」那沙僧即在裡面尋了些米糧,安排了些齋,俱飽餐一頓。牽馬挑擔,請師父出門。行者從他廚中放了一把火,把一座觀霎時燒得煨燼,卻拽步長行。正是:
\begin{quote}
唐僧得命感毘藍,了性消除多目怪。
\end{quote}

畢竟向前去還有甚麼事體,且聽下回分解。
