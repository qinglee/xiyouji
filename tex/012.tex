
\chapter{唐王秉誠修大會 觀音顯聖化金蟬}

卻說鬼使同劉全夫妻二人出了陰司,那陰風遶遶,徑到了長安大國,將劉全的魂靈推入金亭館裡,將翠蓮的靈魂帶進皇宮內院。只見那玉英宮主正在花陰下,徐步綠苔而行,被鬼使撲個滿懷,推倒在地,活捉了他魂,卻將翠蓮的魂靈推入玉英身內。鬼使回轉陰司不題。

卻說宮院中的大小侍婢見玉英跌死,急走金鑾殿,報與三宮皇后道:「宮主娘娘跌死也。」皇后大驚,隨報太宗。太宗聞言,點頭嘆曰:「此事信有之也。朕曾問十代閻君:『老幼安乎?』他道:『俱安,但恐御妹壽促。』果中其言。」合宮人都來悲切,盡到花陰下看時,只見那宮主微微有氣。唐王道:「莫哭!莫哭!休驚了他。」遂上前將御手扶起頭來,叫道:「御妹甦醒甦醒。」那宮主忽的翻身,叫:「丈夫慢行,等我一等。」太宗道:「御妹,是我等在此。」宮主擡頭睜眼觀看道:「你是誰人,敢來扯我?」太宗道:「是你皇兄、皇嫂。」宮主道:「我那裡得個甚麼皇兄、皇嫂?我娘家姓李,我的乳名喚做李翠蓮,我丈夫姓劉名全,兩口兒都是均州人氏。因為我三個月前拔金釵在門首齋僧,我丈夫怪我擅出內門,不遵婦道,罵了我幾句,是我氣塞胸堂,將白綾帶懸梁縊死,撇下一雙兒女,晝夜悲啼。今因我V夫被唐王欽差,赴陰司進瓜果,閻王憐憫,放我夫妻回來。他在前走,因我來遲,趕不上他,我絆了一跌。你等無禮!不知姓名,怎敢扯我?」太宗聞言,與眾宮人道:「想是御妹跌昏了,胡說哩。」傳旨教太醫院進湯藥,將玉英扶入宮中。

唐王當殿,忽有當駕官奏道:「萬歲,今有進瓜果人劉全還魂,在朝門外等旨。」唐王大驚,急傳旨,將劉全召進,俯伏丹墀。太宗問道:「進瓜果之事何如?」劉全道:「臣頂瓜果,徑至鬼門關,引上森羅殿,見了那十代閻君,將瓜果奉上,備言我王慇懃致謝之意。閻君甚喜,多多拜上我王道:『真是個有信有德的太宗皇帝!』」唐王道:「你在陰司見些甚麼來?」劉全道:「臣不曾遠行,沒見甚的,只聞得閻王問臣鄉貫、姓名。臣將棄家捨子,因妻縊死,願來進瓜之事,說了一遍。他急差鬼使,引過我妻,就在森羅殿下相會。一壁廂又檢看死生文簿,說我夫妻都有登仙之壽,便差鬼使送回。臣在前走,我妻後行,幸得還魂。但不知妻投何所。」唐王驚問道:「那閻王可曾說你妻甚麼?」劉全道:「閻王不曾說甚麼,只聽得鬼使說:『李翠蓮歸陰日久,屍首無存。』閻王道:『唐御妹李玉英今該促死,教翠蓮即借玉英屍還魂去罷。』臣不知『唐御妹』是甚地方,家居何處,我還未曾得去找尋哩。」

唐王聞奏,滿心歡喜,當對多官道:「朕別閻君,曾問宮中之事。他言:『老幼俱安,但恐御妹壽促。』卻才御妹玉英花陰下跌死,朕急扶看,須臾甦醒,口叫:『丈夫慢行,等我一等。』朕只道是他跌昏了胡言。又問他詳細,他說的話,與劉全一般。」魏徵奏道:「御妹偶爾壽促,少甦醒即說此言,此是劉全妻借屍還魂之事。此事也有,可請宮主出來,看他有甚話說。」唐王道:「朕才命太醫院去進藥,不知何如。」便教妃嬪入宮去請。那宮主在裡面亂嚷道:「我吃甚麼藥?這裡那是我家?我家是清涼瓦屋,不像這個害黃病的房子,花狸狐哨的門扇,放我出去!放我出去!」

正嚷處,只見四五個女官、兩三個太監扶著他,直至殿上。唐王道:「你可認得你丈夫麼?」玉英道:「說那裡話,我兩個從小兒的結髮夫妻,與他生男育女,怎的不認得?」唐王叫內官攙他下去。那宮主下了寶殿,直至白玉階前,見了劉全,一把扯住道:「丈夫,你往那裡去,就不等我一等?我跌了一跌,被那些沒道理的人圍住我嚷,這是怎的說?」那劉全聽他說的話是妻之言,觀其人非妻之面,不敢相認。唐王道:「這正是山崩地裂有人見,捉生替死卻難逢。」好一個有道的君王,即將御妹的妝奩、衣物、首飾,盡賞賜了劉全,就如陪嫁一般。又賜與他永免差徭的御旨,著他帶領御妹回去。他夫妻兩個便在階前謝了恩,歡歡喜喜還鄉。有詩為證:
\begin{quote}
人生人死是前緣,短短長長各有年。
劉全進瓜回陽世,借屍還魂李翠蓮。
\end{quote}

他兩個辭了君王,徑來均州城裡,見舊家業、兒女俱好,兩口兒宣揚善果不題。

卻說那尉遲恭將金銀一庫,上河南開封府訪看,相良原來賣水為活,同妻張氏在門首販賣烏盆瓦器營生,但賺得些錢兒,只以盤纏為足,其多少齋僧布施,買金銀紙錠,記庫焚燒,故有此善果臻身。陽世間是一條好善的窮漢,那世裡卻是個積玉堆金的長者。尉遲恭將金銀送上他門,諕得那相公、相婆魂飛魄散。又兼有本府官員,茅舍外車馬駢集。那老兩口子如痴如啞,跪在地下,只是磕頭禮拜。尉遲恭道:「老人家請起。我雖是個欽差官,卻齎著我王的金銀送來還你。」他戰兢兢的答道:「小的沒有甚麼金銀放債,如何敢受這不明之財?」尉遲恭道:「我也訪得你是個窮漢,只是你齋僧布施,儘其所用,就買辦金銀紙錠,燒記陰司,陰司裡有你積下的錢鈔。是我太宗皇帝死去三日,還魂復生,曾在那陰司裡借了你一庫金銀,今此照數送還與你。你可一一收下,等我好去回旨。」那相良兩口兒只是朝天禮拜,那裡敢受。道:「小的若受了這些金銀,就死得快了。雖然是燒紙記庫,此乃冥冥之事;況萬歲爺爺那世裡借了金銀,有何憑據?我決不敢受。」尉遲恭道:「陛下說,借你的東西,有崔判官作保可證。你收下罷。」相良道:「就死也是不敢受的。」

尉遲恭見他苦苦推辭,只得具本差人啟奏。太宗見了本,知相良不受金銀,道:「此誠為善良長者。」即傳旨教胡敬德將金銀與他修理寺院,起蓋生祠,請僧作善,就當還他一般。旨意到日,敬德望闕謝恩宣旨,眾皆知之。遂將金銀買到城裡軍民無礙的地基一段,周圍有五十畝寬闊,在上興工,起蓋寺院,名「敕建相國寺」,左有相公、相婆的生祠,鐫碑刻石,上寫著「尉遲恭監造」,即今「大相國寺」是也。

工完回奏,太宗甚喜。卻又聚集多官,出榜招僧,修建水陸大會,超度冥府孤魂。榜行天下,著各處官員推選有道的高僧,上長安做會。那消個月之期,天下多僧俱到。唐王傳旨,著太史丞傅奕選舉高僧,修建佛事。傅奕聞旨,即上疏止浮圖,以言無佛。表曰:
\begin{quote}
西域之法,無君臣父子,以三塗六道,蒙誘愚蠢。追既往之罪,窺將來之福,口誦梵言,以圖偷免。且生死壽夭,本諸自然;刑德威福,係之人主。今聞俗徒矯託,皆云由佛。自五帝三王,未有佛法,君明臣忠,年祚長久。至漢明帝始立胡神,然惟西域桑門自傳其教。實乃夷犯中國,不足為信。
\end{quote}

太宗聞言,遂將此表擲付群臣議之。時有宰相蕭瑀,出班俯顖奏曰:「佛法興自屢朝,弘善遏惡,冥助國家,理無廢棄。佛,聖人也。非聖者無法,請寘嚴刑。」傅奕與蕭瑀論辨,言:「禮本於事親事君,而佛背親出家,以匹夫抗天子,以繼禮悖所親。蕭瑀不生於空桑,乃遵無父之教,正所謂非孝者無親。」蕭瑀但合掌曰:「地獄之設,正為是人。」太宗召太僕卿張道源、中書令張士衡,問佛事營福,其應何如。二臣對曰:「佛在清淨仁恕,果正佛空。周武帝以三教分次;大慧禪師有贊幽遠,歷眾供養而無不顯;五祖投胎,達摩現像。自古以來,皆云三教至尊而不可毀,不可廢。伏乞陛下聖鑒明裁。」太宗甚喜道:「卿之言合理。再有所陳者,罪之。」遂著魏徵與蕭瑀、張道源邀請諸佛,選舉一名有大德行者作壇主,設建道場。眾皆頓首謝恩而退。自此時出了法律:但有毀僧謗佛者,斷其臂。

次日三位朝臣,聚眾僧,在那山川壇裡,逐一從頭查選,內中選得一名有德行的高僧。你道他是誰人?
\begin{quote}
靈通本諱號金蟬,只為無心聽佛講。
轉托塵凡苦受磨,降生世俗遭羅網。
投胎落地就逢兇,未出之前臨惡黨。
父是海州陳狀元,外公總管當朝長。
出身命犯落江星,順水隨波逐浪泱。
海島金山有大緣,遷安和尚將他養。
年方十八認親娘,特赴京都求外長。
總管開山調大軍,洪州剿寇誅兇黨。
狀元光蕊脫天羅,子父相逢堪賀獎。
復謁當今受主恩,凌煙閣上賢名響。
恩官不受願為僧,洪福沙門將道訪。
小字江流古佛兒,法名喚做陳玄奘。
\end{quote}

當日對眾舉出玄奘法師。這個人自幼為僧,出娘胎,就持齋受戒。他外公見是當朝一路總管殷開山。他父親陳光蕊中狀元,官拜文淵殿大學士。一心不愛榮華,只喜修持寂滅。查得他根源又好,德行又高;千經萬典,無所不通;佛號仙音,無般不會。

當時三位引至御前,揚塵舞蹈。拜罷奏曰:「臣瑀等蒙聖旨,選得高僧一名陳玄奘。」太宗聞其名,沉思良久道:「可是學士陳光蕊之兒玄奘否?」江流兒叩頭曰:「臣正是。」太宗喜道:「果然舉之不錯,誠為有德行有禪心的和尚。朕賜你左僧綱,右僧綱,天下大闡都僧綱之職。」玄奘頓首謝恩,受了大闡官爵。又賜五彩織金袈裟一件、毘盧帽一頂。教他用心再拜明僧,排次闍黎班首,書辦旨意,前赴化生寺,擇定吉日良時,開演經法。

玄奘再拜領旨而出,遂到化生寺裡,聚集多僧,打造禪榻,裝修功德,整理音樂。選得大小明僧共計一千二百名,分派上中下三堂。諸所佛前,物件皆齊,頭頭有次。選到本年九月初三日黃道良辰,開啟做七七四十九日水陸大會。即具表申奏。太宗及文武國戚皇親,俱至期赴會,拈香聽講。有詩為證。詩曰:
\begin{quote}
龍集貞觀正十三,王宣大眾把經談。
道場開演無量法,雲霧光乘大願龕。
御敕垂恩修上剎,金蟬脫殼化西涵。
普施善果超沉沒,秉教宣揚前後三。
\end{quote}

貞觀十三年,歲次己巳,九月甲戌,初三日,癸卯良辰,陳玄奘大闡法師聚集一千二百名高僧,都在長安城化生寺開演諸品妙經。那皇帝早朝已畢,帥文武多官,乘鳳輦龍車,出離金鑾寶殿,徑上寺來拈香。怎見那鑾駕?真個是:
\begin{quote}
一天瑞氣,萬道祥光。仁風輕淡蕩,化日麗非常。千官環佩分前後,五衛旌旗列兩旁。執金瓜,擎斧鉞,雙雙對對;絳紗燭,御爐香,靄靄堂堂。龍飛鳳舞,鶚薦鷹揚。聖明天子正,忠義大臣良。介福千年過舜禹,昇平萬代賽堯湯。又見那曲柄傘,滾龍袍,輝光相射;玉連環,彩鳳扇,瑞靄飄揚。珠冠玉帶,紫綬金章。護駕軍千隊,扶輿將兩行。這皇帝沐浴虔誠尊敬佛,皈依善果喜拈香。
\end{quote}

唐王大駕早到寺前,吩咐住了音樂響器。下了車輦,引著多官,拜佛拈香。三匝已畢,擡頭觀看,果然好座道場。但見:
\begin{quote}
幢幡飄舞,寶蓋飛輝。幢幡飄舞,凝空道道彩霞搖;寶蓋飛輝,映日翩翩紅電徹。世尊金像貌臻臻,羅漢玉容威烈烈。瓶插仙花,爐焚檀降。瓶插仙花,錦樹輝輝漫寶剎;爐焚檀降,香雲靄靄透清霄。時新果品砌朱盤,奇樣糖酥堆彩案。高僧羅列誦真經,願拔孤魂離苦難。
\end{quote}

太宗文武俱各拈香,拜了佛祖金身,參了羅漢。又見那大闡都綱陳玄奘法師引眾僧羅拜唐王。禮畢,分班各安禪位。法師獻上濟孤榜文與太宗看。榜曰:
\begin{quote}
至德渺茫,禪宗寂滅。清淨靈通,周流三界。千變萬化,統攝陰陽。體用真常,無窮極矣。觀彼孤魂,深宜哀愍。此奉太宗聖命:選集諸僧,參禪講法。大開方便門庭,廣運慈悲舟楫,普濟苦海群生,脫免沉痾六趣。引歸真路,普玩鴻濛;動止無為,混成純素。仗此良因,邀賞清都絳闕;乘吾勝會,脫離地獄凡籠。早登極樂任逍遙,來往西方隨自在。
\end{quote}

詩曰:
\begin{quote}
一爐永壽香,幾卷超生籙。
無邊妙法宣,無際天恩沐。
冤孽盡消除,孤魂皆出獄。
願保我邦家,清平萬咸福。
\end{quote}

太宗看了,滿心歡喜,對眾僧道:「汝等秉立丹衷,切休怠慢佛事。待後功成完備,各各福有所歸,朕當重賞,決不空勞。」那一千二百僧,一齊頓首稱謝。當日三齋已畢,唐王駕回。待七日正會,復請拈香。時天色將晚,各官俱退。怎見得好晚?你看那:
\begin{quote}
萬里長空淡落暉,歸鴉數點下棲遲。
滿城燈火人煙靜,正是禪僧入定時。
\end{quote}

一宿晚景題過。次早,法師又昇坐,聚眾誦經不題。

卻說南海普陀山觀世音菩薩,自領了如來佛旨,在長安城訪察取經的善人,日久未逢真實有德行者。忽聞得太宗宣揚善果,選舉高僧,開建大會。又見得法師壇主,乃是江流兒和尚,正是極樂中降來的佛子,又是他原引送投胎的長老。菩薩十分歡喜,就將佛賜的寶貝捧上長街,與木叉貨賣。你道他是何寶貝?有一件錦襴異寶袈裟、九環錫杖。還有那金緊禁三個箍兒,密密藏收,以俟後用。只將袈裟、錫杖出賣。

長安城裡,有那選不中的愚僧,倒有幾貫村鈔。見菩薩變化個疥癩形容,身穿破衲,赤腳光頭,將袈裟捧定,豔豔生光,他上前問道:「那癩和尚,你的袈裟要賣多少價錢?」菩薩道:「袈裟價值五千兩,錫杖價值二千兩。」那愚僧笑道:「這兩個癩和尚是瘋子!是傻子!這兩件粗物,就賣得七千兩銀子?只是除非穿上身長生不老,就得成佛作祖,也值不得這許多!拿了去!賣不成!」

那菩薩更不爭吵,與木叉往前又走。行勾多時,來到東華門前,正撞著宰相蕭瑀散朝而回,眾頭踏喝開街道。那菩薩公然不避,當街上拿著袈裟,徑迎著宰相。宰相勒馬觀看,見袈裟豔豔生光,著手下人問那賣袈裟的要價幾何,菩薩道:「袈裟要五千兩,錫杖要二千兩。」蕭瑀道:「有何好處,值這般高價?」菩薩道:「袈裟有好處,有不好處;有要錢處,有不要錢處。」蕭瑀道:「何為好?何為不好?」菩薩道:「著了我袈裟,不入沉淪,不墮地獄,不遭惡毒之難,不遇虎狼之災,便是好處;若貪淫樂禍的愚僧,不齋不戒的和尚,毀經謗佛的凡夫,難見我袈裟之面,這便是不好處。」又問道:「何為要錢,不要錢?」菩薩道:「不遵佛法,不敬三寶,強買袈裟、錫杖,定要賣他七千兩,這便是要錢;若敬重三寶,見善隨喜,皈依我佛,承受得起,我將袈裟、錫杖情願送他,與我結個善緣,這便是不要錢。」蕭瑀聞言,倍添春色,知他是個好人。即便下馬,與菩薩以禮相見,口稱:「大法長老,恕我蕭瑀之罪。我大唐皇帝十分好善,滿朝的文武無不奉行。即今起建水陸大會,這袈裟正好與大都闡陳玄奘法師穿用。我和你入朝見駕去來。」

菩薩欣然從之,拽轉步,徑進東華門裡。黃門官轉奏,蒙旨宣至寶殿。見蕭瑀引著兩個疥癩僧人,立於階下,唐王問曰:「蕭瑀來奏何事?」蕭瑀俯伏階前道:「臣出了東華門前,偶遇二僧,乃賣袈裟與錫杖者。臣思法師玄奘可著此服,故領僧人啟見。」太宗大喜,便問那袈裟價值幾何。菩薩與木叉侍立階下,更不行禮,因問袈裟之價,答道:「袈裟五千兩,錫杖二千兩。」太宗道:「那袈裟有何好處,就值許多?」菩薩道:
\begin{quote}
這袈裟,龍披一縷,免大鵬吞噬之災;鶴掛一絲,得超凡入聖之妙。但坐處,有萬神朝禮;凡舉動,有七佛隨身。這袈裟,是冰蠶造練抽絲,巧匠翻騰為線,仙娥織就,神女機成,方方簇幅繡花縫。片片相幫堆錦簆。玲瓏散碎鬥妝花,色亮飄光噴寶豔。穿上滿身紅霧遶,脫來一段彩雲飛。三天門外透元光,五岳山前生寶氣。重重嵌就西番蓮,灼灼懸珠星斗象。四角上有夜明珠,攢頂間一顆祖母綠。雖無全照原本體,也有生光八寶攢。這袈裟,閑時折疊,遇聖才穿。閑時折疊,千層包裹透虹霓;遇聖才穿,驚動諸天神鬼怕。上邊有如意珠、摩尼珠、辟塵珠、定風珠;又有那紅瑪瑙、紫珊瑚、夜明珠、舍利子。偷月沁白,與日爭紅。條條仙氣盈空,朵朵祥光捧聖。條條仙氣盈空,照徹了天關;朵朵祥光捧聖,影遍了世界。照山川,驚虎豹;影海島,動魚龍。沿邊兩道銷金鎖,叩領連環白玉琮。
\end{quote}

詩曰:
\begin{quote}
三寶巍巍道可尊,四生六道盡評論。
明心解養人天法,見性能傳智慧燈。
護體莊嚴金世界,身心清淨玉壺冰。
自從佛製袈裟後,萬劫誰能敢斷僧?」
\end{quote}

唐王在那寶殿上聞言,十分歡喜。又問:「那和尚,九環杖有甚好處?」菩薩道:「我這錫杖,是那:
\begin{quote}
銅鑲鐵造九連環,九節仙藤永駐顏。
入手厭看青骨瘦,下山輕帶白雲還。
摩啊五祖遊天闕,羅卜尋娘破地關。
不染紅塵些子穢,喜伴神僧上玉山。」
\end{quote}

唐王聞言,即命展開袈裟,從頭細看,果然是件好物。道:「大法長老,實不瞞你。朕今大開善教,廣種福田,見在那化生寺聚集多僧,敷演經法。內中有一個大有德行者,法名玄奘。朕買你這兩件寶物,賜他受用。你端的要價幾何?」菩薩聞言,與木叉合掌皈依,道聲佛號,躬身上啟道:「既有德行,貧僧情願送他,決不要錢。」說罷,抽身便走。唐王急著蕭瑀扯住,欠身立於殿上,問曰:「你原說袈裟五千兩,錫杖二千兩,你見朕要買,就不要錢,敢是說朕心倚恃君位,強要你的物件?更無此理。朕照你原價奉償,卻不可推避。」菩薩起手道:「貧僧有願在前,原說果有敬重三寶,見善隨喜,皈依我佛,不要錢,願送與他。今見陛下明德止善,敬我佛門;況又高僧有德有行,宣揚大法,理當奉上,決不要錢。貧僧願留下此物告回。」唐王見他這等懃懇,甚喜。隨命光祿寺,大排素宴酬謝。菩薩又堅辭不受,暢然而去,依舊望都土地廟中隱避不題。

卻說太宗設午朝,著魏徵賫旨,宣玄奘入朝。那法師正聚眾登壇,諷經誦偈,一聞有旨,隨下壇整衣,與魏徵同往見駕。太宗道:「求證善事,有勞法師,無物酬謝。早間蕭瑀迎著二僧,願送錦襴異寶袈裟一件,九環錫杖一條。今特召法師領去受用。」玄奘叩頭謝恩。太宗道:「法師如不棄,可穿上與朕看看。」長老遂將袈裟抖開,披在身上,手持錫杖,侍立階前。君臣個個忻然。誠為如來佛子。你看他:
\begin{quote}
凜凜威顏多雅秀,佛衣可體如裁就。
暉光豔豔滿乾坤,結綵紛紛凝宇宙。
朗朗明珠上下排,層層金線穿前後。
兜羅四面錦沿邊,萬樣稀奇鋪綺繡。
八寶妝花縛鈕絲,金環束領攀絨扣。
佛天大小列高低,星象尊卑分左右。
玄奘法師大有緣,現前此物堪承受。
渾如極樂活阿羅,賽過西方真覺秀。
錫杖叮噹鬥九環,毘盧帽映多豐厚。
誠為佛子不虛傳,勝似菩提無詐謬。
\end{quote}

當時文武階前喝采。太宗喜之不勝,即著法師穿了袈裟,持了寶杖;又賜兩隊儀從,著多官送出朝門,教他上大街行道,往寺裡去,就如中狀元誇官的一般。這去玄奘再拜謝恩,在那大街上,烈烈轟轟,搖搖擺擺。你看那長安城裡,行商坐賈、公子王孫、墨客文人、大男小女,無不爭看誇獎,俱道:「好個法師,真是個活羅漢下降,活菩薩臨凡。」

玄奘直至寺裡,僧人下榻來迎。一見他披此袈裟,執此錫杖,都道是地藏王來了,各各歸依,侍於左右。玄奘上殿,炷香禮佛。又對眾感述聖恩已畢,各歸禪座。又不覺紅輪西墜。正是那:
\begin{quote}
日落煙迷草樹,帝都鐘鼓初鳴。
叮叮三響斷人行。前後街前寂靜。
上剎輝煌燈火,孤村冷落無聲。
禪僧入定理殘經。正好煉魔養性。
\end{quote}

光陰撚指,卻當七日正會。玄奘又具表,請唐王拈香。此時善聲遍滿天下。太宗即排駕,率文武多官、后妃國戚,早赴寺裡。那一城人,無論大小尊卑,俱詣寺聽講。

當有菩薩與木叉道:「今日是水陸正會,以一七繼七七,可矣了。我和你雜在眾人叢中,一則看他那會何如,二則看金蟬子可有福穿我的寶貝,三則也聽他講的是那一門經法。」兩人隨投寺裡。正是有緣得遇舊相識,般若還歸本道場。入到寺裡觀看,真個是:
\begin{quote}
天朝大國,果勝裟婆。賽過祇園舍衛,也不亞上剎招提。那一派仙音響喨,佛號喧嘩。
\end{quote}

這菩薩直至多寶臺邊,果然是明智金蟬之相。詩曰:
\begin{quote}
萬象澄明絕點埃,大典玄奘坐高臺。
超生孤魂暗中到,聽法高流市上來。
施物應機心路遠,出生隨意藏門開。
對看講出無量法,老幼人人放喜懷。
\end{quote}

又詩曰:
\begin{quote}
因遊法界講堂中,逢見相知不俗同。
盡說目前千萬事,又談塵劫許多功。
法雲容曳舒群岳,教網張羅滿太空。
檢點人生歸善念,紛紛天雨落花紅。
\end{quote}

那法師在臺上念一會《受生度亡經》,談一會《安邦天寶篆》,又宣一會《勸修功卷》。這菩薩近前來,拍著寶臺,厲聲高叫道:「那和尚,你只會談小乘教法,可會談大乘麼?」玄奘聞言,心中大喜,翻身跳下臺來,對菩薩起手道:「老師父,弟子失瞻多罪。見前的蓋眾僧人,都講的是小乘教法,卻不知大乘教法如何。」菩薩道:「你這小乘教法,度不得亡者超昇,只可渾俗和光而已。我有大乘佛法三藏,能超亡者昇天,能度難人脫苦,能修無量壽身,能作無來無去。」

正講處,有那司香巡堂官急奏唐王道:「法師正講談妙法,被兩個疥癩遊僧扯下來亂說胡話。」王令擒來。只見許多人將二僧推擁進後法堂,見了太宗,那僧人手也不起,拜也不拜,仰面道:「陛下問我何事?」唐王卻認得他,道:「你是前日送袈裟的和尚?」菩薩道:「正是。」太宗道:「你既來此處聽講,只該吃些齋便了,為何與我法師亂講,擾亂經堂,誤我佛事?」菩薩道:「你那法師講的是小乘教法,度不得亡者昇天。我有大乘佛法三藏,可以度亡脫苦,壽身無壞。」太宗正色喜問道:「你那大乘佛法在於何處?」菩薩道:「在大西天天竺國大雷音寺我佛如來處,能解百冤之結,能消無妄之災。」太宗道:「你可記得麼?」菩薩道:「我記得。」太宗大喜道:「教法師引去,請上臺開講。」

那菩薩帶了木叉,飛上高臺,遂踏祥雲,直至九霄,現出救苦原身,托了淨瓶楊柳。左邊是木叉惠岸,執著棍,抖擻精神。喜的個唐王朝天禮拜,眾文武跪地焚香。滿寺中僧尼道俗、士人工賈,無一人不拜禱道:「好菩薩!好菩薩!」有讚為證。但見那:
\begin{quote}
瑞靄散繽紛,祥光護法身。九霄華漢裡,現出女真人。那菩薩,頭上戴一頂金葉紐、翠花鋪、放金光、生瑞氣的垂珠纓絡;身上穿一領淡淡色、淺淺妝、盤金龍、飛綵鳳的結素藍袍;胸前掛一面對月明、舞清風、雜寶珠、攢翠玉的砌香環珮;腰間繫一條冰蠶絲、織金邊、登彩雲、促瑤海的錦繡絨裙;面前又領一個飛東洋、遊普世、感恩行孝、黃毛紅嘴白鸚哥。手內托著一個施恩濟世的寶瓶,瓶內插著一枝灑青霄、撒大惡、掃開殘霧垂楊柳。玉環穿繡扣,金蓮足下深。三天許出入。這才是救苦救難觀世音。
\end{quote}

喜的個唐太宗忘了江山,愛的那文武官失卻朝禮,蓋眾多人都念「南無觀世音菩薩」。太宗即傳旨,教巧手丹青描下菩薩真像。旨意一聲,選出個圖神寫聖、遠見高明的吳道子(此人即後圖功臣於凌煙閣者)。當時展開妙筆,圖寫真形。那菩薩祥雲漸遠,霎時間不見了金光。只見那半空中滴溜溜落下一張簡帖,上有幾句頌子,寫得明白。頌曰:
\begin{quote}
禮上大唐君,西方有妙文。程途十萬八千里,大乘進慇懃。此經回上國,能超鬼出群。若有肯去者,求正果金身。
\end{quote}

太宗見了頌子,即命眾僧:「且收勝會,待我差人取得大乘經來,再秉丹誠,重修善果。」眾官無不遵依。當時在寺中問曰:「誰肯領朕旨意,上西天拜佛求經?」問不了,傍邊閃過法師,帝前施禮道:「貧僧不才,願效犬馬之勞,與陛下求取真經,祈保我王江山永固。」唐王大喜,上前將御手扶起道:「法師果能盡此忠賢,不怕程途遙遠,跋涉山川,朕情願與你拜為兄弟。」玄奘頓首謝恩。唐王果是十分賢德,就去那寺裡佛前,與玄奘拜了四拜,口稱「御弟聖僧」。玄奘感謝不盡道:「陛下,貧僧有何德何能,敢蒙天恩眷顧如此?我這一去,定要捐軀努力,直至西天;如不到西天,不得真經,即死也不敢回國,永墮沉淪地獄。」隨在佛前拈香,以此為誓。唐王甚喜,即命回鑾,待選良利日辰,發牒出行,遂此駕回各散。

玄奘亦回洪福寺裡。那本寺多僧與幾個徒弟,早聞取經之事,都來相見,因問:「發誓願上西天,實否?」玄奘道:「是實。」他徒弟道:「師父啊,嘗聞人言,西天路遠,更多虎豹妖魔。只怕有去無回,難保身命。」玄奘道:「我已發了洪誓大願,不取真經,永墮沉淪地獄。大抵是受王恩寵,不得不盡忠以報國耳。我此去真是渺渺茫茫,吉凶難定。」又道:「徒弟們,我去之後,或三二年,或五七年,但看那山門裡松枝頭向東,我即回來;不然,斷不回矣。」眾徒將此言切切而記。

次早,太宗設朝,聚集文武,寫了取經文牒,用了通行寶印。有欽天監奏曰:「今日是人尊吉星,堪宜出行遠路。」唐王大喜。又見黃門官奏道:「御弟法師朝門外候旨。」隨即宣上寶殿道:「御弟,今日是出行吉日。這是通關文牒。朕又有一個紫金缽盂,送你途中化齋而用。再選兩個長行的從者。又欽賜你馬一匹,送為遠行腳力。你可就此行程。」玄奘大喜,即便謝了恩,領了物事,更無留滯之意。唐王排駕,與多官同送至關外。只見那洪福寺僧與諸徒將玄奘的冬夏衣服,俱送在關外相等。唐王見了,先教收拾行囊、馬匹,然後著官人執壺酌酒。太宗舉爵,又問曰:「御弟雅號甚稱?」玄奘道:「貧僧出家人,未敢稱號。」太宗道:「當時菩薩說,西天有經三藏。御弟可指經取號,號作三藏何如?」玄奘又謝恩,接了御酒道:「陛下,酒乃僧家頭一戒,貧僧自為人,不會飲酒。」太宗道:「今日之行,比他事不同,此乃素酒,只飲此一杯,以盡朕奉餞之意。」三藏不敢不受,接了酒,方待要飲,只見太宗低頭,將御指拾一撮塵土,彈入酒中。三藏不解其意,太宗笑道:「御弟啊,這一去,到西天,幾時可回?」三藏道:「只在三年,徑回上國。」太宗道:「日久年深,山遙路遠,御弟可進此酒:寧戀本鄉一捻土,莫愛他鄉萬兩金。」三藏方悟捻土之意,復謝恩飲盡,辭謝出關而去。唐王駕回。

畢竟不知此去何如,且聽下回分解。
