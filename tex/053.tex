
\chapter{禪主吞餐懷鬼孕 黃婆運水解邪胎}

\begin{quote}
德行要修八百,陰功須積三千。
均平物我與親冤。始合西天本願。
魔兕刀兵不怯,空勞水火無愆。
老君降伏卻朝天。笑把青牛牽轉。
\end{quote}

話說那大路傍叫喚者誰?乃金兜山山神、土地,捧著紫金缽盂叫道:「聖僧啊,這缽盂飯是孫大聖向好處化來的。因你等不聽良言,誤入妖魔之手,致令大聖勞苦萬端,今日方救得出。且來吃了飯,再去走路,莫孤負孫大聖一片恭孝之心也。」三藏道:「徒弟,萬分虧你,言謝不盡。早知不出圈痕,那有此殺身之害?」行者道:「不瞞師父說,只因你不信我的圈子,卻教你受別人的圈子。多少苦楚。可嘆,可嘆!」八戒道:「怎麼又有個圈子?」行者道:「都是你這孽嘴孽舌的夯貨,弄師父遭此一場大難,著老孫翻天覆地,請天兵、水火與佛祖丹砂,盡被他使一個白森森的圈子套去。如來暗示了羅漢,對老孫說出那妖的根原,才請老君來收伏,卻是個青牛作怪。」三藏聞言,感激不盡道:「賢徒,今番經此,下次定然聽你吩咐。」

遂此四人分吃那飯,那飯熱氣騰騰的。行者道:「這飯多時了,卻怎麼還熱?」土地跪下道:「是小神知大聖功完,才自熱來伺候。」須臾飯畢,收拾了缽盂,辭了土地、山神,那師父才攀鞍上馬,過了高山。正是:滌慮洗心皈正覺,餐風宿水向西行。

行夠多時,又值早春天氣。聽了些:
\begin{quote}
紫燕呢喃,黃鸝睍睆。紫燕呢喃香嘴困,黃鸝睍睆巧音頻。滿地落紅如佈錦,遍山發翠似堆茵。嶺上青梅結豆,崖前古柏留雲。野潤煙光淡,沙暄日色曛。幾處園林花放蕊,陽回大地柳芽新。
\end{quote}

正行處,忽遇一道小河,澄澄清水,湛湛寒波。唐長老勒過馬觀看,遠見河那邊有柳陰垂碧,微露著茅屋幾椽。行者遙指那廂道:「那裡人家,一定是擺渡的。」三藏道:「我見那廂也似這般,卻不見船隻,未敢開言。」八戒旋下行李,厲聲高叫道:「擺渡的,撐船過來。」連叫幾遍,只見那柳陰裡面,咿咿啞啞的撐出一隻船兒,不多時,相近這岸。師徒們仔細看了那船兒,真個是:
\begin{quote}
短棹分波,輕橈泛浪。橄堂油漆彩,艎板滿平倉。船頭上鐵纜盤窩,船後邊舵樓明亮。雖然是一葦之航,也不亞泛湖浮海。縱無錦纜牙檣,實有松樁桂楫。固不如萬里神舟,真可渡一河之隔。往來只在兩崖邊,出入不離古渡口。
\end{quote}

那船兒須臾頂岸,那梢子叫云:「過河的,這裡去。」三藏縱馬近前看處,那梢子怎生模樣:
\begin{quote}
頭裹錦絨帕,足踏皂絲鞋。身穿百納綿襠襖,腰束千針裙布絛。手腕皮粗筋力硬,眼花眉皺面容衰。聲音嬌細如鶯囀,近觀乃是老裙釵。
\end{quote}

行者近於船邊道:「你是擺渡的?」那婦人道:「是。」行者道:「梢公如何不在,卻著梢婆撐船?」婦人微笑不答,用手拖上跳板。沙和尚將行李挑上去,行者扶著師父上跳,然後順過船來,八戒牽上白馬,收了跳板。那婦人撐開船,搖動槳,頃刻間過了河。身登西岸,長老教沙僧解開包,取幾文錢鈔與他。婦人更不爭多寡,將纜拴在傍水的樁上,笑嘻嘻徑入莊屋裡去了。

三藏見那水清,一時口渴,便著八戒:「取缽盂,舀些水來我吃。」那獃子道:「我也正要些兒吃哩。」即取缽盂,舀了一缽,遞與師父。師父吃了有一少半,還剩了多半,獃子接來,一氣飲乾,卻伏侍三藏上馬。師徒們找路西行,不上半個時辰,那長老在馬上呻吟道:「腹痛。」八戒隨後道:「我也有些腹痛。」沙僧道:「想是吃冷水了。」說未畢,師父聲喚道:「疼的緊。」八戒也道:「疼得緊。」他兩個疼痛難禁,漸漸肚子大了。用手摸時,似有血團肉塊,不住的骨突骨突亂動。三藏正不穩便,忽然見那路傍有一村舍,樹梢頭挑著兩個草把。行者道:「師父,好了,那廂是個賣酒的人家。我們且去化他些熱湯與你吃,就問可有賣藥的,討貼藥,與你治治腹痛。」

三藏聞言甚喜,卻打白馬。不一時,到了村舍門口下馬。但只見那門兒外有一個老婆婆,端坐在草墩上績麻。行者上前,打個問訊道:「婆婆,貧僧是東土大唐來的。我師父乃唐朝御弟,因為過河吃了河水,覺肚腹疼痛。」那婆婆喜哈哈的道:「你們在那邊河裡吃水來?」行者道:「是在此東邊清水河吃的。」那婆婆欣欣的笑道:「好耍子,好耍子。你都進來,我與你說。」

行者即攙唐僧,沙僧即扶八戒,兩人聲聲喚喚,腆著肚子,一個個只疼得面黃眉皺,入草舍坐下。行者只叫:「婆婆,是必燒些熱湯與我師父,我們謝你。」那婆婆且不燒湯,笑唏唏跑走後邊,叫道:「你們來看,你們來看。」那裡面蹼蹼踏的又走出兩三個半老不老的婦人,都來望著唐僧哂笑。行者大怒,喝了一聲,把牙一齜。諕得那一家子跌跌蹡蹡,往後就走。行者上前,扯住那老婆子道:「快早燒湯,我饒了你。」那婆子戰兢兢的道:「爺爺呀!我燒湯也不濟事,也治不得他兩個肚疼。你放了我,等我說。」行者放了他,他說:「我這裡乃是西梁女國。我們這一國盡是女人,更無男子,故此見了你們歡喜。你師父吃的那水不好了。那條河喚做子母河。我那國王城外,還有一座迎陽館驛,驛門外有一個照胎泉。我這裡人,但得年登二十歲以上,方敢去吃那河裡水。吃水之後,便覺腹痛有胎。至三日之後,到那迎陽館照胎水邊照去。若照得有了雙影,便就降生孩兒。你師吃了子母河水,以此成了胎氣,也不日要生孩子,熱湯怎麼治得?」

三藏聞言,大驚失色道:「徒弟啊,似此怎了?」八戒扭腰撒胯的哼道:「爺爺呀!要生孩子,我們卻是男身,那裡開得產門?如何脫得出來?」行者笑道:「古人云:『瓜熟自落。』若到那個時節,一定從脅下裂個窟窿,鑽出來也。」八戒見說,戰兢兢,忍不得疼痛道:「罷了,罷了,死了,死了。」沙僧笑道:「二哥莫扭,莫扭,只怕錯了養兒腸,弄做個胎前病。」那獃子越發慌了,眼中噙淚,扯著行者道:「哥哥,你問這婆婆,看那裡有手輕的穩婆,預先尋下幾個。這半會一陣陣的動蕩得緊,想是摧陣疼,快了,快了。」沙僧又笑道:「二哥既知摧陣疼,不要扭動,只恐擠破漿包耳。」

三藏哼著道:「婆婆啊,你這裡可有醫家?教我徒弟去買一貼墮胎藥吃了,打下胎來罷。」那婆子道:「就有藥也不濟事。只是我們這正南街上有一座解陽山,山中有一個破兒洞,洞裡有一眼落胎泉。須得那泉裡水吃一口,方才解了胎氣。卻如今取不得水了。向年來了一個道人,稱名如意真仙,把那破兒洞改作聚仙庵,護住落胎泉水,不肯善賜與人。但欲求水者,須要花紅表禮,羊酒果盤,志誠奉獻,只拜求得他一碗兒水哩。你們這行腳僧,怎麼得許多錢財買辦?但只可挨命,待時而生產罷了。」行者聞得此言,滿心歡喜道:「婆婆,你這裡到那解陽山有幾多路程?」婆婆道:「有三千里。」行者道:「好了,好了。師父放心,待老孫取些水來你吃。」

好大聖,吩咐沙僧道:「你好仔細看著師父。若這家子無禮,侵哄師父,你拿出舊時手段來,裝諕他。等我取水去。」沙僧依命。只見那婆子端出一個大瓦缽來,遞與行者道:「拿這缽頭兒去,是必多取些來,與我們留著用急。」行者真個接了瓦缽,出草舍,縱雲而去。那婆子才望空禮拜道:「爺爺呀!這和尚會駕雲。」才進去叫出那幾個婦人來,對唐僧磕頭禮拜,都稱為羅漢菩薩。一壁廂燒湯辦飯,供奉唐僧不題。

卻說那孫大聖觔斗雲起,少頃間,見一座山頭阻住雲角。即按雲光,睜睛看處,好山!但見那:
\begin{quote}
幽花擺錦,野草鋪藍。澗水相連落,溪雲一樣閑。重重谷壑藤蘿密,遠遠峰巒樹木蘩。鳥啼雁過,鹿飲猿攀。翠岱如屏嶂,青崖似髻鬟。塵埃滾滾真難到,泉石涓涓不厭看。每見仙童採藥去,常逢樵子負薪還。果然不亞天臺景,勝似三峰西華山。
\end{quote}

這大聖正然觀看那山,又只見背陰處,有一所莊院,忽聞得犬吠之聲。大聖下山,徑至莊所,卻也好個去處。看那:
\begin{quote}
小橋通活水,茅舍倚青山。
村犬汪籬落,幽人自往還。
\end{quote}

不時來至門首,見一個老道人盤坐在綠茵之上。大聖放下瓦缽,近前道問訊。那道人欠身還禮道:「那方來者?至小庵有何勾當?」行者道:「貧僧乃東土大唐欽差西天取經者。因我師父誤飲了子母河之水,如今腹疼腫脹難禁。問及土人,說是結成胎氣,無方可治。訪得解陽山破兒洞有落胎泉可以消得胎氣。故此特來拜見如意真仙,求些泉水,搭救師父。累煩老道指引指引。」那道人笑道:「此間就是破兒洞,今改為聚仙庵了。我卻不是別人,即是如意真仙老爺的大徒弟。你叫做甚麼名字?待我好與你通報。」行者道:「我是唐三藏法師的大徒弟,賤名孫悟空。」那道人問曰:「你的花紅、酒禮都在那裡?」行者道:「我是個過路的掛搭僧,不曾辦得來。」道人笑道:「你好痴呀,我老師父護住山泉,並不曾白送與人。你回去辦將禮來,我好通報。不然請回。莫想,莫想。」行者道:「人情大似聖旨。你去說我老孫的名字,他必然做個人情,或者連井都送我也。」

那道人聞此言,只得進去通報。卻見那真仙撫琴,只待他琴終,方才說道:「師父,外面有個和尚,口稱是唐三藏大徒弟孫悟空,欲求落胎泉水,救他師父。」那真仙不聽說便罷,一聽得說個悟空名字,卻就怒從心上起,惡向膽邊生。急起身,下了琴床,脫了素服,換上道衣,取一把如意鉤子,跳出庵門,叫道:「孫悟空何在?」行者轉頭,觀見那真仙打扮:
\begin{quote}
頭戴星冠飛彩艷,身穿金縷法衣紅。
足下雲鞋堆錦繡,腰間寶帶繞玲瓏。
一雙納錦凌波襪,半露裙襴閃繡絨。
手拿如意金鉤子,鐏利杆長若蟒龍。
鳳眼光明眉菂豎,鋼牙尖利口翻紅。
額下髯飄如烈火,鬢邊赤髮短蓬鬆。
形容惡似溫元帥,爭奈衣冠不一同。
\end{quote}

行者見了,合掌作禮道:「貧僧便是孫悟空。」那先生笑道:「你真個是孫悟空,卻是假名託姓者?」行者道:「你看先生說話。常言道:『君子行不更名,坐不改姓。』我便是悟空,豈有假託之理?」先生道:「你可認得我麼?」行者道:「我因歸正釋門,秉誠僧教,這一向登山涉水,把我那幼時的朋友也都疏失,未及拜訪,少識尊顏。適間問道子母河西鄉人家,言及先生乃如意真仙,故此知之。」那先生道:「你走你的路,我修我的真,你來訪我怎的?」行者道:「因我師父誤飲了子母河水,腹疼成胎,特來仙府,拜求一碗落胎泉水,救解師難也。」

那先生怒目道:「你師父可是唐三藏麼?」行者道:「正是,正是。」先生咬牙恨道:「你們可曾會著一個聖嬰大王麼?」行者道:「他是號山枯松澗火雲洞紅孩兒妖怪的綽號,真仙問他怎的?」先生道:「是我之舍侄,我乃牛魔王的兄弟。前者家兄處有信來報我,稱說唐三藏的大徒弟孫悟空憊𪬯,將他害了。我這裡正沒處尋你報仇,你倒來尋我,還要甚麼水哩。」行者陪笑道:「先生差了。你令兄也曾與我做朋友,幼年間也曾拜七弟兄。但只是不知先生尊府,有失拜望。如今令侄得了好處,現隨著觀音菩薩,做了善財童子,我等尚且不如,怎麼反怪我也?」

先生喝道:「這潑猢猻!還弄巧舌。我舍侄還是自在為王好,還是與人為奴好?不得無禮,吃我這一鉤!」大聖使鐵棒架住道:「先生莫說打的話,且與些泉水去也。」那先生罵道:「潑猢猻!不知死活。如若三合敵得我,與你水去;敵不過,只把你剁為肉醬,方與我侄子報仇。」大聖罵道:「我把你不識起倒的孽障!既要打,起開來看棍。」那先生如意鉤劈手相還。二人在聚仙庵好殺:
\begin{quote}
聖僧誤食成胎水,行者來尋如意仙。那曉真仙原是怪,倚強護住落胎泉。及至相逢講仇隙,爭持決不遂如然。言來語去成僝僽,意惡情兇要報冤。這一個因師傷命來求水,那一個為侄亡身不與泉。如意鉤強如蝎毒,金箍棒狠似龍巔。當胸亂刺施威猛,著腳斜鉤展妙玄。陰手棍丟傷處重,過肩鉤起近頭鞭。鎖腰一棍鷹持雀,壓頂三鉤蜋捕蟬。往往來來爭勝敗,返返復復兩回還。鉤攣棒打無前後,不見輸贏在那邊。
\end{quote}

那先生與大聖戰經十數合,敵不得大聖。這大聖越加猛烈,一條棒似滾滾流星,著頭亂打。先生敗了筋力,倒拖著如意鉤,往山上走了。

大聖不去趕他,卻來庵內尋水。那個道人早把庵門關了。大聖拿著瓦缽,趕至門前,盡力氣一腳,踢破庵門,闖將進去。見那道人伏在井欄上,被大聖喝了一聲,舉棒要打,那道人往後跑了。卻才尋出吊桶來,正要打水,又被那先生趕到前邊,使如意鉤子把大聖鉤著腳一跌,跌了個嘴硍地。大聖爬起來,使鐵棒就打。他卻閃在傍邊,執著鉤子道:「看你可取得我的水去?」大聖罵道:「你上來,你上來,我把你這個孽障直打殺你!」那先生也不上前拒敵,只是禁住了,不許大聖打水。大聖見他不動,卻使左手掄著鐵棒,右手使吊桶。將索子才突轆轆的放下,他又來使鉤。大聖一隻手撐持不得,又被他一鉤鉤著腳,扯了個躘踵,連索子通跌下井去了。大聖道:「這廝卻是無禮。」爬起來,雙手掄棒,沒頭沒臉的打將上去。那先生依然走了,不敢迎敵。大聖又要去取水,奈何沒有吊桶,又恐怕來鉤扯,心中暗暗想道:「且去叫個幫手來。」

好大聖,撥轉雲頭,徑至村舍門首,叫一聲:「沙和尚。」那裡邊三藏忍痛呻吟,豬八戒哼聲不絕。聽得叫喚,二人歡喜道:「沙僧啊,悟空來也。」沙僧連忙出門接著道:「大哥,取水來了?」大聖進門,對唐僧備言前事。三藏滴淚道:「徒弟啊,似此怎了?」大聖道:「我來叫沙兄弟與我同去,到那庵邊,等老孫和那廝敵鬥,教沙僧乘便取水來救你。」三藏道:「兩個沒病的都去了,丟下我兩個有病的,教誰伏侍?」那個老婆婆在傍道:「老羅漢只管放心,不須要你徒弟,我家自然看顧伏侍你。你們早間到時,我等實有愛憐之意。卻才見這位菩薩雲來霧去,方知你是羅漢菩薩,我家決不敢復害你。」

行者咄的一聲道:「汝等女流之輩,敢傷那個?」老婆子笑道:「爺爺呀!還是你們有造化,來到我家!若到第二家,你們也不得囫圇了。」八戒哼哼的道:「不得囫圇,是怎麼的?」婆婆道:「我一家兒四五口,都是有幾歲年紀的,把那風月事盡皆休了,故此不肯傷你;若還到第二家,老小眾大,那年小之人,那個肯放過你去?就要與你交合。假如不從,就要害你性命,把你們身上肉都割了去做香袋兒哩。」八戒道:「若這等,我決無傷。他們都是香噴噴的,好做香袋;我是個臊豬,就割了肉去,也是臊的,故此可以無傷。」行者笑道:「你不要說嘴,省些力氣,好生產也。」那婆婆道:「不必遲疑,快求水去。」行者道:「你家可有吊桶?借個使使。」那婆子即往後邊取出一個吊桶,又窩了一條索子,遞與沙僧。沙僧道:「帶兩條索子去,恐一時井深要用。」

沙僧接了桶索,即隨大聖出了村舍,一同駕雲而去,那消半個時辰,卻到解陽山界。按下雲頭,徑至庵外。大聖吩咐沙僧道:「你將桶索拿了,且在一邊躲著,等老孫出頭索戰。你待我兩人交戰正濃之時,你乘機進去,取水就走。」沙僧謹依言命。

孫大聖掣了鐵棒,近門高叫:「開門,開門!」那守門的看見,急入裡通報道:「師父,那孫悟空又來了也。」那先生心中大怒道:「這潑猴老大無狀。一向聞他有些手段,果然今日方知,他那條棒真是難敵。」道人道:「師父,他的手段雖高,你亦不亞與他,正是個對手。」先生道:「前面兩回,被他贏了。」道人道:「前兩回雖贏,不過是一猛之性;後面兩次打水之時,被師父鉤他兩跌,卻不是相比肩也?先既無奈而去,今又復來,必然是三藏胎成身重,埋怨得緊,不得已而來也。決有慢他師之心,管取我師決勝無疑。」

真仙聞言,喜孜孜滿懷春意,笑盈盈一陣威風,挺如意鉤子,走出門來喝道:「潑猢猻!你又來作甚?」大聖道:「我來只是取水。」真仙道:「泉水乃吾家之井,憑是帝王宰相,也須表禮羊酒來求,方才僅與些須;況你又是我的仇人,擅敢白手來取?」大聖道:「真個不與?」真仙道:「不與,不與。」大聖罵道:「潑孽障!既不與水,看棍!」丟一個架子,搶個滿懷,不容說,著頭便打;那真仙側身躲過,使鉤子急架相還。這一場比前更勝,好殺:
\begin{quote}
金箍棒,如意鉤,二人奮怒各懷仇。飛砂走石乾坤暗,播土揚塵日月愁。大聖救師來取水,妖仙為侄不容求。兩家齊努力,一處賭安休。咬牙爭勝負,切齒定剛柔。添機見,越抖擻,噴雲噯霧鬼神愁。樸樸兵兵鉤棒響,喊聲哮吼振山丘。狂風滾滾催林木,殺氣紛紛過斗牛。大聖愈爭愈喜悅,真仙越打越綢繆。有心有意相爭戰,不定存亡不罷休。
\end{quote}

他兩個在庵門外交手,跳跳舞舞的,鬥到山坡之下,恨苦相持不題。

卻說那沙和尚提著吊桶,闖進門去,只見那道人在井邊擋住道:「你是甚人,敢來取水?」沙僧放下吊桶,取出降妖寶杖,不對話,著頭便打。那道人躲閃不及,把左臂膊打折,道人倒在地下掙命。沙僧罵道:「我要打殺你這孽畜,怎奈你是個人身,我還憐你,饒你去罷。讓我打水。」那道人叫天叫地的,爬到後面去了。沙僧卻才將吊桶向井中滿滿的打了一吊桶水,走出庵門,駕起雲霧,望著行者喊道:「大哥,我已取了水去也。饒他罷,饒他罷。」

大聖聽得,方才使鐵棒支住鉤子道:「我本待斬盡殺絕,爭奈你不曾犯法;二來看你令兄牛魔王的情上。先頭來,我被鉤了兩下,未得水去。才然來,我是個調虎離山計,哄你出來爭戰,卻著我師弟取水去了。老孫若肯拿出本事來打你,莫說你是一個甚麼如意真仙,就是再有幾個,也打死了。正是打死不如放生,且饒你教你活幾年耳。已後再有取水者,切不可勒掯他。」那妖仙不識好歹,演一演,就來鉤腳。被大聖閃過鉤頭,趕上前,喝聲:「休走!」那妖仙措手不及,推了一個蹼辣,掙扎不起。大聖奪過如意鉤來,折為兩段;總拿著又一抉,抉作四段。擲之於地道:「潑孽畜!再敢無禮麼?」那妖仙戰戰兢兢,忍辱無言。這大聖笑呵呵,駕雲而起。有詩為證。詩曰:
\begin{quote}
真鉛若鍊須真水,真水調和真汞乾。
真汞真鉛無母氣,靈砂靈藥是仙丹。
嬰兒枉結成胎像,土母施功不費難。
推倒旁門宗正教,心君得意笑容還。
\end{quote}

大聖縱著祥光,趕上沙僧。得了真水,喜喜歡歡,回於本處。按下雲頭,徑來村舍。只見豬八戒腆著肚子,倚在門枋上哼哩。行者悄悄上前道:「獃子,幾時占房的?」獃子慌了道:「哥哥莫取笑。可曾有水來麼?」行者還要耍他,沙僧隨後就到,笑道:「水來了,水來了。」三藏忍痛欠身道:「徒弟啊,累了你們也。」那婆婆卻也歡喜,幾口兒都出禮拜道:「菩薩呀,卻是難得,難得。」即忙取個花磁盞子,舀了半盞兒,遞與三藏道:「老師父,細細的吃,只消一口,就解了胎氣。」八戒道:「我不用盞子,連吊桶等我喝了罷。」那婆子道:「老爺爺,諕殺人罷了。若吃了這吊桶水,好道連腸子肚子都化盡了。」嚇得獃子不敢胡為,也只吃了半盞。

那裡有頓飯之時,他兩個腹中絞痛,只聽轂轆轂轆三五陣腸鳴。腸鳴之後,那獃子忍不住,大小便齊流。唐僧也忍不住要往靜處解手。行者道:「師父啊,切莫出風地裡去,怕人子,一時冒了風,弄做個產後之疾。」那婆婆即取兩個淨桶來,教他兩個方便。須臾間,各行了幾遍,才覺住了疼痛,漸漸的銷了腫脹,化了那血團肉塊。那婆婆家又煎些白米粥與他補虛。八戒道:「婆婆,我的身子實落,不用補虛。你燒些湯水與我洗個澡,卻好吃粥。」沙僧道:「二哥,洗不得澡。坐月子的人弄了水漿致病。」八戒道:「我又不曾大生,左右只是個小產,怕他怎的?洗洗兒乾淨。」真個那婆子燒些湯與他兩個淨了手腳。唐僧才吃兩盞兒粥湯。八戒就吃了十數碗,還只要添。行者笑道:「夯貨,少吃些,莫弄做個沙包肚,不像模樣。」八戒道:「沒事,沒事,我又不是母豬,怕他做甚?」那家子真個又去收拾煮飯。

老婆婆對唐僧道:「老師父,把這水賜了我罷。」行者道:「獃子,不吃水了?」八戒道:「我的肚腹也不疼了,胎氣想是已行散了,灑然無事,又吃水何為?」行者道:「既是他兩個都好了,將水送你家罷。」那婆婆謝了行者,將餘剩之水裝於瓦罐之中,埋在後邊地下。對眾老小道:「這罐水,夠我的棺材本也。」眾老小無不歡喜,整頓齋飯,調開桌凳。唐僧們吃了齋,消消停停,將息了一宿。

次日天明,師徒們謝了婆婆家,出離村舍。唐三藏攀鞍上馬,沙和尚挑著行囊,孫大聖前邊引路,豬八戒攏了韁繩。這裡才是:
\begin{quote}
洗淨口孽身乾淨,銷化凡胎體自然。
\end{quote}

畢竟不知到國界中還有甚麼理會,且聽下回分解。
