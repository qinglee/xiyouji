
\chapter{豬八戒義激猴王 孫行者智降妖怪}

\begin{quote}
義結孔懷,法歸本性。金順木馴成正果,心猿木母合丹元。共登極樂世界,同來不二法門。經乃修行之總徑,佛配自己之元神。兄和弟會成三契,妖與魔色應五行。剪除六門趣,即赴大雷音。
\end{quote}

卻說那獃子被一窩猴子捉住了,扛擡扯拉,把一件直裰子揪破。口裡勞勞叨叨的,自家念誦道:「罷了,罷了,這一去有個打殺的情了。」不一時,到洞口。那大聖坐在石崖之上,罵道:「你這囊糠的夯貨!你去便罷了,怎麼罵我?」八戒跪在地下道:「哥啊,我不曾罵你;若罵你,就嚼了舌頭根。我只說哥哥不去,我自去報師父便了。怎敢罵你?」行者道:「你怎麼瞞得過我?我這左耳往上一扯,曉得三十三天人說話;我這右耳往下一扯,曉得十代閻王與判官算帳。你今走路把我罵,我豈不聽見?」八戒道:「哥啊,我曉得你賊頭鼠腦的,一定又變作個甚麼東西兒,跟著我聽的。」行者叫:「小的們,選大棍來,先打二十個見面孤拐,再打二十個背花,然後等我使鐵棒與他送行。」八戒慌得磕頭道:「哥哥,千萬看師父面上,饒了我罷。」行者道:「我想那師父好仁義兒哩。」八戒又道:「哥哥,不看師父啊,請看海上菩薩之面,饒了我罷。」

行者見說起菩薩,卻有三分兒轉意道:「兄弟,既這等說,我且不打你。你卻老實說,不要瞞我。那唐僧在那裡有難,你卻來此哄我?」八戒道:「哥哥,沒甚難處,實是想你。」行者罵道:「這個好打的劣貨,你怎麼還要者囂我?老孫身回水簾洞,心逐取經僧。那師父步步有難,處處該災,你趁早兒告訢我,免打。」八戒聞得此言,叩頭上告道:「哥啊,分明要瞞著你,請你去的,不期你這等樣靈。饒我打,放我起來說罷。」行者道:「也罷,起來說。」眾猴撒開手。那獃子跳得起來,兩邊亂張。行者道:「你張甚麼?」八戒道:「看看那條路兒空闊,好跑。」行者道:「你跑到那裡?我就讓你先走三日,老孫自有本事趕轉你來。快早說來,這一惱發我的性子,斷不饒你。」

八戒道:「實不瞞哥哥說,自你回後,我與沙僧保師父前行,只見一座黑松林,師父下馬,教我化齋。我因許遠,無一個人家,辛苦了,略在草裡睡睡。不想沙僧別了師父,又來尋我。你曉得師父沒有坐性,他獨步林間玩景。出得林,見一座黃金寶塔放光,他只當寺院。不期塔下有個妖精,名喚黃袍,被他拿住。後邊我與沙僧回尋,止見白馬、行囊,不見師父。隨尋至洞口,與那怪廝殺。師父在洞,幸虧了一個救星。原是寶象國王第三個公主,被那怪攝來者。他修了一封家書,託師父寄去,遂說方便,解放了師父。到了國中,遞了書子,那國王就請師父降妖,取回公主。哥啊,你曉得,那老和尚可會降妖?我二人復去與戰,不知那怪神通廣大,將沙僧又捉了。我敗陣而走,伏在草中。那怪變做個俊俏文人入朝,與國王認親,把師父變作老虎。又虧了白龍馬夜現龍身,去尋師父,師父倒不曾尋見,卻遇著那怪在銀安殿飲酒。他變一宮娥,與他巡酒、舞刀,欲乘機而砍,反被他用滿堂紅打傷馬腿。就是他教我來請師兄的,說道:『師兄是個有仁有義的君子,君子不念舊惡,一定肯來救師父一難。』萬望哥哥念『一日為師,終身為父』之情,千萬救他一救。」

行者道:「你這個獃子,我臨別之時,曾叮嚀又叮嚀,說道:『若有妖魔捉住師父,你就說老孫是他大徒弟。』怎麼卻不說我?」八戒又思量道:「請將不如激將,等我激他一激。」道:「哥啊,不說你還好哩,只為說你,他一發無狀。」行者道:「怎麼說?」八戒道:「我說:『妖精,你不要無禮,莫害我師父。我還有個大師兄,叫做孫行者,他神通廣大,善能降妖,他來時教你死無葬身之地。』那怪聞言,越加忿怒,罵道:『是個甚麼孫行者,我可怕他?他若來,我剝了他皮,抽了他觔,啃了他骨,吃了他心。饒他猴子瘦,我也把他剁鮓著油烹。』」行者聞言,就氣得抓耳撓腮,暴躁亂跳道:「是那個敢這等罵我?」八戒道:「哥哥息怒,是那黃袍怪這等罵來,我故學與你聽也。」行者道:「賢弟,你起來。不是我去不成,既是妖精敢罵我,我就不能不降他,我和你去。老孫五百年前大鬧天宮,普天的神將看見我,一個個控背躬身,口口稱呼大聖。這妖怪無禮,他敢背前面後罵我。我這去,把他拿住,碎屍萬段,以報罵我之仇!報畢,我即回來。」八戒道:「哥哥,正是。你只去拿了妖精,報了你仇,那時來與不來,任從尊意。」

那大聖才跳下崖,撞入洞裡,脫了妖衣。整一整錦直裰,束一束虎皮裙,執了鐵棒,徑出門來。慌得那群猴攔住道:「大聖爺爺,你往那裡去?帶挈我們耍子幾年也好。」行者道:「小的們,你說那裡話。我保唐僧的這樁事,天上地下,都曉得孫悟空是唐僧的徒弟。他倒不是趕我回來,倒是教我來家看看,送我來家自在耍子。如今只因這件事。你們卻都要仔細看守家業,依時插柳栽松,毋得廢墜。待我還去保唐僧,取經回東土,功成之後,仍回來與你們共樂天真。」眾猴各各領命。

那大聖才和八戒攜手駕雲,離了洞,過了東洋大海,至西岸,住雲光,叫道:「兄弟,你且在此慢行,等我下海去淨淨身子。」八戒道:「忙忙的走路,且淨甚麼身子?」行者道:「你那裡知道。我自從回來,這幾日弄得身上有些妖精氣了。師父是個愛乾淨的,恐怕嫌我。」八戒於此始識得行者是片真心,更無他意。

須臾洗畢,復駕雲西進,只見那金塔放光。八戒指道:「那不是黃袍怪家?沙僧還在他家裡。」行者道:「你在空中,等我下去看看那門前如何,好與妖精見陣。」八戒道:「不要去,妖精不在家。」行者道:「我曉得。」好猴王,按落祥光,徑至洞門外觀看。只見有兩個小孩子,在那裡使彎頭棍,打毛毬,搶窩耍子哩。一個有十來歲,一個有八九歲了。正戲處,被行者趕上前,也不管他是張家李家的,一把抓著頂搭子,提將過來。那孩子吃了諕,口裡夾罵帶哭的亂嚷。驚動那波月洞的小妖,急報與公主道:「奶奶,不知甚人把二位公子搶去也。」原來那兩個孩子是公主與那怪生的。

公主聞言,忙忙走出洞門來,只見行者提著兩個孩子,站在那高崖之上,意欲往下摜。慌得那公主厲聲高叫道:「那漢子,我與你沒甚相干,怎麼把我兒子拿去?他老子利害,有些差錯,決不與你干休。」行者道:「你不認得我?我是那唐僧的大徒弟孫悟空行者。我有個師弟沙和尚在你洞裡,你去放他出來,我把這兩個孩兒還你。似這般兩個換一個,還是你便宜。」那公主聞言,急往裡面,喝退那幾個把門的小妖,親動手,把沙僧解了。沙僧道:「公主,你莫解我,恐你那怪來家,問你要人,帶累你受氣。」公主道:「長老啊,你是我的恩人,你替我折辯了家書,救了我一命,我也留心放你。不期洞門之外,你有個大師兄孫悟空來了,叫我放你哩。」

噫!那沙僧一聞「孫悟空」的三個字,好便似醍醐灌頂,甘露滋心;一面天生喜,滿腔都是春;也不似聞得個人來,就如拾著一方金玉一般。你看他捽手拂衣,走出門來,對行者施禮道:「哥哥,你真是從天而降也。萬乞救我一救。」行者笑道:「你這個沙尼,師父念緊箍兒咒,可肯替我方便一聲?都弄嘴施展,要保師父,如何不走西方路,卻在這裡蹲甚麼?」沙僧道:「哥哥,不必說了,君子人既往不咎。我等是個敗軍之將,不可語勇,救我救兒罷。」行者道:「你上來。」沙僧才縱身跳上石崖。

卻說那八戒停立空中,看見沙僧出洞,即按下雲頭,叫聲:「沙兄弟,心忍,心忍。」沙僧見身道:「二哥,你從那裡來?」八戒道:「我昨日敗陣,夜間進城,會了白馬,知師父有難,被黃袍使法,變做個老虎。那白馬與我商議,請師兄來的。」行者道:「獃子,且休敘闊。把這兩個孩子,各抱著一個,先進那寶象城去激那怪來,等我在這裡打他。」沙僧道:「哥啊,怎麼樣激他?」行者道:「你兩個駕起雲,站在那金鑾殿上,莫分好歹,把那孩子往那白玉階前一摜。有人問你是甚人,你便說是黃袍妖精的兒子,被我兩個拿將來也。那怪聽見,管情回來,我卻不須進城與他鬥了。若在城上廝殺,必要噴雲噯霧,播土揚塵,驚擾那朝廷與多官、黎庶,俱不安也。」八戒笑道:「哥哥,你但幹事,就左我們。」行者道:「如何為左你?」八戒道:「這兩個孩子被你抓來,已此諕破膽了,這一會聲都哭啞,再一會必死無疑。我們拿他往下一摜,摜做個肉子,那怪趕上肯放?定要我兩個償命。你卻還不是個乾淨人?連見證也沒你,你卻不是左我們?」行者道:「他若扯你,你兩個就與他打將這裡來。這裡有戰場寬闊,我在此等候打他。」沙僧道:「正是,正是,大哥說得有理,我們去來。」他兩個才倚仗威風,將孩子拿去。

行者即跳下石崖,到他塔門之下。那公主道:「你這和尚,全無信義:你說放了你師弟,就與我孩兒,怎麼你師弟放去,把我孩兒又留,反來我門首做甚?」行者陪笑道:「公主休怪。你來的日子已久,帶你令郎去認他外公去哩。」公主道:「和尚莫無禮。我那黃袍郎比眾不同,你若諕了我的孩兒,與他柳柳驚是。」行者笑道:「公主啊,為人生在天地之間,怎麼便是得罪?」公主道:「我曉得。」行者道:「你女流家,曉得甚麼?」公主道:「我自幼在宮,曾受父母教訓。記得古書云:『五刑之屬三千,而罪莫大於不孝。』」行者道:「你正是個不孝之人。蓋『父兮生我,母兮鞠我。哀哀父母,生我劬勞。』故孝者,百行之原,萬善之本。卻怎麼將身陪伴妖精,更不思念父母?非得不孝之罪,如何?」公主聞此正言,半晌家耳紅面赤,慚愧無地。忽失口道:「長老之言最善。我豈不思念父母?只因這妖精將我攝騙在此,他的法令又謹,我的步履又難,路遠山遙,無人可傳音信。欲要自盡,又恐父母疑我逃走,事終不明。故沒奈何,苟延殘喘,誠為天地間一大罪人也。」說罷,淚如泉湧。

行者道:「公主不必傷悲。豬八戒曾告訢我,說你有一封書,曾救了我師父一命,你書上也有思念父母之意。老孫來,管與你拿了妖精,帶你回朝見駕,別尋個佳偶,侍奉雙親到老。你意如何?」公主道:「和尚啊,你莫要尋死。昨者你兩個師弟那樣好漢,也不曾打得過我黃袍郎。你這般一個觔多骨少的瘦鬼,一似個螃蟹模樣,骨頭都長在外面,有甚本事,你敢說拿妖魔之話?」行者笑道:「你原來沒眼色,認不得人。俗語云:『尿泡雖大無斤兩,秤鉈雖小壓千斤。』他們相貌,空大無用:走路抗風,穿衣費布,種火心空,頂門腰軟,吃食無功。咱老孫小自小,斤節。」那公主道:「你真個有手段麼?」行者道:「我的手段,你是也不曾看見,絕會降妖,極能伏怪。」公主道:「你卻莫誤了我耶。」行者道:「決然誤你不得。」公主道:「你既會降妖伏怪,如今卻怎樣拿他?」行者說:「你且迴避迴避,莫在我這眼前:倘他來時,不好動手腳,只恐你與他情濃了,捨不得他。」公主道:「我怎的捨不得他?其稽留於此者,不得已耳。」行者道:「你與他做了十三年夫妻,豈無情意?我若見了他,不與他兒戲,一棍便是一棍,一拳便是一拳,須要打倒他,才得你回朝見駕。」

那公主果然依行者之言,往僻靜處躲避。也是他姻緣該盡,故遇著大聖來臨。

那猴王把公主藏了,他卻搖身一變,就變做公主一般模樣,回轉洞中,專候那怪。

卻說八戒、沙僧把兩個孩子拿到寶象國中,往那白玉階前捽下。可憐都摜做個肉餅相似,鮮血迸流,骨骸粉碎。慌得那滿朝多官報道:「不好了,不好了,天上摜下兩個人來了。」八戒厲聲高叫道:「那孩子是黃袍妖精的兒子,被老豬與沙弟拿將來也!」

那怪還在銀安殿,宿酒未醒。正睡夢間,聽得有人叫他名字,他就翻身,擡頭觀看,只見那雲端裡是豬八戒、沙和尚二人吆喝。妖怪心中暗想道:「豬八戒便也罷了;沙和尚是我綁在家裡,他怎麼得出來?我的渾家怎麼肯放他?我的孩兒怎麼得到他手?這怕是豬八戒不得我出去與他交戰,故將此計來羈我。我若認了這個泛頭,就與他打啊。噫!我卻還害酒哩。假若被他築上一鈀,卻不滅了這個威風,識破了那個關竅?且等我回家看看,是我的兒子不是我的兒子,再與他說話不遲。」好妖怪,他也不辭王駕,轉山林,徑去洞中查信息。

此時朝中已知他是個妖怪了。原來他夜裡吃了一個宮娥,還有十七個脫命去的,五更時奏了國王,說他如此如此。又因他不辭而去,越發知他是怪。那國王即著多官看守著假老虎不題。

卻說那怪徑回洞口。行者見他來時,設法哄他,把眼擠了一擠,撲簌簌淚如雨落,兒天兒地的跌腳搥胸,於此洞裡嚎啕痛哭。那怪一時間那裡認得,上前摟住道:「渾家,你有何事,這般煩惱?」那大聖編成的鬼話,捏出的虛詞,淚汪汪的告道:「郎君啊,常言道:『男子少妻財沒主,婦女無夫身落空。』你昨日進朝認親,怎不回來?今早被豬八戒劫了沙和尚,又把我兩個孩兒搶去,是我苦告,更不肯饒。他說拿去朝中認認外公,這半日不見孩兒,又不知存亡如何,你又不見來家,教我怎生割捨?故此止不住傷心痛哭。」那怪聞言,心中大怒道:「真個是我的兒子?」行者道:「正是,被豬八戒搶去了。」

那妖魔氣得亂跳道:「罷了,罷了,我兒被他摜殺了,已是不可活也。只好拿那和尚來與我兒子償命報仇罷,渾家,你且莫哭。你如今心裡覺道怎麼?且醫治一醫治。」行者道:「我不怎的,只是捨不得孩兒,哭得我有些心疼。」妖魔道:「不打緊,你請起來,我這裡有件寶貝,只在你那疼上摸一摸兒,就不疼了。卻要仔細,休使大指兒彈著;若使大指兒彈著啊,就看出我本相來了。」行者聞言,心中暗笑道:「這潑怪,倒也老實,不動刑法,就自家供了。等他拿出寶貝來,我試彈他一彈,看他是個甚麼妖怪。」那怪攜著行者,一直行到洞裡深遠密閉之處。卻從口中吐出一件寶貝,有雞子大小,是一顆舍利子玲瓏內丹。行者心中暗喜道:「好東西耶。這件物不知打了多少坐工,煉了幾年磨難,配了幾轉雌雄,煉成這顆內丹舍利。今日大有緣法,遇著老孫。」那猴子拿將過來,那裡有甚麼疼處,特故意摸了一摸,一指頭彈將去。那妖慌了,劈手來搶。你思量,那猴子好不溜撒,把那寶貝一口吸在肚裡。那妖魔揝著拳頭就打。被行者一手隔住,把臉抹了一抹,現出本相,道聲:「妖怪,不要無禮。你且認認看我是誰?」

那妖怪見了,大驚道:「呀!渾家,你怎麼拿出這一副嘴臉來耶?」行者罵道:「我把你這個潑怪!誰是你渾家?連你祖宗也還不認得哩。」那怪忽然省悟道:「我像有些認得你哩。」行者道:「我且不打你,你再認認看。」那怪道:「我雖見你眼熟,一時間卻想不起姓名,你果是誰?從那裡來的?你把我渾家估倒在何處,卻來我家詐誘我的寶貝?著實無禮,可惡!」行者道:「你是也不認得我。我是唐僧的大徒弟,叫做孫悟空行者。我是你五百年前的舊祖宗哩。」那怪道:「沒有這話,沒有這話。我拿住唐僧時,止知他有兩個徒弟,叫做豬八戒、沙和尚,何曾見有人說個姓孫的?你不知是那裡來的個怪物,到此騙我。」行者道:「我不曾同他二人來,是我師父因老孫慣打妖怪,殺傷甚多,他是個慈悲好善之人,將我逐回,故不曾同他一路行走。你是不知你祖宗名姓。」那怪道:「你好不丈夫啊,既受了師父趕逐,卻有甚麼嘴臉又來見人?」行者道:「你這個潑怪,豈知『一日為師,終身為父』?『父子無隔宿之仇』?你傷害我師父,我怎麼不來救他?你害他便也罷?卻又背前面後罵我,是怎的說?」妖怪道:「我何嘗罵你?」行者道:「是豬八戒說的。」那怪道:「你不要信他。那個豬八戒尖著嘴,有些會說老婆舌頭,你怎聽他?」行者道:「且不必講此閑話。只說老孫今日到你家裡,你好怠慢了遠客。雖無酒饌款待,頭卻是有的。快快將頭伸過來,等老孫打一棍兒當茶。」那怪聞得說打,呵呵大笑道:「孫行者,你差了計較了,你既說要打,不該跟我進來。我這裡大小群妖還有百十,饒你滿身是手,也打不出我的門去。」行者道:「不要胡說,莫說百十個,就有幾千幾萬,只要一個個查明白了好打,棍棍無空,教你斷根絕跡。」

那怪聞言,急傳號令,把那山前山後群妖,洞裡洞外諸怪,一齊點起,各執器械,把那三四層門,密密攔阻不放。行者見了,滿心歡喜,雙手理棍,喝聲叫:「變!」變的三頭六臂。把金箍棒幌一幌,變做三根金箍棒。你看他六隻手使著三根棒,一路打將去,好便似虎入羊群,鷹來雞柵。可憐那小怪,湯著的,頭如粉碎;刮著的,血似水流。往來縱橫,如入無人之境。止剩一個老妖,趕出門來罵道:「你這潑猴,其實憊𪬯!怎麼上門子欺負人家?」行者急回頭,用手招呼道:「你來,你來,打倒你,才是功績。」那怪物,舉寶刀,分頭便砍;好行者,掣鐵棒,覿面相迎。這一場,在那山頂上,半雲半霧的殺哩:
\begin{quote}
大聖神通大,妖魔本事高。這個橫理生鐵棒,那個斜舉蘸鋼刀。悠悠刀起明霞亮,輕輕棒架彩雲飄。往來護頂翻多次,反覆渾身轉數遭。一個隨風更面目,一個立地把身搖。那個大睜火眼伸猿臂,這個明幌金睛折虎腰。你來我去交鋒戰,刀迎棒架不相饒。猴王鐵棍依三略,怪物鋼刀按六韜。一個慣行手段為魔主,一個廣施法力保唐僧。猛烈的猴王添猛烈,英豪的怪物長英豪。死生不顧空中打,都為唐僧拜佛遙。
\end{quote}

他兩個戰有五六十合,不分勝負。行者心中暗喜道:「這個潑怪,他那口刀倒也抵得住老孫的這根棒。等老孫丟個破綻與他,看他可認得?」好猴王,雙手舉棍,使一個「高探馬」的勢子。那怪不識是計,見有空兒,舞著寶刀,徑奔下三路砍;被行者急轉個「大中平」,挑開他那口刀,又使個「葉底偷桃勢」,望妖精頭頂一棍,就打得他無影無蹤。急收棍子看處,不見了妖精。行者大驚道:「我兒啊,不禁打,就打得不見了。果是打死,好道也有些膿血,如何沒一毫蹤影?想是走了。」急縱身跳在雲端裡看處,四邊更無動靜。「老孫這雙眼睛,不管那裡,一抹都見,卻怎麼走得這等溜撒?我曉得了,那怪說有些兒認得我,想必不是凡間的怪,多是天上來的精。」

那大聖一時忍不住怒發,揝著鐵棒,打個觔斗,只跳到南天門上。慌得那龐、劉、苟、畢,張、陶、鄧、辛等眾,兩邊躬身控背,不敢攔阻,讓他打入天門,直至通明殿下。早有張、葛、許、丘四大天師問道:「大聖何來?」行者道:「因保唐僧至寶象國,有一妖魔欺騙國女,傷害吾師,老孫與他賭鬥。正鬥間,不見了這怪。想那怪不是凡間之怪,多是天上之精,特來查勘,那一路走了甚麼妖神?」天師聞言,即進靈霄殿上啟奏。蒙差查勘九曜星官、十二元辰、東西南北中央五斗、河漢群辰、五岳四瀆、普天神聖都在天上,更無一個敢離方位。又查那斗牛宮外二十八宿,顛倒只有二十七位,內獨少了奎星。天師回奏道:「奎木狼下界了。」玉帝道:「多少時不在天了?」天師道:「四卯不到,三日點卯一次,今已十三日了。」玉帝道:「天上十三日,下界已是十三年。」即命本部收他上界。

那二十七宿星員領了旨意,出了天門,各念咒語,驚動奎星。你道他在那裡躲避?他原來是孫大聖大鬧天宮時打怕了的神將,閃在那山澗裡潛災,被水氣隱住妖雲,所以不曾看見他。他聽得本部星員念咒,方敢出頭,隨眾上界。被大聖攔住天門要打,幸虧眾星勸住,押見玉帝。那怪腰間取出金牌,在殿下叩頭納罪。玉帝道:「奎木狼,上界有無邊的勝景,你不受用,卻私走一方,何也?」奎宿叩頭奏道:「萬歲,赦臣死罪。那寶象國王公主,非凡人也。他本是披香殿侍香的玉女,因欲與臣私通,臣恐點污了天宮勝境。他思凡先下界去,托生於皇宮內院。是臣不負前期,變作妖魔,占了名山,攝他到洞府,與他配了一十三年夫妻。『一飲一啄,莫非前定。』今被孫大聖到此成功。」玉帝聞言,收了金牌,貶他去兜率宮與太上老君燒火,帶俸差操,有功復職,無功重加其罪。

行者見玉帝如此發放,心中歡喜,朝上唱個大喏,又向眾神道:「列位,起動了。」天師笑道:「那個猴子還是這等村俗,替他收了怪神,也倒不謝天恩,卻就唱喏而退。」玉帝道:「只得他無事,落得天上清平是幸。」

那大聖按落祥光,徑轉碗子山波月洞,尋出公主,將那思凡下界收妖的言語正然陳訴。只聽得半空中八戒、沙僧厲聲高叫道:「師兄,有妖精,留幾個兒我們打耶。」行者道:「妖精已盡絕矣。」沙僧道:「既把妖精打絕,無甚罣礙,將公主引入朝中去罷。不要睜眼,兄弟們使個縮地法來。」

那公主只聞得耳內風響,霎時間徑回城裡,他三人將公主帶上金鑾殿上。那公主恭拜了父王、母后,會了姊妹。各官俱來拜見。那公主才啟奏道:「多虧孫長老法力無邊,降了黃袍怪,救奴回國。」那國王問曰:「黃袍是個甚怪?」行者道:「陛下的駙馬是上界的奎星,令愛乃侍香的玉女,因思凡降落人間。不非小可,都因前世前緣,該有這些姻眷。那怪被老孫上天宮啟奏玉帝,玉帝查得他四卯不到,下界十三日,就是十三年了。——蓋天上一日,下界一年。——隨差本部星宿收他上界,貶在兜率宮立功去訖。老孫卻救得令愛來也。」那國王謝了行者的恩德,便教:「看你師父去來。」

他三人徑下寶殿,與眾官到朝房裡,擡出鐵籠,將假虎解了鐵索。別人看他是虎,獨行者看他是人。原來那師父被妖術魘住,不能行走,心上明白,只是口眼難開。行者笑道:「師父啊,你是個好和尚,怎麼弄出這般個惡模樣來也?你怪我行兇作惡,趕我回去,你要一心向善,怎麼一旦弄出個這等嘴臉?」八戒道:「哥啊,救他救兒罷,不要只管揭挑他了。」行者道:「你凡事攛唆,是他個得意的好徒弟,你不救他,又尋老孫怎的?原與你說來,待降了妖精,報了罵我之仇,就回去的。」沙僧近前跪下道:「哥啊,古人云:『不看僧面看佛面。』兄長既是到此,萬望救他一救。若是我們能救,也不敢許遠的來奉請你也。」行者用手挽起道:「我豈有安心不救之理?快取水來。」那八戒飛星去驛中,取了行李、馬匹,將紫金缽盂取出,盛水半盂,遞與行者。行者接水在手,念動真言,望那虎劈頭一口噴上,退了妖術,解了虎氣。

長老現了原身,定性睜睛,才認得是行者。一把攙住道:「悟空,你從那裡來也?」沙僧侍立左右,把那請行者,降妖精,救公主,解虎氣,並回朝上項事,備陳了一遍。三藏謝之不盡道:「賢徒,虧了你也,虧了你也。這一去,早詣西方,徑回東土,奏唐王,你的功勞第一。」行者笑道:「莫說,莫說,但不念那話兒,足感愛厚之情也。」國王聞此言,又勸謝了他四眾。整治素筵,大開東閣。他師徒受了皇恩,辭王西去。國王又率多官遠送。這正是:
\begin{quote}
君回寶殿定江山,僧去雷音參佛祖。
\end{quote}

畢竟不知此後又有甚事,幾時得到西天,且聽下回分解。
