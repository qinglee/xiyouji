
\chapter{滌垢洗心惟掃塔 縛魔歸主乃修身}

\begin{quote}
十二時中忘不得,行功百刻全收。五年十萬八千周。休教神水涸,莫縱火光愁。
水火調停無損處,五行聯絡如鉤。陰陽和合上雲樓。乘鸞登紫府,跨鶴赴瀛洲。
\end{quote}

這一篇詞,牌名《臨江仙》,單道唐三藏師徒四眾水火既濟,本性清涼。借得純陰寶扇,搧息燥火過山。不一日行過了八百之程,師徒們散誕逍遙,向西而去。正值秋末冬初時序,見了些:
\begin{quote}
野菊殘英落,新梅嫩蕊生。村村納禾稼,處處食香羹。平林木落遠山現,曲澗霜濃幽壑清。應鍾氣,閉蟄營。純陰陽,月帝玄溟;盛水德,舜日憐晴。地氣下降,天氣上升。虹藏不見影,池沼漸生冰。懸崖掛索藤花敗,松竹凝寒色更青。
\end{quote}

四眾行夠多時,前又遇城池相近。唐僧勒住馬,叫徒弟:「悟空,你看那廂樓閣崢嶸,是個甚麼去處?」行者擡頭觀看,乃是一座城池。真個是:
\begin{quote}
龍蟠形勢,虎踞金城。四垂華蓋近,百轉紫墟平。玉石橋欄排巧獸,黃金臺座列賢明。真個是神洲都會,天府瑤京。萬里邦畿固,千年帝業隆。蠻夷拱服君恩遠,海岳朝元聖會盈。御階潔淨,輦路清寧。酒肆歌聲鬧,花樓喜氣生。未央宮外長春樹,應許朝陽彩鳳鳴。
\end{quote}

行者道:「師父,那座城池是一國帝王之所。」八戒笑道:「天下府有府城,縣有縣城,怎麼就見是帝王之所?」行者道:「你不知帝王之居,與府縣自是不同。你看他四面有十數座門,週圍有百十餘里,樓臺高聳,雲霧繽紛。非帝京邦國,何以有此壯麗?」沙僧道:「哥哥眼明,雖識得是帝王之處,卻喚做甚麼名色?」行者道:「又無牌匾旌號,何以知之?須到城中詢問,方可知也。」

長老策馬,須臾到門。下馬過橋,進門觀看。只見六街三市,貨殖通財;又見衣冠隆盛,人物豪華。正行時,忽見有十數個和尚,一個個披枷戴鎖,沿門乞化,著實的藍縷不堪。三藏嘆曰:「兔死狐悲,物傷其類。」叫:「悟空,你上前去問他一聲,為何這等遭罪?」行者依言,即叫:「那和尚,你是那寺裡的?為甚事披枷戴鎖?」眾僧跪倒道:「爺爺,我等是金光寺負屈的和尚。」行者道:「金光寺坐落何方?」眾僧道:「轉過隅頭就是。」行者將他帶在唐僧前,問道:「怎生負屈,你說我聽。」眾僧道:「爺爺,不知你們是那方來的,我等似有些面善。不敢在此奉告,請到荒山,具說苦楚。」長老道:「也是,我們且到他那寺中去,仔細詢問緣由。」

同至山門,門上橫寫七個金字:「敕建護國金光寺」。師徒們進得門來觀看,但見那:
\begin{quote}
古殿香燈冷,虛廊葉掃風。凌雲千尺塔,養性幾株松。滿地落花無客過,簷前蛛網任攀籠。空架鼓,枉懸鐘,繪壁塵多彩像朦。講座幽然僧不見,禪堂靜矣鳥常逢。淒涼堪歎息,寂寞苦無窮。佛前雖有香爐設,灰冷花殘事事空。
\end{quote}

三藏心酸,止不住眼中出淚。眾僧們頂著枷鎖,將正殿推開,請長老上殿拜佛。長老進殿,奉上心香,叩齒三咂。卻轉於後面,見那方丈簷柱上又鎖著六七個小和尚,三藏甚不忍見。

及到方丈,眾僧俱來叩頭,問道:「列位老爺像貌不一,可是東土大唐來的麼?」行者笑道:「這和尚有甚未卜先知之法?我們正是。你怎麼認得?」眾僧道:「爺爺,我等有甚未卜先知之法?只是痛負了屈苦,無處分明,日逐家只是叫天叫地。想是驚動天神,昨日夜間,各人都得一夢:說有個東土大唐來的聖僧,救得我等性命,庶此冤苦可伸。今日果見老爺這般異像,故認得也。」

三藏聞言,大喜道:「你這裡是何地方?有何冤屈?」眾僧跪告:「爺爺,此城名喚祭賽國,乃西邦大去處。當年有四夷朝貢:南,月陀國;北,高昌國;東,西梁國;西,本缽國。年年進貢美玉、明珠、嬌妃、駿馬。我這裡不動干戈,不去征討,他那裡自然拜為上邦。」三藏道:「既拜為上邦,想是你這國王有道,文武賢良。」眾僧道:「爺爺,文也不賢,武也不良,國君也不是有道。我這金光寺,自來寶塔上祥雲籠罩,瑞靄高升:夜放霞光?萬里有人曾見;晝噴彩氣,四國無不同瞻。故此以為天府神京,四夷朝貢。只是三年之前,孟秋朔日,夜半子時,下了一場血雨。天明時,家家害怕,戶戶生悲。眾公卿奏上國王,不知天公甚事見責。當時延請道士打醮,和尚看經,答天謝地。誰曉得我這寺裡黃金寶塔污了,這兩年外國不來朝貢。我王欲要征伐,眾臣諫道:我寺裡僧人偷了塔上寶貝,所以無祥雲瑞靄,外國不朝。昏君更不察理。那些贓官將我僧眾拿了去,千般拷打,萬樣追求。當時我這裡有三輩和尚:前兩輩已被拷打不過,死了;如今又捉我輩,問罪枷鎖。老爺在上,我等怎敢欺心,盜取塔中之寶!萬望爺爺憐念,方以類聚,物以群分,捨大慈大悲,廣施法力,拯救我等性命。」

三藏聞言,點頭歎道:「這樁事暗昧難明。一則是朝廷失政,二來是汝等有災。既然天降血雨,污了寶塔,那時節何不啟本奏君,致令受苦?」眾僧道:「爺爺,我等凡人,怎知天意?況前輩俱未辨得,我等如何處之?」三藏道:「悟空,今日甚時分了?」行者道:「有申時前後。」三藏道:「我欲面君倒換關文,奈何這眾僧之事不得明白,難以對君奏言。我當時離了長安,在法門寺裡立願:上西方逢廟燒香,遇寺拜佛,見塔掃塔。今日至此,遇有受屈僧人,乃因寶塔之累。你與我辦一把新笤帚,待我沐浴了,上去掃掃,即看這污穢之事何如,不放光之故何如,訪著端的,方好面君奏言,解救他們這苦難也。」

這些枷鎖的和尚聽說,連忙去廚房取把廚刀,遞與八戒道:「爺爺,你將此刀打開那柱子上鎖的小和尚鐵鎖,放他去安排齋飯香湯,伏侍老爺進齋沐浴。我等且上街化把新笤帚來與老爺掃塔。」八戒笑道:「開鎖有何難哉?不用刀斧,教我那一位毛臉老爺,他是開鎖的積年。」行者真個近前,使個解鎖法,用手一抹,幾把鎖俱退落下。那小和尚俱跑到廚中,淨刷鍋灶,安排茶飯。三藏師徒們吃了齋,漸漸天昏。只見那枷鎖的和尚拿了兩把笤帚進來,三藏甚喜。

正說處,一個小和尚點了燈來請洗澡。此時滿天星月光輝,譙樓上更鼓齊發。正是那:
\begin{quote}
四壁寒風起,萬家燈火明。
六街關戶牖,三市閉門庭。
釣艇歸深樹,耕犁罷短繩。
樵夫柯斧歇,學子誦書聲。
\end{quote}

三藏沐浴畢,穿了小袖褊衫,束了環絛,足下換一雙軟公鞋,手裡拿一把新笤帚,對眾僧道:「你等安寢,待我掃塔去來。」行者道:「塔上既被血雨所污,又況日久無光,恐生惡物;一則夜靜風寒,又沒個伴侶:自去恐有差池,老孫與你同上如何?」三藏道:「甚好,甚好。」兩人各持一把,先到大殿上,點起琉璃燈,燒了香,佛前拜道:「弟子陳玄奘奉東土大唐差往靈山參見我佛如來取經,今至祭賽國金光寺,遇本僧言寶塔被污,國王疑僧盜寶,啣冤取罪,上下難明。弟子竭誠掃塔,望我佛威靈,早示污塔之原因,莫致凡夫之冤屈。」祝罷,與行者開了塔門,自下層望上而掃。只見這塔,真是:
\begin{quote}
崢嶸倚漢,突兀凌空。正喚做五色琉璃塔,千金舍利峰。梯轉如穿窟,門開似出籠。寶瓶影射天邊月,金鐸聲傳海上風。但見那虛簷拱斗,絕頂留雲。虛簷拱斗,作成巧石穿花鳳;絕頂留雲,造就浮屠遶霧龍。遠眺可觀千里外,高登似在九霄中。層層門上琉璃燈,有塵無火;步步簷前白玉欄,積垢飛蟲。塔心裡,佛座上,香煙盡絕;窗櫺外,神面前,蛛網牽朦。爐中多鼠糞,盞內少油鎔。只因暗失中間寶,苦殺僧人命落空。三藏發心將塔掃,管教重見舊時容。
\end{quote}

唐僧用帚子掃了一層,又上一層。如此掃至第七層上,卻早二更時分。那長老漸覺困倦,行者道:「困了,你且坐下,等老孫替你掃罷。」三藏道:「這塔是多少層數?」行者道:「怕不有十三層哩。」長老耽著勞倦道:「是必掃了,方趁本願。」又掃了三層,腰酸腿痛,就於十層上坐倒道:「悟空,你替我把那三層掃淨下來罷。」行者抖擻精神,登上第十一層,霎時又上到第十二層。正掃處,只聽得塔頂上有人言語。行者道:「怪哉!怪哉!這早晚有三更時分,怎麼得有人在這頂上言語?斷乎是邪物也,且看看去。」

好猴王,輕輕的挾著笤帚,撒起衣服,鑽出前門,踏著雲頭觀看。只見第十三層塔心裡坐著兩個妖精,面前放一盤下飯、一隻碗、一把壺,在那裡猜拳吃酒哩。行者使個神通,丟了笤帚,掣出金箍棒,攔住塔門,喝道:「好怪物,偷塔上寶貝的原來是你。」兩個怪物慌了,急起身,拿壺拿碗亂摜。被行者橫鐵棒攔住道:「我若打死你,沒人供狀。」只把棒逼將去。那怪貼在壁上,莫想掙扎得動。口裡只叫:「饒命,饒命。不干我事,自有偷寶貝的在那裡也。」行者使個拿法,一隻手抓將過來,徑拿下第十層塔中,報道:「師父,拿住個偷寶貝之賊了。」三藏正自盹睡,忽聞此言,又驚又喜道:「是那裡拿來的?」行者把怪物揪到面前跪下道:「他在塔頂上猜拳吃酒耍子,是老孫聽得喧譁,一縱雲,跳到頂上攔住。未曾著力,但恐一棒打死,沒人供狀,故此輕輕捉來。師父可取他個口詞,看他是那裡妖精,偷的寶貝在於何處。」

那怪物戰戰兢兢,口叫「饒命」,遂從實供道:「我兩個是亂石山碧波潭萬聖龍王差來巡塔的。他叫做奔波兒灞,我叫做灞波兒奔;他是鮎魚怪,我是黑魚精。因我萬聖老龍生了一個女兒,就喚做萬聖公主。那公主花容月貌,有二十分人才。招得一個駙馬,喚做九頭駙馬,神通廣大。前年與龍王來此,顯大法力,下了一陣血雨,污了寶塔,偷了塔中的舍利子佛寶。公主又去大羅天上,靈霄殿前,偷了王母娘娘的九葉靈芝草,養在那潭底下,金光霞彩,晝夜光明。近日聞得有個孫悟空往西天取經,說他神通廣大,沿路上專一尋人的不是,所以這些時常差我等來此巡探,若還有那孫悟空到時,好準備也。」行者聞言,嘻嘻冷笑道:「那孽畜等這等無禮,怪道前日請牛魔王在那裡赴會,原來他結交這夥潑魔,專幹不良之事。」

說未了,只見八戒與兩三個小和尚自塔下提著兩個燈籠,走上來道:「師父,掃了塔不去睡覺,在這裡講甚麼哩?」行者道:「師弟,你來正好。塔上的寶貝,乃是萬聖老龍偷了去。今著這兩個小妖巡塔,探聽我等來的消息,卻才被我拿住也。」八戒道:「叫做甚麼名字?甚麼妖精?」行者道:「才然供了口詞,一個叫做奔波兒灞,一個叫做灞波兒奔;一個是鮎魚怪,一個是黑魚精。」八戒掣鈀就打,道:「既是妖精,取了口詞,不打死待何待?」行者道:「你不知,且留著活的,好去見皇帝講話,又好做鑿眼去尋賊追寶。」好獃子,真個收了鈀,一家一個,都抓下塔來。那怪只叫:「饒命。」八戒道:「正要你鮎魚、黑魚做些鮮湯,與那負冤屈的和尚吃哩。」

兩三個小和尚喜喜歡歡,提著燈籠,引長老下了塔。一個先跑報眾僧道:「好了,好了,我們得見青天了,偷寶貝的妖怪已是爺爺們捉將來矣。」行者教:「拿鐵索來,穿了琵琶骨,鎖在這裡。汝等看守,我們睡覺去,明日再做理會。」那些和尚都緊緊的守著,讓三藏們安寢。

不覺的天曉。長老道:「我與悟空入朝,倒換關文去來。」長老即穿了錦襴袈裟,戴了毘盧帽,整束威儀,拽步前進。行者也束一束虎皮裙,整一整綿布直裰,取了關文同去。八戒道:「怎麼不帶這兩個妖賊去?」行者道:「待我們奏過了,自有駕帖著人來提他。」遂行至朝門外。看不盡那朱雀黃龍,清都絳闕。三藏到東華門,對閣門大使作禮道:「煩大人轉奏,貧僧是東土大唐差去西天取經者,意欲面君,倒換關文。」那黃門官果與通報,至階前奏道:「外面有兩個異容異服僧人,稱言南贍部洲東土唐朝差往西方拜佛求經,欲朝我王,倒換關文。」

國王聞言,傳旨教宣。長老即引行者入朝。文武百官見了行者,無不驚怕。有的說是猴和尚,有的說是雷公嘴和尚。個個悚然,不敢久視。長老在階前舞蹈山呼的行拜。大聖叉著手,斜立在傍,公然不動。長老啟奏道:「臣僧乃南贍部洲東土大唐國差來拜西方天竺國大雷音寺佛,求取真經者。路經寶方,不敢擅過,有隨身關文,乞倒驗方行。」那國王聞言大喜,傳旨教宣唐朝聖僧上金鑾殿,安繡墩賜坐。長老獨自上殿,先將關文捧上,然後謝恩敢坐。

那國王將關文看了一遍,心中喜悅道:「似你大唐王有疾,能選高僧,不避路途遙遠,拜佛取經;寡人這裡和尚,專心只是做賊,敗國傾君。」三藏聞言,合掌道:「怎見得敗國傾君?」國王道:「寡人這國,乃是西域上邦,常有四夷朝貢,皆因國內有個金光寺,寺內有座黃金寶塔,塔上有光彩沖天。近被本寺賊僧暗竊了其中之寶,三年無有光彩,外國這三年也不來朝,寡人心痛恨之。」三藏合掌笑道:「萬歲,『差之毫釐,失之千里』矣。貧僧昨晚到於天府,一進城門,就見十數個枷紐之僧。問及何罪,他道是金光寺負冤屈者。因到寺細審,更不干本寺僧人之事。貧僧入夜掃塔,已獲那偷寶之妖賊矣。」國王大喜道:「妖賊安在?」三藏道:「現被小徒鎖在金光寺裡。」

那國王急降金牌:「著錦衣衛快到金光寺取妖賊來,寡人親審。」三藏又奏道:「萬歲,雖有錦衣衛,還得小徒去方可。」國王道:「高徒在那裡?」三藏用手指道:「那玉階旁立者便是。」國王見了,大驚道:「聖僧如此丰姿,高徒怎麼這等像貌?」孫大聖聽見了,厲聲高叫道:「陛下,『人不可貌相,海水不可斗量』。若愛丰姿者,如何捉得妖賊也?」國王聞言,回驚作喜道:「聖僧說的是。朕這裡不選人材,只要獲賊得寶歸塔為上。」再著當駕官看車蓋,教錦衣衛好生伏侍聖僧去取妖賊來。那當駕官即備大轎一乘、黃傘一柄,錦衣衛點起校尉,將行者八擡八綽,大四聲喝路,徑至金光寺。自此驚動滿城百姓,無處無一人不來看聖僧及那妖賊。

八戒、沙僧聽得喝道,只說是國王差官,急出迎接,原來是行者坐在轎上。獃子當面笑道:「哥哥,你得了本身也。」行者下了轎,攙著八戒道:「我怎麼得了本身?」八戒道:「你打著黃傘,擡著八人轎,卻不是猴王之職分?故說你得了本身。」行者道:「且莫取笑。」遂解下兩個妖物,押見國王。沙僧道:「哥哥,也帶挈小弟帶挈。」行者道:「你只在此看守行李、馬匹。」那枷鎖之僧道:「爺爺們都去承受皇恩,等我們在此看守。」行者道:「既如此,等我去奏過國王,卻來放你。」八戒揪著一個妖賊,沙僧揪著一個妖賊,孫大聖依舊坐了轎,擺開頭搭,將兩個妖怪押赴當朝。

須臾,至白玉階對國王道:「那妖賊已取來了。」國王下降龍床,與唐僧及文武多官,同目視之。那怪一個是暴腮烏甲,尖嘴利牙;一個是滑皮大肚,巨口長鬚。雖然是有足能行,大抵是變成的人像。國王問曰:「你是何方賊怪,那處妖精?幾年侵吾國土,何年盜我寶貝?一夥共有多少賊徒,都喚做甚麼名字?從實一一供來。」二怪朝上跪下,頸內血淋淋的,更不知疼痛。供道:「三載之外,七月初一,有個萬聖龍王,帥領許多親戚,住居在本國東南,離此處路有百十。潭號碧波,山名亂石。生女多嬌,妖嬈美色。招贅一個九頭駙馬,神通無敵。他知你塔上珍奇,與龍王合盤做賊,先下血雨一場,後把舍利偷訖。見如今照耀龍宮,縱黑夜明如白日。公主施能,寂寂密密,又偷了王母靈芝,在潭中溫養寶物。我兩個不是賊頭,乃龍王差來小卒。今夜被擒,所供是實。」國王道:「既取了供,如何不供自家名字?」那怪道:「我喚做奔波兒灞,他喚做灞波兒奔。奔波兒灞是個鮎魚怪,灞波兒奔是個黑魚精。」國王教錦衣衛好生收監。傳旨:「赦了金光寺眾僧的枷鎖。快教光祿寺排宴,就於麒麟殿上謝聖僧獲賊之功,議請聖僧捕擒賊首。」

光祿寺即時備了葷素兩樣筵席。國王請唐僧四眾上麒麟殿敘坐,問道:「聖僧尊號?」唐僧合掌道:「貧僧俗家姓陳,法名玄奘。蒙君賜姓唐,賤號三藏。」國王又問:「聖僧高徒何號?」三藏道:「小徒俱無號。第一個名孫悟空,第二個名豬悟能,第三個名沙悟淨:此乃南海觀世音菩薩起的名字。因拜貧僧為師,貧僧又將悟空叫做行者,悟能叫做八戒,悟淨叫做和尚。」國王聽畢,請三藏坐了上席,孫行者坐了側首左席,豬八戒、沙和尚坐了側首右席。俱是素果、素菜、素茶、素飯。前面一席葷的,坐了國王;下首有百十席葷的,坐了文武多官。眾臣謝了君恩,徒告了師罪,坐定。國王把盞,三藏不敢飲酒,他三個各受了安席酒。下邊只聽得管弦齊奏,乃是教坊司動樂。你看八戒放開食嗓,真個是虎咽狼吞,將一席果菜之類,吃得罄盡。少頃間,添換湯飯又來,又吃得一毫不剩。巡酒的來,又杯杯不辭。這場筵席,直樂到午後方散。

三藏謝了盛宴。國王又留住道:「這一席聊表聖僧獲怪之功。」教光祿寺:「快翻席到建章宮裡,再請聖僧定捕賊首、取寶歸塔之計。」三藏道:「既要捕賊取寶,不勞再宴。貧僧等就此辭王,就擒捉妖怪去也。」國王不肯,一定請到建章宮,又吃了一席。國王舉酒道:「那位聖僧帥眾出師,降妖捕怪?」三藏道:「教大徒弟孫悟空去。」大聖拱手應承。國王道:「孫長老既去,用多少人馬?幾時出城?」八戒忍不住高聲叫道:「那裡用甚麼人馬?又那裡管甚麼時辰?趁如今酒醉飯飽,我共師兄去,手到擒來。」三藏甚喜道:「八戒這一向勤緊啊!」行者道:「既如此,著沙僧弟保護師父,我兩個去來。」那國王道:「二位長老既不用人馬,可用兵器?」八戒笑道:「你家的兵器,我們用不得,我弟兄自有隨身器械。」國王聞說,即取大觥來,與二位長老送行。孫大聖道:「酒不吃了,只教錦衣衛把兩個小妖拿來,我們帶了他去做鑿眼。」國王傳旨,即時提出。二人扯著兩個小妖,駕風頭,使個攝法,徑上東南去了。噫!他那:
\begin{quote}
君臣一見騰風霧,才識師徒是聖僧。
\end{quote}

畢竟不知此去如何擒獲,且聽下回分解。
