
\chapter{脫難江流來國土 承恩八戒轉山林}

詩曰:
\begin{quote}
妄想不復強滅,真如何必希求。
本原自性佛前修,迷悟豈居前後。
悟即剎那成正,迷而萬劫沉流。
若能一念合真修,滅盡恆沙罪垢。
\end{quote}

卻說那八戒、沙僧與怪鬥經個三十回合,不分勝負。你道怎麼不分勝負?若論賭手段,莫說兩個和尚,就是二十個也敵不過那妖精。只為唐僧命不該死,暗中有那護法神祇保著他;空中又有那六丁六甲、五方揭諦、四值功曹、一十八位護教伽藍助著八戒、沙僧。

且不言他三人戰鬥。卻說那長老在洞裡悲啼,思量他那徒弟,眼中流淚道:「悟能啊,不知你在那個村中逢了善友,貪著齋供?悟淨啊,你又不知在那裡尋他,可能得會?豈知我遇妖魔,在此受難?幾時得會你們,脫了大難,早赴靈山?」正當悲啼煩惱,忽見那洞內走出一個婦人來,扶著定魂樁,叫道:「那長老,你從何來?為何被他縛在此處?」長老聞言,淚眼偷看,那婦人約有三十年紀。遂道:「女菩薩,不消問了。我已是該死的,走進你家門來也,要吃就吃了罷,又問怎的?」那婦人道:「我不是吃人的。我家離此西下有三百餘里,那裡有座城,叫做寶象國。我是那國王的第三個公主,乳名叫做百花羞。只因十三年前八月十五日夜,玩月中間,被這妖魔一陣狂風攝將來,與他做了十三年夫妻,在此生兒育女,杳無音信回朝。思量我那父母,不能相見。你從何來,被他拿住?」唐僧道:「貧僧乃是差往西天取經者,不期閑步,誤撞在此。如今要拿住我兩個徒弟,一齊蒸吃哩。」那公主陪笑道:「長老寬心,你既是取經的,我救得你。那寶象國是你西方去的大路,你與我捎一封書兒去,拜上我那父母,我就教他饒了你罷。」三藏點頭道:「女菩薩,若還救得貧僧命,願做捎書寄信人。」

那公主急轉後面,即修了一紙家書,封固停當。到樁前解放了唐僧,將書付與。唐僧得解脫,捧書在手道:「女菩薩,多謝你活命之恩。貧僧這一去,過貴處,定送國王處。只恐日久年深,你父母不肯相認,奈何?切莫怪我貧僧打了誑語。」公主道:「不妨,我父王無子,止生我三個姊妹,若見此書,必有相看之意。」三藏緊緊袖了家書,謝了公主,就往外走。被公主扯住道:「前門裡你出不去,那些大小妖精都在門外搖旗吶喊,擂鼓篩鑼,助著大王,與你徒弟廝殺哩。你往後門裡去罷。若是大王拿住,還審問審問;只恐小妖兒捉了,不分好歹,挾生兒傷了你的性命。等我去他面前說個方便。若是大王放了你啊,待你徒弟討個示下,尋著你一同好走。」三藏聞言,磕了頭,謹依吩咐,辭別公主,躲離後門之外,不敢自行,將身藏在荊棘叢中。

卻說公主娘娘心生巧計,急往前來,出門外,分開了大小群妖。只聽得叮叮噹噹,兵刃亂響。原來是八戒、沙僧與那怪在半空裡廝殺哩。這公主厲聲高叫道:「黃袍郎!」那妖王聽得公主叫喚,即丟了八戒、沙僧,按落雲頭,揪了鋼刀,攙著公主道:「渾家,有甚話說?」公主道:「郎君啊,我才時睡在羅幃之內,夢魂中,忽見個金甲神人。」妖魔道:「那個金甲神?上我門怎的?」公主道:「是我幼時在宮內,對人暗許下一樁心願:若得招個賢郎駙馬,上名山,拜仙府,齋僧佈施。自從配了你,夫妻們歡會,到今不曾題起。那金甲神人來討誓願,喝我醒來,卻是南柯一夢。因此,急整容來郎君處訴知,不期那樁上綁著一個僧人。萬望郎君慈憫,看我薄意,饒了那個和尚罷,只當與我齋僧還願。不知郎君肯否?」那怪道:「渾家,你卻多心吶,甚麼打緊之事。我要吃人,那裡不撈幾個吃吃,這個把和尚到得那裡?放他去罷。」公主道:「郎君,放他從後門裡去罷。」妖魔道:「奈煩哩,放他去便罷,又管他甚麼後門前門哩。」他遂綽了鋼刀,高叫道:「那豬八戒,你過來。我不是怕你,不與你戰;看著我渾家的分上,饒了你師父也。趁早去後門首尋著他,往西方去罷。若再來犯我境界,斷乎不饒。」

那八戒與沙僧聞得此言,就如鬼門關上放回來的一般,即忙牽馬挑擔,鼠攛而行。轉過那波月洞後門之外,叫聲:「師父。」那長老認得聲音,就在那荊棘中答應。沙僧就剖開草徑,攙著師父,慌忙的上馬。這裡:
\begin{quote}
狠毒險遭青面鬼,慇懃幸有百花羞。
鰲魚脫卻金鉤釣,擺尾搖頭逐浪遊。
\end{quote}

八戒當頭領路,沙僧後隨,出了那松林,上了大路。你看他兩個嚌嚌嘈嘈,埋埋怨怨,三藏只是解和。遇晚先投宿,雞鳴早看天。一程一程,長亭短亭,不覺的就走了二百九十九里。猛擡頭,只見一座好城,就是寶象國。真好個處所也:
\begin{quote}
雲渺渺,路迢迢。地雖千里外,景物一般饒。瑞靄祥煙籠罩,清風明月招搖。嵂嵂崒崒的遠山,大開圖畫;潺潺湲湲的流水,碎濺瓊瑤。可耕的連阡帶陌,足食的密蕙新苗。漁釣的幾家三澗曲,樵採的一擔兩峰椒。廓的廓,城的城,金湯鞏固;家的家,戶的戶,只鬥逍遙。九重的高閣如殿宇,萬丈的層臺似錦標。也有那太極殿、華蓋殿、燒香殿、觀文殿、宣政殿、延英殿,一殿殿的玉陛金階,擺列著文冠武弁;也有那大明宮、昭陽宮、長樂宮、華清宮、建章宮、未央宮,一宮宮的鐘鼓管籥,撒抹了閨怨春愁。也有禁苑的露花勻嫩臉,也有御溝的風柳舞纖腰。通衢上,也有個頂冠束帶的,盛儀容,乘五馬;幽僻中,也有個持弓挾矢的,撥雲霧,貫雙鵰。花柳的巷,管弦的樓,春風不讓洛陽橋。取經的長老,回首大唐肝膽裂;伴師的徒弟,息肩小驛夢魂消。
\end{quote}

看不盡寶象國的景致。師徒三眾收拾行李、馬匹,安歇館驛中。

唐僧步行至朝門外,對閣門大使道:「有唐朝僧人,特來面駕,倒換文牒,乞為轉奏轉奏。」那黃門奏事官連忙走至白玉階前奏道:「萬歲,唐朝有個高僧,欲求見駕,倒換文牒。」那國王聞知是唐朝大國,且又說是個方上聖僧,心中甚喜,即時准奏。叫:「宣他進來。」把三藏宣至金階,舞蹈山呼禮畢。兩邊文武多官無不嘆道:「上邦人物,禮樂雍容如此。」那國王道:「長老,你到我國中何事?」三藏道:「小僧是唐朝釋子,承我天子敕旨,前往西方取經。原領有文牒,到陛下上國,理合倒換。故此不識進退,驚動龍顏。」國王道:「既有唐天子文牒,取上來看。」三藏雙手捧上去,展開放在御案上。牒云:
\begin{quote}
南贍部洲大唐國奉天承運唐天子牒行:切惟朕以涼德,嗣續丕基,事神治民,臨深履薄,朝夕是惴。前者失救涇河老龍,獲譴於我皇皇后帝,三魂七魄,倏忽陰司,已作無常之客。因有陽壽未絕,感冥君放送回生,廣陳善會,修建度亡道場。感蒙救苦觀世音菩薩金身出現,指示西方有佛有經,可度幽亡,超脫孤魂。特著法師玄奘,遠歷千山,詢求經偈。倘到西邦諸國,不滅善緣,照牒放行。須至牒者。大唐貞觀一十三年秋吉日,御前文牒。(上有寶印九顆)
\end{quote}

國王見了,取本國玉寶,用了花押,遞與三藏。

三藏謝了恩,收了文牒,又奏道:「貧僧一來倒換文牒,二來與陛下寄有家書。」國王大喜道:「有甚書?」三藏道:「陛下第三位公主娘娘,被碗子山波月洞黃袍妖攝將去,貧僧偶爾相遇,故寄書來也。」國王聞言,滿眼垂淚道:「自十三年前不見了公主,兩班文武官也不知貶退了多少,宮內宮外大小婢子、太監也不知打死了多少;只說是走出皇宮,迷失路徑,無處找尋。滿城中百姓人家,也盤詰了無數,更無下落。怎知道是妖怪攝了去。今日乍聽得這句話,故此傷情流淚。」三藏袖中取出書來獻上。國王接了,見有「平安」二字,一發手軟,拆不開書。傳旨宣翰林院大學士上殿讀書。學士隨即上殿。殿前有文武多官,殿後有后妃宮女,俱側耳聽書。學士拆開朗誦。上寫著:
\begin{quote}
不孝女百花羞頓首百拜大德父王萬歲龍鳳殿前,暨三宮母后昭陽宮下,及舉朝文武賢卿臺次:拙女幸托坤宮,感激劬勞萬種。不能竭力怡顏,盡心奉孝。乃於十三年前八月十五日良夜佳辰,蒙父王恩旨,著各宮排宴,賞玩月華,共樂清霄盛會。正歡娛之間,不覺一陣香風,閃出個金睛藍面青髮魔王,將女擒住,駕祥光,直帶至半野山中無人處,難分難辨,被妖倚強,霸占為妻。是以無奈捱了一十三年,產下兩個妖兒,盡是妖魔之種。論此真是敗壞人倫,有傷風化,不當傳書玷辱。但恐女死之後,不顯分明。正含怨思憶父母,不期唐朝聖僧亦被魔王擒住。是女滴淚修書,大膽放脫,特托寄此片楮,以表寸心。伏望父王垂憫,遣上將早至碗子山波月洞捉獲黃袍怪,救女回朝,深為恩念。草草欠恭,面聽不一。
逆女百花羞再頓首頓首。
\end{quote}

那學士讀罷家書,國王大哭,三宮滴淚,文武傷情,前前後後,無不哀念。

國王哭之許久,便問兩班文武:「那個敢興兵領將,與寡人捉獲妖魔,救我百花公主?」連問數聲,更無一人敢答。真是木雕成的武將,泥塑就的文官。那國王心生煩惱,淚若湧泉。只見那多官齊俯伏奏道:「陛下且休煩惱。公主已失,至今一十三載無音,偶遇唐朝聖僧,寄書來此,未知的否。況臣等俱是凡人凡馬,習學兵書武略,止可佈陣安營,保國家無侵陵之患。那妖精乃雲來霧去之輩,不得與他覿面相見,何以征救?想東土取經者,乃上邦聖僧。這和尚道高龍虎伏,德重鬼神欽,必有降妖之術。自古道:『來說是非者,就是是非人。』可就請這長老降妖邪,救公主,庶為萬全之策。」

那國王聞言,急回頭,便請三藏道:「長老若有手段,放法力,捉了妖魔,救我孩兒回朝,也不須上西方拜佛,長髮留頭,朕與你結為兄弟,同坐龍床,共享富貴如何?」三藏慌忙啟上道:「貧僧粗知念佛,其實不會降妖。」國王道:「你既不會降妖,怎麼敢上西天拜佛?」那長老瞞不過,說出兩個徒弟來了。奏道:「陛下,貧僧一人,實難到此。貧僧有兩個徒弟,善能逢山開路,遇水疊橋,保貧僧到此。」國王怪道:「你這和尚大沒理,既有徒弟,怎麼不與他一同進來見朕?若到朝中,雖無中意賞賜,必有隨分齋供。」三藏道:「貧僧那徒弟醜陋,不敢擅自入朝,但恐驚傷了陛下的龍體。」國王笑道:「你看你這和尚說話,終不然朕當怕他?」三藏道:「不敢說。我那大徒弟姓豬,名悟能八戒,他生得長嘴獠牙,剛鬃扇耳,身粗肚大,行路生風。第二個徒弟姓沙,法名悟淨和尚,他生得身長丈二,臂闊三停,臉如藍靛,口似血盆,眼光閃灼,牙齒排釘。他都是這等個模樣,所以不敢擅領入朝。」國王道:「你既這等樣說了一遍,寡人怕他怎的?宣進來。」隨即著金牌至館驛相請。

那獃子聽見來請,對沙僧道:「兄弟,你還不教下書哩,這才見了下書的好處。想是師父下了書,國王道,捎書人不可怠慢,一定整治筵宴待他;他的食腸不濟,有你我之心,舉出名來,故此著金牌來請。大家吃一頓,明日好行。」沙僧道:「哥啊,知道是甚緣故,我們且去來?」遂將行李、馬匹俱交付驛丞,各帶隨身兵器,隨金牌入朝。早行到白玉階前,左右立下,朝上唱個喏,再也不動。那文武多官,無人不怕。都說道:「這兩個和尚貌醜也罷,只是粗俗太甚,怎麼見我王更不下拜,喏畢平身,挺然而立?可怪,可怪。」八戒聽見道:「列位,莫要議論,我們是這般:乍看果有些醜,只是看下些時來,卻也耐看。」

那國王見他醜陋,已是心驚。及聽得那獃子說出話來,越發膽顫,就坐不穩,跌下龍床。幸有近侍官員扶起。慌得個唐僧跪在殿前,不住的叩頭道:「陛下,貧僧該萬死,萬死。我說徒弟醜陋,不敢朝見,恐傷龍體,果然驚了駕也。」那國王戰兢兢走近前,攙起道:「長老,還虧你先說過了;若未說,猛然見他,寡人一定諕殺了也。」國王定性多時,便問:「豬長老、沙長老,是那一位善於降妖?」那獃子不知好歹,答道:「老豬會降。」國王道:「怎麼家降?」八戒道:「我乃是天蓬元帥,只因罪犯天條,墮落下世,幸今皈正為僧。自從東土來此,第一會降妖的是我。」國王道:「既是天將臨凡,必然善能變化。」八戒道:「不敢,不敢,也將就曉得幾個變化兒。」國王道:「你試變一個我看看。」八戒道:「請出題目,照依樣子好變。」國王道:「變一個大的罷。」

那八戒他也有三十六般變化,就在階前賣弄手段,卻便捻訣念咒,喝一聲叫:「長!」把腰一躬,就長有八九丈長,卻似個開路神一般。嚇得那兩班文武戰戰兢兢,一國君臣呆呆掙掙。時有鎮殿將軍問道:「長老,似這等變得身高,必定長到甚麼去處,才有止極?」那獃子又說出獃話來道:「看風。東風猶可,西風也將就;若是南風起,把青天也拱個大窟窿。」那國王大驚道:「收了神通罷,曉得是這般變化了。」八戒把身一矬,依然現了本相,侍立階前。

國王又問道:「長老此去,有何兵器與他交戰?」八戒腰裡掣出鈀來道:「老豬使的是釘鈀。」國王笑道:「可敗壞門面。我這裡有的是鞭、簡、瓜、鎚,刀、槍、鉞、斧,劍、戟、矛、鐮,隨你選稱手的拿一件去。那鈀算做甚麼兵器?」八戒道:「陛下不知。我這鈀雖然粗夯,實是自幼隨身之器。曾在天河水府為帥,轄押八萬水兵,全仗此鈀之力。今臨凡世,保護吾師,逢山築破虎狼窩,遇水掀翻龍蜃穴,皆是此鈀。」

國王聞得此言,十分歡喜心信。即命九嬪妃子:「將朕親用的御酒整瓶取來,權與長老送行。」遂滿斟一爵,奉與八戒道:「長老,這杯酒,聊引奉勞之意。待捉得妖魔,救回小女,自有大宴相酬,千金重謝。」那獃子接杯在手,人物雖是粗魯,行事倒有斯文,對三藏唱個大喏道:「師父,這酒本該從你飲起;但君王賜我,不敢違背,讓老豬先吃了,助助興頭,好捉妖怪。」那獃子一飲而乾,才斟一爵,遞與師父。三藏道:「我不飲酒,你兄弟們吃罷。」沙僧近前接了。八戒就足下生雲,直上空裡。國王見了道:「豬長老又會騰雲?」

獃子去了,沙僧將酒亦一飲而乾,道:「師父,那黃袍怪拿住你時,我兩個與他交戰,只戰個手平。今二哥獨去,恐戰不過他。」三藏道:「正是,徒弟啊,你可去與他幫幫功。」沙僧聞言,也縱雲跳將起去。那國王慌了,扯住唐僧道:「長老,你且陪寡人坐坐,也莫騰雲去了。」唐僧道:「可憐,可憐,我半步兒也去不得。」此時二人在殿上敘話不題。

卻說那沙僧趕上八戒道:「哥哥,我來了。」八戒道:「兄弟,你來怎的?」沙僧道:「師父叫我來幫幫功的。」八戒大喜道:「說得是,來得好。我兩個努力齊心,去捉那怪物,雖不怎的,也在此國揚揚姓名。」你看他:
\begin{quote}
靉靆祥光辭國界,氤氳瑞氣出京城。
領王旨意來山洞,努力齊心捉怪靈。
\end{quote}

他兩個不多時到了洞口,按落雲頭。八戒掣鈀,往那波月洞的門上盡力氣一築,把他那石門築了斗來大小的個窟窿。嚇得那把門的小妖開門,看見是他兩個,急跑進去報道:「大王,不好了,那長嘴大耳的和尚,與那晦色氣臉的和尚,又來把門都打破了。」那怪驚道:「這個還是豬八戒、沙和尚二人。我饒了他師父,怎麼又敢復來打我的門?」小妖道:「想是忘了甚麼物件,來取的。」老怪咄的一聲道:「胡纏,忘了物件,就敢打上門來?必有緣故。」急整束了披掛,綽了鋼刀,走出來問道:「那和尚,我既饒了你師父,你怎麼又敢來打上我門?」八戒道:「你這潑怪幹得好事兒。」老魔道:「甚麼事?」八戒道:「你把寶象國三公主騙來洞內,倚強霸占為妻,住了一十三載,也該還他了。我奉國王旨意,特來擒你。你快快進去,自家把繩子綁縛出來,還免得老豬動手。」那老怪聞言,十分發怒。你看他屹迸迸,咬響鋼牙;滴溜溜,睜圓環眼;雄糾糾,舉起刀來;赤淋淋,攔頭便砍。八戒側身躲過,使釘鈀劈面迎來;隨後又有沙僧舉寶杖趕上前齊打。這一場在山頭上賭鬥,比前不同。真個是:
\begin{quote}
言差語錯招人惱,意毒情傷怒氣生。這魔王大鋼刀,著頭便砍;那八戒九齒鈀,對面來迎。沙悟淨丟開寶杖,那魔王抵架神兵。一猛怪,二神僧,來來往往甚消停。這個說:「你騙國理該死罪。」那個說:「你羅閑事報不平。」這個說:「你強婚公主傷國體。」那個說:「不干你事莫閑爭。」算來只為捎書故,致使僧魔兩不寧。
\end{quote}

他們在那山坡前戰經八九個回合,八戒漸漸不濟將來,釘鈀難舉,氣力不加。你道如何這等戰他不過?當時初相戰鬥,有那護法諸神,為唐僧在洞,暗助八戒、沙僧,故僅得個手平;此時諸神都在寶象國護定唐僧,所以二人難敵。

那獃子道:「沙僧,你且上前來與他鬥著,讓老豬出恭來。」他就顧不得沙僧,一溜往那蒿草薜蘿荊棘葛藤裡,不分好歹,一頓鑽進。那管刮破頭皮,搠傷嘴臉,一轂轆睡倒,再也不敢出來。但留半邊耳朵,聽著梆聲。

那怪見八戒走了,就奔沙僧。沙僧措手不及,被怪一把抓住,捉進洞去。小妖將沙僧四馬攢蹄綑住。

畢竟不知端的性命如何,且聽下回分解。
