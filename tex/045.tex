
\chapter{三清觀大聖留名 車遲國猴王顯法}

卻說孫大聖左手把沙和尚捻一把,右手把豬八戒捻一把,他二人卻就省悟。坐在高處,倥著臉,不言不語。憑那些道士點燈著火,前後照看,他三個就如泥塑金裝一般模樣。虎力大仙道:「沒有歹人,如何把供獻都吃了?」鹿力大仙道:「卻像人吃的勾當,有皮的都剝了皮,有核的都吐出核,卻怎麼不見人形?」羊力大仙道:「師兄勿疑。想是我們虔心敬意,在此晝夜誦經,前後申文,又是朝廷名號,斷然驚動天尊。想是三清爺爺聖駕降臨,受用了這些供養。趁今仙從未返,鶴駕在斯,我等可拜告天尊,懇求些聖水金丹,進與陛下,卻不是長生永壽,見我們的功果也?」虎力大仙道:「說的是。」教:「徒弟們動樂誦經。一壁廂取法衣來,等我步罡拜禱。」那些小道士俱遵命,兩班兒擺列齊整。噹的一聲磬響,齊念一卷《黃庭道德真經》。虎力大仙披了法衣,擎著玉簡,對面前舞蹈揚塵,拜伏於地,朝上啟奏道:
\begin{quote}
「誠惶誠恐,稽首歸依。臣等興教,仰望清虛。滅僧鄙俚,敬道光輝。敕修寶殿,御製庭闈。廣陳供養,高掛龍旗。通宵秉燭,鎮日香焚。一誠達上,寸敬虔歸。今蒙降駕,未返仙車。望賜些金丹聖水,進與朝廷,壽比南山。」
\end{quote}

八戒聞言,心中忐忑,默對行者道:「這是我們的不是:吃了東西,且不走路,只等這般禱祝。卻怎麼答應?」行者又捻一把,忽地開口,叫聲:「晚輩小仙,且休拜祝。我等自蟠桃會上來的,不曾帶得金丹聖水,待改日再來垂賜。」那些大小道士聽見說出話來,一個個抖衣而戰道:「爺爺呀!活天尊臨凡,是必莫放,好歹求個長生的法兒。」鹿力大仙上前,又拜云:
\begin{quote}
「揚塵頓首,謹辦丹誠。微臣歸命,俯仰三清。自來此界,興道除僧。國王心喜,敬重玄齡。羅天大醮,徹夜看經。幸天尊之不棄,降聖駕而臨庭。俯求垂念,仰望恩榮。是必留些聖水,與弟子們延壽長生。」
\end{quote}

沙僧捻著行者,默默的道:「哥呀,要得緊,又來禱告了。」行者道:「與他些罷。」八戒寂寂道:「那裡有得?」行者道:「你只看著我,我有時,你們也都有了。」那道士吹打已畢,行者開言道:「那晚輩小仙,不須拜伏。我欲不留些聖水與你們,恐滅了苗裔;若要與你,又忒容易了。」眾道聞言,一齊俯伏叩頭道:「萬望天尊念弟子恭敬之意,千乞喜賜些須。我弟子廣宣道德,奏國王普敬玄門。」行者道:「既如此,取器皿來。」那道士一齊頓首謝恩。虎力大仙愛強,就擡一口大缸,放在殿上;鹿力大仙端一砂盆,安在供桌之上;羊力大仙把花瓶摘了花,移在中間。行者道:「你們都出殿前,掩上格子,不可洩了天機,好留與你些聖水。」眾道一齊跪伏丹墀之下,掩了殿門。

那行者立將起來,掀著虎皮裙,撒了一花瓶臊溺。豬八戒見了,歡喜道:「哥啊,我和你做這幾年兄弟,只這些兒不曾弄過。我才吃了些東西,倒要幹這個事兒哩。」那獃子揭衣服,忽喇喇,就似呂梁洪倒下板來,沙沙的溺了一砂盆。沙和尚卻也撒了半缸。依舊整衣端坐在上,道:「小仙領聖水。」

那些道士推開格子,磕頭禮拜謝恩,擡出缸去,將那瓶、盆總歸一處,教:「徒弟,取個鍾子來嘗嘗。」小道士即便拿了一個茶鍾,遞與老道士。道士舀出一鍾來,喝下口去,只情抹唇咂嘴。鹿力大仙道:「師兄,好吃麼?」老道士努著嘴道:「不甚好吃,有些酣之味。」羊力大仙道:「等我嘗嘗。」也喝了一口,道:「有些豬溺臊氣。」行者坐在上面,聽見說出這話兒來,已此識破了,道:「我弄個手段,索性留個名罷。」大叫云:
\begin{quote}
「道號道號,你好胡思。那個三清,肯降凡基?吾將真姓,說與你知。大唐僧眾,奉旨來西。良宵無事,下降宮闈。吃了供養,閑坐嬉嬉。蒙你叩拜,何以答之?那裡是甚麼聖水,你們吃的都是我一溺之尿。」
\end{quote}

那道士聞得此言,攔住門,一齊動叉鈀、掃帚、瓦塊、石頭,沒頭沒臉,往裡面亂打。好行者,左手挾了沙僧,右手挾了八戒,闖出門,駕著雲光,徑轉智淵寺方丈。不敢驚動師父,三人又復睡下。

早是五鼓三點。那國王設朝,聚集兩班文武、四百朝官,但見絳紗燈火光明,寶鼎香雲靉靆。

此時唐三藏醒來,叫:「徒弟,徒弟,伏侍我倒換關文去來。」行者與沙僧、八戒急起身,穿了衣服,侍立左右道:「上告師父。這國君信著那些道士,興道滅僧,恐言語差錯,不肯倒換關文,我等護持師父,都進朝去也。」唐僧大喜,披了錦襴袈裟。行者帶了通關文牒,教悟淨捧著缽盂,悟能拿了錫杖;將行囊、馬匹,交與智淵寺僧看守。徑到五鳳樓前,對黃門官作禮,報了姓名,言是東土大唐取經的和尚來此倒換關文,煩為轉奏。那閣門大使進朝俯伏金階,奏曰:「外面有四個和尚,說是東土大唐取經的,欲來倒換關文,現在五鳳樓前候旨。」國王聞奏道:「這和尚沒處尋死,卻來這裡尋死。那巡捕官員,怎麼不拿他解來?」傍邊閃過當駕的太師啟奏道:「東土大唐,乃南贍部洲,號曰中華大國。到此有萬里之遙,路多妖怪。這和尚一定有些法力,方敢西來。望陛下看中華之遠僧,且召來驗牒放行,庶不失善緣之意。」國王准奏,把唐僧等宣至金鑾殿下。師徒們排列階前,捧關文遞與國王。

國王展開方看,又見黃門官來奏:「三位國師來也。」慌得國王收了關文,急下龍座,著近侍的設了繡墩,躬身迎接。三藏等回頭觀看,見那大仙搖搖擺擺,後帶著一雙丫髻蓬頭的小童兒,往裡直進。兩班官控背躬身,不敢仰視。他上了金鑾殿,對國王徑不行禮。那國王道:「國師,朕未曾奉請,今日如何肯降?」老道士云:「有一事奉告,故來也。那四個和尚是那國來的?」國王道:「是東土大唐差去西天取經的,來此倒換關文。」那三道士鼓掌大笑道:「我說他走了,原來還在這裡。」國王驚道:「國師有何話說?他才來報了姓名,正欲拿送國師使用,怎奈當駕太師所奏有理,朕因看遠來之意,不滅中華善緣,方才召入驗牒,不期國師有此問。想是他冒犯尊顏,有得罪處也?」道士笑云:「陛下不知。他昨日來的,在東門外打殺了我兩個徒弟,放了五百個囚僧,捽碎車輛;夜間闖進觀來,把三清聖像毀壞,偷吃了御賜供養。我等被他蒙蔽了,只道是天尊下降,求些聖水金丹,進與陛下,指望延壽長生;不期他遺些小便,哄瞞我等。我等各喝了一口,嘗出滋味,正欲下手擒拿,他卻走了。今日還在此間,正所謂『冤家路兒窄』也。」那國王聞言發怒,欲誅四眾。

孫大聖合掌開言,厲聲高叫道:「陛下暫息雷霆之怒,容僧等啟奏。」國王道:「你衝撞了國師,國師之言,豈有差謬?」行者道:「他說我昨日到城外打殺他兩個徒弟,是誰知證?我等且屈認了,著兩個和尚償命,還放兩個去取經。他又說我捽碎車輛,放了囚僧,此事亦無見證,料不該死,再著一個和尚領罪罷了。他說我毀了三清,鬧了觀宇,這又是栽害我也。」國王道:「怎見栽害?」行者道:「我僧乃東土之人,乍來此處,街道尚且不通,如何夜裡就知他觀中之事?既遺下小便,就該當時捉住,卻這早晚坐名害人。天下假名託姓的無限,怎麼就說是我?望陛下回嗔詳察。」那國王本來昏亂,被行者說了一遍,他就決斷不定。

正疑惑之間,又見黃門官來奏:「陛下,門外有許多鄉老聽宣。」國王道:「有何事幹?」即命宣來。宣至殿前,有三四十名鄉老,朝上磕頭道:「萬歲,今年一春無雨,但恐夏月乾荒,特來啟奏,請那位國師爺爺祈一場甘雨,普濟黎民。」國王道:「鄉老且退,就有雨來也。」鄉老謝恩而出。國王道:「唐朝僧眾,朕敬道滅僧為何?只為當年求雨,我朝僧人更未嘗求得一點;幸天降國師,拯援塗炭。你今遠來,冒犯國師,本當即時問罪,姑且恕你,敢與我國師賭勝求雨麼?若祈得一場甘雨,濟度萬民,朕即饒你罪名,倒換關文,放你西去,若賭不過,無雨,就將汝等推赴殺場,典刑示眾。」行者笑道:「小和尚也曉得些兒求禱。」

國王見說,即命打掃壇場。一壁廂教:「擺駕,寡人親上五鳳樓觀看。」當時多官擺駕,須臾,上樓坐了。唐三藏隨著行者、沙僧、八戒,侍立樓下。那三道士陪國王坐在樓上。少時間,一員官飛馬來報:「壇場諸色皆備,請國師爺爺登壇。」

那虎力大仙欠身拱手,辭了國王,徑下樓來。行者向前攔住道:「先生那裡去?」大仙道:「登壇祈雨。」行者道:「你也忒自重了,更不讓我遠鄉之僧。也罷,這正是『強龍不壓地頭蛇』。先生先去,必須對君前講開。」大仙道:「講甚麼?」行者道:「我與你都上壇祈雨,知雨是你的,是我的?不見是誰的功績了。」國王在上聽見,心中暗喜道:「那小和尚說話,倒有些筋節。」沙僧聽見,暗笑道:「不知他一肚子筋節,還不曾拿出來哩。」大仙道:「不消講,陛下自然知之。」行者道:「雖然知之,奈我遠來之僧,未曾與你相會。那時彼此混賴,不成勾當,須講開方好行事。」大仙道:「這一上壇,只看我的令牌為號:一聲令牌響,風來;二聲響,雲起;三聲響,雷閃齊鳴;四聲響,雨至;五聲響,雲散雨收。」行者笑道:「妙啊!我僧是不曾見。請了,請了。」

大仙拽開步進前,三藏等隨後,徑到了壇門外。擡頭觀看,那裡有一座高臺,約有三丈多高。臺左右插著二十八宿旗號。頂上放一張桌子,桌上有一個香爐,爐中香煙靄靄。兩邊有兩隻燭臺,臺上風燭煌煌。爐邊靠著一個金牌,牌上鐫的是雷神名號。底下有五個大缸,都注著滿缸清水,水上浮著楊柳枝,楊柳枝上托著一面鐵牌,牌上書的是雷霆都司的符字。左右有五個大樁,樁上寫著五方蠻雷使者的名錄。每一樁邊立兩個道士,各執鐵鎚,伺候著打樁。臺後面有許多道士,在那裡寫作文書。正中間設一架紙爐,又有幾個像生的人物,都是那執符使者、土地贊教之神。

那大仙走進去,更不謙遜,直上高臺立定。傍邊有個小道士捧了幾張黃紙書就的符字、一口寶劍,遞與大仙。大仙執著寶劍,念聲咒語,將一道符在燭上燒了。那底下兩三個道士拿過一個執符的像生、一道文書,亦點火焚之。那上面乒的一聲令牌響,只見那半空裡悠悠的風色飄來。豬八戒口裡作念道:「不好了,不好了,這道士果然有本事。令牌響了一下,果然就刮風。」行者道:「兄弟悄悄的,你們再莫與我說話,只管護持師父,等我幹事去來。」

好大聖,拔下一根毫毛,吹口仙氣,叫:「變!」就變作一個假行者,立在唐僧手下。他的真身出了元神,趕到半空中,高叫:「那司風的是那個?」慌得那風婆婆捻住布袋,巽二郎劄住口繩,上前施禮。行者道:「我保護唐朝聖僧西天取經,路過車遲國,與那妖道賭勝祈雨,你怎麼不助老孫,反助那道士?我且饒你,把風收了;若有一些風兒,把那道士的鬍子吹得動動,各打二十鐵棒。」風婆婆道:「不敢,不敢。」遂而沒些風氣。八戒忍不住亂嚷道:「那先生請退,令牌已響,怎麼不見一些風兒?你下來,讓我們上去。」

那道士又執令牌,燒了符檄,撲的又打了一下,只見那空中雲霧遮滿。孫大聖又當頭叫道:「佈雲的是那個?」慌得那推雲童子、佈霧郎君當面施禮。行者又將前事說了一遍。那雲童、霧子也收了雲霧,放出太陽星耀耀,一天萬里更無雲。八戒笑道:「這先兒只好哄這皇帝,搪塞黎民,全沒些真實本事。令牌響了兩個,如何又不見雲生?」

那道士心中焦躁,仗寶劍,解散了頭髮,念著咒,燒了符,再一令牌打將下去。只見那南天門裡,鄧天君領著雷公、電母到當空,迎著行者進禮。行者又將前項事說了一遍,道:「你們怎麼來的志誠?是何法旨?」天君道:「那道士五雷法是個真的,他發了文書,燒了文檄,驚動玉帝,玉帝擲下旨意,徑至九天應元雷聲普化天尊府下。我等奉旨前來,助雷電下雨。」行者道:「既如此,且都住了,同候老孫行事。」果然雷也不鳴,電也不灼。

那道士愈加著忙,又添香、燒符、念咒、打下令牌。半空中,又有四海龍王一齊擁至。行者當頭喝道:「敖廣,那裡去?」那敖廣、敖順、敖欽、敖閏上前施禮。行者又將前項事說了一遍,道:「向日有勞,未曾成功;今日之事,望為助力。」龍王道:「遵命,遵命。」行者又謝了敖順道:「前日虧令郎縛怪,搭救師父。」龍王道:「那廝還鎖在海中,未敢擅便,正欲請大聖發落。」行者道:「憑你怎麼處治了罷。如今且助我一功。那道士四聲令牌已畢,卻輪到老孫上去幹事了。但我不會發符、燒檄,打甚令牌,你列位卻要助我行行。」鄧天君道:「大聖吩咐,誰敢不從?但只是得一個號令,方敢依令而行;不然,雷雨亂了,顯得大聖無款也。」行者道:「我將棍子為號罷。」那雷公大驚道:「爺爺呀!我們怎吃得這棍子?」行者道:「不是打你們,但看我這棍子往上一指,就要刮風。」那風婆婆、巽二郎沒口的答應道:「就放風。」「棍子第二指,就要佈雲。」那推雲童子、佈霧郎君道:「就佈雲,就佈雲。」「棍子第三指,就要雷鳴電灼。」那雷公、電母道:「奉承,奉承。」「棍子第四指,就要下雨。」那龍王道:「遵命,遵命。」「棍子第五指,就要大日晴天。卻莫違誤。」

吩咐已畢,遂按下雲頭,把毫毛一抖,收上身來。那些人肉眼凡胎,那裡曉得。行者遂在傍邊高叫道:「先生請了。四聲令牌俱已響畢,更沒有風雲雷雨,該讓我了。」那道士無奈,不敢久占,只得下了臺讓他。努著嘴,徑往樓上見駕。行者道:「等我跟他去,看他說些甚的。」只聽得那國王問道:「寡人這裡洗耳誠聽,你那裡四聲令響,不見風雨,何也?」道士云:「今日龍神都不在家。」行者厲聲道:「陛下,龍神俱在家,只是這國師法不靈,請他不來。等和尚請來你看。」國王道:「即去登壇,寡人還在此候雨。」

行者得旨,急抽身到壇所,扯著唐僧道:「師父請上臺。」唐僧道:「徒弟,我卻不會祈雨。」八戒笑道:「他害你了,若還沒雨,拿上柴蓬,一把火了帳。」行者道:「你不會求雨,好的會念經。等我助你。」那長老才舉步登壇,到上面,端然坐下,定性歸神,默念那《密多心經》。正坐處,忽見一員官飛馬來問:「那和尚,怎麼不打令牌,不燒符檄?」行者高聲答道:「不用!不用!我們是靜功祈禱。」那官去回奏不題。

行者聽得老師父經文念盡,卻去耳朵內取出鐵棒,迎風幌了一幌,就有丈二長短,碗來粗細,將棍望空一指。那風婆婆見了,急忙扯開皮袋;巽二郎解放口繩。只聽得呼呼風響,滿城中揭瓦翻磚,揚砂走石。看起來,真個好風,卻比那尋常之風不同也。但見:
\begin{quote}
折柳傷花,摧林倒樹。九重殿損壁崩牆,五鳳樓搖梁撼柱。天邊紅日無光,地下黃砂有翅。演武廳前武將驚,會文閣內文官懼。三宮粉黛亂青絲,六院嬪妃蓬寶髻。侯伯金冠落繡纓,宰相烏紗飄展翅。當駕有言不敢談,黃門執本無由遞。金魚玉帶不依班,象簡羅衫無品敘。彩閣翠屏盡損傷,綠窗朱戶皆狼狽。金鑾殿瓦走磚飛,錦雲堂門歪槅碎。這陣狂風果是兇,刮得那君王父子難相會;六街三市沒人蹤,萬戶千門皆緊閉。
\end{quote}

正是那狂風大作。

孫行者又顯神通,把金箍棒鑽一鑽,望空又一指。只見那:
\begin{quote}
推雲童子,佈霧郎君。推雲童子顯神威,骨都都觸石垂天;佈霧郎君施法力,濃漠漠飛煙蓋地。茫茫三市暗,冉冉六街昏。因風離海上,隨雨出崑崙。頃刻漫天地,須臾蔽世塵。宛然如混沌,不見鳳樓門。
\end{quote}

此時昏霧朦朧,濃雲靉靆。

孫行者又把金箍棒鑽一鑽,望空又一指。慌得那:
\begin{quote}
雷公奮怒,電母生嗔。雷公奮怒,倒騎火獸下天關;電母生嗔,亂掣金蛇離斗府。唿喇喇施霹靂,振碎了鐵叉山;淅瀝瀝閃紅綃,飛出了東洋海。呼呼隱隱滾車聲,燁燁煌煌飄稻米。萬萌萬物精神改,多少昆蟲蟄已開。君臣樓上心驚駭,商賈聞聲膽怯忙。
\end{quote}

那沉雷護閃,乒乒乓乓,一似那地裂山崩之勢。諕得那滿城人,戶戶焚香,家家化紙。孫行者高呼:「老鄧,仔細替我看那貪贓壞法之官、忤逆不孝之子,多打死幾個示眾。」那雷越發振響起來。

行者卻又把鐵棒望上一指。只見那:
\begin{quote}
龍施號令,雨漫乾坤。勢如銀漢傾天塹,疾似雲流過海門。樓頭聲滴滴,窗外響瀟瀟。天上銀河瀉,街前白浪滔。淙淙如瓮撿,滾滾似盆澆。孤莊將漫屋,野岸欲平橋。真個桑田變滄海,霎時陸岸滾波濤。神龍藉此來相助,擡起長江望下澆。
\end{quote}

這場雨自辰時下起,只下到午時前後;下得那車遲城裡裡外外,水漫了街衢。

那國王傳旨道:「雨夠了,雨夠了;十分再多,又渰壞了禾苗,反為不美。」五鳳樓下聽事官策馬冒雨來報:「聖僧,雨夠了。」行者聞言,將金箍棒往上又一指。只見霎時間,雷收風息,雨散雲收。國王滿心歡喜,文武盡皆稱贊道:「好和尚!這正是『強中更有強中手』。就是我國師求雨雖靈,若要晴,細雨兒還下半日,便不清爽。怎麼這和尚要晴就晴,頃刻間杲杲日出,萬里就無雲也?」

國王教回鑾,倒換關文,打發唐僧過去。正用御寶時,又被那三個道士上前阻住道:「陛下,這場雨全非和尚之功,還是我道門之力。」國王道:「你才說龍王不在家,不曾有雨;他走上去,以靜功祈禱,就雨下來,怎麼又與他爭功,何也?」虎力大仙道:「我上壇發了文書,燒了符檄,擊了令牌,那龍王誰敢不來?想是別方召請,風、雲、雷、雨五司俱不在,一聞我令,隨趕而來,適遇著我下他上,一時撞著這個機會,所以就雨。從根算來,還是我請的龍,下的雨,怎麼算作他的功果?」那國王昏亂,聽此言,卻又疑惑未定。

行者近前一步,合掌奏道:「陛下,這些傍門法術,也不成個功果,算不得我的他的。如今有四海龍王現在空中,我僧未曾發放,他還不敢遽退。那國師若能叫得龍王現身,就算他的功勞。」國王大喜道:「寡人做了二十三年皇帝,更不曾看見活龍是怎麼模樣。你兩家各顯法力,不論僧道,但叫得來的,就是有功;叫不出的,有罪。」那道士怎麼有那樣本事?就叫,那龍王見大聖在此,也不敢出頭。道士云:「我輩不能,你是叫來。」

那大聖仰面朝空,厲聲高叫:「敖廣何在?弟兄們都現原身來看。」那龍王聽喚,即忙現了本身,四條龍。在半空中度霧穿雲,飛舞向金鑾殿上。但見:
\begin{quote}
飛騰變化,遶霧盤雲。玉爪垂鉤白,銀鱗舞鏡明。髯飄素練根根爽,角聳軒昂挺挺清。磕額崔巍,圓睛晃亮。隱顯莫能測,飛揚不可評。禱雨隨時佈雨,求晴即便天晴。這才是有靈有聖真龍像,祥瑞繽紛遶殿庭。
\end{quote}

那國王在殿上焚香,眾公卿在階前禮拜。國王道:「有勞貴體降臨,請回。寡人改日醮謝。」行者道:「列位眾神各自歸去,這國王改日醮謝。」那龍王徑自歸海,眾神各各回天。這正是:
\begin{quote}
廣大無邊真妙法,至真了性劈傍門。
\end{quote}

畢竟不知怎麼除邪,且聽下回分解。
