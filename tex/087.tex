
\chapter{鳳仙郡冒天止雨 孫大聖勸善施霖}

\begin{quote}
大道幽深,如何消息,說破鬼神驚駭。挾藏宇宙,剖判玄光,真樂世間無賽。靈鷲峰前,寶珠拈出,明映五般光彩。照乾坤上下群生,知者壽同山海。
\end{quote}

卻說三藏師徒四眾別樵子,下了隱霧山,奔上大路。行經數日,忽見一座城池相近。三藏道:「悟空,你看那前面城池,可是天竺國麼?」行者搖手道:「不是,不是。如來處雖稱極樂,卻沒有城池,乃是一座大山,山中有樓臺殿閣,喚做靈山大雷音寺。就到了天竺國,也不是如來住處,天竺國還不知離靈山有多少路哩。那城想是天竺之外郡,到邊前方知明白。」

不一時至城外,三藏下馬,入到三層門裡,見那民事荒涼,街衢冷落。又到市口之間,見許多穿青衣者左右擺列,有幾個冠帶者立於房簷之下。他四眾順街行走,那些人更不遜避。豬八戒村愚,把長嘴掬一掬,叫道:「讓路,讓路。」那些人猛擡頭,看見模樣,一個個骨軟筋麻,跌跌蹡蹡,都道:「妖精來了,妖精來了!」諕得那簷下冠帶者戰兢兢躬身問道:「那方來者?」三藏恐他們闖禍,一力當先,對眾道:「貧僧乃東土大唐駕下拜天竺國大雷音寺佛祖求經者。路過寶方,一則不知地名,二則未落人家,才進城甚失迴避,望列公恕罪。」那官人卻才施禮道:「此處乃天竺外郡,地名鳳仙郡。連年乾旱,郡侯差我等在此出榜,招求法師祈雨救民也。」行者聞言道:「你的榜文何在?」眾官道:「榜文在此,適間才打掃廊簷,還未張掛。」行者道:「拿來我看看。」眾官即將榜文展開,掛在簷下。行者四眾上前同看。榜上寫著:
\begin{quote}
大天竺國鳳仙郡郡侯上官,為榜聘明師,招求大法事:茲因郡土寬弘,軍民殷實,連年亢旱,累歲乾荒,民田菑而軍地薄,河道淺而溝澮空。井中無水,泉底無津。富室聊以全生,窮民難以活命。斗粟百金之價,束薪五兩之資。十歲女易米三升,五歲男隨人帶去。城中懼法,典衣當物以存身;鄉下欺公,打劫吃人而顧命。為此出給榜文,仰望十方賢哲,禱雨救民。恩當重報,願以千金奉謝,決不虛言。須至榜者。
\end{quote}

行者看罷,對眾官道:「郡侯上官何也?」眾官道:「上官乃是姓,此我郡侯之姓也。」行者笑道:「此姓卻少。」八戒道:「哥哥不曾讀書。《百家姓》後有一句『上官歐陽』。」三藏道:「徒弟們,且休閑講。那個會求雨,與他求一場甘雨,以濟民瘼,此乃萬善之事;如不會,就行,莫誤了走路。」行者道:「祈雨有甚難事?我老孫翻江攪海、換斗移星,踢天弄井、吐霧噴雲,擔山趕月、喚雨呼風:那一件兒不是幼年耍子的勾當?何為稀罕?」

眾官聽說,著兩個急去郡中報道:「老爺,萬千之喜至也。」那郡侯正焚香默祝,聽得報聲喜至,即問:「何喜?」那官道:「今日領榜,方至市口張掛,即有四個和尚,稱是東土大唐差往天竺國大雷音拜佛求經者,見榜即道能祈甘雨,特來報知。」

那郡侯即整衣步行,不用轎馬多人,徑至市口,以禮敦請。忽有人報道:「郡侯老爺來了。」眾人閃過。那郡侯一見唐僧,不怕他徒弟醜惡,當街心倒身下拜道:「下官乃鳳仙郡郡侯上官氏,熏沐拜請老師祈雨救民。望師大捨慈悲,運神功,拔濟拔濟。」三藏答禮道:「此間不是講話處,待貧僧到那寺觀,卻好行事。」郡侯道:「老師同到小衙,自有潔淨之處。」

師徒們遂牽馬挑擔,徑至府中,一一相見。郡侯即命看茶擺齋,少頃齋至。那八戒放量吞餐,如同餓虎。諕得那些捧盤的心驚膽戰,一往一來,添湯添飯,就如走馬燈兒一般,剛剛供上,直吃得飽滿方休。齋畢,唐僧謝了齋,卻問:「郡侯大人,貴處乾旱幾時了?」郡侯道:
\begin{quote}
「敝地大邦天竺國,風仙外郡吾司牧。
一連三載遇乾荒,草子不生絕五穀。
大小人家買賣難,十門九戶俱啼哭。
三停餓死二停人,一停還似風中燭。
下官出榜遍求賢,幸遇真僧來我國。
若施寸雨濟黎民,願奉千金酬厚德!」
\end{quote}

行者聽說,滿面喜生,呵呵的笑道:「莫說,莫說。若說千金為謝,半點甘雨全無;但論積功累德,老孫送你一場大雨。」那郡侯原來十分清正賢良,愛民心重,即請行者上坐,低頭下拜道:「老師果捨慈悲,下官必不敢悖德。」行者道:「且莫講話,請起。但煩你好生看著我師父,等老孫行事。」沙僧道:「哥哥,怎麼行事?」行者道:「你和八戒過來,就在他這堂下隨著我做個羽翼,等老孫喚龍來行雨。」八戒、沙僧謹依使令,三個人都在堂下。郡侯焚香禮拜。三藏坐著念經。

行者念動真言,誦動咒語,即時見正東上一朵烏雲,漸漸落至堂前,乃是東海老龍王敖廣。那敖廣收了雲腳,化作人形,走向前,對行者躬身施禮道:「大聖喚小龍來,那方使用?」行者道:「請起。累你遠來,別無甚事。此間乃鳳仙郡,連年乾旱,問你如何不來下雨?」老龍道:「啟上大聖得知:我雖能行雨,乃上天遣用之輩。上天不差,豈敢擅自來此行雨?」行者道:「我因路過此方,見久旱民苦,特著你來此施雨求濟,如何推託?」龍王道:「豈敢推託?但大聖念真言呼喚,不敢不來。一則未奉上天御旨,二則未曾帶得行雨神將,怎麼動得雨部?大聖既有拔濟之心,容小龍回海點兵,煩大聖到天宮奏准,請一道降雨的聖旨,請水官放出龍來,我卻好照旨意數目下雨。」

行者見他說出理來,只得發放老龍回海。他即跳出罡斗,對唐僧備言龍王之事。唐僧道:「既然如此,你去為之,切莫打誑語。」行者即吩咐八戒、沙僧:「保著師父,我上天宮去也。」好大聖,說聲去,寂然不見。那郡侯膽戰心驚道:「孫老爺那裡去了?」八戒笑道:「駕雲上天去了。」郡侯十分恭敬,傳出飛報,教滿城大街小巷,不拘公卿士庶、軍民人等,家家供養龍王牌位,門設清水缸,缸插楊柳枝,侍奉香火,拜天不題。

卻說行者一駕觔斗雲,徑到西天門外,早見護國天王引天丁、力士上前迎接道:「大聖,取經之事完乎?」行者道:「也差不遠矣。今行至天竺國界,有一外郡,名鳳仙郡。彼處三年不雨,民甚艱苦,老孫欲祈雨拯救。呼得龍王到彼,他言無旨,不敢私自為之,特來朝見玉帝請旨。」天王道:「那壁廂敢是不該下雨哩。我向時聞得說:那郡侯撒潑,冒犯天地,上帝見罪,立有米山、麵山、黃金大鎖,直等此三事倒斷,才該下雨。」行者不知此意是何,要見玉帝。天王不敢攔阻,讓他進去。徑至通明殿外,又見四大天師迎道:「大聖到此何幹?」行者道:「因保唐僧,路至天竺國界,鳳仙郡無雨,郡侯召師祈雨。老孫呼得龍王,意命降雨,他說未奉玉帝旨意,不敢擅行,特來求旨,以甦民困。」四大天師道:「那方不該下雨。」行者笑道:「該與不該,煩為引奏引奏,看老孫的人情何如。」葛仙翁道:「俗語云:『蒼蠅包網兒——好大面皮。』」許旌陽道:「不要亂談,且只帶他進去。」

丘洪濟、張道陵與葛、許四真人引至靈霄殿下,啟奏道:「萬歲,有孫悟空路至天竺國鳳仙郡,欲與求雨,特來請旨。」玉帝道:「那廝三年前十二月二十五日,朕出行監觀萬天,浮遊三界。駕至他方,見那上官正不仁,將齋天素供推倒喂狗,口出穢言,造有冒犯之罪,朕即立以三事,在於披香殿內。汝等引孫悟空去看:若三事倒斷,即降旨與他;如不倒斷,且休管閑事。」

四天師即引行者至披香殿裡看時,見有一座米山,約有十丈高下;一座麵山,約有二十丈高下。米山邊有一隻拳大之雞,在那裡緊一嘴,慢一嘴,嗛那米吃。麵山邊有一隻金毛哈巴狗兒,在那裡長一舌,短一舌,餂那麵吃。左邊懸一座鐵架子,架上掛一把金鎖,約有一尺三四寸長短,鎖梃有指頭粗細,下面有一盞明燈,燈焰兒燎著那鎖梃。行者不知其意,回頭問天師曰:「此何意也?」天師道:「那廝觸犯了上天,玉帝立此三事,直等雞嗛了米盡,狗餂得麵盡,燈焰燎斷鎖梃,那方才該下雨哩。」

行者聞言,大驚失色,再不敢啟奏,走出殿,滿面含羞。四天師笑道:「大聖不必煩惱,這事只宜作善可解。若有一念善慈,驚動上天,那米、麵山即時就倒,鎖梃即時就斷。你去勸他歸善,福自來矣。」行者依言,不上靈霄辭玉帝,徑來下界復凡夫。須臾,到西天門,又見護國天王。天王道:「請旨如何?」行者將米山、麵山、金鎖之事說了一遍,道:「果依你言,不肯傳旨。適間天師送我,教勸那廝歸善,即福原也。」遂相別,降雲下界。

那郡侯同三藏、八戒、沙僧、大小官員人等接著,都簇簇攢攢來問。行者將郡侯喝了一聲道:「只因你這廝三年前十二月二十五日冒犯了天地,致令黎民有難,如今不肯降雨。」慌得郡侯跪伏在地道:「老師如何得知三年前事?」行者道:「你把那齋天的素供,怎麼推倒喂狗?可實實說來。」那郡侯不敢隱瞞,道:「三年前十二月二十五日,獻供齋天,在於本衙之內,因妻不賢,惡言相鬥,一時怒發無知,推倒供桌,潑了素饌,果是喚狗來吃了。這兩年憶念在心,神思恍惚,無處可以解釋。不知上天見罪,遺害黎民。今遇老師降臨,萬望明示,上界怎麼樣計較?」行者道:「那一日正是玉皇下界之日,見你將齋供喂狗,又口出穢言,玉帝即立三事記汝。」八戒問道:「是甚三事?」行者道:「披香殿立一座米山,約有十丈高下;一座麵山,約有二十丈高下。米山邊有拳大的一隻小雞,在那裡緊一嘴、慢一嘴的,嗛那米吃;麵山邊有一個金毛哈巴狗兒,在那裡長一舌、短一舌的,餂那麵吃。左邊又一座鐵架子,架上掛一把黃金大鎖,鎖梃兒有指頭粗細。下面有一盞明燈,燈焰兒燎著那鎖梃。直等那雞嗛米盡,狗餂麵盡,燈燎斷鎖梃,他這裡方才該下雨哩。」八戒笑道:「不打緊,不打緊。哥哥肯帶我去,變出法身來,一頓把他的米麵都吃了,鎖梃弄斷了,管取下雨。」行者道:「獃子莫胡說。此乃上天所設之計,你怎麼得見?」三藏道:「似這等說,怎生是好?」行者道:「不難,不難。我臨行時,四天師曾對我言,但只作善可解。」那郡侯拜伏在地,哀告道:「但憑老師指教,下官一一皈依也。」行者道:「你若回心向善,趁早念佛看經,我還替你作為;汝若仍前不改,我亦不能解釋,不久天即誅之,性命不能保矣。」

那郡侯磕頭禮拜,誓願皈依。當時召請本處僧道,啟建道場,各各寫發文書,申奏三天。郡侯領眾拈香瞻拜,答天謝地,引罪自責。三藏也與他念經。一壁廂又出飛報,教城裡城外大家小戶,不論男女人等,都要燒香念佛。自此時,一片善聲盈耳。行者卻才歡喜,對八戒、沙僧道:「你兩個好生護持師父,等老孫再與他去去來。」八戒道:「哥哥又往那裡去?」行者道:「這郡侯聽信老孫之言,果然受教,恭敬善慈,誠心念佛。我這去再奏玉帝,求些雨來。」沙僧道:「哥哥既要去,不必遲疑,且耽擱我們行路。必求雨一壇,庶成我們之正果也。」

好大聖,又縱雲頭,直至天門外,還遇著護國天王。天王道:「你今又來做甚?」行者道:「那郡侯已歸善矣。」天王亦喜。正說處,早見直符使者捧定了道家文書、僧家關牒,到天門外傳遞。那符使見了行者,施禮道:「此意乃大聖勸善之功。」行者道:「你將此文牒送去何處?」符使道:「直送至通明殿上,與天師傳遞到玉皇大天尊前。」行者道:「如此,你先行,我當隨後而去。」那符使入天門去了。護國天王道:「大聖,不消見玉帝了。你只往九天應元府下借點雷神,徑自聲雷掣電,還他就有雨下也。」

真個行者依言,入天門裡,不上靈霄殿求請旨意,轉雲步,徑往九天應元府,見那雷門使者、糾錄典者、廉訪典者都來迎著,施禮道:「大聖何來?」行者道:「有事要見天尊。」三使者即為傳奏。天尊隨下九鳳丹霞之扆,整衣出迎。相見禮畢,行者道:「有一事特來奉求。」天尊道:「何事?」行者道:「我因保唐僧至鳳仙郡,見那乾旱之甚,已許他求雨,特來告借貴部官將到彼聲雷。」天尊道:「我知那郡侯冒犯上天,立有三事,不知可該下雨哩?」行者笑道:「我昨日已見玉帝請旨。玉帝著天師引我去披香殿看那三事,乃是米山、麵山、金鎖,只要三事倒斷,方該下雨。我愁難得倒斷,天師教我勸化郡侯等眾作善,以為人有善念,天必從之。庶幾可以回天心,解災難也。今已善念頓生,善聲盈耳。適間直符使者已將改行從善的文牒奏上玉帝去了,老孫因特造尊府,告借雷部官將相助相助。」天尊道:「既如此,差鄧、辛、張、陶,帥領閃電娘子,即隨大聖下降鳳仙郡聲雷。」

那四將同大聖不多時,至於鳳仙境界,即於半空中作起法來。只聽得唿的雷聲,又見那淅瀝瀝的閃電。真個是:
\begin{quote}
電掣紫金蛇,雷轟群蟄鬨。熒煌飛火光,霹靂崩山洞。列缺滿天明,震驚連地縱。紅銷一閃發萌芽,萬里江山都撼動。
\end{quote}

那鳳仙郡城裡城外,大小官員,軍民人等,整三年不曾聽見雷電;今日見有雷聲霍閃,一齊跪下,頭頂著香爐,有的手拈著柳枝,都念:「南無阿彌陀佛!南無阿彌陀佛!」這一聲善念,果然驚動上天。正是那古詩云:
\begin{quote}
人心生一念,天地悉皆知。
善惡若無報,乾坤必有私。
\end{quote}

且不說孫大聖指揮雷將,掣電轟雷於鳳仙郡,人人歸善。卻說那上界直符使者將僧、道兩家的文牒送至通明殿,四天師傳奏靈霄殿。玉帝見了道:「那廝們既有善念,看三事如何?」正說處,忽有披香殿看管的將官報道:「所立米、麵山俱倒了,霎時間米、麵皆無,鎖梃亦斷。」奏未畢,又有當駕天官引鳳仙郡土地、城隍、社令等神齊來拜奏道:「本郡郡主並滿城大小黎庶之家,無一家一人不皈依善果,禮佛敬天。今啟垂慈,普降甘雨,求濟黎民。」玉帝聞言大喜,即傳旨:「著風部、雲部、雨部各遵號令,去下方,按鳳仙郡界,即於今日今時,聲雷佈雲,降雨三尺零四十二點。」時有四大天師奉旨,傳與各部隨時下界,各逞神威,一齊振作。

行者正與鄧、辛、張、陶令閃電娘子在空中調弄,只見眾神都到,合會一天。那其間風雲際會,甘雨滂沱。好雨:
\begin{quote}
漠漠濃雲,濛濛黑霧。雷車轟轟,閃電灼灼。滾滾狂風,淙淙驟雨。所謂一念回天,萬民滿望。全虧大聖施元運,萬里江山處處陰。好雨傾河倒海,蔽野迷空。簷前垂瀑布,窗外響玲瓏。萬戶千門人念佛,六街三市水流洪。東西河道條條滿,南北溪灣處處通。槁苗得潤,枯木回生。田疇麻麥盛,村堡荳糧升。客旅喜通販賣,農夫愛爾耘耕。從今黍稷多條暢,自然稼穡得豐登。風調雨順民安樂,海晏河清享太平。
\end{quote}

一日雨下足了三尺零四十二點,眾神祇漸漸收回。孫大聖厲聲高叫道:「那四部眾神,且暫停雲從,待老孫去叫郡侯拜謝列位。列位可撥開雲霧,各現真身,與這凡夫親眼看看,他才信心供奉也。」眾神聽說,只得都停在空中。

這行者按落雲頭,徑至郡裡,早見三藏、八戒、沙僧都來迎接;那郡侯一步一拜來謝。行者道:「且慢謝我。我已留住四部神祇,你可傳召多人同此拜謝,教他向後好來降雨。」郡侯隨傳飛報,召眾同酬,都一個個拈香朝拜。只見那四部神祇開明雲霧,各現真身。四部者,乃雨部、雷部、雲部、風部。只見那:
\begin{quote}
龍王顯像,雷將舒身;雲童出現,風伯垂真。龍王顯像,銀鬚蒼貌世無雙;雷將舒身,鉤嘴威顏誠莫比。雲童出現,誰如玉面金冠;風伯垂真,曾似燥眉環眼。齊齊顯露青霄上,各各挨排現聖儀。鳳仙郡界人才信,頂禮拈香惡性回。今日仰朝天上將,洗心向善盡皈依。
\end{quote}

眾神祇寧待了一個時辰,人民拜之不已。孫行者又起在雲端,對眾作禮道:「有勞,有勞。請列位各歸本部。老孫還教郡界中人家供養高真,遇時節醮謝。列位從此後,五日一風,十日一雨,還來拯救拯救。」眾神依言,各各轉部不題。

卻說大聖墜落雲頭,與三藏道:「事畢民安,可收拾走路矣。」那郡侯聞言,急忙行禮道:「孫老爺說那裡話。今此一場,乃無量無邊之恩德。下官這裡差人辦備小宴,奉答厚恩。仍買治民間田地,與老爺起建寺院,立老爺生祠,勒碑刻名,四時享祀。雖刻骨鏤心,難報萬一,怎麼就說走路的話?」三藏道:「大人之言雖當,但我等乃西方掛搭行腳之僧,不敢久住,一二日間,定走無疑。」那郡侯那裡肯放,連夜差多人治辦酒席,起蓋祠宇。

次日,大開佳宴,請唐僧高坐,孫大聖與八戒、沙僧列坐。郡侯同本郡大小官員部臣把杯獻饌,細吹細打,款待了一日。這場果是欣然。有詩為證:
\begin{quote}
田疇久旱逢甘雨,河道經商處處通。
深感神僧來郡界,多蒙大聖上天宮。
解除三事從前惡,一念皈依善果弘。
此後願如堯舜世,五風十雨萬年豐。
\end{quote}

一日筵,二日宴;今日酬,明日謝。扳留將有半月,只等寺院生祠完備。一日,郡侯請四眾往觀。唐僧驚訝道:「功程浩大,何成之如此速耶?」郡侯道:「下官催趲人工,晝夜不息,急急命完,特請列位老爺看看。」行者笑道:「果是賢才能幹的好賢侯也。」即時都到新寺,見那殿閣巍峨,山門壯麗,俱稱贊不已。行者請師父留一寺名。三藏道:「有,留名當喚做『甘霖普濟寺』。」郡侯稱道:「甚好,甚好。」用金貼廣招僧眾,侍奉香火。殿左邊立起四眾生祠,每年四時祭祀;又起蓋雷神、龍神等廟,以答神功。看畢,即命趲行。

那一郡人民,知久留不住,各備贐儀,分文不受。因此,合郡官員人等,盛張鼓樂,大展旌幢,送有三十里遠近,猶不忍別,遂掩淚目送,直至望不見方回。這正是:
\begin{quote}
碩德神僧留普濟,齊天大聖廣施恩。
\end{quote}

畢竟不知此去還有幾日方見如來,且聽下回分解。
