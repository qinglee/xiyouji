
\chapter{官封弼馬心何足 名注齊天意未寧}

那太白金星與美猴王同出了洞天深處,一齊駕雲而起。原來悟空觔斗雲比眾不同,十分快疾,把個金星撇在腦後,先至南天門外。正欲收雲前進,被增長天王領著龐、劉、苟、畢、鄧、辛、張、陶一路大力天丁,槍刀劍戟,擋住天門,不肯放進。猴王道:「這個金星老兒乃奸詐之徒,既請老孫,如何教人動刀動槍,阻塞門路?」正嚷間,金星倏到。悟空就覿面發狠道:「你這老兒,怎麼哄我?被你說奉玉帝招安旨意來請,卻怎麼教這些人阻住天門,不放老孫進去?」金星笑道:「大王息怒。你自來未曾到此天堂,卻又無名,眾天丁又與你素不相識,他怎肯放你擅入?等如今見了天尊,授了仙籙,注了官名,向後隨你出入,誰復擋也?」悟空道:「這等說,也罷,我不進去了。」金星又用手扯住道:「你還同我進去。」

將近天門,金星高叫道:「那天門天將、大小吏兵,放開路者。此乃下界仙人,我奉玉帝聖旨,宣他來也。」那增長天王與眾天丁俱才斂兵退避。猴王始信其言,同金星緩步入裡觀看。真個是:
\begin{quote}
初登上界,乍入天堂。金光萬道滾紅霓,瑞氣千條噴紫霧。只見那南天門,碧沉沉,琉璃造就;明幌幌,寶玉粧成。兩邊擺數十員鎮天元帥,一員員頂梁靠柱,持銑擁旄;四下列十數個金甲神人,一個個執戟懸鞭,持刀仗劍。外廂猶可,入內驚人:裡壁廂有幾根大柱,柱上纏繞著金鱗耀日赤鬚龍;又有幾座長橋,橋上盤旋著彩羽凌空丹頂鳳。明霞幌幌映天光,碧霧濛濛遮斗口。這天上有三十三座天宮,乃遣雲宮、毘沙宮、五明宮、太陽宮、花樂宮,一宮宮脊吞金穩獸;又有七十二重寶殿,乃朝會殿、凌虛殿、寶光殿、天王殿、靈官殿,一殿殿柱列玉麒麟。壽星臺上,有千千年不卸的名花;煉藥爐邊,有萬萬載常青的繡草。又至那朝聖樓前,絳紗衣,星辰燦爛;芙蓉冠,金璧輝煌。玉簪珠履,紫綬金章。金鐘撞動,三曹神表進丹墀;天鼓鳴時,萬聖朝王參玉帝。又至那靈霄寶殿,金釘攢玉戶,彩鳳舞朱門。復道迴廊,處處玲瓏剔透;三簷四簇,層層龍鳳翱翔。上面有個紫巍巍,明幌幌,圓丟丟,亮灼灼,大金葫蘆頂。下面有天妃懸掌扇,玉女捧仙巾,惡狠狠掌朝的天將,氣昂昂護駕的仙卿。正中間,琉璃盤內,放許多重重疊疊太乙丹;瑪瑙瓶中,插幾枝彎彎曲曲珊瑚樹。正是天宮異物般般有,世上如他件件無。金闕銀鑾並紫府,琪花瑤草暨瓊葩。朝王玉兔壇邊過,參聖金烏著底飛。猴王有分來天境,不墮人間點污泥。
\end{quote}

太白金星領著美猴王,到於靈霄殿外,不等宣詔,直至御前,朝上禮拜。悟空挺身在傍,且不朝禮,但側耳以聽金星啟奏。金星奏道:「臣領聖旨,已宣妖仙到了。」玉帝垂簾問曰:「那個是妖仙?」悟空卻才躬身答應道:「老孫便是。」仙卿們都大驚失色道:「這個野猴,怎麼不拜伏參見,輒敢這等答應道:『老孫便是。』卻該死了,該死了。」玉帝傳旨道:「那孫悟空乃下界妖仙,初得人身,不知朝禮,且姑恕罪。」眾仙卿叫聲:「謝恩。」猴王卻才朝上唱個大喏。玉帝宣文選武選仙卿,看那處少甚官職,著孫悟空去除授。傍邊轉過武曲星君啟奏道:「天宮裡各宮各殿,各方各處,都不少官,只是御馬監缺個正堂管事。」玉帝傳旨道:「就除他做個弼馬溫罷。」眾臣叫謝恩,他也只朝上唱個大喏。玉帝又差木德星官送他去御馬監到任。

當時猴王歡歡喜喜,與木德星官徑去到任。事畢,木德星官回宮。他在監裡,會聚了監丞、監副、典簿、力士、大小官員人等,查明本監事務,止有天馬千匹。乃是:
\begin{quote}
驊騮騏驥,騄駬纖離;龍媒紫燕,挾翼驌驦;駃騠銀騔,騕褭飛黃;騊駼翻羽,赤兔超光;踰輝彌景,騰霧勝黃;追風絕地,飛奔霄;逸飄赤電,銅爵浮雲;驄瓏虎,絕塵紫鱗;四極大宛,八駿九逸,千里絕群。此等良馬,一個個嘶風逐電精神壯,踏霧登雲氣力長。
\end{quote}

這猴王查看了文簿,點明了馬數。本監中典簿管徵備草料;力士官管刷洗馬匹、扎草、飲水、煮料;監丞、監副輔佐催辦。弼馬晝夜不睡,滋養馬匹。日間舞弄猶可,夜間看管慇懃:但是馬睡的,趕起來吃草;走的,捉將來靠槽。那些天馬見了他,泯耳攢蹄。倒養得肉肥膘滿。不覺的半月有餘。

一朝閑暇,眾監官都安排酒席,一則與他接風,二則與他賀喜。正在歡飲之間,猴王忽停杯問曰:「我這弼馬溫是個甚麼官銜?」眾曰:「官名就是此了。」又問:「此官是個幾品?」眾道:「沒有品從。」猴王道:「沒品,想是大之極也?」眾道:「不大,不大,只喚做未入流。」猴王道:「怎麼叫做『未入流』?」眾道:「末等。這樣官兒最低最小,只可與他看馬。似堂尊到任之後,這等慇懃,喂得馬肥,只落得道聲『好』字;如稍有些尪羸,還要見責;再十分傷損,還要罰贖問罪。」猴王聞此,不覺心頭火起,咬牙大怒道:「這般藐視老孫!老孫在那花果山稱王稱祖,怎麼哄我來替他養馬?養馬者,乃後生小輩下賤之役,豈是待我的?不做他,不做他,我將去也。」忽喇的一聲,把公案推倒,耳中取出寶貝,幌一幌,碗來粗細,一路解數,直打出御馬監,徑至南天門。眾天丁知他受了仙籙,乃是個弼馬溫,不敢阻當,讓他打出天門去了。

須臾,按落雲頭,回至花果山上。只見那四健將與各洞妖王,在那裡操演兵卒。這猴王厲聲高叫道:「小的們,老孫來了。」一群猴都來叩頭,迎接進洞天深處,請猴王高登寶位,一壁廂辦酒接風。都道:「恭喜大王,上界去十數年,想必得意榮歸也?」猴王道:「我才半月有餘,那裡有十數年?」眾猴道:「大王,你在天上不覺時辰。天上一日,就是下界一年哩。請問大王,官居何職?」猴王搖手道:「不好說,不好說,活活的羞殺人。那玉帝不會用人,他見老孫這般模樣,封我做個甚麼『弼馬溫』,原來是與他養馬,未入流品之類。我初到任時不知,只在御馬監中頑耍。及今日問我同寮,始知是這等卑賤。老孫心中大惱,推倒席面,不受官銜,因此走下來了。」眾猴道:「來得好,來得好。大王在這福地洞天之處為王,多少尊重快樂,怎麼肯去與他做馬夫?」教小的們快辦酒來,與大王釋悶。

正飲酒歡會間,有人來報道:「大王,門外有兩個獨角鬼王,要見大王。」猴王道:「教他進來。」那鬼王整衣跑入洞中,倒身下拜。美猴王問他:「你見我何幹?」鬼王道:「久聞大王招賢,無由得見;今見大王授了天籙,得意榮歸,特獻赭黃袍一件,與大王稱慶。肯不棄鄙賤,收納小人,亦得效犬馬之勞。」猴王大喜,將赭黃袍穿起。眾等欣然排班朝拜。即將鬼王封為前部總督先鋒。鬼王謝恩畢,復啟道:「大王在天許久,所授何職?」猴王道:「玉帝輕賢,封我做個甚麼『弼馬溫』。」鬼王聽言,又奏道:「大王有此神通,如何與他養馬?就做個齊天大聖,有何不可?」猴王聞說,歡喜不勝,連道幾個「好!好!好!」教四健將:「就替我快置個旌旗,旗上寫『齊天大聖』四大字,立竿張掛。自此以後,只稱我為齊天大聖,不許再稱大王。亦可傳與各洞妖王,一體知悉。」此不在話下。

卻說那玉帝次日設朝,只見張天師引御馬監監丞、監副在丹墀下拜奏道:「萬歲,新任弼馬溫孫悟空,因嫌官小,昨日反下天宮去了。」正說間,又見南天門外增長天王領眾天丁,亦奏道:「弼馬溫不知何故,走出天門去了。」玉帝聞言,即傳旨:「著兩路神元,各歸本職。朕遣天兵,擒拿此怪。」班部中閃上托塔李天王與哪吒三太子,越班奏上道:「萬歲,微臣不才,請旨降此妖怪。」玉帝大喜,即封托塔天王李靖為降魔大元帥,哪吒三太子為三壇海會大神,即刻興師下界。

李天王與哪吒叩頭謝辭,徑至本宮,點起三軍,帥眾頭目,著巨靈神為先鋒,魚肚將掠後,藥叉將催兵。一霎時出南天門外,徑來到花果山,選平陽處安了營寨,傳令教巨靈神挑戰。巨靈神得令,結束整齊,掄著宣花斧,到了水簾洞外。只見小洞門外許多妖魔,都是些狼蟲虎豹之類,丫丫叉叉,掄槍舞劍,在那裡跳鬥咆哮。這巨靈神喝道:「那業畜!快早去報與弼馬溫知道:吾乃上天大將,奉玉帝旨意,到此收伏;教他早早出來受降,免致汝等皆傷殘也。」那些妖怪奔奔波波,傳報洞中道:「禍事了!禍事了!」猴王問:「有甚禍事?」眾妖道:「門外有一員天將,口稱大聖官銜,道奉玉帝聖旨,來此收伏。教早早出去受降,免傷我等性命。」猴王聽說,教:「取我披掛來。」就戴上紫金冠,貫上黃金甲,登上步雲鞋,手執如意金箍棒,領眾出門,擺開陣勢。這巨靈神睜睛觀看,真好猴王:
\begin{quote}
身穿金甲亮堂堂,頭戴金冠光映映。
手舉金箍棒一根,足踏雲鞋皆相稱。
一雙怪眼似明星,兩耳過肩查又硬。
挺挺身才變化多,聲音響喨如鐘磬。
尖嘴咨牙弼馬溫,心高要做齊天聖。
\end{quote}

巨靈神厲聲高叫道:「那潑猴!你認得我麼?」大聖聽言,急問道:「你是那路毛神?老孫不曾會你,你快報名來。」巨靈神道:「我把你那欺心的猢猻!你是認不得我。我乃高上神霄托塔李天王部下先鋒巨靈天將,今奉玉帝聖旨,到此收降你。你快卸了裝束,歸順天恩,免得這滿山諸畜遭誅;若道半個不字,教你頃刻化為齏粉。」猴王聽說,心中大怒道:「潑毛神!休誇大口,少弄長舌。我本待一棒打死你,恐無人去報信。且留你性命,快早回天,對玉皇說:他甚不用賢。老孫有無窮的本事,為何教我替他養馬?你看我這旌旗上字號,若依此字號陞官,我就不動刀兵,自然的天地清泰;如若不依,時間就打上靈霄寶殿,教他龍床定坐不成。」這巨靈神聞此言,急睜睛迎風觀看,果見門外豎一高竿,竿上有旌旗一面,上寫著「齊天大聖」四大字。巨靈神冷笑三聲道:「這潑猴,這等不知人事,輒敢無狀,你就要做齊天大聖。好好的吃吾一斧。」劈頭就砍將去。那猴王正是會家不忙,將金箍棒應手相迎。這一場好殺:
\begin{quote}
棒名如意,斧號宣花。他兩個乍相逢,不知深淺,斧和棒,左右交加。一個暗藏神妙,一個大口稱誇。使動法,噴雲曖霧;展開手,播土揚沙。天將神通就有道,猴王變化實無涯。棒舉卻如龍戲水,斧來猶似鳳穿花。巨靈名望傳天下,原來本事不如他:大聖輕輕掄鐵棒,著頭一下滿身麻。
\end{quote}

巨靈神抵敵他不住,被猴王劈頭一棒,慌忙將斧架隔,扢扠的一聲,把個斧柄打做兩截,急撤身敗陣逃生。猴王笑道:「膿包,膿包。我已饒了你,你快去報信,快去報信。」

巨靈神回至營門,徑見托塔天王,忙哈哈跪下道:「弼馬溫果是神通廣大,末將戰他不得,敗陣回來請罪。」李天王發怒道:「這廝剉吾銳氣,推出斬之!」傍邊閃出哪吒太子拜告:「父王息怒,且恕巨靈之罪。待孩兒出師一遭,便知深淺。」天王聽諫,且教回營待罪管事。

這哪吒太子甲冑齊整,跳出營盤,撞至水簾洞外。那悟空正來收兵,見哪吒來的勇猛。好太子:
\begin{quote}
總角才遮顖,披毛未蓋肩。神奇多敏悟,骨秀更清妍。誠為天上麒麟子,果是煙霞彩鳳仙。龍種自然非俗相,妙齡端不類塵凡。身帶六般神器械,飛騰變化廣無邊。今受玉皇金口詔,敕封海會號三壇。
\end{quote}

悟空迎近前來問曰:「你是誰家小哥?闖近吾門,有何事幹?」哪吒喝道:「潑妖猴!豈不認得我?我乃托塔天王三太子哪吒是也,今奉玉帝欽差,至此捉你。」悟空笑道:「小太子,你的嬭牙尚未退,胎毛尚未乾,怎敢說這般大話?我且留你的性命,不打你。你只看我旌旗上是甚麼字號,拜上玉帝:是這般官銜,再也不須動眾,我自皈依;若是不遂我心,定要打上靈霄寶殿。」哪吒擡頭看處,乃「齊天大聖」四字。哪吒道:「這妖猴能有多大神通,就敢稱此名號?不要怕,吃吾一劍。」悟空道:「我只站下不動,任你砍幾劍罷。」那哪吒奮怒,大喝一聲,叫:「變!」即變做三頭六臂,惡狠狠,手持著六般兵器,乃是斬妖劍、砍妖刀、縛妖索、降妖杵、繡毬兒、火輪兒,丫丫叉叉,撲面來打。悟空見了,心驚道:「這小哥倒也會弄些手段。莫無禮,看我神通。」好大聖,喝聲:「變!」也變做三頭六臂;把金箍棒幌一幌,也變作三條。六隻手拿著三條棒架住。這場鬥,真個是地動山搖,好殺也:
\begin{quote}
六臂哪吒太子,天生美石猴王,相逢真對手,正遇本源流。那一個蒙差來下界,這一個欺心鬧斗牛。斬妖寶劍鋒芒快,砍妖刀狠鬼神愁;縛妖索子如飛蟒,降妖大杵似狼頭;火輪掣電烘烘艷,往往來來滾繡毬。大聖三條如意棒,前遮後擋運機謀。苦爭數合無高下,太子心中不肯休。把那六件兵器多教變,百千萬億照頭丟。猴王不懼呵呵笑,鐵棒翻騰自運籌。以一化千千化萬,滿空亂舞賽飛虯。諕得各洞妖王都閉戶,遍山鬼怪盡藏頭。神兵怒氣雲慘慘,金箍鐵棒響颼颼。那壁廂,天丁吶喊人人怕;這壁廂,猴怪搖旗個個憂。發狠兩家齊鬥勇,不知那個剛強那個柔。
\end{quote}

三太子與悟空各騁神威,鬥了個三十回合。那太子六般兵,變做千千萬萬;孫悟空金箍棒,變作萬萬千千。半空中似雨點流星,不分勝負。

原來悟空手疾眼快,正在那混亂之時,他拔下一根毫毛,叫聲:「變!」就變做他的本相,手挺著棒,演著哪吒;他的真身,卻一縱,趕至哪吒腦後,著左膊上一棒打來。哪吒正使法間,聽得棒頭風響,急躲閃時,不能措手,被他著了一下,負痛逃走。收了法,把六件兵器依舊歸身,敗陣而回。

那陣上李天王早已看見,急欲提兵助戰,不覺太子倏至面前,戰兢兢報道:「父王,弼馬溫真個有本事,孩兒這般法力,也戰他不過,已被他打傷膊也。」天王大驚失色道:「這廝恁的神通,如何取勝?」太子道:「他洞門外豎一竿旗,上寫『齊天大聖』四字。親口誇稱,教玉帝就封他做齊天大聖,萬事俱休;若還不是此號,定要打上靈霄寶殿哩。」天王道:「既然如此,且不要與他相持,且去上界,將此言回奏,再多遣天兵,圍捉這廝,未為遲也。」太子負痛,不能復戰,故同天王回天啟奏不題。

你看那猴王得勝歸山,那七十二洞妖王與那六弟兄,俱來賀喜,在洞天福地,飲樂無比。他卻對六弟兄說:「小弟既稱齊天大聖,你們亦可以大聖稱之。」內有牛魔王忽然高叫道:「賢弟言之有理,我即稱做平天大聖。」蛟魔王道:「我稱做覆海大聖。」鵬魔王道:「我稱混天大聖。」獅狔王道:「我稱移山大聖。」獼猴王道:「我稱通風大聖。」狨王道:「我稱驅神大聖。」此時七大聖自作自為,自稱自號,耍樂一日,各散訖。

卻說那李天王與三太子領著眾將,直至靈霄寶殿,啟奏道:「臣等奉聖旨出師下界,收伏妖仙孫悟空,不期他神通廣大,不能取勝,仍望萬歲添兵剿除。」玉帝道:「諒一妖猴,有多少本事,還要添兵?」太子又近前奏道:「望萬歲赦臣死罪。那妖猴使一條鐵棒,先敗了巨靈神,又打傷臣臂膊。洞門外立一竿旗,上書『齊天大聖』四字。道是封他這官職,即便休兵來投;若不是此官,還要打上靈霄寶殿也。」玉帝聞言,驚訝道:「這妖猴何敢這般狂妄?著眾將即刻誅之。」

正說間,班部中又閃出太白金星,奏道:「那妖猴只知出言,不知大小。欲加兵與他爭鬥,想一時不能收伏,反又勞師。不若萬歲大捨恩慈,還降招安旨意,就教他做個齊天大聖。只是加他個空銜,有官無祿便了。」玉帝道:「怎麼喚做『有官無祿』?」金星道:「名是齊天大聖,只不與他事管,不與他俸祿,且養在天壤之間,收他的邪心,使不生狂妄,庶乾坤安靖,海宇得清寧也。」玉帝聞言道:「依卿所奏。」即命降了詔書,仍著金星領去。

金星復出南天門,直至花果山水簾洞外觀看。這番比前不同,威風凜凜,殺氣森森,各樣妖精,無般不有。一個個都執劍拈槍,拿刀弄杖的在那裡咆哮跳躍。一見金星,皆上前動手。金星道:「那眾頭目來,累你去報你大聖知之:吾乃上帝遣來天使,有聖旨在此請他。」眾妖即跑入報道:「外面有一老者,他說是上界天使,有旨意請你。」悟空道:「來得好,來得好。想是前番來的那太白金星。那次請我上界,雖是官爵不堪,卻也天上走了一次,認得那天門內外之路。今番又來,定有好意。」教眾頭目大開旗鼓,擺隊迎接。大聖即帶引群猴,頂冠貫甲,甲上罩了赭黃袍,足踏雲履,急出洞門,躬身施禮,高叫道:「老星請進,恕我失迎之罪。」

金星趨步向前,徑入洞內,面南立著道:「今告大聖:前者因大聖嫌惡官小,躲離御馬監,當有本監中大小官員奏了玉帝。玉帝傳旨道:『凡授官職,皆由卑而尊,為何嫌小?』即有李天王領哪吒下界取戰。不知大聖神通,故遭敗北,回天奏道:『大聖立一竿旗,要做齊天大聖。眾武將還要支吾,是老漢力為大聖冒罪奏聞,免興師旅,請大王授籙。玉帝准奏,因此來請。」悟空笑道:「前番動勞,今又蒙愛,多謝,多謝!但不知上天可與我齊天大聖之官銜也?」金星道:「老漢以此銜奏准,方敢領旨而來;如有不遂,只坐罪老漢便是。」

悟空大喜,懇留飲宴不肯,遂與金星縱著祥雲,到南天門外。那些天丁天將都拱手相迎。徑入靈霄殿下。金星拜奏道:「臣奉詔宣弼馬溫孫悟空已到。」玉帝道:「那孫悟空過來,今宣你做個齊天大聖,官品極矣,但切不可胡為。」這猴亦止朝上唱個喏,道聲「謝恩」。玉帝即命工幹官張、魯二班,在蟠桃園右首,起一座齊天大聖府,府內設個二司:一名安靜司,一名寧神司。司俱有仙吏,左右扶持。又差五斗星君送悟空去到任,外賜御酒二瓶,金花十朵,著他安心定志,再勿胡為。

那猴王信受奉行,即日與五斗星君到府,打開酒瓶,同眾盡飲。送星官回轉本宮,他才遂心滿意,喜地歡天,在於天宮快樂,無掛無礙。正是:
\begin{quote}
仙名永注長生籙,不墮輪迴萬古傳。
\end{quote}

畢竟不知向後如何,且聽下回分解。
