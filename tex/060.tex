
\chapter{牛魔王罷戰赴華筵 孫行者二調芭蕉扇}

土地說:「大力王即牛魔王也。」行者道:「這山本是牛魔王放的火,假名火焰山?」土地道:「不是,不是。大聖若肯赦小神之罪,方敢直言。」行者道:「你有何罪?直說無妨。」土地道:「這火原是大聖放的。」行者怒道:「我在那裡?你這等亂談。我可是放火之輩?」土地道:「是你也認不得我了。此間原無這座山。因大聖五百年前大鬧天宮時,被顯聖擒了,壓赴老君,將大聖安於八卦爐內。煅煉之後開鼎,被你蹬倒丹爐,落了幾個磚來,內有餘火,到此處化為火焰山。我本是兜率宮守爐的道人,當被老君怪我失守,降下此間,就做了火焰山土地也。」豬八戒聞言,恨道:「怪道你這等打扮,原來是道士變的土地。」

行者半信不信道:「你且說,早尋大力王何故?』土地道:「大力王乃羅剎女丈夫。他這向撇了羅剎,現在積雷山摩雲洞。有個萬年狐王,那狐王死了,遺下一個女兒,叫做玉面公主。那公主有百萬家私,無人掌管。二年前,訪著牛魔王神通廣大,情願倒陪家私,招贅為夫。那牛王棄了羅剎,久不回顧。若大聖尋著牛王,拜求來此,方借得真扇:一則搧息火焰,可保師父前進;二來永除火患,可保此地生靈;三者赦我歸天,回繳老君法旨。」行者道:「積雷山坐落何處?到彼有多少程途?」土地道:「在正南方。此間到彼,有三千餘里。」

行者聞言,即吩咐沙僧、八戒保護師父,又教土地陪伴勿回。隨即忽的一聲,渺然不見。那裡消半個時辰,早見一座高山凌漢。按落雲頭,停立巔峰之上觀看,真是好山:
\begin{quote}
高不高,頂摩碧漢;大不大,根扎黃泉。山前日暖,嶺後風寒。山前日暖,有三冬草木無知;嶺後風寒,見九夏冰霜不化。龍潭接澗水長流,虎穴依崖花放早。水流千派似飛瓊,花放一心如布錦。灣環嶺上灣環樹,扢扠石外扢扠松。真個是:高的山,峻的嶺;陡的崖,深的澗;香的花,美的果;紅的藤,紫的竹;青的松,翠的柳。八節四時顏不改,千年萬古色如龍。
\end{quote}

大聖看夠多時,步下尖峰,入深山,找尋路徑。正自沒個消息,忽見松陰下有一女子,手折了一枝香蘭,嬝嬝娜娜而來。大聖閃在怪石之傍,定睛觀看,那女子怎生模樣:
\begin{quote}
嬌嬌傾國色,緩緩步移蓮。貌若王嬙,顏如楚女。如花解語,似玉生香。高髻堆青碧鴉,雙睛蘸綠橫秋水。湘裙半露弓鞋小,翠袖微舒粉腕長。說甚麼暮雨朝雲,真個是朱唇皓齒。錦江滑膩蛾眉秀,賽過文君與薛濤。
\end{quote}

那女子漸漸走近石邊,大聖躬身施禮,緩緩而言曰:「女菩薩何往?」那女子未曾觀看,聽得叫問,卻自擡頭。忽見大聖的相貌醜陋,老大心驚,欲退難退,欲行難行,只得戰兢兢,勉強答道:「你是何方來者?敢在此間問誰?」大聖沉思道:「我若說出取經求扇之事,恐這廝與牛王有親。且只以假親托意,來請魔王之言而答方可。」那女子見他不語,變了顏色,怒聲喝道:「你是何人,敢來問我?」大聖躬身陪笑道:「我是翠雲山來的,初到貴處,不知路徑。敢問菩薩,此間可是積雷山?」那女子道:「正是。」大聖道:「有個摩雲洞,坐落何處?」那女子道:「你尋那洞做甚?」大聖道:「我是翠雲山芭蕉洞鐵扇公主央來請牛魔王的。」

那女子一聽鐵扇公主請牛魔王之言,心中大怒,徹耳根子通紅,潑口罵道:「這賤婢,著實無知。牛王自到我家,未及二載,也不知送了他多少珠翠金銀、綾羅緞疋,年供柴,月供米,自自在在受用,還不識羞,又來請他怎的?」大聖聞言,情知是玉面公主,故意掣出金箍棒,大喝一聲道:「你這潑賤,將家私買住牛王,誠然是陪錢嫁漢,你倒不羞,卻敢罵誰?」那女子見了,諕得魄散魂飛,沒好步,亂金蓮,戰兢兢回頭便走。這大聖吆吆喝喝,隨後相跟。原來穿過松陰,就是摩雲洞口。女子跑進去,撲的把門關了。大聖卻收了金箍棒,停步看時,好所在:
\begin{quote}
樹林森密,崖削崚嶒。薜蘿陰冉冉,蘭蕙味馨馨。流泉漱玉穿修竹,巧石知機帶落英。煙霞籠遠岫,日月照雲屏。龍吟虎嘯,鶴唳鶯鳴。一片清幽真可愛,琪花瑤草景常明。不亞天臺仙洞,勝如海上蓬瀛。
\end{quote}

且不言行者這裡觀看景致。卻說那女子跑得粉汗淋淋,諕得蘭心吸吸,徑入書房裡面。原來牛魔王正在那裡靜玩丹書。這女子沒好氣倒在懷裡,抓耳撓腮,放聲大哭。牛王滿面陪笑道:「美人,休得煩惱。有甚話說?」那女子跳天索地,口中罵道:「潑魔害殺我也!」牛王笑道:「你為甚事罵我?」女子道:「我因父母無依,招你護身養命。江湖中說你是條好漢,你原來是個懼內的庸夫。」牛王聞說,將女子抱住道:「美人,我有那些不是處?你且慢慢說來,我與你陪禮。」女子道:「適才我在洞外閑步花陰,折蘭採蕙,忽有一個毛臉雷公嘴的和尚,猛地前來施禮,把我嚇了個呆掙。及定性問是何人,他說是鐵扇公主央他來請牛魔王的。被我說了兩句,他倒罵了我一場,將一根棍子趕著我打。若不是走得快些,幾乎被他打死。這不是招你為禍?害殺我也。」牛王聞言,卻與他整容陪禮,溫存良久,女子方才息氣。魔王卻發狠道:「美人在上,不敢相瞞。那芭蕉洞雖是僻靜,卻清幽自在。我山妻自幼修持,也是個得道的女仙,卻是家門嚴謹,內無一尺之童,焉得有雷公嘴的男子央來?這想是那裡來的妖怪,或者假綽名聲,至此訪我。等我出去看看。」

好魔王,拽開步,出了書房,上大廳取了披掛,結束了。拿了一條混鐵棍,出門高叫道:「是誰人在我這裡無狀?」行者在傍,見他那模樣,與五百年前又大不同。只見:
\begin{quote}
頭上戴一頂水磨銀亮熟鐵盔,身上貫一副絨穿錦繡黃金甲,足下踏一雙捲尖粉底麂皮靴;腰間束一條攢絲三股獅蠻帶。一雙眼光如明鏡,兩道眉艷似紅霓。口若血盆,齒排銅板。吼聲響震山神怕,行動威風惡鬼慌。四海有名稱混世,西方大力號魔王。
\end{quote}

這大聖整衣上前,深深的唱個大喏道:「長兄,還認得小弟麼?」牛王答禮道:「你是齊天大聖孫悟空麼?」大聖道:「正是,正是。一向久別未拜。適才到此問一女子,方得見兄。丰采果勝常,可賀也。」牛王喝道:「且休巧舌。我聞你鬧了天宮,被佛祖降壓在五行山下,近解脫天災,保護唐僧西天見佛求經,怎麼在號山枯松澗火雲洞把我小兒牛聖嬰害了?正在這裡惱你,你卻怎麼又來尋我?」大聖作禮道:「長兄勿得誤怪小弟。當時令郎捉住吾師,要食其肉,小弟近他不得,幸觀音菩薩欲救我師,勸他歸正。現今做了善財童子,比兄長還高,享極樂之門堂,受逍遙之永壽,有何不可,返怪我耶?」牛王罵道:「這個乖嘴的猢猻!害子之情,被你說過;你才欺我愛妾,打上我門何也?」大聖笑道:「我因拜謁長兄不見,向那女子拜問,不知就是二嫂嫂。因他罵了我幾句,是小弟一時粗鹵,驚了嫂嫂。望長兄寬恕寬恕。」牛王道:「既如此說,我看故舊之情,饒你去罷。」

大聖道:「既蒙寬恩,感謝不盡。但尚有一事奉瀆,萬望周濟周濟。」牛王罵道:「這猢猻不識起倒,饒了你,倒還不走,反來纏我。甚麼周濟周濟?」大聖道:「實不瞞長兄,小弟因保唐僧西進,路阻火焰山,不能前進。詢問土人,知尊嫂羅剎女有一柄芭蕉扇,欲求一用。昨到舊府奉拜嫂嫂,嫂嫂堅執不借。是以特求長兄,望兄長開天地之心,同小弟到大嫂處一行,千萬借扇搧滅火焰,保得唐僧過山,即時完璧。」牛王聞言,心如火發,咬響鋼牙罵道:「你說你不無禮,你原來是借扇之故。一定先欺我山妻,山妻想是不肯,故來尋我,且又趕我愛妾。常言道:『朋友妻,不可欺;朋友妾,不可滅。』你既欺我妻又滅我妾,多大無禮?上來吃我一棍。」大聖道:「哥要說打,弟也不懼。但求寶貝,是我真心,萬乞借我使使。」牛王道:「你若三合敵得我,我著山妻借你;如敵不過,打死你,與我雪恨。」大聖道:「哥說得是。小弟這一向疏懶,不曾與兄相會,不知這幾年武藝比昔日如何,我兄弟們請演演棍看。」這牛王那容分說,掣混鐵棍,劈頭就打;這大聖持金箍棒,隨手相迎。兩個這場好鬥:
\begin{quote}
金箍棒,混鐵棍,變臉不以朋友論。那個說:「正怪你這猢猻害子情。」這個說:「你令郎已得道休嗔恨。」那個說:「你無知怎敢上我門?」這個說:「我有因特地來相問。」一個要求扇子保唐僧,一個不借芭蕉忒鄙吝。語去言來失舊情,舉家無義皆生忿。牛王棍起賽蛟龍,大聖棒迎神鬼遁。初時爭鬥在山前,後來齊駕祥雲進。半空之內顯神通,五彩光中施妙運。兩條棍響振天關,不見輸贏皆傍寸。
\end{quote}

這大聖與那牛王鬥經百十回合,不分勝負。正在難解難分之際,只聽得山峰上有人叫道:「牛爺爺,我大王多多拜上,幸賜早臨,好安座也。」牛王聞說,使混鐵棍支住金箍棒,叫道:「猢猻,你且住了,等我去一個朋友家赴會來者。」言畢,按下雲頭,徑至洞裡,對玉面公主道:「美人,才那雷公嘴的男子乃孫悟空猢猻,被我一頓棍打走了,再不敢來。你放心耍子。我到一個朋友處吃酒去也。」他才卸了盔甲,穿一領鴉青剪絨襖子,走出門,跨上辟水金睛獸,著小的們看守門庭,半雲半霧,一直向西北方而去。

大聖在高峰上看著,心中暗想道:「這老牛不知又結識了甚麼朋友,往那裡去赴會。等老孫跟他走走。」好行者,將身幌一幌,變作一陣清風趕上,隨著同走。不多時,到了一座山中,那牛王寂然不見。大聖聚了原身,入山尋看。那山中有一面清水深潭,潭邊有一座石碣,碣上有六個大字,乃「亂石山碧波潭」。大聖暗想道:「老牛斷然下水去了。水底之精,若不是蛟精,必是龍精、魚精,或是龜鱉黿鼉之精。等老孫也下去水看看。」

好大聖,捻著訣,念個咒語,搖身一變,變作一個螃蟹,不大不小的有三十六斤重。撲的跳在水中,徑沉潭底。忽見一座玲瓏剔透的牌樓,樓下拴著那個辟水金睛獸。進牌樓裡面,卻就沒水。大聖爬進去,仔細看時,只見那壁廂一派音樂之聲。但見:
\begin{quote}
朱宮貝闕,與世不殊。黃金為屋瓦,白玉作門樞。屏開玳瑁甲,檻砌珊瑚珠。祥雲瑞藹輝蓮座,上接三光下入衢。非是天宮並海藏,果然此處賽蓬壺。高堂設宴羅賓主,大小官員冠冕珠。忙呼玉女捧牙槃,催喚仙娥調律呂。長鯨鳴,巨蟹舞,鱉吹笙,鼉擊鼓,驪頷之珠照樽俎。鳥篆之文列翠屏,蝦鬚之簾掛廊廡。八音迭奏雜仙韶,宮商響徹遏雲霄。青頭鱸妓撫瑤瑟,紅眼馬郎品玉簫。鱖婆頂獻香獐脯,龍女頭簪金鳳翹。吃的是,天廚八寶珍饈味;飲的是,紫府瓊漿熟醞醪。
\end{quote}

那上面坐的是牛魔王,左右有三四個蛟精,前面坐著一個老龍精,兩邊乃龍子、龍孫、龍婆、龍女。

正在那裡觥籌交錯之際,孫大聖一直走將上去,被老龍看見,即命:「拿下那個野蟹來。」龍子、龍孫一擁上前,把大聖拿住。大聖忽作人言,叫:「饒命,饒命。」老龍道:「你是那裡來的野蟹?怎麼敢上廳堂,在尊客之前,橫行亂走?快早供來,免汝死罪。」好大聖,假捏虛言,對眾供道:
\begin{quote}
「生自湖中為活,傍崖作窟權居。
蓋因日久得身舒。官受橫行介士。
踏草拖泥落索,從來未習行儀。
不知法度冒王威。伏望尊慈恕罪!」
\end{quote}

座上眾精聞言,都拱身對老龍作禮道:「蟹介士初入瑤宮,不知王禮,望尊公饒他去罷。」老龍稱謝了。眾精即教:「放了那廝,且記打,外面伺候。」大聖應了一聲,往外逃命,徑至牌樓之下。心中暗想道:「這牛王在此貪杯,那裡等得他散?就是散了,也不肯借扇與我。不如偷了他的金睛獸,變做牛魔王,去哄那羅剎女,騙他扇子,送我師父過山為妙。」

好大聖,即現本像,將金睛獸解了韁繩,撲一把,跨上雕鞍,徑直騎出水底。到於潭外,將身變作牛王模樣。打著獸,縱著雲,不多時,已至翠雲山芭蕉洞口。叫聲:「開門。」那洞門裡有兩個女童,聞得聲音開了門,看見是牛魔王嘴臉,即入報:「奶奶,爺爺來家了。」那羅剎聽言,忙整雲鬟,急移蓮步,出門迎接。這大聖下雕鞍,牽進金睛獸;弄大膽,誆騙女佳人。羅剎女肉眼,認他不出,即攜手而入,著丫鬟設座看茶。一家子見是主公,無不敬謹。

須臾間敘及寒溫,「牛王」道:「夫人久闊。」羅剎道:「大王萬福。」又云:「大王寵幸新婚,拋撇奴家,今日是那陣風兒吹你來的?』大聖笑道:「非敢拋撇,只因玉面公主招後,家事繁冗,朋友多顧,是以稽留在外,卻也又治得一個家當了。」又道:「近聞悟空那廝保唐僧,將近火焰山界,恐他來問你借扇子。我恨那廝害子之仇未報,但來時,可差人報我,等我拿他,分屍萬段,以雪我夫妻之恨。」羅剎聞言,滴淚告道:「大王,常言說:『男兒無婦財無主,女子無夫身無主。』我的性命,險些兒被這猢猻害了。」大聖聽得,故意發怒,罵道:「那潑猴幾時過去了?」羅剎道:「還未去。昨日到我這裡借扇子,我因他害孩兒之故,披掛了,掄寶劍出門,就砍那猢猻。他忍著疼,叫我做嫂嫂,說大王曾與他結義。」大聖道:「是五百年前曾拜為七弟兄。」羅剎道:「被我罵也不敢回言,砍也不敢動手。後被我一扇子搧去。不知在那裡尋得個定風法兒,今早又在門外叫喚。是我又使扇搧,莫想得動。急掄劍砍時,他就不讓我了。我怕他棒重,就走入洞裡,緊關上門。不知他又從何處,鑽在我肚腹之內,險被他害了性命。是我叫他幾聲叔叔,將扇與他去也。」

大聖又假意搥胸道:「可惜,可惜。夫人錯了,怎麼就把這寶貝與那猢猻?惱殺我也。」羅剎笑道:「大王息怒。與他的是假扇,但哄他去了。」大聖問:「真扇在於何處?」羅剎道:「放心,放心,我收著哩。」叫丫鬟整酒接風賀喜。遂擎杯奉上道:「大王,燕爾新婚,千萬莫忘結髮,且吃一杯鄉中之水。」大聖不敢不接,只得笑吟吟,舉觴在手道:「夫人先飲。我因圖治外產,久別夫人,早晚蒙護守家門,權為酬謝。」羅剎復接杯斟起,遞與大王道:「自古道:『妻者,齊也。』夫乃養身之父,謝甚麼?」他兩人謙謙講講,方才坐下巡酒。大聖不敢破葷,只吃幾個果子,與他言言語語。

酒至數巡,羅剎覺有半酣,色情微動,就和孫大聖挨挨擦擦,搭搭拈拈:攜著手,俏語溫存;並著肩,低聲俯就。將一杯酒,你喝一口,我喝一口,卻又哺果。大聖假意虛情,相陪相笑,沒奈何,也與他相倚相偎。果然是:
\begin{quote}
釣詩鉤,掃愁帚,破除萬事無過酒。男兒立節放襟懷,女子忘情開笑口。面赤似夭桃,身搖如嫩柳。絮絮叨叨話語多,捻捻掐掐風情有。時見掠雲鬟,又見掄尖手。幾番常把腳兒蹺,數次每將衣袖抖。粉項自然低,蠻腰漸覺扭。合歡言語不曾丟,酥胸半露鬆金鈕。醉來真個玉山頹,餳眼摩娑幾弄醜。
\end{quote}

大聖見他這等酣然,暗自留心,挑鬥道:「夫人,真扇子你收在那裡?早晚仔細,但恐孫行者變化多端,卻又來騙去。」羅剎笑嘻嘻的,口中吐出,只有一個杏葉兒大小,遞與大聖道:「這個不是寶貝?」大聖接在手中,卻又不信,暗想著:「這些些兒,怎生搧得火滅?怕又是假的。」羅剎見他看著寶貝沉思,忍不住上前,將粉面搵在行者臉上,叫道:「親親,你收了寶貝吃酒罷,只管出神想甚麼哩?」大聖就趁腳兒蹺,問他一句道:「這般小小之物,如何搧得八百里火焰?」羅剎酒陶真性,無忌憚,就說出方法道:「大王,與你別了二載,你想是晝夜貪歡,被那玉面公主弄傷了神思,怎麼自家的寶貝事情,也都忘了?只將左手大指頭捻著那柄兒上第七縷紅絲,念一聲『噓啊吸嘻吹呼』,即長一丈二尺長短。這寶貝變化無窮!那怕他八萬里火焰,可一扇而消也。」

大聖聞言,切切記在心上。卻把扇兒也噙在口裡,把臉抹一抹,現了本像。厲聲高叫道:「羅剎女,你看看我可是你親老公?就把我纏了這許多醜勾當,不羞,不羞。」那女子一見是孫行者,慌得推倒桌席,跌落塵埃,羞愧無比,只叫:「氣殺我也!氣殺我也!」

這大聖不管他死活,捽脫手,拽大步,徑出了芭蕉洞。正是:無心貪美色,得意笑顏回。將身一縱,踏祥雲,跳上高山,將扇子吐出來,演演方法。將左手大指頭捻著那柄上第七縷紅絲,念了一聲「噓啊吸嘻吹呼」,果然長了有一丈二尺長短。拿在手中,仔細看了又看,比前番假的果是不同。只見祥光晃晃,瑞氣紛紛,上有三十六縷紅絲,穿經度絡,表裡相聯。原來行者只討了個長的方法,不曾討他個小的口訣,左右只是那等長短。沒奈何,只得搴在肩上,找舊路而回,不題。

卻說那牛魔王在碧波潭底與眾精散了筵席,出得門來,不見了辟水金睛獸。老龍王聚眾精問道:「是誰偷放牛爺的金睛獸也?」眾精跪下道:「沒人敢偷。我等俱在筵前供酒捧盤,供唱奏樂,更無一人在前。」老龍道:「家樂兒斷乎不敢,可曾有甚生人進來?」龍子、龍孫道:「適才安座之時,有個蟹精到此,那個便是生人。」牛王聞說,頓然省悟道:「不消講了。早間賢友著人邀我時,有個孫悟空保唐僧取經,路遇火焰山難過,曾問我求借芭蕉扇。我不曾與他,他和我賭鬥一場,未分勝負。我卻丟了他,徑赴盛會。那猴子千般伶俐,萬樣機關,斷乎是那廝變作蟹精,來此打探消息,偷了我獸,去山妻處騙了那一把芭蕉扇兒也。」眾精見說,一個個膽戰心驚,問道:「可是那大鬧天宮的孫悟空麼?」牛王道:「正是。列公若在西天路上,有不是處,切要躲避他些兒。」老龍道:「似這般說,大王的駿騎卻如之何?」牛王笑道:「不妨,不妨。列公各散,等我趕他去來。」

遂而分開水路,跳出潭底,駕黃雲,徑至翠雲山芭蕉洞。只聽得羅剎女跌腳搥胸,大呼小叫。推開門,又見辟水金睛獸拴在下邊。牛王高叫:「夫人,孫悟空那廂去了?」眾女童看見牛魔,一齊跪下道:「爺爺來了?」羅剎女扯住牛王,磕頭撞腦,口裡罵道:「潑老天殺的!怎樣這般不謹慎,著那猢猻偷了金睛獸,變作你的模樣,到此騙我?」牛王切齒道:「猢猻那廂去了?」羅剎搥著胸膛罵道:「那潑猴賺了我的寶貝,現出原身走了。氣殺我也!」牛王道:「夫人保重,勿得心焦。等我趕上猢猻,奪了寶貝,剝了他皮,剉碎他骨,擺出他的心肝,與你出氣。」叫:「拿兵器來。」女童道:「爺爺的兵器不在這裡。」牛王道:「拿你奶奶的兵器來罷。」侍婢將兩把青鋒寶劍捧出。牛王脫了那赴宴的鴉青絨襖,束一束貼身的小衣,雙手綽劍,走出芭蕉洞,徑奔火焰山上趕來。正是那:
\begin{quote}
忘恩漢騙了痴心婦,烈性魔來近木叉人。
\end{quote}

畢竟不知此去吉凶如何,且聽下回分解。
