
\chapter{嬰兒問母知邪正 金木參玄見假真}

\begin{quote}
逢君只說受生因,便作如來會上人。
一念靜觀塵世佛,十方同看降威神。
欲知今日真明主,須問當年嫡母身。
別有世間曾未見,一行一步一花新。
\end{quote}

卻說那烏雞國王太子自別大聖,不多時,回至城中。果然不奔朝門,不敢報傳宣詔,徑至後宰門首,見幾個太監在那裡把守。見太子來,不敢阻滯,讓他進去了。好太子,夾一夾馬,撞入裡面,忽至錦香亭下。只見那正宮娘娘坐在錦香亭上,兩邊有數十個嬪妃掌扇,那娘娘倚雕欄兒流淚哩。你道他流淚怎的?原來他四更時也做了一夢,記得一半,含糊了一半,沉沉思想。

這太子下馬,跪於亭下,叫:「母親。」那娘娘強整歡容,叫聲:「孩兒,喜呀!喜呀!這二三年在前殿與你父王開講,不得相見,我甚思量。今日如何得暇來看我一面?誠萬千之喜!誠萬千之喜!孩兒,你怎麼聲音悲慘?你父王年紀高邁,有一日龍歸碧海,鳳返丹霄,你就傳了帝位,還有甚麼不悅?」太子叩頭道:「母親,我問你:即位登龍是那個?稱孤道寡果何人?」娘娘聞言道:「這孩兒發風了?做皇帝的是你父王,你問怎的?」太子叩頭道:「萬望母親赦子無罪,敢問;不赦,不敢問。」娘娘道:「子母家有何罪?赦你,赦你,快快說來。」太子道:「母親,我問你三年前夫妻宮裡之事,與後三年恩愛同否如何?」娘娘見說,魂飄魄散,急下亭抱起,緊摟在懷,眼中滴淚道:「孩兒,我與你久不相見,怎麼今日來宮問此?」太子發怒道:「母親有話早說;不說時,且誤了大事。」娘娘才喝退左右,淚眼低聲道:「這樁事,孩兒不問,我到九泉之下,也不得明白。既問時,聽我說:
\begin{quote}
三載之前溫又暖,三年之後冷如冰。
枕邊切切將言問,他說老邁身衰事不興。
\end{quote}

太子聞言,撒手脫身,攀鞍上馬。那娘娘一把扯住道:「孩兒,你有甚事,話不終就走?」太子跪在面前道:「母親,不敢說。今日早朝,蒙欽差架鷹逐犬,出城打獵。偶遇東土駕下來的個取經聖僧,有大徒弟乃孫行者,極善降妖。原來我父王死在御花園八角琉璃井內,這全真假變父王,侵了龍位。今夜三更,父王託夢,請他到城捉怪。孩兒不敢盡信,特來問母。母親才說出這等言語,必然是個妖精。」那娘娘道:「兒啊,外人之言,你怎麼就信為實?」太子道:「兒還不敢認實,父王遺下表記與他了。」娘娘問是何物,太子袖中取出那金廂白玉珪,遞與娘娘。那娘娘認得是當時國王之寶,止不住淚如泉湧。叫聲:「主公,你怎麼死去三年,不來見我,卻先見聖僧,後見太子?」太子道:「母親,這話是怎的說?」娘娘道:「兒啊,我四更時分,也做了一夢,夢見你父王水淋淋的站在我跟前,親說他死了,鬼魂兒拜請了唐僧,降假皇帝,救他前身。記便記得是這等言語,只是一半兒不得分明。正在這裡狐疑,怎知今日你又來說這話,又將寶貝拿出。我且收下,你且去請那聖僧急急為之。果然掃蕩妖氛,辨明邪正,庶報你父王養育之恩也。」

太子急忙上馬,出後宰門,躲離城池。真個是噙淚叩頭辭國母,含悲頓首復唐僧。不多時,出了城門,徑至寶林寺山門前下馬。眾軍士接著太子,又見紅輪將墜。太子傳令:不許軍士亂動。他又獨自個入了山門,整束衣冠,拜請行者。只見那猴王從正殿搖搖擺擺走來,那太子雙膝跪下道:「師父,我來了。」行者上前攙住道:「請起。你到城中,可曾問誰麼?」太子道:「問母親來。」將前言盡說了一遍。行者微微笑道:「若是那般冷啊,想是個甚麼冰冷的東西變的。不打緊,不打緊,等我老孫與你掃蕩。卻只是今日晚了,不好行事。你先回去,待明早我來。」太子跪地叩拜道:「師父,我只在此,伺候到明日,同師父一路去罷。」行者道:「不好,不好。若是與你一同入城,那怪物生疑,不說是我撞著你,卻說是你請老孫,卻不惹他反怪你也?」太子道:「我如今進城,他也怪我。」行者道:「怪你怎麼?」太子道:「我自早朝蒙差,帶領若干人馬鷹犬出城,今一日更無一件野物,怎麼見駕?若問我個不才之罪,監陷羑里,你明日進城,卻將何倚?況那班部中,更沒個相知人也。」行者道:「這甚打緊?你肯早說時,卻不尋下些等你?」

好大聖,你看他就在太子面前,顯個手段,將身一縱,跳在雲端裡。捻著訣,念一聲「唵藍淨法界」的真言,拘得那山神、土地在半空中施禮道:「大聖,呼喚小神,有何使令?」行者道:「老孫保護唐僧到此,欲拿邪魔,奈何那太子打獵無物,不敢回朝。問汝等討個人情,快將獐鹿兔、走獸飛禽,各尋些來,打發他回去。」山神、土地聞言,敢不承命,又問各要幾何。大聖道:「不拘多少,取些來便罷。」那各神即著本處陰兵,刮一陣聚獸陰風,捉了些野雞山雉、角鹿肥獐、狐獾狢兔、虎豹狼蟲,共有百千餘隻,獻與行者。行者道:「老孫不要,你可把他都捻就了觔,單擺在那四十里路上兩傍,教那些人不放鷹犬,拿回城去,算了汝等之功。」眾神依言,收了陰風,擺在左右。

行者才按雲頭,對太子道:「殿下請回,路上已有物了,你自收去。」太子見他在半空中弄此神通,如何不信,只得叩頭拜別。出山門傳了令,教軍士們回城。只見那路傍果有無限的野物,軍士們不放鷹犬,一個個俱著手擒捉喝采,俱道是千歲殿下的洪福,怎知是老孫的神功。你聽凱歌聲唱,一擁回城。

這行者保護了三藏。那本寺中的和尚見他們與太子這樣綢繆,怎不恭敬?卻又安排齋供,管待了唐僧,依然還歇在禪堂裡。將近有一更時分,行者心中有事,急睡不著。他一轂轆爬起來,到唐僧床前,叫:「師父。」此時長老還未睡哩,他曉得行者會失驚打怪的,推睡不應。行者摸著他的光頭,亂搖道:「師父怎睡著了?」唐僧怒道:「這個頑皮,這早晚還不睡,吆喝甚麼?」行者道:「師父,有一樁事兒,和你計較計較。」長老道:「甚麼事?」行者道:「我日間與那太子誇口,說我的手段比山還高,比海還深,拿那妖精如探囊取物一般,伸了手去就拿將轉來。卻也睡不著,想起來,有些難哩。」唐僧道:「你說難,便就不拿了罷。」行者道:「拿是還要拿,只是理上不順。」唐僧道:「這猴頭亂說。妖精奪了人君位,怎麼叫做理上不順?」行者道:「你老人家只知念經拜佛,打坐參禪,那曾見那蕭何的律法?常言道:『拿賊拿贓。』那怪物做了三年皇帝,又不曾走了馬腳,漏了風聲。他與三宮妃后同眠,又和兩班文武共樂,我老孫就有本事拿住他,也不好定個罪名。」唐僧道:「怎麼不好定罪?」行者道:「他就是個沒嘴的葫蘆,也與你滾上幾滾。他敢道:『我是烏雞國王,有甚逆天之事,你來拿我?』將甚執照與他折辯?」

唐僧道:「憑你怎生裁處?」行者笑道:「老孫的計已成了。只是干礙著你老人家,有些兒護短。」唐僧道:「我怎麼護短?」行者道:「八戒生得夯,你有些兒偏向他。」唐僧道:「我怎麼向他?」行者道:「你若不向他啊,且如今把膽放大些,與沙僧只在這裡。待老孫與八戒趁此時先入那烏雞國城中,尋著御花園,打開琉璃井,把那皇帝屍首撈將上來,包在我們包袱裡。明日進城,且不管甚麼倒換文牒,見了那怪,掣棍來就打。他但有言語,就將骨襯與他看,說:『你殺的是這個人。』卻教太子上來哭父,皇后出來認夫,文武多官見主,我老孫與兄弟們動手。這才是有對頭的官事好打。」唐僧聞言,暗喜道:「只怕八戒不肯去。」行者笑道:「如何?我說你護短。你怎麼就知他不肯去?你只像我叫你時不答應,半個時辰便了。我這去,但憑三寸不爛之舌,莫說是豬八戒,就是豬九戒,也有本事教他跟著我走。」唐僧道:「也罷,隨你去叫他。」

行者離了師父,徑到八戒床邊,叫:「八戒,八戒。」那獃子是走路辛苦的人,丟倒頭,只情打,那裡叫得醒。行者揪著耳朵,抓著鬃,把他一拉,拉起來,叫聲:「八戒。」那獃子還打棱掙。行者又叫一聲,獃子道:「睡了罷,莫頑,明日要走路哩。」行者道:「不是頑,有一樁買賣,我和你做去。」八戒道:「甚麼買賣?」行者道:「你可曾聽得那太子說麼?」八戒道:「我不曾見面,不曾聽見說甚麼。」行者說:「那太子告訴我說,那妖精有件寶貝,萬夫不當之勇。我們明日進城,不免與他爭敵,倘那怪執了寶貝,降倒我們,卻不反成不美?我想著打人不過,不如先下手。我和你去偷他的來,卻不是好?」八戒道:「哥哥,你哄我去做賊哩。這個買賣,我也去得。果是曉得實實的幫寸,我也與你講個明白:偷了寶貝,降了妖精,我卻不奈煩甚麼小家罕氣的分寶貝,我就要了。」行者道:「你要作甚?」八戒道:「我不如你們乖巧能言,人面前化得出齋來。老豬身子又夯,言語又粗,不能念經,若到那無濟無生處,可好換齋吃麼。」行者道:「老孫只要圖名,那裡圖甚寶貝?就與你便了。」那獃子聽見說都與他,他就滿心歡喜,一轂轆爬將起來,套上衣服,就和行者走路。這是清酒紅人面,黃金動道心。

兩個密密的開了門,躲離三藏,縱祥光,徑奔那城。不多時到了,按落雲頭,只聽得樓頭方二鼓矣。行者道:「兄弟,二更時分了。」八戒道:「正好,正好,人都在頭覺裡正濃睡也。」二人不奔正陽門,徑到後宰門首,只聽得梆鈴聲響。行者道:「兄弟,前後門皆緊急,如何得入?」八戒道:「那見做賊的從門裡走麼,瞞牆跳過便罷。」行者依言,將身一縱,跳上裡羅城牆。八戒也跳上去。二人潛入裡面,找著門路,徑尋那御花園。

正行時,只見有一座三簷白簇的門樓,上有三個亮灼灼的大字,映著那星月光輝,乃是「御花園」。行者近前看了,有幾重封皮,公然將鎖門鏽住了,即命八戒動手。那獃子掣鐵鈀,盡力一築,把門築得粉碎。行者先舉步䟕入,忍不住跳將起來,大呼小叫。諕得八戒上前扯住道:「哥呀,害殺我也。那見做賊的這般吆喝?驚醒了人,把我們拿住,發到官司,就不該死罪,也要解回原籍充軍。」行者道:「兄弟啊,你說我發急為何?你看這:
\begin{quote}
彩畫雕欄狼狽,寶妝亭閣攲歪。莎汀蓼岸盡塵埋,芍藥荼俱敗。茉莉玫瑰香暗,牡丹百合空開。芙蓉木槿草垓垓,異卉奇葩壅壞。巧石山峰俱倒,池塘水涸魚衰。青松紫竹似乾柴,滿路茸茸蒿艾。丹桂碧桃枝損,海榴棠棣根歪。橋頭曲徑有蒼苔,冷落花園境界。」
\end{quote}

八戒道:「且嘆他做甚?快幹我們的買賣去來。」行者雖然感慨,卻留心想起唐僧的夢來,說芭蕉樹下方是井。正行走,果見一株芭蕉,生得茂盛,比眾花木不同。真是:
\begin{quote}
一種靈苗秀,天生體性空。
枝枝抽片紙,葉葉捲芳叢。
翠縷千條細,丹心一點紅。
淒涼愁夜雨,憔悴怯秋風。
長養元丁力,栽培造化工。
緘書成妙用,揮灑有奇功。
鳳翎寧得似,鸞尾迥相同。
薄露瀼瀼滴,輕煙淡淡籠。
青陰遮戶牖,碧影上簾櫳。
不許棲鴻雁,何堪繫玉驄。
霜天形槁悴,月夜色朦朧。
僅可消炎暑,猶宜避日烘。
愧無桃李色,冷落粉牆東。
\end{quote}

行者道:「八戒,動手麼,寶貝在芭蕉樹下埋著哩。」那獃子雙手舉鈀,築倒了芭蕉。然後用嘴一拱,拱了有三四尺深,見一塊石板蓋著。獃子歡喜道:「哥呀,造化了,果有寶貝,是一片石板蓋著哩。不知是罈兒盛著,是櫃兒裝著哩。」行者道:「你掀起來看看。」那獃子果又一嘴拱開,看處,又見霞光灼灼,白氣明明。八戒笑道:「造化,造化,寶貝放光哩。」又近前細看時,呀!原來是星月之光,映得那井中水亮。八戒道:「哥呀,你但幹事,便要留根。」行者道:「我怎留根?」八戒道:「這是一眼井,你在寺裡早說是井中有寶貝,我卻帶將兩條綑包袱的繩來,怎麼作個法兒,把老豬放下去。如今空手,這裡面東西,怎麼得下去上來耶?」行者道:「你下去麼?」八戒道:「正是要下去,只是沒繩索。」行者笑道:「你脫了衣服,我與你個手段。」八戒道:「有甚麼好衣服?解了這直裰子就是了。」

好大聖,把金箍棒拿出來,兩頭一扯,叫:「長!」足有七八丈長。教:「八戒,你抱著一頭兒,把你放下井去。」八戒道:「哥呀,放便放下去,若到水邊,就住了罷。」行者道:「我曉得。」那獃子抱著鐵棒,被行者輕輕提將起來,將他放下去,不多時,放至水邊。八戒道:「到水了。」行者聽見他說,卻將棒往下一按。那獃子撲通的一個沒頭蹲,丟了鐵棒,便就負水,口裡哺哺的嚷道:「這天殺的,我說到水莫放,他卻就把我一按。」行者掣上棒來,笑道:「兄弟,可有寶貝麼?」八戒道:「見甚麼寶貝,只是一井水。」行者道:「寶貝沉在水底下哩,你下去摸一摸來。」呆子真個深知水性,卻就打個猛子,淬將下去。呀!那井底深得緊。他卻著實又一淬,忽睜眼見有一座牌樓,上有「水晶宮」三個字。八戒大驚道:「罷了,罷了,錯走了路了,蹡下海來也。海內有個水晶宮,井裡如何有之?」原來八戒不知此是井龍王的水晶宮。

八戒正敘話處,早有一個巡水的夜叉開了門,看見他的模樣,急抽身進去報道:「大王,禍事了,井上落一個長嘴大耳的和尚來了,赤淋淋的,衣服全無,還不死,逼法說話哩。」那井龍王忽聞此言,心中大驚道:「這是天蓬元帥來也。昨夜夜遊神奉上敕旨,來取烏雞國王魂靈去拜見唐僧,請齊天大聖降妖。這怕是齊天大聖、天蓬元帥來了,卻不可怠慢他,快接他去也。」

那龍王整衣冠,領眾水族,出門來厲聲高叫道:「天蓬元帥,請裡面坐。」八戒卻才歡喜道:「原來是個故知。」那獃子不管好歹,徑入水晶宮裡。其實不知上下,赤淋淋的,就坐在上面。龍王道:「元帥,近聞你得了性命,皈依釋教,保唐僧西天取經,如何得到此處?」八戒道:「正為此說。我師兄孫悟空多多拜上,著我來問你取甚麼寶貝哩。」龍王道:「可憐,我這裡怎麼得個寶貝?比不得那江、河、淮、濟的龍王,飛騰變化,便有寶貝。我久困於此,日月且不能長見,寶貝果何自而來也?」八戒道:「不要推辭,有便拿出來罷。」龍王道:「有便有一件寶貝,只是拿不出來,就元帥親自來看看,何如?」八戒道:「妙妙妙,須是看看來也。」

那龍王前走,這獃子隨後。轉過了水晶宮殿,只見廊廡下,橫躺著一個六尺長軀。龍王用手指定道:「元帥,那廂就是寶貝了。」八戒上前看了,呀!原來是個死皇帝,戴著沖天冠,穿著赭黃袍,踏著無憂履,繫著藍田帶,直挺挺睡在那廂。八戒笑道:「難難難,算不得寶貝。想老豬在山為怪時,時常將此物當飯,且莫說見的多少,吃也吃夠無數,那裡叫做甚麼寶貝?」龍王道:「元帥原來不知。他本是烏雞國王的屍首,自到井中,我與他定顏珠定住,不曾得壞。你若肯馱他出去,見了齊天大聖,假有起死回生之意啊,莫說寶貝,憑你要甚麼東西都有。」八戒道:「既這等說,我與你馱出去,只說把多少燒埋錢與我?」龍王道「其實無錢。」八戒道:「你好白使人?果然沒錢,不馱。」龍王道:「不馱,請行。」八戒就走。龍王差兩個有力量的夜叉,把屍擡將出去,送到水晶宮門外,丟在那廂,摘了辟水珠,就有水響。

八戒急回頭看,不見水晶宮門,一把摸著那皇帝的屍首,慌得他腳軟筋麻,攛出水面,扳著井牆,叫道:「師兄,伸下棒來救我一救。」行者道:「可有寶貝麼?」八戒道:「那裡有,只是水底下有一個井龍王,教我馱死人,我不曾馱,他就把我送出門來,就不見那水晶宮了,只摸著那個屍首。諕得我手軟筋麻,掙搓不動了。哥呀,好歹救我救兒。」行者道:「那個就是寶貝,如何不馱上來?」八戒道:「知他死了多少時了,我馱他怎的?」行者道:「你不馱,我回去耶。」八戒道:「你回那裡去?」行者道:「我回寺中,同師父睡覺去。」八戒道:「我就不去了?」行者道:「你爬得上來,便帶你去;爬不上來,便罷。」八戒慌了:「怎生爬得動?你想,城牆也難上,這井肚子大,口兒小,壁陡的圈牆,又是幾年不曾打水的井,團團都長的是苔痕,好不滑也,教我怎爬?哥哥,不要失了兄弟們和氣,等我馱上來罷。」行者道:「正是,快快馱上來,我同你回去睡覺。」那獃子又一個猛子,淬將下去,摸著屍首,拽過來,背在身上,攛出水面,扶井牆道:「哥哥,馱上來了。」那行者睜睛看處,真個的背在身上,卻才把金箍棒伸下井底。那獃子著了惱的人,張開口,咬著鐵棒,被行者輕輕的提將出來。

八戒將屍放下,撈過衣服穿了。行者看時,那皇帝容顏依舊,似生時未改分毫。行者道:「兄弟啊,這人死了三年,怎麼還容顏不壞?」八戒道:「你不知之。這井龍王對我說,他使了定顏珠定住了,屍首未曾壞得。」行者道:「造化,造化。一則是他的冤仇未報,二來該我們成功。兄弟快把他馱了去。」八戒道:「馱往那裡去?」行者道:「馱了去見師父。」八戒口中作念道:「怎的起?怎的起?好好睡覺的人,被這猢猻花言巧語,哄我教做甚麼買賣,如今卻幹這等事,教我馱死人。馱著他,腌臢水淋將下來,污了衣服,沒人與我漿洗。上面有幾個補丁,天陰發潮,如何穿麼?」行者道:「你只管馱了去,到寺裡,我與你換衣服。」八戒道:「不羞,連你穿的也沒有,又替我換?」行者道:「這般弄嘴,便不馱罷?」八戒道:「不馱。」行者道:「便伸過孤拐來,打二十棒。」八戒慌了道:「哥哥,那棒子重,若是打上二十,我與這皇帝一般了。」行者道:「怕打時,趁早兒馱著走路。」八戒果然怕打,沒好氣,把屍首拽將過來,背在身上,拽步出園就走。

好大聖,捻著訣,念聲咒語,往巽地上吸一口氣,吹將去,就是一陣狂風,把八戒撮出皇宮內院,躲離了城池,息了風頭,二人落地,徐徐卻走將來。那獃子心中暗惱,算計要報恨行者,道:「這猴子捉弄我,我到寺裡也捉弄他捉弄:攛道師父,只說他醫得活;醫不活,教師父念緊箍兒咒,把這猴子的腦漿勒出來,方趁我心。」走著路,再再尋思道:「不好,不好。若教他醫人,卻是容易:他去閻王家討將魂靈兒來,就醫活了。只說不許赴陰司,陽世間就能醫活,這法兒才好。」

說不了,卻到了山門前,徑直進去,將屍首丟在那禪堂門前。道:「師父,起來看邪。」那唐僧睡不著,正與沙僧講行者哄了八戒去久不回之事。忽聽得他來叫了一聲,唐僧連忙起身道:「徒弟,看甚麼?」八戒道:「行者的外公,教老豬馱將來了。」行者道:「你這饢糟的獃子,我那裡有甚麼外公?」八戒道:「哥,不是你外公,卻教老豬馱他來怎麼?也不知費了多少力了。」

那唐僧與沙僧開門看處,那皇帝容顏未改,似活的一般。長老忽然慘悽道:「陛下,你不知那世裡冤家,今生遇著他,暗喪其身,拋妻別子,致令文武不知,多官不曉。可憐你妻子昏蒙,誰曾見焚香獻茶?」忽失聲淚如雨下。八戒笑道:「師父,他死了可干你事?又不是你家父祖,哭他怎的?」三藏道:「徒弟啊,出家人慈悲為本,方便為門。你怎的這等心硬?」八戒道:「不是心硬,師兄和我說來,他會醫得活;若醫不活,我也不馱他來了。」那長老原來是一頭水的,被那獃子搖動了,他就叫:「悟空,若果有手段醫活這個皇帝,正是『救人一命,勝造七級浮圖』。我等也強似靈山拜佛。」行者道:「師父,你怎麼信這獃子亂談?人若死了,或三七五七,盡七百日,受滿了陽間罪過,就轉生去了。如今已死三年,如何救得?」三藏聞其言道:「也罷了。」八戒苦恨不息,道:「師父,你莫被他瞞了,他有些夾腦風。你只念念那話兒,管他還你一個活人。」真個唐僧就念緊箍兒咒,勒得那猴子眼脹頭疼。

畢竟不知怎生醫救,且聽下回分解。
