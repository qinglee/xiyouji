
\chapter{黃獅精虛設釘鈀宴 金木土計鬧豹頭山}

卻說那院中幾個鐵匠因連日辛苦,夜間俱自睡了。及天明起來打造,篷下不見了三般兵器,一個個呆掙神驚,四下尋找。只見那三個王子出宮來看,那鐵匠一齊磕頭道:「小主啊,神師的三般兵器,都不知那裡去了。」

小王子聽言,心驚膽戰道:「想是師父今夜收拾去了。」急奔暴紗亭看時,見白馬尚在廊下,忍不住叫道:「師父還睡哩?」沙僧道:「起來了。」即將房門開了,讓王子進裡看時,不見兵器,慌慌張張問道:「師父的兵器都收來了?」行者跳起道:「不曾收啊。」王子道:「三般兵器,今夜都不見了。」八戒連忙爬起道:「我的鈀在麼?」小王道:「適才我等出來,只見眾人前後找尋不見,弟子恐是師父收了,卻才來問。老師的寶貝,俱是能長能消,想必藏在身邊哄弟子哩。」行者道:「委的未收。都尋去來。」

隨至院中篷下,果然不見蹤影。八戒道:「定是這夥鐵匠偷了。快拿出來,略遲了些兒,就都打死,打死。」那鐵匠慌得磕頭滴淚道:「爺爺,我們連日辛苦,夜間睡著,乃至天明起來,遂不見了。我等乃一概凡人,怎麼拿得動?望爺爺饒命,饒命。」行者無語,暗恨道:「還是我們的不是,既然看了式樣,就該收在身邊,怎麼卻丟放在此?那寶貝霞彩光生,想是驚動甚麼歹人,乘夜竊去也。」八戒不信道:「哥哥說那裡話?這般個太平境界,又不是曠野深山,怎得個歹人來?定是鐵匠欺心,他見我們的兵器光彩,認得是三件寶貝,連夜走出王府,夥些人來,擡的擡,拉的拉,偷出去了。拿過來打呀,打呀。」眾匠只是磕頭發誓。

正嚷處,只見老王子出來,問及前事,卻也面無人色,沉吟半晌道:「神師兵器,本不同凡,就有百十餘人也禁挫不動。況孤在此城,今已五代,不是大膽海口,孤也頗有個賢名在外;這城中軍民匠作人等,也頗懼孤之法度,斷是不敢欺心。望神師再思可矣。」行者笑道:「不用再思,也不須苦賴鐵匠。我問殿下:你這州城四面,可有甚麼山林妖怪?」王子道:「神師此問,甚是有理。孤這州城之北,有一座豹頭山,山中有一座虎口洞。往往人言洞內有仙,又言有虎狼,又言有妖怪。孤未曾訪得端的,不知果是何物。」行者笑道:「不消講了,定是那方歹人,知道俱是寶貝,一夜偷將去了。」叫:「八戒、沙僧,你都在此保著師父,護著城池,等老孫尋訪去來。」又叫鐵匠們不可住了爐火,一一煉造。

好猴王,辭了三藏,唿哨一聲,形影不見,早跨到豹頭山上。原來那城相去只有三十里,一瞬即到。徑上山峰觀看,果然有些妖氣。真是:
\begin{quote}
龍脈悠長,地形遠大。尖峰挺挺插天高,陡澗沉沉流水急。山前有瑤草鋪茵,山後有奇花佈錦。喬松老柏,古樹修篁。山鴉山鵲亂飛鳴,野鶴野猴皆嘯唳。懸崖下,麋鹿雙雙;峭壁前,獾狐對對。一起一伏遠來龍,九曲九灣潛地脈。埂頭相接玉華州,萬古千秋興勝處。
\end{quote}

行者正然看時,忽聽得山背後有人言語。急回頭視之,乃兩個狼頭妖怪,朗朗的說著話,向西北上走。行者揣道:「這定是巡山的怪物,等老孫跟他去聽聽,看他說些甚的。」捻著訣,念個咒,搖身一變,變做個蝴蝶兒,展開翅,翩翩翻翻,徑自趕上。果然變得有樣範:
\begin{quote}
一雙粉翅,兩道銀鬚。乘風飛去急,映日舞來徐。渡水過牆能疾俏,偷香弄絮甚歡娛。體輕偏愛鮮花味,雅態芳情任卷舒。
\end{quote}

他飛在那個妖精頭頂上,飄飄蕩蕩,聽他說話。那妖猛的叫道:「二哥,我大王連日僥倖:前月裡得了一個美人兒,在洞內盤桓,十分快樂;昨夜裡又得了三般兵器,果然是無價之寶。明朝開宴慶釘鈀會哩,我們都有受用。」這個道:「我們也有些僥倖:拿這二十兩銀子買豬羊去。如今到了乾方集上,先吃幾壺酒兒。把東西開個花帳兒,落他二三兩銀子,買件綿衣過寒,卻不是好?」兩個怪說說笑笑的,上大路急走如飛。

行者聽得要慶釘鈀會,心中暗喜。欲要打殺他,爭奈不干他事,況手中又無兵器。他即飛向前邊,現了本相,在路口上立定。那怪看看走到身邊,被他一口法唾噴將去,念一聲「唵吽吒唎」,即使個定身法,把兩個狼頭精定住。眼睜睜,口也難開;直挺挺,雙腳站住。又將他扳翻倒,揭衣搜檢,果是有二十兩銀子,著一條搭包兒打在腰間裙帶上;又各掛著一個粉漆牌兒,一個上寫著「刁鑽古怪」,一個上寫著「古怪刁鑽」。

好大聖,取了他銀子,解了他牌兒,返跨步回至州城。到王府中,見了王子、唐僧並大小官員、匠作人等,具言前事。八戒笑道:「想是老豬的寶貝霞彩光明,所以買豬羊,治筵席慶賀哩。但如今怎得他來?」行者道:「我兄弟三人俱去。這銀子是買辦豬羊的,且將這銀子賞了匠人,教殿下尋幾個豬羊。八戒,你變做刁鑽古怪,我變做古怪刁鑽,沙僧裝做個販豬羊的客人,走進那虎口洞裡,得便處,各人拿了兵器,打絕那妖邪,回來卻收拾走路。」沙僧笑道:「妙妙妙,不宜遲,快走。」老王果依此計,即教管事的買辦了七八口豬、四五腔羊。

他三人辭了師父,在城外大顯神通。八戒道:「哥哥,我未曾看見那刁鑽古怪,怎生變得他模樣?」行者道:「那怪被老孫使了定身法定住在那裡,直到明日此時方醒。我記得他的模樣,你站下,等我教你變。如此,如彼,就是他的模樣了。」那獃子真個口裡念著咒,行者吹口仙氣,霎時就變得與那刁鑽古怪一般無二,將一個粉牌兒帶在腰間。行者即變做古怪刁鑽,腰間也帶了一個牌兒。沙僧打扮得像個販豬羊的客人。一起兒趕著豬羊,上大路,徑奔山來。

不多時,進了山凹裡,又遇見一個小妖,他生得嘴臉也恁地兇惡!看那:
\begin{quote}
圓滴溜兩隻眼,如燈晃亮;紅剌媸一頭毛,似火飄光。糟鼻子,猍口,獠牙尖利;查耳朵,砍額頭,青臉泡浮。身穿一件淺黃衣,足踏一雙莎蒲履。雄雄糾糾若兇神,急急忙忙如惡鬼。
\end{quote}

那怪左脅下挾著一個彩漆的請書匣兒,迎著行者叫道:「古怪刁鑽,你兩個來了?買了幾口豬羊?」行者道:「這趕的不是?」那怪朝沙僧道:「此位是誰?」行者道:「就是販豬羊的客人。還少他幾兩銀子,帶他來家取的。你往那裡去?」那怪道:「我往竹節山去請老大王明早赴會。」行者綽他的口氣兒,就問:「共請多少人?」那怪道:「請老大王坐首席,連本山大王共頭目等眾,約有四十多位。」正說處,八戒道:「去罷,去罷,豬羊都四散走了。」行者道:「你去邀著,等我討他帖兒看看。」那怪見自家人,即揭開取出,遞與行者。行者展開看時,上寫著:
\begin{quote}
明辰敬治餚酌,慶釘鈀嘉會,屈尊過山一敘。幸勿外,至感。右啟祖翁九靈元聖老大人尊前。門下孫黃獅頓首百拜。
\end{quote}

行者看畢,仍遞與那怪。那怪放在匣內,徑往東南上去了。

沙僧問道:「哥哥,帖兒上是甚麼話頭?」行者道:「乃慶釘鈀會的請帖。名字寫著『門下孫黃獅頓首百拜』。請的是祖翁九靈元聖老大人。」沙僧笑道:「黃獅想必是個金毛獅子成精。但不知九靈元聖是個何物?」八戒聽言,笑道:「是老豬的貨了。」行者道:「怎見得是你的貨?」八戒道:「古人云:『癩母豬專趕金毛獅子。』故知是老豬之貨物也。」他三人說說笑笑,趕著豬羊,卻就望見虎口洞門。但見那門兒外:
\begin{quote}
周圍山遶翠,一脈氣連城。
峭壁扳青蔓,高崖掛紫荊。
鳥聲深樹匝,花影洞門迎。
不亞桃源洞,堪宜避世情。
\end{quote}

漸漸近於門口,又見一叢大大小小的雜項妖精,在那花樹之下頑耍。忽聽得八戒「呵呵」趕豬羊到時,都來迎接。便就捉豬的捉豬,捉羊的捉羊,一齊綑倒。早驚動裡面妖王,領十數個小妖,出來問道:「你兩個來了?買了多少豬羊?」行者道:「買了八口豬,七腔羊,共十五個牲口。豬銀該一十六兩,羊銀該九兩。前者領銀二十兩,仍欠五兩。這個就是客人,跟來找銀子的。」妖王聽說,即喚:「小的們,取五兩銀子,打發他去。」行者道:「這客人一則來找銀子,二來要看看嘉會。」那妖大怒,罵道:「你這個刁鑽兒憊𪬯!你買東西罷了,又與人說甚麼會不會?」八戒上前道:「主人公得了寶貝,誠是天下之奇珍,就教他看看怕怎的?」那怪咄的一聲道:「你這古怪也可惡!我這寶貝乃是玉華州城中得來的,倘這客人看了,去那州中傳說,說得人知,那王子一時來訪求,卻如之何?」行者道:「主公,這個客人乃乾方集後邊的人,去州許遠,又不是他城中人也,那裡去傳說?二則他肚裡也饑了,我兩個也未曾吃飯,家中有現成酒飯,賞他些吃了,打發他去罷。」說不了,有一小妖取了五兩銀子,遞與行者。行者將銀子遞與沙僧道:「客人,收了銀子,我與你進後面去吃些飯來。」

沙僧仗著膽,同八戒、行者進於洞內。到二層廠廳之上,只見正中間桌上,高高的供養著一柄九齒釘鈀,真個是光彩映目;東山頭靠著一條金箍棒,西山頭靠著一條降妖杖。那怪王隨後跟著道:「客人,那中間放光亮的就是釘鈀,你看便看,只是出去,千萬莫與人說。」沙僧點頭稱謝了。

噫!這正是:物見主,必定取。那八戒一生是個魯夯的人,他見了釘鈀,那裡與他敘甚麼情節,跑上去,拿下來,掄在手中,現了本相,丟了解數,望妖精劈臉就築。這行者、沙僧也奔至兩山頭各拿器械,現了原身,三兄弟一齊亂打。慌得那怪王急抽身閃過,轉入後邊,取一柄四明鏟,桿長鐏利,趕到天井中,支住他三般兵器,厲聲喝道:「你是甚人,敢弄虛頭,騙我寶貝?」行者罵道:「我把你這個賊毛團!你是認我不得。我們乃東土聖僧唐三藏的徒弟。因至玉華州倒換關文,蒙賢王教他三個王子拜我們為師,學習武藝,將我們寶貝作樣,打造如式兵器。因放在院中,被你這賊毛團夤夜入城偷來,倒說我弄虛頭騙你寶貝。不要走,就把我們這三件兵器各奉承你幾下嘗嘗。」那妖精就舉鏟來敵。這一場,從天井中鬥出前門,看他三僧攢一怪,好殺:
\begin{quote}
呼呼棒若風,滾滾鈀如雨。降妖杖舉滿天霞,四明鏟伸雲生綺。好似三仙煉大丹,火光彩晃驚神鬼。行者施威甚有能,妖精盜寶多無禮。天蓬八戒顯神通,大將沙僧英更美。弟兄合意運機謀,虎口洞中興鬥起。那怪豪強弄巧乖,四個英雄堪廝比。當時殺至日頭西,妖邪力軟難相抵。
\end{quote}

他們在豹頭山戰鬥多時,那妖精抵敵不住,向沙僧前喊一聲:「看鏟。」沙僧讓個身法躲過。妖精得空而走,向東南巽宮上,乘風飛去。八戒拽步要趕,行者道:「且讓他去。自古道:『窮寇勿追。』且只來斷他歸路。」八戒依言。

三人徑至洞口,把那百十個若大若小的妖精盡皆打死。原來都是些虎狼彪豹、馬鹿山羊。被大聖使個手法,將他那洞裡細軟物件並打死的雜項獸身與趕來的豬羊,通皆帶出。沙僧就取出乾柴放起火來、八戒使兩個耳朵搧風,把一個巢穴一時燒得乾淨。卻將帶出的諸物,即轉州城。

此時城門尚開,人家未睡。老王父子與唐僧俱在暴紗亭盼望,只見他們撲哩撲剌的丟下一院子死獸、豬羊及細軟物件。一齊叫道:「師父,我們已得勝回來也。」那殿下喏喏相謝。唐長老滿心歡喜。三個小王子跪拜於地,沙僧攙起道:「且莫謝,都近前看看那物件。」王子道:「此物俱是何來?」行者笑道:「那虎狼彪豹、馬鹿山羊,都是成精的妖怪。被我們取了兵器,打出門來。那老妖是個金毛獅子,他使一柄四明鏟,與我等戰到天晚,敗陣逃生,往東南上走了。我等不曾趕他,卻掃除他歸路,打殺這些群妖,搜尋他這些物件,帶將來的。」老王聽說,又喜又憂:喜的是得勝而回,憂的是那妖日後報仇。行者道:「殿下放心,我已慮之熟,處之當矣。一定與你掃除盡絕,方才起行,決不至貽害於後。我午間去時,撞見一個青臉紅毛的小妖送請書,我看他帖子上寫著:『明辰敬治餚酌,慶釘鈀嘉會,屈尊過山一敘。幸勿外,至感。右啟祖翁九靈元聖老大人尊前。』名字是『門下孫黃獅頓首百拜』。才子那妖精敗陣,必然向他祖翁處去會話,明辰斷然尋我們報仇,當情與你掃蕩乾淨。」老王稱謝了,擺上晚齋。師徒們齋畢,各歸寢處不題。

卻說那妖精果然向東南方奔到竹節山。那山中有一座洞天之處,喚名九曲盤桓洞。洞中的九靈元聖是他的祖翁。當夜足不停風,行至五更時分,到於洞口,敲門而進。小妖見了道:「大王,昨晚有青臉兒下請書,老爺留他住到今早,欲同他來赴你釘鈀會,你怎麼又絕早親來邀請?」妖精道:「不好說,不好說,會成不得了。」正說處,見青臉兒從裡邊走出道:「大王,你來怎的?老大王爺爺起來就同我去赴會哩。」妖精慌張張的,只是搖手不言。

少頃,老妖起來了,喚入。這妖精丟了兵器,倒身下拜,止不住腮邊淚落。老妖道:「賢孫,你昨日下柬,今早正欲來赴會,你又親來,為何發悲煩惱?」妖精叩頭道:「小孫前夜對月閑行,只見玉華州城中有光彩沖空。急去看時,乃是王府院中三般兵器放光:一件是九齒滲金釘鈀,一件是寶杖,一件是金箍棒。小孫即使神法攝來,立名『釘鈀嘉會』,著小的們買豬羊果品等物,設宴慶會,請祖爺爺賞之,以為一樂。昨差青臉來送柬之後,只見原差買豬羊的刁鑽兒等趕著幾個豬羊,又帶了一個販賣的客人來找銀子。他定要看看會去,是小孫恐他外面傳說,不容他看。他又說肚中饑餓,討些飯吃,因教他後邊吃飯。他走到裡邊,看見兵器,說是他的。三人就各搶去一件,現出原身:一個是毛臉雷公嘴的和尚,一個是長嘴大耳朵的和尚,一個是晦氣色臉的和尚。他都不分好歹,喊一聲亂打。是小孫急取四明鏟趕出與他相持,問是甚麼人敢弄虛頭。他道是東土大唐差往西天去的唐僧之徒弟,因過州城,倒換關文,被王子留住,習學武藝,將他這三件兵器作樣子打造,放在院內,被我偷來,遂此不忿相持。不知那三個和尚叫做甚名,卻俱有本事。小孫一人敵他三個不過,所以敗走祖爺處。望拔刀相助,拿那和尚報仇,庶見我祖愛孫之意也。」

老妖聞言,默想片時,笑道:「原來是他。我賢孫,你錯惹了他也。」妖精道:「祖爺知他是誰?」老妖道:「那長嘴大耳者,乃豬八戒;晦氣色臉者,乃沙和尚:這兩個猶可。那毛臉雷公嘴者,叫做孫行者。這個人其實神通廣大:五百年前曾大鬧天宮,十萬天兵也不曾拿得住。他專意尋人的,他便就是個搜山揭海、破洞攻城、闖禍的個都頭,你怎麼惹他?也罷,等我和你去,把那廝連玉華王子,都擒來替你出氣。」那妖精聽說,即叩頭而謝。

當時老妖點猱獅、雪獅、狻猊、白澤、伏狸、摶象諸孫,各執鋒利器械,黃獅引領,各縱狂風,徑至豹頭山界。只聞得煙火之氣撲鼻,又聞得有哭泣之聲。仔細看時,原來是刁鑽、古怪二人在那裡叫主公哭主公哩。妖精近前喝道:「你是真刁鑽兒,假刁鑽兒?」二怪跪倒,噙淚叩頭道:「我們怎是假的?昨日這早晚領了銀子去買豬羊,走至山西邊大路之上,見一個毛臉雷公嘴的和尚,他啐了我們一口,我們就腳軟口強,不能言語,不能移步。被他扳倒,把銀子搜了去,牌兒解了去。我兩個昏昏沉沉,直到此時才醒。及到家,見煙火未息,房舍盡皆燒了。又不見主公並大小頭目。故在此傷心痛哭。不知這火是怎生起的。」

那妖精聞言,止不住淚如泉湧,雙腳齊跌,喊聲振天,恨道:「禿廝!十分作惡,怎麼幹出這般毒事?把我洞府燒盡,美人燒死,家當老小一空。氣殺我也,氣殺我也!」老妖叫猱獅扯他過來道:「賢孫,事已至此,徒惱無益。且養全銳氣,到州城裡拿那和尚去。」那妖精猶不肯住哭,道:「老爺,我那們個山場,非一日治的,今被這禿廝盡毀,我卻要此命做甚的?」掙起來,往石崖上撞頭磕腦。被雪獅、猱獅等苦勸方止。

當時丟了此處,都奔州城。只聽得那風滾滾,霧騰騰,來得甚近。諕得那城外各關廂人等,拖男挾女,顧不得家私,都往州城中走,走入城門,將門閉了。有人報入王府中道:「禍事,禍事。」那王子唐僧等正在暴紗亭吃早齋,聽得人報禍事,卻出門來問。眾人道:「一群妖精,飛沙走石、噴霧掀風的來近城了。」老王大驚道:「怎麼好?」行者笑道:「都放心,都放心。這是虎口洞妖精昨日敗陣,往東南方去夥了那甚麼九靈元聖兒來也。等我同兄弟們出去。吩咐教關了四門,汝等點人夫看守城池。」那王子果傳令把四門閉了,點起人夫上城,他父子並唐僧在城樓上點劄,旌旗蔽日,炮火連天。行者三人,卻半雲半霧,出城迎敵。這正是:
\begin{quote}
失卻慧兵緣不謹,頓教魔起眾邪凶。
\end{quote}

畢竟不知這場勝敗如何,且聽下回分解。
