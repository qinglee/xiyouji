
\chapter{猿熟馬馴方脫殼 功成行滿見真如}

話表寇員外既得回生,復整理了幢旛鼓樂,僧道親友,依舊送行不題。

卻說唐僧四眾上了大路,果然西方佛地,與他處不同:見了些琪花瑤草,古柏蒼松;所過地方,家家向善,戶戶齋僧;每逢山下人修行,又見林間客誦經。

師徒們夜宿曉行,又經有六七日,忽見一帶高樓,幾層傑閣。真個是:
\begin{quote}
沖天百尺,聳漢凌空。低頭觀落日,引手摘飛星。豁達窗軒吞宇宙,嵯峨棟宇接雲屏。黃鶴信來秋樹老,彩鸞書到晚風清。此乃是靈宮寶闕,琳館珠庭。真堂談道,宇宙傳經。花向春來美,松臨雨過青。紫芝仙果年年秀,丹鳳儀翔萬感靈。
\end{quote}

三藏舉鞭遙指道:「悟空,好去處耶。」行者道:「師父,你在那假境界,假佛像處,倒強要下拜;今日到了這真境界,真佛像處,倒還不下馬,是怎的說?」三藏聞言,慌得翻身跳下來,已到了那樓閣門首。只見一個道童斜立在山門之前,應聲叫道:「那來者莫非東土取經人麼?」長老急整衣,擡頭觀看。見他:
\begin{quote}
身披錦衣,手搖玉麈。身披錦衣,寶閣瑤池常赴宴;手搖玉麈,丹臺紫府每揮塵。肘懸仙籙,足踏履鞋。飄然真羽士,秀麗實奇哉。煉就長生居勝境,修成永壽脫塵埃。聖僧不識靈山客,當年金頂大仙來。
\end{quote}

孫大聖認得他,即叫:「師父,此乃是靈山腳下玉真觀金頂大仙,他來接我們哩。」三藏方才醒悟,進前施禮。大仙笑道:「聖僧今年才到,我被觀音菩薩哄了。他十年前領佛金旨,向東土尋取經人,原說二三年就到我處。我年年等候,渺無消息,不意今年才相逢也。」三藏合掌道:「有勞大仙盛意,感激,感激。」遂此四眾牽馬挑擔,同入觀裡。卻又與大仙一一相見。即命看茶擺齋。又叫小童兒燒香湯與聖僧沐浴了,好登佛地。正是那:
\begin{quote}
功滿行完宜沐浴,煉馴本性合天真。
千辛萬苦今方息,九戒三皈始自新。
魔盡果然登佛地,災消故得見沙門。
洗塵滌垢全無染,反本還原不壞身。
\end{quote}

師徒們沐浴了,不覺天色將晚,就於玉真觀安歇。

次早,唐僧換了衣服,披上錦襴袈裟,戴了毘盧帽,手持錫杖,登堂拜辭大仙。大仙笑道:「昨日襤褸,今日鮮明,睹此相,真佛子也。」三藏拜別就行。大仙道:「且住,等我送你。」行者道:「不必你送,老孫認得路。」大仙道:「你認得的是雲路,聖僧還未登雲路,當從本路而行。」行者道:「這個講得是。老孫雖走了幾遭,只是雲來雲去,實不曾踏著此地。既有本路,還煩你送送。我師父拜佛心重,幸勿遲疑。」那大仙笑吟吟,攜著唐僧手,接引旃壇上法門。原來這條路不出山門,就自觀宇中堂,穿出後門便是。大仙指著靈山道:「聖僧,你看那半天中有祥光五色、瑞藹千重的,就是靈鷲高峰,佛祖之聖境也。」唐僧見了就拜。行者笑道:「師父,還不到拜處哩。常言道:『望山走倒馬。』離此鎮還有許遠,如何就拜?若拜到頂上,得多少頭磕是?」大仙道:「聖僧,你與大聖、天蓬、捲簾四位已到福地,望見靈山,我回去也。」三藏遂拜辭而去。

大聖引著唐僧等,徐徐緩步,登了靈山。不上五六里,見了一道活水,響潺潺滾浪飛流,約有八九里寬闊,四無人跡。三藏心驚道:「悟空,這路來得差了,敢莫大仙錯指了?此水這般寬闊,這般洶湧,又不見舟楫,如何可渡?」行者笑道:「不差,你看那壁廂不是一座大橋?要從那橋上行過去,方成正果哩。」長老等又近前看時,橋邊有一扁,扁上有「凌雲渡」三字,原來是一根獨木橋。正是:
\begin{quote}
遠看橫空如玉棟,近觀斷水一枯槎。
維河架海還容易,獨木單梁人怎蹅?
萬丈虹霓平臥影,千尋白練接天涯。
十分細滑渾難渡,除是神仙步彩霞。
\end{quote}

三藏心驚膽戰道:「悟空,這橋不是人走的,我們別尋路徑去來。」行者笑道:「正是路,正是路。」八戒慌了道:「這是路?那個敢走?水面又寬,波浪又湧,獨獨一根木頭,又細又滑,怎生動腳?」行者道:「你都站下,等老孫走個兒你看。」

好大聖,拽開步,跳上獨木橋,搖搖擺擺,須臾跑將過去,在那邊招呼道:「過來,過來。」唐僧搖手。八戒、沙僧咬指道:「難難難。」行者又從那邊跑過來,拉著八戒道:「獃子,跟我走,跟我走。」那八戒臥倒在地道:「滑滑滑,走不得,你饒我罷,讓我駕風霧過去。」行者按住道:「這是甚麼去處,許你駕風霧?必須從此橋上走過,方可成佛。」八戒道:「哥啊,佛做不成也罷,實是走不得。」他兩個在那橋邊扯扯拉拉的耍鬥,沙僧走去勸解,才撒脫了手。

三藏回頭,忽見那下溜中有一人撐一隻船來,叫道:「上渡,上渡。」長老大喜道:「徒弟,休得亂頑。那裡有隻渡船兒來了。」他三個跳起來站定,同眼觀看,那船兒來得至近,原來是一隻無底的船兒。行者火眼金睛,早已認得是接引佛祖,又稱為南無寶幢光王佛。行者卻不題破,只管叫:「這裡來,撐攏來。」霎時撐近岸邊,又叫:「上渡,上渡。」三藏見了,又心驚道:「你這無底的破船兒如何渡人?」佛祖道:「我這船:
\begin{quote}
鴻濛初判有聲名,幸我撐來不變更。
有浪有風還自穩,無終無始樂昇平。
六塵不染能歸一,萬劫安然自在行。
無底船兒難過海,今來古往渡群生。」
\end{quote}

孫大聖合掌稱謝道:「承盛意,接引吾師。——師父,上船去。他這船兒雖是無底卻穩,縱有風浪也不得翻。」長老還自驚疑,行者扠著膊子,往上一推。那師父踏不住腳,轂轆的跌在水裡,早被撐船人一把扯起,站在船上。師父還抖衣服,垛鞋腳,抱怨行者。行者卻引沙僧、八戒,牽馬挑擔,也上了船,都立在之上。那佛祖輕輕用力撐開,只見上溜頭泱下一個死屍。長老見了大驚。行者笑道:「師父莫怕。那個原來是你。」八戒也道:「是你,是你。」沙僧拍著手,也道:「是你,是你!」那撐船的打著號子,也說:「那是你,可賀,可賀。」他們三人也一齊聲相和。撐著船,不一時,穩穩當當的過了凌雲仙渡。三藏才轉身,輕輕的跳上彼岸。有詩為證。詩曰:
\begin{quote}
脫卻胎胞骨肉身,相親相愛是元神。
今朝行滿方成佛,洗淨當年六六塵。
\end{quote}

此誠所謂廣大智慧,登彼岸無極之法。

四眾上岸回頭,連無底船兒卻不知去向。行者方說是接引佛祖。三藏方才省悟,急轉身,反謝了三個徒弟。行者道:「兩不相謝,彼此皆扶持也。我等虧師父解脫,借門路修功,幸成了正果;師父也賴我等保護,秉教伽持,喜脫了凡胎。師父,你看這面前花草松篁、鸞鳳鶴鹿之勝境,比那妖邪顯化之處,孰美孰惡?何善何兇?」三藏稱謝不已。一個個身輕體快,步上靈山。早見那雷音古剎:
\begin{quote}
頂摩霄漢中,根接須彌脈。巧峰排列,怪石參差。懸崖下瑤草琪花,曲徑傍紫芝香蕙。仙猿摘果入桃林,卻似火燒金;白鶴棲松立枝頭,渾如煙捧玉。彩鳳雙雙,青鸞對對。彩鳳雙雙,向日一鳴天下瑞;青鸞對對,迎風耀舞世間稀。又見那黃森森金瓦疊鴛鴦,明幌幌花磚鋪瑪瑙。東一行,西一行,盡都是蕊宮珠闕;南一帶,北一帶,看不了寶閣珍樓。天王殿上放霞光,護法堂前噴紫焰。浮屠塔顯,優缽花香。正是地勝疑天別,雲閑覺晝長。紅塵不到諸緣盡,萬劫無虧大法堂。
\end{quote}

師徒們逍逍遙遙,走上靈山之頂。又見青松林下列優婆,翠柏叢中排善士。長老就便施禮,慌得那優婆塞、優婆夷、比丘僧、比丘尼合掌道:「聖僧且休行禮,待見了牟尼,卻來相敘。」行者笑道:「早哩,早哩,且去拜上位者。」

那長老手舞足蹈,隨著行者,直至雷音寺山門之外。那廂有四大金剛迎住道:「聖僧來耶?」三藏躬身道:「是,弟子玄奘到了。」答畢,就欲進門。金剛道:「聖僧少待,容稟過再進。」那金剛著一個轉山門報與二門上四大金剛,說唐僧到了;二門上又傳入三門上,說唐僧到了。三山門內原是打供的神僧,聞得唐僧到時,急至大雄殿下,報與如來至尊釋迦牟尼文佛說:「唐朝聖僧,到於寶山,取經來了。」佛爺爺大喜。即召聚八菩薩、四金剛、五百阿羅、三千揭諦、十一大曜、十八伽藍,兩行排列。卻傳金旨,召唐僧進。那裡邊一層一節,欽依佛旨,叫:「聖僧進來。」這唐僧循規蹈矩,同悟空、悟能、悟淨,牽馬挑擔,徑入山門。正是:
\begin{quote}
當年奮志奉欽差,領牒辭王出玉階。
清曉登山迎霧露,黃昏枕石臥雲霾。
挑禪遠步三千水,飛錫長行萬里崖。
念念在心求正果,今朝始得見如來。
\end{quote}

四眾到大雄寶殿殿前,對如來倒身下拜。拜罷,又向左右再拜。各各三匝已遍,復向佛祖長跪,將通關文牒奉上。如來一一看了,還遞與三藏。三藏頫顖作禮,啟上道:「弟子玄奘,奉東土大唐皇帝旨意,遙詣寶山,拜求真經,以濟眾生。望我佛祖垂恩,早賜回國。」如來方開憐憫之口,大發慈悲之心,對三藏言曰:「你那東土乃南贍部洲,只因天高地厚,物廣人稠,多貪多殺,多淫多誑,多欺多詐;不遵佛教,不向善緣,不敬三光,不重五穀;不忠不孝,不義不仁,瞞心昧己,大斗小秤,害命殺牲:造下無邊之孽,罪盈惡滿,致有地獄之災。所以永墮幽冥,受那許多碓搗磨舂之苦,變化畜類。有那許多披毛頂角之形,將身還債,將肉飼人。其永墮阿鼻,不得超昇者,皆此之故也。雖有孔氏在彼立下仁義禮智之教,帝王相繼,治有徒流絞斬之刑,其如愚昧不明,放縱無忌之輩何耶!我今有經三藏,可以超脫苦惱,解釋災愆。三藏:有法一藏,談天;有論一藏,說地;有經一藏,度鬼。共計三十五部,該一萬五千一百四十四卷。真是修真之徑,正善之門。凡天下四大部洲之天文、地理、人物、鳥獸、花木、器用、人事,無般不載。汝等遠來,待要全付與汝取去,但那方之人愚蠢村強,毀謗真言,不識我沙門之奧旨。」叫:「阿儺、伽葉,你兩個引他四眾到珍樓之下,先將齋食待他。齋罷,開了寶閣,將我那三藏經中,三十五部之內,各檢幾卷與他,教他傳流東土,永注洪恩。」

二尊者即奉佛旨,將他四眾領至樓下。看不盡那奇珍異寶,擺列無窮。只見那設供的諸神鋪排齋宴,並皆是仙品、仙餚、仙茶、仙果,珍饈百味,與凡世不同。師徒們頂禮了佛恩,隨心享用。其實是:
\begin{quote}
寶焰金光映目明,異香奇品更微精。
千層金閣無窮麗,一派仙音入耳清。
素味仙花人罕見,香茶異食得長生。
向來受盡千般苦,今日榮華喜道成。
\end{quote}

這番造化了八戒,便宜了沙僧。佛祖處正壽長生,脫胎換骨之饌,儘著他受用。二尊者陪奉四眾餐畢,卻入寶閣,開門登看。那廂有霞光瑞氣,籠罩千重;彩霧祥雲,遮漫萬道。經櫃上,寶篋外,都貼了紅簽,楷書著經卷名目。乃是:
\begin{quote}
《涅槃經》一部七百四十八卷
《菩薩經》一部一千二十一卷
《虛空藏經》一部四百卷
《首楞嚴經》一部一百一十卷
《恩意經大集》一部五十卷
《決定經》一部一百四十卷
《寶藏經》一部四十五卷
《華嚴經》一部五百卷
《禮真如經》一部九十卷
《大般若經》一部九百一十六卷
《光明經》一部三百卷
《未曾有經》一部一千一百一十卷
《維摩經》一部一百七十卷
《三論別經》一部二百七十卷
《金剛經》一部一百卷
《正法論經》一部一百二十卷
《佛本行經》一部八百卷
《五龍經》一部三十二卷
《菩薩戒經》一部一百一十六卷
《大集經》一部一百三十卷
《摩竭經》一部三百五十卷
《法華經》一部一百卷
《瑜伽經》一部一百卷
《寶常經》一部二百二十卷
《西天論經》一部一百三十卷
《僧祇經》一部一百五十七卷
《佛國雜經》一部一千九百五十卷
《起信論經》一部一千卷
《大智度經》一部一千八十卷
《寶威經》一部一千二百八十卷
《本閣經》一部八百五十卷
《正律文經》一部二百卷
《大孔雀經》一部二百二十卷
《維識論經》一部一百卷
《具舍論經》一部二百卷
\end{quote}

阿儺、伽葉引唐僧看遍經名,對唐僧道:「聖僧東土到此,有些甚麼人事送我們?快拿出來,好傳經與你去。」三藏聞言道:「弟子玄奘,來路迢遙,不曾備得。」二尊者笑道:「好好好,白手傳經繼世,後人當餓死矣。」行者見他講口扭捏,不肯傳經,他忍不住叫噪道:「師父,我們去告如來,教他自家來把經與老孫也。」阿儺道:「莫嚷,此是甚麼去處,你還撒野放刁?到這邊來接著經。」八戒、沙僧耐住了性子,勸住了行者,轉身來接,一卷卷收在包裡,馱在馬上,又綑了兩擔,八戒與沙僧挑著,卻來寶座前叩頭,謝了如來,一直出門。逢一位佛祖,拜兩拜;見一尊菩薩,拜兩拜。又到大門,拜了比丘僧、尼,優婆夷、塞,一一相辭,下山奔路不題。

卻說那寶閣上有一尊燃燈古佛,他在閣上暗暗的聽著那傳經之事,心中甚明:原是阿儺、伽葉將無字之經傳去。卻自笑云:「東土眾僧愚迷,不識無字之經,卻不枉費了聖僧這場跋涉?」問:「座邊有誰在此?」只見白雄尊者閃出。古佛分付道:「你可作起神威,飛星趕上唐僧,把那無字之經奪了,教他再來求取有字真經。」白雄尊者即駕狂風,滾離了雷音寺山門之外,大作神威。那陣好風,真個是:
\begin{quote}
佛前勇士,不比巽二風神;仙竅怒號,遠賽吹噓少女。這一陣,魚龍皆失穴,江海逆波濤。玄猿捧果難來獻,黃鶴回雲找舊巢。丹鳳清音鳴不美,錦雞喔運叫聲嘈。青松枝折,優缽花飄。翠竹竿竿倒,金蓮朵朵搖。鐘聲遠送三千里,經韻輕飛萬壑高。崖下奇花殘美色,路傍瑤草偃鮮苗。彩鸞難舞翅,白鹿躲山崖。蕩蕩異香漫宇宙,清清風氣徹雲霄。
\end{quote}

那唐長老正行間,忽聞香風滾滾,只道是佛祖之禎祥,未曾隄防。又聞得響一聲,半空中伸下一隻手來,將馬馱的經輕輕搶去。諕得個三藏搥胸叫喚,八戒滾地來追,沙和尚護守著經擔,孫行者急趕去如飛。那白雄尊者,見行者趕得將近,恐他棒頭上沒眼,一時間不分好歹,打傷身體,即將經包捽碎,拋落塵埃。行者見經包破落,又被香風吹得飄零,卻就按下雲頭顧經,不去追趕。那白雄尊者收風斂霧,回報古佛不題。

八戒去追趕,見經本落下,遂與行者收拾,背著來見唐僧。唐僧滿眼垂淚道:「徒弟呀,這個極樂世界,也還有兇魔欺害。」沙僧接了抱著的散經,打開看時,原來雪白,並無半點字跡。慌忙遞與三藏道:「師父,這一卷沒字。」行者又打開一卷看時,也無字。八戒打開一卷,也無字。三藏叫:「通打開來看看。」卷卷俱是白紙。長老短嘆長吁的道:「我東土人果是沒福,似這般無字的空本,取去何用?怎麼敢見唐王?誑君之罪,誠不容誅也。」行者早已知之,對唐僧道:「師父,不消說了,這就是阿儺、伽葉那廝問我要人事,沒有,故將此白紙本子與我們來了。快回去告在如來之前,問他掯財作弊之罪。」八戒嚷道:「正是,正是,告他去來。」四眾急急回山無好步,忙忙又轉上雷音。

不多時到於山門之外,眾皆拱手相迎,笑道:「聖僧是換經來的?」三藏點頭稱謝。眾金剛也不阻擋,讓他進去,直至大雄殿前。行者嚷道:「如來,我師徒們受了萬蜇千魔,千辛萬苦,自東土拜到此處,蒙如來分付傳經,被阿儺、伽葉掯財不遂,通同作弊,故意將無字的白紙本兒教我們拿去。我們拿他去何用?望如來敕治。」佛祖笑道:「你且休嚷。他兩個問你要人事之情,我已知矣。但只是經不可輕傳,亦不可以空取。向時眾比丘聖僧下山,曾將此經在舍衛國趙長者家與他誦了一遍,保他家生者安全,亡者超脫,只討得他三斗三升米粒黃金回來。我還說他們忒賣賤了,教後代兒孫沒錢使用。你如今空手來取,是以傳了白本。白本者,乃無字真經,倒也是好的。因你那東土眾生愚迷不悟,只可以此傳之耳。」即叫:「阿儺、伽葉,快將有字的真經,每部中各檢幾卷與他,來此報數。」

二尊者復領四眾,到珍樓寶閣之下,仍問唐僧要些人事。三藏無物奉承,即命沙僧取出紫金缽盂,雙手奉上道:「弟子委是窮寒路遙,不曾備得人事。這缽盂乃唐王親手所賜,教弟子持此沿路化齋。今特奉上,聊表寸心。萬望尊者將此收下,待回朝奏上唐王,定有厚謝。只是以有字真經賜下,庶不孤欽差之意,遠涉之勞也。」那阿儺接了,但微微而笑。被那些管珍樓的力士、管香積的庖丁、看閣的尊者,你抹他臉,我撲他背,彈指的,扭唇的,一個個笑道:「不羞,不羞,需索取經的人事。」須臾,把臉皮都羞皺了,只是拿著缽盂不放。伽葉卻才進閣檢經,一一查與三藏。三藏卻叫:「徒弟們,你們都好生看看,莫似前番。」他三人接一卷,看一卷,卻都是有字的。傳了五千零四十八卷,乃一藏之數。收拾齊整,馱在馬上;剩下的,還裝了一擔,八戒挑著。自己行囊,沙僧挑著。行者牽了馬,唐僧拿了錫杖,按一按毘盧帽,抖一抖錦袈裟,才喜喜歡歡,到我佛如來之前。正是那:
\begin{quote}
大藏真經滋味甜,如來造就甚精嚴。
須知玄奘登山苦,可笑阿儺卻愛錢。
先次未詳虧古佛,後來真實始安然。
至今得意傳東土,大眾均將雨露沾。
\end{quote}

阿儺、伽葉引唐僧來見如來。如來高陞蓮座,指令降龍、伏虎二大羅漢敲響雲磬,遍請三千諸佛、三千揭諦、八金剛、四菩薩、五百尊羅漢、八百比丘僧、大眾優婆塞、比丘尼、優婆夷,各天各洞,福地靈山,大小尊者聖僧,該坐的請登寶座,該立的侍立兩傍。一時間,天樂遙聞,仙音嘹喨,滿空中祥光疊疊,瑞氣重重,諸佛畢集,參見了如來。如來問:「阿儺、伽葉,傳了多少經卷與他?可一一報數。」二尊者即開報:「現付去唐朝:
\begin{quote}
《涅槃經》四百卷
《菩薩經》三百六十卷
《虛空藏經》二十卷
《首楞嚴經》三十卷
《恩意經大集》四十卷
《決定經》四十卷
《寶藏經》二十卷
《華嚴經》八十一卷
《禮真如經》三十卷
《大般若經》六百卷
《大光明經》五十卷
《未曾有經》五百五十卷
《維摩經》三十卷
《三論別經》四十二卷
《金剛經》一卷
《正法論經》二十卷
《佛本行經》一百一十六卷
《五龍經》二十卷
《菩薩戒經》六十卷
《大集經》三十卷
《摩竭經》一百四十卷
《法華經》十卷
《瑜伽經》三十卷
《寶常經》一百七十卷
《西天論經》三十卷
《僧祗經》一百一十卷
《佛國雜經》一千六百三十八卷
《起信論經》五十卷
《大智度經》九十卷
《寶威經》一百四十卷
《本閣經》五十六卷
《正律文經》十卷
《大孔雀經》十四卷
《維識論經》十卷
《具舍論經》十卷
\end{quote}

在藏總經共三十五部,各部中檢出五千零四十八卷,與東土聖僧傳留在唐。現俱收拾整頓於馬馱人擔之上,專等謝恩。」

三藏四眾拴了馬,歇了擔,一個個合掌躬身,朝上禮拜。如來對唐僧言曰:「此經功德,不可稱量。雖為我門之龜鑑,實乃三教之源流。若到你那南贍部洲,示與一切眾生,不可輕慢。非沐浴齋戒,不可開卷。寶之,重之。蓋此內有成仙了道之奧妙,有發明萬化之奇方也。」三藏叩頭謝恩,信受奉行,依然對佛祖遍禮三匝,承謹歸誠,領經而去。去到三山門,一一又謝了眾聖不題。

如來因打發唐僧去後,才散了傳經之會。傍又閃上觀世音菩薩合掌啟佛祖道:「弟子當年領金旨向東土尋取經之人,今已成功,共計得一十四年,乃五千零四十日,還少八日,不合藏數。望我世尊早賜聖僧回東轉西,須在八日之內庶完藏數。准弟子繳還金旨。」如來大喜道:「所言甚當,准繳金旨。」即叫八大金剛分付道:「汝等快使神威,駕送聖僧回東,把真經傳留,即引聖僧西回。須在八日之內,以完一藏之數,勿得遲違。」金剛隨即趕上唐僧,叫道:「取經的,跟我來。」唐僧等俱身輕體健,蕩蕩飄飄,隨著金剛,駕雲而起。這才是:
\begin{quote}
見性明心參佛祖,功完行滿即飛昇。
\end{quote}

畢竟不知回東土怎生傳授,且聽下回分解。
