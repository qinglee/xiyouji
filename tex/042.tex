
\chapter{大聖慇懃拜南海 觀音慈善縛紅孩}

話說那六健將出洞門,徑往西南上,依路而走。行者心中暗想道:「他要請老大王吃我師父,老大王斷是牛魔王。自老孫當年與他相會,真個意合情投,交遊甚厚。至如今我歸正道,他還是邪魔。雖則久別,還記得他模樣。且等老孫變作牛魔王,哄他一哄,看是何如。」

好行者,躲離了六個小妖,展開翅,飛向前邊,離小妖有十數里遠近,搖身一變,變作個牛魔王;拔下幾根毫毛,叫:「變!」即變作幾個小妖。在那山凹裡,駕鷹牽犬,搭弩張弓,充作打圍的樣子,等候那六健將。

那一夥廝拖廝扯正行時,忽然看見牛魔王坐在中間,慌得興烘掀、掀烘興撲的跪下道:「老大王爺爺在這裡也。」那雲裡霧、霧裡雲、急如火、快如風都是肉眼凡胎,那裡認得真假,也就一同跪倒,磕頭道:「爺爺,小的們是火雲洞聖嬰大王處差來,請老大王爺爺去吃唐僧肉,壽延千紀哩。」行者借口答道:「孩兒們起來,同我回家去,換了衣服來也。」小妖叩頭道:「望爺爺方便,不消回府罷。路程遙遠,恐我大王見責,小的們就此請行。」行者笑道:「好乖兒女。也罷,也罷,向前開路,我和你去來。」六怪抖擻精神,向前喝路。大聖隨後而來。

不多時,早到了本處。快如風、急如火撞進洞裡:「報大王:老大王爺爺來了。」妖王歡喜道:「你們卻中用,這等來的快。」即便叫各路頭目擺隊伍,開旗鼓,迎接老大王爺爺。滿洞群妖遵依旨令,齊齊整整,擺將出去。這行者昂昂烈烈,挺著胸脯,把身子抖了一抖,卻將那架鷹犬的毫毛都收回身上。拽開大步,徑走入門裡,坐在南面當中。紅孩兒當面跪下,朝上叩頭道:「父王,孩兒拜揖。」行者道:「孩兒免禮。」那妖王四大拜,拜畢,立於下手。行者道:「我兒,請我來有何事?」妖王躬身道:「孩兒不才,昨日獲得一人,乃東土大唐和尚。常聽得人講,他是一個十世修行之人,有人吃他一塊肉,壽似蓬瀛不老仙。愚男不敢自食,特請父王同享唐僧之肉,壽延千紀。」

行者聞言,打了個失驚道:「我兒,是那個唐僧?」妖王道:「是往西天取經的人也。」行者道:「我兒,可是孫行者師父麼?」妖王道:「正是。」行者擺手搖頭道:「莫惹他,莫惹他。別的還好惹,孫行者是那樣人哩。我賢郎你不曾會他,那猴子神通廣大,變化多端。他曾大鬧天宮,玉皇上帝差十萬天兵,佈下天羅地網,也不曾捉得他。你怎麼敢吃他師父?快早送出去還他,不要惹那猴子。他若打聽著你吃了他師父,他也不來和你打,他只把那金箍棒往山腰裡搠個窟窿,連山都掬了去。我兒,弄得你何處安身?教我倚靠何人養老?」

妖王道:「父王說那裡話,長他人志氣,滅孩兒的威風。那孫行者共有兄弟三人,領唐僧在我半山之中,被我使個變化,將他師父攝來。他與那豬八戒當時尋到我的門前,講甚麼攀親託熟之言,被我怒發沖天,與他交戰幾合,也只如此,不見甚麼高作。那豬八戒刺邪裡就來助戰,是孩兒吐出三昧真火,把他燒敗了一陣。慌得他去請四海龍王助雨,又不能滅得我三昧真火,被我燒了一個小發昏,連忙著豬八戒去請南海觀音菩薩。是我假變觀音,把豬八戒賺來,見吊在如意袋中,也要蒸他與眾小的們吃哩。那行者今早又來我的門首吆喝,我傳令教拿他,慌得他把包袱都丟下走了。卻才去請父王來看看唐僧活像,方可蒸與你吃,延壽長生不老也。」

行者笑道:「我賢郎啊,你只知有三昧火贏得他,不知他有七十二般變化哩。」妖王道:「憑他怎麼變化,我也認得,諒他決不敢進我門來。」行者道:「我兒,你雖然認得他,他卻不變大的,如狼犺大象,恐進不得你門;他若變作小的,你卻難認。」妖王道:「憑他變甚小的,我這裡每一層門上有四五個小妖把守,他怎生得入?」行者道:「你是不知。他會變蒼蠅、蚊子、虼蚤,或是蜜蜂、蝴蝶並蟭蟟蟲等項,又會變我模樣,你卻那裡認得?」妖王道:「勿慮,他就是鐵膽銅心,也不敢近我門來也。」

行者道:「既如此說,賢郎甚有手段,實是敵得他過,方來請我吃唐僧的肉,奈何我今日還不吃哩。」妖王道:「如何不吃?」行者道:「我近來年老,你母親常勸我作些善事。我想無甚作善,且持些齋戒。」妖王道:「不知父王是長齋,是月齋?」行者道:「也不是長齋,也不是月齋,喚做雷齋,每月只該四日。」妖王問:「是那四日?」行者道:「三辛逢初六。今朝是辛酉日,一則當齋,二來酉不會客。且等明日,我去親自刷洗蒸他,與兒等同享罷。」

那妖王聞言,心中暗想道:「我父王平日吃人為生,今活夠有一千餘歲,怎麼如今又吃起齋來了?想當初作惡多端,這三四日齋戒,那裡就積得過來?此言有假,可疑,可疑。」即抽身走出二門之下,叫六健將來問:「你們老大王是那裡請來的?」小妖道:「是半路請來的。」妖王道:「我說你們來的快。不曾到家麼?」小妖道:「是,不曾到家。」妖王道:「不好了,著了他假也,這不是老大王。」小妖一齊跪下道:「大王,自家父親也認不得?」妖王道:「觀其形容動靜都像,只是言語不像。只怕著了他假,吃了人虧。你們都要仔細:會使刀的刀要出鞘,會使槍的槍要磨明;會使棍的使棍,會使繩的使繩。待我再去問他,看他言語如何。若果是老大王,莫說今日不吃,明日不吃,便遲個月何妨?假若言語不對,只聽我哏的一聲,就一齊下手。」群魔各各領命訖。

這妖王復轉身到於裡面,對行者當面又拜。行者道:「孩兒,家無常禮,不須拜。但有甚話,只管說來。」妖王伏於地下道:「愚男一則請來奉獻唐僧之肉,二來有句話兒上請:我前日閑行,駕祥光,直至九霄空內,忽逢著祖庭道齡張先生。」行者道:「可是做天師的張道齡麼?」妖王道:「正是。」行者問曰:「有甚話說?」妖王道:「他見孩兒生得五官周正,三停平等,他問我是幾年那月那日那時出世。兒因年幼,記得不真。先生子平精熟,要與我推看五星。今請父王,正欲問此。倘或下次再得會他,好煩他推算。」行者聞言,坐在上面暗笑道:「好妖怪啞!老孫自歸佛果,保唐師父,一路上也捉了幾個妖精,不似這廝剋剝。他問我甚麼家長禮短、少米無柴的話說,我也好信口捏膿答他。他如今問我生年月日,我卻怎麼知道?」好猴王,也十分乖巧:巍巍端坐中間,也無一些兒懼色,面上反喜盈盈的笑道:「賢郎請起。我因年老,連日有事不遂心懷,把你生時果偶然忘了,且等到明日回家,問你母親便知。」

妖王道:「父王把我八個字時常不離口論說,說我有同天不老之壽,怎麼今日一旦忘了?豈有此理,必是假的。」哏的一聲,群妖槍刀簇擁,望行者沒頭沒臉的劄來。這大聖使金箍棒架住了,現出本像,對妖精道:「賢郎,你卻沒理那裡兒子好打爺的?」那妖王滿面羞慚,不敢回視。行者化金光,走出他的洞府。小妖道:「大王,孫行者走了。」妖王道:「罷罷罷,讓他走了罷,我吃他這一場虧也。且關了門,莫與他打話,只來刷洗唐僧,蒸吃便罷。」

卻說那行者搴著鐵棒,呵呵大笑,自澗那邊而來。沙僧聽見,急出林迎著道:「哥啊,這半日方回,如何這等哂笑,想救出師父來也?」行者道:「兄弟,雖不曾救得師父,老孫卻得個上風來了。」沙僧道:「甚麼上風?」行者道:「原來豬八戒被那怪假變觀音哄將回來,吊於皮袋之內。我欲設法救援,不期他著甚麼六健將去請老大王來吃師父肉。是老孫想著他老大王必是牛魔王,就變了他的模樣,充將進去,坐在中間。他叫父王,我就應他;他便叩頭,我就直受。著實快活,果然得了上風。」沙僧道:「哥啊,你便圖這般小便宜,恐師父性命難保。」行者道:「不須慮,等我去請菩薩來。」沙僧道:「你還腰疼哩。」行者道:「我不疼了。古人云:『人逢喜事精神爽。』你看著行李、馬匹,等我去。」沙僧道:「你置下仇了,恐他害我師父,你須快去快來。」行者道:「我來得快,只消頓飯時,就回來矣。」

好大聖,說話間躲離了沙僧,縱觔斗雲,徑投南海。在那半空裡,那消半個時辰,望見普陀山景。須臾,按下雲頭,直至落伽崖上。端肅正行,只見二十四路諸天迎著道:「大聖,那裡去?」行者作禮畢,道:「要見菩薩。」諸天道:「少停,容通報。」時有鬼子母諸天來潮音洞外報道:「菩薩得知:孫悟空特來參見。」菩薩聞報,即命進去。大聖斂衣皈命,捉定步,徑入裡邊,見菩薩倒身下拜。菩薩道:「悟空,你不領金蟬子西方求經去,卻來此何幹?」行者道:「上告菩薩:弟子保護唐僧前行,至一方,乃號山枯松澗火雲洞。有一個紅孩兒妖精,喚作聖嬰大王,把我師父攝去。是弟子與豬悟能等尋至門前,與他交戰。他放出三昧火來,我等不能取勝,救不出師父。急上東洋大海,請到四海龍王,施雨水,又不能勝火,把弟子都燻壞了,幾乎喪了殘生。」菩薩道:「既他是三昧火,神通廣大,怎麼去請龍王,不來請我?」行者道:「本欲來的,只是弟子被煙燻了,不能駕雲,卻教豬八戒來請菩薩。」菩薩道:「悟能不曾來啞。」行者道:「正是。未曾到得寶山,被那妖精假變做菩薩模樣,把豬八戒又賺入洞中,現吊在一個皮袋裡,也要蒸吃哩。」

菩薩聽說,心中大怒道:「那潑妖敢變我的模樣?」恨了一聲,將手中寶珠、淨瓶往海心裡撲的一摜。諕得那行者毛骨竦然,即起身侍立下面,道:「這菩薩火性不退,好是怪老孫說的話不好,壞了他的德行,就把淨瓶摜了,可惜,可惜。早知送了我老孫,卻不是一件大人事?」

說不了,只見那海當中翻波跳浪,鑽出個瓶來。原來是一個怪物馱著出來。行者仔細看那馱瓶的怪物,怎生模樣:
\begin{quote}
根源出處號幫泥,水底增光獨顯威。
世隱能知天地性,安藏偏曉鬼神機。
藏身一縮無頭尾,展足能行快似飛。
文王畫卦曾元卜,常納庭臺伴伏羲。
雲龍透出千般俏,號水推波把浪吹。
條條金線穿成甲,點點裝成彩玳瑁。
九宮八卦袍披定,散碎鋪遮綠燦衣。
生前好勇龍王幸,死後還馱佛祖碑。
要知此物名和姓,興風作浪惡烏龜。
\end{quote}

那龜馱著淨瓶,爬上崖邊,對菩薩點頭二十四點,權為二十四拜。行者見了,暗笑道:「原來是看瓶的。想是不見瓶,就問他要。」菩薩道:「悟空,你在下面說甚麼?」行者道:「沒說甚麼。」菩薩教:「拿上瓶來。」這行者即去拿瓶。唉!莫想拿得他動。好便似蜻蜓撼石柱,怎生搖得半分毫?行者上前跪下道:「菩薩,弟子拿不動。」菩薩道:「你這猴頭,只會說嘴。瓶兒你也拿不動,怎麼去降妖縛怪?」行者道:「不瞞菩薩說。平日拿得動,今日拿不動。想是吃了妖精虧,觔力弱了。」菩薩道:「常時是個空瓶;如今是淨瓶拋下海去,這一時間,轉過了三江五湖、八海四瀆、溪源潭洞之間,共借了一海水在裡面。你那裡有架海的斤量?此所以拿不動也。」行者合掌道:「是,弟子不知。」

那菩薩走上前,將右手輕輕的提起淨瓶,托在左手掌上。只見那龜點點頭,鑽下水去了。行者道:「原來是個養家看瓶的夯貨。」菩薩坐定道:「悟空,我這瓶中甘露水漿,比那龍王的私雨不同,能滅那妖精的三昧火。待要與你拿了去,你卻拿不動;待要著善財龍女與你同去,你卻又不是好心,專一只會騙人。你見我這龍女貌美,淨瓶又是個寶物,你假若騙了去,卻那有工夫又來尋你?你須是留些甚麼東西作當。」行者道:「可憐!菩薩這等多心。我弟子自秉沙門,一向不幹那樣事了。你教我留些當頭,卻將何物?我身上這件綿布直裰,還是你老人家賜的。這條虎皮裙子,能值幾個銅錢?這根鐵棒,早晚卻要護身。但只是頭上這個箍兒,是個金的,卻又被你弄了個方法兒長在我頭上,取不下來。你今要當頭,情願將此為當。你念個鬆箍兒咒,將此除去罷;不然,將何物為當?」菩薩道:「你好自在啊!我也不要你的衣服、鐵棒、金箍,只將你那腦後救命的毫毛拔一根與我作當罷。」行者道:「這毫毛也是你老人家與我的。但恐拔下一根,就拆破群了,又不能救我性命。」菩薩罵道:「你這猴子!你便一毛也不拔,教我這善財也難捨。」行者笑道:「菩薩,你卻也多疑。正是『不看僧面看佛面』。千萬救我師父一難罷。」那菩薩:
\begin{quote}
逍遙欣喜下蓮臺,雲步香飄上石崖。
只為聖僧遭障害,要降妖怪救回來。
\end{quote}

孫大聖十分歡喜,請觀音出了潮音仙洞。諸天大神都列在普陀巖上。菩薩道:「悟空,過海。」行者躬身道:「請菩薩先行。」菩薩道:「你先過去。」行者磕頭道:「弟子不敢在菩薩面前施展。若駕觔斗雲啊,掀露身體,恐菩薩怪我不敬。」菩薩聞言,即著善財龍女去蓮花池裡劈一瓣蓮花,放在石巖下邊水上。教行者:「你上那蓮花瓣兒,我渡你過海。」行者見了道:「菩薩,這花瓣兒又輕又薄,如何載得我起?這一屣翻跌下水去,卻不濕了虎皮裙?走了硝,天冷怎穿?」菩薩喝道:「你且上去看。」行者不敢推辭,捨命往上跳。果然先見輕小,到上面比海船還大三分。行者歡喜道:「菩薩,載得我了。」菩薩道:「既載得,如何不過去?」行者道:「又沒了篙、槳、篷、桅,怎生得過?」菩薩道:「不用。」只把他一口氣吹開吸攏,又著實一口氣吹過南洋苦海,得登彼岸。行者卻腳屣實地,笑道:「這菩薩賣弄神通,把老孫這等呼來喝去,全不費力也。」

那菩薩吩咐概眾諸天各守仙境,著善財龍女閉了洞門。他卻縱祥雲,躲離普陀巖,到那邊叫:「惠岸何在?」惠岸(乃托塔李天王第二個太子,俗名木叉是也。)乃菩薩親傳授的徒弟,不離左右,稱為護法惠岸行者。惠岸即對菩薩合掌伺候。菩薩道:「你快上界去,見你父王,問他借天罡刀來一用。」惠岸道:「師父用著幾何?」菩薩道:「全副都要。」

惠岸領命,即駕雲頭,徑入南天門裡,到雲樓宮殿,見父王下拜。天王見了,問:「兒從何來?」木叉道:「師父是孫悟空請來降妖,著兒拜上父王,將天罡刀借了一用。」天王即喚哪吒將刀取三十六把,遞與木叉。木叉對哪吒說:「兄弟,你回去多拜上母親:我事緊急,等送刀來再磕頭罷。」忙忙相別,按落祥光,徑至南海,將刀捧與菩薩。

菩薩接在手中,拋將去,念個咒語,只見那刀化作一座千葉蓮臺。菩薩縱身上去,端坐在中間。行者在傍暗笑道:「這菩薩省使儉用。那蓮花池裡有五色寶蓮臺,捨不得坐將來,卻又問別人去借。」菩薩道:「悟空,休言語,跟我來也。」卻才都駕著雲頭,離了海上。白鸚哥展翅前飛,孫大聖與惠岸隨後。

頃刻間,早見一座山頭。行者道:「這山就是號山了。從此處到那妖精門首,約摸有四百餘里。」菩薩聞言,即命住下祥雲,在那山頭上念一聲「唵」字咒語。只見那山左山右,走出許多神鬼,卻乃是本山土地眾神,都到菩薩寶蓮座下磕頭。菩薩道:「汝等俱莫驚張。我今來擒此魔王,你與我把這團圍打掃乾淨,要三百里遠近地方,不許一個生靈在地。將那窩中小獸,窟內雛蟲,都送在巔峰之上安生。」眾神遵依而退。須臾間,又來回覆。菩薩道:「既然乾淨,俱各回祠。」遂把淨瓶扳倒,唿喇喇傾出水來,就如雷響。真個是:
\begin{quote}
漫過山頭,沖開石壁。漫過山頭如海勢,沖開石壁似汪洋。黑霧漲天全水氣,滄波影日晃寒光。遍崖沖玉浪,滿海長金連。菩薩大展降魔法,袖中取出定身禪。化做落伽仙景界,真如南海一般般。秀蒲挺出曇花嫩,香草舒開貝葉鮮。紫竹幾竿鸚鵡歇,青松數簇鷓鴣喧。萬疊波濤蓮四野,只聞風吼水漫天。
\end{quote}

孫大聖見了,暗中讚嘆道:「果然是一個大慈大悲的菩薩!若老孫有此法力,將瓶兒望山一倒,管甚麼禽獸蛇蟲哩。」菩薩叫:「悟空,伸手過來。」行者即忙斂袖,將左手伸出。菩薩拔楊柳枝,蘸甘露,把手心裡寫一個「迷」字。教他:「捏著拳頭,快去與那妖精索戰,許敗不許勝。敗將來我這跟前,我自有法力收他。」

行者領命,返雲光,徑來至洞口。一隻手使拳,一隻手使棒,高叫道:「妖怪開門!」那些小妖又進去報道:「孫行者又來了。」妖王道:「緊關了門,莫睬他。」行者叫道:「好兒子!把老子趕在門外,還不開門?」小妖又報道:「孫行者罵出那話兒來了。」妖王只教:「莫睬他。」行者叫兩次,見不開門,心中大怒,舉鐵棒,將門一下,打了一個窟窿。慌得那小妖跌將進去道:「孫行者打破門了。」妖王見報幾次,又聽說打破前門,急縱身,跳將出去,挺長槍,對行者罵道:「這猴子,老大不識起倒。我讓你得些便宜,你還不知盡足,又來欺我。打破我門,你該個甚麼罪名?」行者道:「我兒,你趕老子出門,你該個甚麼罪名?」

那妖王羞怒,綽長槍,劈胸便刺;這行者,舉鐵棒,架隔相還。一番搭上手,鬥經四五個回合,行者捏著拳頭,拖著棒,敗將下來。那妖王立在山前道:「我要刷洗唐僧去哩。」行者道:「好兒子,天看著你哩。你來。」那妖精聞言,愈加嗔怒,喝一聲,趕到面前,挺槍又刺;這行者掄棒,又戰幾合,敗陣又走。那妖王罵道:「猴子,你在前有二三十合的本事,你怎麼如今正鬥時就要走了,何也?」行者笑道:「賢郎,老子怕你放火。」妖精道:「我不放火了,你上來。」行者道:「既不放火,走開些,好漢子莫在家門前打人。」那妖精不知是詐,真個舉槍又趕。行者拖了棒,放了拳頭。那妖王著了迷亂,只情追趕。前走的如流星過度,後走的如弩箭離弦。

不一時,望見那菩薩了。行者道:「妖精,我怕你了,你饒我罷。你如今趕至南海觀音菩薩處,怎麼還不回去?」那妖王不信,咬著牙,只管趕來。行者將身一幌,藏在那菩薩的神光影裡。這妖精見沒了行者。走近前,睜圓眼,對菩薩道:「你是孫行者請來的救兵麼?」菩薩不答應。妖王撚轉長槍,喝道:「咄!你是孫行者請來的救兵麼?」菩薩也不答應。妖精望菩薩劈心刺一槍來。那菩薩化道金光,徑走上九霄空內。行者跟定道:「菩薩,你好欺伏我罷了,那妖精再三問你,你怎麼推聾裝啞,不敢做聲,被他一槍搠走了,卻把那個蓮臺都丟下耶?」菩薩只教:「莫言語,看他再要怎的。」

此時行者與木叉俱在空中,並肩同看。只見那妖呵呵冷笑道:「潑猴頭,錯認了我也。他不知把我聖嬰當作個甚人,幾番家戰我不過,又去請個甚麼膿包菩薩來卻被我一槍,搠得無形無影去了,又把個寶蓮臺兒丟了。且等我上去坐坐。」好妖精,他也學菩薩,盤手盤腳的坐在當中。行者看見道:「好好好,蓮花臺兒好送人了。」菩薩道:「悟空,你又說甚麼?」行者道:「說甚?說甚?蓮臺送了人了。那妖精坐放臀下,終不得你還要哩?」菩薩道:「正要他坐哩。」行者道:「他的身軀小巧,比你還坐得穩當。」菩薩叫:「莫言語,且看法力。」

他將楊柳枝往下指定,叫一聲:「退!」只見那蓮臺花彩俱無,祥光盡散,原來那妖王坐在刀尖之上。即命木叉:「使降妖杵,把刀柄兒打打去來。」那木叉按下雲頭,將降魔杵如築牆一般,築了有千百餘下。那妖精穿通兩腿刀尖出,血注成汪皮肉開。好怪物,你看他咬著牙,忍著痛,且丟了長槍,用手將刀亂拔。行者卻道:「菩薩啊,那怪物不怕痛,還拔刀哩。」菩薩見了,喚上木叉:「且莫傷他生命。」卻又把楊柳枝垂下,念聲「唵」字咒語,那天罡刀都變做倒鬚鉤兒,狼牙一般,莫能褪得。那妖精卻才慌了,扳著刀尖,痛聲苦告道:「菩薩,我弟子有眼無珠,不識你廣大法力。千乞垂慈,饒我性命,再不敢恃惡,願入法門戒行也。」

菩薩聞言,卻與二行者、白鸚哥低下金光,到了妖精面前,問道:「你可受吾戒行麼?」妖王點頭滴淚道:「若饒性命,願受戒行。」菩薩道:「你可入我門麼?」妖王道:「果饒性命,願入法門。」菩薩道:「既如此,我與你摩頂受戒。」就袖中取出一把金剃頭刀兒,近前去,把那怪分頂剃了幾刀,剃作一個太山壓頂,與他留下三個頂搭,挽起三個窩角揪兒。行者在傍笑道:「這妖精大晦氣,弄得不男不女,不知像個甚麼東西。」菩薩道:「你今既受我戒,我卻也不慢你,稱你做善財童子,如何?」那妖點頭受持,只望饒命。菩薩卻用手一指,叫聲:「退!」撞的一聲,天罡刀都脫落塵埃,那童子身軀不損。

菩薩叫:「惠岸,你將刀送上天宮,還你父王,莫來接我,先到普陀巖會眾諸天等候。」那木叉領命,送刀上界,回海不題。

卻說那童子野性不定,見那腿疼處不疼,臀破處不破,頭挽了三個揪兒,他走去綽起長槍,望菩薩道:「那裡有甚真法力降我?原來是個掩樣術法兒。不受甚戒,看槍!」望菩薩劈臉刺來。恨得個行者掄鐵棒要打。菩薩只叫:「莫打,我自有懲治。」卻又袖中取出一個金箍兒來道:「這寶貝原是我佛如來賜我往東土尋取經人的金、緊、禁三個箍兒。緊箍兒先與你戴了;禁箍兒收了守山大神;這個金箍兒未曾捨得與人,今觀此怪無禮,與他罷。」好菩薩,將箍兒迎風一幌,叫聲:「變!」即變作五個箍兒,望童子身上拋了去,喝聲:「著!」一個套在他頭頂上,兩個套在他左右手上,兩個套在他左右腳上。菩薩道:「悟空,走開些,等我念念金箍兒咒。」行者慌了道:「菩薩呀,請你來此降妖,如何卻要咒我?」菩薩道:「這篇咒不是緊箍兒咒咒你的,是金箍兒咒咒那童子的。」行者卻才放心,緊隨左右,聽他念咒。菩薩捻著訣,默默的念了幾遍,那妖精搓耳揉腮,攢蹄打滾。正是:
\begin{quote}
一句能通遍沙界,廣大無邊法力深。
\end{quote}

畢竟不知那童子怎的皈依,且聽下回分解。
