
\chapter{老龍王拙計犯天條 魏丞相遺書託冥吏}

且不題光蕊盡職,玄奘修行。卻說長安城外涇河岸邊,有兩個賢人:一個是漁翁,名喚張稍;一個是樵子,名喚李定。他兩個是不登科的進士,能識字的山人。一日,在長安城裡賣了肩上柴,貨了籃中鯉,同入酒館之中,吃了半酣,各攜一瓶,順涇河岸邊,徐步而回。張稍道:「李兄,我想那爭名的,因名喪體;奪利的,為利亡身;受爵的,抱虎而眠;承恩的,袖蛇而走。算起來,還不如我們水秀山青,逍遙自在,甘淡薄,隨緣而過。」李定道:「張兄說得有理。但只是你那水秀,不如我的山青。」張稍道:「你山青不如我的水秀。有一《蝶戀花》詞為證。詞曰:
\begin{quote}
煙波萬里扁舟小,靜依孤篷,西施聲音遶。滌慮洗心名利少,閑攀蓼穗蒹葭草。
數點沙鷗堪樂道,柳岸蘆灣,妻子同歡笑。一覺安眠風浪消,無榮無辱無煩惱。」
\end{quote}

李定道:「你的水秀,不如我的山青。也有個《蝶戀花》詞為證。詞曰:
\begin{quote}
雲林一段松花滿,默聽鶯啼,巧舌如調管。紅瘦綠肥春正暖,倏然夏至光陰轉。
又值秋來容易換,黃花香,堪供玩。迅速嚴冬如指撚,逍遙四季無人管。」
\end{quote}

漁翁道:「你山青不如我水秀,受用些好物。有一《鷓鴣天》為證:
\begin{quote}
仙鄉雲水足生涯,擺櫓橫舟便是家。活剖鮮鱗烹綠鱉,旋蒸紫蟹煮紅蝦。
青蘆筍,水荇芽,菱角雞頭更可誇。嬌藕老蓮芹葉嫩,慈菇茭白鳥英花。」
\end{quote}

樵夫道:「你水秀不如我山青,受用些好物。亦有一《鷓鴣天》為證:
\begin{quote}
崔巍峻嶺接天涯,草舍茅庵是我家。醃臘雞鵝強蟹鱉,獐兔鹿勝魚蝦。
香椿葉,黃楝芽,竹筍山茶更可誇。紫李紅桃梅杏熟,甜梨酸棗木樨花。」
\end{quote}

漁翁道:「你山青真個不如我的水秀。又有《天仙子》一首:
\begin{quote}
一葉小舟隨所寓,萬疊煙波無恐懼。垂鉤撒網捉鮮鱗,沒醬膩,偏有味,老妻稚子團圓會。
魚多又貨長安市,換得香醪吃個醉。簑衣當被臥秋江,鼾鼾睡,無憂慮,不戀人間榮與貴。」
\end{quote}

樵子道:「你水秀還不如我的山青。也有《天仙子》一首:
\begin{quote}
茆舍數椽山下蓋,松竹梅蘭真可愛。穿林越嶺覓乾柴,沒人怪,從我賣,或少或多憑世界。
將錢沽酒隨心快,瓦缽磁甌殊自在。酕醄醉了臥松陰,無掛礙,無利害,不管人間興與敗。」
\end{quote}

漁翁道:「李兄,你山中不如我水上生意快活。有一《西江月》為證:
\begin{quote}
紅蓼花繁映月,黃蘆葉亂搖風。碧天清遠楚江空,牽攪一潭星動。
入網大魚作隊,吞鉤小鱖成叢。得來烹煮味偏濃,笑傲江湖打鬨。」
\end{quote}

樵夫道:「張兄,你水上還不如我山中的生意快活。亦有《西江月》為證:
\begin{quote}
敗葉枯藤滿路,破梢老竹盈山。女蘿乾葛亂牽攀,折取收繩殺擔。
蟲蛀空心榆柳,風吹斷頭松柟。採來堆積備冬寒,換酒換錢從俺。」
\end{quote}

漁翁道:「你山中雖可比過,還不如我水秀的幽雅。有一《臨江仙》為證:
\begin{quote}
潮落旋移孤艇去,夜深罷棹歌來。簑衣殘月甚幽哉,宿鷗驚不起,天際彩雲開。
困臥蘆洲無個事,三竿日上還捱。隨心儘意自安排,朝臣寒待漏,怎似我寬懷。」
\end{quote}

樵夫道:「你水秀的幽雅,還不如我山青更幽雅。亦有《臨江仙》可證:
\begin{quote}
蒼徑秋高拽斧去,晚涼擡擔回來。野花插鬢更奇哉,撥雲尋路出,待月叫門開。
稚子山妻欣笑接,草床木枕攲捱。蒸梨炊黍旋鋪排,甕中新釀熟,真個壯幽懷。」
\end{quote}

漁翁道:「這都是我兩個生意,贍身的勾當,你卻沒有我閑時節的好處。有詩為證。詩曰:
\begin{quote}
閑看蒼天白鶴飛,停舟溪畔掩蒼扉。
倚篷教子搓鉤線,罷棹同妻晒網圍。
性定果然如浪靜,身安自是覺風微。
綠簑青笠隨時著,勝掛朝中紫綬衣。」
\end{quote}

樵夫道:「你那閑時又不如我的閑時好也。亦有詩為證。詩曰:
\begin{quote}
閑觀縹緲白雲飛,獨坐茅庵掩竹扉。
無事訓兒開卷讀,有時對客把棋圍。
喜來策杖歌芳徑,興到攜琴上翠微。
草履麻絛粗布被,心寬強似著羅衣。」
\end{quote}

張稍道:「李定,我兩個真是微吟可相狎,不須檀板共金樽。但散道詞章,不為稀罕。且各聯幾句,看我們漁樵攀話何如?」李定道:「張兄言之最妙。請兄先吟。」
\begin{quote}
「舟停綠水煙波內,家住深山曠野中。
偏愛溪橋春水漲,最憐岩岫曉雲蒙。
龍門鮮鯉時烹煮,蟲蛀乾柴日燎烘。
釣網多般堪贍老,擔繩二事可容終。
小舟仰臥觀飛雁,草徑斜欹聽唳鴻。
口舌場中無我分,是非海內少吾蹤。
溪邊掛晒繒如錦,石上重磨斧似鋒。
秋月暉暉常獨釣,春山寂寂沒人逢。
魚多換酒同妻飲,柴剩沽壺共子叢。
自唱自斟隨放蕩,長歌長嘆任顛風。
呼兄喚弟邀船夥,挈友攜朋聚野翁。
行令猜拳頻遞盞,拆牌道字漫傳鐘。
烹蝦煮蟹朝朝樂,炒鴨爊雞日日豐。
愚婦煎茶情散淡,山妻造飯意從容。
曉來舉杖淘輕浪,日出擔柴過大沖。
雨後披簑擒活鯉,風前弄斧伐枯松。
潛蹤避世妝痴蠢,隱姓埋名作啞聾。」
\end{quote}

張稍道:「李兄,我才僭先起句,今到我兄,也先起一聯,小弟亦當續之。」
\begin{quote}
「風月佯狂山野漢,江湖寄傲老餘丁。
清閑有分隨瀟灑,口舌無聞喜太平。
月夜身眠茅屋穩,天昏體蓋箬簑輕。
忘情結識松梅友,樂意相交鷗鷺盟。
名利心頭無算計,干戈耳畔不聞聲。
隨時一酌香醪酒,度日三餐野菜羹。
兩束柴薪為活計,一竿鉤線是營生。
閑呼稚子磨鋼斧,靜喚憨兒補舊繒。
春到愛觀楊柳綠,時融喜看荻蘆青。
夏天避暑修新竹,六月乘涼摘嫩菱。
霜降雞肥常日宰,重陽蟹壯及時烹。
冬來日上還沉睡,數九天高自不寒。
八節山中隨放性,四時湖裡任陶情。
採薪自有仙家興,垂釣全無世俗形。
門外野花香豔豔,船頭綠水浪平平。
身安不說三公位,性定強如十里城。
十里城高防閫令,三公位顯聽宣聲。
樂山樂水真是罕,謝天謝地謝神明。」
\end{quote}

他二人既各道詞章,又相聯詩句。行到那分路去處,躬身作別。張稍道:「李兄啊,途中保重,上山仔細看虎。假若有些凶險,正是『明日街頭少故人』。」李定聞言,大怒道:「你這廝憊𪬯!好朋友也替得生死,你怎麼咒我?我若遇虎遭害,你必遇浪翻江。」張稍道:「我永世也不得翻江。」李定道:「『天有不測風雲,人有暫時禍福。』你怎麼就保得無事?」張稍道:「李兄,你雖這等說,你還沒捉摸;不若我的生意有捉摸,定不遭此等事。」李定道:「你那水面上營生,極凶極險,隱隱暗暗,有甚麼捉摸?」張稍道:「你是不曉得。這長安城裡,西門街上,有一個賣卦的先生。我每日送他一尾金色鯉,他就與我袖傳一課,依方位,百下百著。今日我又去買卦,他教我在涇河灣頭東邊下網,西岸拋鉤,定獲滿載魚蝦而歸。明日上城來,賣錢沽酒,再與老兄相敘。」二人從此敘別。

這正是:「路上說話,草裡有人。」原來這涇河水府有一個巡水的夜叉,聽見了百下百著之言,急轉水晶宮,慌忙報與龍王道:「禍事了!禍事了!」龍王問:「有甚禍事?」夜叉道:「臣巡水去到河邊,只聽得兩個漁、樵攀話,相別時,言語甚是利害。那漁翁說:長安城裡,西門街上,有個賣卦先生,算得最準。他每日送他鯉魚一尾,他就袖傳一課,教他百下百著。若依此等算準,卻不將水族盡情打了?何以壯觀水府,何以躍浪翻波,輔助大王威力?」龍王甚怒,急提了劍,就要上長安城,誅滅這賣卦的。旁邊閃過龍子、龍孫、蝦臣、蟹士、鰣軍師、鱖少卿、鯉太宰,一齊啟奏道:「大王且息怒。常言道:『過耳之言,不可聽信。』大王此去,必有雲從,必有雨助,恐驚了長安黎庶,上天見責。大王隱顯莫測,變化無方,但只變一秀士,到長安城內訪問一番。果有此輩,容加誅滅不遲;若無此輩,可不是妄害他人也?」

龍王依奏,遂棄寶劍,也不興雲雨,出岸上,搖身一變,變作一個白衣秀士,真個:
\begin{quote}
丰姿英偉,聳壑昂霄。步履端祥,循規蹈矩。語言遵孔孟,禮貌體周文。身穿玉色羅襴服,頭戴逍遙一字巾。
\end{quote}

上路來,拽開雲步,徑到長安城西門大街上。只見一簇人,擠擠雜雜,鬧鬧哄哄。內有高談闊論的道:「屬龍的本命,屬虎的相沖。寅辰巳亥,雖稱合局,但怕的是日犯歲君。」龍王聞言,情知是賣卜之處。走上前,分開眾人,望裡觀看。只見:
\begin{quote}
四壁珠璣,滿堂綺繡。寶鴨香無斷,磁瓶水恁清。兩邊羅列王維畫,座上高懸鬼谷形。端溪硯,金煙墨,相襯著霜毫大筆;火珠林,郭璞數,謹對了臺政新經。六爻熟諳,八卦精通。能知天地理,善曉鬼神情。一槃子午安排定,滿腹星辰佈列清。真個那未來事,過去事,觀如月鏡;幾家興,幾家敗,鑑若神明。知凶定吉,斷死言生。開談風雨迅,下筆鬼神驚。招牌有字書名姓,神課先生袁守誠。
\end{quote}

此人是誰?原來是當朝欽天監臺正先生袁天罡的叔父,袁守誠是也。那先生果然相貌稀奇,儀容秀麗;名揚大國,術冠長安。龍王入門來,與先生相見。禮畢,請龍上坐,童子獻茶。先生問曰:「公來問何事?」龍王曰:「請卜天上陰晴事如何。」先生即袖傳一課,斷曰:「雲迷山頂,霧罩林梢。若占雨澤,準在明朝。」龍王曰:「明日甚時下雨?雨有多少尺寸?」先生道:「明日辰時布雲,巳時發雷,午時下雨,未時雨足,共得水三尺三寸零四十八點。」龍王笑曰:「此言不可作戲。如是明日有雨,依你斷的時辰、數目,我送課金五十兩奉謝;若無雨,或不按時辰、數目,我與你實說:定要打壞你的門面,扯碎你的招牌,即時趕出長安,不許在此惑眾。」先生忻然而答:「這個一定任你。請了,請了。明朝雨後來會。」

龍王辭別,出長安,回水府。大小水神接著,問曰:「大王訪那賣卦的如何?」龍王道:「有,有,有。但是一個掉嘴口討春的先生。我問他幾時下雨,他就說明日下雨。問他甚麼時辰,甚麼雨數,他就說辰時布雲,巳時發雷,午時下雨,未時雨足,得水三尺三寸零四十八點。我與他打了個賭賽:若果如他言,送他謝金五十兩;如略差些,就打破他門面,趕他起身,不許在長安惑眾。」眾水族笑曰:「大王是八河都總管,司雨大龍神,有雨無雨,惟大王知之。他怎敢這等胡言?那賣卦的定是輸了,定是輸了。」

此時龍子、龍孫與那魚卿、蟹士正歡笑談此事未畢,只聽得半空中叫:「涇河龍王接旨。」眾擡頭上看,是一個金衣力士,手擎玉帝敕旨,徑投水府而來。慌得龍王整衣端肅,焚香接了旨。金衣力士回空而去。龍王謝恩,拆封看時,上寫著:
\begin{quote}
敕命八河總,驅雷掣電行;
明朝施雨澤,普濟長安城。
\end{quote}

旨意上時辰、數目,與那先生判斷者毫髮不差。諕得那龍王魂飛魄散。少頃甦醒,對眾水族曰:「塵世上有此靈人,真個是能通天地理,卻不輸與他啊!」鰣軍師奏云:「大王放心。要贏他有何難處?臣有小計,管教滅那廝的口嘴。」龍王問計,軍師道:「行雨差了時辰,少些點數,就是那廝斷卦不準,怕不贏他?那時捽碎招牌,趕他跑路,果何難也?」龍王依他所奏,果不擔憂。

至次日,點札風伯、雷公、雲童、電母,直至長安城九霄空上。他挨到那巳時方布雲,午時發雷,未時落雨,申時雨止,卻只得三尺零四十點。改了他一個時辰,剋了他三寸八點。雨後發放眾將班師。他又按落雲頭,還變作白衣秀士,到那西門裡大街上,撞入袁守誠卦舖,不容分說,就把他招牌、筆、硯等一齊捽碎。那先生坐在椅上,公然不動。這龍王又掄起門板便打,罵道:「這妄言禍福的妖人,擅惑眾心的潑漢!你卦又不靈,言又狂謬。說今日下雨的時辰、點數俱不相對。你還危然高坐,趁早去,饒你死罪!」守誠猶公然不懼分毫,仰面朝天冷笑道:「我不怕,我不怕。我無死罪,只怕你倒有個死罪哩。別人好瞞,只是難瞞我也。我認得你,你不是秀士,乃是涇河龍王。你違了玉帝敕旨,改了時辰,剋了點數,犯了天條。你在那剮龍臺上,恐難免一刀,你還在此罵我?」龍王見說,心驚膽戰,毛骨悚然。急丟了門板,整衣伏禮,向先生跪下道:「先生休怪。前言戲之耳,豈知弄假成真,果然違犯天條,奈何?望先生救我一救;不然,我死也不放你。」守誠曰:「我救你不得,只是指條生路與你投生便了。」龍曰:「願求指教。」先生曰:「你明日午時三刻,該赴人曹官魏徵處聽斬。你果要性命,須當急急去告當今唐太宗皇帝方好。那魏徵是唐王駕下的丞相,若是討他個人情,方保無事。」

龍王聞言,拜辭含淚而去。不覺紅日西沉,太陰星上。但見:
\begin{quote}
煙凝山紫歸鴉倦,遠路行人投旅店。渡頭新雁宿汀沙,銀河現,催更籌,孤村燈火光無焰。風裊爐煙清道院,蝴蝶夢中人不見。月移花影上欄杆,星光亂,漏聲換,不覺深沉夜已半。
\end{quote}

這涇河龍王也不回水府,只在空中。等到子時前後,收了雲頭,斂了霧角,徑來皇宮門首。此時唐王正夢出宮門之外,步月花陰。忽然龍王變作人相,上前跪拜,口叫:「陛下,救我,救我。」太宗云:「你是何人?朕當救你。」龍王云:「陛下是真龍,臣是業龍。臣因犯了天條,該陛下賢臣人曹官魏徵處斬,故來拜求,望陛下救我一救。」太宗曰:「既是魏徵處斬,朕可以救你,你放心前去。」龍王歡喜,叩謝而去。

卻說那太宗夢醒後,念念在心。早已至五鼓三點,太宗設朝,聚集兩班文武官員。但見那:
\begin{quote}
煙籠鳳闕,香藹龍樓。光搖丹扆動,雲拂翠華流。君臣相契同堯舜,禮樂威嚴近漢周。侍臣燈,宮女扇,雙雙映彩;孔雀屏,麒麟殿,處處光浮。山呼萬歲,華祝千秋。靜鞭三下響,衣冠拜冕旒。宮花燦爛天香襲,堤柳輕柔御樂謳。珍珠簾,翡翠簾,金鉤高控;龍鳳扇,山河扇,寶輦停留。文官英秀,武將抖擻。御道分高下,丹墀列品流。金章紫綬乘三象,地久天長萬萬秋。
\end{quote}

眾官朝賀已畢,各各分班。唐王閃鳳目龍睛,一一從頭觀看,只見那文官內是房玄齡、杜如晦、徐世勣、許敬宗、王珪等,武官內是馬三寶、段志玄、殷開山、程咬金、劉洪紀、胡敬德、秦叔寶等,一個個威儀端肅,卻不見魏徵丞相。唐王召徐世勣上殿道:「朕夜間得一怪夢:夢見一人,迎面拜謁,口稱是涇河龍王,犯了天條,該人曹官魏徵處斬,拜告寡人救他,朕已許諾。今日班前獨不見魏徵,何也?」世勣對曰:「此夢告準。須喚魏徵來朝,陛下不要放他出門,過此一日,可救夢中之龍。」唐王大喜,即傳旨,著當駕官宣魏徵入朝。

卻說魏徵丞相在府,夜觀乾象,正爇寶香,只聞得九霄鶴唳,卻是天差仙使,捧玉帝金旨一道,著他午時三刻,夢斬涇河老龍。這丞相謝了天恩,齋戒沐浴,在府中試慧劍,運元神,故此不曾入朝。一見當駕官齎賫來宣,惶懼無任;又不敢違遲君命,只得急急整衣束帶,同旨入朝,在御前叩頭請罪。唐王道:「赦卿無罪。」那時諸臣尚未退朝,至此,卻命捲簾散朝。獨留魏徵,宣上金鑾,召入便殿,先議論安邦之策,定國之謀。將近巳末午初時候,卻命宮人:「取過大棋來,朕與賢卿對弈一局。」眾嬪妃隨取棋枰,鋪設御案。魏徵謝了恩,即與唐王對弈,一遞一著,擺開陣勢。正合《爛柯經》云:
\begin{quote}
博弈之道,貴乎嚴謹。高者在腹,下者在邊,中者在角,此棋家之常法。法曰:「寧輸一子,不失一先。」擊左則視右,攻後則瞻前。有先而後,有後而先。兩生勿斷,皆活勿連。闊不可太疏,密不可太促。與其戀子以求生,不若棄之而取勝;與其無事而獨行,不若固之而自補。彼眾我寡,先謀其生;我眾彼寡,務張其勢。善勝者不爭,善陣者不戰;善戰者不敗,善敗者不亂。夫棋始以正合,終以奇勝。凡敵無事而自補者,有侵絕之意;棄小而不救者,有圖大之心;隨手而下者,無謀之人;不思而應者,取敗之道。《詩》云:「惴惴小心,如臨于谷。」此之謂也。
\end{quote}

詩曰:
\begin{quote}
棋盤為地子為天,色按陰陽造化全。
下到玄微通變處,笑誇當日爛柯仙。
\end{quote}

君臣兩個對弈,此棋正下到午時三刻,一盤殘局未終,魏徵忽然俯伏在案邊,鼾鼾盹睡。太宗笑曰:「賢卿真是匡扶社稷之心勞,創立江山之力倦,所以不覺盹睡。」太宗任他睡著,更不呼喚。不多時,魏徵醒來,俯伏在地道:「臣該萬死,臣該萬死!卻才倦困,不知所為,望陛下赦臣慢君之罪。」太宗道:「卿有何慢罪?且起來,拂退殘棋,與卿從新更著。」

魏徵謝了恩,卻才撚子在手,忽聽得朝門外大呼小叫。原來是秦叔寶、徐茂公等,將著一個血淋的龍頭,擲在帝前,啟奏道:「陛下,海淺河枯曾有見,這般異事卻無聞。」太宗與魏徵起身道:「此物何來?」叔寶、茂公道:「千步廊南,十字街上,雲端裡落下這顆龍頭,微臣不敢不奏。」唐王驚問魏徵:「此是何說?」魏徵轉身叩頭道:「是臣才一夢斬的。」唐王聞言,大驚道:「賢卿盹睡之時,又不曾見動身動手,又無刀劍,如何卻斬此龍?」魏徵奏道:「主公,臣的身在君前,夢離陛下。身在君前對殘局,合眼朦朧;夢離陛下乘瑞雲,出神抖擻。那條龍在剮龍臺上,被天兵將綁縛其中。是臣道:『你犯天條,合當死罪。我奉天命,斬汝殘生。』龍聞哀苦,臣抖精神。龍聞哀苦,伏爪收鱗甘受死;臣抖精神,撩衣進步舉霜鋒。扢扠一聲刀過處,龍頭因此落虛空。」

太宗聞言,心中悲喜不一。喜者,誇獎魏徵好臣,朝中有此豪傑,愁甚江山不穩?悲者,謂夢中曾許救龍,不期竟致遭誅。只得強打精神,傳旨著叔寶將龍頭懸掛市曹,曉諭長安黎庶。一壁廂賞了魏徵,眾官散訖。

當晚回宮,心中只是憂悶。想那夢中之龍,哭啼啼哀告求生,豈知無常,難免此患。思念多時,漸覺神魂倦怠,身體不安。當夜二更時分,只聽得宮門外有號泣之聲,太宗愈加驚恐。正朦朧睡間,又見那涇河龍王手提著一顆血淋淋的首級,高叫:「唐太宗,還我命來!還我命來!你昨夜滿口許諾救我,怎麼天明時反宣人曹官來斬我?你出來,你出來,我與你到閻君處折辨折辨。」他扯住太宗,再三嚷鬧不放。太宗箝口難言,只掙得汗流遍體。

正在那難分難解之時,只見正南上香雲繚繞,彩霧飄飄,有一個女真人上前,將楊柳枝用手一擺,那沒頭的龍悲悲啼啼,徑往西北而去。原來這是觀音菩薩領佛旨,上東土尋取經人,住此長安城都土地廟裡,夜聞鬼泣神號,特來喝退業龍,救脫皇帝。那龍徑到陰司地獄具告不題。

卻說太宗甦醒回來,只叫:「有鬼!有鬼!」慌得那三宮皇后、六院嬪妃,與近侍太監,戰兢兢,一夜無眠。

不覺五更三點,那滿朝文武多官,都在朝門外候朝。等到天明,猶不見臨朝,諕得一個個驚懼躊躇。及日上三竿,方有旨意出來道:「朕心不快,眾官免朝。」不覺倏五七日,眾官憂惶,都正要撞門見駕問安,只見太后有旨,召醫官入宮用藥。眾人在朝門外等候討信。少時,醫官出來,眾問何疾。醫官道:「皇上脈氣不正,虛而又數,狂言見鬼。又診得十動一代,五臟無氣,恐不諱只在七日之內矣。」眾官聞言,大驚失色。

正愴惶間,又聽得太后有旨宣徐茂公、護國公、尉遲恭見駕。三公奉旨,急入到分宮樓下。拜畢,太宗正色強言道:「賢卿,寡人十九歲領兵,南征北伐,東擋西除,苦歷數載,更不曾見半點邪祟,今日卻反見鬼。」尉遲恭道:「創立江山,殺人無數,何怕鬼乎?」太宗道:「卿是不信。朕這寢宮門外,入夜就拋磚弄瓦,鬼魅呼號,著然難處。白日猶可,昏夜難禁。」叔寶道:「陛下寬心,今晚臣與敬德把守宮門,看有甚麼鬼祟。」太宗准奏。茂公謝恩而出。

當日天晚,各取披掛,他兩個介冑整齊,執金瓜、鉞斧,在宮門外把守。好將軍!你看他怎生打扮:
\begin{quote}
頭戴金盔光爍爍,身披鎧甲龍鱗。護心寶鏡幌祥雲,獅蠻收緊扣,繡帶彩霞新。這一個鳳眼朝天星斗怕,那一個環睛映電月光浮。他本是英雄豪傑舊勳臣,只落得千年稱戶尉,萬古作門神。
\end{quote}

二將軍侍立門傍,一夜天曉,更不曾見一點邪祟。是夜,太宗在宮,安寢無事。曉來宣二將軍,重重賞勞道:「朕自得疾,數日不能得睡,今夜仗二將軍威勢甚安。卿且請出安息安息,待晚間再一護衛。」二將謝恩而出。

遂此二三夜把守俱安。只是御膳減損,病轉覺重。太宗又不忍二將辛苦,又宣叔寶、敬德與杜、房諸公入宮,吩咐道:「這兩日朕雖得安,卻只難為秦、胡二將軍徹夜辛苦。朕欲召巧手丹青,傳二將軍真容,貼於門上,免得勞他。如何?」眾臣即依旨,選兩個會寫真的,著胡、秦二公依前披掛,照樣畫了,貼在門上。夜間也即無事。

如此二三日,又聽得後宰門乒乓乒乓,磚瓦亂響。曉來即宣眾臣曰:「連日前門幸喜無事,今夜後門又響,卻不又驚殺寡人也。」茂公進前奏道:「前門不安,是敬德、叔寶護衛;後門不安,該著魏徵護衛。」太宗准奏,又宣魏徵今夜把守後門。徵領旨,當夜結束整齊,提著那誅龍的寶劍,侍立在後宰門前,真個的好英雄也。他怎生打扮:
\begin{quote}
熟絹青巾抹額,錦袍玉帶垂腰。兜風氅袖采霜飄,壓賽壘荼神貌。腳踏烏靴坐折,手持利刃兇驍。圓睜兩眼四邊瞧,那個邪神敢到?
\end{quote}

一夜通明,也無鬼魅。雖是前後門無事,只是身體漸重。

一日,太后又傳旨,召眾臣商議殯殮後事。太宗又宣徐茂公,吩咐國家大事,叮囑倣劉蜀主託孤之意。言畢,沐浴更衣,待時而已。傍閃魏徵,手扯龍衣,奏道:「陛下寬心,臣有一事,管保陛下長生。」太宗道:「病勢已入膏肓,命將危矣,如何保得?」徵云:「臣有書一封,進與陛下,捎去到陰司,付酆都判官崔珏。」太宗道:「崔珏是誰?」徵云:「崔珏乃是太上先皇帝駕前之臣,先受茲洲令,後陞禮部侍郎。在日與臣八拜為交,相知甚厚。他如今已死,現在陰司做掌生死文簿的酆都判官,夢中常與臣相會。此去若將此書付與他,他念微臣薄分,必然放陛下回來。管教魂魄還陽世,定取龍顏轉帝都。」太宗聞言,接在手中,籠入袖裡,遂瞑目而亡。那三宮六院、皇后嬪妃、侍長儲君及兩班文武,俱舉哀戴孝。又在白虎殿上,停著梓宮不題。

畢竟不知太宗如何還魂,且聽下回分解。
