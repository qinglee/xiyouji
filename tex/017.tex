
\chapter{孫行者大鬧黑風山 觀世音收伏熊羆怪}

話說孫行者一觔斗跳將起去,諕得那觀音院大小和尚並頭陀、幸童、道人等一個個朝天禮拜道:「爺爺呀!原來是騰雲駕霧的神聖下界,怪道火不能傷。恨我那個不識人的老剝皮使心用心,今日反害了自己。」三藏道:「列位請起,不須恨了。這去尋著袈裟,萬事皆休;但恐找尋不著,我那徒弟性子有些不好,汝等性命不知如何,恐一人不能脫也。」眾僧聞得此言,一個個提心弔膽,告天許願,只要尋得袈裟,各全性命不題。

卻說孫大聖到空中,把腰兒扭了一扭,早來到黑風山上。住了雲頭,仔細看,果然是座好山,況正值春光時節,但見:
\begin{quote}
萬壑爭流,千崖競秀。鳥啼人不見,花落樹猶香。雨過天連青壁潤,風來松捲翠屏張。山草發,野花開,懸崖峭嶂;薜蘿生,佳木麗,峻嶺平崗。不遇幽人,那尋樵子?澗邊雙鶴飲,石上野猿狂。矗矗堆螺排黛色,巍巍擁翠弄嵐光。
\end{quote}

那行者正觀山景,忽聽得芳草坡前,有人言語。他卻輕步潛蹤,閃在那石崖之下,偷睛觀看。原來是三個妖魔,席地而坐:上首的是一條黑漢,左首下是一個道人,右首下是一個白衣秀士。都在那裡高談闊論,講的是立鼎安爐,摶砂煉汞,白雪黃芽,傍門外道。正說中間,那黑漢笑道:「後日是我母難之日,二公可光顧光顧。」白衣秀士道:「年年與大王上壽,今年豈有不來之理?」黑漢道:「我夜來得了一件寶貝,名喚錦襴佛衣,誠然是件玩好之物。我明日就以他為壽,大開筵宴,邀請各山道官,慶賀佛衣,就稱為佛衣會如何?」道人笑道:「妙,妙,妙。我明日先來拜壽,後日再來赴宴。」

行者聞得佛衣之言,定以為是他寶貝。他就忍不住怒氣,跳出石崖,雙手舉起金箍棒,高叫道:「我把你這夥賊怪!你偷了我的袈裟,要做甚麼佛衣會?趁早兒將來還我。」喝一聲「休走!」掄起棒,照頭一下。慌得那黑漢化風而逃,道人駕雲而走,只把個白衣秀士一棒打死。拖將過來看處,卻是一條白花蛇怪。索性提起來,捽做五七斷。徑入深山,找尋那個黑漢。轉過尖峰,抹過峻嶺,又見那壁陡崖前,聳出一座洞府。但見那:
\begin{quote}
煙霞渺渺,松柏森森。煙霞渺渺采盈門,松柏森森青遶戶。橋踏枯槎木,峰巔繞薜蘿。鳥啣紅蕊來雲壑,鹿踐芳叢上石臺。那門前時催花發,風送花香。臨堤綠柳轉黃鸝,傍岸夭桃翻粉蝶。雖然曠野不堪誇,卻賽蓬萊山下景。
\end{quote}

行者到於門首,又見那兩扇石門關得甚緊。門上有一橫石板,明書六個大字,乃「黑風山黑風洞」。即便掄棒,叫聲:「開門!」那裡面有把門的小妖,開了門出來,問道:「你是何人,敢來擊吾仙洞?」行者罵道:「你個作死的孽畜!甚麼個去處,敢稱仙洞?『仙』字是你稱的?快進去報與你那黑漢,教他快送老爺的袈裟出來,饒你一窩性命。」小妖急急跑到裡面,報道:「大王,佛衣會做不成了,門外有一個毛臉雷公嘴的和尚來討袈裟哩。」那黑漢被行者在芳草坡前趕將來,卻才關了門,坐還未穩,又聽得那話,心中暗想道:「這廝不知是那裡來的,這般無禮,他敢嚷上我的門來。」教取披掛,隨結束了,綽一桿黑纓槍,走出門來。這行者閃在門外,執著鐵棒,睜睛觀看,只見那怪果生得兇險:
\begin{quote}
碗子鐵盔火漆光,烏金鎧甲亮輝煌。
皂羅袍罩風兜袖,黑綠絲絛穗長。
手執黑纓槍一桿,足踏烏皮靴一雙。
眼晃金睛如掣電,正是山中黑風王。
\end{quote}

行者暗笑道:「這廝真個如燒窰的一般,築煤的無二,想必是在此處刷炭為生,怎麼這等一身烏黑?」那怪厲聲高叫道:「你是個甚麼和尚,敢在我這裡大膽?」行者執鐵棒,撞至面前,大咤一聲道:「不要閑講,快還你老外公的袈裟來。」那怪道:「你是那寺裡和尚?你的袈裟在那裡失落了,敢來我這裡索取?」行者道:「我的袈裟在直北觀音院後方丈裡放著,只因那院裡失了火,你這廝趁鬨擄掠,盜了來,要做佛衣會慶壽,怎敢抵賴?快快還我,饒你性命;若牙迸半個『不』字,我推倒了黑風山,屣平了黑風洞,把你這一洞妖邪都碾為齏粉。」

那怪聞言,呵呵冷笑道:「你這個潑物,原來昨夜那火就是你放的。你在那方丈屋上行兇招風,是我把一件袈裟拿來了,你待怎麼?你是那裡來的?姓甚名誰?有多大手段,敢那等海口浪言。」行者道:「是你也認不得你老外公哩。你老外公乃大唐上國駕前御弟三藏法師之徒弟,姓孫,名悟空行者。若問老孫的手段,說出來,教你魂飛魄散,死在眼前。」那怪道:「我不曾會,你有甚麼手段,說來我聽。」行者笑道:「我兒子,你站穩著,仔細聽之。我:
\begin{quote}
自小神通手段高,隨風變化逞英豪。
養性修真熬日月,跳出輪迴把命逃。
一點誠心曾訪道,靈臺山上採藥苗。
那山有個老仙長,壽年十萬八千高。
老孫拜他為師父,指我長生路一條。
他說身內有丹藥,外邊採取枉徒勞。
得傳大品天仙訣,若無根本實難熬。
回光內照寧心坐,身中日月坎離交。
萬事不思全寡慾,六根清淨體堅牢。
返老還童容易得,超凡入聖路非遙。
三年無漏成仙體,不同俗輩受煎熬。
十洲三島還遊戲,海角天涯轉一遭。
活該三百多餘歲,不得飛昇上九霄。
下海降龍真寶貝,才有金箍棒一條。
花果山前為帥首,水簾洞裡聚群妖。
玉皇大帝傳宣詔,封我齊天極品高。
幾番大鬧靈霄殿,數次曾偷王母桃。
天兵十萬來降我,層層密密布槍刀。
戰退天王歸上界,哪吒負痛領兵逃。
顯聖真君能變化,老孫硬賭跌平交。
道祖觀音同玉帝,南天門上看降妖。
卻被老君助一陣,二郎擒我到天曹。
將身綁在降妖柱,即命神兵把首梟。
刀砍鎚敲不得壞,又教雷打火來燒。
老孫其實有手段,全然不怕半分毫。
送在老君爐裡煉,六丁神火慢煎熬。
日滿開爐我跳出,手持鐵棒遶天跑。
縱橫到處無遮擋,三十三天鬧一遭。
我佛如來施法力,五行山壓老孫腰。
整整壓該五百載,幸逢三藏出唐朝。
吾今皈正西方去,轉上雷音見玉毫。
你去乾坤四海問一問,我是歷代馳名第一妖。」
\end{quote}

那怪聞言笑道:「你原來是那鬧天宮的弼馬溫麼?」行者最惱的是人叫他弼馬溫,聽見這一聲,心中大怒,罵道:「你這賊怪!偷了袈裟不還,倒傷老爺。不要走,看棍。」那黑漢側身躲過,綽長槍,劈手來迎。兩家這場好殺:
\begin{quote}
如意棒,黑纓槍,二人洞口逞剛強。分心劈臉刺,著臂照頭傷。這個橫丟陰棍手,那個直撚急三槍。白虎爬山來探爪,黃龍臥道轉身忙。噴彩霧,吐毫光,兩個妖仙不可量。一個是修正齊天聖,一個是成精黑大王。這場山裡相爭處,只為袈裟各不良。
\end{quote}

那怪與行者鬥了十數回合,不分勝負,漸漸紅日當午。那黑漢舉槍架住鐵棒道:「孫行者,咱兩個且收兵,等我進了膳來,再與你賭鬥。」行者道:「你這個孽畜,教做漢子?好漢子,半日兒就要吃飯?似老孫在山根下,整壓了五百餘年,也未曾嘗些湯水,那裡便餓哩?莫推故,休走,還我袈裟來,方讓你去吃飯。」那怪虛幌一槍,撤身入洞,關了石門,收回小怪,且安排筵宴,書寫請帖,邀請各山魔王慶會不題。

卻說行者攻門不開,也只得回觀音院。那本寺僧人已葬埋了那老和尚,都在方丈裡伏侍唐僧。早齋已畢,又擺上午齋。正那裡添湯換水,只見行者從空降下,眾僧禮拜,接入方丈,見了三藏。三藏道:「悟空,你來了?袈裟如何?」行者道:「已有了根由。早是不曾冤了這些和尚,原來是那黑風山妖怪偷了。老孫去暗暗的尋他,只見他與一個白衣秀士、一個老道人,坐在那芳草坡前講話。也是個不打自招的怪物,他忽然說出道:後日是他母難之日,邀請諸邪來做生日;夜來得了一件錦襴佛衣,要以此為壽,作一大宴,喚做慶賞佛衣會。是老孫搶到面前,打了一棍,那黑漢化風而走,道人也不見了,只把個白衣秀士打死,乃是一條白花蛇成精。我又急急趕到他洞口,叫他出來與他賭鬥。他已承認了,是他拿回。戰夠這半日,不分勝負。那怪回洞,卻要吃飯,關了石門,懼戰不出。老孫卻來回看師父,先報此信。已是有了袈裟的下落,不怕他不還我。」

眾僧聞言,合掌的合掌,磕頭的磕頭,都念聲:「南無阿彌陀佛!今日尋著下落,我等方有了性命矣。」行者道:「你且休喜歡暢快,我還未曾到手,師父還未曾出門哩。只等有了袈裟,打發得我師父好好的出門,才是你們的安樂處;若稍有些須不虞,老孫可是好惹的主子!可曾有好茶飯與我師父吃?可曾有好草料喂馬?」眾僧俱滿口答應道:「有,有,有,更不曾一毫待怠慢了老爺。」三藏道:「自你去了這半日,我已吃過了三次茶湯,兩餐齋供了,他俱不曾敢慢我。但只是你還盡心竭力去尋取袈裟回來。」行者道:「莫忙,既有下落,管情拿住這廝,還你原物。放心,放心。」

正說處,那上房院主又整治素供,請孫老爺吃齋。行者卻吃了些須,復駕祥雲,又去找尋。正行間,只見一個小妖,左脅下夾著一個花梨木匣兒,從大路而來。行者度他匣內必有甚麼柬札,舉起棒,劈頭一下,可憐不禁打,就打得似個肉餅一般。卻拖在路傍,揭開匣兒觀看,果然是一封請帖。帖上寫著:
\begin{quote}
侍生熊羆頓首拜,啟上大闡金池老上人丹房:屢承佳惠,感激淵深。夜觀回祿之難,有失救護,諒仙機必無他害。生偶得佛衣一件,欲作雅會,謹具花酌,奉扳清賞。至期,千乞仙駕過臨一敘。是荷。先二日具。
\end{quote}

行者見了,呵呵大笑道:「那個老剝皮,死得他一毫兒也不虧,他原來與妖精結黨。怪道他也活了二百七十歲,想是那個妖精傳他些甚麼服氣的小法兒,故有此壽。老孫還記得他的模樣,等我就變做那和尚,往他洞裡走走,看我那袈裟放在何處。假若得手,即便拿回,卻也省力。」

好大聖,念動咒語,迎著風一變,果然就像那老和尚一般。藏了鐵棒,拽開步,徑來洞口,叫聲:「開門!」那小妖開了門,見是這般模樣,急轉身報道:「大王,金池長老來了。」那怪大驚道:「剛才差了小的去下簡帖請他,這時候還未到那裡哩,如何他就來得這等迅速?想是小的不曾撞著他,斷是孫行者呼他來討袈裟的。管事的,可把佛衣藏了,莫教他看見。」

行者進了前門,但見那天井中松篁交翠,桃李爭妍,叢叢花發,簇簇蘭香,卻也是個洞天之處。又見那二門上有一聯對子,寫著:「靜隱深山無俗慮;幽居仙洞樂天真。」行者暗道:「這廝也是個脫垢離塵,知命的怪物。」入門裡,往前又進,到於三層門裡,都是些畫棟雕梁,明窗彩戶。只見那黑漢子穿的是黑綠紵絲袢襖,罩一領鴉青花綾披風,戴一頂烏角軟巾,穿一雙麂皮皂靴。見行者進來,整頓衣巾,降階迎接道:「金池老友,連日欠親。請坐,請坐。」行者以禮相見。見畢而坐,坐定而茶。茶罷,妖精欠身道:「適有小簡奉啟,後日一敘,何老友今日就下顧也?」行者道:「正來進拜,不期路遇華翰,見有佛衣雅會,故此急急奔來,願求見見。」那怪笑道:「老友差矣。這袈裟本是唐僧的,他在你處住錫,你豈不曾看見,反來就我看看?」行者道:「貧僧借來,因夜晚還不曾展看,不期被大王取來。又被火燒了荒山,失落了家私。那唐僧的徒弟又有些驍勇,亂忙中,四下裡都尋覓不見。原來是大王的洪福收來,故特來一見。」

正講處,只見有一個巡山的小妖來報道:「大王,禍事了,下請書的小校被孫行者打死在大路傍邊,他綽著經兒,變化做金池長老,來騙佛衣也。」那怪聞言,暗道:「我說那長老怎麼今日就來,又來得迅速,果然是他。」急縱身,拿過槍來,就刺行者。行者耳朵裡急掣出棍子,現了本相,架住槍尖,就在他那中廳裡跳出,自天井中鬥到前門外。諕得那洞裡群魔都喪膽,家間老幼盡無魂。這場在山頭好賭鬥,比前番更是不同。好殺:
\begin{quote}
那猴王膽大充和尚,這黑漢心靈隱佛衣。語去言來機會巧,隨機應變不差池。袈裟欲見無由見,寶貝玄微真妙微。小怪巡山言禍事,老妖發怒顯神威。翻身打出黑風洞,槍棒爭持辨是非。棒架長槍聲響喨,槍迎鐵棒放光輝。悟空變化人間少,妖怪神通世上稀。這個要把佛衣來慶壽,那個不得袈裟肯善歸?這番苦戰難分手,就是活佛臨凡也解不得圍。
\end{quote}

他兩個從洞口打上山頭,自山頭殺在雲外,吐霧噴風,飛砂走石,只鬥到紅日沉西,不分勝敗。那怪道:「姓孫的,你且住了手,今日天晚,不好相持。你去,你去,待明早來,與你定個死活。」行者叫道:「兒子莫走,要戰便像個戰的,不可以天晚相推。」看他沒頭沒臉的,只情使棍子打來。這黑漢又化陣清風,轉回本洞,緊閉石門不出。

行者卻無計策奈何,只得也回觀音院裡,按落雲頭,道聲:「師父。」那三藏眼兒巴巴的正望他哩,忽見到了面前,甚喜;又見他手裡沒有袈裟,又懼。問道:「怎麼這番還不曾有袈裟來?」行者袖中取出個簡帖兒來,遞與三藏道:「師父,那怪物與這死的老剝皮原是朋友。他著一個小妖送此帖來,還請他去赴佛衣會。是老孫就把那小妖打死,變做那老和尚,進他洞去,騙了一鍾茶吃。欲問他討袈裟看看,他不肯拿出。正坐間,忽被一個甚麼巡山的走了風信,他就與我打將起來。只鬥到這早晚,不分上下。他見天晚,閃回洞去,緊閉石門。老孫無奈,也暫回來。」三藏道:「你手段比他何如?」行者道:「我也硬不多兒,只戰個手平。」

三藏才看了簡帖,又遞與那院主道:「你師父敢莫也是妖精麼?」那院主慌忙跪下道:「老爺,我師父是人。只因那黑大王修成人道,常來寺裡與我師父講經,他傳了我師父些養神服氣之術,故以朋友相稱。」行者道:「這夥和尚沒甚妖氣,他一個個頭圓頂天,足方履地,但比老孫肥胖長大些兒,非妖精也。你看那帖兒上寫著『侍生熊羆』,此物必定是個黑熊成精。」三藏道:「我聞得古人云:『熊與猩猩相類。』都是獸類。他卻怎麼成精?」行者笑道:「老孫是獸類,見做了齊天大聖,與他何異?大抵世間之物,凡有九竅者,皆可以修行成仙。」三藏又道:「你才說他本事與你手平,你卻怎生得勝,取我袈裟回來?」行者道:「莫管,莫管,我有處治。」

商議間,眾僧擺上晚齋,請他師徒們吃了。三藏教掌燈,仍去前面禪堂安歇。眾僧都挨牆倚壁,苫搭窩棚,各各睡下,只把後方丈讓與那上下院主安身。此時夜靜,但見:
\begin{quote}
銀河現影,玉宇無塵。滿天星燦爛,一水浪收痕。萬籟聲寧,千山鳥絕。溪邊漁火息,塔上佛燈昏。昨夜闍黎鐘鼓響,今宵一遍哭聲聞。
\end{quote}

是夜在禪堂歇宿。那三藏想著袈裟,那裡得穩睡?忽翻身見窗外透白,急起叫道:「悟空,天明了,快尋袈裟去。」行者一骨魯跳將起來,一見眾僧侍立,供奉湯水,行者道:「你等用心伏侍我師父,老孫去也。」三藏下床,扯住道:「你往那裡去?」行者道:「我想這樁事都是觀音菩薩沒理,他有這個禪院在此,受了這裡人家香火,又容那妖精鄰住。我去南海尋他,與他講一講,教他親來問妖精討袈裟還我。」三藏道:「你這去,幾時回來?」行者道:「時少只在飯罷,時多只在晌午,就成功了。那些和尚可好伏侍,老孫去也。」

說聲去,早已無蹤。須臾間到了南海,停雲觀看。但見那:
\begin{quote}
汪洋海遠,水勢連天。祥光籠宇宙,瑞氣照山川。千層雪浪吼青霄,萬疊煙波滔白晝。水飛四野,浪滾週遭。水飛四野振轟雷,浪滾週遭鳴霹靂。休言水勢,且看中間。五色朦朧寶疊山,紅黃紫皂綠和藍。才見觀音真勝境,試看南海落伽山。好去處,山峰高聳,頂透虛空。中間有千樣奇花,百般瑞草。風搖寶樹,日映金蓮。觀音殿,瓦蓋琉璃;潮音洞,門鋪玳瑁。綠楊影裡語鸚哥,紫竹林中啼孔雀。羅紋石上,護法威嚴;瑪瑙灘前,木叉雄壯。
\end{quote}

這行者觀不盡那異景非常,徑直按雲頭,到竹林之下。早有諸天迎接道:「菩薩前者對眾言大聖歸善,甚是宣揚。今保唐僧,如何得暇到此?」行者道:「因保唐僧,路逢一事,特見菩薩,煩為通報。」諸天遂來洞口報知,菩薩喚入。行者遵法而行,至寶蓮臺下拜了。菩薩問曰:「你來何幹?」行者道:「我師父路遇你的禪院,你受了人間香火,容一個黑熊精在那裡鄰住,著他偷了我師父袈裟,屢次取討不與,今特來問你要的。」菩薩道:「這猴子說話,這等無狀。既是熊精偷了你的袈裟,你怎來問我取討?都是你這個孽猴大膽,將寶貝賣弄,拿與小人看見,你卻又行兇,喚風發火,燒了我的留雲下院,反來我處放刁。」行者見菩薩說出這話,知他曉得過去未來之事,慌忙禮拜道:「菩薩,乞恕弟子之罪,果是這般這等。但恨那怪物不肯與我袈裟,師父又要念那話兒咒語,老孫忍不得頭疼,故此來拜煩菩薩。望菩薩慈悲之心,助我去拿那妖精,取衣西進也。」菩薩道:「那怪物有許多神通,卻也不亞於你。也罷,我看唐僧面上,和你去走一遭。」行者聞言,謝恩再拜。即請菩薩出門,遂同駕祥雲,早到黑風山,墜落雲頭,依路找洞。

正行處,只見那山坡前走出一個道人,手拿著一個玻璃盤兒,盤內安著兩粒仙丹,往前正走。被行者撞個滿懷,掣出棒,就照頭一下,打得腦裡漿流出,腔中血迸攛。菩薩大驚道:「你這個猴子,還是這等放潑。他又不曾偷你袈裟,又不與你相識,又無甚冤仇,你怎麼就將他打死?」行者道:「菩薩,你認他不得,他是那黑熊精的朋友。他昨日和一個白衣秀士,都在芳草坡前坐講。後日是黑精的生日,請他們來慶佛衣會。今日他先來拜壽,明日來慶佛衣會。所以我認得,定是今日替那妖去上壽。」菩薩說:「既是這等說來,也罷。」行者才去把那道人提起來看,卻是一隻蒼狼。傍邊那個盤兒底下卻有字,刻道「凌虛子製」。

行者見了,笑道:「造化,造化,老孫也是便益,菩薩也是省力。這怪叫做不打自招,那怪教他今日了劣。」菩薩說道:「悟空,這教怎麼說?」行者道:「菩薩,我悟空有一句話兒,叫做將計就計,不知菩薩可肯依我?」菩薩道:「你說。」行者說道:「菩薩,你看這盤兒中是兩粒仙丹,便是我們與那妖魔的贄見;這盤兒後面刻的四個字,說『凌虛子製』,便是我們與那妖魔的勾頭。菩薩若要依得我時,我好替你作個計較,也就不須動得干戈,也不須勞得征戰,妖魔眼下遭瘟,佛衣眼下出現;菩薩要不依我時,菩薩往西,我悟空往東,佛衣只當相送,唐三藏只當落空。」菩薩笑道:「這猴熟嘴。」行者道:「不敢,倒是一個計較。」菩薩說:「你這計較怎說?」行者道:「這盤上刻那『凌虛子製』,想這道人就叫做凌虛子。菩薩,你要依我時,可就變做這個道人,我把這丹吃了一粒,變上一粒,略大些兒。菩薩,你就捧了這個盤兒、兩粒仙丹,去與那妖上壽,把這丸大些的讓與那妖。待那妖一口吞之,老孫便於中取事:他若不肯獻出佛衣,老孫將他肚腸就也織將一件出來。」菩薩沒法,只得也點點頭兒依他。行者笑道:「如何?」

爾時菩薩迺以廣大慈悲,無邊法力,億萬化身,以心會意,以意會身,恍惚之間,變作凌虛仙子:
\begin{quote}
鶴氅仙風颯,飄颻欲步虛。
蒼顏松柏老,秀色古今無。
去去還無住,如如自有殊。
總來歸一法,只是隔邪軀。
\end{quote}

行者看道:「妙啊,妙啊!還是妖精菩薩,還是菩薩妖精?」菩薩笑道:「悟空,菩薩、妖精,總是一念;若論本來,皆屬無有。」行者心下頓悟,轉身卻就變做一粒仙丹:
\begin{quote}
走盤無不定,圓明未有方。
三三勾漏合,六六少翁商。
瓦鑠黃金焰,牟尼白晝光。
外邊鉛與汞,未許易論量。
\end{quote}

行者變了那顆丹,終是略大些兒。菩薩認定,拿了那個玻璃盤兒,徑到妖洞門口看時,果然是:
\begin{quote}
崖深岫險,雲生嶺上;柏蒼松翠,風颯林間。崖深岫險,果是妖邪出沒人煙少;柏蒼松翠,也可仙真修隱道情多。山有澗,澗有泉,潺潺流水咽鳴琴,便堪洗耳;崖有鹿,林有鶴,幽幽仙籟動間岑,亦可賞心。這是妖仙有分降菩提,弘誓無邊垂惻隱。
\end{quote}

菩薩看了,心中暗喜道:「這孽畜占了這座山洞,卻是也有些道分。」因此心中已是有個慈悲。

走到洞口,只見守洞小妖都有些認得道:「凌虛仙長來了。」一邊傳報,一邊接引。那妖早已迎出門道:「凌虛,有勞仙駕珍顧,蓬蓽有輝。」菩薩道:「小道敬獻一粒仙丹,敢稱千壽。」他二人拜畢,方才坐定,又敘起他昨日之事。菩薩不答,連忙拿丹盤道:「大王,且見小道鄙意。」覷定一粒大的,推與那妖道:「願大王千壽。」那妖亦推一粒,遞與菩薩道:「願與凌虛子同之。」讓畢,那妖才待要咽,那藥順口兒一直滾下。現了本相,理起四平。那妖滾倒在地。菩薩現相,問妖取了佛衣。行者早已從鼻孔中出去。菩薩又怕那妖無禮,卻把一個箍兒丟在那妖頭上。那妖起來,提槍要刺,行者、菩薩早已起在空中,將真言念起。那怪依舊頭疼,丟了槍,滿地亂滾。半空裡笑倒個美猴王,平地下滾壞個黑熊怪。

菩薩道:「孽畜,你如今可皈依麼?」那怪滿口道:「心願皈依,只望饒命。」行者恐耽擱了工夫,意欲就打。菩薩急止住道:「休傷他命,我有用他處哩。」行者道:「這樣怪物,不打死他,反留他在何處用哩?」菩薩道:「我那落伽山後無人看管,我要帶他去做個守山大神。」行者笑道:「誠然是個救苦慈尊,一靈不損。若是老孫有這樣咒語,就念上他娘千遍。這回兒就有許多黑熊,都教他了帳。」卻說那怪甦醒多時,公道難禁疼痛,只得跪在地下哀告道:「但饒性命,願皈正果。」菩薩方墜落祥光,又與他摩頂受戒,教他執了長槍,跟隨左右。那黑熊才一片野心今日定,無窮頑性此時收。

菩薩吩咐道:「悟空,你回去罷,好生伏侍唐僧,以後再休懈惰生事。」行者道:「深感菩薩遠來,弟子還當回送回送。」菩薩道:「免送。」行者才捧著袈裟,叩頭而別。菩薩亦帶了熊羆,徑回大海。有詩為證。詩曰:
\begin{quote}
祥光靄靄凝金像,萬道繽紛實可誇。
普濟世人垂憫恤,遍觀法界現金蓮。
今來多為傳經意,此去原無落點瑕。
降怪成真歸大海,空門復得錦袈裟。
\end{quote}

畢竟不知向後事情如何,且聽下回分解。
