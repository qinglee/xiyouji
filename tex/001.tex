
\chapter{靈根育孕源流出 心性修持大道生}

詩曰:
\begin{quote}
混沌未分天地亂,茫茫渺渺無人見。
自從盤古破鴻濛,開闢從茲清濁辨。
覆載群生仰至仁,發明萬物皆成善。
欲知造化會元功,須看西遊釋厄傳。
\end{quote}

蓋聞天地之數,有十二萬九千六百歲為一元。將一元分為十二會,乃子、丑、寅、卯、辰、巳、午、未、申、酉、戌、亥之十二支也。每會該一萬八百歲。且就一日而論:子時得陽氣,而丑則雞鳴;寅不通光,而卯則日出;辰時食後,而巳則挨排;日午天中,而未則西蹉;申時晡,而日落酉,戌黃昏,而入定亥。譬於大數,若到戌會之終,則天地昏曚而萬物否矣。再去五千四百歲,交亥會之初,則當黑暗,而兩間人物俱無矣,故曰混沌。又五千四百歲,亥會將終,貞下起元,近子之會,而復逐漸開明。邵康節曰:「冬至子之半,天心無改移。一陽初動處,萬物未生時。」到此,天始有根。再五千四百歲,正當子會,輕清上騰,有日,有月,有星,有辰。日、月、星、辰,謂之四象。故曰,天開於子。又經五千四百歲,子會將終,近丑之會,而逐漸堅實。《易》曰:「大哉乾元!至哉坤元!萬物資生,乃順承天。」至此,地始凝結。再五千四百歲,正當丑會,重濁下凝,有水,有火,有山,有石,有土。水、火、山、石、土,謂之五形。故曰,地闢於丑。又經五千四百歲,丑會終而寅會之初,發生萬物。曆曰:「天氣下降,地氣上升;天地交合,群物皆生。」至此,天清地爽,陰陽交合。再五千四百歲,正當寅會,生人,生獸,生禽,正謂天地人,三才定位。故曰,人生於寅。

感盤古開闢,三皇治世,五帝定倫,世界之間,遂分為四大部洲:曰東勝神洲,曰西牛賀洲,曰南贍部洲,曰北俱蘆洲。這部書單表東勝神洲。海外有一國土,名曰傲來國。國近大海,海中有一座名山,喚為花果山。此山乃十洲之祖脈,三島之來龍,自開清濁而立,鴻濛判後而成。真個好山!有詞賦為證。賦曰:
\begin{quote}
勢鎮汪洋,威寧瑤海。勢鎮汪洋,潮湧銀山魚入穴;威寧瑤海,波翻雪浪蜃離淵。水火方隅高積上,東海之處聳崇巔。丹崖怪石,削壁奇峰。丹崖上,彩鳳雙鳴;削壁前,麒麟獨臥。峰頭時聽錦雞鳴,石窟每觀龍出入。林中有壽鹿仙狐,樹上有靈禽玄鶴。瑤草奇花不謝,青松翠柏長春。仙桃常結果,修竹每留雲。一條澗壑籐蘿密,四面原堤草色新。正是百川會處擎天柱,萬劫無移大地根。
\end{quote}

那座山正當頂上,有一塊仙石。其石有三丈六尺五寸高,有二丈四尺圍圓。三丈六尺五寸高,按周天三百六十五度;二丈四尺圍圓,按政曆二十四氣。上有九竅八孔,按九宮八卦。四面更無樹木遮陰,左右倒有芝蘭相襯。

蓋自開闢以來,每受天真地秀,日精月華,感之既久,遂有靈通之意。內育仙胞,一日迸裂,產一石卵,似圓毬樣大。因見風,化作一個石猴,五官俱備,四肢皆全。便就學爬學走,拜了四方。目運兩道金光,射沖斗府。驚動高天上聖大慈仁者玉皇大天尊玄穹高上帝,駕座金闕雲宮靈霄寶殿,聚集仙卿,見有金光燄燄,即命千里眼、順風耳開南天門觀看。二將果奉旨出門外,看的真,聽的明。須臾回報道:「臣奉旨觀聽金光之處,乃東勝神洲海東傲來小國之界,有一座花果山,山上有一仙石,石產一卵,見風化一石猴,在那裡拜四方,眼運金光,射沖斗府。如今服餌水食,金光將潛息矣。」玉帝垂賜恩慈曰:「下方之物,乃天地精華所生,不足為異。」

那猴在山中,卻會行走跳躍,食草木,飲澗泉,採山花,覓樹果;與狼蟲為伴,虎豹為群,獐鹿為友,獼猿為親;夜宿石崖之下,朝遊峰洞之中。真是:「山中無甲子,寒盡不知年。」

一朝天氣炎熱,與群猴避暑,都在松陰之下頑耍。你看他一個個:
\begin{quote}
跳樹攀枝,採花覓果;拋彈子,邷麼兒;跑沙窩,砌寶塔;趕蜻蜓,撲蜡;參老天,拜菩薩;扯葛籐,編草;捉虱子,咬又掐;理毛衣,剔指甲。挨的挨,擦的擦;推的推,壓的壓;扯的扯,拉的拉:青松林下任他頑,綠水澗邊隨洗濯。
\end{quote}

一群猴子耍了一會,卻去那山澗中洗澡。見那股澗水奔流,真個似滾瓜湧濺。古云:「禽有禽言,獸有獸語。」眾猴都道:「這股水不知是那裡的水。我們今日趕閑無事,順澗邊往上溜頭尋看源流,耍子去耶!」喊一聲,都拖男挈女,呼弟呼兄,一齊跑來,順澗爬山,直至源流之處,乃是一股瀑布飛泉。但見那:
\begin{quote}
一派白虹起,千尋雪浪飛。
海風吹不斷,江月照還依。
冷氣分青嶂,餘流潤翠微。
潺湲名瀑布,真似掛簾帷。
\end{quote}

眾猴拍手稱揚道:「好水,好水!原來此處遠通山腳之下,直接大海之波。」又道:「那一個有本事的,鑽進去尋個源頭出來,不傷身體者,我等即拜他為王。」連呼了三聲,忽見叢雜中跳出一個石猴,應聲高叫道:「我進去,我進去。」好猴!也是他:
\begin{quote}
今日芳名顯,時來大運通。
有緣居此地,王遣入仙宮。
\end{quote}

你看他瞑目蹲身,將身一縱,徑跳入瀑布泉中,忽睜睛擡頭觀看,那裡邊卻無水無波,明明朗朗的一架橋梁。他住了身,定了神,仔細再看,原來是座鐵板橋。橋下之水,沖貫於石竅之間,倒掛流出去,遮閉了橋門。卻又欠身上橋頭,再走再看,卻似有人家住處一般,真個好所在。但見那:
\begin{quote}
翠蘚堆藍,白雲浮玉,光搖片片煙霞。虛窗靜室,滑凳板生花。乳窟龍珠倚掛,縈迴滿地奇葩。鍋灶傍崖存火跡,樽罍靠案見殽渣。石座石床真可愛,石盆石碗更堪誇。又見那一竿兩竿修竹,三點五點梅花。幾樹青松常帶雨,渾然像個人家。
\end{quote}

看罷多時,跳過橋中間,左右觀看。只見正當中有一石碣,碣上有一行楷書大字,鐫著「花果山福地,水簾洞洞天」。

石猿喜不自勝,急抽身往外便走,復瞑目蹲身,跳出水外,打了兩個呵呵道:「大造化!大造化!」眾猴把他圍住,問道:「裡面怎麼樣?水有多深?」石猴道:「沒水!沒水!原來是一座鐵板橋,橋那邊是一座天造地設的家當。」眾猴道:「怎見得是個家當?」石猴笑道:「這股水乃是橋下沖貫石橋,倒掛下來遮閉門戶的。橋邊有花有樹,乃是一座石房。房內有石窩、石灶、石碗、石盆、石床、石凳。中間一塊石碣上,鐫著『花果山福地,水簾洞洞天』。真個是我們安身之處。裡面且是寬闊,容得千百口老小。我們都進去住,也省得受老天之氣。這裡邊:
\begin{quote}
刮風有處躲,下雨好存身。
霜雪全無懼,雷聲永不聞。
煙霞常照耀,祥瑞每蒸熏。
松竹年年秀,奇花日日新。」
\end{quote}

眾猴聽得,個個歡喜。都道:「你還先走,帶我們進去,進去。」石猴卻又瞑目蹲身,往裡一跳,叫道:「都隨我進來,進來。」那些猴有膽大的,都跳進去了;膽小的,一個個伸頭縮頸,抓耳撓腮,大聲叫喊,纏一會,也都進去了。跳過橋頭,一個個搶盆奪碗,佔灶爭床,搬過來,移過去,正是猴性頑劣,再無一個寧時,只搬得力倦神疲方止。石猿端坐上面道:「列位啊,『人而無信,不知其可。』你們才說有本事進得來,出得去,不傷身體者,就拜他為王。我如今進來又出去,出去又進來,尋了這一個洞天與列位安眠穩睡,各享成家之福,何不拜我為王?」眾猴聽說,即拱伏無違,一個個序齒排班,朝上禮拜,都稱「千歲大王」。自此,石猿高登王位,將「石」字兒隱了,遂稱「美猴王」。有詩為證。詩曰:
\begin{quote}
三陽交泰產群生,仙石胞含日月精。
借卵化猴完大道,假他名姓配丹成。
內觀不識因無相,外合明知作有形。
歷代人人皆屬此,稱王稱聖任縱橫。
\end{quote}

美猴王領一群猿猴、獼猴、馬猴等,分派了君臣佐使。朝遊花果山,暮宿水簾洞,合契同情,不入飛鳥之叢,不從走獸之類,獨自為王,不勝歡樂。是以:
\begin{quote}
春採百花為飲食,夏尋諸果作生涯。
秋收芋栗延時節,冬覓黃精度歲華。
\end{quote}

美猴王享樂天真,何期有三五百載。一日,與群猴喜宴之間,忽然憂惱,墮下淚來。眾猴慌忙羅拜道:「大王何為煩惱?」猴王道:「我雖在歡喜之時,卻有一點兒遠慮,故此煩惱。」眾猴又笑道:「大王好不知足。我等日日歡會,在仙山福地,古洞神洲,不伏麒麟轄,不伏鳳凰管,又不伏人間王位所拘束,自由自在,乃無量之福,為何遠慮而憂也?」猴王道:「今日雖不歸人王法律,不懼禽獸威嚴,將來年老血衰,暗中有閻王老子管著,一旦身亡,可不枉生世界之中,不得久注天人之內?」眾猴聞此言,一個個掩面悲啼,俱以無常為慮。

只見那班部中,忽跳出一個通背猿猴,厲聲高叫道:「大王若是這般遠慮,真所謂道心開發也。如今五蟲之內,惟有三等名色不伏閻王老子所管。」猴王道:「你知那三等人?」猿猴道:「乃是佛與仙與神聖三者,躲過輪迴,不生不滅,與天地山川齊壽。」猴王道:「此三者居於何所?」猿猴道:「他只在閻浮世界之中,古洞仙山之內。」猴王聞之,滿心歡喜道:「我明日就辭汝等下山,雲遊海角,遠涉天涯,務必訪此三者,學一個不老長生,常躲過閻君之難。」噫!這句話,頓教跳出輪迴網,致使齊天大聖成。眾猴鼓掌稱揚,都道:「善哉,善哉!我等明日越嶺登山,廣尋些果品,大設筵宴送大王也。」

次日,眾猴果去採仙桃,摘異果,刨山藥,斸黃精。芝蘭香蕙,瑤草奇花,般般件件,整整齊齊,擺開石凳石桌,排列仙酒仙殽。但見那:
\begin{quote}
金丸珠彈,紅綻黃肥。金丸珠彈臘櫻桃,色真甘美;紅綻黃肥熟梅子,味果香酸。鮮龍眼,肉甜皮薄;火荔枝,核小囊紅。林檎碧實連枝獻,枇杷緗苞帶葉擎。兔頭梨子雞心棗,消渴除煩更解酲。香桃爛杏,美甘甘似玉液瓊漿;脆李楊梅,酸蔭蔭如脂酥膏酪。紅囊黑子熟西瓜,四瓣黃皮大柿子。石榴裂破,丹砂粒現火晶珠;芋栗剖開,堅硬肉團金瑪瑙。胡桃銀杏可傳茶,椰子葡萄能做酒。榛松榧柰滿盤盛,橘蔗柑橙盈案擺。熟煨山藥,爛煮黃精。搗碎茯苓並薏苡,石鍋微火漫炊羹。人間縱有珍饈味,怎比山猴樂更寧。
\end{quote}

群猴尊美猴王上坐,各依齒肩排於下邊,一個個輪流上前奉酒、奉花、奉果,痛飲了一日。

次日,美猴王早起,教:「小的們,替我折些枯松,編作栰子,取個竹竿作篙,收拾些果品之類,我將去也。」果獨自登栰,儘力撐開,飄飄蕩蕩,徑向大海波中,趁天風,來渡南贍部洲地界。這一去,正是那:
\begin{quote}
天產仙猴道行隆,離山駕栰趁天風。
飄洋過海尋仙道,立志潛心建大功。
有分有緣休俗願,無憂無慮會元龍。
料應必遇知音者,說破源流萬法通。
\end{quote}

也是他運至時來,自登木栰之後,連日東南風緊,將他送到西北岸前,乃是南贍部洲地界。持篙試水,偶得淺水,棄了栰子,跳上岸來。只見海邊有人捕魚、打雁、穵蛤、淘鹽。他走近前,弄個把戲,妝個虎,嚇得那些人丟筐棄網,四散奔跑。將那跑不動的拿住一個,剝了他衣裳,也學人穿在身上。搖搖擺擺,穿州過府,在市廛中學人禮,學人話。朝餐夜宿,一心裡訪問佛、仙、神聖之道,覓個長生不老之方。見世人都是為名為利之徒,更無一個為身命者。正是那:
\begin{quote}
爭名奪利幾時休?早起遲眠不自由!
騎著驢騾思駿馬,官居宰相望王侯。
只愁衣食耽勞碌,何怕閻君就取勾。
繼子蔭孫圖富貴,更無一個肯回頭。
\end{quote}

猴王參訪仙道,無緣得遇。在於南贍部洲,串長城,遊小縣,不覺八九年餘。忽行至西洋大海,他想著海外必有神仙。獨自個依前作栰,又飄過西海,直至西牛賀洲地界。登岸遍訪多時,忽見一座高山秀麗,林麓幽深。他也不怕狼蟲,不懼虎豹,登上山頂上觀看。果是好山:
\begin{quote}
千峰排戟,萬仞開屏。日映嵐光輕鎖翠,雨收黛色冷含青。瘦籐纏老樹,古渡界幽程。奇花瑞草,修竹喬松。修竹喬松,萬載常青欺福地;奇花瑞草,四時不謝賽蓬瀛。幽鳥啼聲近,源泉響溜清。重重谷壑芝蘭繞,處處巉崖苔蘚生。起伏巒頭龍脈好,必有高人隱姓名。
\end{quote}

正觀看間,忽聞得林深之處有人言語。急忙趨步,穿入林中,側耳而聽,原來是歌唱之聲。歌曰:
\begin{quote}
「觀棋柯爛,伐木丁丁,雲邊谷口徐行。賣薪沽酒,狂笑自陶情。蒼逕秋高,對月枕松根,一覺天明。認舊林,登崖過嶺,持斧斷枯籐。收來成一擔,行歌市上,易米三升。更無些子爭競,時價平平。不會機謀巧算,沒榮辱,恬淡延生。相逢處,非仙即道,靜坐講黃庭。」
\end{quote}

美猴王聽得此言,滿心歡喜道:「神仙原來藏在這裡!」即忙跳入裡面,仔細再看,乃是一個樵子,在那裡舉斧砍柴。但看他打扮非常:
\begin{quote}
頭上戴箬笠,乃是新筍初脫之籜。身上穿布衣,乃是木綿撚就之紗。腰間繫環絛,乃是老蠶口吐之絲。足下踏草履,乃是枯莎槎就之爽。手執衠鋼斧,擔挽火麻繩。扳松劈枯樹,爭似此樵能。
\end{quote}

猴王近前叫道:「老神仙,弟子起手。」那樵漢慌忙丟了斧,轉身答禮道:「不當人,不當人。我拙漢衣食不全,怎敢當『神仙』二字?」猴王道:「你不是神仙,如何說出神仙的話來?」樵夫道:「我說甚麼神仙話?」猴王道:「我才來至林邊,只聽的你說:『相逢處,非仙即道,靜坐講《黃庭》。』《黃庭》乃道德真言,非神仙而何?」樵夫笑道:「實不瞞你說,這個詞名做《滿庭芳》,乃一神仙教我的。那神仙與我舍下相鄰,他見我家事勞苦,日常煩惱,教我遇煩惱時,即把這詞兒念念,一則散心,二則解困。我才有些不足處思慮,故此念念,不期被你聽了。」猴王道:「你家既與神仙相鄰,何不從他修行?學得個不老之方,卻不是好?」樵夫道:「我一生命苦:自幼蒙父母養育至八九歲,才知人事,不幸父喪,母親居孀。再無兄弟姊妹,只我一人,沒奈何,早晚侍奉。如今母老,一發不敢拋離。卻又田園荒蕪,衣食不足,只得斫兩束柴薪,挑向市廛之間,貨幾文錢,糴幾升米,自炊自造,安排些茶飯,供養老母。所以不能修行。」

猴王道:「據你說起來,乃是一個行孝的君子,向後必有好處。但望你指與我那神仙住處,卻好拜訪去也。」樵夫道:「不遠,不遠。此山叫做靈臺方寸山,山中有座斜月三星洞,那洞中有一個神仙,稱名須菩提祖師。那祖師出去的徒弟,也不計其數,見今還有三四十人從他修行。你順那條小路兒,向南行七八里遠近,即是他家了。」猴王用手扯住樵夫道:「老兄,你便同我去去,若還得了好處,決不忘你指引之恩。」樵夫道:「你這漢子甚不通變,我方才這般與你說了,你還不省?假若我與你去了,卻不誤了我的生意?老母何人奉養?我要斫柴,你自去,自去。」

猴王聽說,只得相辭。出深林,找上路徑,過一山坡,約有七八里遠,果然望見一座洞府。挺身觀看,真好去處!但見:
\begin{quote}
煙霞散彩,日月搖光。千株老柏,萬節修篁。千株老柏,帶雨半空青冉冉;萬節修篁,含煙一壑色蒼蒼。門外奇花佈錦,橋邊瑤草噴香。石崖突兀青苔潤,懸壁高張翠蘚長。時聞仙鶴唳,每見鳳凰翔。仙鶴唳時,聲振九皋霄漢遠;鳳凰翔起,翎毛五色彩雲光。玄猿白鹿隨隱見,金獅玉象任行藏。細觀靈福地,真個賽天堂。
\end{quote}

又見那洞門緊閉,靜悄悄杳無人跡。忽回頭,見崖頭立一石碑,約有三丈餘高,八尺餘闊,上有一行十個大字,乃是「靈臺方寸山,斜月三星洞」。美猴王十分歡喜道:「此間人果是樸實,果有此山此洞。」看夠多時,不敢敲門。且去跳上松枝梢頭,摘松子吃了頑耍。

少頃間,只聽得呀的一聲,洞門開處,裡面走出一個仙童,真個丰姿英偉,像貌清奇,比尋常俗子不同。但見他:
\begin{quote}
髽髻雙絲綰,寬袍兩袖風。
貌和身自別,心與相俱空。
物外長年客,山中永壽童。
一塵全不染,甲子任翻騰。
\end{quote}

那童子出得門來,高叫道:「甚麼人在此搔擾?」猴王撲的跳下樹來,上前躬身道:「仙童,我是個訪道學仙之弟子,更不敢在此搔擾。」仙童笑道:「你是個訪道的麼?」猴王道:「是。」童子道:「我家師父正才下榻,登壇講道,還未說出原由,就教我出來開門。說:『外面有個修行的來了,可去接待接待。』想必就是你了?」猴王笑道:「是我,是我。」童子道:「你跟我進來。」

這猴王整衣端肅,隨童子徑入洞天深處觀看:一層層深閣瓊樓,一進進珠宮貝闕,說不盡那靜室幽居。直至瑤臺之下,見那菩提祖師端坐在臺上,兩邊有三十個小仙侍立臺下。果然是:
\begin{quote}
大覺金仙沒垢姿,西方妙相祖菩提。不生不滅三三行,全氣全神萬萬慈。空寂自然隨變化,真如本性任為之。與天同壽莊嚴體,歷劫明心大法師。
\end{quote}

美猴王一見,倒身下拜,磕頭不計其數,口中只道:「師父,師父,我弟子志心朝禮,志心朝禮。」祖師道:「你是那方人氏?且說個鄉貫、姓名明白,再拜。」猴王道:「弟子乃東勝神洲傲來國花果山水簾洞人氏。」祖師喝令:「趕出去!他本是個撒詐搗虛之徒,那裡修甚麼道果!」猴王慌忙磕頭不住道:「弟子是老實之言,決無虛詐。」祖師道:「你既老實,怎麼說東勝神洲?那去處到我這裡隔兩重大海,一座南贍部洲,如何就得到此?」猴王叩頭道:「弟子飄洋過海,登界遊方,有十數個年頭,方才訪到此處。」

祖師道:「既是逐漸行來的也罷。你姓甚麼?」猴王又道:「我無性。人若罵我,我也不惱;若打我,我也不嗔。只是陪個禮兒就罷了。一生無性。」祖師道:「不是這個性。你父母原來姓甚麼?」猴王道:「我也無父母。」祖師道:「既無父母,想是樹上生的?」猴王道:「我雖不是樹上生,卻是石裡長的。我只記得花果山上有一塊仙石,其年石破,我便生也。」祖師聞言暗喜,道:「這等說,卻是個天地生成的。你起來走走我看。」猴王縱身跳起,拐呀拐的走了兩遍。祖師笑道:「你身軀雖是鄙陋,卻像個食松果的猢猻。我與你就身上取個姓氏,意思教你姓『猢』。猢字去了個獸傍,乃是個古月。古者,老也;月者,陰也。老陰不能化育,教你姓『猻』倒好。猻字去了獸傍,乃是個子系。子者,兒男也;系者,嬰細也,正合嬰兒之本論。教你姓『孫』罷。」猴王聽說,滿心歡喜,朝上叩頭道:「好!好!好!今日方知姓也。萬望師父慈悲,既然有姓,再乞賜個名字,卻好呼喚。」祖師道:「我門中有十二個字,分派起名,到你乃第十輩之小徒矣。」猴王道:「那十二個字?」祖師道:「乃廣、大、智、慧、真、如、性、海、穎、悟、圓、覺十二字。排到你,正當『悟』字。與你起個法名叫做『孫悟空』,好麼?」猴王笑道:「好!好!好!自今就叫做孫悟空也。」正是:
\begin{quote}
鴻濛初闢原無姓,打破頑空須悟空。
\end{quote}

畢竟不知向後修些甚麼道果,且聽下回分解。
