
\chapter{情亂性從因愛慾 神昏心動遇魔頭}

詞曰:
\begin{quote}
心地頻頻掃,塵情細細除。莫教坑塹陷毘盧。本體常清淨,方可論元初。
性燭須挑剔,曹溪任吸呼。勿令猿馬氣聲粗。晝夜綿綿息,方顯是功夫。
\end{quote}

這一首詞,牌名《南柯子》,單道著唐僧脫卻通天河寒冰之災,踏白黿負登彼岸。師徒四眾,順著大路,望西而進。正遇嚴冬之景,但見那林光漠漠煙中淡,山骨稜稜水外清。

師徒們正當行處,忽然又遇一座大山,阻住去道。路窄崖高,石多嶺峻,人馬難進。三藏在馬上兜住韁繩,叫聲:「徒弟。」時有孫行者引豬八戒、沙僧近前侍立道:「師父,有何吩咐?」三藏道:「你看那前面山高,恐有虎狼作怪,妖獸傷人,今番是必仔細!」行者道:「師父放心莫慮。我等兄弟三人心和意合,歸正求真,使出蕩怪降妖之法,怕甚麼虎狼妖獸?」三藏聞言,只得放懷前進。到於谷口,促馬登崖,擡頭仔細觀看,好山:
\begin{quote}
嵯峨矗矗,變削巍巍。嵯峨矗矗沖霄漢,變削巍巍礙碧空。怪石亂堆如坐虎,蒼松斜掛似飛龍。嶺上鳥啼嬌韻美,崖前梅放異香濃。澗水潺湲流出冷,巔雲黯淡過來兇。又見那飄飄雪,凜凜風,咆哮餓虎吼山中。寒鴉揀樹無棲處,野鹿尋窩沒定蹤。可嘆行人難進步,皺眉愁臉把頭蒙。
\end{quote}

師徒四眾冒雪沖寒,戰澌澌行過那巔峰峻嶺,遠望見山凹中有樓臺高聳,房舍清幽。唐僧馬上欣然道:「徒弟啊,這一日又飢又寒,幸得那山凹裡有樓臺房舍,斷乎是莊戶人家,菴觀寺院;且去化些齋飯,吃了再走。」行者聞言,急睜睛看,只見那壁廂兇雲隱隱,惡氣紛紛。回首對唐僧道:「師父,那廂不是好處。」三藏道:「見有樓臺亭宇,如何不是好處?」行者笑道:「師父啊,你那裡知道。西方路上多有妖怪邪魔,善能點化莊宅。不拘甚麼樓臺房舍,館閣亭宇,俱能指化了哄人。你知道『龍生九種」,內有一種名蜃。蜃氣放光,就如樓閣淺池。若遇大江昏迷,蜃現此勢。倘有鳥鵲飛騰,定來歇翅。那怕你上萬論千,盡被他一氣吞之。此意害人最重。那壁廂氣色兇惡,斷不可入。」

三藏道:「既不可入,我卻著實飢了。」行者道:「師父果飢,且請下馬,就在這平處坐下,待我別處化些齋來你吃。」三藏依言下馬,八戒採定韁繩。沙僧放下行李,即去解開包裹,取出缽盂,遞與行者。行者接缽盂在手中,吩咐沙僧道:「賢弟,卻不可前進。好生保護師父穩坐於此,待我化齋回來,再往西去。」沙僧領諾。行者又向三藏道:「師父,這去處少吉多凶,切莫要動身別往。老孫化齋去也。」唐僧道:「不必多言,但要你快去快來。我在這裡等你。」行者轉身欲行,卻又回來道:「師父,我知你沒甚坐性,我與你個安身法兒。」即取金箍棒,幌了一幌,將那平地下週圍畫了一道圈子,請唐僧坐在中間;著八戒、沙僧侍立左右,把馬與行李都放在近身。對唐僧合掌道:「老孫畫的這圈,強似那銅牆鐵壁。憑他甚麼虎豹狼蟲,妖魔鬼怪,俱莫敢近。但只不許你們走出圈外,只在中間穩坐,保你無虞;但若出了圈兒,定遭毒手。千萬千萬,至祝至祝。」三藏依言,師徒俱端然坐下。

行者縱起雲頭,尋莊化齋,一直南行,忽見那古樹參天,乃一村莊舍。按下雲頭,仔細觀看,但只見:
\begin{quote}
雪欺衰柳,冰結方塘。疏疏修竹搖青,鬱鬱喬松凝翠。幾間茅屋半裝銀,一座小橋斜砌粉。籬邊微吐水仙花,簷下長垂冰凍箸。颯颯寒風送異香,雪漫不見梅開處。
\end{quote}

行者隨步觀看莊景,只聽得呀的一聲,柴扉響處,走出一個老者,手拖藜杖,頭頂羊裘,身穿破衲,足踏蒲鞋,拄著杖,仰身朝天道:「西北風起,明日晴了。」說不了,後邊跑出一個哈巴狗兒來,望著行者,汪汪的亂吠。老者卻才轉過頭來,看見行者捧著缽盂。打個問訊道:「老施主,我和尚是東土大唐欽差上西天拜佛求經者,適路過寶方,我師父腹中飢餒,特造尊府募化一齋。」老者聞言,點頭頓杖道:「長老,你且休化齋,你走錯路了。」行者道:「不錯。」老者道:「往西天大路,在那直北下。此間到那裡有千里之遙,還不去找大路而行?」行者笑道:「正是直北下。我師父現在大路上端坐,等我化齋哩。」那老者道:「這和尚胡說了。你師父在大路上等你化齋,似這千里之遙,就會走路,也須得六七日,走回去又要六七日,卻不餓壞他也?」行者笑道:「不瞞老施主說,我才然離了師父,還不上一盞熱茶之時,卻就走到此處。如今化了齋,還要尚趕去作午齋哩。」

老者見說,心中害怕道:「這和尚是鬼,是鬼。」急抽身往裡就走。行者一把扯住道:「施主那裡去?有齋快化些兒。」老者道:「不方便,不方便,別轉一家兒罷。」行者道:「你這施主好不會事。你說我離此有千里之遙,若再轉一家,卻不又有千里?真是餓殺我師父也。」那老者道:「實不瞞你說,我家老小六七口,才淘了三升米下鍋,還未曾煮熟。你且到別處去轉轉再來。」行者道:「古人云:『走三家不如坐一家。』我貧僧在此等一等罷。」那老者見纏得緊,惱了,舉藜杖就打。行者公然不懼,被他照光頭上打了七八下,只當與他拂癢。那老者道:「這是個撞頭的和尚。」行者笑道:「老官兒,憑你怎麼打,只要記得杖數明白:一杖一升米,慢慢量來。」那老者聞言,急丟了藜杖,跑進去把門關了,只嚷:「有鬼,有鬼。」慌得那一家兒戰戰兢兢,把前後門俱關了。

行者見他關了門,心中暗想:「這老賊才說淘米下鍋,不知是虛是實?常言道:『道化賢良釋化愚。』且等老孫進去看看。」好大聖,捻著訣,使個隱身遁法,徑走入廚中看處,果然那鍋裡氣騰騰的,煮了半鍋乾飯。就把缽盂往裡一掗,滿滿的掗了一缽盂,即駕雲回轉不題。

卻說唐僧坐在圈子裡,等待多時,不見行者回來,欠身悵望道:「這猴子往那裡化齋去了?」八戒在傍笑道:「知他往那裡耍子去來?化甚麼齋,卻教我們在此坐牢。」三藏道:「怎麼謂之坐牢?」八戒道:「師父,你原來不知,古人劃地為牢?他將棍子劃個圈兒,強似鐵壁銅牆,假如有虎狼妖獸來時,如何擋得他住?只好白白的送與他吃罷了。」三藏道:「悟能,憑你怎麼處治?」八戒道:「此間又不藏風,又不避冷,若依老豬,只該順著路,往西且行。師兄化了齋,駕了雲,必然來快,讓他趕來。如有齋,吃了再走。如今坐了這一會,老大腳冷!」

三藏聞此言,就是晦氣星到了。遂依獃子,一齊出了圈外。八戒牽了馬,沙僧擔了擔,那長老順路步行前進。不一時,到了樓閣之所,卻原來是坐北向南之家。門外八字粉牆,有一座倒垂蓮升斗門樓,都是五色裝的。那門兒半開半掩。八戒就把馬拴在門枕石鼓上;沙僧歇了擔子;三藏畏風,坐於門限之上。八戒道:「師父,這所在想是公侯之宅,相輔之家。前門外無人,想必都在裡面烘火。你們坐著,讓我進去看看。」唐僧道:「仔細耶,莫要衝撞了人家。」獃子道:「我曉得。自從歸正禪門,這一向也學了些禮數,不比那村莽之夫也。」

那獃子把釘鈀撒在腰裡,整一整青錦直裰,斯斯文文,走入門裡。只見是三間大廳,簾櫳高控,靜悄悄全無人跡,也無桌椅家火。轉過屏門,往裡又走,乃是一座穿堂。堂後有一座大樓,樓上窗格半開,隱隱見一頂黃綾帳幔。獃子道:「想是有人怕冷,還睡哩。」他也不分內外,拽步只管走上樓來。用手掀開看時,把獃子諕了一個躘踵。原來那帳裡象牙床上,白媸媸的一堆骸骨,骷髏有巴斗大,腿挺骨有四五尺長。那獃子定了性,止不住腮邊淚落,對骷髏點頭嘆云:「你不知是:
\begin{quote}
那代那朝元帥體,何邦何國大將軍。
當時豪傑爭強勝,今日淒涼露骨筋。
不見妻兒來侍奉,那逢士卒把香焚。
謾觀這等真堪嘆,可惜興王霸業人。」
\end{quote}

八戒正才感嘆,只見那帳幔後有火光一幌。獃子道:「想是有侍奉香火之人在後面哩。」急轉步,過帳觀看,卻是穿樓的窗扇透光。那壁廂有一張彩漆的桌子,桌子上亂搭著幾件錦繡綿衣。獃子提起來看時,卻是三件納錦背心兒。

他也不管好歹,拿下樓來,出廳房,徑到門外道:「師父,這裡全沒人煙,是一所亡靈之宅。老豬走進裡面,直至高樓之上,黃綾帳內,有一堆骸骨。串樓傍有三件納錦的背心,被我拿來了,也是我們一程兒造化。此時天氣寒冷,正當用處。師父,且脫了褊衫,把他且穿在底下,受用受用,免得吃冷。」三藏道:「不可,不可。律云:『公取竊取皆為盜。』倘或有人知覺,趕上我們,到了當官,斷然是一個竊盜之罪。還不送進去與他搭在原處。我們在此避風坐一坐,等悟空來時走路。出家人不要這等愛小。」八戒道:「四顧無人,雖雞犬亦不知之,但只我們知道,誰人告我?有何證見?就如拾得的一般,那裡論甚麼公取竊取也?」三藏道:「你胡做啊。雖是人不知之,天何蓋焉?玄帝垂訓云:『暗室虧心,神目如電。』趁早送去還他,莫愛非禮之物。」

那獃子莫想肯聽,對唐僧笑道:「師父啊,我自為人,也穿了幾件背心,不曾見這等納錦的。你不穿,且待老豬穿一穿,試試新,晤晤脊背。等師兄來,脫了還他走路。」沙僧道:「既如此說,我也穿一件兒。」兩個齊脫了上蓋直裰,將背心套上。才緊帶子,不知怎麼立站不穩,撲的一跌。原來這背心兒賽過綁縛手,霎時間,把他兩個背剪手貼心綑了。慌得個三藏跌足報怨,急忙來解,那裡便解得開。三個人在那裡吆喝之聲不絕,卻早驚動了魔頭。

原來那座樓房果是妖精點化的,終日在此拿人。他在洞裡正坐,忽聞得怨恨之聲,急出門來看,果見綑住幾個人了。妖魔即喚小妖,同到那廂,收了樓臺房屋之形。把唐僧攙住,牽了白馬,挑了行李,將八戒、沙僧一齊捉到洞裡。老妖魔登臺高坐,眾小妖把唐僧推近臺邊,跪伏於地。妖魔問道:「你是那方和尚?怎麼這般膽大,白日裡偷盜我的衣服?」三藏滴淚告曰:「貧僧是東土大唐欽差往西天取經的。因腹中飢餒,著大徒弟去化齋未回,不曾依得他的言語,誤撞仙庭避風。不期我這兩個徒弟愛小,拿出這衣物來。貧僧決不敢壞心,當教送還本處。他不聽吾言,要穿此晤晤脊背,不料中了大王機會,把貧僧拿來。萬望慈憫,留我殘生,求取真經,永註大王恩情,回東土千古傳揚也。」那妖魔笑道:「我這裡常聽得人言:有人吃了唐僧一塊肉,髮白還黑,齒落更生。幸今日不請自來,還指望饒你哩。你那大徒弟叫做甚麼名字?往何方化齋?」八戒聞言,即開口稱揚道:「我師兄乃五百年前大鬧天宮齊天大聖孫悟空也。」

那妖魔聽說是齊天大聖孫悟空,老大有些悚懼,口內不言,心中暗想道:「久聞那廝神通廣大,如今不期而會。」教:「小的們,把唐僧綑了;將那兩個解下寶貝,換兩條繩子,也綑了。且擡在後邊,待我拿住他大徒弟,一發刷洗,卻好湊籠蒸吃。」眾小妖答應一聲,把三人一齊綑了,擡在後邊。將白馬拴在槽頭,行李挑在屋裡。眾妖都磨兵器,準備擒拿行者不題。

卻說孫行者自南莊人家攝了一缽盂齋飯,駕雲回返舊路,徑至山坡平處,按下雲頭,早已不見唐僧,不知何往,棍劃的圈子還在,只是人馬都不見了。回看那樓臺處所,亦俱無矣,惟見山根怪石。行者心驚道:「不消說了,他們定是遭那毒手也。」急依路看著馬蹄,向西而趕。

行有五六里,正在悽愴之際,只聞得北坡外有人言語。看時,乃一個老翁,氈衣蓋體,暖帽蒙頭,足下踏一雙半新半舊的油靴,手持著一根龍頭拐棒,後邊跟一個年幼的僮僕,折一枝臘梅花,自坡前念歌而走。行者放下缽盂,覿面道個問訊,叫:「老公公,貧僧問訊了。」那老翁即便回禮道:「長老那裡來的?」行者道:「我們東土來的,往西天拜佛求經,一行師徒四眾。我因師父飢了,特去化齋,教他三眾坐在那山坡平處相候。及回來不見,不知往那條路上去了。動問公公,可曾看見?」老者聞言,呵呵冷笑道:「你那三眾,可有一個長嘴大耳的麼?」行者道:「有有有。」「又有一個晦氣色臉的,牽著一匹白馬,領著一個白臉的胖和尚麼?」行者道:「是是是。」老翁道:「你們走錯路了,你休尋他,各人顧命去也。」行者道:「那白臉者是我師父,那怪樣者是我師弟。我與他共發虔心,要往西天取經,如何不尋他去?」老翁道:「我才然從此過時,看見他們錯走了路徑,闖入妖魔口裡去了。」行者道:「煩公公指教指教,是個甚麼妖魔?居於何方我好上門取索他等,往西天去也。」老翁道:「這座山叫做金山。山前有個金洞,那洞中有個獨角兕大王。那大王神通廣大,威武高強。那三眾此回斷沒命了,你若去尋他,只怕連你也難保,不如不去之為愈也。我也不敢阻你,也不敢留你,只憑你心中度量。」

行者再拜稱謝道:「多蒙公公指教。我豈有不尋之理?」把這齋飯倒與他,將這空缽盂自家收拾。那老翁放下拐棒,接了缽盂,遞與僮僕,現出本相,雙雙跪下磕頭,叫:「大聖,小神不敢隱瞞。我們兩個就是此山山神、土地,在此候接大聖。這齋飯連缽盂,小神收下,讓大聖身輕好施法力。待救唐僧出難,將此齋飯還奉唐僧,方顯得大聖至恭至孝。」行者喝道:「你這毛鬼討打。既知我到,何不早迎,卻又這般藏頭露尾,是甚道理?」土地道:「大聖性急,小神不敢造次,恐犯威顏,故此隱像告知。」行者息怒道:「你且記打。好生與我收著缽盂,待我拿那妖精去來。」土地、山神遵領。

這大聖卻才束一束虎筋絛,拽起虎皮裙,執著金箍棒,徑奔山前,找尋妖洞。轉過山崖,只見那亂石磷磷,翠崖邊有兩扇石門,門外有許多小妖,在那裡掄槍舞劍。真個是:
\begin{quote}
煙雲凝瑞,苔蘚堆青。崚嶒怪石列,崎嶇曲道縈。猿嘯鳥啼風景麗,鸞飛鳳舞若蓬瀛。向陽幾樹梅初放,弄暖千竿竹自青。陡崖之下,深澗之中,陡崖之下雪堆粉,深澗之中水結冰。兩林松柏千年秀,幾簇山茶一樣紅。
\end{quote}

這大聖觀看不盡,拽開步徑至門前,厲聲高叫道:「那小妖,你快進去與你那洞主說,我本是唐朝聖僧徒弟齊天大聖孫悟空。快教他送我師父出來,免教你等喪了性命。」

那夥小妖急入洞裡報道:「大王,前面有一個毛臉勾嘴的和尚,稱是齊天大聖孫悟空,來要他師父哩。」那魔王聞得此言,滿心歡喜道:「正要他來哩。我自離了本宮,下降塵世,更不曾試試武藝。今日他來,必是個對手。」即命小妖們取出兵器。那洞中大小群妖,一個個精神抖擻,即忙擡出一根丈二長的點鋼槍,遞與老怪。老怪傳令,教:「小的們,各要整齊。進前者賞,退後者誅!」眾妖得令,隨著老怪,走出門來,叫道:「那個是孫悟空?」

行者在傍閃過,見那魔王生得好不兇醜:
\begin{quote}
獨角參差,雙眸晃亮。頂上粗皮突,耳根黑肉光。舌長時攪鼻,口闊版牙黃。毛皮青似靛,筋攣硬如鋼。比犀難照水,像牯不耕荒。全無喘月犁雲用,倒有欺天振地強。兩隻焦筋藍靛手,雄威直挺點鋼槍。細看這等兇模樣,不枉名稱兕大王。
\end{quote}

孫大聖上前道:「你孫外公在這裡也。快早還我師父,兩無毀傷;若道半個『不』字,我教你死無葬身之地!」那魔喝道:「我把你這個大膽潑猴精!你有些甚麼手段,敢出這般大言?」行者道:「你這潑物!是也不曾見我老孫的手段。」那妖魔道:「你師父偷盜我的衣服,實是我拿住了,如今待要蒸吃。你是個甚麼好漢,就敢上我的門來取討?」行者道:「我師父乃忠良正直之僧,豈有偷你甚麼妖物之理?」妖魔道:「我在山路邊點化一座仙莊,你師父潛入裡面,心愛情慾,將我三領納錦綿裝背心兒偷穿在身,見有贓證,故此我才拿他。你今果有手段,即與我比勢:假若三合敵得我,饒了你師之命;如敵不過我,教你一路歸陰。」

行者笑道:「潑物!不須講口,但說比勢,正合老孫之意。走上來,吃吾之棒。」那怪物那怕甚麼賭鬥,挺鋼槍劈面迎來。這一場好殺!你看那:
\begin{quote}
金箍棒舉,長桿槍迎。金箍棒舉,亮爍爍似電掣金蛇;長桿槍迎,明晃晃如龍離黑海。那門前小妖擂鼓,排開陣勢助威風;這壁廂大聖施功,使出縱橫逞本事。他那裡一桿槍,精神抖擻;我這裡一條棒,武藝高強。正是英雄相遇英雄漢,果然對手才逢對手人。那魔王口噴紫氣盤煙霧,這大聖眼放光華結繡雲。只為大唐僧有難,兩家無義苦爭論。
\end{quote}

他兩個戰經三十合,不分勝負。那魔王見孫悟空棍法齊整,一往一來,全無些破綻,喜得他連聲喝采道:「好猴兒,好猴兒,真個是那鬧天宮的本事。」這大聖也愛他槍法不亂,右遮左擋,甚有解數,也叫道:「好妖精,好妖精。果然是一個偷丹的魔頭。」二人又鬥了一二十合,那魔王把槍尖點地,喝令小妖齊來。那些潑怪一個個拿刀弄杖,執劍掄槍,把個孫大聖圍在中間。行者公然不懼,只叫:「來得好,來得好,正合吾意。」使一條金箍棒,前迎後架,東擋西除。那夥群妖莫想肯退。行者忍不住焦躁,把金箍棒丟將起去,喝聲:「變!」即變作千百條鐵棒,好便似飛蛇走蟒,盈空裡亂落下來。那夥妖精見了,一個個魄散魂飛,抱頸縮頭,盡往洞中逃命。老魔王唏唏冷笑道:「那猴不要無禮,看手段。」即忙袖中取出一個亮灼灼白森森的圈子來,望空拋起,叫聲:「著!」唿喇一下,把金箍棒收做一條,套將去了。弄得孫大聖赤手空拳,翻觔斗逃了性命。那妖魔得勝回歸洞,行者朦朧失主張。這正是:
\begin{quote}
道高一尺魔高丈,性亂情昏錯認家。
可恨法身無坐位,當時行動念頭差。
\end{quote}

畢竟不知這番怎麼結果,且聽下回分解。
