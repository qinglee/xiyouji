
\chapter{長庚傳報魔頭狠 行者施為變化能}

\begin{quote}
情慾原因總一般,有情有慾自如然。
沙門修煉紛紛士,斷慾忘情即是禪。
須著意,要心堅,一塵不染月當天。
行功進步休教錯,行滿功完大覺仙。
\end{quote}

話表三藏師徒們打開慾網,跳出情牢,放馬西行。走不多時,又是夏盡秋初,新涼透體。但見那:
\begin{quote}
急雨收殘暑,梧桐一葉驚。
螢飛莎徑晚,蛩語月華明。
黃葵開映露,紅蓼遍沙汀。
蒲柳先零落,寒蟬應律鳴。
\end{quote}

三藏正然行處,忽見一座高山,峰插碧空,真個是摩星礙日。長老心中害怕,叫悟空道:「你看前面這山十分高聳,但不知有路通行否?」行者笑道:「師父說那裡話,自古道:『山高自有客行路,水深自有渡船人。』豈無通達之理?可放心前去。」長老聞言,喜笑花生,揚鞭策馬而進,徑上高岩。

行不數里,見一老者,鬢蓬鬆,白髮飄搔;鬚稀朗,銀絲擺動;項掛一串數珠子,手持拐杖現龍頭。遠遠的立在那山坡上高呼:「西進的長老,且暫住驊騮,緊兜玉勒。這山上有一夥妖魔,吃盡了閻浮世上人,不可前進。」三藏聞言,大驚失色。一是馬的足下不平,二是坐個雕鞍不穩,撲的跌下馬來,掙挫不動,睡在草裡哼哩。行者近前攙起道:「莫怕,莫怕,有我哩。」長老道:「你聽那高岩上老者報道這山上有夥妖魔,吃盡閻浮世上人,誰敢去問他一個真實端的?」行者道:「你且坐地,等我去問他。」三藏道:「你的相貌醜陋,言語粗俗,怕衝撞了他,問不出個實信。」行者笑道:「我變個俊些兒的去問他。」三藏道:「你是變了我看。」好大聖,捻著訣,搖身一變,變做個乾乾淨淨的小和尚兒,真個是目秀眉清,頭圓臉正;行動有斯文之氣象,開口無俗類之言辭。抖一抖錦衣直裰,拽步上前,向唐僧道:「師父,我看變得好麼?」三藏見了大喜道:「變得好。」八戒道:「怎麼不好?只是把我們都比下去了。老豬就滾上二三年,也變不得這等俊俏。」

好大聖,躲離了他們,徑直近前,對那老者躬身道:「老公公,貧僧問訊了。」那老兒見他生得俊雅,年少身輕,待答不答的,還了他個禮,用手摸著他頭兒,笑嘻嘻問道:「小和尚,你是那裡來的?」行者道:「我們是東土大唐來的,特上西天拜佛求經。適到此間,聞得公公報道有妖怪,我師父膽小怕懼,著我來問一聲:端的是甚妖精,他敢這般短路?煩公公細說與我知之,我好把他貶解起身。」那老兒笑道:「你這小和尚年幼,不知好歹,言不幫襯。那妖魔神通廣大得緊,怎敢就說貶解他起身?」行者笑道:「據你之言,似有護他之意,必定與他有親,或是緊鄰契友;不然,怎麼長他的威智,興他的節概,不肯傾心吐膽說他個來歷?」公公點頭笑道:「這和尚倒會弄嘴。想是跟你師父遊方,到處兒學些法術,或者會驅縛魍魎,與人家鎮宅降邪。你不曾撞見十分狠怪哩。」行者道:「怎的狠?」公公道:「那妖精一封書到靈山,五百阿羅都來迎接;一紙簡上天宮,十一大曜個個相欽。四海龍曾與他為友,八洞仙常與他作會;十地閻君以兄弟相稱,社令、城隍以賓朋相愛。」

大聖聞言,忍不住呵呵大笑,用手扯著老者道:「不要說,不要說。那妖精與我後生小廝為兄弟、朋友,也不見十分高作。若知是我小和尚來啊,他連夜就搬起身去了。」公公道:「你這小和尚胡說,不當人子。那個神聖是你的後生小廝?」行者笑道:「實不瞞你說,我小和尚祖居傲來國花果山水簾洞,姓孫,名悟空。當年也曾做過妖精,幹過大事。曾因會眾魔,多飲了幾杯酒睡著,夢中見二人將批勾我去到陰司。一時怒發,將金箍棒打傷鬼判,諕倒閻王,幾乎掀翻了森羅殿。嚇得那掌案的判官拿紙,十閻王簽名畫字,教我饒他打,情願與我做後生小廝。」那公公聞說道:「阿彌陀佛!這和尚說了這過頭話,莫想再長得大了。」行者道:「官兒,似我這般大也夠了。」公公道:「你年幾歲了?」行者道:「你猜猜看。」老者道:「有七八歲罷了。」行者笑道:「有一萬個七八歲。我把舊嘴臉拿出來你看看,你卻莫怪。」公公道:「怎麼又有個嘴臉?」行者道:「我小和尚果有七十二副嘴臉哩。」

那公公不識竅,只管問他。他就把臉抹一抹,即現出本像,咨牙徠嘴,兩股通紅,腰間繫一條虎皮裙,手裡執一根金箍棒,立在石崖之下,就像個活雷公。那老者見了,嚇得面容失色,腿腳酸麻,站不穩,撲的一跌;爬起來,又一個躘踵。大聖上前道:「老官兒,不要虛驚,我等面惡人善,莫怕,莫怕。適間蒙你好意,報有妖魔。委的有多少怪?一發累你說說,我好謝你。」那老兒戰戰兢兢,口不能言,又推耳聾,一句不應。

行者見他不言,即抽身回坡。長老道:「悟空,你來了?所問如何?」行者笑道:「不打緊,不打緊。西天有便有個把妖精兒,只是這裡人膽小,把他放在心上。沒事,沒事,有我哩。」長老道:「你可曾問他此處是甚麼山?甚麼洞?有多少妖怪?那條路通得雷音?」八戒道:「師父,莫怪我說。若論賭變化,使促掐,捉弄人,我們三五個也不如師兄;若論老實,像師兄就擺一隊伍,也不如我。」唐僧道:「正是,正是,你還老實。」八戒道:「他不知怎麼鑽過頭不顧尾的問了兩聲,不尷不尬的就跑回來了。等老豬去問他個實信來。」唐僧道:「悟能,你仔細著。」

好獃子,把釘鈀撒在腰裡,整一整皂直裰,扭扭捏捏,奔上山坡,對老者叫道:「公公,唱喏了。」那老兒見行者回去,方拄著杖掙得起來,戰戰兢兢的要走,忽見八戒,愈覺驚怕道:「爺爺呀!今夜做的甚麼惡夢,遇著這夥惡人?為先的那和尚醜便醜,還有三分人相;這個和尚,怎麼這等個碓梃嘴,蒲扇耳朵,鐵片臉,毛頸項,一分人氣兒也沒有了?」八戒笑道:「你這老公公不藏興,有些兒好褒貶人。你是怎的看我哩?我醜便醜,奈看,再停一時就俊了。」那老者見他說出人話來,只得開言問他:「你是那裡來的?」八戒道:「我是唐僧第二個徒弟,法名叫做悟能八戒。才自先問的,叫做悟空行者,是我師兄。師父怪他衝撞了公公,不曾問得實信,所以特著我來拜問。此處果是甚山?甚洞?洞裡果是甚妖精?那裡是西去大路?煩公公指示指示。」老者道:「可老實麼?」八戒道:「我生平不敢有一毫虛的。」老者道:「你莫像才來的那個和尚走花溜水的胡纏。」八戒道:「我不像他。」

公公拄著杖,對八戒說:「此山叫做八百里獅駝嶺。中間有座獅駝洞。洞裡有三個魔頭。」八戒啐了一聲:「你這老兒卻也多心,三個妖魔也費心勞力的來報遭信?」公公道:「你不怕麼?」八戒道:「不瞞你說,這三個妖魔,我師兄一棍就打死一個;我一鈀就築死一個。我還有個師弟,他一降妖杖又打死一個:三個都打死,我師父就過去了,有何難哉?」那老者笑道:「這和尚不知深淺。那三個魔頭,神通廣大得緊哩。他手下小妖,南嶺上有五千,北嶺上有五千;東路口有一萬,西路口有一萬;巡哨的有四五千,把門的也有一萬;燒火的無數,打柴的也無數:共計算有四萬七八千。這都是有名字帶牌兒的,專在此吃人。」

那獃子聞得此言,戰兢兢跑將轉來,相近唐僧,且不回話,放下鈀,在那裡出恭。行者見了,喝道:「你不回話,卻蹲在那裡怎的?」八戒道:「諕出屎來了。如今也不消說,趕早兒各自顧命去罷。」行者道:「這個獃根,我問信偏不驚恐,你去問就這等慌張失智。」長老道:「端的何如?」八戒道:「這老兒說:此山叫做八百里獅駝山。中間有座獅駝洞。洞裡有三個老妖,有四萬八千小妖,專在那裡吃人。我們若屣著他些山邊兒,就是他口裡食了。莫想去得。」三藏聞言,戰兢兢,毛骨悚然道:「悟空,如何是好?」行者笑道:「師父放心,沒大事。想是這裡有便有幾個妖精,只是這裡人膽小,把他就說出許多人,許多大,所以自驚自怪。有我哩。」八戒道:「哥哥說的是那裡話?我比你不同,我問的是實,決無虛謬之言。滿山滿谷都是妖魔,怎生前進?」行者笑道:「獃子嘴臉,不要虛驚。若論滿山滿谷之魔,只消老孫一路棒,半夜打個罄盡。」八戒道:「不羞,不羞,莫說大話。那些妖精點卯也得七八日,怎麼就打得罄盡?」行者道:「你說怎樣打?」八戒道:「憑你抓倒,綑倒,使定身法定倒,也沒有這等快的。」行者笑道:「不用甚麼抓、拿、綑縛。我把這棍子兩頭一扯,叫:『長!』就有四十丈長短。幌一幌,叫:『粗!』就有八丈圍圓粗細。往山南一滾,滾殺五千;山北一滾,滾殺五千;從東往西一滾,只怕四五萬砑做肉泥爛醬。」八戒道:「哥哥,若是這等趕麵打,或者二更時也都了了。」沙僧在傍笑道:「師父,有大師兄恁樣神通,怕他怎的?請上馬走啊。」唐僧見他們講論手段,沒奈何,只得寬心上馬而走。

正行間,不見了那報信的老者。沙僧道:「他就是妖怪,故意狐假虎威的來傳報,恐諕我們哩。」行者道:「不要忙,等我去看看。」好大聖,跳上高峰,四顧無跡,急轉面,見半空中有彩霞晃亮,即縱雲趕上看時,乃是太白金星。走到身邊,用手扯住,口口聲聲只叫他的小名道:「李長庚,李長庚,你好憊𪬯。有甚話,當面來說便好,怎麼裝做個山林之老,魘樣老孫?」金星慌忙施禮道:「大聖,報信來遲,乞勿罪,乞勿罪。這魔頭果是神通廣大,勢要崢嶸。只看你那移變化,乖巧機謀,可便過去;如若怠慢些兒,其實難去。」行者謝道:「感激,感激。果然此處難行,望老星上界與玉帝說聲,借些天兵,幫助老孫幫助。」金星道:「有有有,你只口信帶去,就是十萬天兵,也是有的。」

大聖別了金星,按落雲頭,見了三藏道:「適才那個老兒,原是太白星來與我們報信的。」長老合掌道:「徒弟,快趕上他,問他那裡另有個路,我們轉了去罷。」行者道:「轉不得。此山徑過有八百里,四週圍不知更有多少路哩,怎麼轉得?」三藏聞言,止不住眼中流淚道:「徒弟,似此艱難,怎生拜佛?」行者道:「莫哭,莫哭,一哭便膿包行了。他這報信,必有幾分虛話,只是要我們著意留心,誠所謂:『以告者,過也。』你且下馬來坐著。」八戒道:「又有甚商議?」行者道:「沒甚商議。你且在這裡用心保守師父,沙僧好生看守行李、馬匹。等老孫先上嶺打聽打聽,看前後共有多少妖怪,拿住一個,問他個詳細,教他寫個執結,開個花名,把他老老小小一一查明,吩咐他關了洞門,不許阻路,卻請師父靜靜悄悄的過去,方顯得老孫手段。」沙僧只教:「仔細,仔細。」行者笑道:「不消囑付。我這一去,就是東洋大海也湯開路,就是鐵裹銀山也撞透門。」

好大聖,唿哨一聲,縱觔斗雲,跳上高峰。扳藤負葛,平山觀看,那山裡靜悄無人。忽失聲道:「錯了,錯了,不該放這金星老兒去了,他原來恐諕我。這裡那有個甚麼妖精?他就出來跳風頑耍,必定拈槍弄棒,操演武藝,如何沒有一個?」正自家揣度,只聽得山背後叮叮噹噹、辟辟剝剝梆鈴之聲。急回頭看處,原來是個小妖兒,掮著一桿「令」字旗,腰間懸著鈴子,手裡敲著梆子,從北向南而走。仔細看他,有一丈二尺的身子。行者暗笑道:「他必是個鋪兵,想是送公文下報帖的。且等我去聽他一聽,看他說些甚話。」

好大聖,捻著訣,念個咒,搖身一變,變做個蒼蠅兒,輕輕飛在他帽子上,側耳聽之。只見那小妖走上大路,敲著梆,搖著鈴,口裡作念道:「我等巡山的,各人要謹慎隄防孫行者,他會變蒼蠅。」行者聞言,暗自驚疑道:「這廝看見我了?若未看見,怎麼就知我的名字,又知我會變蒼蠅?」原來那小妖也不曾見他,只是那魔頭不知怎麼就吩咐他這話,卻是個謠言,著他這等胡念。行者不知,反疑他看見,就要取出棒來打他,卻又停住,暗想道:「曾記得八戒問金星時,他說老妖三個,小妖有四萬七八千名。似這小妖,再多幾萬,也不打緊。卻不知這三個老魔有多大手段。等我問他一問,動手不遲。」

好大聖,你道他怎麼去問?跳下他的帽子來,釘在樹頭上,讓那小妖先行幾步。急轉身騰那,也變做個小妖兒,照依他敲著梆,搖著鈴,掮著旗,一般衣服,只是比他略長了三五寸,口裡也那般念著。趕上前叫道:「走路的,等我一等。」那小妖回頭道:「你是那裡來的?」行者笑道:「好人呀,一家人也不認得?」小妖道:「我家沒你呀。」行者道:「怎的沒我?你認認看。」小妖道:「面生,認不得,認不得。」行者道:「可知道面生。我是燒火的,你會得我少。」小妖搖頭道:「沒有,沒有。我洞裡就是燒火的那些兄弟,也沒有這個嘴尖的。」行者暗想道:「這個嘴好的變尖了些了。」即低頭,把手侮著嘴揉一揉道:「我的嘴不尖啊。」真個就不尖了。那小妖道:「你剛才是個尖嘴,怎麼揉一揉就不尖了?疑惑人子,大不好認,不是我一家的。少會少會,可疑可疑。我那大王家法甚嚴,燒火的只管燒火,巡山的只管巡山。終不然教你燒火,又教你來巡山?」行者口乖,就趁過來道:「你不知道。大王見我燒得火好,就陞我來巡山。」

小妖道:「也罷;我們這巡山的,一班有四十名,十班共四百名,各自年貌,各自名色。大王怕我們亂了班次,不好點卯,一家與我們一個牌兒為號。你可有牌兒?」行者只見他那般打扮,那般報事,遂照他的模樣變了;因不曾看見他的牌兒,所以身上沒有。好大聖,更不說沒有,就滿口應承道:「我怎麼沒牌?但只是剛才領的新牌。拿你的出來我看。」那小妖那裡知這個機關,即揭起衣服,貼身帶著個金漆牌兒,穿條絨線繩兒,扯與行者看看。行者見那牌背是個「威鎮諸魔」的金牌,正面有三個真字,是「小鑽風」。他卻心中暗想道:「不消說了,但是巡山的,必有個『風』字墜腳。」便道:「你且放下衣走過,等我拿牌兒你看。」即轉身,插下手,將尾巴梢兒的小毫毛拔下一根,捻他把,叫:「變!」即變做個金漆牌兒,也穿上個綠絨繩兒,上書三個真字,乃「總鑽風」。拿出來,遞與他看了。小妖大驚道:「我們都叫做個小鑽風,偏你又叫做個甚麼『總鑽風』。」行者幹事找絕,說話合宜,就道:「你實不知。大王見我燒得火好,把我陞個巡風;又與我個新牌,叫做『總巡風』,教我管你這一班四十名兄弟也。」那妖聞言,即忙唱喏道:「長官,長官,新點出來的,實是面生,言語衝撞,莫怪。」行者還著禮笑道:「怪便不怪你,只是一件:見面錢卻要哩,每人拿出五兩來罷。」小妖道:「長官不要忙,待我向南嶺頭會了我這一班的人,一總打發罷。」行者道:「既如此,我和你同去。」那小妖真個前走,大聖隨後相跟。

不數里,忽見一座筆峰。何以謂之筆峰?那山頭上長出一條峰來,約有四五丈高,如筆插在架上一般,故以為名。行者到邊前,把尾巴掬一掬,跳上去,坐在峰尖兒上。叫道:「鑽風,都過來。」那些小鑽風在下面躬身道:「長官,伺候。」行者道:「你可知大王點我出來之故?」小妖道:「不知。」行者道:「大王要吃唐僧,只怕孫行者神通廣大,說他會變化,只恐他變作小鑽風,來這裡屣著路徑,打探消息,把我陞作總鑽風,來查勘你們這一班可有假的?」小鑽風連聲應道:「長官,我們俱是真的。」行者道:「你既是真的,大王有甚本事,你可曉得?」小鑽風道:「我曉得。」行者道:「你曉得,快說來我聽。如若說得合著我,便是真的;若說差了一些兒,便是假的,我定拿去見大王處治。」那小鑽風見他坐在高處,弄獐弄智,呼呼喝喝的,沒奈何,只得實說道:「我大王神通廣大,本事高強,一口曾吞了十萬天兵。」行者聞說,吐出一聲道:「你是假的。」小鑽風慌了道:「長官老爺,我是真的,怎麼說是假的?」行者道:「你既是真的,如何胡說?大王身子能有多大,一口就吞了十萬天兵?」小鑽風道:「長官原來不知。我大王會變化,要大能撐天堂,要小就如菜子。因那年王母娘娘設蟠桃大會,邀請諸仙,他不曾具柬來請,我大王意欲爭天,被玉皇差十萬天兵來降我大王。是我大王變化法身,張開大口,似城門一般,用力吞將去。諕得眾天兵不敢交鋒,關了南天門。故此是一口曾吞十萬兵。」行者聞言,暗笑道:「若是講手頭之話,老孫也曾幹過。」又應聲道:「二大王有何本事?」小鑽風道:「二大王身高三丈,臥蠶眉,丹鳳眼,美人聲,匾擔牙,鼻似蛟龍。若與人爭鬥,只消一鼻子捲去,就是鐵背銅身,也就魂亡魄喪。」行者道:「鼻子捲人的妖精也好拿。」又應聲道:「三大王也有幾多手段?」小鑽風道:「我三大王不是凡間之怪物,名號雲程萬里鵬。行動時,摶風運海,振北圖南。隨身有一件兒寶貝,喚做陰陽二氣瓶。假若是把人裝在瓶中,一時三刻,化為漿水。」

行者聽說,心中暗驚道:「妖魔倒也不怕,只是仔細防他瓶兒。」又應聲道:「三個大王的本事,你倒也說得不差,與我知道的一樣。但只是那個大王要吃唐僧哩?」小鑽風道:「長官,你不知道?」行者喝道:「我比你不知些兒。因恐汝等不知底細,吩咐我來著實盤問你哩。」小鑽風道:「我大大王與二大王久住在獅駝嶺獅駝洞。三大王不在這裡住,他原住處離此西下有四百里遠近。那廂有座城,喚做獅駝國。他五百年前吃了這城國王及文武官僚,滿城大小男女也盡被他吃了乾淨,因此上奪了他的江山。如今盡是些妖怪。不知那一年打聽得東土唐朝差一個僧人去西天取經,說那唐僧乃十世修行的好人,有人吃他一塊肉,就延壽長生不老。只因怕他一個徒弟孫行者十分利害,自家一個難為,徑來此處與我這兩個大王結為兄弟,合意同心,打夥兒捉那個唐僧也。」

行者聞言,心中大怒道:「這潑魔十分無禮。我保唐僧成正果,他怎麼算計要吃我的人?」恨一聲,咬響鋼牙,掣出鐵棒,跳下高峰,把棍子望小妖頭上砑了一砑,可憐,就砑得像一個肉陀。自家見了,又不忍道:「咦!他倒是個好意,把些家常話兒都與我說了,我怎麼卻這一下子就結果了他?也罷,也罷,左右是左右。」好大聖,只為師父阻路,沒奈何幹出這件事來。就把他牌兒解下,帶在自家腰裡,將「令」字旗掮在背上,腰間掛了鈴,手裡敲著梆子。迎風捻個訣,口裡念個咒語,搖身一變,變的就像小鑽風模樣。拽回步,徑轉舊路,找尋洞府,去打探那三個老妖魔的虛實。這正是:
\begin{quote}
千般變化美猴王,萬樣騰那真本事!
\end{quote}

闖入深山,依著舊路。正走處,忽聽得人喊馬嘶之聲。即舉目觀之,原來是獅駝洞口有萬數小妖排列著槍刀劍戟,旗幟旌旄。這大聖心中暗喜道:「李長庚之言,真是不妄,真是不妄。」原來這擺列的有些路數:二百五十名作一大隊伍。他只見有四十名雜彩長旗,迎風亂舞,就知有萬名人馬。卻又自揣自度道:「老孫變作小鑽風,這一進去,那老魔若問我巡山的話,我必隨機答應。倘或一時言語差訛,認得我啊,怎生脫體?就要往外跑時,那夥把門的擋住,如何出得門去?要拿洞裡妖王,必先除了門前眾怪。」你道他怎麼除得眾怪?好大聖,想著:「那老魔不曾與我會面,就知我老孫的名頭,我且倚著我的這個名頭,仗著威風,說些大話,嚇他一嚇看。果然中土眾生有緣有分,取得經回,這一去,只消我幾句英雄之言,就嚇退那門前若干之怪;假若眾生無緣無分,取不得真經啊,就是縱然說得蓮花現,也除不得西方洞外精。」心問口,口問心,思量此計,敲著梆,搖著鈴,徑直闖到獅駝洞口。早被前營上小妖擋住道:「小鑽風來了?」行者不應,低著頭就走。

走至二層營裡,又被小妖扯住道:「小鑽風來了?」行者道:「來了。」眾妖道:「你今早巡風去,可曾撞見甚麼孫行者麼?」行者道:「撞見的,正在那裡磨杠子哩。」眾妖害怕道:「他怎麼個模樣?磨甚麼杠子?」行者道:「他蹲在那澗邊,還似個開路神;若站起來,好道有十數丈長。手裡拿著一條鐵棒,就似碗來粗細的一根大杠子,在那石崖上抄一把水,磨一磨,口裡又念著:『杠子啊,這一向不曾拿你出來顯顯神通,這一去就有十萬妖精,也都替我打死,等我殺了那三個魔頭祭你。』他要磨得明了,先打死你門前一萬精哩。」那些小妖聞得此言,一個個心驚膽戰,魂散魄飛。行者又道:「列位,那唐僧的肉也不多幾斤,也分不到我處,我們替他頂這個缸怎的?不如我們各自散一散罷。」眾妖都道:「說得是,我們各自顧命去來。」原來此輩都是些狼蟲虎豹,走獸飛禽,嗚的一聲,都鬨然而去了。這個倒不像孫大聖幾句鋪頭話,卻就如楚歌聲吹散了八千兵。

行者暗自喜道:「好了!老妖是死了。聞言就走,怎敢覿面相逢?這進去還似此言方好;若說差了,才這夥小妖有一兩個倒走進去聽見,卻不走了風汛?」你看:
\begin{quote}
他存心來古洞,仗膽入深門。
\end{quote}

畢竟不知見那個老魔頭有甚吉凶,且聽下回分解。
