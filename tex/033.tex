
\chapter{外道迷真性 元神助本心}

卻說那怪將八戒拿進洞去,道:「哥哥啊,拿將一個來了。」老魔喜道:「拿來我看。」二魔道:「這不是?」老魔道:「兄弟,錯拿了,這個和尚沒用。」八戒就綽經說道:「大王,沒用的和尚,放他出去罷。不當人子。」二魔道:「哥哥,不要放他。雖然沒用,也是唐僧一起的,叫做豬八戒。把他且浸在後邊淨水池中,浸退了毛衣,使鹽醃著,晒乾了,等天陰下酒。」八戒聽言道:「蹭蹬啊,撞著個販醃臘的妖怪了。」那小妖把八戒擡進去,拋在水裡不題。

卻說三藏坐在坡前,耳熱眼跳,身體不安,叫聲:「悟空,怎麼悟能這番巡山,去之久而不來?」行者道:「師父還不曉得他的心哩。」三藏道:「他有甚心?」行者道:「師父啊,此山若是有怪,他半步難行,一定虛張聲勢,跑將回來報我。想是無怪,路途平靜,他一直去了。」三藏道:「假若真個去了,卻在那裡相會?此間乃是山野空闊之處,比不得那店市城井之間。」行者道:「師父莫慮,且請上馬。那獃子有些懶惰,斷然走的遲慢。你把馬打動些兒,我們定趕上他,一同去罷。」真個唐僧上馬,沙僧挑擔,行者前面引路上山。

卻說那老怪又喚二魔道:「兄弟,你既拿了八戒,斷然就有唐僧。再去巡巡山來,切莫錯過他去。」二魔道:「就行,就行。」你看他急點起五十名小妖,上山巡邏。正走處,只見祥雲縹緲,瑞氣盤旋,二魔道:「唐僧來了。」眾妖道:「唐僧在那裡?」二魔道:「好人頭上祥雲照頂,惡人頭上黑氣沖天。那唐僧原是金蟬長老臨凡,十世修行的好人,所以有這祥雲縹緲。」眾怪都不看見。二魔用手指道:「那不是?」那三藏就在馬上打了一個寒噤;又一指,又打個寒噤。一連指了三指,他就一連打了三個寒噤。心神不寧道:「徒弟啊,我怎麼打寒噤麼?」沙僧道:「打寒噤想是傷食病發了。」行者道:「胡說,師父是走著這深山峻嶺,必然小心虛驚。莫怕,莫怕,等老孫把棒打一路與你壓壓驚。」

好行者,理開棒,在馬前丟幾個解數,上三下四,左五右六,盡按那六韜三略,使起神通。那長老在馬上觀之,真個是寰中少有,世上全無。剖開路一直前行,險些兒不諕倒那怪物。他在山頂上看見,魂飛魄喪,忽失聲道:「幾年間聞說孫行者,今日才知話不虛傳果是真。」眾怪上前道:「大王,怎麼長他人之志氣,滅自己之威風?你誇誰哩?」二魔道:「孫行者神通廣大,那唐僧吃他不成。」眾怪道:「大王,你沒手段,等我們著幾個去報大大王,教點起本洞大小兵來,擺開陣勢,合力齊心,怕他走了那裡去?」二魔道:「你們不曾見他那條鐵棒,有萬夫不當之勇。我洞中不過有四五百兵,怎禁得他那一棒?」眾妖道:「這等說,唐僧吃不成,卻不把豬八戒錯拿了?如今送還他罷。」二魔道:「拿便也不曾錯拿,送便也不好輕送。唐僧終是要吃,只是眼下還尚不能。」眾妖道:「這般說,還過幾年麼?」二魔道:「也不消幾年。我看見那唐僧只可善圖,不可惡取。若要倚勢拿他,聞也不得一聞。只可以善去感他,賺得他心與我心相合,卻就善中取計,可以圖之。」眾妖道:「大王如定計拿他,可用我等?」二魔道:「你們都各回本寨,但不許報與大王知道;若是驚動了他,必然走了風汛,敗了我計策。我自有個神通變化,可以拿他。」眾妖散去。

他獨跳下山來,在那道路之傍,搖身一變,變做個年老的道者。真個是怎生打扮?但見他:
\begin{quote}
星冠晃亮,鶴髮蓬鬆。羽衣圍繡帶,雲履綴黃棕。神清目朗如仙客,體健身輕似壽翁。說甚麼清牛道士,也強如素券先生。妝成假像如真像,捏作虛情似實情。
\end{quote}

他在那大路傍,妝做個跌折腿的道士,腳上血淋淋,口裡哼哼的,只叫:「救人!救人!」

卻說這三藏仗著孫大聖與沙僧,歡喜前來。正行處,只聽得叫:「師父救人!」三藏聞得,道:「善哉!善哉!這曠野山中,四下裡更無村舍,是甚麼人叫?想必是虎豹狼蟲諕倒的。」這長老兜回駿馬,叫道:「那有難者是甚人?可出來。」這怪從草科裡爬出,對長老馬前,乒乓的只情磕頭。三藏在馬上見他是個道者,卻又年紀高大,甚不過意。連忙下馬攙道:「請起,請起。」那怪道:「疼,疼,疼。」丟了手看處,只見他腳上流血。三藏驚問道:「先生啊,你從那裡來?因甚傷了尊足?」那怪巧語花言,虛情假意道:「師父啊,此山西去,有一座清幽觀宇,我是那觀裡的道士。」三藏道:「你不在本觀中侍奉香火,演習經法,為何在此閑行?」那魔道:「因前日山南里施主家邀道眾禳星散福來晚,我師徒二人一路而行。行至深衢,忽遇著一隻斑斕猛虎,將我徒弟銜去。貧道戰兢兢的無奔走,一跤跌在亂石坡上,傷了腿足,不知回路。今日大有天緣,得遇師父。萬望師父大發慈悲,救我一命。若得到觀中,就是典身賣命,一定重謝深恩。」三藏聞言,認為真實,道:「先生啊,你我都是一命之人,我是僧,你是道,衣冠雖別,修行之理則同。我不救你啊,就不是出家之輩。救便救你,你卻走不得路哩。」那怪道:「立也立不起來,怎生走路?」

三藏道:「也罷,也罷,我還走得路,將馬讓與你騎一程,到你上宮,還我馬去罷。」那怪道:「師父,感蒙厚情,只是腿胯跌傷,不能騎馬。」三藏道:「正是。」叫沙和尚:「你把行李捎在我馬上,你馱他一程罷。」沙僧道:「我馱他。」那怪急回頭,抹了他一眼,道:「師父啊,我被那猛虎諕怕了,見這晦氣色臉的師父,愈加驚怕,不敢要他馱。」三藏叫道:「悟空,你馱罷。」行者連聲答應道:「我馱,我馱。」那妖就認定了行者,順順的要他馱,再不言語。沙僧笑道:「這個沒眼色的老道。我馱著不好,顛倒要他馱。他若看不見師父時,三尖石上,把觔都摜斷了你的哩。」

行者聽了,口中笑道:「你這個潑魔,怎麼敢來惹我?你也問問老孫是幾年的人兒?你這般鬼話兒,只好瞞唐僧,又好來瞞我?我認得你是這山中的怪物,想是要吃我師父哩。我師父又非是等閑之輩,是你吃的?你要吃他,也須是分多一半與老孫是。」那魔聞得行者口中念誦,道:「師父,我是好人家兒孫,做了道士。今日不幸,遇著虎狼之厄,我不是妖怪。」行者道:「你既怕虎狼,怎麼不念北斗經?」三藏正然上馬,聞得此言,罵道:「這個潑猴!『救人一命,勝造七級浮屠。』你馱他馱兒便罷了,且講甚麼『北斗經』、『南斗經』。」行者聞言道:「這廝造化哩,我那師父是個慈悲好善之人,又有些外好裡枒槎。我待不馱你,他就怪我。馱便馱,須要與你講開:若是大小便,先和我說;若在脊梁上淋下來,臊氣不堪,且污了我的衣服,沒人漿洗。」那怪道:「我這般一把子年紀,豈不知你的話說?」行者才拉將起來,背在身上,同長老、沙僧,奔大路西行。那山上高低不平之處,行者留心慢走,讓唐僧前去。

行不上三五里路,師父與沙僧下了山凹之中,行者卻望不見,心中埋怨道:「師父偌大年紀,再不曉得事體。這等遠路,就是空身子也還嫌手重,恨不得捽了,卻又教我馱著這個妖怪。莫說他是妖怪,就是好人,這們年紀,也死得著了。摜殺他罷,馱他怎的?」這大聖正算計要摜,原來那怪就知道了,且會遣山。就使一個「移山倒海」的法術,就在行者背上捻訣,念動真言,把一座須彌山遣在空中,劈頭來壓行者。這大聖慌得把頭偏一偏,壓在左肩臂上,笑道:「我的兒,你使甚麼重身法來壓老孫哩?這個倒也不怕,只是正擔好挑,偏擔兒難挨。」那魔道:「一座山壓他不住。」卻又念咒語,把一座峨嵋山遣在空中來壓。行者又把頭偏一偏,壓在右肩臂上。看他挑著兩座大山,飛星來趕師父。那魔頭看見,就嚇得渾身是汗,遍體生津道:「他卻會擔山。」又整性情,把真言念動,將一座泰山遣在空中,劈頭壓住行者。那大聖力軟觔麻,遭逢他這泰山下頂之法,只壓得三尸神咋,七竅噴紅。

好妖魔,使神通壓倒行者,卻疾駕長風,去趕唐三藏,就於雲端裡伸下手來,馬上撾人。慌得個沙僧丟了行李,掣出降妖棒,當頭擋住。那妖魔舉一口七星劍,對面來迎。這一場好殺:
\begin{quote}
七星劍,降妖棒,萬映金光如閃亮。這個圜眼兇如黑殺神,那個鐵臉真是捲簾將。那怪山前大顯能,一心要捉唐三藏。這個努力保真僧,一心寧死不肯放。他兩個噴雲噯霧照天宮,播土揚塵遮斗象。殺得那一輪紅日淡無光,大地乾坤昏蕩蕩。來往相持八九回,不期戰敗沙和尚。
\end{quote}

那魔十分兇猛,使口寶劍,流星的解數滾來,把個沙僧戰得軟弱難搪,回頭要走。早被他逼住寶杖,掄開大手,撾住沙僧,挾在左脅下;將右手去馬上拿了三藏,腳尖兒鉤著行李,張開口咬著馬鬃;使起攝法,把他們一陣風,都拿到蓮花洞裡,厲聲高叫道:「哥哥!這和尚都拿來了。」

老魔聞言,大喜道:「拿來我看。」二魔道:「這不是?」老魔道:「賢弟呀,又錯拿來了也。」二魔道:「你說拿唐僧的。」老魔道:「是便就是唐僧,只是還不曾拿住那有手段的孫行者。須是要拿住他,才好吃唐僧哩;若不曾拿得他,切莫動他的人。那猴王神通廣大,變化多般,我們若吃了他師父,他肯甘心?來那門前吵鬧,莫想能得安生。」二魔笑道:「哥啊,你也忒會擡舉人。若依你誇獎他,天上少有,地下全無;自我觀之,也只如此,沒甚手段。」老魔道:「你拿住了?」二魔道:「他已被我遣三座大山壓在山下,寸步不能舉移。所以才把唐僧、沙和尚連馬、行李,都攝將來也。」那老魔聞言,滿心歡喜道:「造化,造化。拿住這廝,唐僧才是我們口裡的食哩。」叫小妖:「快安排酒來,且與你二大王奉一個得功的杯兒。」二魔道:「哥哥,且不要吃酒,叫小的們把豬八戒撈上水來吊起。」遂把八戒吊在東廊,沙僧吊在西邊,唐僧吊在中間,白馬送在槽上,行李收將進去。

老魔笑道:「賢弟好手段,兩次捉了三個和尚。但孫行者雖是有山壓住,也須要作個法,怎麼拿他來湊蒸,才好哩。」二魔道:「兄長請坐。若要拿孫行者,不消我們動身,只教兩個小妖,拿兩件寶貝,把他裝將來罷。」老魔道:「拿甚麼寶貝去?」二魔道:「拿我的紫金紅葫蘆,你的羊脂玉淨瓶。」老魔將寶貝取出道:「差那兩個去?」二魔道:「差精細鬼、伶俐蟲二人去。」吩咐道:「你兩個拿著這寶貝,徑至高山絕頂,將底兒朝天,口兒朝地,叫一聲:『孫行者。』他若應了,就已裝在裡面,隨即貼上『太上老君急急如律令奉敕』的帖兒,他就一時三刻化為膿了。」二小妖叩頭,將寶貝領出去拿行者不題。

卻說那大聖被魔使法壓住在山根之下,遇苦思三藏,逢災念聖僧。厲聲叫道:「師父啊,想當時你到兩界山,揭了壓帖,老孫脫了大難,秉教沙門。感菩薩賜與法旨,我和你同住同修,同緣同相,同見同知。乍想到了此處,遭逢魔障,又被他遣山壓了。可憐,可憐!你死該當,只難為沙僧、八戒與那小龍化馬一場。這正是:樹大招風風撼樹,人為名高名喪人。」嘆罷,那珠淚如雨。

早驚了山神、土地與五方揭諦神眾,會金頭揭諦道:「這山是誰的?」土地道:「是我們的。」「你山下壓的是誰?」土地道:「不知是誰。」揭諦道:「你等原來不知。這壓的是五百年前大鬧天宮的齊天大聖孫悟空行者。如今皈依正果,跟唐僧做了徒弟。你怎麼把山借與妖魔壓他?你們是死了,他若有一日脫身出來,他肯饒你?就是從輕,土地也問個擺站,山神也問個充軍,我們也領個大不應是。」那山神、土地才怕道:「委實不知,不知。只聽得那魔頭念起遣山咒法,我們就把山移將來了。誰曉得是孫大聖?」揭諦道:「你且休怕。律上有云:『不知者不坐罪。』我與你計較,放他出來,不要教他動手打我們。」土地道:「就沒理了,既放出來又打?」揭諦道:「你不知。他有一條如意金箍棒,十分利害:打著的就死,挽著的就傷;磕一磕兒觔斷,擦一擦兒皮塌哩。」

那土地、山神心中恐懼,與五方揭諦商議了,卻來到三山門外叫道:「大聖,山神、土地、五方揭諦來見。」好行者,他虎瘦雄心還在,自然的氣象昂昂,聲音朗朗道:「見我怎的?」土地道:「告大聖得知:遣開山,請大聖出來,赦小神不恭之罪。」行者道:「遣開山,不打你。」喝聲:「起去!」就如官府發放一般,那眾神念動真言咒語,把山仍遣歸本位,放起行者。行者跳將起來,抖抖土,束束裙,耳後掣出棒來,叫山神、土地:「都伸過孤拐來,每人先打兩下,與老孫散散悶。」眾神大驚道:「剛才大聖已吩咐,恕我等之罪,怎麼出來就變了言語要打?」行者道:「好土地,好山神,你道不怕老孫,卻怕妖怪?」土地道:「那魔神通廣大,法術高強,念動真言咒語,拘喚我等在他洞裡,一日一個輪流當值哩。」

行者聽見「當值」二字,卻也心驚。仰面朝天,高聲大叫道:「蒼天,蒼天!自那混沌初分,天開地闢,花果山生了我,我也曾遍訪明師,傳授長生秘訣。想我那隨風變化,伏虎降龍,大鬧天宮,名稱大聖,更不曾把山神、土地欺心使喚。今日這個妖魔無狀,怎敢把山神、土地喚為奴僕,替他輪流當值?天啊!既生老孫,怎麼又生此輩?」

那大聖正感嘆間,又見那山凹裡霞光焰焰而來。行者道:「山神、土地,你既在這洞中當值,那放光的是甚物件?」土地道:「那是妖魔的寶貝放光,想是有妖精拿寶貝來降你。」行者道:「這個卻好耍子兒啊。我且問你,他這洞中有甚人與他相往?」土地道:「他愛的是燒丹煉藥,喜的是全真道人。」行者道:「怪道他變個老道士,把我師父騙去了。既這等,你都且記打,回去罷。等老孫自家拿他。」那眾神俱騰空而散。

這大聖搖身一變,變做個老真人。你道他怎生打扮:
\begin{quote}
頭挽雙髽髻,身穿百衲衣。
手敲漁鼓簡,腰繫呂公絛。
斜倚大路下,專候小魔妖。
頃刻妖來到,猴王暗放刁。
\end{quote}

不多時,那兩個小妖到了。行者將金箍棒伸開,那妖不曾防備,絆著腳,撲的一跌。爬起來,才看見行者,口裡嚷道:「憊𪬯,憊𪬯!若不是我大王敬重你這行人,就和你比較起來。」行者陪笑道:「比較甚麼?道人見道人,都是一家人。」那怪道:「你怎麼睡在這裡絆我一跌?」行者道:「小道童見我這老道人,要跌一跌兒做見面錢。」那妖道:「我大王見面錢只要幾兩銀子,你怎麼跌一跌兒做見面錢?你別是一鄉風,決不是我這裡道士。」行者道:「我當真不是,我是蓬萊山來的。」那妖道:「蓬萊山是海島神仙境界。」行者道:「我不是神仙,誰是神仙?」那妖卻回嗔作喜,上前道:「老神仙,老神仙,我等肉眼凡胎,不能識認,言語衝撞,莫怪,莫怪。」行者道:「我不怪你。常言道:『仙體不踏凡地。』你怎知之?我今日到你山上,要度一個成仙了道的好人。那個肯跟我去?」精細鬼道:「師父,我跟你去。」伶俐蟲道:「師父,我跟你去。」

行者明知故問道:「你二位從那裡來的?」那怪道:「自蓮花洞來的。」「要往那裡去?」那怪道:「奉我大王教命,拿孫行者去的。」行者道:「拿那個?」那怪又道:「拿孫行者。」孫行者道:「可是跟唐僧取經的那個孫行者麼?」那妖道:「正是,正是。你也認得他?」行者道:「那猴子有些無禮。我認得他,我也有些惱他。我與你同拿他去,就當與你助功。」那怪道:「師父,不須你助功。我二大王有些法術,遣了三座大山,把他壓在山下,寸步難移,教我兩個拿寶貝來裝他的。」行者道:「是甚寶貝?」精細鬼道:「我的是紅葫蘆,他的是玉淨瓶。」行者道:「怎麼樣裝他?」小妖道:「把這寶貝的底兒朝天,口兒朝地,叫他一聲,他若應了,就裝在裡面;貼上一張『太上老君急急如律令奉敕』的帖子,他就一時三刻,化為膿了。」行者見說,心中暗驚道:「利害,利害!當時日值功曹報信,說有五件寶貝,這是兩件了。不知那三件又是甚麼東西?」行者笑道:「二位,你把寶貝借我看看。」那小妖那知甚麼訣竅,就於袖中取出兩件寶貝,雙手遞與行者。行者見了,心中暗喜道:「好東西,好東西。我若把尾子一抉,颼的跳起走了,只當是送老孫。」忽又思道:「不好,不好。搶便搶去,只是壞了老孫的名頭。這叫做白日搶奪了。」復遞與他去道:「你還不曾見我的寶貝哩。」那怪道:「師父有甚寶貝?也借與我凡人看看壓災。」

好行者,伸下手,把尾上毫毛拔了一根,捻一捻,叫:「變!」即變做一個一尺七寸長的大紫金紅葫蘆,自腰裡拿將出來道:「你看我的葫蘆麼?」那伶俐蟲接在手,看了道:「師父,你這葫蘆長大,有樣範,好看,卻只是不中用。」行者道:「怎的不中用?」那怪道:「我這兩件寶貝,每一個可裝千人哩。」行者道:「你這裝人的,何足稀罕?我這葫蘆,連天都裝在裡面哩。」那怪道:「就可以裝天?」行者道:「當真的裝天。」那怪道:「只怕是謊,就裝與我們看看才信;不然,決不信你。」行者道:「天若惱著我,一月之間,常裝他七八遭,不惱著我,就半年也不裝他一次。」伶俐蟲道:「哥啊,裝天的寶貝,與他換了罷。」精細鬼道:「他裝天的,怎肯與我裝人的相換?」伶俐蟲道:「若不肯啊,貼他這個淨瓶也罷。」行者心中暗喜道:「葫蘆換葫蘆,餘外貼淨瓶:一件換兩件,其實甚相應。」即上前扯住那伶俐蟲道:「裝天可換麼?」那怪道:「但裝天就換,不換我是你的兒子。」行者道:「也罷,也罷,我裝與你們看看。」

好大聖,低頭捻訣,念個咒語,叫那日遊神、夜遊神、五方揭諦神:「即去與我奏上玉帝,說老孫皈依正果,保唐僧去西天取經,路阻高山,師逢苦厄。妖魔那寶,吾欲誘他換之。萬千拜上,將天借與老孫裝閉半個時辰,以助成功,若道半聲不肯,即上靈霄殿,動起刀兵。」

那日遊神徑至南天門裡,靈霄殿下,啟奏玉帝,備言前事。玉帝道:「這潑猴頭,出言無狀。前者觀音來說放了他,保護唐僧,朕這裡又差五方揭諦、四值功曹,輪流護持。如今又借天裝,天可裝乎?」才說裝不得,那班中閃出哪吒三太子,奏道:「萬歲,天也裝得。」玉帝道:「天怎樣裝?」哪吒道:「自混沌初分,以輕清為天,重濁為地。天是一團清氣而扶托瑤天宮闕,以理論之,其實難裝;但只孫行者保唐僧西去取經,誠所謂泰山之福緣,海深之善慶,今日當助他成功。」玉帝道:「卿有何助?」哪吒道:「請降旨意,往北天門問真武借皂雕旗,在南天門上一展,把那日月星辰閉了。對面不見人,捉白不見黑,哄那怪道,只說裝了天,以助行者成功。」玉帝聞言:「依卿所奏。」那太子奉旨,前來北天門,見真武,備言前事。那祖師隨將旗付太子。

早有遊神急降大聖耳邊道:「哪吒太子來助功了。」行者仰面觀之,只見祥雲繚繞,果是有神。卻回頭對小妖道:「裝天罷。」小妖道:「要裝就裝,只管阿綿花屎怎的?」行者道:「我方才運神念咒來。」那小妖都睜著眼,看他怎麼樣裝天。這行者將一個假葫蘆兒拋將上去。你想,這是一根毫毛變的,能有多重?被那山頂上風吹去,飄飄蕩蕩,足有半個時辰,方才落下。只見那南天門上,哪吒太子把皂旗撥喇喇展開,把日月星辰俱遮閉了。真是乾坤墨染就,宇宙靛裝成。二小妖大驚道:「才說話時,只好向午,卻怎麼就黃昏了?」行者道:「天既裝了,不辨時候,怎不黃昏?」「如何又這等樣黑?」行者道:「日月星辰都裝在裡面,外卻無光,怎麼不黑?」小妖道:「師父,你在那廂說話哩?」行者道:「我在你面前不是?」小妖伸手摸著道:「只見說話,更不見面目。師父,此間是甚麼去處?」行者又哄他道:「不要動腳,此間乃是渤海岸上,若塌了腳,落下去啊,七八日還不得到底哩。」小妖大驚道:「罷罷罷,放了天罷,我們曉得是這樣裝了。若弄一會子,落下海去,不得歸家。」

好行者,見他認了真實,又念咒語,驚動太子,把旗捲起,卻早見日光正午。小妖笑道:「妙啊!妙啊!這樣好寶貝,若不換啊,誠為不是養家的兒子。」那精細鬼交了葫蘆,伶俐蟲拿出淨瓶,一齊兒遞與行者。行者卻將假葫蘆兒遞與小妖換了。既換了寶貝,卻又幹事找絕:臍下拔一根毫毛,吹口仙氣,變作一個銅錢。叫道:「小童,你拿這個錢去買張紙來。」小妖道:「何用?」行者道:「我與你寫個合同文書。你將這兩件裝人的寶貝換了我一件裝天的寶貝,恐人心不平,向後去日久年深,有甚反悔不便,故寫此各執為照。」小妖道:「此間又無筆墨,寫甚文書?我與你賭個咒罷。」行者道:「怎麼樣賭?」小妖道:「我兩件裝人之寶,貼換你一件裝天之寶,若有反悔,一年四季遭瘟。」行者笑道:「我是決不反悔;如有反悔,也照你四季遭瘟。」

說了誓,將身一縱,把尾子趬了一趬,跳在南天門前,謝了哪吒太子麾旗相助之功。太子回宮繳旨,將旗送還真武不題。這行者佇立霄漢之間,觀看那個小妖。

畢竟不知怎生區處,且聽下回分解。
