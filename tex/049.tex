
\chapter{三藏有災沉水宅 觀音救難現魚籃}

卻說孫大聖與八戒、沙僧辭陳老來至河邊,道:「兄弟,你兩個議定,那一個先下水?」八戒道:「哥啊,我兩個手段不見怎的,還得你先下水。」行者道:「不瞞賢弟說,若是山裡妖精,全不用你們費力;水中之事,我去不得。就是下海行江,我須要捻著避水訣,或者變化甚麼魚蟹之形,才去得;若是那般捻訣,卻掄不得鐵棒,使不得神通,打不得妖怪。我久知你兩個乃慣水之人,所以要你兩個下去。」沙僧道:「哥啊,小弟雖是去得,但不知水底如何。我等大家都去。哥哥變作甚麼模樣,或是我馱著你,分開水道,尋著妖怪的巢穴,你先進去打聽打聽。若是師父不曾傷損,還在那裡,我們好努力征討;假若不是這怪弄法,或者渰死師父,或者被妖吃了,我等不須苦求,早早的別尋道路何如?」行者道:「賢弟說得有理。你們那個馱我?」八戒暗喜道:「這猴子不知捉弄了我多少,今番原來不會水,等老豬馱他,也捉弄他捉弄。」獃子笑嘻嘻的叫道:「哥哥,我馱你。」行者就知有意,卻便將計就計道:「是,也好,你比悟淨還有些膂力。」八戒就背著他。

沙僧剖開水路,弟兄們同入通天河內。向水底下行有百十里遠近,那獃子要捉弄行者。行者隨即拔下一根毫毛,變做假身,伏在八戒背上;真身變作一個豬虱子,緊緊的貼在他耳朵裡。八戒正行,忽然打個躘踵,得故子把行者往前一摜,撲的跌了一跤。原來那個假身本是毫毛變的,卻就飄起去,無影無形。沙僧道:「二哥,你是怎麼說?不好生走路,就跌在泥裡,便也罷了,卻把大哥不知跌在那裡去了。」八戒道:「那猴子不禁跌,一跌就跌化了。兄弟,莫管他死活,我和你且去尋師父去。」沙僧道:「不好,還得他來。他雖不知水性,他比我們乖巧。若無他來,我不與你去。」行者在八戒耳朵裡,忍不住高叫道:「悟淨,老孫在這裡也。」沙僧聽得,笑道:「罷了,這獃子是死了,你怎麼就敢捉弄他?如今弄得聞聲不見面,卻怎是好?」八戒慌得跪在泥裡磕頭道:「哥哥,是我不是了。待救了師父,上岸陪禮。你在那裡做聲?就諕殺我也。你請現原身出來,我馱著你,再不敢衝撞你了。」行者道:「是你還馱著我哩。我不弄你,你快走,快走。」那獃子絮絮叨叨,只管念誦著陪禮,爬起來與沙僧又進。

行了又有百十里遠近,忽擡頭望見一座樓臺,上有「水黿之第」四個大字。沙僧道:「這廂想是妖精住處,我兩個不知虛實,怎麼上門索戰?」行者道:「悟淨,那門裡外可有水麼?」沙僧道:「無水。」行者道:「既無水,你再藏隱在左右,待老孫去打聽打聽。」

好大聖,爬離了八戒耳朵裡,卻又搖身一變,變作個長腳蝦婆,兩三跳跳到門裡。睜眼看時,只見那怪坐在上面,眾水族擺列兩邊,有個斑衣鱖婆坐於側手,都商議要吃唐僧。行者留心,兩邊尋找不見。忽看見一個大肚蝦婆走將來,徑往西廊下立定。行者跳到面前,稱呼道:「姆姆,大王與眾商議要吃唐僧,唐僧卻在那裡?」蝦婆道:「唐僧被大王降雪結冰,昨日拿在宮後石匣中間。只等明日,他徒弟們不來吵鬧,就奏樂享用也。」

行者聞言,演了一會,徑直尋到宮後看,果有一個石匣,卻像人家槽房裡的豬槽,又似人間一口石棺材之樣,量量足有六尺長短。卻伏在上面,聽了一會,只聽得三藏在裡面嚶嚶的哭哩。行者不言語,側耳再聽,那師父銼得牙響,哏了一聲道:
\begin{quote}
自恨江流命有愆,生時多少水災纏。
出娘胎腹淘波浪,拜佛西天墮渺淵。
前遇黑河身有難,今逢冰解命歸泉。
不知徒弟能來否,可得真經返故園?
\end{quote}

行者忍不住叫道:「師父莫恨水災。經云:『土乃五行之母,水乃五行之源。無土不生,無水不長。』老孫來了。」三藏聞得道:「徒弟啊,救我耶。」行者道:「你且放心,待我們擒住妖精,管教你脫難。」三藏道:「快些兒下手,再停一日,足足悶殺我也。」行者道:「沒事,沒事。我去也!」

急回頭,跳將出去,到門外現了原身,叫:「八戒。」那獃子與沙僧近前道:「哥哥,如何?」行者道:「正是此怪騙了師父。師父未曾傷損,被怪物蓋在石匣之下。你兩個快早挑戰,讓老孫先出水面。你若擒得他就擒;擒不得,做個佯輸,引他出水,等我打他。」沙僧道:「哥哥放心先去,待小弟們鑒貌辨色。」這行者捻著避水訣,鑽出河中,停立岸邊等候不題。

你看那豬八戒行兇,闖至門前,厲聲高叫:「潑怪物!送我師父出來。」慌得那門裡小妖急報:「大王,門外有人要師父哩。」妖邪道:「這定是那潑和尚來了。」教:「快取披掛兵器來。」眾小妖連忙取出。妖邪結束了,執兵器在手,即命開門,走將出來。八戒與沙僧對列左右,見妖邪怎生披掛?好怪物,你看他:
\begin{quote}
頭戴金盔晃且輝,身披金甲掣虹霓。
腰圍寶帶團珠翠,足踏煙黃靴樣奇。
鼻準高隆如嶠聳,天庭廣闊若龍儀。
眼光閃灼圓還暴,牙齒鋼鋒尖又齊。
短髮蓬鬆飄火焰,長鬚瀟灑挺金錐。
口咬一枝青嫩藻,手拿九瓣赤銅鎚。
一聲咿啞門開處,響似三春驚蟄雷。
這等形容人世少,敢稱靈顯大王威。
\end{quote}

妖邪出得門來,隨後有百十個小妖,一個個掄槍舞劍,擺開兩哨。對八戒道:「你是那寺裡和尚?為甚到此喧嚷?」八戒喝道:「我把你這打不死的潑物!你前夜與我頂嘴,今日如何推不知來問我?我本是東土大唐聖僧之徒弟,往西天拜佛求經者。你弄玄虛,假做甚麼靈感大王,專在陳家莊要吃童男童女。我本是陳清家一秤金,你不認得我麼?」那妖怪道:「你這和尚,甚沒道理。你變做一秤金,該一個冒名頂替之罪。我倒不曾吃你,反被你傷了我手背。已此讓了你,你怎麼又尋上我的門來?」八戒道:「你既讓我,卻怎麼又弄冷風,下大雪,凍結堅冰,害我師父?快早送我師父出來,萬事皆休;牙迸半個『不』字,你只看看手中鈀,決不饒你。」妖邪聞言,微微冷笑道:「這和尚賣此長舌,胡誇大口。果然是我作冷下雪凍河,攝你師父。你今嚷上門來,思量取討,只怕這一番不比那一番了:那時節,我因赴會,不曾帶得兵器,誤中你傷;你如今且休要走,我與你交敵三合。三合敵得我過,還你師父;敵不過,連你一發吃了。」

八戒道:「好乖兒子,正是這等說。仔細看鈀。」妖邪道:「你原來是半路上出家的和尚。」八戒道:「我的兒,你真個有些靈感,怎麼就曉得我是半路出家的?」妖邪道:「你會使鈀,想是雇在那裡種園,把他釘鈀拐將來也。」八戒道:兒子,我這鈀,不是那築地之鈀。你看:
\begin{quote}
巨齒鑄就如龍爪,遜金妝來似蟒形。
若逢對敵寒風灑,但遇相持火焰生。
能與聖僧除怪物,西方路上捉妖精。
掄動煙雲遮日月,使開霞彩照分明。
築倒太山千虎怕,掀翻大海萬龍驚。
饒你威靈有手段,一築須教九窟窿。」
\end{quote}

那個妖邪那裡肯信,舉銅鎚劈頭就打。八戒使釘鈀架住道:「你這潑物,原來也是半路上成精的邪魔。」那怪道:「你怎麼認得我是半路上成精的?」八戒道:「你會使銅鎚,想是雇在那個銀匠家扯爐,被你得了手,偷將出來的。」妖邪道:「這不是打銀之鎚。你看:
\begin{quote}
九瓣攢成花骨朵,一竿虛孔萬年青。
原來不比凡間物,出處還從仙苑名。
綠房紫菂瑤池老,素質清香碧沼生。
因我用功摶鍊過,堅如鋼銳徹通靈。
槍刀劍戟渾難賽,鉞斧戈矛莫敢經。
縱讓他鈀能利刃,湯著吾鎚迸折釘。」
\end{quote}

沙和尚見他兩個攀話,忍不住近前高叫道:「那怪物休得朗言。古人云:『口說無憑,做出便見。』不要走,且吃我一杖。」妖邪使鎚桿架住道:「你也是半路裡出家的和尚。」沙僧道:「你怎麼認得?」妖邪道:「你這模樣,像一個磨博士出身。」沙僧道:「如何認得我像個磨博士?」妖邪道:「你不是磨博士,怎麼會使趕麵杖?」沙僧罵道:「你這孽障,是也不曾見:
\begin{quote}
這般兵器人間少,故此難知寶杖名。
出自月宮無影處,梭羅仙木琢磨成。
外邊嵌寶霞光耀,內裡鑽金瑞氣凝。
先日也曾陪御宴,今朝秉正保唐僧。
西方路上無知識,上界宮中有大名。
喚做降妖真寶杖,管教一下碎天靈。」
\end{quote}

那妖邪不容分說,三家變臉,這一場在水底下好殺:
\begin{quote}
銅鎚寶杖與釘鈀,悟能悟淨戰妖邪。一個是天蓬臨世界,一個是上將降天涯。他兩個夾攻水怪施威武,這一個獨抵神僧勢可誇。有分有緣成大道,相生相剋秉恆沙。土剋水,水乾見底;水生木,木旺開花。禪法參修歸一體,還丹炮煉伏三家。土是母,發金芽,金生神水產嬰娃;水為本,潤木華,木有輝煌烈火霞。攢簇五行皆別異,故然變臉各爭差。看他那銅鎚九瓣光明好,寶杖千絲彩繡佳。鈀按陰陽分九曜,不明解數亂如麻。捐軀棄命因僧難,捨死忘生為釋迦。致使銅鎚忙不墜,左遮寶杖右遮鈀。
\end{quote}

三人在水底下鬥經兩個時辰,不分勝敗。豬八戒料道不得贏他,對沙僧丟了個眼色。二人詐敗佯輸,各拖兵器,回頭就走。那怪物教:「小的們,扎住在此,等我追趕上這廝,捉將來與汝等湊吃啞。」你看他如風吹敗葉,似雨打殘花,將他兩個趕出水面。

那孫大聖在東岸上眼不轉睛,只望著河邊水勢。忽然見波浪翻騰,喊聲號吼,八戒先跳上岸道:「來了,來了。」沙僧也到岸邊道:「來了,來了。」那妖邪隨後叫:「那裡走?」才出頭,被行者喝道:「看棍。」那妖邪閃身躲過,使銅鎚急架相還。一個在河邊湧浪,一個在岸上施威。搭上手未經三合,那妖遮架不住,打個花,又淬於水裡,遂此風平浪息。

行者回轉高崖道:「兄弟們,辛苦啊。」沙僧道:「哥啊,這妖精他在岸上覺得不濟,在水底也盡利害哩,我與二哥左右齊攻,只戰得個兩平。卻怎麼處置,救師父也?」行者道:「不必疑遲,恐被他傷了師父。」八戒道:「哥哥,我這一去哄他出來,你莫做聲,但只在半空中等候。估著他鑽出頭來,卻使個搗蒜打,照他頂門上著著實實一下。縱然打不死他,好道也護疼發暈,卻等老豬趕上一鈀,管教他了帳。」行者道:「正是,正是,這叫做『裡迎外合』,方可濟事。」他兩個復入水中不題。

卻說那妖邪敗陣逃生,回歸本宅,眾妖接到宮中,鱖婆上前問道:「大王趕那兩個和尚到那方來?」妖邪道:「那和尚原來還有一個幫手。他兩個跳上岸去,那幫手掄一條鐵棒打我,我閃過與他相持。也不知他那棍子有多少斤重,我的銅鎚莫想架得他住,戰未三合,我卻敗回來也。」鱖婆道:「大王,可記得那幫手是甚相貌?」妖邪道:「是一個毛臉雷公嘴、查耳朵、折鼻梁、火眼金睛和尚。」鱖婆聞說,打了一個寒噤道:「大王啊,虧了你識俊,逃了性命;若再三合,決然不得全生。那和尚我認得他。」妖邪道:「你認得他是誰?」鱖婆道:「我當年在東洋海內,曾聞得老龍王說他的名譽,乃是五百年前大鬧天宮混元一氣上方太乙金仙美猴王齊天大聖。如今歸依佛教,保唐僧往西天取經,改名喚做孫悟空行者。他的神通廣大,變化多端,大王你怎麼惹他?今後再莫與他戰了。」

說不了,只見門裡小妖來報:「大王,那兩個和尚又來門前索戰哩。」妖精道:「賢妹所見甚長,再不出去,看他怎麼。」急傳令教:「小的們,把門關緊了。正是『任君門外叫,只是不開門』。讓他纏兩日,性攤了回去時,我們卻不自在受用唐僧也?」那小妖一齊都搬石頭,塞泥塊,把門閉殺。

八戒與沙僧連叫不出,獃子心焦,就使釘鈀築門。那門已此緊閉牢關,莫想能夠。被他七八鈀,築破門扇,裡面卻都是泥土石塊,高疊千層。沙僧見了道:「二哥,這怪物懼怕之甚,閉門不出,我和你且回上河崖,再與大哥計較去來。」八戒依言,徑轉東岸。

那行者半雲半霧,提著鐵棒等哩。看見他兩個上來,不見妖怪,即按雲頭,迎至岸邊,問道:「兄弟,那話兒怎麼不上來?」沙僧道:「那怪物緊閉宅門,再不出來見面。被二哥打破門扇看時,那裡面都是些泥土石塊,實實的疊住了。故此不能得戰,卻來與哥哥計議,再怎麼設法去救師父。」行者道:「似這般卻也無法可治。你兩個只在河岸上巡視著,不可放他往別處走了,待我去來。」八戒道:「哥哥,你往那裡去?」行者道:「我上普陀巖拜問菩薩,看這妖怪是那裡出身,姓甚名誰。尋著他的祖居,拿了他的家屬,捉了他的四鄰,卻來此擒怪救師。」八戒笑道:「哥啊,這等幹,只是忒費事,擔擱了時辰了。」行者道:「管你不費事,不擔擱,我去就來。」

好大聖,急縱祥光,躲離河口,徑赴南海。那裡消半個時辰,早望見落伽山不遠。低下雲頭,徑至普陀崖上。只見那二十四路諸天與守山大神、木叉行者、善財童子、捧珠龍女,一齊上前,迎著施禮道:「大聖何來?」行者道:「有事要見菩薩。」眾神道:「菩薩今早出洞,不許人隨,自入竹林裡觀玩。知大聖今日必來,吩咐我等在此候接大聖,不可就見。請在翠巖前聊坐片時,待菩薩出來。」

行者依言,還未坐下,又見那善財童子上前施禮道:「孫大聖,前蒙盛意,幸菩薩不棄收留,早晚不離左右,專侍蓮臺之下,甚得善慈。」行者知是紅孩兒,笑道:「你那時節魔孽迷心,今朝得成正果,才知老孫是好人也。」

行者久等不見,心焦道:「列位與我傳報一聲,但遲了,恐傷吾師之命。」諸天道:「不敢報,菩薩吩咐,只等他自出來哩。」行者性急,那裡等得,急縱身往裡便走。噫!
\begin{quote}
這個美猴王,性急能鵲薄。
諸天留不住,要往裡邊蹕。
拽步入深林,睜眼偷覷著。
遠觀救苦尊,盤坐襯殘箬。
懶散怕梳妝,容顏多綽約。
散挽一窩絲,未曾戴纓絡。
不掛素藍袍,貼身小襖縛。
漫腰束錦裙,赤了一雙腳。
披肩繡帶無,精光兩臂膊。
玉手執鋼刀,正把竹皮削。
\end{quote}
行者見了,忍不住厲聲高叫道:「菩薩,弟子孫悟空志心朝禮。」菩薩教:「外面俟候。」行者叩頭道:「菩薩,我師父有難,特來拜問通天河妖怪根源。」菩薩道:「你且出去,待我出來。」

行者不敢強,只得走出竹林,對眾諸天道:「菩薩今日又重置家事哩。怎麼不坐蓮臺,不妝飾,不喜歡,在林裡削篾做甚?」諸天道:「我等卻不知。今早出洞,未曾妝束,就入林中去了。又教我等在此接候大聖,必然為大聖有事。」行者沒奈何,只得等候。

不多時,只見菩薩手提一個紫竹籃兒,出林道:「悟空,我與你救唐僧去來。」行者慌忙跪下道:「弟子不敢催促,且請菩薩著衣登座。」菩薩道:「不消著衣,就此去也。」那菩薩撇下諸天,縱祥雲騰空而去。孫大聖只得相隨。

頃刻間,到了通天河界。八戒與沙僧看見道:「師兄性急,不知在南海怎麼亂嚷亂叫,把一個未梳妝的菩薩逼將來也。」說不了,到於河岸。二人下拜道:「菩薩,我等擅干,有罪,有罪。」菩薩即解下一根束襖的絲絛,將籃兒拴定,提著絲絛,半踏雲彩,拋在河中,往上溜頭扯著,口念頌子道:「死的去,活的住。死的去,活的住!」念了七遍,提起籃兒,但見那籃裡亮灼灼一尾金魚,還斬眼動鱗。菩薩叫:「悟空,快下水救你師父耶。」行者道:「未曾拿住妖邪,如何救得師父?」菩薩道:「這籃兒裡不是?」八戒與沙僧拜問道:「這魚兒怎生有那等手段。」菩薩道:「他本是我蓮花池裡養大的金魚,每日浮頭聽經,修成手段。那一柄九瓣銅鎚,乃是一枝未開的菡萏,被他運鍊成兵。不知是那一日海潮泛漲,走到此間。我今早扶欄看花,卻不見這廝出拜。掐指巡紋,算著他在此成精,害你師父,故此未及梳妝,運神功,織個竹籃兒擒他。」

行者道:「菩薩,既然如此,且待片時,我等叫陳家莊眾信人等,看看菩薩的金面:一則留恩;二來說此收怪之事,好教凡人信心供養。」菩薩道:「也罷,你快去叫來。」那八戒與沙僧一齊飛跑至莊前,高呼道:「都來看活觀音菩薩,都來看活觀音菩薩。」一莊老幼男女,都向河邊,也不顧泥水,都跪在裡面,磕頭禮拜。內中有善圖畫者,傳下影神,這才是魚籃觀音現身。當時菩薩就歸南海。

八戒與沙僧分開水道,徑往那水黿之第找尋師父。原來那裡邊水怪魚精,盡皆死爛。卻入後宮,揭開石匣,馱著唐僧,出離波津,與眾相見。那陳清兄弟叩頭稱謝道:「老爺不依小人勸留,致令如此受苦。」行者道:「不消說了。你們這裡人家,下年再不用祭賽,那大王已此除根,永無傷害。陳老兒,如今才好累你,快尋一隻船兒,送我們過河去也。」那陳清道:「有有有。」就教解板打船。眾莊客聞得此言,無不喜捨。那個道:「我買桅篷。」這個道:「我辦篙槳。」有的說:「我出繩索。」有的說:「我雇水手。」

正都在河邊上吵鬧,忽聽得河中間高叫:「孫大聖不要打船,花費人家財物。我送你師徒們過去。」眾人聽說,個個心驚,膽小的走了回家,膽大的戰兢兢貪看。須臾,那水裡鑽出一個怪來,你道怎生模樣:
\begin{quote}
方頭神物非凡品,九助靈機號水仙。
曳尾能延千紀壽,潛身靜隱百川淵。
翻波跳浪沖江岸,向日朝風臥海邊。
養氣含靈真有道,多年粉蓋癩頭黿。
\end{quote}

那老黿又叫:「大聖,不要打船,我送你師徒過去。」行者掄著鐵棒道:「我把你這個孽畜!若到邊前,這一棒就打死你。」老黿道:「我感大聖之恩,情願辦好心送你師徒,你怎麼反要打我?」行者道:「與你有甚恩惠?」老黿道:「大聖,你不知這底下水黿之第,乃是我的住宅,自歷代以來,祖上傳留到我。我因省悟本根,養成靈氣,在此處修行,被我將祖居翻蓋了一遍,立做一個水黿之第。那妖邪乃九年前海嘯波翻,他趕潮頭,來於此處,仗逞兇頑,與我爭鬥,被他傷了我許多兒女,奪了我許多眷族。我鬥他不過,將巢穴白白的被他占了。今蒙大聖至此搭救唐師父,請了觀音菩薩掃淨妖氛,收去怪物,將第宅還歸於我。我如今團圞老小,再不須挨土幫泥,得居舊舍。此恩重若丘山,深如大海。且不但我等蒙惠,只這一莊上人,免得年年祭賽,全了多少人家兒女。此誠所謂一舉而兩得之恩也,敢不報答。」行者聞言,心中暗喜,收了鐵棒道:「你端的是真實之情麼?」老黿道:「因大聖恩德洪深,怎敢虛謬?」行者道:「既是真情,你朝天賭咒。」那老黿張著紅口,朝天發誓道:「我若真情不送唐僧過此通天河,將身化為血水。」行者笑道:「你上來,你上來。」

老黿卻才負近岸邊,將身一縱,爬上河崖。眾人近前觀看,有四丈圍圓的一個大白蓋。行者道:「師父,我們上他身,渡過去也。」三藏道:「徒弟呀,那層冰厚凍,尚且邅迍,況此黿背,恐不穩便。」老黿道:「師父放心。我比那層冰厚凍,穩得緊哩,但歪一歪,不成功果。」行者道:「師父啊,凡諸眾生,會說人話,決不打誑語。」教:「兄弟們,快牽馬來。」

到了河邊,陳家莊老幼男女一齊來拜送。行者教把馬牽在白黿蓋上,請唐僧站在馬的頸項左邊,沙僧站在右邊,八戒站在馬後,行者站在馬前。又恐那黿無禮,解下虎觔絛子,穿在老黿的鼻之內,扯起來,像一條韁繩。卻使一隻腳踏在蓋上,一隻腳登在頭上;一隻手執著鐵棒。一隻手扯著韁繩;叫道:「老黿,慢慢走啊,歪一歪兒,就照頭一下。」老黿道:「不敢,不敢。」他卻蹬開四足,踏水面如行平地。眾人都在岸上焚香叩頭,都念「南無阿彌陀佛」。這正是:真羅漢臨凡,活菩薩出現。眾人只拜的望不見形影方回,不題。

卻說那師父駕著白黿,那消一日,行過了八百里通天河界,乾手乾腳的登岸。三藏上崖,合手稱謝道:「老黿累你,無物可贈,待我取經回謝你罷。」老黿道:「不勞師父賜謝。我聞得西天佛祖無滅無生,能知過去未來之事。我在此間整修行了一千三百餘年,雖然延壽身輕,會說人語,只是難脫本殼。萬望老師父到西天與我問佛祖一聲,看我幾時得脫本殼,可得一個人身?」三藏響允道:「我問,我問。」那老黿才淬水中去了。

行者遂伏侍唐僧上馬,八戒挑著行囊,沙僧跟隨左右,師徒們找大路,一直奔西。這的是:
\begin{quote}
聖僧奉旨拜彌陀,水遠山遙災難多。
意志心誠不懼死,白黿馱渡過天河。
\end{quote}

畢竟不知此後有多少路程,還有甚麼凶吉,且聽下回分解。
