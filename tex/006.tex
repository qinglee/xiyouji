
\chapter{觀音赴會問原因 小聖施威降大聖}

且不言天神圍繞,大聖安歇。話表南海普陀落伽山大慈大悲救苦救難靈感觀世音菩薩,自王母娘娘請赴蟠桃大會,與大徒弟惠岸行者,同登寶閣瑤池,見那裡荒荒涼涼,席面殘亂;雖有幾位天仙,俱不就座,都在那裡亂紛紛講論。菩薩與眾仙相見畢,眾仙備言前事。菩薩道:「既無盛會,又不傳杯,汝等可跟貧僧去見玉帝。」眾仙怡然隨往。至通明殿前,早有四大天師、赤腳大仙等眾俱在此,迎著菩薩,即道玉帝煩惱,調遣天兵,擒怪未回等因。菩薩道:「我要見見玉帝,煩為轉奏。」天師丘弘濟即入靈霄寶殿,啟知宣入。時有太上老君在上,王母娘娘在後。

菩薩引眾同入裡面,與玉帝禮畢,又與老君、王母相見,各坐下。便問:「蟠桃盛會如何?」玉帝道:「每年請會,喜喜歡歡;今年被妖猴作亂,甚是虛邀也。」菩薩道:「妖猴是何出處?」玉帝道:「妖猴乃東勝神洲傲來國花果山石卵化生的。當時生出,即目運金光,射沖斗府。始不介意,繼而成精,降龍伏虎,自削死籍。當有龍王、閻王啟奏。朕欲擒拿,是長庚星啟奏道:『三界之間,凡有九竅者,可以成仙。』朕即施教育賢,宣他上界,封為御馬監弼馬溫官。那廝嫌惡官小,反了天宮。即差李天王與哪吒太子收降,又降詔撫安,宣至上界,就封他做個齊天大聖,只是有官無祿。他因沒事幹管理,東遊西蕩。朕又恐別生事端,著他代管蟠桃園。他又不遵法律,將老樹大桃,盡行偷吃。及至設會,他乃無祿人員,不曾請他。他就設計賺哄赤腳大仙,卻自變他相貌入會,將仙殽仙酒盡偷吃了,又偷老君仙丹,又偷御酒若干,去與本山眾猴享樂。朕心為此煩惱,故調十萬天兵,天羅地網收伏。這一日不見回報,不知勝負如何。」菩薩聞言,即命惠岸行者道:「你可快下天宮,到花果山,打探軍情如何。如遇相敵,可就相助一功,務必的實回話。」

惠岸行者整整衣裙,執一條鐵棍,駕雲離闕,徑至山前。見那天羅地網,密密層層,各營門提鈴喝號,將那山圍繞的水泄不通。惠岸立住叫:「把營門的天丁,煩你傳報:我乃李天王二太子木吒——南海觀音大徒弟惠岸,特來打探軍情。」那營裡五岳神兵,即傳入轅門之內。早有虛日鼠、昴日雞、星日馬、房日兔,將言傳到中軍帳下。李天王發下令旗,教開天羅地網,放他進來。此時東方才亮,惠岸隨旗進入,見四大天王與李天王下拜。拜訖,李天王道:「孩兒,你自那廂來者?」惠岸道:「愚男隨菩薩赴蟠桃會,菩薩見勝會荒涼,瑤池寂寞,引眾仙並愚男去見玉帝。玉帝備言父王等下界收伏妖猴,一日不見回報,勝負未知,菩薩因命愚男到此打聽虛實。」李天王道:「昨日到此安營下寨,著九曜星挑戰,被這廝大弄神通,九曜星俱敗走而回。後我等親自提兵,那廝也排開陣勢。我等十萬天兵,與他混戰至晚,他使個分身法戰退。及收兵查勘時,止捉得些狼蟲虎豹之類,不曾捉得他半個妖猴。今日還未出戰。」

說不了,只見轅門外有人來報道:「那大聖引一群猴精,在外面叫戰。」四大天王與李天王並太子正議出兵,木叉道:「父王,愚男蒙菩薩吩咐,下來打探消息,就說若遇戰時,可助一功。今不才願往,看他怎麼個大聖。」天王道:「孩兒,你隨菩薩修行這幾年,想必也有些神通,切須在意。」

好太子,雙手掄著鐵棍,束一束繡衣,跳出轅門,高叫:「那個是齊天大聖?」大聖挺如意棒,應聲道:「老孫便是。你是甚人,輒敢問我?」木叉道:「吾乃李天王第二太子木叉,今在觀音菩薩寶座前為徒弟護教,法名惠岸是也。」大聖道:「你不在南海修行,卻來此見我做甚?」木叉道:「我蒙師父差來打探軍情,見你這般猖獗,特來擒你。」大聖道:「你敢說那等大話,且休走,吃老孫這一棒。」木叉全然不懼,使鐵棒劈手相迎。他兩個立那半山中,轅門外,這場好鬥:
\begin{quote}
棍雖對棍鐵各異,兵縱交兵人不同。一個是太乙散仙呼大聖,一個是觀音徒弟正元龍。渾鐵棍乃千鎚打,六丁六甲運神功;如意棒是天河定,鎮海神珍法力洪。兩個相逢真對手,往來解數實無窮。這個的陰手棍萬千兇,繞腰貫索疾如風;那個的夾槍棒不放空,左遮右擋怎相容。那陣上旌旗閃閃,這陣上鼉鼓鼕鼕。萬員天將團團繞,一洞妖猴簇簇叢。怪霧愁雲漫地府,狼煙煞氣射天宮。昨朝混戰還猶可,今日爭持更又兇。堪羨猴王真本事,木叉復敗又逃生。
\end{quote}

這大聖與惠岸戰經五六十合,惠岸臂膊酸麻,不能迎敵,虛幌一幌,敗陣而走。大聖也收了猴兵,安扎在洞門之外。只見天王營門外,大小天兵接住了太子,讓開大路,徑入轅門,對四天王、李托塔、哪吒,氣哈哈的喘息未定:「好大聖,好大聖!著實神通廣大,孩兒戰不過,又敗陣而來也!」李天王見了心驚,即命寫表求助,便差大力鬼王與木叉太子上天啟奏。

二人當時不敢停留,闖出天羅地網,駕起瑞靄祥雲。須臾,徑至通明殿下,見了四大天師,引至靈霄寶殿,呈上表章。惠岸又見菩薩施禮。菩薩道:「你打探的如何?」惠岸道:「始領命到花果山,叫開天羅地網門,見了父親,道師父差命之意。父王道:『昨日與那猴王戰了一場,止捉得他虎豹獅象之類,更未捉他一個猴精。』正講間,他又索戰,是弟子使鐵棍與他戰經五六十合,不能取勝,敗走回營。父親因此差大力鬼王同弟子上界求助。」菩薩低頭思忖。

卻說玉帝拆開表章,見有求助之言,笑道:「叵耐這個猴精,能有多大手段,就敢敵過十萬天兵?李天王又來求助,卻將那路神兵助之?」言未畢,觀音合掌啟奏:「陛下寬心,貧僧舉一神,可擒這猴。」玉帝道:「所舉者何神?」菩薩道:「乃陛下令甥顯聖二郎真君,見居灌洲灌江口,享受下方香火。他昔日曾力誅六怪,又有梅山兄弟與帳前一千二百草頭神,神通廣大。奈他只是聽調不聽宣,陛下可降一道調兵旨意,著他助力,便可擒也。」玉帝聞言,即傳調兵的旨意,就差大力鬼王賫調。

那鬼王領了旨,即駕起雲,徑至灌江口,不消半個時辰,直至真君之廟。早有把門的鬼判傳報至裡道:「外有天使,捧旨而至。」二郎即與眾弟兄出門迎接旨意,焚香開讀。旨意上云:
\begin{quote}
花果山妖猴齊天大聖作亂:因在宮偷桃、偷酒、偷丹,攪亂蟠桃大會,見著十萬天兵、一十八架天羅地網,圍山收伏,未曾得勝。今特調賢甥同義兄弟即赴花果山助力剿除。成功之後,高陞重賞。
\end{quote}

真君大喜道:「天使請回,吾當就去拔刀相助也。」鬼王回奏不題。

這真君即喚梅山六兄弟乃康、張、姚、李四太尉,郭申、直健二將軍,聚集殿前道:「適才玉帝調遣我等往花果山收降妖猴,同去去來。」眾兄弟俱忻然願往。即點本部神兵,駕鷹牽犬,搭弩張弓,縱狂風,霎時過了東洋大海,徑至花果山。見那天羅地網密密層層,不能前進,因叫道:「把天羅地網的神將聽著:吾乃二郎顯聖真君,蒙玉帝調來,擒拿妖猴者,快開營門放行。」一時,各神一層層傳入。四大天王與李天王俱出轅門迎接。相見畢,問及勝敗之事,天王將上項事備陳一遍。真君笑道:「小聖來此,必須與他鬥個變化。列公將天羅地網不要幔了頂上,只四圍緊密,讓我賭鬥。若我輸與他,不必列公相助,我自有兄弟扶持;若贏了他,也不必列公綁縛,我自有兄弟動手。只請托塔天王與我使個照妖鏡,住立空中。恐他一時敗陣,逃竄他方,切須與我照耀明白,勿走了他。」天王各居四維,眾天兵各挨排列陣去訖。

這真君領著四太尉、二將軍,連本身七兄弟,出營挑戰;吩咐眾將緊守營盤,收全了鷹犬。眾草頭神得令。真君只到那水簾洞外,見那一群猴齊齊整整,排作個蟠龍陣勢。中軍裡立一竿旗,上書「齊天大聖」四字。真君道:「那潑妖,怎麼稱得起齊天之職?」梅山六弟道:「且休讚嘆,叫戰去來。」那營口小猴見了真君,急走去報知。那猴王即掣金箍棒,整黃金甲,登步雲履,按一按紫金冠,騰出營門,急睜睛觀看,那真君的相貌果是清奇,打扮得又秀氣。真個是:
\begin{quote}
儀容清俊貌堂堂,兩耳垂肩目有光。
頭戴三山飛鳳帽,身穿一領淡鵝黃。
縷金靴襯盤龍襪,玉帶團花八寶粧。
腰挎彈弓新月樣,手執三尖兩刃槍。
斧劈桃山曾救母,彈打棕羅雙鳳凰。
力誅八怪聲名遠,義結梅山七聖行。
心高不認天家眷,性傲歸神住灌江。
赤城昭惠英靈聖,顯化無邊號二郎。
\end{quote}

大聖見了,笑嘻嘻的將金箍棒掣起,高叫道:「你是何方小將,輒敢大膽到此挑戰?」真君喝道:「你這廝有眼無珠,認不得我麼?吾乃玉帝外甥、敕封昭惠靈顯王二郎是也。今蒙上命,到此擒你這造反天宮的弼馬溫猢猻,你還不知死活。」大聖道:「我記得當年玉帝妹子思凡下界,配合楊君,生一男子,曾使斧劈桃山的,是你麼?我行要罵你幾聲,曾奈無甚冤仇;待要打你一棒,可惜了你的性命。你這郎君小輩,可急急回去,喚你四大天王出來。」真君聞言,心中大怒道:「潑猴!休得無禮,吃吾一刃。」大聖側身躲過,疾舉金箍棒,劈手相還。他兩個這場好殺:
\begin{quote}
昭惠二郎神,齊天孫大聖。這個心高欺敵美猴王,那個面生壓伏真梁棟。兩個乍相逢,各人皆賭興。從來未識淺和深,今日方知輕與重。鐵棒賽飛龍,神鋒如舞鳳。左擋右攻,前迎後映。這陣上梅山六弟助威風,那陣上馬流四將傳軍令。搖旗擂鼓各齊心,吶喊篩鑼都助興。兩個鋼刀有見機,一來一往無絲縫。金箍棒是海中珍,變化飛騰能取勝。若還身慢命該休,但要差池為蹭蹬。
\end{quote}

真君與大聖鬥經三百餘合,不知勝負。那真君抖擻神威,搖身一變,變得身高萬丈,兩隻手舉著三尖兩刃神鋒,好便似華山頂上之峰,青臉獠牙,朱紅頭髮,惡狠狠,望大聖著頭就砍。這大聖也使神通,變得與二郎身軀一樣,嘴臉一般,舉一條如意金箍棒,卻就是崑崙頂上擎天之柱,抵住二郎神。諕得那馬、流元帥戰兢兢,搖不得旌旗;崩、芭二將虛怯怯,使不得刀劍。這陣上,康、張、姚、李、郭申、直健傳號令,撒放草頭神,向他那水簾洞外縱著鷹犬,搭弩張弓,一齊掩殺。可憐沖散妖猴四健將,捉拿靈怪二三千。那些猴拋戈棄甲,撇劍丟槍,跑的跑,喊的喊,上山的上山,歸洞的歸洞。好似夜貓驚宿鳥,飛灑滿天星。眾兄弟得勝不題。

卻說真君與大聖變做法天象地的規模,正鬥時,大聖忽見本營中妖猴驚散,自覺心慌,收了法象,掣棒抽身就走。真君見他敗走,大步趕上道:「那裡走?趁早歸降,饒你性命。」大聖不戀戰,只情跑起。將近洞口,正撞著康、張、姚、李四太尉,郭申、直健二將軍,一齊帥眾擋住道:「潑猴!那裡走?」大聖慌了手腳,就把金箍棒捏做繡花針,藏在耳內。搖身一變,變作個麻雀兒,飛在樹梢頭釘住。那六兄弟慌慌張張,前後尋覓不見,一齊吆喝道:「走了這猴精也!走了這猴精也!」

正嚷處,真君到了,問:「兄弟們,趕到那廂不見了?」眾神道:「才在這裡圍住,就不見了。」二郎圓睜鳳目觀看,見大聖變了麻雀兒,釘在樹上。就收了法象,撇了神鋒,卸下彈弓。搖身一變,變作個鷂鷹兒,抖開翅,飛將去撲打。大聖見了,颼的一翅飛起去,變作一只大鶿老,沖天而去。二郎見了,急抖翎毛,搖身一變,變作一隻大海鶴,鑽上雲霄來嗛。大聖又將身按下,入澗中,變作一個魚兒,淬入水內。二郎趕至澗邊,不見蹤跡。心中暗想道:「這猢猻必然下水去也,定變作魚蝦之類。等我再變變拿他。」果一變,變作個魚鷹兒,飄蕩在下溜頭波面上,等待片時。那大聖變魚兒,順水正游,忽見一隻飛禽:似青鷂,毛片不青;似鷺鷥,頂上無纓;似老鸛,腿又不紅:「想是二郎變化了等我哩!」急轉頭,打個花就走。二郎看見道:「打花的魚兒:似鯉魚,尾巴不紅;似鱖魚,花鱗不見;似黑魚,頭上無星;似魴魚,鰓上無針。他怎麼見了我就回去了?必然是那猴變的。」趕上來,刷的啄一嘴。那大聖就攛出水中,一變,變作一條水蛇,游近岸,鑽入草中。二郎因嗛他不著,他見水響中,見一條蛇攛出去,認得是大聖。急轉身,又變了一隻朱繡頂的灰鶴,伸著一個長嘴,與一把尖頭鐵鉗子相似,徑來吃這水蛇。水蛇跳一跳,又變做一隻花鴇,木木樗樗的,立在蓼汀之上。二郎見他變得低賤,(花鴇乃鳥中至賤至淫之物,不拘鸞、鳳、鷹、鴉,都與交群)故此不去攏傍。即現原身,走將去,取過彈弓,拽滿,一彈子把他打個躘踵。

那大聖趁著機會,滾下山崖,伏在那裡又變,變一座土地廟兒:大張著口,似個廟門;牙齒變做門扇;舌頭變做菩薩;眼睛變做窗櫺;只有尾巴不好收拾,豎在後面,變做一根旗竿。真君趕到崖下,不見打倒的鴇鳥,只有一間小廟。急睜鳳眼,仔細看之,見旗竿立在後面,笑道:「是這猢猻了,他今又在那裡哄我。我也曾見廟宇,更不曾見一個旗竿豎在後面的。斷是這畜生弄諠。他若哄我進去,他便一口咬住。我怎肯進去?等我掣拳先搗窗櫺,後踢門扇。」大聖聽得,心驚道:「好狠,好狠!門扇是我牙齒,窗櫺是我眼睛,若打了牙,搗了眼,卻怎麼是好?」撲的一個虎跳,又冒在空中不見。

真君前前後後亂趕,只見四太尉、二將軍一齊擁至,道:「兄長,拿住大聖了麼?」真君笑道:「那猴兒才自變座廟宇哄我。我正要搗他窗櫺,踢他門扇,他就縱一縱,又渺無蹤跡。可怪,可怪!」眾皆愕然,四望更無形影。真君道:「兄弟們在此看守巡邏,等我上去尋他。」急縱身駕雲,起在半空。見那李天王高擎照妖鏡,與哪吒住立雲端,真君道:「天王,曾見那猴王麼?」天王道:「不曾上來,我這裡照著他哩。」真君把那賭變化,弄神通,拿群猴一事說畢。卻道:「他變廟宇,正打處,就走了。」李天王聞言,又把照妖鏡四方一照,呵呵的笑道:「真君,快去,快去。那猴使了個隱身法,走出營圍,往你那灌江口去也。」二郎聽說,即取神鋒,回灌江口來趕。

卻說那大聖已至灌江口,搖身一變,變作二郎爺爺的模樣,按下雲頭,徑入廟裡。鬼判不能相認,一個個磕頭迎接。他坐中間,點查香火:見李虎拜還的三牲,張龍許下的保福,趙甲求子的文書,錢丙告病的良願。正看處,有人報:「又一個爺爺來了。」眾鬼判急急觀看,無不驚心。真君卻道:「有個甚麼齊天大聖,才來這裡否?」眾鬼判道:「不曾見甚麼大聖,只有一個爺爺在裡面查點哩。」真君撞進門,大聖見了,現出本相道:「郎君不消嚷,廟宇已姓孫了。」這真君即舉三尖兩刃神鋒,劈臉就砍。那猴王使個身法,讓過神鋒。掣出那繡花針兒,幌一幌,碗來粗細,趕到前,對面相還。兩個嚷嚷鬧鬧,打出廟門,半霧半雲,且行且戰,復打到花果山。慌得那四大天王等眾,隄防愈緊。這康、張太尉等迎著真君,合心努力,把那美猴王圍繞不題。

話表大力鬼王既調了真君與六兄弟提兵擒魔去後,卻上界回奏。玉帝與觀音菩薩、王母並眾仙卿,正在靈霄殿講話,道:「既是二郎已去赴戰,這一日還不見回報。」觀音合掌道:「貧僧請陛下同道祖出南天門外,親去看看虛實如何?」玉帝道:「言之有理。」即擺駕,同道祖、觀音、王母與眾仙卿至南天門。早有些天丁、力士接著,開門遙觀。只見眾天丁佈羅網,圍住四面;李天王與哪吒擎照妖鏡,立在空中;真君把大聖圍繞中間,紛紛賭鬥哩。

菩薩開口對老君說:「貧僧所舉二郎神如何?果有神通,已把那大聖圍困,只是未得擒拿。我如今助他一功,決拿住他也。」老君道:「菩薩將甚兵器?怎麼助他?」菩薩道:「我將那淨瓶楊柳拋下去,打那猴頭,即不能打死,也打個一跌,教二郎小聖好去拿他。」老君道:「你這瓶是個磁器,能打著他便好,如打不著他的頭,或撞著他的鐵棒,卻不打碎了?你且莫動手,等我老君助他一功。」菩薩道:「你有甚麼兵器?」老君道:「有,有,有。」捋起衣袖,左膊上取下一個圈子,說道:「這件兵器,乃錕鋼摶煉的,被我將還丹點成,養就一身靈氣,善能變化,水火不侵,又能套諸物。一名『金鋼琢』,又名『金鋼套』。當年過函關,化胡為佛,甚是虧他。早晚最可防身。等我丟下去打他一下。」

話畢,自天門上往下一摜,滴流流,徑落花果山營盤裡,可可的著猴王頭上一下。猴王只顧苦戰七聖,卻不知天上墜下這兵器,打中了天靈,立不穩腳,跌了一跤,爬將起來就跑。被二郎爺爺的細犬趕上,照腿肚子上一口,又扯了一跌。他睡倒在地,罵道:「這個亡人!你不去妨家長,卻來咬老孫!」急翻身爬不起來,被七聖一擁按住,即將繩索捆綁,使勾刀穿了琵琶骨,再不能變化。

那老君收了金鋼琢,請玉帝同觀音、王母、眾仙等,俱回靈霄殿。這下面四大天王與李天王諸神,俱收兵拔寨,近前向小聖賀喜,都道:「此小聖之功也。」小聖道:「此乃天尊洪福,眾神威權,我何功之有?」康、張、姚、李道:「兄長不必多敘,且押這廝去上界見玉帝,請旨發落去也。」真君道:「賢弟,汝等未受天籙,不得面見玉帝。教天甲神兵押著,我同天王等上界回旨。你們帥眾在此搜山,搜淨之後,仍回灌口。待我請了賞,討了功,回來同樂。」四太尉、二將軍依言領諾。這真君與眾即駕雲頭,唱凱歌,得勝朝天。不多時,到通明殿外。天師啟奏道:「四大天王等眾已捉了妖猴齊天大聖了,來此聽宣。」玉帝傳旨,即命大力鬼王與天丁等眾,押至斬妖臺,將這廝碎剁其屍。咦!正是:
\begin{quote}
欺誑今遭刑憲苦,英雄氣概等時休。
\end{quote}

畢竟不知那猴王性命何如,且聽下回分解。
