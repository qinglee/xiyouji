
\chapter{二心攪亂大乾坤 一體難修真寂滅}

這行者與沙僧拜辭了菩薩,縱起兩道祥光,離了南海。原來行者觔斗雲快,沙和尚仙雲覺遲,行者就要先行。沙僧扯住道:「大哥不必這等藏頭露尾,先去安根。待小弟與你一同走。」大聖本是良心,沙僧卻有疑意。真個二人同駕雲而去。不多時,果見花果山。按下雲頭,二人洞外細看,果見一個行者,高坐石臺之上,與群猴飲酒作樂。模樣與大聖無異:也是黃髮金箍,金睛火眼;身穿也是綿布直裰,腰繫虎皮裙;手中也拿一條兒金箍鐵棒;足下也踏一雙麂皮靴;也是這等毛臉雷公嘴,朔腮別土星,查耳額顱闊,獠牙向外生。

這大聖怒發,一撒手,撇了沙和尚,掣鐵棒上前罵道:「你是何等妖邪,敢變我的相貌,敢占我的兒孫,擅居吾仙洞,擅作這威福?」那行者見了,公然不答,也使鐵棒來迎。二行者在一處,果是不分真假,好打呀:
\begin{quote}
兩條棒,二猴精,這場相敵實非輕。都要護持唐御弟,各施功績立英名。真猴實受沙門教,假怪虛稱佛子情。蓋為神通多變化,無真無假兩相平。一個是混元一氣齊天聖,一個是久煉千靈縮地精。這個是如意金箍棒,那個是隨心鐵桿兵。隔架遮攔無勝敗,撐持抵敵沒輸贏。先前交手在洞外,少頃爭持起半空。
\end{quote}

他兩個各踏雲光,跳鬥上九霄雲內。

沙僧在傍,不敢下手。見他們戰此一場,誠然難認真假。欲待拔刀相助,又恐傷了真的。忍耐良久,且縱身跳下山崖,使降妖寶杖,打近水簾洞外,驚散群妖,掀翻石凳,把飲酒食肉的器皿盡情打碎。尋他的青氈包袱,四下裡全然不見。原來他水簾洞本是一股瀑布飛泉,遮掛洞門,遠看似一條白布簾兒,近看乃是一股水脈,故曰水簾洞。沙僧不知進步來歷,故此難尋。即便縱雲,趕到九霄雲裡,掄著寶杖,又不好下手。大聖道:「沙僧,你既助不得力,且回覆師父,說我等這般這般,等老孫與此妖打上南海落伽山菩薩前辨個真假。」道罷,那行者也如此說。沙僧見兩個相貌、聲音,更無一毫差別,皂白難分,只得依言,撥轉雲頭,回覆唐僧不題。

你看那兩個行者,且行且鬥,直來到南海,徑至落伽山,打打罵罵,喊聲不絕。早驚動護法諸天,即報入潮音洞裡道:「菩薩,果然兩個孫悟空打將來也。」那菩薩與木叉行者、善財童子、龍女降蓮臺,出門喝道:「那孽畜那裡走?」這兩個遞相揪住道:「菩薩,這廝果然像弟子模樣。才自水簾洞打起,戰鬥多時,不分勝負。沙悟淨肉眼愚蒙,不能分識,有力難助,是弟子教他回西路去回覆師父。我與這廝打到寶山,借菩薩慧眼,與弟子認個真假,辨明邪正。」道罷,那行者也如此說一遍。眾諸天與菩薩都看良久,莫想能認。菩薩道:「且放了手,兩邊站下,等我再看。」果然撒手,兩邊站定。這邊說:「我是真的。」那邊說:「他是假的。」

菩薩喚木叉與善財上前,悄悄吩咐:「你一個幫住一個,等我暗念緊箍兒咒,看那個害疼的便是真,不疼的便是假。」他二人果各幫一個。菩薩暗念真言,兩個一齊喊疼,都抱著頭,地下打滾,只叫:「莫念,莫念。」菩薩不念,他兩個又一齊揪住,照舊嚷鬥。菩薩無計奈何,即令諸天、木叉上前助力。眾神恐傷真的,亦不敢下手。菩薩叫聲「孫悟空」,兩個一齊答應。菩薩道:「你當年官拜弼馬溫,大鬧天宮時,神將皆認得你,你且上界去分辨回話。」這大聖謝恩,那行者也謝恩。

二人扯扯拉拉,口裡不住的嚷鬥,徑至南天門外。慌得那廣目天王帥馬、趙、溫、關四大天將,及把門大小眾神,各使兵器擋住道:「那裡走?此間可是爭鬥之處?」大聖道:「我因保護唐僧往西天取經,在路上打殺賊徒,那三藏趕我回去,我徑到普陀崖見觀音菩薩訴苦。不想這妖精幾時就變作我的模樣,打倒唐僧,搶去包袱。有沙僧至花果山尋討,只見這妖精占了我的巢穴。後到普陀崖告請菩薩,又見我侍立臺下,沙僧誑說是我駕觔斗雲,又先在菩薩處遮飾。菩薩卻是個正明,不聽沙僧之言,命我同他到花果山看驗。原來這妖精果像老孫模樣,才自水簾洞打到落伽山見菩薩,菩薩也難識認。故打至此間,煩諸天眼力,與我認個真假。」道罷,那行者也似這般這般說了一遍。眾天神看夠多時,也不能辨。他兩個吆喝道:「你們既不能認,讓開路,等我們去見玉帝!」

眾神搪抵不住,放開天門,直至靈霄寶殿。馬元帥同張、葛、許、丘四天師奏道:「下界有一般兩個孫悟空打進天門,口稱見王。」說不了,兩個直嚷進來。諕得那玉帝即降立寶殿,問曰:「你兩個因甚事擅鬧天宮,嚷至朕前尋死?」大聖口稱:「萬歲,萬歲,臣今皈命,秉教沙門,再不敢欺心誑上。只因這個妖精變作臣的模樣,」如此如彼,把前情備陳了一遍,「望乞與臣辨個真假。」那行者也如此陳了一遍。玉帝即傳旨宣托塔李天王,教:「把照妖鏡來照這廝誰真誰假,教他假滅真存。」天王即取鏡照住,請玉帝同眾神觀看。鏡中乃是兩個孫悟空的影子,金箍、衣服,毫髮不差。玉帝亦辨不出,趕出殿外。

這大聖呵呵冷笑,那行者也哈哈歡喜。揪頭抹頸,復打出天門,墜落西方路上道:「我和你見師父去,我和你見師父去。」

卻說那沙僧自花果山辭他兩個,又行了三晝夜,回至本莊,把前事對唐僧說了一遍。唐僧自家悔恨道:「當時只說是孫悟空打我一棍,搶去包袱,豈知卻是妖精假變的行者。」沙僧又告道:「這妖又假變一個長老,一匹白馬;又有一個八戒挑著我們包袱,又有一個變作是我。我忍不住惱怒,一杖打死,原是一個猴精。因此驚散,又到菩薩處訴苦。菩薩著我與師兄又同去識認,那妖果與師兄一般模樣。我難助力,故先來回覆師父。」三藏聞言,大驚失色。八戒哈哈大笑道:「好好好,應了這施主家婆婆之言了:他說有幾起取經的,這卻不又是一起?」

那家老老小小的都來問沙僧道:「你這幾日往何處討盤纏去的?」沙僧笑道:「我往東勝神洲花果山尋大師兄取討行李,又到南海普陀山拜見觀音菩薩,卻又到花果山,方才轉回至此。」那老者又問:「往返有多少路程?」沙僧道:「約有二十餘萬里。」老者道:「爺爺呀!似這幾日,就走了這許多路,只除是駕雲,方能夠得到!」八戒道:「不是駕雲,如何過海?」沙僧道:「我們那算得走路,若是我大師兄,只消一二日,可往回也。」那家子聽言,都說是神仙。八戒道:「我們雖不是神仙,神仙還是我們的晚輩哩!」

正說間,只聽半空中喧嘩亂嚷。慌得都出來看,卻是兩個行者打將來。八戒見了,忍不住手癢道:「等我去認認看。」好獃子,急縱身跳起,望空高叫道:「師兄莫嚷,我老豬來也!」那兩個一齊應道:「兄弟,來打妖精,來打妖精。」那家子又驚又喜道:「是幾位騰雲駕霧的羅漢歇在我家,就是發願齋僧的,也齋不著這等好人。」更不計較茶飯,愈加供養。又說:「這兩個行者只怕鬥出不好來,地覆天翻,作禍在那裡!」三藏見那老者當面是喜,背後是憂,即開言道:「老施主放心,莫生憂嘆。貧僧收伏了徒弟,去惡歸善,自然謝你。」那老者滿口回答道:「不敢,不敢。」沙僧道:「施主休講。師父可坐在這裡,等我和二哥去,一家扯一個來到你面前,你就念念那話兒,看那個害疼的就是真的,不疼的就是假的。」三藏道:「言之極當。」

沙僧果起在半空道:「二位住了手,我同你到師父面前辨個真假去。」這大聖放了手,那行者也放了手。沙僧攙住一個,叫道:「二哥,你也攙住一個。」果然攙住,落下雲頭,徑至草舍門外。三藏見了,就念緊箍兒咒。二人一齊叫苦道:「我們這等苦鬥,你還咒我怎的?莫念,莫念。」那長老本心慈善,遂住了口不念,卻也不認得真假。他兩個掙脫手,依然又打。這大聖道:「兄弟們保著師父,等我與他打到閻王前折辨去也。」那行者也如此說。二人抓抓掗掗,須臾又不見了。

八戒道:「沙僧,你既到水簾洞,看見假八戒挑著行李,怎麼不搶將來?」沙僧道:「那妖精見我使寶杖打他假沙僧,他就亂圍上來要拿,是我顧性命走了。及告菩薩,與行者復至洞口,他兩個打在空中,是我去掀翻他的石凳,打散他的小妖。只見一股瀑布泉水流,竟不知洞門開在何處,尋不著行李,所以空手回覆師命也。」八戒道:「你原來不曉得。我前年請他去時,先在洞門外相見。後被我說泛了他,他就跳下,去洞裡換衣來時,我看見他將身往水裡一鑽。那一股瀑布水流,就是洞門。想必那怪將我們包袱收在那裡面也。」三藏道:「你既知此門,你可趁他都不在,可先到他洞裡取出包袱,我們往西天去罷。他就來,我也不用他了。」八戒道:「我去。」沙僧說:「二哥,他那洞前有千數小猴,你一人恐弄他不過,反為不美。」八戒笑道:「不怕,不怕。」急出門,縱著雲霧,徑上花果山尋取行李不題。

卻說那兩個行者又打嚷到陰山背後,諕得那滿山鬼戰戰兢兢,藏藏躲躲。有先跑的,撞入陰司門裡,報上森羅寶殿道:「大王,背陰山上有兩個齊天大聖打將來也。」慌得那第一殿秦廣王傳報與二殿楚江王、三殿宋帝王、四殿卞城王、五殿閻羅王、六殿平等王、七殿泰山王、八殿都市王、九殿忤官王、十殿轉輪王:一殿轉一殿,霎時間,十王會齊,又著人飛報與地藏王。盡在森羅殿上,點聚陰兵,等擒真假。只聽得那強風滾滾,慘霧漫漫,二行者一翻一滾的打至森羅殿下。

陰君近前擋住道:「大聖有何事,鬧我幽冥?」這大聖道:「我因保唐僧西天取經,路過西梁國,至一山,有強賊截劫我師。是老孫打死幾個,師父怪我,把我逐回。我隨到南海菩薩處訴告,不知那妖精怎麼就綽著口氣,假變作我的模樣,在半路上打倒師父,搶奪了行李。師弟沙僧向我本山取討包袱,這妖假立師名,要往西天取經。沙僧逃遁至南海見菩薩,我正在側。他備說原因,菩薩又命我同他至花果山觀看,果被這廝占了我巢穴。我與他爭辨到菩薩處,其實相貌、言語等俱一般,菩薩也難辨真假。又與這廝打上天堂,眾神亦果難辨。因見我師,我師念緊箍咒試驗,與我一般疼痛。故此鬧至幽冥,望陰君與我查看生死簿,看假行者是何出身,快早追他魂魄,免教二心沌亂。」那怪亦如此說一遍。陰君聞言,即喚管簿判官一一從頭查勘,更無個假行者之名。再看毛蟲文簿,那猴子一百三十條,已是孫大聖幼年得道之時,大鬧陰司,消死名,一筆勾之。自後來,凡是猴屬,盡無名號。查勘畢,當殿回報。陰君各執笏,對行者道:「大聖,幽冥處既無名號可查,你還到陽間去折辨。」

正說處,只聽得地藏王菩薩道:「且住,且住。等我著諦聽與你聽個真假。」原來那諦聽是地藏菩薩經案下伏的一個獸名。他若伏在地下,一霎時,將四大部洲山川社稷,洞天福地之間,蠃蟲、鱗蟲、毛蟲、羽蟲、昆蟲、天仙、地仙、神仙、人仙、鬼仙,可以照鑒善惡,察聽賢愚。那獸奉地藏鈞旨,就於森羅庭院之中,俯伏在地。須臾,擡起頭來,對地藏道:「怪名雖有,但不可當面說破,又不能助力擒他。」地藏道:「當面說出便怎麼?」諦聽道:「當面說出,恐妖精惡發,搔擾寶殿,致令陰府不安。」又問:「何為不能助力擒拿?」諦聽道:「妖精神通,與孫大聖無二。幽冥之神,能有多少法力?故此不能擒拿。」地藏道:「似這般怎生祛除?」諦聽言:「佛法無邊。」地藏早已省悟,即對行者道:「你兩個形容如一,神通無二,若要辨明,須到雷音寺釋迦如來那裡,方得明白。」兩個一齊嚷道:「說的是,說的是。我和你西天佛祖之前折辨去。」那十殿陰君送出,謝了地藏,回上翠雲宮,著鬼使閉了幽冥關隘不題。

看那兩個行者飛雲奔霧,打上西天。有詩為證。詩曰:
\begin{quote}
人有二心生禍災,天涯海角致疑猜。
欲思寶馬三公位,又憶金鑾一品臺。
南征北討無休歇,東擋西除未定哉。
禪門須學無心訣,靜養嬰兒結聖胎。
\end{quote}

他兩個在那半空裡扯扯拉拉,抓抓掗掗,且行且鬥,直嚷至大西天靈鷲仙山雷音寶剎之外。早見那四大菩薩、八大金剛、五百阿羅、三千揭諦、比丘尼、比丘僧、優婆塞、優婆夷諸大聖眾,都到七寶蓮臺之下,淨聽如來說法。那如來正講到這:
\begin{quote}
「不有中有,不無中無。不色中色,不空中空。非有為有,非無為無。非色為色,非空為空。空即是空,色即是色。色無定色,色即是空。空無定空,空即是色。知空不空,知色不色。名為照了,始達妙音。」
\end{quote}

概眾稽首皈依,流通誦讀之際,如來降天花普散繽紛,即離寶座,對大眾道:「汝等俱是一心,且看二心競鬥而來也。」

大眾舉目看之,果是兩個行者,吆天喝地,打至雷音勝境。慌得那八大金剛上前擋住道:「汝等欲往那裡去?」這大聖道:「妖精變作我的模樣,欲至寶蓮臺下,煩如來為我辨個虛實也。」眾金剛抵擋不住,直嚷至臺下,跪於佛祖之前,拜告道:「弟子保護唐僧,來造寶山,求取真經,一路上煉魔縛怪,不知費了多少精神。前至中途,偶遇強徒劫擄,委是弟子二次打傷幾人。師父怪我趕回,不容同拜如來金身。弟子無奈,只得投奔南海,見觀音訴苦。不期這個妖精假變弟子聲音、相貌,將師父打倒,把行李搶去。師弟悟淨尋至我山,被這妖假捏巧言,說有真僧取經之故。悟淨脫身至南海,備說詳細。觀音知之,遂令弟子同悟淨再至我山。因此,兩人比併真假,打至南海,又打到天宮,又曾打見唐僧,打見冥府,俱莫能辨認。故此大膽輕造,千乞大開方便之門,廣垂慈憫之念,與弟子辨明邪正,庶好保護唐僧親拜金身,取經回東土,永揚大教。」大眾聽他兩張口一樣聲,俱說一遍,眾亦莫辨。惟如來則通知之,正欲道破,忽見南下彩雲之間,來了觀音,參拜我佛。

我佛合掌道:「觀音尊者,你看那兩個行者,誰是真假?」菩薩道:「前日在弟子荒境,委不能辨。他又至天宮、地府,亦俱難認。特來拜告如來,千萬與他辨明辨明。」如來笑道:「汝等法力廣大,只能普閱周天之事,不能遍識周天之物,亦不能廣會周天之種類也。」菩薩又請示周天種類。如來才道:「周天之內有五仙:乃天、地、神、人、鬼。有五蟲:乃蠃、鱗、毛、羽、昆。這廝非天、非地、非神、非人、非鬼,亦非蠃、非鱗、非毛、非羽、非昆。又有四猴混世,不入十類之種。」菩薩道:「敢問是那四猴?」如來道:「第一是靈明石猴,通變化,識天時,知地利,移星換斗;第二是赤尻馬猴,曉陰陽,會人事,善出入,避死延生;第三是通臂猿猴,拿日月,縮千山,辨休咎,乾坤摩弄;第四是六耳獼猴,善聆音,能察理,知前後,萬物皆明。此四猴者,不入十類之種,不達兩間之名。我觀假悟空乃六耳獼猴也。此猴若立一處,能知千里外之事;凡人說話,亦能知之。故此善聆音,能察理,知前後,萬物皆明。與真悟空同像同音者,六耳獼猴也。」

那獼猴聞得如來說出他的本像,膽戰心驚,急縱身,跳起來就走。如來見他走時,即令大眾下手。早有四菩薩、八金剛、五百阿羅、三千揭諦、比丘僧、比丘尼、優婆塞、優婆夷、觀音、木叉,一齊圍繞。孫大聖也要上前,如來道:「悟空休動手,待我與你擒他。」那獼猴毛骨悚然,料著難脫,即忙搖身一變,變作個蜜蜂兒,往上便飛。如來將金缽盂撇起去,正蓋著那蜂兒,落下來。大眾不知,以為走了。如來笑云:「大眾休言。妖精未走,見在我這缽盂之下。」大眾一發上前,把缽盂揭起,果然見了本像,是一個六耳獼猴。孫大聖忍不住,掄起鐵棒,劈頭一下打死,至今絕此一種。如來不忍,道聲:「善哉!善哉!」大聖道:「如來不該慈憫他。他打傷我師父,搶奪我包袱,依律問他個得財傷人,白晝搶奪,也該個斬罪哩。」如來道:「你自快去保護唐僧來此求經罷。」大聖叩頭謝道:「上告如來得知:那師父定是不要我,我此去,若不收留,卻不又勞一番神思?望如來方便,把鬆箍兒咒念一念,褪下這個金箍,交還如來,放我還俗去罷。」如來道:「你休亂想,切莫放刁。我教觀音送你去,不怕他不收。好生保護他去,那時功成歸極樂,汝亦坐蓮臺。」

那觀音在傍聽說,即合掌謝了聖恩,領悟空,輒駕雲而去。隨後木叉行者、白鸚哥一同趕上。不多時,到了中途草舍人家。沙和尚看見,急請師父拜門迎接。菩薩道:「唐僧,前日打你的,乃假行者六耳獼猴也。幸如來知識,已被悟空打死。你今須是收留悟空,一路上魔障未消,必得他保護你,才得到靈山,見佛取經。再休嗔怪。」三藏叩頭道:「謹遵教旨。」

正拜謝時,只聽得正東上狂風滾滾,豬八戒背著兩個包袱,駕風而至。獃子見了菩薩,倒身下拜道:「弟子前日別了師父,至花果山水簾洞尋取包袱,果見一個假唐僧、假八戒,都被弟子打死,原是兩個猴身。卻入裡,方尋著包袱,當時查點,一物不少,卻駕風轉此。更不知兩行者下落如何?」菩薩把如來識怪之事說了一遍。那獃子十分歡喜,稱謝不盡。師徒們拜謝了,菩薩回海。卻都照舊合意同心,洗冤解怒。又謝了那村舍人家,整束行囊、馬匹,找大路而西。正是:
\begin{quote}
中道分離亂五行,降妖聚會合元明。
神歸心舍禪方定,六識祛降丹自成。
\end{quote}

畢竟這去,不知三藏幾時得面佛求經,且聽下回分解。
