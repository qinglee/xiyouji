
\chapter{外道弄強欺正法 心猿顯聖滅諸邪}

話說那國王見孫行者有呼龍使聖之法,即將關文用了寶印,便要遞與唐僧,放行西路。那三個道士慌得拜倒在金鑾殿上啟奏。那皇帝即下龍位,御手忙攙道:「國師今日行此大禮,何也?」道士說:「陛下,我等至此匡扶社稷,保國安民,苦歷二十年來今日這和尚弄法力,抓了丟去,敗了我們聲名。陛下以一場之雨,就恕殺人之罪,可不輕了我等也?望陛下且留住他的關文,讓我兄弟與他再賭一賭,看是何如?」那國王著實昏亂,東說向東,西說向西,真個收了關文,道:「國師,你怎麼與他賭?」虎力大仙道:「我與他賭坐禪。」國王道:「國師差矣。那和尚乃禪教出身,必然先會禪機,才敢奉旨求經,你怎與他賭此?」大仙道:「我這坐禪,比常不同,有一異名,教做雲梯顯聖。」國王道:「何為雲梯顯聖?」大仙道:「要一百張桌子,五十張作一禪臺,一張一張疊將起去,不許手攀而上,亦不用梯凳而登,各駕一朵雲頭,上臺坐下,約定幾個時辰不動。」

國王見此有些難處,就便傳旨問道:「那和尚,我國師要與你賭『雲梯顯聖』坐禪,那個會麼?」行者聞言,沉吟不答。八戒道:「哥哥,怎麼不言語?」行者道:「兄弟,實不瞞你說。若是踢天弄井、攪海翻江、擔山趕月、換斗移星諸般巧事,我都幹得;就是砍頭剁腦、剖腹剜心、異樣騰那卻也不怕;但說坐禪,我就輸了。我那裡有這坐性?你就把我鎖在鐵柱子上,我也要上下爬蹅,莫想坐得住。」三藏忽的開言道:「我會坐禪。」行者歡喜道:「卻好,卻好。可坐得多少時?」三藏道:「我幼年遇方上禪僧講道,那性命根本上,定性存神,在死生關裡,也坐二三個年頭。」行者道:「師父若坐二三年,我們就不取經罷。多也不上二三個時辰,就下來了。」三藏道:「徒弟呀,卻是不能上去。」行者道:「你上前答應,我送你上去。」

那長老果然合掌當胸道:「貧僧會坐禪。」國王教傳旨,立禪臺。國家有倒山之力,不消半個時辰,就設起兩座臺,在金鑾殿左右。

那虎力大仙下殿,立於階心,將身一縱,踏一朵席雲,徑上西邊臺上坐下。行者拔一根毫毛,變做假像,陪著八戒、沙僧,立於下面;他卻作五色祥雲,把唐僧撮起空中,徑至東邊臺上坐下;他又斂祥光,變作一個蟭蟟蟲,飛在八戒耳朵邊道:「兄弟,仔細看著師父,再莫與老孫替身說話。」那獃子笑道:「理會得,理會得。」

卻說那鹿力大仙在繡墩上坐看多時,他兩個在高臺上不分勝負,這道士就助他師兄一功:將腦後短髮拔了一根,捻著一團,彈將上去,徑至唐僧頭上,變作一個大臭蟲,咬住長老。那長老先前覺癢,然後覺疼。原來坐禪的不許動手,動手算輸。一時間疼痛難禁,他縮著頭,就著衣襟擦癢。八戒道:「不好了,師父羊兒風發了。」沙僧道:「不是,是頭風發了。」行者聽見道:「我師父乃志誠君子,他說會坐禪,斷然會坐;說不會,只是不會。君子家,豈有謬乎?你兩個休言,等我上去看看。」

好行者,嚶的一聲,飛在唐僧頭上,只見有豆粒大小一個臭蟲叮他師父,慌忙用手捻下,替師父撓撓摸摸。那長老不疼不癢,端坐上面。行者暗想道:「和尚頭光,虱子也安不得一個,如何有此臭蟲?想是那道士弄的玄虛,害我師父。哈哈,枉自也不見輸贏,等老孫去弄他一弄!」這行者飛將上去,金殿獸頭上落下,搖身一變,變作一條七寸長的蜈蚣,徑來道士鼻凹裡叮了一下。那道士坐不穩,一個觔斗,翻將下去,幾乎喪了性命,幸虧大小官員人多救起。國王大驚,即著當駕太師領他往文華殿裡梳洗去了。行者仍駕祥雲,將師父馱下階前,已是長老得勝。

那國王只教放行。鹿力大仙又奏道:「陛下,我師兄原有暗風疾,因到了高處,冒了天風,舊疾舉發,故令和尚得勝。且留下他,等我與他賭隔板猜枚。」國王道:「怎麼叫做『隔板猜枚』?」鹿力道:「貧道有隔板知物之法,看那和尚可能夠?他若猜得過我,讓他出去;猜不著,憑陛下問擬罪名,雪我昆仲之恨,不污了二十年保國之恩也。」真個那國王十分昏亂,依此讒言,即傳旨:將一硃紅漆的櫃子,命內官擡到宮殿。教娘娘放上件寶貝。須臾擡出,放在白玉階前,教僧道:「你兩家各賭法力,猜那櫃中是何寶貝。」

三藏道:「徒弟,櫃中之物,如何得知?」行者斂祥光,還變作蟭蟟蟲,釘在唐僧頭上道:「師父放心,等我去看看來。」好大聖,輕輕飛到櫃上,爬在那櫃腳之下,見有一條板縫兒。他鑽將進去,見一個紅漆丹盤,內放一套宮衣,乃是山河社稷襖、乾坤地理裙。用手拿起來,抖亂了,咬破舌尖上,一口血哨噴將去,叫聲:「變!」即變作一件破爛流丟一口鐘。臨行又撒上一泡臊溺。卻還從板縫裡鑽出來,飛在唐僧耳朵上道:「師父,你只猜是破爛流丟一口鐘。」三藏道:「他教猜寶貝哩,流丟是件甚寶貝?」行者道:「莫管他,只猜著便是。」

唐僧進前一步,正要猜,那鹿力大仙道:「我先猜,那櫃裡是山河社稷襖、乾坤地理裙。」唐僧道:「不是,不是,櫃裡是件破爛流丟一口鐘。」國王道:「這和尚無禮,敢笑我國中無寶,猜甚麼流丟一口鐘。」教:「拿了!」那兩班校尉就要動手。慌得唐僧合掌高呼:「陛下,且赦貧僧一時,待打開櫃看。端的是寶,貧僧領罪;如不是寶,卻不屈了貧僧也?」國王教打開看。當駕官即開了,捧出丹盤來看,果然是件破爛流丟一口鐘。國王大怒道:「是誰放上此物?」龍座後面閃上三宮皇后道:「我主,是梓童親手放的山河社稷襖、乾坤地理裙,卻不知怎麼變成此物?」國王道:「御妻請退,寡人知之。宮中所用之物,無非是緞絹綾羅,那有此甚麼流丟?」教:「擡上櫃來,等朕親藏一寶貝,再試如何。」那皇帝即轉後宮,把御花園裡仙桃樹上結得一個大桃子,有碗來大小,摘下,放在櫃內,又擡下叫猜。

唐僧道:「徒弟啊,又來猜了。」行者道:「放心,等我再去看看。」又嚶的一聲,飛將去,還從板縫兒鑽進去,見是一個桃子,正合他意。即現了原身,坐在櫃裡,將桃子一頓口啃得乾乾淨淨,連兩邊腮凹兒都啃淨了,將核兒安在裡面。仍變蟭蟟蟲,飛將出去,釘在唐僧耳朵上道:「師父,只猜是個桃核子。」長老道:「徒弟啊,休要弄我。先前不是口快,幾乎拿去典刑。這番須猜寶貝方好。桃核子是甚寶貝?」行者道:「休怕,只管贏他便了。」

三藏正要開言,聽得那羊力大仙道:「貧道先猜,是一顆仙桃。」三藏猜道:「不是桃,是個光桃核子。」那國王喝道:「是朕放的仙桃,如何是核?三國師猜著了。」三藏道:「陛下,打開來看就是。」當駕官又擡上去打開,捧出丹盤,果然是一個核子,皮肉俱無。國王見了,心驚道:「國師,休與他賭鬥了,讓他去罷。寡人親手藏的仙桃,如今只是一核子,是甚人吃了?想是有鬼神暗助他也。」八戒聽說,與沙僧微微冷笑道:「還不知他是會吃桃子的積年哩。」

正話間,只見那虎力大仙從文華殿梳洗了,走上殿道:「陛下,這和尚有搬運抵物之術。擡上櫃來,我破他術法,與他再猜。」國王道:「國師還要猜甚?」虎力道:「術法只抵得物件,卻抵不得人身。將這道童藏在裡面,管教他抵換不得。」這小童果藏在櫃裡,掩上櫃蓋,擡將下去,教:「那和尚再猜,這三番是甚寶貝?」

三藏道:「又來了!」行者道:「等我再去看看。」嚶的又飛去,鑽入裡面,見是一個小童兒。好大聖,他卻有見識,果然是騰那天下少,似這伶俐世間稀。他就搖身一變,變作個老道士一般容貌,進櫃裡,叫聲:「徒弟。」童兒道:「師父,你從那裡來的?」行者道:「我使遁法來的。」童兒道:「你來有甚麼教誨?」行者道:「那和尚看見你進櫃來了,他若猜個道童,卻不又輸了?是特來和你計較計較:剃了頭,我們猜和尚罷。」童兒道:「但憑師父處治,只要我們贏他便了;若是再輸與他,不但低了聲名,又恐朝廷不敬重了。」行者道:「說得是。我兒過來,贏了他,我重重賞你。」將金箍棒就變作一把剃頭刀,摟抱著那童兒,口裡叫道:「乖乖,忍著疼,莫放聲,等我與你剃頭。」須臾,剃下髮來,窩作一團,塞在那櫃腳紇絡裡。收了刀兒,摸著他的光頭道:「我兒,頭便像個和尚,只是衣裳不趁。脫下來,我與你變一變。」那道童穿的一領蔥白色雲頭花絹繡錦沿邊的鶴氅,真個脫下來。被行者吹一口仙氣,叫:「變!」即變做一件土黃色的直裰兒,與他穿了。卻又拔下兩根毫毛,變作一個木魚兒,遞在他手裡道:「徒弟,須聽著:但叫道童,千萬莫出去;若叫和尚,你就與我頂開櫃蓋,敲著木魚,念一卷佛經鑽出來,方得成功也。」童兒道:「我只會念《三官經》、《北斗經》、《消災經》,不會念佛家經。」行者道:「你可會念佛?」童兒道:「阿彌陀佛,那個不會念?」行者道:「也罷,也罷,就念佛,省得我又教你。切記著,我去也。」還變蟭蟟蟲,鑽出去,飛在唐僧耳輪邊道:「師父,你只猜是個和尚。」三藏道:「這番他準贏了。」行者道:「你怎麼定得?」三藏道:「經上有云:『佛、法、僧三寶。』和尚卻也是一寶。」

正說處,只見那虎力大仙道:「陛下,第三番是個道童。」只管叫,他那裡肯出來。三藏合掌道:「是個和尚。」八戒盡力高叫道:「櫃裡是個和尚。」那童兒忽的頂開櫃蓋,敲著木魚,念著佛,鑽出來。喜得那兩班文武齊聲喝采。諕得那三個道士拑口無言。

國王道:「這和尚是有鬼神輔佐。怎麼道士入櫃,就變做和尚?縱有待詔跟進去,也只剃得頭便了,如何衣服也能趁體,口裡又會念佛?國師啊,讓他去罷。」虎力大仙道:「陛下,左右是棋逢對手,將遇良材。貧道將鍾南山幼時學的武藝,索性與他賭一賭。」國王道:「有甚麼武藝?」虎力道:「弟兄三個,都有些神通:會砍下頭來,又能安上;剖腹剜心,還再長完;滾油鍋裡,又能洗澡。」國王大驚道:「此三事都是尋死之路。」虎力道:「我等有此法力,才敢出此朗言,斷要與他賭個才休。」那國王叫道:「東土的和尚,我國師不肯放你,還要與你賭砍頭、剖腹、下滾油鍋洗澡哩。」

行者正變作蟭蟟蟲,往來報事,忽聽此言,即收了毫毛,現出本相,哈哈大笑道:「造化,造化,買賣上門了。」八戒道:「這三件都是喪性命的事,怎麼說買賣上門?」行者道:「你還不知我的本事。」八戒道:「哥哥,你只像這等變化騰那也夠了,怎麼還有這等本事?」行者道:「我啊:
\begin{quote}
砍下頭來能說話,剁了臂膊打得人。
斬去腿腳會走路,剖腹還平妙絕倫。
就似人家包匾食,一捻一個就囫圇。
油鍋洗澡更容易,只當溫湯滌垢塵。」
\end{quote}

八戒、沙僧聞言,呵呵大笑。

行者上前道:「陛下,小和尚會砍頭。」國王道:「你怎麼會砍頭?」行者道:「我當年在寺裡修行,曾遇著一個方上禪和子,教我一個砍頭法,不知好也不好,如今且試試新。」國王笑道:「那和尚年幼不知事,砍頭那裡好試新?頭乃六陽之首,砍下即便死矣。」虎力道:「陛下,正要他如此,方才出得我們之氣。」那昏君信他言語,即傳旨,教設殺場。

一聲傳旨,即有羽林軍三千,擺列朝門之外。國王教:「和尚先去砍頭。」行者欣然應道:「我先去,我先去。」拱著手,高呼道:「國師,恕大膽,占先了。」拽回頭,往外就走。唐僧一把扯住道:「徒弟呀,仔細些,那裡不是耍處。」行者道:「怕他怎的?撒了手,等我去來。」

那大聖徑至殺場裡面,被劊子手撾住了,綑做一團,按在那土墩高處,只聽喊一聲:「開刀!」颼的把個頭砍將下來。又被劊子手一腳踢了去,好似滾西瓜一般,滾有三四十步遠近。行者腔子中更不出血。只聽得肚裡叫聲:「頭來!」慌得鹿力大仙見有這般手段,即念咒語,教本坊土地、神祇:「將人頭扯住,待我贏了和尚,奏了國王,與你把小祠堂蓋作大廟宇,泥塑像改作正金身。」原來那些土地、神祇因他有五雷法,也服他使喚,暗中真個把行者頭按住了。行者又叫聲:「頭來!」那頭一似生根,莫想得動。行者心焦,捻著拳,掙了一掙,將綑的繩子就皆掙斷,喝聲:「長!」颼的腔子內長出一個頭來。諕得那劊子手個個心驚,羽林軍人人膽戰。那監斬官急走入朝奏道:「萬歲,那小和尚砍了頭,又長出一顆來了。」八戒冷笑道:「沙僧,那知哥哥還有這般手段。」沙僧道:「他有七十二般變化,就有七十二個頭哩。」

說不了,行者走來,叫聲:「師父。」三藏大喜道:「徒弟,辛苦麼?」行者道:「不辛苦,倒好耍子。」八戒道:「哥哥,可用刀瘡藥麼?」行者道:「你是摸摸看,可有刀痕?」那獃子伸手一摸,就笑得呆呆睜睜道:「妙哉,妙哉!卻也長得完全,截疤兒也沒些兒。」

兄弟們正都歡喜,又聽得國王叫領關文:「赦你無罪。快去,快去。」行者道:「關文雖領,必須國師也赴曹砍砍頭,也當試新去來。」國王道:「大國師,那和尚也不肯放你哩。你與他賭勝,且莫諕了寡人。」虎力也只得去,被幾個劊子手也綑翻在地,幌一幌,把頭砍下,一腳也踢將去,滾了有三十餘步。他腔子裡也不出血,也叫一聲:「頭來!」行者即忙拔下一根毫毛,吹口仙氣,叫:「變!」變作一條黃犬,跑入場中,把那道士頭一口銜來,徑跑到御水河邊丟下不題。

卻說那道士連叫三聲,人頭不到,怎似行者的手段,長不出來,腔子中,骨都都紅光迸出。可憐空有喚雨呼風法,怎比長生果正仙。須臾,倒在塵埃。眾人觀看,乃是一隻無頭的黃毛虎。

那監斬官又來奏:「萬歲,大國師砍下頭來,不能長出,死在塵埃,是一隻無頭的黃毛虎。」國王聞奏,大驚失色,目不轉睛,看那兩個道士。鹿力起身道:「我師兄已是命倒祿絕了,如何是隻黃虎?這都是那和尚憊𪬯,使的掩樣法兒,將我師兄變作畜類。我今定不饒他,定要與他賭那剖腹剜心。」

國王聽說,方才定性回神。又叫:「小和尚,二國師還要與你賭哩。」行者道:「小和尚久不吃煙火食,前日西來,忽遇齋公家勸飯,多吃了幾個饝饝,這幾日腹中作痛,想是生蟲,正欲借陛下之刀,剖開肚皮,拿出臟腑,洗淨脾胃,方好上西天見佛。」國王聽說,教:「拿他赴曹。」那許多人攙的攙,扯的扯。行者展脫手道:「不用人攙,自家走去。但一件:不許縛手,我好用手洗刷臟腑。」國王傳旨,教:「莫綁他手。」

行者搖搖擺擺,徑至殺場。將身靠著大樁,解開衣帶,露出肚腹。那劊子手將一條繩套在他膊項上,一條繩紮住他腿足,把一口牛耳短刀幌一幌,著肚皮下一割,搠個窟窿。這行者雙手爬開肚腹,拿出腸臟來,一條條理夠多時,依然安在裡面,照舊盤曲。捻著肚皮,吹口仙氣,叫:「長!」依然長合。

國王大驚,將他那關文捧在手中道:「聖僧莫誤西行,與你關文去罷。」行者笑道:「關文小可,也請二國師剖剖剜剜,何如?」國王對鹿力說:「這事不與寡人相干,是你要與他做對頭的,請去,請去。」鹿力道:「寬心,料我決不輸與他。」

你看他也像孫大聖,搖搖擺擺,徑入殺場。被劊子手套上繩,將牛耳短刀唿喇的一聲,割開肚腹。他也拿出肝腸,用手理弄。行者即拔一根毫毛,吹口仙氣,叫:「變!」即變作一隻餓鷹,展開翅爪,颼的把他五臟心肝,盡情抓去,不知飛向何方受用。這道士弄做一個空腔破肚淋漓鬼,少臟無腸浪蕩魂。那劊子手蹬倒大樁,拖屍來看,呀!原來是一隻白毛角鹿。

慌得那監斬官又來奏道:「二國師晦氣,正剖腹時,被一隻餓鷹將臟腑肝腸都刁去了,死在那裡。原身是個白毛角鹿也。」國王害怕道:「怎麼是個角鹿?」那羊力大仙又奏道:「我師兄既死,如何得現獸形?這都是那和尚弄術法坐害我等。等我與師兄報仇者。」國王道:「你有甚麼法力贏他?」羊力道:「我與他賭下滾油鍋洗澡。」國王便教取一口大鍋,滿著香油,教他兩個賭去。行者道:「多承下顧。小和尚一向不曾洗澡,這兩日皮膚燥癢,好歹盪盪去。」

那當駕官果安下油鍋,架起乾柴,燃著烈火,將油燒滾,教和尚先下去。」行者合掌道:「不知文洗,武洗?」國王道:「文洗如何?武洗如何?」行者道:「文洗不脫衣服,似這般叉著手,下去打個滾,就起來,不許污壞了衣服,若有一點油膩算輸。武洗要取一張衣架,一條手巾,脫了衣服,跳將下去,任意翻觔斗,豎蜻蜓,當耍子洗也。」國王對羊力說:「你要與他文洗,武洗?」羊力道:「文洗恐他衣服是藥鍊過的,隔油。武洗罷。」

行者又上前道:「恕大膽,屢次占先了。」你看他脫了布直裰,褪了虎皮裙,將身一縱,跳在鍋內,翻波鬥浪,就似負水一般頑耍。八戒見了,咬著指頭對沙僧道:「我們也錯看了這猴子了。平時間劖言訕語,鬥他耍子,怎知他有這般真實本事。」他兩個唧唧噥噥,誇獎不盡。

行者望見,心疑道:「那獃子笑我哩。正是『巧者多勞拙者閑』。老孫這般舞弄,他倒自在。等我作成他綑一繩,看他可怕?」正洗浴,打個水花,淬在油鍋底上,變作個棗核釘兒,再也不起來了。

那監斬官近前又奏:「萬歲,小和尚被滾油烹死了。」國王大喜,教撈上骨骸來看。劊子手將一把鐵笊籬在油鍋裡撈。原來那笊籬眼稀,行者變得釘小,往往來來,從眼孔漏下去了,那裡撈得著。又奏道:「和尚身微骨嫩,俱炸化了。」國王教:「拿三個和尚下去。」兩邊校尉見八戒面兇,先揪翻,把背心綑了。

慌得三藏高叫:「陛下,赦貧僧一時。我那個徒弟自從歸教,歷歷有功。今日衝撞國師,死在油鍋之內,奈何先死者為神。我貧僧怎敢貪生?正是天下官員也管著天下百姓。陛下若教臣死,臣豈敢不死?只望寬恩,賜我半盞涼漿水飯、三張紙馬,容到油鍋前,燒此一陌紙,也表我師徒一念,那時再領罪也。」國王聞言道:「也是,那中華人多有義氣。」命取些漿飯、黃錢與他。果然取了,遞與唐僧。

唐僧教沙和尚同去,行至階下,有幾個校尉把八戒揪著耳朵,拉在鍋邊。三藏對鍋祝曰:「徒弟孫悟空:
\begin{quote}
自從受戒拜禪林,護我西來恩愛深。
指望同時成大道,何期今日你歸陰。
生前只為求經意,死後還存念佛心。
萬里英魂須等候,幽冥做鬼上雷音。」
\end{quote}

八戒聽見道:「師父,不是這般祝了。——沙和尚,你替我奠漿飯,等我禱。」那獃子綑在地下,氣呼呼的道:「闖禍的潑猴子,無知的弼馬溫;該死的潑猴子,油烹的弼馬溫。猴兒了帳,馬溫斷根。」孫行者在油鍋底上,聽得那獃子亂罵,忍不住現了本相,赤淋淋的站在油鍋底道:「饢糟的夯貨!你罵那個哩?」唐僧見了道:「徒弟,諕殺我也!」沙僧道:「大哥乾淨推佯死慣了。」

慌得那兩班文武上前來奏道:「萬歲,那和尚不曾死,又在油鍋裡鑽出來了。」監斬官恐怕虛誑朝廷,卻又奏道:「死是死了,只是日期犯兇,小和尚來顯魂哩。」行者聞言大怒,跳出鍋來,揩了油膩,穿上衣服,掣出棒,撾過監斬官,著頭一下,打做了肉團,道:「我顯甚麼魂哩?」諕得多官連忙解了八戒,跪地哀告:「恕罪,恕罪。」國王走下龍座,行者上殿扯住道:「陛下不要走,且教你三國師也下下油鍋去。」那皇帝戰戰兢兢道:「三國師,你救朕之命,快下鍋去,莫教和尚打我。」

羊力下殿,照依行者脫了衣服,跳下油鍋,也那般支吾洗浴。行者放了國王,近油鍋邊,叫燒火的添柴。卻伸手探了一把,呀!那滾油都冰冷。心中暗想道:「我洗時滾熱,他洗時卻冷。我曉得了,這不知是那個龍王,在此護持他哩。」急縱身跳在空中,念聲「唵」字咒語,把那北海龍王喚來:「我把你這個帶角的蚯蚓,有鱗的泥鰍!你怎麼助道士,冷龍護住鍋底,教他顯聖贏我?」諕得那龍王喏喏連聲道:「敖順不敢相助。大聖原來不知,這個孽畜苦修行了一場,脫得本殼,卻只是五雷法真受,其餘都屣了傍門,難歸仙道。這個是他在小茅山學來的『大開剝』。那兩個已是大聖破了他法,現了本相。這一個也是他自己煉的冷龍,只好哄瞞世俗之人耍子,怎瞞得大聖?小龍如今收了他冷龍,管教他骨碎皮焦。」行者道:「趁早收了,免打。」那龍王化一陣旋風,到油鍋邊,將冷龍捉下海去不題。

行者下來,與三藏、八戒、沙僧立在殿前,見那道士在滾油鍋裡打掙,爬不出來。滑了一跌,霎時間骨脫皮焦肉爛。

監斬官又來奏道:「萬歲,三國師煠化了也。」那國王滿眼垂淚,手撲著御案,放聲大哭道:
\begin{quote}
「人身難得果然難,不遇真傳莫煉丹。
空有驅神咒水術,卻無延壽保生丸。
圓明混,怎涅槃?徒用心機命不安。
早覺這般輕折挫,何如秘食穩居山?」
\end{quote}

這正是:
\begin{quote}
點金煉汞成何濟,喚雨呼風總是空!
\end{quote}

畢竟不知師徒們怎的維持,且聽下回分解。
