
\chapter{尋洞擒妖逢老壽 當朝正主救嬰兒}

卻說那錦衣官把假唐僧扯出館驛,與羽林軍圍圍繞繞,直至朝門外,對黃門官言:「我等已請唐僧到此,煩為轉奏。」黃門官急進朝,依言奏上昏君,遂請進去。眾官都在階下跪拜,惟假唐僧挺立階心,口中高叫:「比丘王,請我貧僧何說?」君王笑道:「朕得一疾,纏綿日久不愈。幸國丈賜得一方,藥餌俱已完備,只少一味引子。特請長老,求些藥引。若得病愈,與長老修建祠堂,四時奉祭,永為傳國之香火。」假唐僧道:「我乃出家人,隻身至此,不知陛下問國丈要甚東西作引?」昏君道:「特求長老的心肝。」假唐僧道:「不瞞陛下說,心便有幾個兒,不知要的甚麼色樣?」那國丈在傍指定道:「那和尚,要你的黑心。」假唐僧道:「既如此,快取刀來,剖開胸腹,若有黑心,謹當奉命。」那昏君歡喜相謝,即著當駕官取一把牛耳短刀,遞與假僧。假僧接刀在手,解開衣服,挺起胸膛,將左手抹腹,右手持刀,唿喇的響一聲,把肚皮剖開,那裡頭就骨都都的滾出一堆心來。諕得文官失色,武將身麻。國丈在殿上見了道:「這是個多心的和尚。」假僧將那些心,血淋淋的一個個撿開與眾觀看,卻都是些紅心、白心、黃心、慳貪心、利名心、嫉妒心、計較心、好勝心、望高心、侮慢心、殺害心、狠毒心、恐怖心、謹慎心、邪妄心、無名隱暗之心、種種不善之心,更無一個黑心。那昏君諕得呆呆掙掙,口不能言,戰兢兢的教:「收了去,收了去。」那假唐僧忍耐不住,收了法心,現出本相,對昏君道:「陛下全無眼力。我和尚家都是一片好心,惟你這國丈是個黑心,好做藥引。你不信,等我替你取他的出來看看。」

那國丈聽見,急睜睛仔細觀看,見那和尚變了面皮,不是那般模樣。咦!
\begin{quote}
認得當年孫大聖,五百年前舊有名。
\end{quote}

卻抽身,騰雲就起,被行者翻觔斗,跳在空中喝道:「那裡走?吃吾一棒。」那國丈即使蟠龍拐杖來迎。他兩個在半空中這場好殺:
\begin{quote}
如意棒,蟠龍拐,虛空一片雲靉靉。原來國丈是妖精,故將怪女稱嬌色。國主貪歡病染身,妖邪要把兒童宰。相逢大聖顯神通,捉怪救人將難解。鐵棒當頭著實兇,拐棍迎來堪喝采。殺得那滿天霧氣暗城池,城裡人家都失色。文武多官魂魄飛,嬪妃繡女容顏改。諕得那比丘昏主亂身藏,戰戰兢兢沒佈擺。棒起猶如虎出山,拐掄卻似龍離海。今番大鬧比丘國,致令邪正分明白。
\end{quote}

那妖精與行者苦戰二十餘合,蟠龍拐抵不住金箍棒,虛幌了一拐,將身化作一道寒光,落入皇宮內院,把進貢的妖后帶出宮門,並化寒光,不知去向。

大聖按落雲頭,到了宮殿下,對多官道:「你們的好國丈啊!」多官一齊禮拜,感謝神僧。行者道:「且休拜,且去看你那昏主何在?」多官道:「我主見爭戰時,驚恐潛藏,不知向那座宮中去也。」行者即命:「快尋,莫被美后拐去。」多官聽言,不分內外,同行者先奔美后宮,漠然無蹤,連美后也通不見了。正宮、東宮、西宮、六院,概眾后妃,都來拜謝大聖。大聖道:「且請起,不到謝處哩。且去尋你主公。」少時,見四五個太監攙著那昏君,自謹身殿後面而來。眾臣俯伏在地,齊聲啟奏道:「主公,主公,感得神僧到此,辨明真假。那國丈乃是個妖邪,連美后亦不見矣。」國王聞言,即請行者出皇宮,到寶殿,拜謝了,道:「長老,你早間來的模樣那般俊偉,這時如何就改了形容?」行者笑道:「不瞞陛下說,早間來者,是我師父,乃唐朝御弟三藏。我是他徒弟孫悟空,還有兩個師弟豬悟能、沙悟淨,見在金亭館驛。因知你信了妖言,要取我師父心肝做藥引,是老孫變作師父模樣,特來此降妖也。」那國王聞說,即傳旨著閣下太宰快去驛中請師眾來朝。

那三藏聽見行者現了相,在空中降妖,嚇得魂飛魄散。幸有八戒、沙僧護持,他又臉上戴著一片子臊泥。正悶悶不快,只聽得人叫道:「法師,我等乃比丘國王差來的閣下太宰,特請入朝謝恩也。」八戒笑道:「師父,莫怕,莫怕。這不是又請你取心,想是師兄得勝,請你酬謝哩。」三藏道:「雖是得勝來請,但我這個臊臉,怎麼見人?」八戒道:「沒奈何,我們且去見了師兄,自有解釋。」真個那長老無計,只得跟著八戒、沙僧,挑著擔,牽著馬,同去驛庭之上。那太宰見了,害怕道:「爺爺呀!這都像似妖頭怪腦之類。」沙僧道:「朝士休怪醜陋,我等乃是生成的遺體。若我師父,來見了我師兄,他就俊了。」

他三人與眾來朝,不待宣召,直至殿下。行者看見,即轉身下殿,迎著面,把師父的泥臉子抓下,吹口仙氣,叫:「變!」那唐僧即時復了原身,精神愈覺爽利。國王下殿親迎,口稱:「法師老佛。」師徒們將馬拴住,都上殿來相見。行者道:「陛下可知那怪來自何方?等老孫去與你一併擒來,剪除後患。」三宮六院、諸嬪群妃都在那翡翠屏後;聽見行者說剪除後患,也不避內外男女之嫌,一齊出來拜告道:「萬望神僧老佛大施法力,斬草除根,把他剪除盡絕,誠為莫大之恩,自當重報。」行者忙忙答禮,只教國王說他住居。國王含羞告道:「三年前他到時,朕曾問他。他說離城不遠,只在向南去七十里路,有一座柳林坡清華莊上。國丈年老無兒,止後妻生一女,年方十六,不曾配人,願進與朕。朕因愛那女,遂納了,寵幸在宮。不期得疾,太醫屢藥無功。他說:『我有仙方,止用小兒心煎湯為引。』是朕不才,輕信其言,遂選民間小兒,選定今日午時開刀取心。不料神僧下降,恰恰又遇籠兒都不見了。他就說神僧十世修真,元陽未泄,得其心,比小兒心更加萬倍。一時誤犯,不知神僧識透妖魔。敢望廣施大法,剪其後患,朕以傾國之資酬謝。」行者笑道:「實不相瞞,籠中小兒,是我師慈悲,著我藏了。你且休題甚麼資財相謝,待我捉了妖怪,是我的功行。」叫:「八戒,跟我去來。」八戒道:「謹依兄命。但只是腹中空虛,不好著力。」國王即傳旨,教光祿寺快辦齋供。不一時齋到。八戒盡飽一餐,抖擻精神,隨行者駕雲而起。諕得那國王、妃后並文武多官,一個個朝空禮拜,都道:「是真仙真佛降臨凡也。」

那大聖攜著八戒,徑到南方七十里之地,住下風雲,找尋妖處。但只見一股清溪,兩邊夾岸,岸上有千千萬萬的楊柳,更不知清華莊在於何處。正是那:
\begin{quote}
萬頃野田觀不盡,千堤煙柳隱無蹤。
\end{quote}

孫大聖尋覓不著,即捻訣,念一聲「唵」字真言,拘出一個當方土地,戰兢兢近前跪下叫道:「大聖,柳林坡土地叩頭。」行者道:「你休怕,我不打你。我問你:柳林坡有個清華莊,在於何方?」土地道:「此間有個清華洞,不曾有個清華莊。小神知道了,大聖想是自比丘國來的?」行者道:「正是,正是。比丘國王被一個妖精哄了,是老孫到那廂,識得是妖怪,當時戰退那怪,化一道寒光,不知去向。及問比丘王,他說三年前進美女時,曾問其由,怪言居住城南七十里柳林坡清華莊。適尋到此,只見林坡,不見清華莊,是以問你。」土地叩頭道:「望大聖恕罪。比丘王亦我地之主也,小神理當鑒察。奈何妖精神威法大,如我泄漏他事,就來欺凌,故此未獲。大聖今來,只去那南岸九叉頭一顆楊樹根下,左轉三轉,右轉三轉,用兩手齊撲樹上,連叫三聲『開門』,即現清華洞府。」

大聖聞言,即令土地回去,與八戒跳過溪來,尋那顆楊樹。果然有九條叉枝,總在一顆根上。行者吩咐八戒:「你且遠遠的站定,待我叫開門,尋著那怪,趕將出來,你卻接應。」八戒聞命,即離樹有半里遠近立下。這大聖依土地之言,繞樹根,左轉三轉,右轉三轉,雙手齊撲其樹,叫:「開門,開門。」霎時間,一聲響喨,唿喇喇的門開兩扇,更不見樹的蹤跡。那裡邊光明霞采,亦無人煙。行者趁神威,撞將進去,但見那裡好個去處:
\begin{quote}
煙霞晃亮,日月偷明。白雲常出洞,翠蘚亂漫庭。一徑奇花爭豔麗,遍階瑤草鬥芳榮。溫暖氣,景常春,渾如閬苑,不亞蓬瀛。滑凳攀長蔓,平橋掛亂藤。蜂啣紅蕊來巖窟,蝶戲幽蘭過石屏。
\end{quote}

行者急拽步,行近前邊細看,見石屏上有四個大字:「清華仙府」。他忍不住,跳過石屏看處,只見那老怪懷中摟著個美女,喘噓噓的,正講比丘國事,齊聲叫道:「好機會來,三年事,今日得完,被那猴頭破了。」行者跑近身,掣棒高叫道:「我把你這夥毛團!甚麼『好機會』?吃我一棒。」那老怪丟了美人,掄起蟠龍拐,急架相迎。他兩個在洞前,這場好殺,比前又甚不同:
\begin{quote}
棒舉迸金光,拐掄兇氣發。那怪道:「你無知敢進我門來。」行者道:「我有意降妖怪。」那怪道:「我戀國主你無干,怎的欺心來展抹?」行者道:「僧修政教本慈悲,不忍兒童活見殺。」語去言來各恨仇,棒迎拐架當心扎。促損琪花為顧生,踢破翠苔因把滑。只殺得那洞中霞采欽光明,崖上芳菲俱掩壓。乒乓驚得鳥難飛,吆喝嚇得美人散。只存老怪與猴王,呼呼捲地狂風刮。看看殺出洞門來,又撞悟能獃性發。
\end{quote}

原來八戒在外邊,聽見他們裡面嚷鬧,激得他心癢難撓,掣釘鈀,把一顆九叉楊樹鈀倒,使鈀築了幾下,築得那鮮血直冒,嚶嚶的似乎有聲。他道:「這顆樹成了精也,這顆樹成了精也。」八戒舉鈀,又正築處,只見行者引怪出來。那獃子不打話,趕上前,舉鈀就築。那老怪戰行者已是難敵,見八戒鈀來,愈覺心慌,敗了陣,將身一幌,化道寒光,徑投東走。他兩個決不放鬆,向東趕來。

正當喊殺之際,又聞得鸞鶴聲鳴,祥光縹緲。舉目視之,乃南極老人星也。那老人把寒光罩住,叫道:「大聖慢來,天蓬休趕,老道在此施禮哩。」行者即答禮道:「壽星兄弟,那裡來?」八戒笑道:「肉頭老兒罩住寒光,必定捉住妖怪了。」壽星陪笑道:「在這裡,在這裡。望二公饒他命罷。」行者道:「老怪不與老弟相干,為何來說人情?」壽星笑道:「他是我的一副腳力,不意走將來,成此妖怪。」行者道:「既是老弟之物,只教他現出本相來看看。」壽星聞言,即把寒光放出,喝道:「孽畜!快現本相,饒你死罪。」那怪打個轉身,原來是隻白鹿。壽星拿起拐杖道:「這孽畜,連我的拐棒也偷來也。」那隻鹿俯伏在地,口不能言,只管叩頭滴淚。但見他:
\begin{quote}
一身如玉簡斑斑,兩角參差七叉彎。
幾度饑時尋藥圃,有朝渴處飲雲潺。
年深學得飛騰法,日久修成變化顏。
今見主人呼喚處,現身抿耳伏塵寰。
\end{quote}

壽星謝了行者,就跨鹿而行。被行者一把扯住道:「老弟,且慢走,還有兩件事未完哩。」壽星道:「還有甚麼未完之事?」行者道:「還有美人未獲,不知是個甚麼怪物;還又要同到比丘城見那昏君,現相回旨也。」壽星道:「既這等說,我且寧耐。你與天蓬下洞擒捉那美人來,同去現相可也。」行者道:「老弟略等等兒,我們去了就來。」

那八戒抖擻精神,隨行者徑入清華仙府,吶聲喊,叫:「拿妖精,拿妖精!」那美人戰戰兢兢,正自難逃,又聽得喊聲大振,即轉石屏之內,又沒個後門出頭。被八戒喝聲:「那裡走?我把你這個哄漢子的臊精,看鈀。」那美人手中又無兵器,不能迎敵,將身一閃,化道寒光,往外就走。被大聖抵住寒光,乒乓一棒。那怪立不住腳,倒在塵埃,現了本相,原來是一個白面狐狸。獃子忍不住手,舉鈀照頭一築。可憐把那個傾城傾國千般笑,化作毛團狐狸形。行者叫道:「莫打爛他,且留他此身去見昏君。」

那獃子不嫌穢污,一把揪住尾子,拖拖扯扯,跟隨行者出得門來。只見那壽星老兒手摸著鹿頭罵道:「好孽畜啊,你怎麼背主逃去,在此成精?若不是我來,孫大聖定打死你了。」行者跳出來道:「老弟說甚麼?」壽星道:「我囑鹿哩,我囑鹿哩。」八戒將個死狐狸摜在鹿的面前道:「這可是你的女兒麼?」那鹿點頭幌腦,伸著嘴,聞他幾聞,呦呦發聲,似有眷戀不捨之意。被壽星劈頭撲了一掌道:「孽畜!你得命足矣,又聞他怎的?」即解下勒袍腰帶,把鹿扣住頸項,牽將起來,道:「大聖,我和你比丘國相見去也。」行者道:「且住,索性把這邊都掃個乾淨,庶免他年復生妖孽。」

八戒聞言,舉鈀將柳樹亂築。行者又念聲「唵」字真言,依然拘出當方土地,叫:「尋些枯柴,點起烈火,與你這方消除妖患,以免欺凌。」那土地即轉身,陰風颯颯,帥起陰兵,搬取了些迎霜草、秋青草、蓼節草、山蕊草、蔞蒿柴、龍骨柴、蘆荻柴,都是隔年乾透的枯焦之物,見火如同油膩一般。行者叫:「八戒,不必築樹,但得此物填塞洞裡,放起火來,燒得個乾淨。」火一起,果然把一座清華妖怪宅,燒作火池坑。

這裡才喝退土地,同壽星牽著鹿,拖著狐狸,一齊回到殿前,對國王道:「這是你的美后,與他耍子兒麼?」那國王膽戰心驚。又只見孫大聖引著壽星,牽著白鹿,都到殿前,諕得那國裡君臣妃后一齊下拜。行者近前,攙住國王,笑道:「且休拜我。這鹿兒卻是國丈,你只拜他便是。」那國王羞愧無地,只道:「感謝神僧救我一國小兒,真天恩也。」即傳旨,教光祿寺安排素宴,大開東閣,請南極老人與唐僧四眾,共坐謝恩。三藏拜見了壽星,沙僧亦以禮見。都問道:「白鹿既是老壽星之物,如何得到此間為害?」壽星笑道:「前者,東華帝君過我荒山,我留坐著棋,一局未終,這孽畜走了。及客去尋他不見,我因屈指一算,知他走在此處,特來尋他,正遇著孫大聖施威。若果來遲,此畜休矣。」

敘不了,只見報道:「宴已完備。」好素宴:
\begin{quote}
五彩盈門,異香滿座。桌掛繡緯生錦豔,地鋪紅毯晃霞光。寶鴨內,沉檀香裊;御筵前,蔬品香馨。看盤高果砌樓臺,龍纏斗糖擺走獸。鴛鴦錠,獅仙糖,似模似樣;鸚鵡杯,鷺鶿杓,如相如形。席前果品般般盛,案上齋殽件件精。魁圓繭栗,鮮荔桃子。棗兒柿餅味甘甜,松子葡萄香膩酒。幾般蜜食,數品蒸酥。油炸糖澆,花團錦砌。金盤高壘大饝饝,銀碗滿盛香稻飯。辣煼煼湯水粉條長,香噴噴相連添換美。說不盡蘑菇、木耳、嫩筍、黃精,十香素菜,百味珍饈。往來綽摸不曾停,進退諸般皆盛設。
\end{quote}

當時敘了坐次:壽星首席,長老次席,國王前席,行者、八戒、沙僧側席。傍又有兩三個大師相陪左右。即命教坊司動樂。國王擎著紫霞杯,一一奉酒。惟唐僧不飲。八戒向行者道:「師兄,果子讓你,湯飯等須請讓我受用受用。」那獃子不分好歹,一齊亂上,但來的吃個精空。

一席筵宴已畢,壽星告辭。那國王又近前跪拜壽星,求祛病延年之法。壽星笑道:「我因尋鹿,未帶丹藥。欲傳你修養之方,你又筋衰神敗,不能還丹。我這衣袖中只有三個棗兒,是與東華帝君獻茶的,我未曾吃,今送你罷。」國王吞之,漸覺身輕病退。後得長生者,皆原於此。八戒看見,就叫道:「老壽,有火棗,送我幾個吃吃。」壽星道:「未曾帶得,待改日我送你幾斤。」遂出了東閣,道了謝意,將白鹿一聲喝起,飛跨背上,踏雲而去。這朝中君王妃后、城中黎庶居民,各各焚香禮拜不題。

三藏叫:「徒弟,收拾辭王。」那國王又苦留求教。行者道:「陛下,從此色欲少貪,陰功多積,凡百事將長補短,自足以祛病延年,就是教也。」遂拿出兩盤散金碎銀,奉為路費。唐僧堅辭,分文不受。國王無已,命擺鑾駕,請唐僧端坐鳳輦龍車,王與嬪后,俱推輪轉轂,方送出朝。六街三市,百姓群黎,亦皆盞添淨水,爐焚真香,又送出城。

忽聽得半空中一聲風響,路兩邊落下一千一百一十一個鵝籠,內有小兒啼哭,暗中有原護的城隍、土地、社令、真官、五方揭諦、四值功曹、六丁六甲、護教伽藍等眾,應聲高叫道:「大聖,我等前蒙吩咐,攝去小兒鵝籠,今知大聖功成起行,一一送來也。」那國王妃后與一應臣民,又俱下拜。行者望空道:「有勞列位,請各歸祠,我著民間祭祀謝你。」呼呼淅淅,陰風又起而退。

行者叫城裡人家來認領小兒。當時傳播,俱來各認出籠中之兒,歡歡喜喜,抱出叫哥哥,叫肉兒,跳的跳,笑和笑,都叫:「扯住唐朝爺爺,到我家奉謝救兒之恩。」無大無小,若男若女,都不怕他相貌之醜,擡著豬八戒,扛著沙和尚,頂著孫大聖,撮著唐三藏,牽著馬,挑著擔,一擁回城。那國王也不能禁止。這家也開宴,那家也設席。請不及的,或做僧帽、僧鞋、褊衫、布襪,裡裡外外,大小衣裳,都來相送。如此盤桓,將有個月,才得離城。又有傳下影神,立起牌位,頂禮焚香供養。這才是:
\begin{quote}
陰功高疊恩山重,救活千千萬萬人。
\end{quote}

畢竟不知向後又有甚麼事體,且聽下回分解。
