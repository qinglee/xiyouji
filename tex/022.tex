
\chapter{八戒大戰流沙河 木叉奉法收悟淨}

話說唐僧師徒三眾脫難前來,不一日行過了黃風嶺,進西卻是一脈平陽之地。光陰迅速,歷夏經秋,見了些寒蟬鳴敗柳,大火向西流。正行處,只見一道大水狂瀾,渾波湧浪。三藏在馬上忙呼道:「徒弟,你看那前邊水勢寬闊,怎不見船隻行走,我們從那裡過去?」八戒見了道:「果是狂瀾,無舟可渡。」那行者跳在空中,用手搭涼篷而看,他也心驚道:「師父啊,真個是難,真個是難。這條河若論老孫去時,只消把腰兒扭一扭,就過去了;若師父,誠千分難渡,萬載難行。」三藏道:「我這裡一望無邊,端的有多少寬闊?」行者道:「經過有八百里遠近。」八戒道:「哥哥怎的定得個遠近之數?」行者道:「不瞞賢弟說,老孫這雙眼,白日裡常看得千里路上的吉凶。卻才在空中看出,此河上下不知多遠,但只見這經過足有八百里。」長老憂嗟煩惱,兜回馬,忽見岸上有一通石碑。三眾齊來看時,見上有三個篆字,乃「流沙河」;腹上有小小的四行真字云:
\begin{quote}
八百流沙界,三千弱水深。
鵝毛飄不起,蘆花定底沉。
\end{quote}

師徒們正看碑文,只聽得那浪湧如山,波翻若嶺,河當中滑辣的鑽出一個妖精,十分兇醜:
\begin{quote}
一頭紅燄髮蓬鬆,兩隻圓睛亮似燈。
不黑不青藍靛臉,如雷如鼓老龍聲。
身披一領鵝黃氅,腰束雙攢露白藤。
項下骷髏懸九個,手持寶杖甚崢嶸。
\end{quote}

那怪一個旋風,奔上岸來,徑搶唐僧。慌得行者把師父抱住,急登高岸,回身走脫。那八戒放下擔子,掣出釘鈀;望妖精便築。那怪使寶杖架住。他兩個在流沙河岸,各逞英雄。這一場好鬥:
\begin{quote}
九齒鈀,降妖杖,二人相敵河岸上。這個是總督大天蓬,那個是謫下捲簾將。昔年曾會在靈霄,今日爭持賭猛壯。這一個鈀去探爪龍,那一個杖架磨牙象。伸開大四平,鑽入迎風戧。這個沒頭沒臉抓,那個無亂無空放。一個是久占流沙界吃人精,一個是秉教迦持修行將。
\end{quote}

他兩個來來往往,戰經二十回合,不分勝負。

那大聖護了唐僧,牽著馬,守定行李。見八戒與那怪交戰,就恨得咬牙切齒,擦掌磨拳,忍不住要去打他,掣出棒來道:「師父,你坐著,莫怕。等老孫和他耍耍兒來。」那師父苦留不住。他打個唿哨,跳到前邊。原來那怪與八戒正戰到好處,難解難分。被行者掄起鐵棒,望那怪著頭一下,那怪急轉身,慌忙躲過,徑鑽入流沙河裡。氣得個八戒亂跳道:「哥啊,誰著你來的?那怪漸漸手慢,難架我鈀,再不上三五合,我就擒住他了。他見你兇險,敗陣而逃,怎生是好?」行者笑道:「兄弟,實不瞞你說,自從降了黃風怪,下山來,這個把月不曾耍棍,我見你和他戰的甜美,我就忍不住腳癢,故就跳將來耍耍的。那知那怪不識耍,就走了。」

他兩個攙著手,說說笑笑,轉回見了唐僧。唐僧道:「可曾捉得妖怪?」行者道:「那妖怪不奈戰,敗回鑽入水去也。」三藏道:「徒弟,這怪久住于此,他知道淺深。似這般無邊的弱水,又沒了舟楫,須是得個知水性的引領引領才好哩。」行者道:「正是這等說。常言道:『近硃者赤,近墨者黑。』那怪在此,斷知水性。我們如今拿住他,且不要打殺,只教他送師父過河,再做理會。」八戒道:「哥哥不必遲疑,讓你先去拿他,等老豬看守師父。」行者笑道:「賢弟呀,這樁兒我不敢說嘴,水裡勾當,老孫不大十分熟。若是空走,還要捻訣,又念念避水咒,方才走得;不然,就要變化做甚麼魚蝦蟹鱉之類,我才去得。若論賭手段,憑你在高山雲裡,幹甚麼蹊蹺異樣事兒,老孫都會;只是水裡的買賣,有些兒榔杭。」八戒道:「老豬當年總督天河,掌管了八萬水兵大眾,倒學得知些水性。卻只怕那水裡有甚麼眷族老小,七窩八代的都來,我就弄他不過,一時不被他撈去耶?」行者道:「你若到他水中與他交戰,卻不要戀戰,許敗不許勝,把他引將出來,等老孫下手助你。」八戒道:「言得是,我去耶。」說聲去,就剝了青錦直裰,脫了鞋,雙手舞鈀,分開水路,使出那當年舊手段,躍浪翻波,撞將進去,徑至水底之下,往前正走。

卻說那怪敗了陣回,方才喘定,又聽得有人推得水響。忽起身觀看,原來是八戒執了鈀推水。那怪舉杖當面高呼道:「那和尚,那裡走?仔細看打。」八戒使鈀架住道:「你是個甚麼妖精,敢在此間擋路?」那妖道:「你是也不認得我。我不是那妖魔鬼怪,也不是少姓無名。」八戒道:「你既不是妖魔鬼怪,卻怎生在此傷生?你端的甚麼姓名,實實說來,我饒你性命。」那怪道:「我:
\begin{quote}
自小生來神氣壯,乾坤萬里曾遊蕩。
英雄天下顯威名,豪傑人家做模樣。
萬國九州任我行,五湖四海從吾撞。
皆因學道蕩天涯,只為尋師遊地曠。
常年衣缽謹隨身,每日心神不可放。
沿地雲遊數十遭,到處閑行百餘趟。
因此才得遇真人,引開大道金光亮。
先將嬰兒姹女收,後把木母金公放。
明堂腎水入華池,重樓肝火投心臟。
三千功滿拜天顏,志心朝禮明華向。
玉皇大帝便加陞,親口封為捲簾將。
南天門裡我為尊,靈霄殿前吾稱上。
腰間懸掛虎頭牌,手中執定降妖杖。
頭頂金盔晃日光,身披鎧甲明霞亮。
往來護駕我當先,出入隨朝予在上。
只因王母降蟠桃,設宴瑤池邀眾將。
失手打破玉玻璃,天神個個魂飛喪。
玉皇即便怒生嗔,卻令掌朝左輔相:
卸冠脫甲摘官銜,將身推在殺場上。
多虧赤腳大天仙,越班啟奏將吾放。
饒死回生不點刑,遭貶流沙東岸上。
飽時困臥此山中,餓去翻波尋食餉。
樵子逢吾命不存,漁翁見我身皆喪。
來來往往吃人多,翻翻覆覆傷生瘴。
你敢行兇到我門,今日肚皮有所望。
莫言粗糙不堪嘗,拿住消停剁鮓醬。」
\end{quote}

八戒聞言大怒,罵道:「你這潑物!全沒一些兒眼色。我老豬還掐出水沫兒來哩,你怎敢說我粗糙,要剁鮓醬?看起來,你把我認做個老走硝哩。休得無禮,吃你祖宗這一鈀。」那怪見鈀來,使一個「鳳點頭」躲過。兩個在水中打出水面,各人踏浪登波。這一場賭鬥,比前不同,你看那:
\begin{quote}
捲簾將,天蓬帥,各顯神通真可愛。那個降妖寶杖著頭輪,這個九齒釘鈀隨手快。躍浪振山川,推波昏世界。兇如太歲撞幢幡,惡似喪門掀寶蓋。這一個赤心凜凜保唐僧,那一個犯罪滔滔為水怪。鈀抓一下九條痕,杖打之時魂魄敗。努力喜相持,用心要賭賽。算來只為取經人,怒氣沖天不忍耐。攪得那鯁鮊鯉鱖退鮮鱗,龜鱉黿鼉傷嫩蓋;紅蝦紫蟹命皆亡,水府諸神朝上拜。只聽得波翻浪滾似雷轟,日月無光天地怪。
\end{quote}

二人整鬥有兩個時辰,不分勝敗。這才是銅盆逢鐵帚,玉磬對金鐘。

卻說那大聖保著唐僧,立於左右,眼巴巴的望著他兩個在水上爭持,只是他不好動手。只見那八戒虛幌一鈀,佯輸詐敗,轉回頭往東岸上走。那怪隨後趕來,將近到了岸邊。這行者忍耐不住,撇了師父,掣鐵棒,跳到河邊,望妖精劈頭就打。那妖物不敢相迎,颼的又鑽入河內。八戒嚷道:「你這弼馬溫,徹是個急猴子!你再緩緩些兒,等我哄他到了高處,你卻阻住河邊,教他不能回首啊,卻不拿住他也?他這進去,幾時又肯出來?」行者笑道:「獃子,莫嚷,莫嚷。我們且回去見師父去來。」

八戒卻同行者到高岸上,見了三藏。三藏欠身道:「徒弟辛苦呀。」八戒道:「且不說辛苦,只是降了妖精,送得你過河,方是萬全之策。」三藏道:「你才與妖精交戰何如?」八戒道:「那妖的手段,與老豬是個對手。正戰處,使一個詐敗,他才趕到岸上。見師兄舉著棍子,他就跑了。」三藏道:「如此怎生奈何?」行者道:「師父放心,且莫焦惱。如今天色又晚,且坐在這崖岸之上,待老孫去化些齋飯來,你吃了睡去,待明日再處。」八戒道:「說得是,你快去快來。」

行者急縱雲跳起去,正到直北下人家化了一缽素齋,回獻師父。師父見他來得甚快,便叫:「悟空,我們去化齋的人家,求問他一個過河之策,不強似與這怪爭持?」行者笑道:「這家子遠得狠哩,相去有五七千里之路,他那裡得知水性?問他何益?」八戒道:「哥哥又來扯謊了,五七千里路,你怎麼這等去來得快?」行者道:「你那裡曉得,老孫的觔斗雲,一縱有十萬八千里。像這五七千路,只消把頭點上兩點,把腰躬上一躬,就是個往回,有何難哉?」八戒道:「哥啊,既是這般容易,你把師父背著,只消點點頭,躬躬腰,跳過去罷了,何必苦苦的與這怪廝戰?」行者道:「你不會駕雲?你把師父馱過去不是?」八戒道:「師父的凡胎肉骨,重似泰山,我這駕雲的,怎稱得起?須是你的觔斗方可。」行者道:「我的觔斗,好道也是駕雲,只是去的有遠近些兒。你是馱不動,我卻如何馱得動?自古道:『遣泰山輕如芥子,攜凡夫難脫紅塵。』像這潑魔毒怪,使攝法,弄風頭,卻是扯扯拉拉,就地而行,不能帶得空中而去。像那樣法兒,老孫也會使會弄。還有那隱身法、縮地法,老孫件件皆知。但只是師父要窮歷異邦,不能夠超脫苦海,所以寸步難行也。我和你只做得個擁護,保得他身在命在,替不得這些苦惱,也取不得經來;就是有能先去見了佛,那佛也不肯把經善與你我。正叫做『若將容易得,便作等閑看』。」那獃子聞言,喏喏聽受。遂吃了些無菜的素食,師徒們歇在流沙河東崖次之下。

次早,三藏道:「悟空,今日怎生區處?」行者道:「沒甚區處,還須八戒下水。」八戒道:「哥哥,你要圖乾淨,只作成我下水。」行者道:「賢弟,這番我再不急性了,只讓你引他上來,我攔住河沿,不讓他回去,務要將他擒了。」

好八戒,抹抹臉,抖擻精神,雙手拿鈀,到河沿,分開水路,依然又下至窩巢。那怪方才睡醒,忽聽推得水響,急回頭睜睛觀看,見八戒執鈀來至。他跳出來,當頭阻住,喝道:「慢來,慢來,看杖。」八戒舉鈀架住道:「你是個甚麼哭喪杖,斷叫你祖宗看杖?」那怪道:「你這廝甚不曉得哩。我這:
\begin{quote}
寶杖原來名譽大,本是月裡梭羅派。
吳剛伐下一枝來,魯班製造工夫蓋。
裡邊一條金趁心,外邊萬道珠絲玠。
名稱寶杖善降妖,永鎮靈霄能伏怪。
只因官拜大將軍,玉皇賜我隨身帶。
或長或短任吾心,要細要粗憑意態。
也曾護駕宴蟠桃,也曾隨朝居上界。
值殿曾經眾聖參,捲簾曾見諸仙拜。
養成靈性一神兵,不是人間凡器械。
自從遭貶下天門,任意縱橫遊海外。
不當大膽自稱誇,天下槍刀難比賽。
看你那個鏽釘鈀,只好鋤田與築菜。」
\end{quote}

八戒笑道:「我把你少打的潑物,且莫管甚麼築菜,只怕蕩了一下兒,教你沒處貼膏藥,九個眼子一齊流血。縱然不死,也是個到老的破傷風。」那怪丟開架手,在那水底下,與八戒依然打出水面。這一番鬥,比前果更不同,你看他:
\begin{quote}
寶杖掄,釘鈀築,言語不通非眷屬。只因木母剋刀圭,致令兩下相戰觸。沒輸贏,無反覆,翻波淘浪不和睦。這個怒氣怎含容,那個傷心難忍辱。鈀來杖架逞英雄,水滾流沙能惡毒。氣昂昂,勞碌碌,多因三藏朝西域。釘鈀老大兇,寶杖十分熟。這個揪住要往岸上拖,那個抓來就將水裡沃。聲如霹靂動魚龍,雲暗天昏神鬼伏。
\end{quote}

這一場,來來往往,鬥經三十回合,不見強弱。八戒又使個佯輸計,拖了鈀走。那怪隨後又趕來,擁波捉浪,趕至崖邊。八戒罵道:「我把你這個潑怪,你上來,這高處,腳踏實地好打。」那妖罵道:「你這廝哄我上去,又教那幫手來哩。你下來,還在水裡相鬥。」原來那妖乖了,再不肯上岸,只在河沿與八戒鬧吵。

卻說行者見他不肯上岸,急得他心焦性爆,恨不得一把捉來。行者道:「師父,你自坐下,等我與他個『餓鷹叼食』。」就縱觔斗,跳在半空,刷的落下來,要抓那妖。那妖正與八戒嚷鬧,忽聽得風響,急回頭,見是行者落下雲來,卻又收了那杖,一頭淬下水,隱跡潛蹤,渺然不見。行者佇立岸上,對八戒說:「兄弟呀,這妖也弄得滑了,他再不肯上岸,如之奈何?」八戒道:「難,難,難,戰不勝他。就把吃奶的氣力也使盡了,只繃得個手平。」行者道:「且見師父去。」

二人又到高岸,見了唐僧,備言難捉。那長老滿眼下淚道:「似此艱難,怎生得渡?」行者道:「師父莫要煩惱。這怪深潛水底,其實難行。——八戒,你只在此保守師父,再莫與他廝斗,等老孫往南海走走去來。」八戒道:「哥哥,你去南海何幹?」行者道:「這取經的勾當,原是觀音菩薩;及脫解我等,也是觀音菩薩。今日路阻流沙河,不能前進,不得他,怎生處治?等我去請他,還強如和這妖精相鬥。」八戒道:「也是,也是。師兄,你去時,千萬與我上覆一聲:向日多承指教。」三藏道:「悟空,若是去請菩薩,卻也不必遲疑,快去快來。」

行者即縱觔斗雲,徑上南海。咦!那消半個時辰,早望見普陀山境。須臾間,墜下觔斗,到紫竹林外,又只見那二十四路諸天上前迎著道:「大聖何來?」行者道:「我師有難,特來謁見菩薩。」諸天道:「請坐,容報。」那輪日的諸天徑至潮音洞口報道:「孫悟空有事朝見。」菩薩正與捧珠龍女在寶蓮池畔扶欄看花,聞報,即轉雲巖,開門喚入。大聖端肅皈依參拜。

菩薩問曰:「你怎麼不保唐僧,為甚事又來見我?」行者啟上道:「菩薩,我師父前在高老莊,又收了一個徒弟,喚名豬八戒,多蒙菩薩又賜法諱悟能。才行過黃風嶺,今至八百里流沙河,乃是弱水三千,師父已是難渡;河中又有個妖怪,武藝高強,甚虧了悟能與他水面上大戰三次,只是不能取勝,被他攔阻,不能渡河。因此,特告菩薩,望垂憐憫,濟渡他一濟渡。」菩薩道:「你這猴子,又逞自滿,不肯說出保唐僧的話來麼?」行者道:「我們只是要拿住他,教他送我師父渡河。水裡事,我又弄不得精細。只是悟能尋著他窩巢,與他打話,想是不曾說出取經的勾當。」菩薩道:「那流沙河的妖怪,乃是捲簾大將臨凡,也是我勸化的善信,教他保護取經之輩。你若肯說出是東土取經人時,他決不與你爭持,斷然歸順矣。」行者道:「那怪如今怯戰,不肯上崖,只在水裡潛蹤,如何得他歸順?我師如何得渡弱水?」菩薩即喚惠岸,袖中取出一個紅葫蘆兒,吩咐道:「你可將此葫蘆,同孫悟空到流沙河水面上,只叫『悟淨』,他就出來了。先要引他歸依了唐僧。然後把他那九個骷髏穿在一處,按九宮佈列,卻把這葫蘆安在當中,就是法船一隻,能渡唐僧過流沙河界。」

惠岸聞言,謹遵師命,與大聖捧葫蘆出了潮音洞,奉法旨辭了紫竹林。有詩為證。
\begin{quote}
五行匹配合天真,認得從前舊主人。
煉已立基為妙用,辨明邪正見原因。
金來歸性還同類,木去求情共復淪。
二土全功成寂寞,調和水火沒纖塵。
\end{quote}

他兩個不多時,按落雲頭,早來到流沙河岸。豬八戒認得是木叉行者,引師父上前迎接。那木叉與三藏禮畢,又與八戒相見。八戒道:「向蒙尊者指示,得見菩薩,我老豬果遵法教,今喜拜了沙門。這一向在途中奔碌,未及致謝,恕罪,恕罪。」行者道:「且莫敘闊,我們叫喚那廝去來。」三藏道:「叫誰?」行者道:「老孫見菩薩,備陳前事。菩薩說,這流沙河的妖怪,乃是捲簾大將臨凡,因為在天有罪,墮落此河,忘形作怪。他曾被菩薩勸化,願歸師父往西天去的。但是我們不曾說出取經的事情,故此苦苦爭鬥。菩薩今差木叉將此葫蘆,要與這廝結作法船,渡你過去哩。」三藏聞言,頂禮不盡,對木叉作禮道:「萬望尊者作速一行。」那木叉捧定葫蘆,半雲半霧,徑到了流沙河水面上,厲聲高叫道:「悟淨,悟淨,取經人在此久矣,你怎麼還不歸順?」

卻說那怪懼怕猴王,回於水底,正在窩中歇息,只聽得叫他法名。情知是觀音菩薩;又聞得說「取經人在此」:他也不懼斧鉞,急翻波伸出頭來,又認得是木叉行者。你看他笑盈盈,上前作禮道:「尊者失迎。菩薩今在何處?」木叉道:「我師未來,先差我來吩咐你早跟唐僧做個徒弟。叫把你項下掛的骷髏與這個葫蘆,按九宮結做一隻法船,渡他過此弱水。」悟淨道:「取經人卻在那裡?」木叉用手指道:「那東岸上坐的不是?」悟淨看見了八戒道:「他不知是那裡來的個潑物,與我整鬥了這兩日,何曾言著一個取經的字兒?」又看見行者,道:「這個主子,是他的幫手,好不利害,我不去了。」木叉道:「那是豬八戒,這是孫行者,俱是唐僧的徒弟,俱是菩薩勸化的,怕他怎的?我且和你見唐僧去。」

那悟淨才收了寶杖,整一整黃錦直裰,跳上岸來,對唐僧雙膝跪下道:「師父,弟子有眼無珠,不認得師父的尊容,多有衝撞,萬望恕罪。」八戒道:「你這膿包,怎的早不皈依,只管要與我打?是何說話?」行者笑道:「兄弟,你莫怪他,還是我們不曾說出取經的事情與姓名耳。」長老道:「你果肯誠心皈依吾教麼?」悟淨道:「弟子向蒙菩薩教化,指沙為姓,與我起個法名,喚做沙悟淨,豈有不從師父之理?」三藏道:「既如此,」叫:「悟空,取戒刀來,與他落了髮。」大聖依言,即將戒刀與他剃了頭。又來拜了三藏,拜了行者與八戒,分了大小。三藏見他行禮真像個和尚家風,故又叫他做沙和尚。木叉道:「既秉了迦持,不必絮煩,早與作法船去來。」

那悟淨不敢怠慢,即將頸項下掛的骷髏取下,用索子結作九宮,把菩薩葫蘆安在當中,請師父下岸。那長老遂登法船,坐於上面,果然穩似輕舟。左有八戒扶持,右有悟淨捧托;孫行者在後面牽了龍馬,半雲半霧相跟;頭直上又有木叉擁護。那師父才飄然穩渡流沙河界,浪靜風平過弱河。真個也如飛似箭,不多時,身登彼岸,得脫洪波;又不拖泥帶水,幸喜腳乾手燥,清淨無為,師徒們腳踏實地。那木叉按祥雲,收了葫蘆。又只見那骷髏一時解化作九股陰風,寂然不見。三藏拜謝了木叉,頂禮了菩薩。正是:
\begin{quote}
木叉徑回東洋海,三藏上馬卻投西。
\end{quote}

畢竟不知幾時才得正果求經,且聽下回分解。
