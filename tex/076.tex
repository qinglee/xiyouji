
\chapter{心神居舍魔歸性 木母同降怪體真}

話表孫大聖在老魔肚裡支吾一會,那魔頭倒在塵埃,無聲無氣,若不言語,想是死了,卻又把手放放。魔頭回過氣來,叫一聲:「大慈大悲齊天大聖菩薩。」行者聽見道:「兒子,莫廢工夫,省幾個字兒,只叫孫外公罷。」那妖魔惜命,真個叫:「外公,外公,是我的不是了。一差二誤吞了你,你如今卻反害我。萬望大聖慈悲,可憐螻蟻貪生之意,饒了我命,願送你師父過山也。」大聖雖英雄,甚為唐僧進步。他見妖魔哀告,好奉承的人,也就回了善念,叫道:「妖怪,我饒你,你怎麼送我師父?」老魔道:「我這裡也沒甚麼金銀、珠翠、瑪瑙、珊瑚、琉璃、琥珀、玳瑁珍奇之寶相送。我兄弟三個擡一乘香藤轎兒,把你師父送過此山。」行者笑道:「既是擡轎相送,強如要寶。你張開口,我出來。」那魔頭真個就張開口。那三魔走近前,悄悄的對老魔道:「大哥,等他出來時,把口往下一咬,將猴兒嚼碎,嚥下肚,卻不得磨害你了。」

原來行者在裡面聽得,便不先出去,卻把金箍棒伸出,試他一試。那怪果往下一口,扢喳的一聲,把個門牙都迸碎了。行者抽回棒道:「好妖怪,我倒饒你性命出來,你反咬我,要害我命。我不出來,活活的只弄殺你。不出來,不出來。」老魔報怨三魔道:「兄弟,你是自家人弄自家人了。且是請他出來好了,你卻教我咬他。他倒不曾咬著,卻迸得我牙齦疼痛。這是怎麼起的?」

三魔見老魔怪他,他又作個激將法,厲聲高叫道:「孫行者,聞你名如轟雷貫耳,說你在南天門外施威,靈霄殿下逞勢,如今在西天路上降妖縛怪,原來是個小輩的猴頭。」行者道:「我何為小輩?」三怪道:「『好看千里客,萬里去傳名。』你出來,我與你賭鬥,才是好漢。怎麼在人肚裡做勾當?非小輩而何?」行者聞言,心中暗想道:「是是是。我若如今扯斷他腸,揌破他肝,弄殺這怪,有何難哉?但真是壞了我的名頭。也罷,也罷,你張口,我出來與你比併。但只是你這洞口窄逼,不好使家火,須往寬處去。」三魔聞說,即點大小怪,前前後後,有三萬多精,都執著精銳器械,出洞擺開一個三才陣勢,專等行者出口,一齊上陣。那二怪攙著老魔,徑至門外,叫道:「孫行者,好漢出來,此間有戰場,好鬥。」

大聖在他肚裡,聞得外面鴉鳴鵲噪,鶴唳風聲,知道是寬闊之處。卻想著:「我不出去,是失信與他;若出去,這妖精人面獸心:先時說送我師父,哄我出來咬我,今又調兵在此。也罷,也罷,與他個兩全其美:出去便出去,還與他肚裡生下一個根兒。」即轉手,將尾上毫毛拔了一根,吹口仙氣,叫:「變!」即變一條繩兒,只有頭髮粗細,倒有四十丈長短。那繩兒理出去,見風就長粗了。把一頭拴在妖怪的心肝上,打做個活扣兒。那扣兒不扯不緊,扯緊就痛。卻拿著一頭,笑道:「這一出去,他送我師父便罷;如若不送,亂動刀兵,我也沒工夫與他打,只消扯此繩兒,就如我在肚裡一般。」又將身子變得小小的,往外爬。爬到咽喉之下,見妖精大張著方口,上下鋼牙排如利刃,忽思量道:「不好,不好。若從口裡出去扯這繩兒,他怕疼,往下一嚼,卻不咬斷了?我打他沒牙齒的所在出去。」好大聖,理著繩兒,從他那上齶子往前爬,爬到他鼻孔裡。那老魔鼻子發癢,阿啛的一聲,打了個噴嚏,直迸出行者。

行者見了風,把腰躬一躬,就長了有三丈長短,一隻手扯著繩兒,一隻手拿著鐵棒。那魔頭不知好歹,見他出來了,就舉鋼刀,劈臉來砍。這大聖一隻手使鐵棒相迎。又見那二怪使槍,三怪使戟,沒頭沒臉的亂上。大聖放鬆了繩,收了鐵棒,急縱身駕雲走了。原來怕那夥小妖圍繞,不好幹事。他卻跳出營外,去那空闊山頭上,落下雲,雙手把繩盡力一扯,老魔心裡才疼。他害疼,往上一掙,大聖復往下一扯。眾小妖遠遠看見,齊聲高叫道:「大王,莫惹他,讓他去罷。這猴兒不按時景:清明還未到,他卻那裡放風箏也。」大聖聞言,著力氣蹬了一蹬。那老魔從空中拍剌剌,似紡車兒一般跌落塵埃。就把那山坡下死硬的黃土跌做個二尺淺深之坑。

慌得那二怪、三怪一齊按下雲頭,上前扯住繩兒,跪在坡下,哀告道:「大聖啊,只說你是個寬洪海量之仙,誰知是個鼠腹蝸腸之輩。實實的哄你出來,與你見陣,不期你在我家兄心上拴了一根繩子。」行者笑道:「你這夥潑魔,十分無禮。前番哄我出來就咬我,這番哄我出來卻又擺陣敵我。似這幾萬妖兵戰我一個,理上也不通。扯了去,扯了去見我師父。」那怪一齊叩頭道:「大聖慈悲,饒我性命,願送老師父過山。」行者笑道:「你要性命,只消拿刀把繩子割斷罷了。」老魔道:「爺爺呀!割斷外邊的,這裡邊的拴在心上,喉嚨裡又的惡心,怎生是好?」行者道:「既如此,張開口,等我再進去解出繩來。」老魔慌了道:「這一進去,又不肯出來,卻難也,卻難也。」行者道:「我有本事外邊就可以解得裡面繩頭也。解了可實實的送我師父麼?」老魔道:「但解就送,決不敢打誑語。」大聖審得是實,即便將身一抖,收了毫毛。那怪的心就不疼了。這是孫大聖掩樣的法兒,使毫毛拴著他的心,收了毫毛,所以就不害疼也。三個妖縱身而起,謝道:「大聖請回,上覆唐僧,收拾下行李,我們就擡轎來送。」眾怪偃干戈,盡皆歸洞。

大聖收繩子,徑轉山東,遠遠的看見唐僧睡在地下打滾痛哭,豬八戒與沙僧解了包袱,將行李搭分兒,在那裡分哩。行者暗暗嗟嘆道:「不消講了,這定是八戒對師父說我被妖精吃了,師父捨不得我,痛哭;那獃子卻分東西散火哩。咦!不知可是此意?且等我叫他一聲看。」落下雲頭叫道:「師父。」沙僧聽見,報怨八戒道:「你是個棺材座子——專一害人。師兄不曾死,你卻說他死了,在這裡幹這個勾當,那裡不叫將來了?」八戒道:「我分明看見他被妖精一口吞了。想是日辰不好,那猴子來顯魂哩。」行者到跟前,一把撾住八戒臉,一個巴掌打了個踉蹌道:「夯貨!我顯甚麼魂?」獃子侮著臉道:「哥哥,你實是那怪吃了,你、你怎麼又活了?」行者道:「像你這個不濟事的膿包?他吃了我,我就抓他腸,捏他肺,又把這條繩兒穿住他的心,扯得疼痛難禁,一個個叩頭哀告,我才饒了他性命。如今擡轎來送我師父過山也。」那三藏聞言,一骨魯爬起來,對行者躬身道:「徒弟啊,累殺你了。若信悟能之言,我已絕矣。」行者掄拳打著八戒罵道:「這個饢糠的獃子,十分懈怠,甚不成人。師父,你切莫惱,那怪就來送你也。」沙僧也甚生慚愧,連忙遮掩,收拾行李,扣背馬匹,都在途中等候不題。

卻說三個魔頭帥群精回洞,二怪道:「哥哥,我只道是個九頭八尾的孫行者,原來是恁的個小小猴兒。你不該吞他,只與他鬥時,他那裡鬥得過你我?洞裡這幾萬妖精,吐唾沫也可渰殺他。你卻將他吞在肚裡,他便弄起法來,教你受苦,怎麼敢與他比較?才自說送唐僧,都是假意,實為兄長性命要緊,所以哄他出來,決不送他。」老魔道:「賢弟不送之故,何也?」二怪道:「你與我三千小妖,擺開陣勢,我有本事拿住這個猴頭。」老魔道:「莫說三千,憑你起老營去,只是拿住他,便大家有功。」

那二魔即點三千小妖,徑到大路傍擺開,著一個藍旗手往來傳報道:「孫行者趕早出來,與我二大王爺爺交戰。」八戒聽見,笑道:「哥啊,常言道:『說謊不瞞當鄉人。』就來弄虛頭,搗鬼:怎麼說降了妖精,就擡轎來送師父,卻又來叫戰,何也?」行者道:「老怪已被我降了,不敢出頭,聞著個『孫』字兒,也害頭疼。這定是二妖魔不伏氣送我們,故此叫戰。我道兄弟,這妖精有弟兄三個,這般義氣;我弟兄也是三個,就沒些義氣?我已降了大魔,二魔出來,你就與他戰戰,未為不可。」八戒道:「怕他怎的?等我去打他一仗來。」行者道:「要去便去罷。」八戒笑道:「哥啊,去便去,你把那繩兒借與我使使。」行者道:「你要怎的?你又沒本事鑽在肚裡,你又沒本事拴在他心上,要他何用?」八戒道:「我要扣在這腰間,做個救命索。你與沙僧扯住後手,放我出去,與他交戰。估著贏了他,你便放鬆,我把他拿住;若是輸與他,你把我扯回來,莫教他拉了去。」真個行者暗笑道:「也是捉弄獃子一番。」就把繩兒扣在他腰裡,撮弄他出戰。

那獃子舉釘鈀跑上山崖,叫道:「妖精出來,與你豬祖宗打來。」那藍旗手急報道:「大王,有一個長嘴大耳朵的和尚來了。」二怪即出營,見了八戒,更不打話,挺槍劈面刺來;這獃子舉鈀上前迎住。他兩個在山坡前搭上手,鬥不上七八回合,獃子手軟,架不得妖魔,急回頭叫:「師兄,不好了,扯扯救命索,扯扯救命索。」這壁廂大聖聞言,轉把繩子放鬆了,拋將去。那獃子敗了陣,住後就跑。原來那繩子拖著走,還不覺;轉回來,因鬆了,倒有些絆腳,自家絆倒了一跌,爬起來又一跌。始初還跌個躘踵,後面就跌了個嘴搶地。被妖精趕上,捽開鼻子,就如蛟龍一般,把八戒一鼻子捲住,得勝回洞。眾妖凱歌齊唱,一擁而歸。

這坡下三藏看見,又惱行者道:「悟空,怪不得悟能咒你死哩,原來你兄弟全無相親相愛之意,專懷相嫉相妒之心。他那般說,教你扯扯救命索,你怎麼不扯,還將索子丟去?如今教他被害,卻如之何?」行者笑道:「師父也忒護短,忒偏心。罷了,像老孫拿去時,你略不掛念,左右是捨命之材;這獃子才自遭擒,你就怪我。也教他受些苦惱,方見取經之難。」三藏道:「徒弟啊,你去,我豈不掛念?想著你會變化,斷然不至傷身。那獃子生得狼犺,又不會騰那,這一去,少吉多凶。你還去救他一救。」行者道:「師父不得報怨,等我去救他一救。」

急縱身,趕上山,暗中恨道:「這獃子咒我死,且莫與他個快活。且跟去看那妖精怎麼擺佈他,等他受些罪,再去救他。」即捻訣念起真言,搖身一變,即變做個蟭蟟蟲,飛將去,釘在八戒耳朵根上,同那妖精到了洞裡。二魔帥三千小怪,大吹大打的至洞口屯下。自將八戒拿入裡邊道:「哥哥,我拿了一個來也。」老怪道:「拿來我看。」他把鼻子放鬆,捽下八戒道:「這不是?」老怪道:「這廝沒用。」八戒聞言道:「大王,沒用的放出去,尋那有用的捉來罷。」三怪道:「雖是沒用,也是唐僧的徒弟豬八戒。且綑了,送在後邊池塘裡浸著。待浸退了毛,破開肚子,使鹽醃了晒乾,等天陰下酒。」八戒大驚道:「罷了,罷了,撞見那販醃的妖怪也。」眾怪一齊下手,把獃子四馬攢蹄綑住,扛扛擡擡,送至池塘邊,往中間一推,盡皆轉去。

大聖卻飛起來看處,那獃子四肢朝上,掘著嘴,半浮半沉,嘴裡呼呼的,著然好笑:倒像八九月經霜落了子兒的一個大黑蓮蓬。大聖見他那嘴臉,又恨他,又憐他,說道:「怎的好麼?他也是龍華會上的一個人。但只恨他動不動分行李散火,又要攛掇師父念緊箍咒咒我。我前日曾聞得沙僧說,他攢了些私房,不知可有否?等我且嚇他一嚇看。」

好大聖,飛近他耳邊,假捏聲音,叫聲:「豬悟能,豬悟能。」八戒慌了道:「晦氣呀,我這悟能是觀世音菩薩起的,自跟了唐僧,又呼做八戒,此間怎麼有人知道我叫做悟能?」獃子忍不住問道:「是那個叫我的法名?」行者道:「是我。」獃子道:「你是那個?」行者道:「我是勾司人。」那獃子慌了道:「長官,你是那裡來的?」行者道:「我是五閻王差來勾你的。」獃子道:「長官,你且回去,上覆五閻王,他與我師兄孫悟空交得甚好,教他讓我一日兒,明日來勾罷。」行者道:「胡說。『閻王註定三更死,誰敢留人到四更。』趁早跟我去,免得套上繩子扯拉。」獃子道:」長官,那裡不是方便?看我這般嘴臉,還想活哩。死是一定死,只等一日,這妖精連我師父們都拿來,會一會,就都了帳也。」行者暗笑道:「也罷,我這批上有三十個人,都在這中前後,等我拘將來就你,便有一日耽閣。你可有盤纏?把些兒我去。」八戒道:「可憐啊,出家人那裡有甚麼盤纏?」行者道:「若無盤纏,索了去,跟著我走。」獃子慌了道:「長官不要索。我曉得你這繩兒叫做追命繩,索上就要斷氣。有有有,有便有些兒,只是不多。」行者道:「在那裡?快拿出來。」八戒道:「可憐,可憐,我自做了和尚,到如今,有些善信的人家齋僧,見我食腸大,襯錢比他們略多些兒,我拿了攢在這裡,零零碎碎有五錢銀子。因不好收拾,前者到城中,央了個銀匠煎在一處,他又沒天理,偷了我幾分,只得四錢六分一塊兒。你拿去罷。」行者暗笑道:「這獃子褲子也沒得穿,卻藏在何處?咄!你銀子在那裡?」八戒道:「在我左耳朵眼兒裡揌著哩。我綑了拿不得,你自家拿了去罷。」

行者聞言,即伸手在耳朵竅中摸出,真個是塊馬鞍兒銀子,足有四錢五六分重。拿在手裡,忍不住哈哈的一聲大笑。那獃子認是行者聲音,在水裡亂罵道:「天殺的弼馬溫,到這們苦處,還來打詐財物哩。」行者又笑道:「我把你這饢糟的,老孫保師父,不知受了多少苦難,你倒攢下私房。」八戒道:「嘴臉,這是甚麼私房?都是牙齒上刮下來的,我不捨得買來嘴吃,留了買匹布兒做件衣服,你卻嚇了我的。還分些兒與我。」行者道:「半分也沒得與你。」八戒罵道:「買命錢讓與你罷,好道也救我出去是。」行者道:「莫發急,等我救你。」將銀子藏了,即現原身,掣鐵棒,把獃子划攏,用手提著腳,扯上來,解了繩。八戒跳起來,脫下衣裳,整乾了水,抖一抖,潮漉漉的披在身上,道:「哥哥,開後門走了罷。」行者道:「後門裡走,可是個長進的?還打前門上去。」八戒道:「我的腳綑麻了,跑不動。」行者道:「快跟我來。」

好大聖,把鐵棒一路丟開解數,打將出去。那獃子忍著麻,只得跟定他。只看見二門下靠著的是他的釘鈀,走上前,推開小妖,撈過來往前亂築,與行者打出三四層門,不知打殺了多少小妖。那老魔聽見,對二魔道:「拿得好人,拿得好人。你看孫行者劫了豬八戒,門上打傷小妖也。」那二魔急縱身,綽槍在手,趕出門來,高聲罵道:「潑猢猻,這般無禮,怎敢渺視我等?」大聖聽得,即應聲站下。那怪物不容講,使槍便刺;行者正是會家不忙,掣鐵棒,劈面相迎。他兩個在洞門外,這一場好殺:
\begin{quote}
黃牙老變人形,義結獅王為弟兄。因為大魔來說合,同心計算吃唐僧。齊天大聖神通廣,輔正除邪要滅精。八戒無能遭毒手,悟空拯救出門行。妖王趕上施英猛,槍棒交加各顯能。那一個槍來好似穿林蟒,這一個棒起猶如出海龍。龍出海門雲靄靄,蟒穿林樹霧騰騰。算來都為唐和尚,恨苦相持太沒情。
\end{quote}

那八戒見大聖與妖精交戰,他在山嘴上豎著釘鈀,不來幫打,只管呆呆的看著。那妖精見行者棒重,滿身解數,全無破綻,就把槍架住,捽開鼻子,要來捲他。行者知道他的勾當,雙手把金箍棒橫起來,往上一舉。被妖精一鼻子捲住腰胯,不曾捲手。你看他兩隻手在妖精鼻頭上丟花棒兒耍子。八戒見了,搥胸道:「咦!那妖怪晦氣呀。捲我這夯的,連手都捲住了,不能得動;捲那們滑的,倒不捲手。他那兩隻手拿著棒,只消往鼻裡一搠,那孔子裡害疼流涕,怎能捲得他住?」

行者原無此意,倒是八戒教了他。他就把棒幌一幌,小如雞子,長有丈餘,真個往他鼻孔裡一搠。那妖精害怕,沙的一聲,把鼻子捽放。被行者轉手過來,一把撾住,用氣力往前一拉。那妖精護疼,隨著手,舉步跟來。八戒方才敢近,拿釘鈀望妖精胯子上亂築。行者道:「不好,不好。那鈀齒兒尖,恐築破皮,淌出血來,師父看見,又說我們傷生。只調柄子來打罷。」

真個獃子舉鈀柄,走一步,打一下;行者牽著鼻子:就似兩個象奴,牽至坡下。只見三藏凝睛盼望,見他兩個嚷嚷鬧鬧而來,即喚:「悟淨,你看悟空牽的是甚麼?」沙僧見了,笑道:「師父,大師兄把妖精揪著鼻子拉來,真愛殺人也。」三藏道:「善哉!善哉!那般大個妖精,那般長個鼻子。你且問他:他若喜喜歡歡送我等過山,可饒了他,莫傷他性命。」沙僧急縱前迎著,高聲叫道:「師父說:那怪果送師父過山,教不要傷他命哩。」那怪聞說,連忙跪下,口裡嗚嗚的答應。原來被行者揪著鼻子,捏儾了,就如重傷風一般。叫道:「唐老爺,若肯饒命,即便擡轎相送。」行者道:「我師徒俱是善勝之人,依你言,且饒你命。快擡轎來,如再變卦,拿住決不再饒。」那怪得脫手,磕頭而去。行者同八戒見唐僧,備言前事。八戒慚愧不勝,在坡前晾晒衣服,等候不題。

那二魔戰戰兢兢回洞。未到時,已有小妖報知老魔、三魔,說二魔被行者揪著鼻子拉去。老魔悚懼,與三魔帥眾方出,見二魔獨回,又皆接入,問及放回之故。二魔把三藏慈憫善勝之言,對眾說了一遍。一個個面面相覷,更不敢言。二魔道:「哥哥可送唐僧麼?」老魔道:「兄弟,你說那裡話?孫行者是個廣施仁義的猴頭:他先在我肚裡,若肯害我性命,一千個也被他弄殺了;卻才揪住你鼻子,若是扯了去不放回,只捏破你的鼻子頭兒,卻也惶恐。快早安排送他去罷。」三魔笑道:「送送送。」老魔道:「賢弟這話,卻又像尚氣的了。你不送,我兩個送去罷。」

三魔又笑道:「二位兄長在上:那和尚倘不要我們送,只這等瞞過去,還是他的造化;若要送,不知正中了我的調虎離山之計哩。」老怪道:」何為『調虎離山』?」三怪道:「如今把滿洞群妖點將起來,萬中選千,千中選百,百中選十六個,又選三十個。」老怪道:「怎麼既要十六,又要三十?」三怪道:「要三十個會烹煮的,與他些精米、細麵、竹筍、茶芽、香蕈、蘑菇、豆腐、麵筋,著他二十里或三十里,搭下窩鋪,安排茶飯,管待唐僧。」老怪道:「又要十六個何用?」三怪道:「著八個擡,八個喝路。我弟兄相隨左右,送他一程。此去向西四百餘里,就是我的城池,我那裡自有接應的人馬。若至城邊,如此如此,著他師徒首尾不能相顧。要捉唐僧,全在此十六個身上成功。」老怪聞言,歡欣不已;真是如醉方醒,似夢方覺。道:「好好好!」即點眾妖,先選三十,與他物件;又選十六,擡一頂香藤轎子:同出門來。又吩咐眾妖:「俱不許上山閑走:孫行者是個多心的猴子,若見汝等往來,他必生疑,識破此計。」

老怪遂帥眾至大路傍高叫道:「唐老爺,今日不犯紅沙,請老爺早早過山。」三藏聞之道:「悟空,是甚人叫我?」行者指定道:「那廂是老孫降伏的妖精擡轎來送你哩。」三藏合掌朝天道:「善哉!善哉!若不是賢徒如此之能,我怎生得去?」徑直向前,對眾妖作禮道:「多承列位之愛,我弟子取經東回向長安,當傳揚善果也。」眾妖叩首道:「請老爺上轎。」那三藏肉眼凡胎,不知是計。孫大聖又是太乙金仙,忠正之性,只以為擒縱之功,降了妖怪,亦豈期他都有異謀?卻也不曾詳察,盡著師父之意。即命八戒將行李捎在馬上,與沙僧緊隨。他使鐵棒向前開路,顧盼吉凶。八個擡起轎子,八個一遞一聲喝道,三個妖扶著轎扛。師父喜喜歡歡的端坐轎上。上了高山,依大路而行。

此一去,豈知歡喜之間愁又至。經云:「泰極否還生。」時運相逢真太歲,又值喪門吊客星。那夥妖魔同心合意的侍衛左右,早晚慇懃。行經三十里獻齋,五十里又齋,未晚請歇,沿路齊齊整整。一日三餐,遂心滿意;良宵一宿,好處安身。

西進有四百里餘程,忽見城池相近。大聖舉鐵棒,離轎僅有一里之遙,見城池,把他嚇了一跌,掙挫不起。你道他只這般大膽,如何見此著諕?原來望見那城中有許多惡氣。乃是:
\begin{quote}
攢攢簇簇妖魔怪,四門都是狼精靈。
斑斕老虎為都管,白面雄彪作總兵。
丫叉角鹿傳文引,伶俐狐狸當道行。
千尺大蟒圍城走,萬丈長蛇占路程。
樓下蒼狼呼令使,臺前花豹作人聲。
搖旗擂鼓皆妖怪,巡更坐鋪盡山精。
狡兔開門弄買賣,野豬挑擔幹營生。
先年原是天朝國,如今翻作虎狼城。
\end{quote}

那大聖正當悚懼,只聽得耳後風響。急回頭觀看,原來是三魔雙手舉一柄畫桿方天戟,往大聖頭上打來;大聖急翻身爬起,使金箍棒劈面相迎。他兩個各懷惱怒,氣呼呼,更不打話;咬著牙,各要相爭。又見那老魔頭傳號令,舉鋼刀便砍八戒;八戒慌得丟了馬,掄著鈀,向前亂築。那二魔纏長槍,望沙僧刺來;沙僧使降妖杖,支開架子敵住。三個魔頭與三個和尚,一個敵一個,在那山頭捨死忘生苦戰。

那十六個小妖卻遵號令,各各效能:搶了白馬、行囊,把三藏一擁,擡著轎子,徑至城邊,高叫道:「大王爺爺定計,已拿得唐僧來了。」那城上大小妖精,一個個跑下,將城門大開。吩咐各營捲旗息鼓,不許吶喊篩鑼。說:「大王原有令在前,不許嚇了唐僧。唐僧禁不得恐嚇,一嚇就肉酸,不中吃了。」眾妖都歡天喜地邀三藏,控背躬身接主僧。把唐僧一轎子擡上金鑾殿,請他坐在當中,一壁廂獻茶獻飯,左右旋繞。那長老昏昏沉沉,舉眼無親。

畢竟不知性命何如,且聽下回分解。
