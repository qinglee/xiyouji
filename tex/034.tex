
\chapter{魔王巧算困心猿 大聖騰那騙寶貝}

卻說那兩個小妖將假葫蘆拿在手中,爭看一會,忽擡頭不見了行者。伶俐蟲道:「哥啊,神仙也會打誑語。他說換了寶貝,度我等成仙,怎麼不辭就去了?」精細鬼道:「我們相應便宜的多哩,他敢去得成?拿過葫蘆來,等我裝裝天,也試演試演看。」真個把葫蘆往上一拋,撲的就落將下來。慌得個伶俐蟲道:「怎麼不裝?不裝?莫是孫行者假變神仙,將假葫蘆換了我們的真的去耶?」精細鬼道:「不要胡說,孫行者是那三座山壓住了,怎生得出?拿過來,等我念他那幾句咒兒裝了看。」這怪也把葫蘆兒望空丟起,口中念道:「若有半聲不肯,就上靈霄殿上,動起刀兵。」念不了,撲的又落將下來。兩妖道:「不裝,不裝,一定是個假的。」

正嚷處,孫大聖在半空裡聽得明白,看得真實,恐怕他弄得時辰多了,緊要處走了風汛,將身一抖,把那變葫蘆的毫毛收上身來,弄得那兩個妖四手皆空。精細鬼道:「兄弟,拿葫蘆來。」伶俐蟲道:「你拿著的。」「天呀!怎麼不見了?」都去地下亂摸,草裡胡尋,吞袖子,揣腰間,那裡得有?二妖嚇得呆呆掙掙道:「怎的好?怎的好?當時大王將寶貝付與我們,教拿孫行者。今行者既不曾拿得,連寶貝都不見了,我們怎敢去回話?這一頓直直的打死了也。怎的好!怎的好!」伶俐蟲道:「我們走了罷。」精細鬼道:「往那裡走麼?」伶俐蟲道:「不管那裡走罷。若回去說沒寶貝,斷然是送命了。」精細鬼道:「不要走,還回去。二大王平日看你甚好,我推一句兒在你身上。他若肯將就,留得性命;說不過,就打死,還在此間。莫弄得兩頭不著。去來,去來。」那怪商議了,轉步回山。

行者在半空中見他回去,又搖身一變,變作蒼蠅兒,飛下去,跟著小妖。你道他既變了蒼蠅,那寶貝卻放在何處?如丟在路上,藏在草裡,被人看見拿去,卻不是勞而無功?他還帶在身上。帶在身上啊,蒼蠅不過豆粒大小,如何容得?原來他那寶貝,與他金箍棒相同,叫做如意佛寶,隨身變化,可以大,可以小,故身上亦可容得。

他嚶的一聲飛下去,跟定那怪。不一時,到了洞裡。只見那兩個魔頭坐在那裡飲酒,小妖朝上跪下。行者就釘在那門櫃上,側耳聽著。小妖道:「大王。」二老魔即停杯道:「你們來了?」小妖道:「來了。」又問:「拿著孫行者否?」小妖叩頭,不敢聲言。老魔又問,又不敢應,只是叩頭。問之再三,小妖俯伏在地:「赦小的萬千死罪,赦小的萬千死罪。我等執著寶貝,走到半山之中,忽遇著蓬萊山一個神仙。他問我們那裡去,我們答道:『拿孫行者去。』那神仙聽見說孫行者,他也惱他,要與我們幫工。是我們不曾叫他幫工,卻將拿寶貝裝人的情由,與他說了。那神仙也有個葫蘆,善能裝天。我們也是妄想之心,養家之意:他的裝天,我的裝人,與他換了罷。原說葫蘆換葫蘆,伶俐蟲又貼他個淨瓶。誰想他仙家之物,近不得凡人之手。正試演處,就連人都不見了。萬望饒小的們死罪。」

老魔聽說,暴躁如雷道:「罷了,罷了,這就是孫行者假妝神仙騙哄去了。那猴頭神通廣大,處處人熟,不知那個毛神放他出來,騙去寶貝。」二魔道:「兄長息怒。叵耐那猴頭著然無禮,既有手段,便走了也罷,怎麼又騙寶貝?我若沒本事拿他,永不在西方路上為怪。」老魔道:「怎生拿他?」二魔道:「我們有五件寶貝,去了兩件,還有三件,務要拿住他。」老魔道:「還有那三件?」二魔道:「還有七星劍與芭蕉扇在我身邊,那一條幌金繩,在壓龍山壓龍洞老母親那裡收著哩。如今差兩個小妖去請母親來吃唐僧肉,就教他帶幌金繩來拿孫行者。」老魔道:「差那個去?」二魔道:「不差這樣廢物去。」將精細鬼、伶俐蟲一聲喝起。二人道:「造化,造化,打也不曾打,罵也不曾罵,卻就饒了。」二魔道:「叫那常隨的伴當巴山虎、倚海龍來。」二人跪下,二魔吩咐道:「你卻要小心。」俱應道:「小心。」「卻要仔細。」俱應道:「仔細。」又問道:「你認得老奶奶家麼?」又俱應道:「認得。」「你既認得,你快早走動,到老奶奶處,多多拜上,說請吃唐僧肉哩;就著帶幌金繩來,要拿孫行者。」

二怪領命疾走。怎知那行者在傍,一一聽得明白。他展開翅,飛將去,趕上巴山虎,釘在他身上。行經二三里,就要打殺他兩個。又思道:「打死他,有何難事?但他奶奶身邊有那幌金索,又不知住在何處,等我且問他一問再打。」好行者,嚶的一聲,躲離小妖,讓他先行有百十步。卻又搖身一變,也變做個小妖兒,戴一頂狐皮帽子,將虎皮裙子倒插上來勒住,趕上道:「走路的,等我一等。」那倚海龍回頭問道:「是那裡來的?」行者道:「好哥啊,連自家人也認不得?」小妖道:「我家沒有你。」行者道:「怎麼沒我?你再認認我。」小妖道:「面生,面生,不曾相會。」行者道:「正是。你們不曾會著我,我是外班的。」小妖道:「外班長官,是不曾會。你往那裡去?」行者道:「大王說差你二位請老奶奶來吃唐僧肉,教他就帶幌金繩來,拿孫行者。恐你二位走得緩,有些貪頑,誤了正事,又差我來催你們快去。」

小妖見說著海底眼,更不疑惑,把行者果認做一家人。急急忙忙,往前飛跑,一氣又跑有八九里。行者道:「忒走快了些。我們離家有多少路了?」小怪道:「有十五六里了。」行者道:「還有多遠?」倚海龍用手指道:「烏林子裡就是。」行者擡頭見一帶黑林不遠,料得那老怪只在林子裡外。卻立定步,讓那小怪前走。即取出鐵棒,走上前,著腳後一刮。可憐忒不禁打,就把兩個小妖刮做一團肉餅。卻拖著腳,藏在路傍深草科裡。即便拔下一根毫毛,吹口仙氣,叫:「變!」變做個巴山虎,自身卻變做個倚海龍。假妝做兩個小妖,徑往那壓龍洞請老奶奶。這叫做七十二變神通大,指物騰那手段高。

三五步,跳到林子裡。正找尋處,只見有兩扇石門,半開半掩,不敢擅入。只得洋叫一聲:「開門!開門!」早驚動那把門的一個女怪,將那半扇兒開了,道:「你是那裡來的?」行者道:「我是平頂山蓮花洞裡差來請老奶奶的。」女怪道:「進去。」到了二層門下,閃著頭,往裡觀看,又見那正當中坐著一個老媽媽兒。你道他怎生模樣?但見:
\begin{quote}
雪鬢蓬鬆,星光晃亮。臉皮紅潤皺文多,牙齒稀疏神氣壯。貌似花殘霜裡色,形如松老雨餘顏。頭纏白練攢絲帕,耳墜黃金嵌寶環。
\end{quote}

孫大聖見了,不敢進去,只在二門外仵著臉,脫脫的哭起來,你道他哭怎的,莫成是怕他?就怕也便不哭,況先哄了他的寶貝,又打死他的小妖,卻為何而哭?他當時曾下九鼎油鍋,就煠了七八日,也不曾有一點淚兒。只為想起唐僧取經的苦惱,他就淚出痛腸,放眼便哭。心卻想道:「老孫既顯手段,變做小妖,來請這老怪,沒有個直直的站了說話之理,一定見他磕頭才是。我為人做了一場好漢,止拜了三個人:西天拜佛祖;南海拜觀音;兩界山師父救了我,我拜了他四拜。為他使碎六葉連肝肺,用盡三毛七孔心。一卷經能值幾何?今日卻教我去拜此怪。若不跪拜,必定走了風汛。苦啊!算來只為師父受困,故使我受辱於人。」

到此際也沒及奈何,撞將進去,朝上跪下道:「奶奶磕頭。」那怪道:「我兒,起來。」行者暗道:「好好好,叫得結實!」老怪問道:「你是那裡來的?」行者道:「平頂山蓮花洞,蒙二位大王有令,差來請奶奶去吃唐僧肉,教帶幌金繩,要拿孫行者哩。」老怪大喜道:「好孝順的兒子。」就去叫擡出轎來。行者道:「我的兒啊,妖精也擡轎?」後壁廂即有兩個女怪擡出一頂香藤轎,放在門外,掛上青絹緯幔。老怪起身出洞,坐在轎裡。後有幾個小女妖捧著減粧,端著鏡架,提著手巾,托著香盒,跟隨左右。那老怪道:「你們來怎的?我往自家兒子去處,愁那裡沒人伏侍,要你們去獻勤塌嘴?都回去,關了門看家。」那幾個小妖果俱回去,止有兩個擡轎的。老怪問道:「那差來的叫做甚麼名字?」行者連忙答應道:「他叫做巴山虎,我叫做倚海龍。」老怪道:「你兩個前走,與我開路。」行者暗想道:「可是晦氣,經倒不曾取得,且來替他做皂隸。」卻又不敢抵強,只得向前引路,大四聲喝起。

行了五六里遠近,他就坐在石崖上,等候那擡轎的到了,行者道:「略歇歇如何?壓得肩頭疼啊。」小怪那知甚麼訣竅,就把轎子歇下。行者在轎後,胸脯上拔下一根毫毛,變做一個大燒餅,抱著啃。轎夫道:「長官,你吃的是甚麼?」行者道:「不好說。這遠的路,來請奶奶,沒些兒賞賜,肚裡饑了,原帶來的乾糧,等我吃些兒再走。」轎夫道:「把些兒我們吃吃。」行者笑道:「來麼,都是一家人,怎麼計較?」那小妖不知好歹,圍住行者,分其乾糧。被行者掣出棒,著頭一磨:一個搪著的,打得稀爛;一個擦著的,不死還哼。那老怪聽得人哼,轎子裡伸出頭來看時,被行者跳到轎前,劈頭一棍,打了個窟窿,腦漿迸流,鮮血直冒。拖出轎來看處,原是個九尾狐狸。行者笑道:「造孽畜,叫甚麼老奶奶。你叫老奶奶,就該稱老孫做上太祖公公是。」

好猴王,把他那幌金繩搜出來,籠在袖裡,歡喜道:「那潑魔縱有手段,已此三件兒寶貝姓孫了。」卻又拔兩根毫毛變做個巴山虎、倚海龍;又拔兩根變做兩個擡轎的;他卻變做老奶奶模樣,坐在轎裡。將轎子擡起,徑回本路。

不多時,到了蓮花洞口,那毫毛變的小妖俱在前道:「開門!開門!」內有把門的小妖開了門道:「巴山虎、倚海龍來了?」毫毛道:「來了。」「你們請的奶奶呢?」毫毛用手指道:「那轎內的不是?」小怪道:「你且住,等我進去先報。」報道:「大王,奶奶來耶。」兩個魔頭聞說,即命排香案來接。行者聽得,暗喜道:「造化,也輪到我為人了。我先變小妖,去請老怪,磕了他一個頭;這番來,我變老怪,是他母親,定行四拜之禮,雖不怎的,好道也賺他兩個頭兒。」好大聖,下了轎子,抖抖衣服,把那四根毫毛收在身上。那把門的小妖把空轎擡入門裡。他卻隨後徐行,那般嬌嬌啻啻,扭扭捏捏,就像那老怪的行動,徑自進去。又只見大小群妖,都來跪接。鼓樂簫韶,一派響喨;博山爐裡,靄靄香煙。他到正廳中,南面坐下。兩個魔頭,雙膝跪倒,朝上叩頭,叫道:「母親,孩兒拜揖。」行者道:「我兒起來。」

卻說豬八戒吊在梁上,哈哈的笑了一聲。沙僧道:「二哥,好啊,吊出笑來也。」八戒道:「兄弟,我笑中有故。」沙僧道:「甚故?」八戒道:「我們只怕是奶奶來了,就要蒸吃。原來不是奶奶,是舊話來了。」沙僧道:「甚麼舊話?」八戒笑道:「弼馬溫來了。」沙僧道:「你怎麼認得是他?」八戒道:「彎倒腰,叫『我兒起來』,那後面就掬起猴尾巴子。我比你吊得高,所以看得明也。」沙僧道:「且不要言語,聽他說甚麼話。」八戒道:「正是,正是。」

那孫大聖坐在中間,問道:「我兒,請我來有何事幹?」魔頭道:「母親啊,連日兒等少禮,不曾孝順得。今早愚兄弟拿得東土唐僧,不敢擅吃,請母親來獻獻生,好蒸與母親吃了延壽。」行者道:「我兒,唐僧的肉,我倒不吃。聽見有個豬八戒的耳朵甚好,可割將下來,整治整治,我下酒。」那八戒聽見慌了道:「遭瘟的,你來為割我耳朵的,我喊出來不好聽啊。」

噫!只為獃子一句通情話,走了猴王變化的風。那裡有幾個巡山的小怪,把門的眾妖,都撞將進來,報道:「大王,禍事了!孫行者打殺奶奶,他妝來耶。」魔頭聞此言,那容分說,掣七星寶劍,望行者劈面砍來。好大聖,將身一幌,只見滿洞紅光,預先走了。似這般手段,著實好耍子。正是那聚則成形,散則成氣。

諕得個老魔頭魂飛魄散,眾群精噬指搖頭。老魔道:「兄弟,把唐僧與沙僧、八戒、白馬、行李都送還那孫行者,閉了是非之門罷。」二魔道:「哥哥,你說那裡話?我不知費了多少辛勤,施這計策,將那和尚都攝將來。如今似你這等怕懼孫行者的詭譎,就俱送去還他,真所謂畏刀避劍之人,豈大丈夫之所為也?你且請坐勿懼。我聞你說孫行者神通廣大,我雖與他相會一場,卻不曾與他比試。取披掛來,等我尋他交戰三合。假若他三戰勝我不過,唐僧還是我們之食;如三戰我不能勝他,那時再送唐僧與他未遲。」老魔道:「賢弟說得是。」教取披掛。

眾妖擡出披掛,二魔結束齊整,執寶劍,出門外,叫聲:「孫行者,你往那裡走了?」此時大聖已在雲端裡,聞得叫他名字,急回頭觀看,原來是那二魔。你看他怎生打扮:
\begin{quote}
頭戴鳳盔欺臘雪,身披戰甲幌鑌鐵。
腰間帶是蟒龍觔,粉皮靴靿梅花摺。
顏如灌口活真君,貌比巨靈無二別。
七星寶劍手中擎,怒氣沖霄威烈烈。
\end{quote}

二魔高叫道:「孫行者,快還我寶貝與我母親來,我饒你唐僧取經去。」大聖忍不住罵道:「這潑怪物,錯認了你孫外公。趕早兒送還我師父、師弟、白馬、行囊,仍打發我些盤纏,往西走路;若牙縫裡道半個『不』字,就自家搓根繩兒去罷,也免得你外公動手。」二魔聞言,急縱雲,跳在空中,掄寶劍來刺;行者掣鐵棒劈面相迎。他兩個在半空中,這場好殺:
\begin{quote}
棋逢對手,將遇良才。棋逢對手難藏興,將遇良才可用功。那兩員神將相交,好便似南山虎鬥,北海龍爭。龍爭處,鱗甲生輝;虎鬥時,爪牙亂落。爪牙亂落撒銀鉤,鱗甲生輝支鐵葉。這一個翻翻復復,有千般解數;那一個來來往往,無半點放閑。金箍棒,離頂門只隔三分;七星劍,向心窩惟爭一蹍。那個威風逼得斗牛寒,這個怒氣勝如雷電險。
\end{quote}

他兩個戰了有三十回合,不分勝負。

行者暗喜道:「這潑怪倒也架得住老孫的鐵棒。我已得了他三件寶貝,卻這般苦苦的與他廝殺,可不誤了我的工夫?不若拿葫蘆或淨瓶裝他去,多少是好?」又想道:「不好,不好。常言道:『物隨主便。』倘若我叫他不答應,卻又不誤了事業?且使幌金繩扣頭罷。」好大聖,一隻手使棒,架住他的寶貝;一隻手把那繩執起,刷喇的扣了魔頭。

原來那魔頭有個緊繩咒,有個鬆繩咒。若扣住別人,就念緊繩咒,莫能得脫;若扣住自家人,就念鬆繩咒,不得傷身。他認得是自家的寶貝,即念鬆繩咒,把繩鬆動,便脫出來,反望行者拋將去,卻早扣住了大聖。大聖正要使「瘦身法」,想要脫身,卻被那魔念動緊繩咒,緊緊扣住,怎能得脫?褪至頸項之下,原是一個金圈子套住。那怪將繩一扯,扯將下來,照光頭上砍了七八寶劍。行者頭皮兒也不曾紅了一紅。那魔道:「這猴子,你這等頭硬,我不砍你;且帶你回去,再打你。將我那兩件寶貝趁早還我。」行者道:「我拿你甚麼寶貝,你問我要?」那魔頭將身上細細搜檢,卻將那葫蘆、淨瓶都搜出來。又把繩子牽著,帶至洞裡道:「兄長,拿將來了。」老魔道:「拿了誰來?」二魔道:「孫行者。你來看,你來看。」老魔一見,認得是行者,滿面喜笑道:「是他,是他。把他長長的繩兒拴在柱科上耍子。」真個把行者拴住,兩個魔頭卻進後面堂裡飲酒。

那大聖在柱根下爬蹉,忽驚動八戒。那獃子吊在梁上,哈哈的笑道:「哥哥啊,耳朵吃不成了。」行者道:「獃子,可吊得自在麼?我如今就出去,管情救了你們。」八戒道:「不羞,不羞。本身難脫,還想救人。罷罷罷,師徒們都在一處死了,好到陰司裡問路。」行者道:「不要胡說,你看我出去。」八戒道:「我看你怎麼出去?」那大聖口裡與八戒說話,眼裡卻抹著那兩個妖怪。見他在裡邊吃酒,有幾個小妖拿盤拿盞,執壺釃酒,不住的兩頭亂跑,關防的略鬆了些兒。他見面前無人,就弄神通:順出棒來,吹口仙氣,叫:「變!」即變做一個純鋼的銼兒;扳過那頸項的圈子,三五銼,銼做兩段。拔開銼口,脫將出來。拔了一根毫毛,叫變做一個假身,拴在那裡。真身卻幌一幌,變做個小妖,立在旁邊。八戒又在梁上喊道:「不好了,不好了,拴的是假貨,吊的是正身。」老魔停杯便問:「那豬八戒吆喝的是甚麼?」行者已變做小妖,上前道:「豬八戒攛道孫行者教變化走了罷,他不肯走,在那裡吆喝哩。」二魔道:「還說豬八戒老實,原來這等不老實,該打二十多嘴棍。」

這行者就去拿條棍來打。八戒道:「你打輕些兒;若重了些兒,我又喊起,我認得你。」行者道:「老孫變化,也只為你們,你怎麼倒走了風息?這一洞裡妖精,都認不得,怎的偏你認得?」八戒道:「你雖變了頭臉,還不曾變得屁股,那屁股上兩塊紅不是?我因此認得是你。」行者隨往後面,演到廚中,鍋底上摸了一把,將兩臀擦黑,行至前邊。八戒看見,又笑道:「那個猴子去那裡混了這一會,弄做個黑屁股來了。」

行者仍站在跟前,要偷他寶貝。真個甚有見識:走上廳,對那怪扯個腿子道:「大王,你看那孫行者拴在柱上,左右爬蹉,磨壞那根金繩,得一根粗壯些的繩子換將下來才好。」老魔道:「說得是。」即將腰間的獅蠻帶解下,遞與行者。行者接了帶,把假妝的行者拴住。換下那條繩子,一窩兒窩兒籠在袖內。又拔一根毫毛,吹口仙氣,變作一根假幌金繩,雙手送與那怪。那怪只因貪酒,那曾細看,就便收下。這個是大聖騰那弄本事,毫毛又換幌金繩。

得了這件寶貝,急轉身跳出門外,現了原身,高叫:「妖怪!」那把門的小妖問道:「你是甚人,在此呼喝?」行者道:「你快早進去報與你那潑魔,說者行孫來了。」那小妖如言報告,老魔大驚道:「拿住孫行者,又怎麼有個者行孫?」二魔道:「哥哥,怕他怎的?寶貝都在我手裡,等我拿那葫蘆出去,把他裝將來。」老魔道:「兄弟仔細。」二魔拿了葫蘆,走出山門,忽看見與孫行者模樣一般,只是略矮些兒。問道:「你是那裡來的?」行者道:「我是孫行者的兄弟,聞說你拿了我家兄,卻來與你尋事的。」二魔道:「是我拿了,鎖在洞中。你今既來,必要索戰。我也不與你交兵,我且叫你一聲,你敢應我麼?」行者道:「可怕你叫上千聲,我就答應你萬聲!」那魔執了寶貝,跳在空中,把底兒朝天,口兒朝地,叫聲:「者行孫。」行者卻不敢答應,心中暗想道:「若是應了,就裝進去哩。」那魔道:「你怎麼不應我?」行者道:「我有些耳閉,不曾聽見。你高叫。」那怪物又叫聲:「者行孫。」行者在底下掐著指頭算了一算,道:「我真名字叫做孫行者,起的鬼名字叫做者行孫。真名字可以裝得,鬼名字好道裝不得。」卻就忍不住應了他一聲。颼的被他吸進葫蘆去,貼上帖兒。原來那寶貝,那管甚麼名字真假,但綽個應的氣兒,就裝了去也。

大聖到他葫蘆裡,渾然烏黑。把頭往上一頂,那裡頂得動,且是塞得甚緊,卻才心中焦躁道:「當時我在山上遇著那兩個小妖,他曾告訴我說:不拘葫蘆、淨瓶,把人裝在裡面,只消一時三刻,就化為膿了,敢莫化了我麼?」一條心又想著道:「沒事,化不得我。我老孫五百年前大鬧天宮,被太上老君放在八卦爐中煉了四十九日,煉成個金子心肝,銀子肺腑,銅頭鐵背,火眼金睛,那裡一時三刻就化得我?且跟他進去,看他怎的。」

二魔拿入裡面道:「哥哥,拿來了。」老魔道:「拿了誰?」二魔道:「者行孫是我裝在葫蘆裡也。」老魔歡喜道:「賢弟,請坐。不要動,只等搖得響再揭帖兒。」行者聽得道:「我這般一個身子,怎麼便搖得響?只除化成稀汁,才搖得響是。等我撒泡溺罷,他若搖得響時,一定揭帖起蓋,我乘空走他娘罷。」又思道,「不好,不好。溺雖可響,只是污了這直裰。等他搖時,我但聚些唾津漱口,稀漓呼喇的,哄他揭開,老孫再走罷。」大聖作了準備,那怪貪酒不搖。大聖作個法,意思只是哄他來搖,忽然叫道:「天呀!孤拐都化了。」那魔也不搖。大聖又叫道:「娘啊!連腰截骨都化了。」老魔道:「化至腰時,都化盡矣,揭起帖兒看看。」

那大聖聞言,就拔了一根毫毛,叫:「變!」變作個半截的身子,在葫蘆底上。真身卻變做個蟭蟟蟲兒,釘在那葫蘆口邊。只見那二魔揭起帖子看時,大聖早已飛出。打個滾,又變做個倚海龍。倚海龍卻是原去請老奶奶的那個小妖,他變了,站在旁邊。那老魔扳著葫蘆口張了一張,見是個半截身子動耽,他也不認真假,慌忙叫:「兄弟,蓋上,蓋上,還不曾化得了哩。」二魔依舊貼上。大聖在傍暗笑道:「不知老孫已在此矣。」

那老魔拿了壺,滿滿的斟了一杯酒,近前雙手遞與二魔道:「賢弟,我與你遞個鍾兒。」二魔道:「兄長,我們已吃了這半會酒,又遞甚鍾?」老魔道:「你拿住唐僧、八戒、沙僧猶可,又索了孫行者,裝了者行孫,如此功勞,該與你多遞幾鍾。」二魔見哥哥恭敬,怎敢不接,但一隻手托著葫蘆,一隻手不敢去接,卻把葫蘆遞與倚海龍,雙手去接杯。不知那倚海龍是孫行者變的。你看他端葫蘆,慇懃奉侍。二魔接酒吃了,也要回奉一杯。老魔道:「不消回酒,我這裡陪你一杯罷。」兩人只管謙遜。行者頂著葫蘆,眼不轉睛,看他兩個左右傳杯,全無計較,他就把個葫蘆揌入衣袖。拔根毫毛,變個假葫蘆,一樣無二,捧在手中。那魔遞了一會酒,也不看真假,一把接過寶貝。各上席,安然坐下,依然飲酒。孫大聖撤身走過,得了寶貝,心中暗喜道:「饒這魔頭有手段,畢竟葫蘆還姓孫。」

畢竟不知向後怎麼施為,方得救師滅怪,且聽下回分解。
