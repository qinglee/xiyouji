
\chapter{妖邪假設小雷音 四眾皆遭大厄難}

\begin{quote}
這回因果,勸人為善,切休作惡。一念生,神明照鑒,任他為作。拙蠢乖能君怎學,兩般還是無心藥。趁生前有道正該修,莫浪泊。認根源,脫本殼。訪長生,須把捉。要時時明見,醍醐斟酌。貫徹三關填黑海,管教善者乘鸞鶴。那其間愍故更慈悲,登極樂。
\end{quote}

話表唐三藏一念虔誠,且休言天神保護,似這草木之靈,尚來引送,雅會一宵,脫出荊棘針刺,再無蘿蓏攀纏。四眾西進,行夠多時,又值冬殘,正是那三春之日:
\begin{quote}
物華交泰,斗柄回寅。草芽遍地綠,柳眼滿堤青。一嶺桃花紅錦涴,半溪煙水碧羅明。幾多風雨,無限心情。日晒花心豔,燕啣苔蕊輕。山色王維畫濃淡,鳥聲季子舌縱橫。芳菲鋪繡無人賞,蝶舞蜂歌卻有情。
\end{quote}

師徒們也自尋芳踏翠,緩隨馬步。正行之間,忽見一座高山,遠望著與天相接。三藏揚鞭指道:「悟空,那座山也不知有多少高,可便似接著青天,透沖碧漢。」行者道:「古詩不云:『只有天在上,更無山與齊。』但言山之極高,無可與他比並,豈有接天之理?」八戒道:「若不接天,如何把崑崙山號為天柱?」行者道:「你不知。自古『天不滿西北』。崑崙山在西北乾位上,故有頂天塞空之意,遂名天柱。」沙僧笑道:「大哥把這好話兒莫與他說,他聽了去,又降別人。我們且走路,等上了那山,就知高下也。」

那獃子趕著沙僧,廝耍廝鬥。老師父馬快如飛。須臾,到那山崖之邊,一步步往上行來。只見那山:
\begin{quote}
林中風颯颯,澗底水潺潺。鴉雀飛不過,神仙也道難。千崖萬壑,億曲百灣。塵埃滾滾無人到,怪石森森不厭看。有處有雲如水滉,是方是樹鳥聲繁。鹿啣芝去,猿摘桃還。狐貉往來崖上跳,麖獐出入嶺頭頑。忽聞虎嘯驚人膽,斑豹蒼狼把路攔。
\end{quote}

唐三藏一見心驚。孫行者神通廣大,你看他一條金箍棒,哮吼一聲,嚇過了狼蟲虎豹,剖開路,引師父直上高山。行過嶺頭,下西平處,忽見祥光藹藹,彩霧紛紛,有一所樓臺殿閣,隱隱的鐘磬悠揚。三藏道:「徒弟們,看是個甚麼去處?」行者擡頭,用手搭涼篷,仔細觀看,那壁廂好個所在。真個是:
\begin{quote}
珍樓寶座,上剎名方。谷虛繁地籟,境寂散天香。青松帶雨遮高閣,翠竹留雲護講堂。霞光縹緲龍宮顯,彩色飄颻沙界長。朱欄玉戶,畫棟雕梁。談經香滿座,語籙月當窗。鳥啼丹樹內,鶴飲石泉傍。四圍花發琪園秀,三面門開舍衛光。樓臺突兀門迎嶂,鐘磬虛徐聲韻長。窗開風細,簾捲煙茫。有僧情散淡,無俗意和昌。紅塵不到真仙境,靜土招提好道場。
\end{quote}

行者看罷,回覆道:「師父,那去處便是座寺院,卻不知禪光瑞藹之中,又有些凶氣何也。觀此景象,也似雷音,卻又路道差池。我們到那廂,決不可擅入,恐遭毒手。」唐僧道:「既有雷音之景,莫不就是靈山?你休誤了我誠心,擔擱了我來意。」行者道:「不是,不是。靈山之路,我也走過幾遍,那是這路途?」八戒道:「縱然不是,也必有個好人居住。」沙僧道:「不必多疑,此條路未免從那門首過,是不是一見可知也。」行者道:「悟淨說得有理。」

那長老策馬加鞭,至山門前,見「雷音寺」三個大字,慌得滾下馬來,倒在地下,口裡罵道:「潑猢猻!害殺我也。現是雷音寺,還哄我哩。」行者陪笑道:「師父莫惱,你再看看。山門上乃四個字,你怎麼只念出三個來,倒還怪我?」長老戰兢兢的爬起來再看,真個是四個字,乃「小雷音寺」。三藏道:「就是小雷音寺,必定也有個佛祖在內。經上言三千諸佛,想是不在一方:似觀音在南海,普賢在峨眉,文殊在五臺。這不知是那一位佛祖的道場。古人云:『有佛有經,無方無寶。』我們可進去來。」行者道:「不可進去,此處少吉多凶。若有禍患,你莫怪我。」三藏道:「就是無佛,也必有個佛像。我弟子心願,遇佛拜佛,如何怪你?」

即命八戒取袈裟,換僧帽,結束了衣冠,舉步前進。只聽得山門裡有人叫道:「唐僧,你自東土來拜見我佛,怎麼還這等怠慢?」三藏聞言,即便下拜;八戒也磕頭,沙僧也跪倒。惟大聖牽馬,收拾行李在後。方入到二層門內,就見如來大殿。殿門外寶臺之下,擺列著五百羅漢、三千揭諦、四金剛、八菩薩、比丘尼、優婆塞,無數的聖僧、道者。真個也香花豔麗,瑞氣繽紛。慌得那長老與八戒、沙僧一步一拜,拜上靈臺之間。行者公然不拜。又聞得蓮臺座上厲聲高叫道:「那孫悟空,見如來怎麼不拜?」不知行者又仔細觀看,見得是假,遂丟了馬匹、行囊,掣棒在手,喝道:「你這夥孽畜,十分膽大,怎麼假倚佛名,敗壞如來清德?不要走。」雙手掄棒,上前便打。只聽得半空中叮噹一聲,撇下一副金鐃,把行者連頭帶足,合在金鐃之內。慌得個豬八戒、沙和尚連忙使起鈀杖,就被些阿羅、揭諦、聖僧、道者一擁近前圍繞,他兩個措手不及,盡被拿了。將三藏捉住。一齊都繩纏索綁,緊縛牢拴。

原來那蓮花座上裝佛祖者乃是個妖王,眾阿羅等都是些小怪。遂收了佛祖體像,依然現出妖身。將三眾擡入後邊收藏。把行者合在金鐃之中,永不開放,只擱在寶臺之上,限三晝夜化為膿血。化後,才將鐵籠蒸他三個受用。這正是:
\begin{quote}
碧眼猢兒識假真,禪機見像拜金身。
黃婆盲目同參禮,木母痴心共話論。
邪怪生強欺本性,魔頭懷惡詐天人。
誠為道小魔頭大,錯入傍門枉費身。
\end{quote}

那時群妖將唐僧三眾收藏在後;把馬拴在後邊;把他的袈裟、僧帽安在行李擔內,亦收藏了。一壁廂嚴緊不題。

卻說行者合在金鐃裡,黑洞洞的,燥得滿身流汗,左拱右撞,不能得出。急得他使鐵棒亂打,莫想得動分毫。他心裡沒了算計,將身往外一掙,卻要掙破那金鐃。遂捻著一個訣,就長有千百丈高;那金鐃也隨他身長,全無一些瑕縫光明。卻又捻訣把身子往下一小,小如芥菜子兒;那鐃也就隨身小了,更無些些孔竅。他又把鐵棒,吹口仙氣,叫:「變!」即變做旛竿一樣,撐住金鐃。他卻把腦後毫毛,選長的拔下兩根,叫:「變!」即變做梅花頭五瓣鑽兒,挨著棒下,鑽有千百下,只鑽得蒼蒼響喨,再不鑽動一些。行者急了,卻捻個訣,念一聲「唵㘕靜法界,乾元亨利貞」的咒語,拘得那五方揭諦、六丁六甲、一十八位護教伽藍,都在金鐃之外道:「大聖,我等俱保護著師父,不教妖魔傷害,你又拘喚我等做甚?」行者道:「我那師父不聽我勸解,就弄死他也不虧。但只你等怎麼快作法將這鐃鈸掀開,放我出來,再作處治。這裡面不通光亮,滿身暴燥,卻不悶殺我也?」眾神真個掀鐃,就如長就的一般,莫想揭動分毫。金頭揭諦道:「大聖,這鐃鈸不知是件甚麼寶貝,連上帶下,合成一塊。小神力薄,不能掀動。」行者道:「我在裡面,不知使了多少神通,也不得動。」

揭諦聞言,即著六丁神保護著唐僧,六甲神看守著金鐃,眾伽藍前後照察。他卻縱起祥光,須臾間,闖入南天門裡。不待宣召,直上靈霄寶殿之下,見玉帝,俯伏啟奏道:「主公,臣乃五方揭諦使。今有齊天大聖保唐僧取經,路遇一山,名小雷音寺。唐僧錯認靈山進拜,原來是妖魔假設,困陷他師徒,將大聖合在一副金鐃之內,進退無門,看看至死,特來啟奏。」即傳旨:「差二十八宿星辰,快去釋厄降妖。」

那星宿不敢少緩,隨同揭諦,出了天門,至山門之內,有二更時分。那些大小妖精,因獲了唐僧,老妖俱犒賞了,各去睡覺。眾星宿更不驚張,都到鐃鈸之外,報道:「大聖,我等是玉帝差來二十八宿,到此救你。」行者聽說大喜,便教:「動兵器打破,老孫就出來了。」眾星宿道:「不敢打。此物乃渾金之寶,打著必響,響時驚動妖魔,卻難救拔。等我們用兵器捎他。你那裡但見有一些光處就走。」行者道:「正是。」你看他們使槍的使槍,使劍的使劍,使刀的使刀,使斧的使斧;扛的扛,擡的擡,掀的掀,捎的捎。弄到有三更天氣,漠然不動,就是鑄成了囫圇的一般。那行者在裡邊東張張,西望望,爬過來,滾過去,莫想看見一些光亮。

亢金龍道:「大聖啊,且休焦躁。觀此寶定是個如意之物,斷然也能變化。你在那裡面,於那合縫之處,用手摸著,等我使角尖兒拱進來,你可變化了,順鬆處脫身。」行者依言,真個在裡面亂摸。這星宿把身變小了,那角尖兒就似個針尖一樣,順著鈸合縫口上伸將進去。可憐用盡千斤之力,方能穿透裡面。卻將本身與角使法像,叫:「長!長!長!」角就長有碗來粗細。那鈸口倒也不像金鑄的,好似皮肉長成的,順著亢金龍的角,緊緊噙住,四下裡更無一絲拔縫。行者摸著他的角,叫道:「不濟事,上下沒有一毫鬆處。沒奈何,你忍著些兒疼,帶我出去。」好大聖,即將金箍棒變作一把鋼鑽兒,將他那角尖上鑽了一個孔竅,把身子變得似個芥菜子兒,拱在那鑽眼裡蹲著,叫:「扯出角去,扯出角去。」這星宿又不知費了多少力,方才拔出,使得力盡觔柔,倒在地下。

行者卻自他角尖鑽眼裡鑽出,現了原身,掣出鐵棒,照鐃鈸噹的一聲打去,就如崩倒銅山,咋開金礦。可惜把個佛門之器,打做個千百塊散碎之金。諕得那二十八宿驚張,五方揭諦髮豎,大小群妖皆夢醒。老妖王睡裡慌張,急起來,披衣擂鼓,聚點群妖,各執器械。此時天將黎明。一擁趕到寶臺之下,只見孫行者與列宿圍在碎破金鐃之外,大驚失色。即令:「小的們!緊關了前門,不要放出人去。」

行者聽說,即攜星眾,駕雲跳在九霄空裡。那妖王收了碎金,排開妖卒,列在山門外。妖王懷恨,沒奈何披掛了,使一根短軟狼牙棒,出營高叫:「孫行者,好男子不可遠走高飛,快向前與我交戰三合。」行者忍不住,即引星眾,按落雲頭,觀看那妖精怎生模樣。但見他:
\begin{quote}
蓬著頭,勒一條扁薄金箍;光著眼,簇兩道黃眉的豎。懸膽鼻,孔竅開查;四方口,牙齒尖利。穿一副叩結連環鎧,勒一條生絲攢穗絛。腳踏烏喇鞋一對,手執狼牙棒一根。此形似獸不如獸,相貌非人卻似人。
\end{quote}

行者挺著鐵棒喝道:「你是個甚麼怪物,擅敢假裝佛祖,侵占山頭,虛設小雷音寺?」那妖王道:「這猴兒是也不知我的姓名,故來冒犯仙山。此處喚做小西天。因我修行,得了正果,天賜與我的寶閣珍樓。我名乃是黃眉老佛。這裡人不知,但稱我為黃眉大王、黃眉爺爺。一向久知你往西去,有些手段,故此設像顯能,誘你師父進來,要和你打個賭賽。如若鬥得過我,饒你師徒,讓汝等成個正果;如若不能,將汝等打死,等我去見如來取經,果正中華也。」行者笑道:「妖精,不必海口,既要賭,快上來領棒。」那妖王喜孜孜,使狼牙棒抵住。這一場好殺:
\begin{quote}
兩條棒,不一樣,說將起來有形狀:一條短軟佛家兵,一條堅硬藏海藏。都有隨心變化功,今番相遇爭強壯。短軟狼牙雜錦妝,堅硬金箍蛟龍像。若粗若細實可誇,要短要長甚停當。猴與魔,齊打仗,這場真個無虛誑。馴猴秉教作心猿,潑怪欺天弄假像。嗔嗔恨恨各無情,惡惡兇兇都有樣。那一個當頭手起不放鬆,這一個架丟劈面難推讓。噴雲照日昏,吐霧遮峰嶂。棒來棒去兩相迎,忘生忘死因三藏。
\end{quote}

看他兩個鬥經五十回合,不見輸贏。那山門口鳴鑼擂鼓,眾妖精吶喊搖旗。這壁廂有二十八宿天兵共五方揭諦眾聖,各掮器械,吆喝一聲,把那魔頭圍在中間,嚇得那山門外群妖難擂鼓,戰兢兢手軟不敲鑼。老妖魔公然不懼,一隻手使狼牙棒,架著眾兵;一隻手去腰間解下一條舊白布搭包兒,往上一拋,滑的一聲響喨,把孫大聖、二十八宿與五方揭諦,一搭包兒通裝將去,挎在肩上,拽步回身。眾小妖個個歡然得勝而回。老妖教小的們取了三五十條麻索,解開搭包,拿一個,綑一個。一個個都骨軟觔麻,皮膚窊皺。綑了擡去後邊,不分好歹,俱擲之於地。妖王又命排筵暢飲,自旦至暮方散,各歸寢處不題。

卻說孫大聖與眾神綑至夜半,忽聞有悲泣之聲。側耳聽時,卻原來是三藏聲音,哭道:「悟空啊,我:
\begin{quote}
自恨當時不聽伊,致令今日受災危。
金鐃之內傷了你,麻繩綑我有誰知。
四人遭逢緣命苦,三千功行盡傾頹。
何由解得迍邅難,坦蕩西方去復歸?
\end{quote}

行者聽言,暗自憐憫道:「那師父雖是未聽吾言,今遭此害,然於患難之中,還有憶念老孫之意。趁此夜靜妖眠,無人防備,且去解脫眾等逃生也。」

好大聖,使了個遁身法,將身一小,脫下繩來,走近唐僧身邊,叫聲:「師父。」長老認得聲音,叫道:「你為何到此?」行者悄悄的把前項事告訴了一遍。長老甚喜道:「徒弟,快救我一救。向後事,但憑你處,再不強了。」行者才動手,先解了師父,放了八戒、沙僧。又將二十八宿、五方揭諦,個個解了。又牽過馬來,教快先走出去。方出門,卻不知行李在何處,又來找尋。亢金龍道:「你好重物輕人。既救了你師父就夠了,又還尋甚行李?」行者道:「人固要緊,衣缽尤要緊。包袱中有通關文牒、錦襴袈裟、紫金缽盂,俱是佛門至寶,如何不要?」八戒道:「哥哥,你去找尋,我等先去路上等你。」你看那星眾簇擁著唐僧,使個攝法,共弄神通,一陣風,撮出垣圍,奔大路,下了山坡,卻屯於平處等候。

約有三更時分,孫大聖輕那慢步,走入裡面,原來一層層門戶甚緊。他就爬上高樓看時,窗牖皆關。欲要下去,又恐怕窗櫺兒響,不敢推動。捻著訣,搖身一變,變做一個仙鼠,俗名蝙蝠。你道他怎生模樣:
\begin{quote}
頭尖還似鼠,眼亮亦如之。
有翅黃昏出,無光白晝居。
藏身穿瓦穴,覓食撲蚊兒。
偏喜晴明月,飛騰最識時。
\end{quote}

他順著不封瓦口椽子之下,鑽將進去,越門過戶,到了中間看時,只見那第三重樓窗之下,閃灼灼一道毫光,也不是燈燭之光、螢火之光,又不是飛霞之光、掣電之光。他半飛半跳,近於窗前看時,卻是包袱放光。那妖精把唐僧的袈裟脫了,不曾摺,就亂亂的揌在包袱之內。那袈裟本是佛寶,上邊有如意珠、摩尼珠、紅瑪瑙、紫珊瑚、舍利子、夜明珠,所以透的光彩。他見了此衣缽,心中一喜,就現了本像,拿將過來,也不管擔繩偏正,擡上肩,往下就走。不期脫了一頭,撲的落在樓板上,唿喇的一聲響。噫!有這般事:可可的老妖精在樓下睡覺,一聲響,把他驚醒,跳起來,亂叫道:「有人了,有人了!」那些大小妖都起來,點燈打火,一齊吆喝,前後去看。有的來報道:「唐僧走了。」又有的來報道:「行者眾人俱走了。」老妖急傳號令,教:「各門上謹慎。」行者聽言,恐又遭他羅網,挑不成包袱,縱觔斗,就跳出樓窗外走了。

那妖精前前後後尋不著唐僧等,又見天色將明,取了棒,帥眾來趕,只見那二十八宿與五方揭諦等神雲霧騰騰,屯住山坡之下。妖王喝了一聲:「那裡去?吾來也。」角木蛟急喚:「兄弟們,怪物來了。」亢金龍、氐土蝠、房日兔、心月狐、尾火虎、箕水豹、斗木獬、牛金牛、女土貉、虛日鼠、危月燕、室火豬、壁水㺄、奎木狼、婁金狗、胃土彘、昴日雞、畢月烏、觜火猴、參水猿、井木犴、鬼金羊、柳土獐、星日馬、張月鹿、翼火蛇、軫水蚓,領著金頭揭諦、銀頭揭諦、六甲六丁等神、護教伽藍,同八戒、沙僧,(不領唐三藏,丟了白龍馬)各執兵器,一擁而上。這妖王見了,呵呵冷笑,叫一聲哨子,有四五千大小妖精,一個個威強力勝,渾戰在西山坡上。好殺:
\begin{quote}
魔頭潑惡欺真性,真性溫柔怎奈魔。百計施為難脫苦,千方妙用不能和。諸天來擁護,眾聖助干戈。留情虧木母,定志感黃婆。渾戰驚天並振地,強爭設網與張羅。那壁廂搖旗吶喊,這壁廂擂鼓篩鑼。槍刀密密寒光蕩,劍戟紛紛殺氣多。妖卒兇還勇,神兵怎奈何。愁雲遮日月,慘霧罩山河。苦掤苦拽來相戰,皆因三藏拜彌陀。
\end{quote}

那妖精倍加勇猛,帥眾上前掩殺。

正在那不分勝敗之際,只聞得行者叱咤一聲道:「老孫來了。」八戒迎著道:「行李如何?」行者道:「老孫的性命幾乎難免,卻便說甚麼行李!」沙僧執著寶杖道:「且休敘話,快去打妖精也。」那星宿、揭諦、丁甲等神,被群妖圍在垓心渾殺,老妖使棒來打他三個。這行者、八戒、沙僧丟開棍杖,掄著釘鈀抵住。真個是地暗天昏,不能取勝。只殺得太陽星西沒山根,太陰星東生海嶠。那妖見天晚,打個哨子,教群妖各各留心,他卻取出寶貝。孫行者看得分明:那怪解下搭包,拿在手中。行者道聲:「不好了,走啊!」他就顧不得八戒、沙僧、諸天等眾,一路觔斗,跳上九霄空裡。眾神、八戒、沙僧不解其意,被他拋起去,又都裝在裡面,只是行者走了。那妖王收兵回寺,又教取出繩索,照舊綁了。將唐僧、八戒、沙僧懸梁高吊,白馬拴在後邊,諸神亦俱綁縛,擡在地窖子內,封鎖了蓋。那眾妖遵依,一一收了不題。

卻說孫行者跳在九霄,全了性命。見妖兵回轉,不張旗號,已知眾等遭擒。他卻按下祥光,落在那東山頂上,咬牙恨怪物,滴淚想唐僧,仰面朝天望,悲嗟忽失聲。叫道:「師父啊,你是那世裡造下這迍邅難,今世裡步步遇妖精?似這般苦楚難逃,怎生是好?」獨自一個,嗟嘆多時,復又寧神思慮,以心問心道:「這妖魔不知是個甚麼搭包子,那般裝得許多物件?如今將天神、天將,許多人又都裝進去了。我待求救於天,奈恐玉帝見怪。我記得有個北方真武,號曰蕩魔天尊,他如今現在南贍部洲武當山上,等我去請他來搭救師父一難。」正是:
\begin{quote}
仙道未成猿馬散,心神無主五行枯。
\end{quote}

畢竟不知此去端的如何,且聽下回分解。
