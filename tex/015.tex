
\chapter{蛇盤山諸神暗佑 鷹愁澗意馬收韁}

卻說行者伏侍唐僧西進,行經數日,正是那臘月寒天,朔風凜凜,滑凍凌凌。走的是些懸崖峭壁崎嶇路,疊嶺層巒險峻山。三藏在馬上,遙聞唿喇喇水聲聒耳,回頭叫:「悟空,是那裡水響?」行者道:「我記得此處叫做蛇盤山鷹愁澗,想必是澗裡水響。」說不了,馬到澗邊,三藏勒韁觀看。但見:
\begin{quote}
涓涓寒脈穿雲過,湛湛清波映日紅。
聲搖夜雨聞幽谷,彩發朝霞眩太空。
千仞浪飛噴碎玉,一泓水響吼清風。
流歸萬頃煙波去,鷗鷺相忘沒釣逢。
\end{quote}

師徒兩個正然看處,只見那澗當中響一聲,鑽出一條龍來,推波掀浪,攛出崖山,就搶長老。慌得個行者丟了行李,把師父抱下馬來,回頭便走。那條龍就趕不上,把他的白馬連鞍轡一口吞下肚去,依然伏水潛蹤。行者把師父送在那高阜上坐了,卻來牽馬挑擔,止存得一擔行李,不見了馬匹。他將行李擔送到師父面前道:「師父,那孽龍也不見蹤影,只是驚走我的馬了。」三藏道:「徒弟啊,卻怎生尋得馬著麼?」行者道:「放心,放心,等我去看來。」

他打個唿哨,跳在空中,火眼金睛,用手搭涼篷,四下裡觀看,更不見馬的蹤跡。按落雲頭,報道:「師父,我們的馬斷乎O那龍吃了,四下裡再看不見。」三藏道:「徒弟呀,那廝能有多大口,卻將那匹大馬連鞍轡都吃了?想是驚張溜韁,走在那山凹之中。你再仔細看看。」行者道:「你也不知我的本事。我這雙眼,白日裡常看一千里路的吉凶。像那千里之內,蜻蜓兒展翅,我也看見,何期那匹大馬,我就不見?」三藏道:「既是他吃了,我如何前進?可憐啊,這千山萬水,怎生走得?」說著話,淚如雨落。行者見他哭將起來,他那裡忍得住暴燥,發聲喊道:「師父莫要這等膿包形麼,你坐著,坐著,等老孫去尋著那廝,教他還我馬匹便了。」三藏卻才扯住道:「徒弟啊,你那裡去尋他?只怕他暗地裡攛將出來,卻不又連我都害了?那時節人馬兩亡,怎生是好?」行者聞得這話,越加嗔怒,就叫喊如雷道:「你忒不濟,不濟!又要馬騎,又不放我去,似這般看著行李,坐到老罷。」

哏哏的吆喝,正難息怒,只聽得空中有人言語,叫道:「孫大聖莫惱,唐御弟休哭。我等是觀音菩薩差來的一路神祇,特來暗中保取經者。」那長老聞言,慌忙禮拜。行者道:「你等是那幾個,可報名來,我好點卯。」眾神道:「我等是六丁六甲、五方揭諦、四值功曹、一十八位護教伽藍,各各輪流值日聽候。」行者道:「今日先從誰起?」眾揭諦道:「丁甲、功曹、伽藍輪次。我五方揭諦,惟金頭揭諦晝夜不離左右。」行者道:「既如此,不當值者且退,留下六丁神將與日值功曹和眾揭諦保守著我師父。等老孫尋那澗中的孽龍,教他還我馬來。」眾神遵令。三藏才放下心,坐在石崖之上,吩咐行者仔細。行者道:「只管寬心。」好猴王,束一束綿布直裰,撩起虎皮裙子,揝著金箍鐵棒,抖擻精神,徑臨澗壑,半雲半霧的,在那水面上高叫道:「潑泥鰍,還我馬來!還我馬來!」

卻說那龍吃了三藏的白馬,伏在那澗底中間,潛靈養性,只聽得有人叫罵索馬。他按不住心中火發,急縱身躍浪翻波,跳將上來道:「是那個敢在這裡海口傷吾?」行者見了他,大咤一聲「休走,還我馬來!」掄著棍,劈頭就打。那條龍張牙舞爪來抓。他兩個在澗邊前這一場賭鬥,果是驍雄。但見那:
\begin{quote}
龍舒利爪,猴舉金箍。那個鬚垂白玉線,這個眼幌赤金燈。那個鬚下明珠噴彩霧,這個手中鐵棒舞狂風。那個是迷爺娘的業子,這個是欺天將的妖精。他兩個都因有難遭磨折,今要成功各顯能。
\end{quote}

來來往往,戰夠多時,盤旋良久,那條龍力軟筋麻,不能抵敵,打一個轉身,又攛於水內,深潛澗底,再不出頭。被猴王罵詈不絕,他也只推耳聾。

行者沒及奈何,只得回見三藏道:「師父,這個怪被老孫罵將出來,他與我賭鬥多時,怯戰而走,只躲在水中間,再不出來了。」三藏道:「不知端的可是他吃了我馬?」行者道:「你看你說的話,不是他吃了,他還肯出來招聲,與老孫犯對?」三藏道:「你前日打虎時,曾說有降龍伏虎的手段,今日如何便不能降他?」原來那猴子吃不得人急他。見三藏搶白了他這一句,他就發起神威道:「不要說,不要說,等我與他再見個上下。」

這猴王拽開步,跳到澗邊,使出那翻江攪海的神通,把一條鷹愁陡澗徹底澄清的水,攪得似那九曲黃河泛漲的波。那孽龍在於深澗中坐臥不寧,心中思想道:「這才是福無雙降,禍不單行。我才脫了天條死難,不上一年,在此隨緣度日,又撞著這般個潑魔,他來害我。」你看他越思越惱,受不得屈氣,咬著牙,跳將出去,罵道:「你是那裡來的潑魔,這等欺我?」行者道:「你莫管我那裡不那裡,你只還了馬,我就饒你性命。」那龍道:「你的馬是我吞下肚去,如何吐得出來?不還你,便待怎的?」行者道「不還馬時看棍,只打殺你,償了我馬的性命便罷。」他兩個又在那山崖下苦鬥。鬥不數合,小龍委實難搪,將身一幌,變作一條水蛇兒,鑽入草科中去了。

猴王拿著棍,趕上前來,撥草尋蛇,那裡得些影響。急得他三尸神咋,七竅煙生,念了一聲「唵」字咒語,即喚出當坊土地、本處山神,一齊來跪下道:「山神、土地來見。」行者道:「伸過孤拐來,各打五棍見面,與老孫散散心。」二神叩頭哀告道:「望大聖方便,容小神訴告。」行者道:「你說甚麼?」二神道:「大聖一向久困,小神不知幾時出來,所以不曾接得,萬望恕罪。」行者道:「既如此,我且不打你。我問你:鷹愁澗裡,是那方來的怪龍?他怎麼搶了我師父的白馬吃了?」二神道:「大聖自來不曾有師父,原來是個不伏天不伏地混元上真,如何得有甚麼師父的馬來?」行者道:「你等是也不知。我只為那誑上的勾當,整受了這五百年的苦難。今蒙觀音菩薩勸善,著唐朝駕下真僧救出我來,教我跟他做徒弟,往西天去拜佛求經。因路過此處,失了我師父的白馬。」二神道:「原來是如此。這澗中自來無邪,只是深陡寬闊,水光徹底澄清,鴉鵲不敢飛過;因水清照見自己的形影,便認做同群之鳥,往往身擲於水內:故名『鷹愁陡澗』。只是向年間,觀音菩薩因為尋訪取經人去,救了一條玉龍,送他在此,教他等候那取經人,不許為非作歹。他只是饑了時,上岸來撲些鳥鵲吃,或是捉些獐鹿食用。不知他怎麼無知,今日衝撞了大聖。」行者道:「先一次,他還與老孫侮手,盤旋了幾合;後一次,是老孫叫罵,他再不出。因此使了一個翻江攪海的法兒,攪混了他澗水,他就攛將上來,還要爭持。不知老孫的棍重,他遮架不住,就變做一條水蛇,鑽在草裡。我趕來尋他,卻無蹤跡。」土地道:「大聖不知。這條澗千萬個孔竅相通,故此這波瀾深遠。想是此間也有一孔,他鑽將下去。也不須大聖發怒,在此找尋;要擒此物,只消請將觀世音來,自然伏了。」

行者見說,喚山神、土地,同來見了三藏,具言前事。三藏道:「若要去請菩薩,幾時才得回來?我貧僧饑寒怎忍?」說不了,只聽得暗空中有金頭揭諦叫道:「大聖,你不須動身,小神去請菩薩來也。」行者大喜,道聲:「有累,有累。快行,快行。」那揭諦急縱雲頭,徑上南海。行者吩咐山神、土地守護師父,日值功曹去尋齋供,他又去澗邊巡遶不題。

卻說金頭揭諦一駕雲,早到了南海。按祥光,直至落伽山紫竹林中,託那金甲諸天與木叉惠岸轉達,得見菩薩。菩薩道:「汝來何幹?」揭諦道:「唐僧在蛇盤山鷹愁陡澗失了馬,急得孫大聖進退兩難。及問本處土神,說是菩薩送在澗裡的孽龍吞了。那大聖著小神來告請菩薩降這孽龍,還他馬匹。」菩薩聞言道:「這廝本是西海敖閏之子,他為縱火燒了殿上明珠,他父告他忤逆,天庭上犯了死罪。是我親見玉帝,討他下來,教他與唐僧做個腳力。他怎麼反吃了唐僧的馬?這等說,等我去來。」那菩薩降蓮臺,徑離仙洞,與揭諦駕著祥光,過了南海而來。有詩為證。詩曰:
\begin{quote}
佛說蜜多三藏經,菩薩揚善滿長城。
摩訶妙語通天地,般若真言救鬼靈。
致使金蟬重脫殼,故令玄奘再修行。
只因路阻鷹愁澗,龍子歸真化馬形。
\end{quote}

那菩薩與揭諦不多時到了蛇盤山,卻在那半空裡留住祥雲,低頭觀看,只見孫行者正在澗邊叫罵。菩薩著揭諦喚他來。那揭諦按落雲頭,不經由三藏,直至澗邊,對行者道:「菩薩來也。」行者聞得,急縱雲跳到空中,對他大叫道:「你這個七佛之師,慈悲的教主,你怎麼生方法兒害我?」菩薩道:「我把你這個大膽的馬流,村愚的赤尻。我倒再三盡意,度得個取經人來,叮嚀教他救你性命,你怎麼不來謝我活命之恩,反來與我嚷鬧?」行者道:「你弄得我好哩。你既放我出來,讓我逍遙自在耍子便了。你前日在海上迎著我,傷了我幾句,教我來盡心竭力,伏侍唐僧便罷了,你怎麼送他一頂花帽,哄我戴在頭上受苦?把這個箍子長在老孫頭上,又教他念一卷甚麼『緊箍兒咒』,著那老和尚念了又念,教我這頭上疼了又疼,這不是你害我也?」菩薩笑道:「你這猴子,你不遵教令,不受正果,若不如此拘係你,你又誑上欺天,知甚好歹?再似從前撞出禍來,有誰收管?須是得這個魔頭,你才肯入我瑜伽之門路哩。」行者道:「這樁事,作做是我的魔頭罷。你怎麼又把那有罪的孽龍,送在此處成精,教他吃了我師父的馬匹?此又是縱放歹人為惡,太不善也。」菩薩道:「那條龍,是我親奏玉帝,討他在此,專為求經人做個腳力。你想那東土來的凡馬,怎歷得這萬水千山?怎到得那靈山佛地?須是得這個龍馬,方才去得。」行者道:「像他這般懼怕老孫,潛躲不出,如之奈何?」菩薩叫揭諦道:「你去澗邊叫一聲『敖閏龍王玉龍三太子,你出來,有南海菩薩在此。』他就出來了。」

那揭諦果去澗邊叫了兩遍。那小龍翻波跳浪,跳出水來,變作一個人像,踏了雲頭,到空中對菩薩禮拜道:「向蒙菩薩解脫活命之恩,在此久等,更不聞取經人的音信。」菩薩指著行者道:「這不是取經人的大徒弟?」小龍見了道:「菩薩,這是我的對頭。我昨日腹中饑餒,果然吃了他的馬匹。他倚著有些力量,將我鬥得力怯而回,又罵得我閉門不敢出來。他更不曾提著一個『取經』的字樣。」行者道:「你又不曾問我姓甚名誰,我怎麼就說?」小龍道:「我不曾問你是那裡來的潑魔?你嚷道:『管甚麼那裡不那裡,只還我馬來。』何曾說出半個『唐』字?」菩薩道:「那猴頭專倚自強,那肯稱讚別人?今番前去,還有歸順的哩。若問時,先提起『取經』的字來,卻也不用勞心,自然拱伏。」

行者歡喜領教。菩薩上前,把那小龍的項下明珠摘了,將楊柳枝蘸出甘露,往他身上拂了一拂,吹口仙氣,喝聲叫:「變!」那龍即變做他原來的馬匹毛片。又將言語吩咐道:「你須用心還了業障,功成後超越凡龍,還你個金身正果。」那小龍口啣著橫骨,心心領諾。菩薩教悟空領他去見三藏。「我回海上去也。」行者扯住菩薩不放道:「我不去了,我不去了。西方路這等崎嶇,保這個凡僧,幾時得到?似這等多磨多折,老孫的性命也難全,如何成得甚麼功果?我不去了,我不去了。」菩薩道:「你當年未成人道,且肯盡心修悟;你今日脫了天災,怎麼倒生懶惰?我門中以寂滅成真,須是要信心正果。假若到了那傷身苦磨之處,我許你叫天天應,叫地地靈;十分再到那難脫之際,我也親來救你。你過來,我再贈你一般本事。」菩薩將楊柳葉兒摘下三個,放在行者的腦後,喝聲:「變!」即變做三根救命的毫毛。教他:「若到那無濟無主的時節,可以隨機應變,救得你急苦之災。」行者聞了這許多好言,才謝了大慈大悲的菩薩。那菩薩香風繞繞,彩霧飄飄,徑轉普陀而去。

這行者才按落雲頭,揪著那龍馬的頂鬃,來見三藏道:「師父,馬有了也。」三藏一見,大喜道:「徒弟,這馬怎麼比前反肥盛了些?在何處尋著的?」行者道:「師父,你還做夢哩。卻才是金頭揭諦請了菩薩來,把那澗裡龍化作我們的白馬,其毛片相同,只是少了鞍轡。著老孫揪將來也。」三藏大驚道:「菩薩何在?待我去拜謝他。」行者道:「菩薩此時已到南海,不耐煩矣。」三藏就撮土焚香,望南禮拜。拜罷,起身即與行者收拾前進。行者喝退了山神、土地,吩咐了揭諦、功曹,卻請師父上馬。三藏道:「那無鞍轡的馬,怎生騎得?且待尋船渡過澗去,再作區處。」行者道:「這個師父好不知時務!這個曠野山中,船從何來?這匹馬,他在此久住,必知水勢,就騎著他做個船兒過去罷。」

三藏無奈,只得依言,跨了產馬。行者挑著行囊。到了澗邊。只見那上流頭,有一個漁翁,撐著一個枯木的栰子,順流而下。行者見了,用手招呼道:「那老漁,你來,你來。我是東土取經去的,我師父到此難過,你來渡他一渡。」漁翁聞言,即忙撐攏。行者請師父下了馬,扶持左右。三藏上了栰子,揪上馬匹,安了行李。那老漁撐開栰子,如風似箭,不覺的過了鷹愁陡澗,上了西岸。三藏教行者解開包袱,取出大唐的幾文錢鈔,送與老漁。老漁把栰子一篙撐開道:「不要錢,不要錢。」向中流渺渺茫茫而去。三藏甚不過意,只管合掌稱謝。行者道:「師父休致意了,你不認得他?他是此澗裡的水神。不曾來接得我老孫,老孫還要打他哩。只如今免打就夠了他的,怎敢要錢!」那師父也似信不信,只得又跨著產馬,隨著行者,徑投大路,奔西而去。這正是:廣大真如登彼岸,誠心了性上靈山。

同師前進,不覺的紅日沉西,天光漸晚。但見:
\begin{quote}
淡雲撩亂,山月昏蒙。滿天霜色生寒,四面風聲透體。孤鳥去時蒼渚闊,落霞明處遠山低。疏林千樹吼,空嶺獨猿啼。長途不見行人跡,萬里歸舟入夜時。
\end{quote}

三藏在馬上遙觀,忽見路傍一座莊院。三藏道:「悟空,前面人家,可以借宿,明早再行。」行者擡頭看見道:「師父,不是人家莊院。」三藏道:「如何不是?」行者道:「人家莊院,卻沒飛魚穩獸之脊,這斷是個廟宇庵院。」

師徒們說著話,早已到了門首。三藏下了馬,只見那門上有三個大字,乃「里社祠」,遂入門裡。那裡邊有一個老者,項掛著數珠兒,合掌來迎,叫聲:「師父請坐。」三藏慌忙答禮,上殿去參拜了聖像。那老者即呼童子獻茶。茶罷,三藏問老者道:「此廟何為『里社』?」老者道:「敝處乃西番哈咇國界。這廟後有一莊人家,共發虔心,立此廟宇。里者,乃一鄉里地;社者,乃一社土神。每遇春耕、夏耘、秋收、冬藏之日,各辦三牲花果,來此祭社,以保四時清吉、五穀豐登、六畜茂盛故也。」三藏聞言,點頭誇讚:「正是『離家三里遠,別是一鄉風』。我那裡人家,更無此善。」老者卻問:「師父仙鄉是何處?」三藏道:「貧僧是東土大唐國,奉旨意,上西天拜佛求經的。路過寶坊,天色將晚,特投聖祠,告宿一宵,天光即行。」那老者十分歡喜,道了幾聲「失迎」,又叫童子辦飯。三藏吃畢,謝了。

行者的眼乖,見他房簷下有一條搭衣的繩子,走將去,一把扯斷,將馬腳繫住。那老者笑道:「這馬是那裡偷來的?」行者怒道:「你那老頭子,說話不知高低。我們是拜佛的聖僧,又會偷馬?」老兒笑道:「不是偷的,如何沒有鞍轡韁繩,卻來扯斷我晒衣的索子?」三藏陪禮道:「這個頑皮,只是性燥。——你要拴馬,好生問老人家討條繩子,如何就扯斷他的衣索?——老先生,休怪,休怪。我這馬,實不瞞你說,不是偷的。昨日東來,至鷹愁陡澗,原有騎的一匹白馬,鞍轡俱全。不期那澗裡有條孽龍,在彼成精,他把我的馬連鞍轡一口吞之。幸虧我徒弟有些本事,又感得觀音菩薩來澗邊擒住那龍,教他就變做我原騎的白馬,毛片俱同,馱我上西天拜佛。今此過澗,未經一日,卻到了老先的聖祠,還不曾置得鞍轡哩。」那老者道:「師父休怪,我老漢作笑耍子,誰知你高徒認真。我小時也有幾個村錢,也好騎匹駿馬。只因累歲迍邅,遭喪失火,到此沒了下梢,故充為廟祝,侍奉香火。幸虧這後莊施主家募化度日。我那裡倒還有一副鞍轡,是我平日心愛之物,就是這等貧窮,也不曾捨得賣了。才聽老師父之言,菩薩尚且救護神龍,教他化馬馱你,我老漢卻不能少有周濟。明日將那鞍轡取來,願送老師父,扣背前去,乞為笑納。」三藏聞言,稱謝不盡。早又見童子拿出晚齋。齋罷,掌上燈,安了鋪,各各寢歇。

至次早,行者起來道:「師父,那廟祝老兒昨晚許我們鞍轡,問他要,不要饒他。」說未了,只見那老兒果擎著一副鞍轡、襯屜、韁籠之類,凡馬上一切用的,無不全備,放在廊下道:「師父,鞍轡奉上。」三藏見了,歡喜領受。教行者拿了,背上馬看,可相稱否。行者走上前,一件件的取起看了,果然是些好物。有詩為證。詩曰:
\begin{quote}
雕鞍彩晃柬銀星,寶鐙光飛金線明。
襯屜幾層絨苫疊,牽韁三股紫絲繩。
轡頭皮劄團花粲,雲扇描金舞獸形。
環嚼叩成磨煉鐵,兩垂蘸水結毛纓。
\end{quote}

行者心中暗喜,將鞍轡背在馬上,就似量著做的一般。三藏拜謝那老,那老慌忙攙起道:「惶恐,惶恐。何勞致謝?」那老者也不再留,請三藏上馬。那長老出得門來,攀鞍上馬。行者擔著行李。那老兒復袖中取出一條鞭兒來,卻是皮丁兒寸劄的香籐柄子,虎筋絲穿結的梢兒,在路傍拱手奉上道:「聖僧,我還有一條挽手兒,一發送了你罷。」那三藏在馬上接了道:「多承布施,多承布施。」

正打問訊,卻早不見了那老兒。及回看那里社祠,是一片光地。只聽得半空中有人言語道:「聖僧,多簡慢你。我是落伽山山神、土地,蒙菩薩差送鞍轡與汝等的。汝等可努力西行,卻莫一時怠慢。」慌得個三藏滾鞍下馬,望空禮拜道:「弟子肉眼凡胎,不識尊神尊面,望乞恕罪。煩轉達菩薩,深蒙恩佑。」你看他只管朝天磕頭,也不計其數。路傍邊活活的笑倒個孫大聖,孜孜的喜壞個美猴王,上前來扯住唐僧道:「師父,你起來罷,他已去得遠了,聽不見你禱祝,看不見你磕頭,只管拜怎的?」長老道:「徒弟呀,我這等磕頭,你也就不拜他一拜,且立在傍邊,只管哂笑,是何道理?」行者道:「你那裡知道,像他這個藏頭露尾的,本該打他一頓;只為看菩薩面上,饒他打,儘夠了,他還敢受我老孫之拜?老孫自小兒做好漢,不曉得拜人,就是見了玉皇大帝、太上老君,我也只是唱個喏便罷了。」三藏道:「不當人子,莫說這空頭話。快起來,莫誤了走路。」那師父才起來收拾,投西而去。

此去行有兩個月太平之路,相遇的都是些羅羅、回回、狼蟲虎豹。光陰迅速,又值早春時候。但見山林錦翠色,草木發青芽;梅英落盡,柳眼初開。師徒們行玩春光,又見太陽西墜。三藏勒馬遙觀,山凹裡有樓臺影影,殿閣沉沉。三藏道:「悟空,你看那裡是甚麼去處?」行者擡頭看了道:「不是殿宇,定是寺院。我們趕起些,那裡借宿去。」三藏欣然從之,放開龍馬,徑奔前來。

畢竟不知此去是甚麼去處,且聽下回分解。
