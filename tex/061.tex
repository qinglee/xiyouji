
\chapter{豬八戒助力敗魔王 孫行者三調芭蕉扇}

話表牛魔王趕上孫大聖,只見他肩膊上掮著那柄芭蕉扇,怡顏悅色而行。魔王大驚道:「猢猻原來把運用的方法兒也叨餂得來了。我若當面問他索取,他定然不與;倘若搧我一扇,要去十萬八千里遠,卻不遂了他意?我聞得唐僧在那大路上等候。他二徒弟豬精、三徒弟沙流精,我當年做妖怪時,也曾會他。且變作豬精的模樣,返騙他一場。料猢猻以得意為喜,必不詳細隄防。」好魔王,他也有七十二變,武藝也與大聖一般,只是身子狼犺些。欠鑽疾,不活達些;把寶劍藏了,念個咒語,搖身一變,即變作八戒一般嘴臉。抄下路,當面迎著大聖,叫道:「師兄,我來也。」

這大聖果然歡喜。古人云「得勝的貓兒歡似虎」也,只倚著強能,更不察來人的意思。見是個八戒的模樣,便就叫道:「兄弟,你往那裡去?」牛魔王綽著經兒道:「師父見你許久不回,恐牛魔王手段大,你鬥他不過,難得他的寶貝,教我來迎你的。」行者笑道:「不必費心,我已得了手了。」牛王又問道:「你怎麼得的?」行者道:「那老牛與我戰經百十合,不分勝負。他就撇了我,去那亂石山碧波潭底,與一夥蛟精、龍精飲酒。是我暗跟他去,變作個螃蟹,偷了他所騎的辟水金睛獸,變了老牛的模樣,徑至芭蕉洞哄那羅剎女。那女子與老孫結了一場乾夫妻,是老孫設法騙將來的。」牛王道:「卻是生受了。哥哥勞碌太甚,可把扇子我拿。」孫大聖那知真假,也慮不及此,遂將扇子遞與他。

原來那牛王他知那扇子收放的根本,接過手,不知捻個甚麼訣兒,依然小似一片杏葉,現出本像。開言罵道:「潑猢猻!認得我麼?」行者見了,心中自悔道:「是我的不是了。」恨了一聲,跌足高呼道:「咦!逐年家打雁,今卻被小雁兒嗛了眼睛。」狠得他爆躁如雷,掣鐵棒,劈頭便打;那魔王就使扇子搧他一下。不知那大聖先前變蟭蟟蟲入羅剎女腹中之時,將定風丹噙在口裡,不覺的嚥下肚裡,所以五臟皆牢,皮骨皆固,憑他怎麼搧,再也搧他不動。牛王慌了,把寶貝丟入口中,雙手掄劍就砍。那兩個在那半空中這一場好殺:
\begin{quote}
齊天孫大聖,混世潑牛王,只為芭蕉扇,相逢各騁強。粗心大聖將人騙,大膽牛王把扇誆。這一個,金箍棒起無情義;那一個,雙刃青鋒有智量。大聖施威噴彩霧,牛王放潑吐毫光。齊鬥勇,兩不良,咬牙剉齒氣昂昂。播土揚塵天地暗,飛砂走石鬼神藏。這個說:「你敢無知返騙我?」那個說:「我妻許你共相將。」言村語潑,性烈情剛。那個說:「你哄人妻女真該死,告到官司有罪殃。」伶俐的齊天聖,兇頑的大力王,一心只要殺,更不待商量。棒打劍迎齊努力,有些鬆慢見閻王。
\end{quote}

且不說他兩個相鬥難分。卻表唐僧坐在途中,一則火氣蒸人,二來心焦口渴,對火焰山土地道:「敢問尊神,那牛魔王法力如何?」土地道:「那牛王神通不小,法力無邊,正是孫大聖的敵手。」三藏道:「悟空是個會走路的,往常家二千里路,一霎時便回,怎麼如今去了一日?斷是與那牛王賭鬥。」叫:「悟能、悟淨,你兩個,那一個去迎你師兄一迎?倘或遇敵,就當用力相助,求得扇子來,解我煩躁,早早過山,趕路去也。」八戒道:「今日天晚,我想著要去接他,但只是不認得積雷山路。」土地道:「小神認得,且教捲簾將軍與你師父做伴,我與你去來。」三藏大喜道:「有勞尊神,功成再謝。」

那八戒抖擻精神,束一束皂錦直裰,搴著鈀,即與土地縱起雲霧,徑回東方而去。正行時,忽聽得喊殺聲高,狂風滾滾。八戒按住雲頭看時,原來孫行者與牛王廝殺哩。土地道:「天蓬還不上前,待怎的?」獃子掣釘鈀,厲聲高叫道:「師兄,我來也。」行者恨道:「你這夯貨,誤了我多少大事。」八戒道:「師父教我來迎你,因認不得山路,商議良久,教土地引我,故此來遲,如何誤了大事?」行者道:「不是怪你來遲,這潑牛十分無禮。我向羅剎處弄得扇子來,卻被這廝變作你的模樣,口稱迎我,我一時歡悅,轉把扇子遞在他手,他卻現了本像,與老孫在此比併,所以誤了大事也。」八戒聞言大怒,舉釘鈀,當面罵道:「我把你這血皮脹的遭瘟!你怎敢變作你祖宗的模樣,騙我師兄,使我兄弟不睦?」你看他沒頭沒臉的使釘鈀亂築。

那牛王,一則是與行者鬥了一日,力倦神疲;二則是見八戒的釘鈀兇猛,遮架不住:敗陣就走。只見那火焰山土地帥領陰兵,當面擋住道:「大力王,且住手。唐三藏西天取經,無神不保,無天不佑,三界通知,十方擁護。快將芭蕉扇來搧息火焰,教他無災無障,早過山去;不然,上天責你罪愆,定遭誅也。」牛王道:「你這土地,全不察理。那潑猴奪我子,欺我妾,騙我妻,番番無道,我恨不得囫圇吞他下肚,化作大便喂狗,怎麼肯將寶貝借他?」

說不了,八戒趕上罵道:「我把你個結心癀!快拿出扇來,饒你性命。」那牛王只得回頭,使寶劍又戰八戒。孫大聖舉棒相幫,這一場在那裡好殺:
\begin{quote}
成精豕,作怪牛,兼上偷天得道猴。禪性自來能戰煉,必當用土合元由。釘鈀九齒尖還利,寶劍雙鋒快更柔。鐵棒捲舒為主仗,土神助力結丹頭。三家刑剋相爭競,各展雄才要運籌。捉牛耕地金錢長,喚豕歸爐木氣收。心不在焉何作道,神常守舍要拴猴。胡亂嚷,苦相求,三般兵刃響搜搜。鈀築劍傷無好意,金箍棒起有因由。只殺得星不光兮月不皎,一天寒霧黑悠悠。
\end{quote}

那魔王奮勇爭強,且行且鬥,鬥了一夜,不分上下,早又天明。前面是他的積雷山摩雲洞口,他三個與土地、陰兵又諠譁振耳,驚動那玉面公主,喚丫鬟看是那裡人嚷。只見守門小妖來報:「是我家爺爺與昨日那雷公嘴漢子,並一個長嘴大耳的和尚,同火焰山土地等眾廝殺哩。」玉面公主聽言,即命外護的大小頭目,各執槍刀助力。前後點起七長八短,有百十餘口。一個個賣弄精神,拈槍弄棒,齊告:「大王爺爺,我等奉奶奶內旨,特來助力也。」牛王大喜道:「來得好,來得好。」眾妖一齊上前亂砍。八戒措手不及,倒拽著鈀,敗陣而走。大聖縱觔斗雲,跳出重圍。眾陰兵亦四散奔走。老牛得勝,聚群妖歸洞,緊閉了洞門不題。

行者道:「這廝驍勇,自昨日申時前後與老孫戰起,直到今夜,未定輸贏。卻得你兩個來接力。如此苦鬥半日一夜,他更不見勞困。才這一夥小妖,卻又莽壯。他將洞門緊閉不出,如之奈何?」八戒道:「哥哥,你昨日巳時離了師父,怎麼到申時才與他鬥起?你那兩三個時辰在那裡的?」行者道:「別你後,頃刻就到這座山上,見一個女子,問訊,原來就是他愛妾玉面公主。被我使鐵棒諕他一諕,他就跑進洞,叫出那牛王來。與老孫劖言劖語,嚷了一會。又與他交手,鬥了有一個時辰。正打處,有人請他赴宴去了。是我跟他到那亂石山碧波潭底,變作一個螃蟹,探了消息,偷了他辟水金睛獸,假變牛王模樣,復至翠雲山芭蕉洞,騙了羅剎女,哄得他扇子。出門試演試演方法,把扇子弄長了,只是不會收小。正掮了走處,被他假變做你的嘴臉,返騙了去。故此耽擱兩三個時辰也。」

八戒道:「這正是俗語云:『大海裡翻了豆腐船——湯裡來,水裡去。』如今難得他扇子,如何保得師父過山?且回去,轉路走他娘罷。」土地道:「大聖休焦惱,天蓬莫懈怠。但說轉路,就是入了傍門,不成個修行之類。古語云:『行不由徑。』豈可轉走?你那師父在正路上坐著,眼巴巴只望你們成功哩。」行者發狠道:「正是,正是。獃子莫要胡談,土地說得有理。我們正要與他:
\begin{quote}
賭輸贏,弄手段,等我施為地煞變。
自到西方無對頭,牛王本是心猿變。
今番正好會源流,斷要相持借寶扇。
趁清涼,息火焰,打破頑空參佛面。
行滿超昇極樂天,大家同赴龍華宴。」
\end{quote}

那八戒聽言,便生努力。慇懃道:
\begin{quote}
「是是是,去去去,管甚牛王會不會。
木生在亥配為豬,牽轉牛兒歸土類。
申下生金本是猴,無刑無剋多和氣。
用芭蕉,為水意,焰火消除成既濟。
晝夜休離苦盡功,功完趕赴盂蘭會。」
\end{quote}

他兩個領著土地、陰兵一齊上前,使釘鈀,掄鐵棒,乒乒乓乓,把一座摩雲洞的前門打得粉碎。諕得那外護頭目戰戰兢兢,闖入裡邊報道:「大王,孫悟空率眾打破前門也。」那牛王正與玉面公主備言其事,懊恨孫行者哩。聽說打破前門,十分發怒,急披掛,拿了鐵棍,從裡邊罵出來道:「潑猢猻!你是多大個人兒,敢這等上門撒潑,打破我門扇?」八戒近前亂罵道:「潑老剝皮!你是個甚樣人物,敢量那個大小?不要走,看鈀。」牛王喝道:「你這個囔糟食的夯貨不見怎的,快叫那猴兒上來。」行者道:「不知好歹的䬲草!我昨日還與你論兄弟,今日就是仇人了。仔細吃吾一棒。」那牛王奮勇而迎。這場比前番更勝。三個英雄廝混在一處,好殺:
\begin{quote}
釘鈀鐵棒逞神威,同帥陰兵戰老犧。犧牲獨展兇強性,遍滿同天法力恢。使鈀築,著棍擂,鐵棒英雄又出奇。三般兵器叮噹響,隔架遮攔誰讓誰?他道他為首,我道我奪魁。土兵為證難分解,木土相煎上下隨。這兩個說:「你如何不借芭蕉扇?」那一個道:「你焉敢欺心騙我妻?趕妾害兒仇未報,敲門打戶又驚疑。」這個說:「你仔細隄防如意棒,擦著些兒就破皮。」那個說:「好生躲避鈀頭齒,一傷九孔血淋漓。」牛魔不怕施威猛,鐵棍高擎有見機。翻雲覆雨隨來往,吐霧噴風任發揮。恨苦這場都拚命,各懷惡念喜相持。丟架子,讓高低,前迎後擋總無虧。兄弟二人齊努力,單身一棍獨施為。卯時戰到辰時後,戰罷牛魔束手回。
\end{quote}

他三個舍死忘生,又鬥有百十餘合。八戒發起獃性,仗著行者神通,舉鈀亂築。牛王遮架不住,敗陣回頭,就奔洞門。卻被土地、陰兵攔住洞門,喝道:「大力王,那裡走?吾等在此。」那老牛不得進洞,急抽身。又見八戒、行者趕來,慌得卸了盔甲,丟了鐵棍,搖身一變,變做一隻天鵝,望空飛走。

行者看見,笑道:「八戒,老牛去了。」那獃子漠然不知,土地亦不能曉,一個個東張西覷,只在積雷山前後亂找。行者指道:「那空中飛的不是?」八戒道:「那是一隻天鵝。」行者道:「正是老牛變的。」土地道:「既如此,卻怎麼好?」行者道:「你兩個打進此門,把群妖盡情剿除,拆了他的窩巢,絕了他的歸路。等老孫與他賭變化去。」那八戒與土地,依言攻破洞門不題。

這大聖收了金箍棒,捻訣念咒,搖身一變,變作一個海東青。颼的一翅,鑽在雲眼裡,倒飛下來,落在天鵝身上,抱住頸項嗛眼。那牛王也知是孫行者變化,急忙抖抖翅,變作一隻黃鷹,返來嗛海東青。行者又變作一個烏鳳,專一趕黃鷹。牛王識得,又變作一隻白鶴,長唳一聲,向南飛去。行者立定,抖抖翎毛,又變作一隻丹鳳,高鳴一聲。那白鶴見鳳是鳥王,諸禽不敢妄動,刷的一翅,淬下山崖,將身一變,變作一隻香獐,乜乜些些,在崖前吃草。行者認得,也就落下翅來,變作一隻餓虎,剪尾跑蹄,要來趕獐作食。魔王慌了手腳,又變作一隻金錢花斑的大豹,要傷餓虎。行者見了,迎著風,把頭一幌,又變作一隻金眼狻猊,聲如霹靂,鐵額銅頭,復轉身要食大豹。牛王著了急,又變作一個人熊,放開腳,就來擒那狻猊。行者打個滾,就變作一隻賴象,鼻似長蛇,牙如竹筍,撒開鼻子,要去捲那人熊。

牛王嘻嘻的笑了一笑,現出原身:一隻大白牛,頭如峻嶺,眼若閃光,兩隻角似兩座鐵塔,牙排利刃,連頭至尾有千餘丈長短,自蹄至背有八百丈高下。對行者高叫道:「潑猢猻!你如今將奈我何?」行者也就現了原身,抽出金箍棒來,把腰一躬,喝聲叫:「長!」長得身高萬丈,頭如泰山,眼如日月,口似血池,牙似門扇,手執一條鐵棒,著頭就打。那牛王硬著頭,使角來觸。這一場,真個是撼嶺搖山,驚天動地。有詩為證。詩曰:
\begin{quote}
道高一尺魔千丈,奇巧心猿用力降。
若得火山無烈焰,必須寶扇有清涼。
黃婆矢志扶元老,木母留情掃蕩妖。
和睦五行歸正果,煉魔滌垢上西方。
\end{quote}

他兩個大展神通,在半山中賭鬥,驚得那過往虛空一切神眾與金頭揭諦、六甲六丁、一十八位護教伽藍都來圍困魔王。那魔王公然不懼,你看他東一頭,西一頭,直挺挺、光耀耀的兩隻鐵角,往來抵觸;南一撞,北一撞,毛森森,觔暴暴的一條硬尾,左右敲搖。孫大聖當面迎,眾多神四面打。牛王急了,就地一滾,復本像,便投芭蕉洞去。行者也收了法像,與眾多神隨後追襲。那魔王闖入洞裡,閉門不出。概眾把一座翠雲山圍得水泄不通。

正都上門攻打,忽聽得八戒與土地、陰兵嚷嚷而至。行者見了,問曰:「那摩雲洞事體如何?」八戒笑道:「那老牛的娘子,被我一鈀築死,剝開衣看,原來是個玉面狸精。那夥群妖,俱是些驢、騾、犢、特、獾、狐、狢、獐、羊、虎、麋、鹿等類,已此盡皆剿戮。又將他洞府房廊放火燒了。土地說他還有一處家小,住居此山,故又來這裡掃蕩也。」行者道:「賢弟有功,可喜,可喜。老孫空與那老牛賭變化,未曾得勝。他變做無大不大的白牛,我變了法天象地的身量。正和他抵觸之間,幸蒙諸神下降,圍困多時,他卻復原身,走進洞去矣。」八戒道:「那可是芭蕉洞麼?」行者道:「正是,正是。羅剎女正在此間。」八戒發狠道:「既是這般,怎麼不打進去,剿除那廝,問他要扇子,倒讓他停留長智,兩口兒敘情?」

好獃子,抖擻威風,舉鈀照門一築,忽辣的一聲,將那石崖連門築倒了一邊。慌得那女童忙報:「爺爺,不知甚人把前門都打壞了。」牛王方跑進去,喘噓噓的,正告訴羅剎女與孫行者奪扇子賭鬥之事。聞報,心中大怒,就口中吐出扇子,遞與羅剎女。羅剎女接扇在手,滿眼垂淚道:「大王,把這扇子送與那猢猻,教他退兵去罷。」牛王道:「夫人啊,物雖小而恨則深。你且坐著,等我再和他比併去來。」那魔重整披掛,又選兩口寶劍,走出門來。正遇著八戒使鈀築門,老牛更不打話,掣劍劈頭便砍。八戒舉鈀迎著,向後倒退了幾步,出門來,早有大聖掄棒當頭。那牛魔即駕狂風,跳離洞府,又都在那翠雲山上相持。眾多神四面圍繞,土地兵左右攻擊。這一場,又好殺哩:
\begin{quote}
雲迷世界,霧罩乾坤。颯颯陰風砂石滾,巍巍怒氣海波渾。重磨劍二口,復掛甲全身。結冤深似海,懷恨越生嗔。你看齊天大聖因功績,不講當年老故人。八戒施威求扇子,眾神護法捉牛君。牛王雙手無停息,左遮右擋弄精神。只殺得那過鳥難飛皆斂翅,遊魚不躍盡潛鱗。鬼泣神嚎天地暗,龍愁虎怕日光昏。
\end{quote}

那牛王拚命捐軀,鬥經五十餘合,抵敵不住,敗了陣,往北就走。早有五臺山秘魔巖神通廣大潑法金剛阻住,喝道:「牛魔,你往那裡去?我蒙釋迦牟尼佛祖差來,佈列天羅地網,至此擒汝也。」正說間,隨後有大聖、八戒、眾神趕來。那魔王慌轉身,向南而走,又撞著峨眉山清涼洞法力無量勝至金剛擋住,喝道:「吾奉佛旨,在此正要拿住你也。」牛王心慌腳軟,急抽身往東便走,卻逢著須彌山摩耳崖毘盧沙門大力金剛迎住,喝道:「老牛何往?我蒙如來密令,教來捕獲你也。」牛王又悚然而退,向西就走,又遇著崑崙山金霞嶺不壞尊王永住金剛敵住,喝道:「這廝又將安走?我領西天大雷音寺佛老親言,在此把截,誰放你也?」

那老牛心驚膽戰,悔之不及。見那四面八方都是佛兵天將,真個似羅網高張,不能脫命。正在倉惶之際,又聞得行者帥眾趕來,他就駕雲頭,望上便走。卻好有托塔李天王並哪吒太子,領魚肚藥叉、巨靈神將,幔住空中,叫道:「慢來,慢來。吾奉玉帝旨意,特來此剿除你也。」牛王急了,依前搖身一變,還變做一隻大白牛,使兩隻鐵角去觸天王。天王使刀來砍。隨後孫行者又到。哪吒太子厲聲高叫:「大聖,衣甲在身,不能為禮。愚父子昨日見佛如來發檄奏聞玉帝,言唐僧路阻火焰山,孫大聖難伏牛魔王,玉帝傳旨,特差我父王領眾助力。」行者道:「這廝神通不小,又變作這等身軀,卻怎奈何?」太子笑道:「大聖勿疑,你看我擒他。」

這太子即喝一聲:「變!」變得三頭六臂,飛身跳在牛王背上,使斬妖劍望頸項上一揮,不覺得把個牛頭斬下。天王收刀,卻才與行者相見。那牛王腔子裡又鑽出一個頭來,口吐黑氣,眼放金光。被哪吒又砍一劍,頭落處,又鑽出一個頭來。一連砍了十數劍,隨即長出十數個頭。哪吒取出火輪兒掛在那老牛的角上,便吹真火,焰焰烘烘,把牛王燒得張狂哮吼,搖頭擺尾。才要變化脫身,又被托塔天王將照妖鏡照住本像,騰那不動,無計逃生,只叫:「莫傷我命,情願歸順佛家也。」哪吒道:「既惜身命,快拿扇子出來。」牛王道:「扇子在我山妻處收著哩。」

哪吒見說,將縛妖索子解下,跨在他那頸項上,一把拿住鼻頭,將索穿在鼻孔裡,用手牽來。孫行者卻會聚了四大金剛、六丁六甲、護教伽藍、托塔天王、巨靈神將並八戒、土地、陰兵,簇擁著白牛,回至芭蕉洞口。老牛叫道:「夫人,將扇子出來,救我性命。」羅剎聽叫,急卸了釵環,脫了色服,挽青絲如道姑,穿縞素似比丘,雙手捧那柄丈二長短的芭蕉扇子,走出門。又見有金剛眾聖與天王父子,慌忙跪在地下,磕頭禮拜道:「望菩薩饒我夫妻之命,願將此扇奉承孫叔叔成功去也。」行者近前接了扇,同大眾共駕祥雲,徑回東路。

卻說那三藏與沙僧立一會,坐一會,盼望行者,許久不回,何等憂慮。忽見祥雲滿空,瑞光滿地,飄飄颻颻,蓋眾神行將近,這長老害怕道:「悟淨,那壁廂是誰神兵來也?」沙僧認得道:「師父啊,那是四大金剛、金頭揭諦、六甲六丁、護教伽藍與過往眾神。牽牛的是哪吒三太子,拿鏡的是托塔李天王,大師兄執著芭蕉扇,二師兄並土地隨後,其餘的都是護衛神兵。」三藏聽說,換了毘盧帽,穿了袈裟,與悟淨拜迎眾聖,稱謝道:「我弟子有何德能,敢勞列位尊聖臨凡也。」四大金剛道:「聖僧喜了,十分功行將完。吾等奉佛旨差來助汝,汝當竭力修持,勿得須臾怠惰。」三藏叩齒叩頭,受身受命。

孫大聖執著扇子,行近山邊,盡氣力揮了一扇,那火焰山平平息焰,寂寂除光。行者喜喜歡歡,又搧一扇,只聞得習習瀟瀟,清風微動。第三扇,滿天雲漠漠,細雨落霏霏。有詩為證。詩曰:
\begin{quote}
火焰山遙八百程,火光大地有聲名。
火煎五漏丹難熟,火燎三關道不清。
時借芭蕉施雨露,幸蒙天將助神功。
牽牛歸佛休顛劣,水火相聯性自平。
\end{quote}

此時三藏解燥除煩,清心了意。四眾皈依,謝了金剛,各轉寶山。六丁六甲升空保護。過往神祇四散。天王、太子牽牛,徑歸佛地回繳。止有本山土地押著羅剎女,在傍伺候。

行者道:「那羅剎,你不走路,還立在此等甚?」羅剎跪道:「萬望大聖垂慈,將扇子還了我罷。」八戒喝道:「潑賤人,不知高低。饒了你的性命就夠了,還要討甚麼扇子?我們拿過山去,不會賣錢買點心吃?費了這許多精神力氣,又肯與你?雨濛濛的,還不回去哩。」羅剎再拜道:「大聖原說搧息了火還我,今此一場,誠悔之晚矣。只因不倜儻,致令勞師動眾。我等也修成人道,只是未歸正果。見今真身現象歸西,我再不敢妄作。願賜本扇,從立自新,修身養命去也。」土地道:「大聖,趁此女深知息火之法,斷絕火根,還他扇子,小神居此苟安,拯救這方生民,求些血食,誠為恩便。」行者道:「我當時問著鄉人說:這山搧息火,只收得一年五穀,便又火發。如何治得除根?」羅剎道:「要是斷絕火根,只消連搧四十九扇,永遠再不發了。」

行者聞言,執扇子,使盡筋力,望山頭連搧四十九扇,那山上大雨淙淙。果然是寶貝:有火處下雨,無火處天晴。他師徒們立在這無火處,不遭雨濕。

坐了一夜,次早才收拾馬匹、行李,把扇子還了羅剎。又道:「老孫若不與你,恐人說我言而無信。你將扇子回山,再休生事。看你得了人身,饒你去罷。」那羅剎接了扇子,念個咒語,捏做個杏葉兒,噙在口裡。拜謝了眾聖,隱姓修行,後來也得了正果,經藏中萬古流名。羅剎、土地俱感激謝恩,隨後相送。行者、八戒、沙僧保著三藏,遂此前進,真個是身體清涼,足下滋潤。誠所謂:
\begin{quote}
坎離既濟真元合,水火均平大道成。
\end{quote}

畢竟不知幾年才回東土,且聽下回分解。
