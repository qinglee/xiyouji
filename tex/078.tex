
\chapter{比丘憐子遣陰神 金殿識魔談道德}

\begin{quote}
一念才生動百魔,修持最苦奈他何。
但憑洗滌無塵垢,也用收拴有琢磨。
掃退萬緣歸寂滅,蕩除千怪莫蹉跎。
管教跳出樊籠套,行滿飛昇上大羅。
\end{quote}

話說孫大聖用盡心機,請如來收了眾怪,解脫三藏師徒之難,離獅駝城西行。又經數月,早值冬天。但見那:
\begin{quote}
嶺梅將破玉,池水漸成冰。
紅葉俱飄落,青松色更新。
淡雲飛欲雪,枯草伏山平。
滿目寒光迥,陰陰透骨冷。
\end{quote}

師徒們沖寒冒冷,宿雨餐風。正行間,又見一座城池。三藏問道:「悟空,那廂又是甚麼所在?」行者道:「到跟前自知。若是西邸王位,須要倒換關文;若是府州縣,徑過。」

師徒言語未畢,早至城門之外。三藏下馬,一行四眾,進了月城。見一個老軍在向陽牆下,偎風而睡。行者近前,搖他一下,叫聲:「長官。」那老軍猛然驚覺,麻麻糊糊的睜開眼,看見行者,連忙跪下磕頭,叫:「爺爺。」行者道:「你休胡驚作怪。我又不是甚麼惡神,你叫爺爺怎的?」老軍磕頭道:「你是雷公爺爺。」行者道:「胡說。吾乃東土去西天取經的僧人。適才到此,不知地名,問你一聲的。」那老軍聞言,卻才正了心,打個啊欠,爬起來,伸伸腰道:「長老,長老,恕小人之罪。此處地方,原喚比丘國,今改作小子城。」行者道:「國中有帝王否?」老軍道:「有有有。」行者卻轉身對唐僧道:「師父,此處原是比丘國,今改小子城,但不知改名之意何故也。」唐僧疑惑道:「既云比丘,又何云小子?」八戒道:「想是比丘王崩了,新立王位的是個小子,故名小子城。」唐僧道:「無此理,無此理。我們且進去,到街坊上再問。」沙僧道:「正是。那老軍一則不知,二則被大哥諕得胡說。且入城去詢問。」

又入三層門裡,到通衢大市觀看,倒也衣冠濟楚,人物清秀。但見那:
\begin{quote}
酒樓歌館語聲喧,彩鋪茶房高掛帘。
萬戶千門生意好,六街三市廣財源。
買金販錦人如蟻,奪利爭名只為錢。
禮貌莊嚴風景盛,河清海晏太平年。
\end{quote}

師徒四眾牽著馬,挑著擔,在街市上行夠多時,看不盡繁華氣概,但只見家家門口一個鵝籠。三藏道:「徒弟啊,此處人家都將鵝籠放在門首,何也?」八戒聽說,左右觀之,果是鵝籠,排列五色綵緞遮幔。獃子笑道:「師父,今日想是黃道良辰,宜結婚姻會友,都行禮哩。」行者道:「胡談,那裡就家家都行禮?其間必有緣故,等我上前看看。」三藏扯住道:「你莫去,你嘴臉醜陋,怕人怪你。」行者道:「我變化個兒去來。」

好大聖,捻著訣,念聲咒語,搖身一變,變作一個蜜蜂兒,展開翅,飛近邊前,鑽進幔裡觀看,原來裡面坐的是個小孩兒。再去第二家籠裡看,也是個小孩兒。連看八九家,都是個小孩兒。卻是男身,更無女子。有的坐在籠中頑耍,有的坐在裡邊啼哭;有的吃果子,有的或睡坐。行者看罷,現原身,回報唐僧道:「那籠裡是些小孩子,大者不滿七歲,小者只有五歲,不知何故。」三藏見說,疑思不定。

忽轉街見一衙門,乃金亭館驛。長老喜道:「徒弟,我們且進這驛裡去:一則問他地方,二則撒和馬匹,三則天晚投宿。」沙僧道:「正是,正是,快進去耶。」四眾欣然而入。只見那在官人果報與驛丞,接入門,各各相見。敘坐定,驛丞問:「長老自何方來?」三藏言:「貧僧東土大唐差往西天取經者,今到貴處,有關文理當照驗,權借高衙一歇。」驛丞即命看茶。茶畢,即辦支應,命當直的安排管待。三藏稱謝,又問:「今日可得入朝見駕,照驗關文?」驛丞道:「今晚不能,須待明日早朝。今晚且於敝衙門寬住一宵。」

少頃,安排停當,驛丞即請四眾同吃了齋供。又教手下人打掃客房安歇。三藏感謝不盡。既坐下,長老道:「貧僧有一件不明之事請教,煩為指示。貴處養孩兒,不知怎生看待。」驛丞道:「天無二日,人無二理。養育孩童,父精母血,懷胎十月,待時而生。生下乳哺三年,漸成體相。豈有不知之理。」三藏道:「據尊言與敝邦無異。但貧僧進城時,見街坊人家各設一鵝籠,都藏小兒在內。此事不明,故敢動問。」驛丞附耳低言道:「長老莫管他,莫問他,也莫理他,說他。請安置,明早走路。」長老聞言,一把扯住驛丞,定要問個明白。驛丞搖頭搖指,只叫:「謹言。」三藏一發不放,執死定要問個詳細。驛丞無奈,只得屏去一應在官人等。獨在燈光之下,悄悄而言道:「適所問鵝籠之事,乃是當今國主無道之事。你只管問他怎的?」三藏道:「何為無道?必見教明白,我方得放心。」驛丞道:「此國原是比丘國,近有民謠,改作小子城。三年前,有一老人,打扮做道人模樣,攜一小女子,年方一十六歲。其女形容嬌俊,貌若觀音,進貢與當今,陛下愛其色美,寵幸在宮,號為美后。近來把三宮娘娘、六院妃子,全無正眼相覷。不分晝夜,貪歡不已。如今弄得精神瘦倦,身體尪羸,飲食少進,命在須臾。太醫院檢盡良方,不能療治。那進女子的道人,受我主誥封,稱為國丈。國丈有海外秘方,甚能延壽。前者去十洲、三島採將藥來,俱已完備。但只是藥引子利害:單用著一千一百一十一個小兒的心肝,煎湯服藥。服後有千年不老之功。這些鵝籠裡的小兒,俱是選就的,養在裡面。人家父母懼怕王法,俱不敢啼哭,遂傳播謠言,叫做小兒城。長老明早到朝:只去倒換關文,不得言及此事。」言畢,抽身而退。

諕得個長老骨軟筋麻,止不住腮邊淚墮。忽失聲叫道:「昏君,昏君!為你貪歡愛美,弄出病來,怎麼屈傷這許多小兒性命?苦哉,苦哉,痛殺我也!」有詩為證。詩曰:
\begin{quote}
邪主無知失正真,貪歡不省暗傷身。
因求永壽戕童命,為解天災殺小民。
僧發慈悲難割捨,官言利害不堪聞。
燈前灑淚長吁嘆,痛倒參禪向佛人。
\end{quote}

八戒近前道:「師父,你是怎的起哩?專把別人棺材擡在自家家裡哭。不要煩惱。常言道:『君教臣死,臣不死不忠;父教子亡,子不亡不孝。』他傷的是他的子民,與你何干?且來寬衣服睡覺,莫替古人耽憂。」三藏滴淚道:「徒弟啊,你是一個不慈憫的。我出家人積功累行,第一要行方便。怎麼這昏君一味胡行?從來也不見吃人心肝,可以延壽。似這等之事,教我怎不傷悲?」沙僧道:「師父且莫傷悲。等明早倒換關文,覿面與國王講過。如若不從,看他是怎麼模樣的一個國丈。或恐那國丈是個妖精,欲吃人的心肝,故設此法,未可知也。」

行者道:「悟淨說得有理。師父,你且睡覺,明日等老孫同你進朝,看國丈的好歹。如若是人,只恐他走了傍門,不知正道,徒以採藥為真,待老孫將先天之要旨,化他皈正;若是妖邪,我把他拿住,與這國王看看,教他寬慾養身,斷不教他傷了那些孩童性命。」三藏聞言,急躬身,反對行者施禮道:「徒弟啊,此論極妙,極妙。但只是見了昏君,不可便問此事,恐那昏君不分遠近,並作謠言見罪,卻怎生區處?」行者笑道:「老孫自有法力。如今先將鵝籠小兒攝離此城,教他明日無物取心,地方官自然奏表。那昏君必有旨意,或與國丈商量,或者另行選報。那時節,借此舉奏,決不致罪坐於我也。」三藏甚喜。又道:「如今怎得小兒離城?若果能脫得,真賢徒天大之德。可速為之,略遲緩些,恐無及也。」行者抖擻神威,即起身,吩咐八戒、沙僧:「同師父坐著,等我施為,你看但有陰風刮動,就是小兒出城了。」他三人一齊俱念:「南無救生藥師佛!南無救生藥師佛!」

這大聖出得門外,打個唿哨,起在半空,捻了訣,念動真言,叫一聲「唵淨法界」,拘得那城隍、土地、社令、真官,並五方揭諦、四值功曹、六丁六甲與護教伽藍等眾,都到空中,對他施禮道:「大聖,夜喚吾等,有何急事?」行者道:「今因路過比丘國,那國王無道,聽信妖邪,要取小兒心肝做藥引子,指望長生。我師父十分不忍,欲要救生滅怪。故老孫特請列位,各使神通,與我把這城中各街坊人家鵝籠裡的小兒,連籠都攝出城外山凹中,或樹林深處,收藏一二日,與他些果子食用,不得餓損;再暗的護持,不得使他驚恐啼哭。待我除了邪,治了國,勸正君王,臨行時,送來還我。」

眾神聽令,即便各使神通,按下雲頭。滿城中陰風滾滾,慘霧漫漫:
\begin{quote}
陰風刮暗一天星,慘霧遮昏千里月。起初時還蕩蕩悠悠,次後來就轟轟烈烈。悠悠蕩蕩,各尋門戶救孩童;烈烈轟轟,都看鵝籠援骨血。冷氣侵人怎出頭,寒威透體衣如鐵。父母徒張皇,兄嫂皆悲切。滿地捲陰風,籠兒被神攝。此夜縱孤恓,天明盡歡悅。
\end{quote}

有詩為證,詩曰:
\begin{quote}
釋門慈憫古來多,正善成功說摩訶。
萬聖千真皆積德,三皈五戒要從和。
比丘一國非君亂,小子千名是命訛。
行者因師同救護,這場陰騭勝波羅。
\end{quote}

當夜有三更時分,眾神祇把鵝籠攝去各處安藏。

行者按下祥光,徑至驛庭上,只聽得他三人還念「南無救生藥師佛」哩。他也心中暗喜,近前叫:「師父,我來也。陰風之起何如?」八戒道:「好陰風。」三藏道:「救兒之事,卻怎麼說?」行者道:「已一一救他出去,待我們起身時送還。」長老謝了又謝,方才就寢。

至天曉,三藏醒來,遂結束齊備道:「悟空,我趁早朝,倒換關文去也。」行者道:「師父,你自家去,恐不濟事,待老孫和你同去,看那國丈邪正如何。」三藏道:「你去卻不肯行禮,恐國王見怪。」行者道:「我不現身,暗中跟隨你,就當保護。」三藏甚喜,吩咐八戒、沙僧看守行李、馬匹,卻才舉步。這驛丞又來相見,看這長老打扮起來,比昨日又甚不同。但見他:
\begin{quote}
身上穿一領錦襴異寶佛袈裟,頭戴金頂毘盧帽。九環錫杖手中拿,胸藏一點神光妙。通關文牒緊隨身,包裹袋中纏錦套。行似阿羅降世間,誠如活佛真容貌。
\end{quote}

那驛丞相見禮畢,附耳低言,只教莫管閑事。三藏點頭應聲。大聖閃在門傍,念個咒語,搖身一變,變做個蟭蟟蟲兒,嚶的一聲,飛在三藏帽兒上。出了館驛,徑奔朝中。

及到朝門外,見有黃門官,即施禮道:「貧僧乃東土大唐差往西天取經者。今到貴地,理當倒換關文,意欲見駕,伏乞轉奏轉奏。」那黃門官果為傳奏。國王喜道:「遠來之僧,必有道行。」教請進來。黃門官復奉旨,將長老請入。長老階下朝見畢,復請上殿賜坐。長老又謝恩坐了。只見那國王相貌尪羸,精神倦怠:舉手處,揖讓差池;開言時,聲音斷續。長老將文牒獻上,那國王眼目昏朦,看了又看,方才取寶印,用了花押,遞與長老。長老收訖。

那國王正要問取經原因,只聽得當駕官奏道:「國丈爺爺來矣。」那國王即扶著近侍小宦,掙下龍床,躬身迎接。慌得那長老急起身,側立於傍。回頭觀看,原來是一個老道者,自玉階前,搖搖擺擺而進。但見他:
\begin{quote}
頭上戴一頂淡鵝黃九錫雲錦紗巾,身上穿一領箸頂梅沉香綿絲鶴氅。腰間繫一條紉藍三股攢絨帶,足下踏一對麻經葛緯雲頭履。手中拄一根九節枯藤盤龍拐杖,胸前掛一個描龍刺鳳團花錦囊。玉面多光潤,蒼髯頷下飄。金睛飛火焰,長目過眉梢。行動雲隨步,逍遙香霧饒。階下眾官都拱接,齊呼國丈進王朝。
\end{quote}

那國丈到寶殿前,更不行禮,昂昂烈烈,徑到殿上。國王欠身道:「國丈仙蹤,今喜早降。」就請左手繡墩上坐。

三藏起一步,躬身施禮道:「國丈大人,貧僧問訊了。」那國丈端然高坐,亦不回禮,轉面向國王道:「僧家何來?」國王道:「東土唐朝差上西天取經者,今來倒驗關文。」國丈笑道:「西方之路,黑漫漫有甚好處?」三藏道:「自古西方乃極樂之勝境,如何不好?」那國王問道:「朕聞上古有云:『僧是佛家弟子。』端的不知為僧可能不死,向佛可能長生?」三藏聞言,急合掌應道:
\begin{quote}
「為僧者,萬緣都罷;了性者,諸法皆空。大智閑閑,澹泊在不生之內;真機默默,逍遙於寂滅之中。三界空而百端治,六根淨而千種窮。若乃堅誠知覺,須當識心:心淨則孤明獨照,心存則萬境皆清。真容無欠亦無餘,生前可見;幻相有形終有壞,分外何求?行功打坐,乃為入定之原;佈惠施恩,誠是修行之本。大巧若拙,還知事事無為;善計非籌,必須頭頭放下。但使一心不動,萬行自全;若云採陰補陽,誠為謬語。服餌長壽,實乃虛詞。只要塵塵緣總棄,物物色皆空。素素純純寡愛慾,自然享壽永無窮。」
\end{quote}

那國丈聞言,付之一笑。用手指定唐僧道:「呵呵呵,你這和尚滿口胡柴。寂滅門中,須云認性。你不知那性從何而滅,枯坐參禪,盡是些盲修瞎煉。俗語云:『坐坐坐,你的屁股破。火熬煎,反成禍。』更不知我這:
\begin{quote}
修仙者,骨之堅秀;達道者,神之最靈。攜簞瓢而入山訪友,採百藥而臨世濟人。摘仙花以砌笠,折香蕙以鋪裀。歌之鼓掌,舞罷眠雲。闡道法,揚太上之正教;施符水,除人世之妖氛。奪天地之秀氣,採日月之華精。運陰陽而丹結,按水火而胎凝。二八陰消兮,若恍若惚;三九陽長兮,如杳如冥。應四時而採取藥物,養九轉而修煉丹成。跨青鸞,升紫府;騎白鶴,上瑤京。參滿天之華采,表妙道之慇懃。比你那靜禪釋教,寂滅陰神,涅槃遺臭殼,又不脫凡塵。三教之中無上品,古來惟道獨稱尊。」
\end{quote}

那國王聽說,十分歡喜。滿朝官都喝采道:「好個『惟道獨稱尊』,『惟道獨稱尊』。」長老見人都讚他,不勝羞愧。國王又叫光祿寺安排素齋,待那遠來之僧出城西去。三藏謝恩而退。才下殿,往外正走,行者飛下帽頂兒,來在耳邊叫道:「師父,這國丈是個妖邪,國王受了妖氣。你先去驛中等齋,待老孫在這裡聽他消息。」三藏知會了,獨出朝門不題。

看那行者,一翅飛在金鑾殿翡翠屏中釘下,只見那班部中閃出五城兵馬官奏道:「我主,今夜一陣冷風,將各坊各家鵝籠裡小兒,連籠都刮去了,更無蹤跡。」國王聞奏,又驚又惱,對國丈道:「此事乃天滅朕也。連月病重,御醫無效,幸國丈賜仙方,專待今日午時開刀,取此小兒心肝作引,何期被冷風刮去,非天欲滅朕而何?」國丈笑道:「陛下且休煩惱。此兒刮去,正是天送長生與陛下也。」國王道:「見把籠中之兒刮去,何以返說天送長生?」國丈道:「我才入朝來,見了一個絕妙的藥引,強似那一千一百一十一個小兒之心。那小兒之心,只延得陛下千年之壽;此引子,吃了我的仙藥,就可延萬萬年也。」國王漠然不知是何藥引,請問再三,國丈才說:「那東土差去取經的和尚,我看他器宇清淨,容顏齊整,乃是個十世修行的真體,自幼為僧,元陽未泄,比那小兒更強萬倍。若得他的心肝煎湯,服我的仙藥,足保萬年之壽。」那昏君聞言,十分聽信,對國丈道:「何不早說?若果如此有效,適才留住,不放他去了。」國丈道:「此何難哉?適才吩咐光祿寺辦齋待他,他必吃了齋,方才出城。如今急傳旨,將各門緊閉,點兵圍了金亭館驛,將那和尚拿來,必以禮求其心。如果相從,即時剖而取出,遂御葬其屍,還與他立廟享祭;如若不從,就與他個武不善作,即時綑住,剖開取之。有何難事?」那昏君如其言,即傳旨,把各門閉了。又差羽林衛大小官軍,圍住館驛。

行者聽得這個消息,一翅飛奔館驛,現了本相,對唐僧道:「師父,禍事了,禍事了。」那三藏才與八戒、沙僧領御齋,忽聞此言,諕得三尸神散,七竅煙生,倒在塵埃,渾身是汗,眼不定睛,口不能言。慌得沙僧上前攙住,只叫:「師父甦醒,師父甦醒。」八戒道:「有甚禍事?有甚禍事?你慢些兒說便也罷,卻諕得師父如此。」行者道:「自師父出朝,老孫回視,那國丈是個妖精。少頃,有五城兵馬來奏冷風刮去小兒之事。國王方惱,他卻轉教喜歡,道:『這是天送長生與你。』要取師父的心肝做藥引,可延萬年之壽。那昏君聽信誣言,所以點精兵,來圍館驛,差錦衣官來請師父求心也。」八戒笑道:「行的好慈憫,救的好小兒,刮的好陰風,今番卻撞出禍來了。」

三藏戰兢兢的爬起來,扯著行者,哀告道:「賢徒啊,此事如何是好?」行者道:「若要好,大做小。」沙僧道:「怎麼叫做『大做小』?」行者道:「若要全命,師作徒,徒作師,方可保全。」三藏道:「你若救得我命,情願與你做徒子、徒孫也。」行者道:「既如此,不必遲疑。」教:「八戒,快和些泥來。」那獃子即使釘鈀築了些土。又不敢外面去取水,後就擄起衣服撒溺,和了一團臊泥,遞與行者。行者沒奈何,將泥撲作一片,往自家臉上一安,做下個猴像的臉子。叫唐僧站起休動,再莫言語。貼在唐僧臉上,念動真言,吹口仙氣,叫:「變!」那長老即變做個行者模樣。脫了他的衣服,以行者的衣服穿上。行者卻將師父的衣服穿了,捻著訣,念個咒語,搖身變作唐僧的嘴臉。八戒、沙僧也難識認。

正當合心裝扮停當,只聽得鑼鼓齊鳴,又見那槍刀簇擁。原來是羽林衛官,領三千兵把館驛圍了。又見一個錦衣官走進驛庭問道:「東土唐朝長老在那裡?」慌得那驛丞戰兢兢的跪下,指道:「在下面客房裡。」錦衣官即至客房裡道:「唐長老,我王有請。」八戒、沙僧左右護持假行者。只見假唐僧出門施禮道:「錦衣大人,陛下召貧僧,有何話說?」錦衣官上前一把扯住道:「我與你進朝去,想必有取用也。」咦!這正是:
\begin{quote}
妖誣勝慈善,慈善反招凶。
\end{quote}

畢竟不知此去端的性命何如,且聽下回分解。
