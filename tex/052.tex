
\chapter{悟空大鬧金兜洞 如來暗示主人公}

話說孫大聖得了金箍棒,打出門前,跳上高峰,對眾神滿心歡喜。李天王道:「你這場如何?」行者道:「老孫變化進他洞去,那怪物越發唱唱舞舞的吃得勝酒哩。更不曾打聽得他的寶貝在那裡。我轉他後面,忽聽得馬叫龍吟,知是火部之物。東壁廂靠著我的金箍棒,是老孫拿在手中,一路打將出來也。」眾神道:「你的寶貝得了,我們的寶貝何時到手?」行者道:「不難,不難。我有了這根鐵棒,不管怎的,也要打倒他,取寶貝還你。」

正講處,只聽得那山坡下鑼鼓齊鳴,喊聲振地。原來是兕大王帥眾精靈來趕行者。行者見了,叫道:「好好好,正合吾意。列位請坐,待老孫再去捉他。」好大聖,舉鐵棒劈面迎來,喝道:「潑魔那裡走!看棍!」那怪使槍支住,罵道:「賊猴頭!著實無禮。你怎麼白晝劫吾物件?」行者道:「我把你這個不知死的孽畜!你倒弄圈套白晝搶奪我物,那件兒是你的?不要走,吃老爺一棍!」那怪物掄槍隔架。這一場好戰:
\begin{quote}
大聖施威猛,妖魔不順柔。兩家齊鬥勇,那個肯干休。這一個鐵棒如龍尾,那一個長槍似蟒頭。這一個棒來解數如風響,那一個槍架雄威似水流。只見那彩霧朦朦山嶺暗,祥雲靉靆樹林愁。滿空飛鳥皆停翅,四野狼蟲盡縮頭。那陣上小妖吶喊,這壁廂行者抖擻。一條鐵棒無人敵,打遍西方萬里遊。那桿長槍真對手,永鎮金稱上籌。相遇這場無好散,不見高低誓不休。
\end{quote}

那魔王與孫大聖戰經三個時辰,不分勝敗,早又見天色將晚。妖魔支著長槍道:「悟空,你住了。天昏地暗,不是個賭鬥之時,且各歇息歇息,明朝再與你比迸。」行者罵道:「潑畜休言!老孫的興頭才來,管甚麼天晚?是必與你定個輸贏。」那怪物喝一聲,虛幌一槍,逃了性命,帥群妖收轉干戈,入洞中將門緊緊閉了。

這大聖拽棍方回,天神在岸頭賀喜,都道:「是有能有力的大齊天,無量無邊的真本事。」行者笑道:「承過獎,承過獎。」李天王近前道:「此言實非褒獎,真是一條好漢子。這一陣也不亞當時瞞地網罩天羅也。」行者道:「且休題夙話。那妖魔被老孫打了這一場,必然疲倦。我也說不得辛苦,你們都放懷坐坐,等我再進洞去打聽他的圈子,務要偷了他的,捉住那怪,尋取兵器,奉還汝等歸天。」太子道:「今已天晚,不若安眠一宿,明早去罷。」行者笑道:「這小郎不知世事,那見做賊的好白日裡下手?似這等掏摸的,必須夜去夜來,不知不覺,才是買賣哩。」火德與雷公道:「三太子休言,這件事我們不知。大聖是個慣家熟套,須教他趁此時候,一則魔頭困倦,二來夜黑無防,就請快去,快去。」

好大聖,笑唏唏的,將鐵棒藏了。跳下高峰,又至洞口,搖身一變,變作一個促織兒。真個:
\begin{quote}
嘴硬鬚長皮黑,眼明爪腳丫叉。
風清月明叫牆涯。夜靜如同人話。
泣露淒涼景色,聲音斷續堪誇。
客窗旅思怕聞他。偏在空階床下。
\end{quote}

蹬開大腿,三五跳,跳到門邊,自門縫裡鑽將進去,蹲在那壁根下,迎著裡面燈光,仔細觀看。只見那大小群妖,一個個狼餐虎嚥,正都吃東西哩。行者揲揲鎚鎚的叫了一遍。少時間,收了家火,又都去安排窩鋪,各各安身。

約摸有一更時分,行者才到他後邊房裡,只聽那老魔傳令,教:「各門上小的醒睡,恐孫悟空又變甚麼,私入家偷盜。」又有些該班坐夜的,滌滌托托,梆鈴齊響。這大聖越好行事。鑽入房門,見有一架石床,左右列幾個抹粉搽胭的山精樹鬼,展鋪蓋伏侍老魔,脫腳的脫腳,解衣的解衣。只見那魔王寬了衣服,左肐膊上白森森的套著那個圈子,原來像一個連珠鐲頭模樣。你看他更不取下,轉往上抹了兩抹,緊緊的勒在肐膊上,方才睡下。行者見了,將身又變,變作一個黃皮虼蚤,跳上石床,鑽入被裡,爬在那怪的肐膊上,著實一口。叮的那怪翻身罵道:「這些少打的奴才!被也不抖,床也不拂,不知甚麼東西,咬了我這一下。」他卻把圈子又捋上兩捋,依然睡下。行者爬上那圈子,又咬一口。那怪睡不得,又翻過身來道:「刺鬧殺我也!」

行者見他關防得緊,寶貝又隨身,不肯除下,料偷他的不得。跳下床來,還變做促織兒,出了房門,徑至後面,又聽得龍吟馬嘶。原來那層門緊鎖,火龍、火馬都吊在裡面。行者現了原身,走近門前,使個解鎖法,念動咒語,用手一抹,扢扠一聲,那鎖雙鐄俱就脫落。推開門,闖將進去觀看,原來那裡面被火器照得明晃晃的,如白日一般。忽見東西兩邊斜靠著幾件兵器,都是太子的砍妖刀等物,並那火德的火弓、火箭等物。行者映火光,週圍看了一遍,又見那門背後一張石桌子上有一個篾絲盤兒,放著一把毫毛。大聖滿心歡喜,將毫毛拿起來,啊了兩口熱氣,叫聲:「變!」即變作三五十個小猴。教他都拿了刀、劍、杵、索、裘輪及弓、箭、槍、車、葫蘆、火鴉、火鼠、火馬,一應套去之物,了火龍,縱起火勢,從裡邊往外燒來。只聽得烘烘,撲撲乒乒,好便似咋雷連炮之聲。慌得那些大小妖精夢夢查查的,披著被,朦著頭,喊的喊,哭的哭,一個個走頭無路,被這火燒死大半。美猴王得勝回來,只好有三更時候。

卻說那高峰上,李天王眾位忽見火光晃亮,一擁前來。見行者騎著龍,喝喝呼呼,縱著小猴,徑上峰頭,厲聲高叫道:「來收兵器,來收兵器。」火德與哪吒答應一聲。這行者將身一抖,那把毫毛復上身來。哪吒太子收了他六件兵器,火德星君著眾火部收了火龍等物,都笑吟吟贊賀行者不題。

卻說那金洞裡火焰紛紛,諕得個兕大王魂不附體,急欠身開了房門,雙手拿著圈子,東推東火滅,西推西火消,滿空中冒煙突火,執著寶貝跑了一遍,四下裡煙火俱熄。急忙收救群妖,已此燒殺大半,男男女女,收不上百十餘丁;又查看藏兵之內,各件皆無。又去後面看處,見八戒、沙僧與長老還綑住未解,白龍馬還在槽上,行李擔亦在屋裡。妖魔遂恨道:「不知是那個小妖不仔細,失了火,致令如此。」傍有近侍的告道:「大王,這火不干本家之事。多是個偷營劫寨之賊,放了那火部之物,盜了神兵去也。」老魔方然省悟道:「沒有別人,斷乎是孫悟空那賊。怪道我臨睡時不得安穩。想是那賊猴變化進來,在我這肐膊叮了兩口。一定是要偷我的寶貝,見我抹勒得緊,不能下手,故此盜了兵器,縱著火龍,放此狠毒之心,意欲燒殺我也。賊猴啊!你枉使機關,不知我的本事。我但帶了這件寶貝,就是入大海而不能溺,赴火池而不能焚哩。這番若拿住那賊,只把刮了點垛,方趁我心。」

說著話,懊惱多時,不覺的雞鳴天曉。那高峰上太子得了六件兵器,對行者道:「大聖,天色已明,不須怠慢,我們趁那妖魔挫了銳氣,與火部等扶助你,再去力戰,庶幾這次可擒拿也。」行者笑道:「說得有理。我們齊了心,耍子兒去耶。」一個個抖擻威風,喜弄武藝,徑至洞口。行者叫道:「潑魔出來,與老孫打者。」

原來那裡兩扇石門被火氣化成灰燼,門裡邊有幾個小妖,正然掃地撮灰。忽見眾聖齊來,慌得丟了掃帚,撇下灰耙,跑入裡面,又報道:「孫悟空領著許多天神,又在門外罵戰哩。」那兕怪聞報大驚,扢迸迸,鋼牙咬響;滴溜溜,環眼睜圓。挺著長槍,帶了寶貝,走出門來,潑口亂罵道:「我把你這個偷營放火的賊猴!你有多大手段,敢這等藐視我也?」行者笑臉兒罵道:「潑怪物!你要知我的手段,且上前來,我說與你聽:
\begin{quote}
自小生來手段強,乾坤萬里有名揚。
當時穎悟修仙道,昔日傳來不老方。
立志拜投方寸地,虔心參見聖人鄉。
學成變化無量法,宇宙長空任我狂。
閑在山前將虎伏,悶來海內把龍降。
祖居花果稱王位,水簾洞裡逞剛強。
幾番有意圖天界,數次無知奪上方。
御賜齊天名大聖,敕封又贈美猴王。
只因宴設蟠桃會,無簡相邀我性剛。
暗闖瑤池偷玉液,私行寶閣飲瓊漿。
龍肝鳳髓曾偷吃,百味珍饈我竊嘗。
千載蟠桃隨受用,萬年丹藥任充腸。
天宮異物般般取,聖府奇珍件件藏。
玉帝訪我有手段,即發天兵擺戰場。
九曜惡星遭我貶,五方兇宿被吾傷。
普天神將皆無敵,十萬雄師不敢當。
威逼玉皇傳旨意,灌江小聖把兵揚。
相持七十單二變,各弄精神個個強。
南海觀音來助戰,淨瓶楊柳也相幫。
老君又使金剛套,把我擒拿到上方。
綁見玉皇張大帝,曹官拷較罪該當。
即差大力開刀斬,刀砍頭皮火焰光。
百計千方弄不死,將吾押赴老君堂。
六丁神火爐中煉,煉得渾身硬似鋼。
七七數完開鼎看,我身跳出又兇張。
諸神閉戶無遮擋,眾聖商量把佛央。
其實如來多法力,果然智慧廣無量。
手中賭賽翻觔斗,將山壓我不能強。
玉皇才設安天會,西域方稱極樂場。
壓困老孫五百載,一些茶飯不曾嘗。
金蟬長老臨凡世,東土差他拜佛鄉。
欲取真經回上國,大唐帝主度先亡。
觀音勸我皈依善,秉教迦持不放狂。
解脫高山根下難,如今西去取經章。
潑魔休弄獐狐智,還我唐僧拜法王。」
\end{quote}

那怪聞言,指著行者道:「你原來是個偷天的大賊。不要走,吃吾一槍。」這大聖使棒來迎。兩個正自相持,這壁廂哪吒太子生嗔,火德星君發狠,即將那六件神兵、火部等物,望妖魔身上拋來。孫大聖更加雄勢。一邊又雷公使㨝,天王舉刀,不分上下,一擁齊來。那魔頭巍巍冷笑,袖子中暗暗將寶貝取出,撒手拋起空中,叫聲:「著!」唿喇的一下,把六件神兵、火部等物、雷公㨝、天王刀、行者棒,盡情又都撈去。眾神靈依然赤手,孫大聖仍是空拳。妖魔得勝回身,叫:「小的們,搬石砌門,動土修造,從新整理房廊。待齊備了,殺唐僧三眾來謝土,大家散福受用。」眾小妖領命維持不題。

卻說那李天王帥眾回上高峰,火德怨哪吒性急,雷公怪天王放刁,惟水伯在傍無語。行者見他們面不廝睹,心有縈思,沒奈何,懷恨強歡,對眾笑道:「列位不須煩惱。自古道:『勝敗兵家之常。』我和他論武藝,也只如此;但只是他多了這個圈子,所以為害,把我等兵器又套將去了。你且放心,待老孫再去查查他的腳色來也。」太子道:「你前啟奏玉帝,查勘滿天世界,更無一點蹤跡,如今卻又何處去查?」行者道:「我想起來,佛法無邊。如今且上西天問我佛如來,教他著慧眼觀看大地四部洲,看這怪是那方生長,何處鄉貫住居,圈子是件甚麼寶貝。不管怎的,一定要拿他,與列位出氣,還汝等歡喜歸天。」眾神道:「既有此意,不須久停,快去,快去。」

好行者,說聲去,就縱觔斗雲,早至靈山。落下祥光,四方觀看,好去處:
\begin{quote}
靈峰疏傑,疊障清佳,仙岳頂巔摩碧漢。西天瞻巨鎮,形勢壓中華。元氣流通天地遠,威風飛徹滿臺花。時聞鐘磬音長,每聽經聲明朗。又見那青松之下優婆講,翠柏之間羅漢行。白鶴有情來鷲嶺,青鸞著意佇閑亭。玄猴對對擎仙果,壽鹿雙雙獻紫英。幽鳥聲頻如訴語,奇花色絢不知名。回巒盤繞重重顧,古道彎環處處平。正是清虛靈秀地,莊嚴大覺佛家風。
\end{quote}

那行者正然點看山景,忽聽得有人叫道:「孫悟空,從那裡來?往何處去?」急回頭看,原來是比丘尼尊者。大聖作禮道:「正有一事,欲見如來。」比丘尼道:「你這個頑皮。既然要見如來,怎麼不登寶剎,且在這裡看山?」行者道:「初來貴地,故此大膽。」比丘尼道:「你快跟我來也。」這行者緊隨至雷音寺山門下,又見那八大金剛雄糾糾的,兩邊擋住。比丘尼道:「悟空,暫候片時,等我與你奏上去來。」行者只得住立門外。那比丘尼至佛前合掌道:「孫悟空有事,要見如來。」如來傳旨令入,金剛才閃路放行。

行者低頭禮拜畢,如來問道:「悟空,前聞得觀音尊者解脫汝身,皈依釋教,保唐僧來此求經,你怎麼獨自到此?有何事故?」行者頓首道:「上告我佛。弟子自秉迦持,與唐朝師父西來,行至金山金洞,遇著一個惡魔頭,名喚兕大王,神通廣大,把師父與師弟等攝入洞中。弟子向伊求取,沒好意,兩家比迸,被他將一個白森森的圈子,搶了我的鐵棒。我恐他是天將思凡,急上界查勘不出。蒙玉帝差遣李天王父子助援,又被他搶了太子的六般兵器。及請火德星君放火燒他,又被他將火具搶去。又請水德星君放水渰他,一毫又渰他不著。弟子費若干精神氣力,將那鐵棒等物偷出,復去索戰,又被他將前物依然套去,無法收降。因此特告我佛,望垂慈與弟子看看,果然是何物出身。我好去拿他家屬四鄰,擒此魔頭,救我師父,合拱虔誠,拜求正果。」

如來聽說,將慧眼遙觀,早已知識。對行者道:「那怪物我雖知之,但不可與你說。你這猴兒口敞,一傳道是我說他,他就不與你鬥,定要嚷上靈山,反遺禍於我也。我這裡著法力助你擒他去罷。」行者再拜稱謝道:「如來助我甚麼法力?」如來即令十八尊羅漢開寶庫,取十八粒金丹砂,與悟空助力。行者道:「金丹砂卻如何?」如來道:「你去洞外,叫那妖魔比試。演他出來,卻教羅漢放砂,陷住他,使他動不得身,拔不得腳,憑你揪打便了。」行者笑道:「妙妙妙,趁早去來。」

那羅漢不敢遲延,即取金丹砂出門。行者又謝了如來。一路查看,止有十六尊羅漢,行者嚷道:「這是那個去處,卻賣放人。」眾羅漢道:「那個賣放?」行者道:「原差十八尊,今怎麼只得十六尊?」說不了,裡邊走出降龍、伏虎二尊,上前道:「悟空,怎麼就這等放刁?我兩個在後聽如來吩咐話的。」行者道:「忒賣法,忒賣法。才自若嚷遲了些兒,你敢就不出來了。」眾羅漢笑呵呵駕起祥雲。

不多時,到了金山界。那李天王見了,帥眾相迎,備言前事。羅漢道:「不必絮繁,快去叫他出來。」這大聖捻著拳頭,來於洞口,罵道:「腯潑怪物,快出來與你孫外公見個上下。」那小妖又飛跑去報。魔王怒道:「這賊猴又不知請誰來猖獗也。」小妖道:「更無甚將,止他一人。」魔王道:「那根棒子已被我收來,怎麼卻又一人到此?敢是又要走拳?」隨帶了寶貝,綽槍在手,叫小妖搬開石塊,跳出門來,罵道:「賊猴,你幾番家不得便宜,就該迴避,如何又來吆喝?」行者道:「這潑魔不識好歹!若要你外公不來,除非你服了降,陪了禮,送出我師父、師弟,我就饒你。」那怪道:「你那三個和尚已被我洗淨了,不久便要宰殺,你還不識起倒?去了罷。」

行者聽說「宰殺」二字,扢蹬蹬腮邊火發,按不住心頭之怒,丟了架子,掄著拳,斜行抅步,望妖魔使個掛面;那怪展長槍,劈手相迎。行者左跳右跳,哄那妖魔;妖魔不知是計,趕離洞口南來。行者即招呼羅漢把金丹砂望妖魔一齊拋下。好砂!正是那:
\begin{quote}
似霧如煙初散漫,紛紛靄靄下天涯。白茫茫,到處迷人眼;昏漠漠,飛時找路差。打柴的樵子失了伴,採藥的仙童不見家。細細輕飄如麥麵,粗粗翻復似芝麻。世界朦朧山頂暗,長空迷沒太陽遮。不比囂塵隨駿馬,難言輕軟襯香車。此砂本是無情物,蓋地遮天把怪拿。只為妖魔侵正道,阿羅奉法逞豪華。手中就有明珠現,等時刮得眼生花。
\end{quote}

那妖魔見飛砂迷目,把頭低了一低,足下就有三尺餘深。慌得他將身一縱,跳在浮上一層。未曾立得穩,須臾,又有二尺餘深。那怪急了,拔出腳來,即忙取圈子,往上一撇,叫聲:「著!」唿喇的一下,把十八粒金丹砂又盡套去,拽回步,徑歸本洞。

那羅漢一個個空手停雲。行者近前問道:「眾羅漢,怎麼不下砂了?」羅漢道:「適才響了一聲,金丹砂就不見矣。」行者笑道:「又是那話兒套將去了。」天王等眾道:「這般難伏啊,卻怎麼捉得他?何日歸天,何顏見帝也?」旁有降龍、伏虎二羅漢對行者道:「悟空,你曉得我兩個出門遲滯何也?」行者道:「老孫只怪你躲避不來,卻不知有甚話說。」羅漢道:「如來吩咐我兩個說:『那妖魔神通廣大,如失了金丹砂,就教孫悟空上離恨天兜率宮太上老君處尋他的蹤跡,庶幾可一鼓而擒也。』」行者聞言道:「可恨,可恨!如來卻也閃賺老孫。當時就該對我說了,卻不免教汝等遠涉。」李天王道:「既是如來有此明示,大聖就當早起。」

好行者,說聲去,就縱一道觔斗雲,直入南天門裡。時有四大元帥擎拳拱手道:「擒怪事如何?」行者且行且答道:「未哩,未哩。如今有處尋根去也。」四將不敢留阻,讓他進了天門。不上靈霄殿,不入斗牛宮,徑至三十三天之外離恨天兜率宮前,見兩仙童侍立,他也不通姓名,一直徑走。慌得兩童扯住道:「你是何人?往何處去?」行者才說:「我是齊天大聖,欲尋李老君哩。」仙童道:「你怎這樣粗魯?且住下,讓我們通報。」行者那容分說,喝了一聲,往裡徑走。忽見老君自內而出,撞個滿懷。行者躬身唱個喏道:「老官,一向少看。」老君笑道:「這猴兒不去取經,卻來我處何幹?」行者道:「取經取經,晝夜無停。有些阻礙,到此行行。」老君道:「西天路阻,與我何干?」行者道:「西天西天,你且休言。尋著蹤跡,與你纏纏。」老君道:「我這裡乃是無上仙宮,有甚蹤跡可尋?」

行者入裡,眼不轉睛,東張西看。走過幾層廊宇,忽見那牛欄邊一個童兒盹睡,青牛不在欄中。行者道:「老官,走了牛也,走了牛也。」老君大驚道:「這孽畜幾時走了?」正嚷間,那童兒方醒,跪於當面道:「爺爺,弟子睡著,不知是幾時走的。」老君罵道:「你這廝如何盹睡?」童兒叩頭道:「弟子在丹房裡拾得一粒丹,當時吃了,就在此睡著。」老君道:「想是前日煉的七返火丹,吊了一粒,被這廝拾吃了。那丹吃一粒,該睡七日哩。那孽畜因你睡著,無人看管,遂乘機走下界去,今亦是七日矣。」

即查可曾偷甚寶貝。行者道:「無甚寶貝,只見他有一個圈子,甚是利害。」老君急查看時,諸般俱在,止不見了金剛琢。老君道:「這孽畜偷了我金剛琢去了!」行者道:「原來是這件寶貝。當時打著老孫的是他。如今在下界張狂,不知套了我等多少物件。」老君道:「這孽畜在甚地方?」行者道:「現住金山金洞。他捉了我唐僧進去,搶了我金箍棒。請天兵相助,又搶了太子的神兵。及請火德星君,又搶了他的火具。惟水伯雖不能渰死他,倒還不曾搶他物件。至請如來著羅漢下砂,又將金丹砂搶去。似你這老官縱放怪物,搶奪傷人,該當何罪?」老君道:「我那金剛琢,乃是我過函關化胡之器,自幼煉成之寶。憑你甚麼兵器、水火,俱莫能近他。若偷去我的芭蕉扇兒,連我也不能奈他何矣。」

大聖才歡歡喜喜,隨著老君。老君執了芭蕉扇,駕著祥雲同行,出了仙宮。南天門外,低下雲頭,徑至金山界。見了十八尊羅漢、雷公、水伯、火德、李天王父子,備言前事一遍。老君道:「孫悟空還去誘他出來,我好收他。」

這行者跳下峰頭,又高聲罵道:「腯潑孽畜!趁早出來受死!」那小妖又去報知。老魔道:「這賊猴又不知請誰來也。」急綽槍帶寶,迎出門來。行者罵道:「你這潑魔,今番坐定是死了!不要走,吃吾一掌。」急縱身跳個滿懷,劈臉打了一個耳括子,回頭就跑。那魔掄槍就趕。只聽得高峰上叫道:「那牛兒還不歸家,更待何日?」那魔擡頭,看見是太上老君,就諕得心驚膽戰道:「這賊猴真個是個地裡鬼,卻怎麼就訪得我的主公來也?」

老君念個咒語,將扇子搧了一下,那怪將圈子丟來,被老君一把接住。又一搧,那怪物力軟筋麻,現了本相,原來是一隻青牛。老君將金鋼琢吹口仙氣,穿了那怪的鼻子,解下勒袍帶,繫於琢上,牽在手中。至今留下個拴牛鼻的拘兒,又名賓郎,職此之謂。老君辭了眾神,跨上青牛背上,駕彩雲,徑歸兜率院;縛妖怪,高昇離恨天。

孫大聖才同天王等眾打入洞裡,把那百十個小妖盡皆打死,各取兵器。謝了天王父子回天,雷公入府,火德歸宮,水伯回河,羅漢向西。然後才解放唐僧、八戒、沙僧,拿了鐵棒。他三人又謝了行者,收拾馬匹、行裝,師徒們離洞,找大路方走。

正走間,只聽得路傍叫:「唐聖僧,吃了齋飯去。」那長老心驚。

畢竟不知是甚麼人叫喚,且聽下回分解。
