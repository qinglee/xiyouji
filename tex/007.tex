
\chapter{八卦爐中逃大聖 五行山下定心猿}

\begin{quote}
富貴功名,前緣分定,為人切莫欺心。正大光明,忠良善果彌深。些些狂妄天加譴,眼前不遇待時臨。問東君,因甚如今禍害相侵?只為心高圖罔極,不分上下亂規箴。
\end{quote}

話表齊天大聖被眾天兵押去斬妖臺下,綁在降妖柱上,刀砍斧剁,槍刺劍刳,莫想傷及其身。南斗星奮令火部眾神放火煨燒,亦不能燒著。又著雷部眾神以雷屑釘打,越發不能傷損一毫。那大力鬼王與眾啟奏道:「萬歲,這大聖不知是何處學得這護身之法,臣等用刀砍斧剁,雷打火燒,一毫不能傷損,卻如之何?」玉帝聞言道:「這廝這等這等,如何處治?」太上老君即奏道:「那猴吃了蟠桃,飲了御酒,又盜了仙丹。我那五壺丹,有生有熟,被他都吃在肚裡。運用三昧火,鍛成一塊,所以渾做金鋼之軀,急不能傷。不若與老道領去,放在八卦爐中,以文武火鍛煉,煉出我的丹來,他身自為灰燼矣。」玉帝聞言,即教六丁、六甲將他解下,付與老君。老君領旨去訖。一壁廂宣二郎顯聖,賞賜金花百朵、御酒百瓶、還丹百粒、異寶明珠、錦繡等件,教與義兄弟分享。真君謝恩,回灌江口不題。

那老君到兜率宮,將大聖解去繩索,放了穿琵琶骨之器,推入八卦爐中,命看爐的道人、架火的童子,將火搧起鍛煉。原來那爐是乾、坎、艮、震、巽、離、坤、兌八卦。他即將身鑽在巽宮位下。巽乃風也,有風則無火。只是風攪得煙來,把一雙眼火煼紅了,弄做個老害病眼,故喚作「火眼金睛」。

真個光陰迅速,不覺七七四十九日,老君的火候俱全。忽一日,開爐取丹。那大聖雙手侮著眼,正自揉搓流涕,只聽得爐頭聲響。猛睜睛看見光明,他就忍不住,將身一縱,跳出丹爐,唿喇一聲,蹬倒八卦爐,往外就走。慌得那架火、看爐與丁甲一班人來扯,被他一個個都放倒,好似癲癇的白額虎,風狂的獨角龍。老君趕上抓一把,被他一捽,捽了個倒栽蔥,脫身走了。即去耳中掣出如意棒,迎風幌一幌,碗來粗細,依然拿在手中,不分好歹,卻又大亂天宮,打得那九曜星閉門閉戶,四天王無影無形。好猴精,有詩為證。詩曰:
\begin{quote}
混元體正合先天,萬劫千番只自然。
渺渺無為渾太乙,如如不動號初玄。
爐中久煉非鉛汞,物外長生是本仙。
變化無窮還變化,三皈五戒總休言。
\end{quote}

又詩:
\begin{quote}
一點靈光徹太虛,那條拄杖亦如之。
或長或短隨人用,橫豎橫排任卷舒。
\end{quote}

又詩:
\begin{quote}
猿猴道體配人心,心即猿猴意思深。
大聖齊天非假論,官封弼馬豈知音。
馬猿合作心和意,緊縛牢拴莫外尋。
萬相歸真從一理,如來同契住雙林。
\end{quote}

這一番,那猴王不分上下,使鐵棒東打西敵,更無一神可擋,只打到通明殿裡,靈霄殿外。幸有佑聖真君的佐使王靈官執殿,他看大聖縱橫,掣金鞭近前擋住道:「潑猴何往?有吾在此,切莫猖狂。」這大聖不由分說,舉棒就打;那靈官鞭起相迎。兩個在靈霄殿前廝渾一處,好殺:
\begin{quote}
赤膽忠良名譽大,欺天誑上聲名壞。一低一好幸相持,豪傑英雄同賭賽。鐵棒兇,金鞭快,正直無私怎忍耐?這個是太乙雷聲應化尊,那個是齊天大聖猿猴怪。金鞭鐵棒兩家能,都是神宮仙器械。今日在靈霄寶殿弄威風,各展雄才真可愛。一個欺心要奪斗牛宮,一個竭力匡扶玄聖界。苦爭不讓顯神通,鞭棒往來無勝敗。
\end{quote}

他兩個鬥在一處,勝敗未分。早有佑聖真君又差將佐發文到雷府,調三十六員雷將齊來,把大聖圍在垓心,各騁兇惡鏖戰。那大聖全無一毫懼色,使一條如意棒,左遮右擋,後架前迎。一時見那眾雷將的刀槍劍戟、鞭簡撾鎚、鉞斧金瓜、旄鐮月鏟來的甚緊,他即搖身一變:變做三頭六臂;把如意棒幌一幌,變作三條;六隻手使開三條棒,好便似紡車兒一般,滴流流,在那垓心裡飛舞。眾雷神莫能相近。真個是:
\begin{quote}
圓陀陀,光灼灼,亙古常存人怎學?入火不能焚,入水何曾溺?光明一顆摩尼珠,劍戟刀槍傷不著。也能善,也能惡,眼前善惡憑他作。善時成佛與成仙,惡處披毛並帶角。無窮變化鬧天宮,雷將神兵不可捉。
\end{quote}

當時眾神把大聖攢在一處,卻不能近身,亂嚷亂鬥。早驚動玉帝,遂傳旨著遊奕靈官同翊聖真君上西方請佛老降伏。

那二聖得了旨,徑到靈山勝境雷音寶剎之前,對四金剛、八菩薩禮畢,即煩轉達。眾神隨至寶蓮臺下啟知,如來召請。二聖禮佛三匝,侍立臺下。如來問:「玉帝何事,煩二聖下臨?」二聖即啟道:「向時花果山產一猴,在那裡弄神通,聚眾猴攪亂世界。玉帝降招安旨,封為弼馬溫,他嫌官小反去。當遣李天王、哪吒太子擒拿未獲,復招安他,封做齊天大聖,先有官無祿。著他代管蟠桃園,他即偷桃;又走至瑤池,偷殽、偷酒,攪亂大會;仗酒又暗入兜率宮,偷老君仙丹,反出天宮。玉帝復遣十萬天兵,亦不能收伏。後觀世音舉二郎真君同他義兄弟追殺,他變化多端,虧老君拋金鋼琢打中,二郎方得拿住。解赴御前,即命斬之,刀砍斧剁,火燒雷打,俱不能傷。老君奏准領去,以火鍛煉。四十九日開鼎,他卻又跳出八卦爐,打退天丁,徑入通明殿裡,靈霄殿外。被佑聖真君的佐使王靈官擋住苦戰,又調三十六員雷將把他困在垓心,終不能相近。事在緊急,因此玉帝特請如來救駕。」如來聞說,即對眾菩薩道:「汝等在此穩坐法堂,休得亂了禪位,待我煉魔救駕去來。」

如來即喚阿儺、迦葉二尊者相隨,離了雷音,徑至靈霄門外。忽聽得喊聲振耳,乃三十六員雷將圍困著大聖哩。佛祖傳法旨:「教雷將停息干戈,放開營所,叫那大聖出來,等我問他有何法力。」眾將果退。大聖也收了法象,現出原身近前,怒氣昂昂,厲聲高叫道:「你是那方善士,敢來止住刀兵問我?」如來笑道:「我是西方極樂世界釋迦牟尼尊者。南無阿彌陀佛!今聞你猖狂村野,屢反天宮,不知是何方生長,何年得道,為何這等暴橫?」大聖道:「我本:
\begin{quote}
天地生成靈混仙,花果山中一老猿。
水簾洞裡為家業,拜友尋師悟太玄。
煉就長生多少法,學來變化廣無邊。
因在凡間嫌地窄,立心端要住瑤天。
靈霄寶殿非他久,歷代人王有分傳。
強者為尊該讓我,英雄只此敢爭先。」
\end{quote}

佛祖聽言,呵呵冷笑道:「你那廝乃是個猴子成精,焉敢欺心,要奪玉皇上帝尊位?他自幼修持,苦歷過一千七百五十劫。每劫該十二萬九千六百年,你算他該多少年數,方能享受此無極大道?你那個初世為人的畜生,如何出此大言?不當人子,不當人子,折了你的壽算。趁早皈依,切莫胡說。但恐遭了毒手,性命頃刻而休,可惜了你的本來面目。」大聖道:「他雖年幼修長,也不應久占在此。常言道:『皇帝輪流做,明年到我家。』只教他搬出去,將天宮讓與我,便罷了;若還不讓,定要攪攘,永不清平。」佛祖道:「你除了長生變化之法,再有何能,敢占天宮勝境?」大聖道:「我的手段多哩:我有七十二般變化,萬劫不老長生;會駕觔斗雲,一縱十萬八千里。如何坐不得天位?」佛祖道:「我與你打個賭賽:你若有本事,一觔斗打出我這右手掌中,算你贏,再不用動刀兵,苦爭戰,就請玉帝到西方居住,把天宮讓你;若不能打出手掌,你還下界為妖,再修幾劫,卻來爭吵。」那大聖聞言,暗笑道:「這如來十分好獃。我老孫一觔斗去十萬八千里,他那手掌方圓不滿一尺,如何跳不出去?」急發聲道:「既如此說,你可做得主張?」佛祖道:「做得,做得。」伸開右手,卻似個荷葉大小。

那大聖收了如意棒,抖擻神威,將身一縱,站在佛祖手心裡,卻道聲:「我出去也。」你看他一路雲光,無形無影去了。佛祖慧眼觀看,見那猴王風車子一般相似不住,只管前進。大聖行時,忽見有五根肉紅柱子,撐著一股青氣。他道:「此間乃盡頭路了。這番回去,如來作證,靈霄宮定是我坐也。」又思量說:「且住,等我留下些記號,方好與如來說話。」拔下一根毫毛,吹口仙氣,叫:「變!」變作一管濃墨雙毫筆,在那中間柱子上寫一行大字云:「齊天大聖,到此一遊。」寫畢,收了毫毛。又不莊尊,卻在第一根柱子根下撒了一泡猴尿。翻轉觔斗雲,徑回本處,站在如來掌內道:「我已去,今來了。你教玉帝讓天宮與我。」

如來罵道:「我把你這個尿精猴子,你正好不曾離了我掌哩。」大聖道:「你是不知。我去到天盡頭,見五根肉紅柱,撐著一股青氣,我留個記在那裡,你敢和我同去看麼?」如來道:「不消去,你只自低頭看看。」那大聖睜圓火眼金睛,低頭看時,原來佛祖右手中指寫著「齊天大聖,到此一遊」。大指丫裡,還有些猴尿臊氣。大聖吃了一驚道:「有這等事?有這等事?我將此字寫在撐天柱子上,如何卻在他手指上?莫非有個未卜先知的法術?我決不信,不信。等我再去來。」

好大聖,急縱身又要跳出。被佛祖翻掌一撲,把這猴王推出西天門外,將五指化作金、木、水、火、土五座聯山,喚名「五行山」,輕輕的把他壓住。眾雷神與阿儺、迦葉一個個合掌稱揚道:「善哉,善哉!
\begin{quote}
當年卵化學為人,立志修行果道真。
萬劫無移居勝境,一朝有變散精神。
欺天罔上思高位,凌聖偷丹亂大倫。
惡貫滿盈今有報,不知何日得翻身。」
\end{quote}

如來佛祖殄滅了妖猴,即喚阿儺、迦葉同轉西方極樂世界。時有天蓬、天佑急出靈霄寶殿道:「請如來少待,我主大駕來也。」佛祖聞言,回首瞻仰。須臾,果見八景鸞輿,九光寶蓋,聲奏玄歌妙樂,詠哦無量神章,散寶花,噴真香,直至佛前謝曰:「多蒙大法收殄妖邪,望如來少停一日,請諸仙做一會筵奉謝。」如來不敢違悖,即合掌謝道:「老僧承大天尊宣命來此,有何法力?還是天尊與眾神洪福。敢勞致謝?」玉帝傳旨,即著雷部眾神,分頭請三清、四御、五老、六司、七元、八極、九曜、十都、千真、萬聖來此赴會,同謝佛恩。又命四大天師、九天仙女,大開玉京金闕、太玄寶宮、洞陽玉館,請如來高座七寶靈臺,調設各班坐位,安排龍肝鳳髓,玉液蟠桃。

不一時,那玉清元始天尊、上清靈寶天尊、太清道德天尊、五炁真君、五斗星君、三官四聖、九曜真君、左輔、右弼、天王、哪吒,玄虛一應靈通,對對旌旗,雙雙幡蓋,都捧著明珠異寶,壽果奇花,向佛前拜獻曰:「感如來無量法力,收伏妖猴。蒙大天尊設宴,呼喚我等皆來陳謝。請如來將此會立一名如何?」如來領眾神之託曰:「今欲立名,可作個安天大會。」各仙老異口同聲,俱道:「好個『安天大會』!好個『安天大會』!」言訖,各坐座位,走斝傳觴,簪花鼓瑟,果好會也。有詩為證。詩曰:
\begin{quote}
宴設蟠桃猴攪亂,安天大會勝蟠桃。
龍旗鸞輅祥光藹,寶節幢幡瑞氣飄。
仙樂玄歌音韻美,鳳簫玉管響聲高。
瓊香繚繞群仙集,宇宙清平賀聖朝。
\end{quote}

眾皆暢然喜會,只見王母娘娘引一班仙子、仙娥、美姬、美女飄飄蕩蕩舞向佛前,施禮曰:「前被妖猴攪亂蟠桃一會,請眾仙眾佛俱成功。今蒙如來大法鍊鎖頑猴,喜慶『安天大會』,無物可謝,今是我淨手親摘大株蟠桃數顆奉獻。」真個是:
\begin{quote}
半紅半綠噴甘香,艷麗仙根萬載長。
堪笑武陵源上種,爭如天府更奇強。
紫紋嬌嫩寰中少,緗核清甜世莫雙。
延壽延年能易體,有緣食者自非常。
\end{quote}

佛祖合掌向王母謝訖。王母又著仙姬、仙子唱的唱,舞的舞。滿會群仙又皆賞讚。正是:
\begin{quote}
縹緲天香滿座,繽紛仙蕊仙花。
玉京金闕大榮華。異品奇珍無價。
對對與天齊壽,雙雙萬劫增加。
桑田滄海任更差。他自無驚無訝。
\end{quote}

王母正著仙姬、仙子歌舞,觥籌交錯,不多時,忽又聞得:
\begin{quote}
一陣異香來鼻噢,驚動滿堂星與宿。
天仙佛祖把杯停,各各擡頭迎目候。
霄漢中間現老人,手捧靈芝飛藹繡。
葫蘆藏蓄萬年丹,寶籙名書千紀壽。
洞裡乾坤任自由,壺中日月隨成就。
遨遊四海樂清閑,散淡十洲容輻輳。
曾赴蟠桃醉幾遭,醒時明月還依舊。
長頭大耳短身軀,南極之方稱老壽。
\end{quote}

壽星又到。見玉帝禮畢,又見如來,申謝曰:「始聞那妖猴被老君引至兜率宮鍛煉,以為必致平安,不期他又反出。幸如來善伏此怪,設宴奉謝,故此聞風而來。更無他物可獻,特具紫芝瑤草、碧藕金丹奉上。」詩曰:
\begin{quote}
碧藕金丹奉釋迦,如來萬壽若恆沙。
清平永樂三乘錦,康泰長生九品花。
無相門中真法主,色空天上是仙家。
乾坤大地皆稱祖,丈六金身福壽賒。
\end{quote}

如來忻然領謝。壽星得座,依然走斝傳觴。只見赤腳大仙又至,向玉帝前頫顖禮畢,又對佛祖謝道:「深感法力,降伏妖猴。無物可以表敬,特具交梨二顆、火棗數枚奉獻。」詩曰:
\begin{quote}
大仙赤腳棗梨香,敬獻彌陀壽算長。
七寶蓮臺山樣穩,千金花座錦般粧。
壽同天地言非謬,福比洪波話豈狂。
福壽如期真個是,清閑極樂那西方。
\end{quote}

如來又稱謝了,叫阿儺、迦葉將各所獻之物,一一收起,方向玉帝前謝宴。眾各酩酊。只見個巡視靈官來報道:「那大聖伸出頭來了。」佛祖道:「不妨,不妨。」袖中只取出一張帖子,上有六個金字:「唵嘛呢叭吽」。遞與阿儺,叫貼在那山頂上。這尊者即領帖子,拿出天門,到那五行山頂上,緊緊的貼在一塊四方石上,那座山即生根合縫。可運用呼吸之氣,手兒爬出,可以搖掙搖掙。阿儺回報道:「已將帖子貼了。」

如來即辭了玉帝眾神,與二尊者出天門之外。又發一個慈悲心,念動真言咒語,將五行山召一尊土地神祇,會同五方揭諦,居住此山監押。但他饑時,與他鐵丸子吃;渴時,與他溶化的銅汁飲。待他災愆滿日,自有人救他。正是:
\begin{quote}
妖猴大膽反天宮,卻被如來伏手降。
渴飲溶銅捱歲月,饑餐鐵彈度時光。
天災苦困遭磨折,人事淒涼喜命長。
若得英雄重展掙,他年奉佛上西方。
\end{quote}

又詩曰:
\begin{quote}
伏逞豪強大勢興,降龍伏虎弄乖能。
偷桃偷酒遊天府,受籙承恩在玉京。
惡貫滿盈身受困,善根不絕氣還昇。
果然脫得如來手,且待唐朝出聖僧。
\end{quote}

畢竟不知向後何年何月方滿災殃,且聽下回分解。
