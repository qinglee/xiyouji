
\chapter{觀音院僧謀寶貝 黑風山怪竊袈裟}

卻說他師徒兩個策馬前來,直至山門首觀看,果然是一座寺院。但見那:
\begin{quote}
層層殿閣,疊疊廊房。三山門外,巍巍萬道彩雲遮;五福堂前,豔豔千條紅霧遶。兩路松篁,一林檜柏。兩路松篁,無年無紀自清幽;一林檜柏,有色有顏隨傲麗。又見那鐘鼓樓高,浮屠塔峻。安禪僧定性,啼樹鳥音閑。寂寞無塵真寂寞,清虛有道果清虛。
\end{quote}

詩曰:
\begin{quote}
上剎祇園隱翠窩,招提勝景賽娑婆。
果然淨土人間少,天下名山僧占多。
\end{quote}

長老下了馬,行者歇了擔,正欲進門,只見那門裡走出一眾僧來。你看他怎生模樣:
\begin{quote}
頭戴左笄帽,身穿無垢衣。
銅環雙墜耳,絹帶束腰圍。
草履行來穩,木魚手內提。
口中常作念,般若總皈依。
\end{quote}

三藏見了,侍立門傍,道個問訊。那和尚連忙答禮,笑道:「失瞻。」問:「是那裡來的?請入方丈獻茶。」三藏道:「我弟子乃東土欽差,上雷音寺拜佛求經。至此處天色將晚,欲借上剎一宵。」那和尚道:「請進裡坐,請進裡坐。」三藏方喚行者牽馬進來。那和尚忽見行者相貌,有些害怕,便問:「那牽馬的是個甚麼東西?」三藏道:「悄言,悄言。他的性急,若聽見你說是甚麼東西,他就惱了。他是我的徒弟。」那和尚打了個寒噤,咬著指頭道:「這般一個醜頭怪腦的,好招他做徒弟?」三藏道:「你看不出來哩,醜自醜,甚是有用。」

那和尚只得同三藏與行者進了山門。山門裡,又見那正殿上書四個大字,是「觀音禪院」。三藏又大喜道:「弟子屢感菩薩聖恩,未及叩謝。今遇禪院,就如見菩薩一般,甚好拜謝。」那和尚聞言,即命道人開了殿門,請三藏朝拜。那行者拴了馬,丟了行李,同三藏上殿。三藏展背舒身,鋪胸納地,望金像叩頭。那和尚便去打鼓。行者就去撞鐘。三藏俯伏臺前,傾心禱祝。祝拜已畢,那和尚住了鼓,行者還只管撞鐘不歇,或緊或慢,撞了許久。那道人道:「拜已畢了,還撞鐘怎麼?」行者方丟了鐘杵,笑道:「你那裡曉得!我這是『做一日和尚撞一日鐘』的。」此時卻驚動那寺裡大小僧人、上下房長老,聽得鐘聲亂響,一齊擁出道:「那個野人在這裡亂敲鐘鼓?」行者跳將出來,咄的一聲道:「是你孫外公撞了耍子的。」那些和尚一見了,諕得跌跌滾滾,都爬在地下道:「雷公爺爺!」行者道:「雷公是我的重孫兒哩。起來,起來,不要怕,我們是東土大唐來的老爺。」眾僧方才禮拜。見了三藏,都才放心不怕。內有本寺院主請道:「老爺們到後方丈中奉茶。」遂而解韁牽馬,擡了行李,轉過正殿,徑入後房,序了坐次。

那院主獻了茶,又安排齋供。天光尚早,三藏稱謝未畢,只見那後面有兩個小童,攙著一個老僧出來。看他怎生打扮:
\begin{quote}
頭上戴一頂毘盧方帽,貓睛石的寶頂光輝;身上穿一領錦絨褊衫,翡翠毛的金邊晃亮。一對僧鞋攢八寶,一根拄杖嵌雲星。滿面皺痕,好似驪山老母;一雙昏眼,卻如東海龍君。口不關風因齒落,腰駝背屈為筋攣。
\end{quote}

眾僧道:「師祖來了。」三藏躬身施禮迎接道:「老院主,弟子拜揖。」那老僧還了禮,又各敘坐。老僧道:「適間小的們說,東土唐朝來的老爺,我才出來奉見。」三藏道:「輕造寶山,不知好歹,恕罪,恕罪。」老僧道:「不敢,不敢。」因問:「老爺,東土到此,有多少路程?」三藏道:「出長安邊界,有五千餘里。過兩界山,收了一眾小徒,一路來,行過西番哈咇國,經兩個月,又有五六千里,才到了貴處。」老僧道:「也有萬里之遙了。我弟子虛度一生,山門也不曾出去,誠所謂『坐井觀天』,樗朽之輩。」三藏又問:「老院主高壽幾何?」老僧道:「痴長二百七十歲了。」行者聽見道:「這還是我萬代孫兒哩。」三藏瞅了他一眼道:「謹言,莫要不識高低,衝撞人。」那和尚便問:「老爺,你有多少年紀了?」行者道:「不敢說。」

那老僧也只當一句瘋話,便不介意,也不再問,只叫獻茶。有一個小幸童,拿出一個羊脂玉的盤兒,有三個法藍鑲金的茶鍾。又一童,提一把白銅壺兒,斟了三杯香茶。真個是色欺榴蕊豔,味勝桂花香。三藏見了,誇愛不盡道:「好物件,好物件,真是美食美器。」那老僧道:「污眼,污眼。老爺乃天朝上國,廣覽奇珍,似這般器具,何足過獎?老爺自上邦來,可有甚麼寶貝,借與弟子一觀?」三藏道:「可憐,我那東土無甚寶貝;就有時,路程遙遠,也不能帶得。」行者在傍道:「師父,我前日在包袱裡,曾見那領袈裟,不是件寶貝?拿與他看看如何?」眾僧聽說袈裟,一個個冷笑。行者道:「你笑怎的?」院主道:「老爺才說袈裟是件寶貝,言實可笑。若說袈裟,似我等輩者,不止二三十件;若論我師祖,在此處做了二百五六十年和尚,足有七八百件。」叫:「拿出來看看。」那老和尚也是他一時賣弄,便叫道人開庫房,頭陀擡櫃子,就擡出十二櫃,放在天井中,開了鎖。兩邊設下衣架,四圍牽了繩子,將袈裟一件件抖開掛起,請三藏觀看。果然是滿堂綺繡,四壁綾羅。

行者一一觀之,都是些穿花納錦,刺繡銷金之物,笑道:「好,好,好。收起,收起。把我們的也取出來看看。」三藏把行者扯住,悄悄的道:「徒弟,莫要與人鬥富。你我是單身在外,只恐有錯。」行者道:「看看袈裟,有何差錯?」三藏道:「你不曾理會得。古人有云:『珍奇玩好之物,不可使見貪婪奸偽之人。』倘若一經入目,必動其心;既動其心,必生其計。汝是個畏禍的,索之而必應其求,可也;不然,則殞身滅命,皆起於此,事不小矣。」行者道:「放心,放心,都在老孫身上。」你看他不由分說,急急的走了去,把個包袱解開,早有霞光迸迸,尚有兩層油紙裹定。去了紙,取出袈裟,抖開時,紅光滿室,彩氣盈庭。眾僧見了,無一個不心歡口讚,真個好袈裟。上頭有:
\begin{quote}
千般巧妙明珠墜,萬樣稀奇佛寶攢。
上下龍鬚鋪綵綺,兜羅四面錦沿邊。
體掛魍魎從此滅,身披魑魅入黃泉。
托化天仙親手製,不是真僧不敢穿。
\end{quote}

那老和尚見了這般寶貝,果然動了奸心,走上前,對三藏跪下,眼中垂淚道:「我弟子真是沒緣。」三藏攙起道:「老院師有何話說?」他道:「老爺這件寶貝方才展開,天色晚了,奈何眼目昏花,不能看得明白,豈不是無緣?」三藏教:「掌上燈來,讓你再看。」那老僧道:「爺爺的寶貝已是光亮,再點了燈,一發晃眼,莫想看得仔細。」行者道:「你要怎的看才好?」老僧道:「老爺若是寬恩放心,教弟子拿到後房,細細的看一夜,明早送還老爺西去,不知尊意何如?」三藏聽說,吃了一驚,埋怨行者道:「都是你,都是你。」行者笑道:「怕他怎的?等我包起來,教他拿了去看。但有疏虞,盡是老孫管整。」那三藏阻當不住,他把袈裟遞與老僧道:「憑你看去。只是明早照舊還我,不得損污些須。」老僧喜喜歡歡,著幸童將袈裟拿進去。卻吩咐眾僧,將前面禪堂掃淨,取兩張籐床,安設鋪蓋,請二位老爺安歇;一壁廂又教安排明早齋送行。遂而各散,師徒們關了禪堂,睡下不題。

卻說那和尚把袈裟騙到手,拿在後房燈下,對袈裟號咷痛哭。慌得那本寺僧不敢先睡。小幸童也不知為何,卻去報與眾僧道:「公公哭到二更時候,還不歇聲。」有兩個徒孫是他心愛之人,上前問道:「師公,你哭怎的?」老僧道:「我哭無緣,看不得唐僧寶貝。」小和尚道:「公公年紀高大,發過了。他的袈裟放在你面前,你只消解開看便罷了,何須痛哭?」老僧道:「看的不長久。我今年二百七十歲,空掙了幾百件袈裟。怎麼得有他這一件?怎麼得做個唐僧?」小和尚道:「師公差了。唐僧乃是離鄉背井的一個行腳僧。你這等年高享用,也夠了,倒要像他做行腳僧,何也?」老僧道:「我雖是坐家自在,樂乎晚景,卻不得他這袈裟穿穿。若教我穿得一日兒,就死也閉眼,也是我來陽世間為僧一場。」眾僧道:「好沒正經。你要穿他的,有何難處?我們明日留他住一日,你就穿他一日;留他住十日,你就穿他十日;便罷了,何苦這般痛哭?」老僧道:「縱然留他住了年載,也只穿得年載,到底也不得氣長。他要去時,只得與他去,怎生留得長遠?」

正說話處,有一個小和尚,名喚廣智,出頭道:「公公要得長遠,也容易。」老僧聞言,就歡喜起來道:「我兒,你有甚麼高見?」廣智道:「那唐僧兩個是走路的人,辛苦之甚,如今已睡著了。我們想幾個有力量的,拿了槍刀,打開禪堂,將他殺了,把屍首埋在後園,只我一家知道,卻又謀了他的白馬、行囊,卻把那袈裟留下,以為傳家之寶,豈非子孫長久之計耶?」老和尚見說,滿心歡喜,卻才揩了眼淚道:「好,好,好,此計絕妙。」即便收拾槍刀。

內中又有一個小和尚,名喚廣謀,就是那廣智的師弟,上前來道:「此計不妙。若要殺他,須要看看動靜。那個白臉的似易,那個毛臉的似難,萬一殺他不得,卻不反招己禍?我有一個不動刀槍之法,不知你尊意如何?」老僧道:「我兒,你有何法?」廣謀道:「依小孫之見,如今喚聚東山大小房頭,每人要乾柴一束,捨了那三間禪堂,放起火來,教他欲走無門,連馬一火焚之。就是山前山後人家看見,只說是他自不小心,走了火,將我禪堂都燒了。那兩個和尚,卻不都燒死?又好掩人耳目。袈裟豈不是我們傳家之寶?」那些和尚聞言,無不歡喜,都道:「強,強,強,此計更妙,更妙。」遂教各房頭搬柴來。唉!這一計,正是:弄得個高壽老僧該命盡,觀音禪院化為塵。原來他那寺裡有七八十個房頭,大小有二百餘眾。當夜一擁搬柴,把個禪堂前前後後,四面圍繞不通,安排放火不題。

卻說三藏師徒安歇已定。那行者卻是個靈猴,雖然睡下,只是存神煉氣,朦朧著醒眼。忽聽得外面不住的人走,查查的柴響風生。他心疑惑道:「此時夜靜,如何有人行得腳步之聲?莫敢是賊盜,謀害我們的?」他就一骨魯跳起,欲要開門出看,又恐驚醒師父。你看他弄個精神,搖身一變,變做一個蜜蜂兒。真個是:
\begin{quote}
口甜尾毒,腰細身輕。穿花度柳飛如箭,粘絮尋香似落星。小小微軀能負重,囂囂薄翅會乘風。卻自椽棱下,鑽出看分明。
\end{quote}

只見那眾僧們搬柴運草,已圍住禪堂放火哩。行者暗笑道:「果依我師父之言,他要害我們性命,謀我的袈裟,故起這等毒心。我待要拿棍打他啊,可憐又不禁打,一頓棍都打死了,師父又怪我行兇。罷,罷,罷,與他個順手牽羊,將計就計,教他住不成罷!」

好行者,一觔斗跳上南天門裡。諕得個龐、劉、苟、畢躬身,馬、趙、溫、關控背,俱道:「不好了,不好了!那鬧天宮的主子又來了。」行者搖著手道:「列位免禮,休驚。我來尋廣目天王的。」說不了,卻遇天王早到,迎著行者道:「久闊,久闊。前聞得觀音菩薩來見玉帝,借了四值功曹、六丁六甲並揭諦等,保護唐僧往西天取經去,說你與他做了徒弟,今日怎麼得閑到此?」行者道:「且休敘闊。唐僧路遇歹人,放火燒他,事在萬分緊急,特來尋你借辟火罩兒,救他一救。快些拿來使使,即刻返上。」天王道:「你差了。既是歹人放火,只該借水救他,如何要辟火罩?」行者道:「你那裡曉得就裡。借水救之,卻燒不起來,倒相應了他;只是借此罩,護住了唐僧無傷,其餘管他,盡他燒去。快些,快些,此時恐已無及,莫誤了我下邊幹事。」那天王笑道:「這猴子還是這等起不善之心,只顧了自家,就不管別人。」行者道:「快著,快著,莫要調嘴,害了大事。」那天王不敢不借,遂將罩兒遞與行者。

行者拿了,按著雲頭,徑到禪堂房脊上,罩住了唐僧與白馬、行李。他卻去那後面老和尚住的方丈房上頭坐著,保護那袈裟。看那些人放起火來,他轉捻訣念咒,望巽地上吸一口氣吹將去,一陣風起,把那火轉吹得烘烘亂發。好火,好火!但見:
\begin{quote}
黑煙漠漠,紅燄騰騰。黑煙漠漠,長空不見一天星;紅燄騰騰,大地有光千里赤。起初時,灼灼金蛇;次後來,威威血馬。南方三炁逞英雄,回祿大神施法力。燥乾柴燒烈火性,說甚麼燧人鑽木;熱油門前飄彩燄,賽過了老祖開爐。正是那無情火發,怎禁這有意行兇。不去弭災,反行助虐。風隨火勢,燄飛有千丈餘高;火逞風威,灰迸上九霄雲外。乒乒乓乓,好便似殘年爆竹;潑潑喇喇,卻就如軍中炮聲。燒得那當場佛像莫能逃,東院伽藍無處躲。勝如赤壁夜鏖兵,賽過阿房宮內火。
\end{quote}

這正是星星之火,能燒萬頃之田。須臾間,風狂火盛,把一座觀音院,處處通紅。你看那眾和尚,搬箱擡籠,搶桌端鍋,滿院裡叫苦連天。孫行者護住了後邊方丈,辟火罩罩住了前面禪堂,其餘前後火光大發,真個是照天紅燄輝煌,透壁金光照耀。

不期火起之時,驚動了一山獸怪。這觀音院正南二十里遠近,有座黑風山,山中有一個黑風洞,洞中有一個妖精,正在睡醒翻身。只見那窗間透亮,只道是天明。起來看時,卻是正北下的火光晃亮。妖精大驚道:「呀!這必是觀音院裡失了火。這些和尚好不小心。我看時,與他救一救來。」好妖精,縱起雲頭,即至煙火之下,果然沖天之火,前面殿宇皆空,兩廊煙火方灼。他大拽步,撞將進去,正呼喚叫取水來,只見那後房無火,房脊上有一人放風。他卻情知如此,急入裡面看時,見那方丈中間有些霞光彩氣,臺案上有一個青氈包袱。他解開一看,見是一領錦襴袈裟,乃佛門之異寶。正是財動人心,他也不救火,他也不叫水,拿著那袈裟,趁鬨打劫,拽回雲步,徑轉東山而去。

那場火只燒到五更天明,方才滅息。你看那眾僧們赤赤精精,啼啼哭哭,都去那灰內尋銅鐵,撥腐炭,撲金銀。有的在牆筐裡,苫搭窩棚;有的赤壁根頭,支鍋造飯。叫冤叫屈,亂嚷亂鬧不題。

卻說行者取了辟火罩,一觔斗送上南天門,交與廣目天王道:「謝借,謝借。」天王收了道:「大聖至誠了。我正愁你不還我的寶貝,無處尋討,且喜就送來也。」行者道:「老孫可是那當面騙物之人?這叫做『好借好還,再借不難』。」天王道:「許久不面,請到宮少坐一時,何如?」行者道:「老孫比在前不同,爛板凳,高談闊論了;如今保唐僧,不得身閑。容敘,容敘。」急辭別墜雲,又見那太陽星上。徑來到禪堂前,搖身一變,變做個蜜蜂兒,飛將進去,現了本像看時,那師父還沉睡哩。

行者叫道:「師父,天亮了,起來罷。」三藏才醒覺,翻身道:「正是。」穿了衣服,開門出來,忽擡頭,只見些倒壁紅牆,不見了樓臺殿宇。大驚道:「呀!怎麼這殿宇俱無,都是紅牆,何也?」行者道:「你還做夢哩,今夜走了火的。」三藏道:「我怎不知?」行者道:「是老孫護了禪堂,見師父濃睡,不曾驚動。」三藏道:「你有本事護了禪堂,如何就不救別房之火?」行者笑道:「好教師父得知:果然依你昨日之言,他愛上我們的袈裟,算計要燒殺我們。若不是老孫知覺,到如今皆成灰骨矣。」三藏聞言,害怕道:「是他們放的火麼?」行者道:「不是他是誰?」三藏道:「莫不是怠慢了你,你幹的這個勾當?」行者道:「老孫是這等憊𪬯之人,幹這等不良之事?實實是他家放的。老孫見他心毒,果是不曾與他救火,只是與他略略助些風的。」三藏道:「天那,天那!火起時,只該助水,怎轉助風?」行者道:「你可知古人云:『人沒傷虎心,虎沒傷人意。』他不弄火,我怎肯弄風?」三藏道:「袈裟何在?敢莫是燒壞了也?」行者道:「沒事,沒事,燒不壞,那放袈裟的方丈無火。」三藏恨道:「我不管你,但是有些兒傷損,我只把那話兒念動念動,你就是死了。」行者慌了道:「師父莫念,莫念,管尋還你袈裟就是了。等我去拿來走路。」三藏才牽著馬,行者挑了擔,出了禪堂,徑往後方丈去。

卻說那些和尚正悲切間,忽的看見他師徒牽馬挑擔而來,諕得一個個魂飛魄散道:「冤魂索命來了。」行者喝道:「甚麼冤魂索命?快還我袈裟來。」眾僧一齊跪倒,叩頭道:「爺爺呀!冤有冤家,債有債主。要索命不干我們事,都是廣謀與老和尚定計害你的,莫問我們討命。」行者咄的一聲道:「我把你這些該死的畜生,那個問你討甚麼命。只拿袈裟來還我走路!」其間有兩個膽量大的和尚道:「老爺,你們在禪堂裡已燒死了,如今又來討袈裟,端的還是人,是鬼?」行者笑道:「這夥孽畜,那裡有甚麼火來?你去前面看看禪堂,再來說話。」眾僧們爬起來往前觀看,那禪堂外面的門窗槅扇,更不曾燎灼了半分。眾人悚懼,才認得三藏是種神僧,行者是尊護法。一齊上前叩頭道:「我等有眼無珠,不識真人下界。你的袈裟在後面方丈中老師祖處哩。」三藏行過了三五層敗壁破牆,嗟嘆不已。只見方丈果然無火,眾僧搶入裡面,叫道:「公公,唐僧乃是神人,未曾燒死,如今反害了自己家當。趁早拿出袈裟,還他去也。」

原來這老和尚尋不見袈裟,又燒了本寺的房屋,正在萬分煩惱焦燥之處,一聞此言,怎敢答應。因尋思無計,進退無方,拽開步,躬著腰,往那牆上著實撞了一頭,可憐只撞得腦破血流魂魄散,咽喉氣斷染紅沙。有詩為證。詩曰:
\begin{quote}
堪嘆老衲性愚蒙,枉作人間一壽翁。
欲得袈裟傳遠世,豈知佛寶不凡同。
但將容易為長久,定是蕭條取敗功。
廣智廣謀成甚用?損人利己一場空。
\end{quote}

慌得個眾僧哭道:「師公已撞殺了,又不見袈裟,怎生是好?」行者道:「想是汝等盜藏起也。都出來,開具花名手本,等老孫逐一查點。」那上下房的院主,將本寺和尚、頭陀、幸童、道人盡行開具手本二張,大小人等共計二百三十名。行者請師父高坐,他卻一一從頭唱名搜檢,都要解放衣襟,分明點過,更無袈裟。又將那各房頭搬搶出去的箱籠物件,從頭細細尋遍,那裡得有蹤跡。三藏心中煩惱,懊恨行者不盡,卻坐在上面念動那咒。行者撲的跌倒在地,抱著頭,十分難禁,只教:「莫念,莫念,管尋還了袈裟。」那眾僧見了,一個個戰兢兢的,上前跪下勸解,三藏才合口不念。行者一骨魯跳起來,耳朵裡掣出鐵棒,要打那些和尚,被三藏喝住道:「這猴頭,你頭痛還不怕,還要無禮?休動手,且莫傷人,再與我審問一問。」眾僧們磕頭禮拜,哀告三藏道:「老爺饒命。我等委實的不曾看見。這都是那老死鬼的不是。他昨晚看著你的袈裟,只哭到更深時候,看也不曾敢看,思量要圖長久,做個傳家之寶,設計定策,要燒殺老爺。自火起之候,狂風大作,各人只顧救火,搬搶物件,更不知袈裟去向。」

行者大怒,走進方丈屋裡,把那觸死鬼屍首擡出,選剝了細看,渾身更無那件寶貝。就把個方丈掘地三尺,也無蹤影。行者忖量半晌,問道:「你這裡可有甚麼妖怪成精麼?」院主道:「老爺不問,莫想得知。我這裡正東南有座黑風山,黑風洞內有一個黑大王,我這老死鬼常與他講道,他便是個妖精。別無甚物。」行者道:「那山離此有多遠近?」院主道:「只有二十里,那望見山頭的就是。」行者笑道:「師父放心,不須講了,一定是那黑怪偷去無疑。」三藏道:「他那廂離此有二十里,如何就斷得是他?」行者道:「你不曾見夜間那火,光騰萬里,亮透三天,且休說二十里,就是二百里也照見了。坐定是他見火光焜耀,趁著機會,暗暗的來到這裡,看見我們袈裟是件寶貝,必然趁鬨擄去也。等老孫去尋他一尋。」三藏道:「你去了時,我卻何倚?」行者道:「這個放心,暗中自有神靈保護,明中等我叫那些和尚伏侍。」即喚眾和尚過來,道:「汝等著幾個去埋那老鬼;著幾個伏侍我師父,看守我白馬。」眾僧領諾。行者又道:「汝等莫順口兒答應,等我去了,你就不來奉承。看師父的,要怡顏悅色;養白馬的,要水草調勻。假有一毫兒差了,照依這個樣棍,與你們看看。」他掣出棍子,照那火燒的磚牆上,撲的一下,把那牆打得粉碎,又震倒了有七八層牆。眾僧見了,個個骨軟身麻,跪著磕頭滴淚道:「爺爺寬心前去,我等竭力虔心,供奉老爺,決不敢一毫怠慢。」

好行者,急縱觔斗雲,徑上黑風山,尋找這袈裟。正是那:
\begin{quote}
金禪求正出京畿,仗錫投西涉翠微。
虎豹狼蟲行處有,工商士客見時稀。
路逢異國愚僧妒,全仗齊天大聖威。
火發風生禪院廢,黑熊夜盜錦襴衣。
\end{quote}

畢竟此去不知袈裟有無,吉凶如何,且聽下回分解。
