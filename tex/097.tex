
\chapter{金酬外護遭魔蟄 聖顯幽魂救本原}

且不言唐僧等在華光破屋中苦耐夜雨存身。卻說銅臺府地靈縣城內有夥兇徒,因宿娼、飲酒、賭博,花費了家私,無計過活,遂夥了十數人做賊,算道本城哪家是第一個財主,哪家是第二個財主,去打劫些金銀用度。內有一人道:「也不用緝訪,也不須算計,只有今日送那唐朝和尚的寇員外家十分富厚。我們乘此夜雨,街上人也不防備,火甲等也不巡邏,就此下手,劫他些貲本,我們再去嫖賭兒耍子,豈不美哉?」眾賊歡喜,齊了心,都帶了短刀、蒺藜、拐子、悶棍、麻繩、火把,冒雨前來,打開寇家大門,吶喊殺入。慌得他家裡若大若小,是男是女,俱躲個乾淨:媽媽兒躲在床底;老頭兒閃在門後;寇梁、寇棟與著親的幾個兒女,都戰戰兢兢的四散逃走顧命。那夥賊拿著刀,點著火,將他家箱籠打開,把些金銀寶貝、首飾衣裳、器皿家火,盡情搜劫。那員外割捨不得,拚了命,走出門來,對眾強人哀告道:「列位大王,夠你用的便罷,還留幾件衣物與我老漢送終。」那眾強人哪容分說,趕上前,把寇員外撩陰一腳,踢翻在地。可憐三魂渺渺歸陰府,七魄悠悠別世人!眾賊得了手,走出寇家,順城腳做了軟梯,漫城牆一一繫出,冒著雨連夜奔西而去。那寇家僮僕見賊退了,方才出頭。及看時,老員外已死在地下。放聲哭道:「天呀!主人公已打死了!」眾皆伏屍而哭,悲悲啼啼。

將四更時,那媽媽想恨唐僧等不受他的齋供,因為花撲撲的送他,惹出這場災禍,便生妒害之心,欲陷他四眾。扶著寇梁道:「兒啊,不須哭了。你老子今日也齋僧,明日也齋僧,豈知今日做圓滿,齋著那一夥送命的僧也。」他兄弟道:「母親,怎麼是送命僧?」媽媽道:「賊勢兇勇,殺進房來,我就躲在床下,戰兢兢的留心向燈火處看得明白。你說是誰?點火的是唐僧,持刀的是豬八戒,搬金銀的是沙和尚,打死你老子的是孫行者。」二子聽言,認了真實道:「母親既然看得明白,必定是了。他四人在我家住了半月,將我家門戶牆垣、窗櫺巷道,俱看熟了,財動人心,所以乘此夜雨,復到我家,既劫去財物,又害了父親,此情何毒!待天明到府裡遞失狀,坐名告他。」寇棟道:「失狀如何寫?」寇梁道:「就依母親之言。」寫道:
\begin{quote}
唐僧點著火,八戒叫殺人。沙和尚劫出金銀去,孫行者打死我父親。
\end{quote}

一家子吵吵鬧鬧,不覺天曉。一壁廂傳請親人,置辦棺木;一壁廂寇梁兄弟,赴府投詞。原來這銅臺府刺史正堂大人:
\begin{quote}
平生正直,素性賢良。少年向雪案攻書,早歲在金鑾對策。常懷忠義之心,每切仁慈之念。名揚青史播千年,龔黃再見;聲振黃堂傳萬古,卓魯重生。
\end{quote}

當時坐了堂,發放了一應事務,即令擡出放告牌。這寇梁兄弟抱牌而入,跪倒高叫道:「爺爺,小的們是告強盜得財,殺傷人命重情事。」刺史接上狀去,看了這般這的,如此如彼,即問道:「昨日有人傳說,你家齋僧圓滿,齋得四眾高僧,乃東土唐朝的羅漢,花撲撲的滿街鼓樂送行,怎麼卻有這般事情?」寇梁等磕頭道:「爺爺,小的父親寇洪,齋僧二十四年。因這四僧遠來,恰足萬僧之數,因此做了圓滿,留他住了半月。他就將路道、門窗都看熟了。當日送出,當晚復回,乘黑夜風雨,遂明火執杖,殺進房來,劫去金銀財寶、衣服首飾,又將父打死在地。望爺爺與小民做主。」刺史聞言,即點起馬步快手并民壯人役,共有百五十人,各執鋒利器械,出西門,一直來趕唐僧四眾。

卻說他師徒們在那華光行院破屋下挨至天曉,方才出門,上路奔西。可可的那些強盜當夜打劫了寇家,繫出城外,也向西方大路上,行經天曉,走過華光院西去,有二十里遠近,藏於山凹中,分撥金銀等物。分還未了,忽見唐僧四眾順路而來,眾賊心猶不歇,指定唐僧道:「那不是昨日送行的和尚來了?」眾賊笑道:「來得好,來得好。我們也是幹這般沒天理的買賣,這些和尚緣路來,又在寇家許久,不知身邊有多少東西,我們索性去截住他,奪了盤纏,搶了白馬湊分,卻不是遂心滿意之事?」眾賊遂持兵器,吶一聲喊,跑上大路,一字兒擺開,叫道:「和尚,不要走,快留下買路錢,饒你性命;牙迸半個『不』字,一刀一個,決不留存。」諕得唐僧在馬上亂戰,沙僧與八戒心慌,對行者道:「怎的了?怎的了?苦耐得半夜雨天,又早遇強徒斷路,誠所謂『禍不單行』也。」行者笑道:「師父莫怕,兄弟勿憂,等老孫去問他一問。」

好大聖,束一束虎皮裙,抖一抖錦布直裰,走近前,叉手當胸道:「列位是做甚麼的?」賊徒喝道:「這廝不知死活,敢來問我。你額顱下沒眼,不認得我是大王爺爺?快將買路錢來,放你過去。」行者聞言,滿面陪笑道:「你原來是剪徑的強盜。」賊徒發狠叫:「殺了。」行者假假的驚恐道:「大王,大王,我是鄉村中的和尚,不會說話,衝撞莫怪,莫怪。若要買路錢,不要問那三個,只消問我。我是個管帳的,凡有經錢、襯錢,哪裡化緣的、布施的,都在包袱中,盡是我管出入。那個騎馬的雖是我的師父,他卻只會念經,不管閑事,財色俱忘,一毫沒有。那個黑臉的是我半路上收的個後生,只會養馬。那個長嘴的是我雇的長工,只會挑擔。你把三個放過去,我將盤纏、衣缽盡情送你。」眾賊皆說:「這個和尚倒是個老實頭兒。既如此,饒了你命,教那三個丟下行李,放他過去。」行者回頭使個眼色,沙僧就丟了行李擔子,與師父牽著馬,同八戒往西徑走。

行者低頭打開包袱,就地撾把塵土,往上一灑,念個咒語,乃是個定身之法;喝一聲:「住!」那夥賊共有三十來名,一個個咬著牙,睜著眼,撒著手,直直的站定,莫能言語,不得動身。行者跳出路口,叫道:「師父,回來,回來。」八戒慌了道:「不好,不好,師兄供出我們來了。他身上又無錢財,包裡又無金銀,必定是叫師父要馬哩。叫我們是剝衣服了。」沙僧笑道:「二哥莫亂說。大哥是個了得的,向者那般毒魔狠怪,也能收服,怕這幾個毛賊?他那裡招呼,必有話說,快回去看看。」長老聽言,欣然轉馬,回至邊前,叫道:「悟空,有甚事叫回來也?」行者道:「你們看這些賊是怎的說?」八戒近前推著他,叫道:「強盜,你怎的不動彈了?」那賊渾然無知,不言不語。八戒道:「好的痴啞了。」行者笑道:「是老孫使個定身法定住也。」八戒道:「既定了身,未曾定口,怎麼連聲也不做?」行者道:「師父請下馬坐著。常言道:『只有錯捉,沒有錯放。』兄弟,你們把賊都扳翻倒綑了,教他供一個供狀,看他是個雛兒強盜,把勢強盜。」沙僧道:「沒繩索哩。」行者即拔下些毫毛,吹口仙氣,變作三十條繩索。一齊下手,把賊扳翻,都四馬攢蹄綑住。卻又念念解咒,那夥賊漸漸甦醒。

行者請唐僧坐在上首,他三人各執兵器喝道:「毛賊!你們一起有多少人?做了幾年買賣?打劫了有多少東西?可曾殺傷人口?還是初犯,卻是二犯、三犯?」眾賊開口道:「爺爺饒命。」行者道:「莫叫喚,從實供來。」眾賊道:「老爺,我們不是久慣做賊的,都是好人家子弟。只因不才,吃酒賭錢、宿娼頑耍,將父祖家業,盡花費了,一向無幹,又無錢用。訪知銅臺府城中寇員外家貲財豪富,昨日合夥,當晚乘夜雨昏黑,就去打劫。劫的有些金銀服飾,在這路北下山凹裡正自分贓,忽見老爺們來,內中有認得是寇員外送行的,必定身邊有物;又見行李沉重,白馬快走;人心不足,故又來邀截。豈知老爺有大神通法力,將我們困住。萬望老爺慈悲,收去那劫的財物,饒了我們性命也。」

三藏聽說是寇家劫的財物,猛然吃了一驚,慌忙站起道:「悟空,寇老員外十分好善,如何招此災厄?」行者笑道:「只為送我們起身,那等彩帳花幢,盛張鼓樂,驚動了人眼目,所以這夥光棍就去下手他家。今又幸遇著我們,奪下他這許多金銀服飾。」三藏道:「我們擾他半月,感激厚恩,無以為報,不如將此財物護送他家,卻不是一件好事?」行者依言。即與八戒、沙僧,去山凹裡取將那些贓物,收拾了,馱在馬上。又教八戒挑了一擔金銀,沙僧挑著自己行李。行者欲將這夥強盜一棍盡情打死,又恐唐僧怪他傷人性命,只得將身一抖,收上毫毛。那夥賊鬆了手腳,爬起來,一個個落荒逃生而去。這唐僧轉步回身,將財物送還員外。這一去,卻似飛蛾投火,反受其殃。有詩為證。詩曰:
\begin{quote}
恩將恩報人間少,反把恩慈變作仇。
下水救人終有失,三思行事卻無憂。
\end{quote}

三藏師徒們將著金銀服飾拿轉,正行處,忽見那槍刀簇簇而來。三藏大驚道:「徒弟,你看那兵器簇擁相臨,是甚好歹?」八戒道:「禍來了,禍來了,這是那放去的強盜,他取了兵器,又夥了些人,轉過路來與我們鬥殺也。」沙僧道:「二哥,那來的不是賊勢。——大哥,你仔細觀之。」行者悄悄的向沙僧道:「師父的災星又到了,此必是官兵捕賊之意。」說不了,眾兵卒至邊前,撒開個圈子陣,把他師徒圍住道:「好和尚!打劫了人家東西,還在這裡搖擺哩。」一擁上前,先把唐僧抓下馬來,用繩綑了;又把行者三人,也一齊綑了。穿上杠子,兩個擡一個,趕著馬,奪了擔,徑轉府城。只見那:
\begin{quote}
唐三藏,戰戰兢兢,滴淚難言;豬八戒,絮絮叨叨,心中報怨。沙和尚,囊突突,意下躊躇;孫行者,笑嘻嘻,要施手段。
\end{quote}

眾官兵攢擁扛擡,須臾間,拿到城裡,徑自解上黃堂報道:「老爺,民快人等,捕獲強盜來了。」那刺史端坐堂上,賞勞了民快,檢看了賊贓,當叫寇家領去。卻將三藏等提近廳前,問道:「你這起和尚,口稱是東土遠來,向西天拜佛,卻原來是些設法屣看門路,打家劫舍之賊。」三藏道:「大人容告:貧僧實不是賊,決不敢假,隨身現有通關文牒可照。只因寇員外家齋我等半月,情意深重,我等路遇強盜,奪轉打劫寇家的財物,因送還寇家報恩,不期民快人等捉獲,以為是賊,實不是賊。望大人詳察。」刺史道:「你這廝見官兵捕獲,卻巧言報恩。既是路遇強盜,何不連他捉來,報官報恩?如何只是你四眾?你看,寇梁遞得失狀,坐名告你,你還敢展掙?」三藏聞言,一似大海吞舟,魂飛魄喪。叫:「悟空,你何不上來折辨?」行者道:「有贓是實,折辨何為?」刺史道:「正是啊,贓證現存,還敢抵賴?」叫手下:「拿腦箍來,把這禿賊的光頭箍他一箍,然後再打。」行者慌了,心中暗想道:「雖是我師父該有此難,還不可教他十分受苦。」他見那皂隸們收拾索子,結腦箍,即便開口道:「大人且莫箍那個和尚。昨夜打劫寇家,點燈的也是我,持刀的也是我,劫財的也是我,殺人的也是我。我是個賊頭,要打只打我,與他們無干,但只不放我便是。」刺史聞言,就教先箍起這個來。皂隸們齊來上手,把行者套上腦箍,收緊了一勒,扢撲的把索子斷了。又結又箍,又扢撲的斷了。一連箍了三四次,他的頭皮皺也不曾皺一些兒。

卻又換索子再結時,只聽得有人來報道:「老爺,都下陳少保爺爺到了,請老爺出郭迎接。」那刺史即命刑房吏:「把賊收監,好生看轄。待我接過上司,再行拷問。」刑房吏遂將唐僧四眾推進監門。八戒、沙僧將自己行李擔進隨身。三藏道:「徒弟,這是怎麼起的?」行者笑道:「師父,進去,進去,這裡邊沒狗叫,倒好耍子。」可憐把四眾捉將進去,一個個都推入轄床,扣拽了滾肚、敵腦、攀胸。禁子們又來亂打。三藏苦痛難禁,只叫:「悟空,怎的好?怎的好?」行者道:「他打是要錢哩。常言道:『好處安身,苦處用錢。』如今與他些錢,便罷了。」三藏道:「我的錢自何來?」行者道:「若沒錢,衣物也是,把那袈裟與了他罷。」三藏聽說,就如刀刺其心。一時間見他打不過,只得開言道:「悟空,隨你罷。」

行者便叫:「列位長官,不必打了。我們擔進來的那兩個包袱中,有一件錦襴袈裟,價值千金,你們解開拿了去罷。」眾禁子聽言,一齊動手,把兩個包袱解看。雖有幾件布衣,雖有個引袋,俱不值錢。只見幾層油紙包裹著一物,霞光焰焰,知是好物。抖開看時,但只見:
\begin{quote}
巧妙明珠綴,稀奇佛寶攢。
盤龍鋪繡結,飛鳳錦沿邊。
\end{quote}

眾皆爭看,又驚動本司獄官,走來喝道:「你們在此嚷甚的?」禁子們跪道:「老爹,才子提審,送下四個和尚,乃是大夥強盜。他見我們打了他幾下,把這兩個包袱與我。我們打開看時,見有此物,無可處置:若眾人扯破分之,其實可惜;若獨歸一人,眾人無利。幸老爹來,憑老爹做個劈著。」獄官見了,乃是一件袈裟;又將別項衣服,並引袋兒通檢看了。又打開袋內關文一看,見有各國的寶印花押,道:「早是我來看呀,不然,你們都撞出事來了。這和尚不是強盜,切莫動他衣物。待明日太爺再審,方知端的。」眾禁子聽言,將包袱還與他,照舊包裹,交與獄官收訖。

漸漸天晚,聽得樓頭起鼓,火甲巡更。捱至四更三點,行者見他們都不呻吟,盡皆睡著,他暗想道:「師父該有這一夜牢獄之災。老孫不開口折辨,不使法力者,蓋為此耳。如今四更將盡,災將滿矣,我須去打點打點,天明好出牢門。」你看他弄本事,將身小一小,脫出轄床。搖身一變,變做個蜢蟲兒,從房簷瓦縫裡飛出。見那星光月皎,正是清和夜靜之天。他認了方向,徑飛向寇家門首,只見那街西下一家兒燈火明亮。又飛近他門口看時,原來是個做豆腐的。見一個老頭兒燒火,媽媽兒擠漿。那老兒忽的叫聲:「媽媽,寇大官且是有子有財,只是沒壽。我和他小時同學讀書,我還大他五歲。他老子叫做寇銘,當時也不上千畝田地,放些租帳,也討不起。他到二十歲時,那銘老兒死了,他掌著家當。其實也是他一步好運:娶的妻是那張旺之女,小名叫做穿針兒,卻倒旺夫,自進他門,種田又收,放帳又起,買著的有利,做著的賺錢,被他如今掙了有十萬家私。他到四十歲上,就回心向善,齋了萬僧,不期昨夜被強盜踢死。可憐!今年才六十四歲,正好享用。何期這等向善,不得好報,乃死於非命,可嘆,可嘆!」

行者一一聽之,卻早五更初點。他就飛入寇家,只見那堂屋裡已停著棺材,材頭邊點著燈,擺列著香燭花果,媽媽在傍啼哭;又見他兩個兒子也來拜哭,兩個媳婦拿兩盞飯兒供獻。行者就釘在他材頭上,咳嗽了一聲。諕得那兩個媳婦查手舞腳的往外跑;寇梁兄弟伏在地下不敢動,只叫:「爹爹!嚛嚛嚛」那媽媽子膽大,把材頭撲了一把道:「老員外,你活了?」行者學著那員外的聲音道:「我不曾活。」兩個兒子一發慌了,不住的叩頭垂淚,只叫:「爹爹!嚛嚛嚛」媽媽子硬著膽,又問道:「員外,你不曾活,如何說話?」行者道:「我是閻王差鬼使押將來家與你們講話的。那張氏穿針兒枉口誑舌,陷害無辜。」那媽媽子聽見叫他小名,慌得跪倒磕頭道:「好老兒啊!這等大年紀還叫我的小名兒!我哪些枉口誑舌,害甚麼無辜?」行者喝道:「有個甚麼『唐僧點著火,八戒叫殺人。沙僧劫出金銀去,行者打死你父親』。只因你誑言,把那好人受難。那唐朝四位老師路遇強徒,奪將財物,送來謝我,是何等好意!你卻假捏失狀,著兒子們首官。官府又未細審,又如今把他們監禁。那獄神、土地、城隍俱慌了,坐立不寧,報與閻王。閻王轉差鬼使押解我來家,教你們趁早解放他去;不然,教我在家攪鬧一月,將合家老幼並雞狗之類,一個也不存留。」寇梁兄弟又磕頭哀告道:「爹爹請回,切莫傷殘老幼。待天明就去本府投遞解狀,願認招回,只求存歿均安也。」行者聽了,即叫:「燒紙,我去呀。」他一家兒都來燒紙。

行者一翅飛起,徑又飛至刺史住宅裡面,低頭觀看,那房內裡已有燈光,見刺史已起來了。他就飛進中堂看時,只見中間後壁掛著一軸畫兒,是一個官兒騎著一匹點子馬,有幾個從人打著一把青傘,搴著一張校床,更不識是甚麼故事。行者就丁在中間。忽然那刺史自房裡出來,彎著腰梳洗。行者猛的裡咳嗽一聲,把刺史諕得慌慌張張,走入房內。梳洗畢,穿了大衣,即出來對著畫兒焚香禱告道:「伯考姜公乾一神位:孝姪姜坤三,蒙祖上德廕,忝中甲科,今叨受銅臺府刺史,旦夕侍奉香火不絕,為何今日發聲?切勿為邪為祟,恐諕家眾。」行者暗笑道:「此是他大爺的神子。」卻就綽著經兒叫道:「坤三賢姪,你做官雖承祖廕,一向清廉,怎的昨日無知,把四個聖僧當賊,不審來音,囚於禁內?那獄神、土地、城隍不安,報與閻君,閻君差鬼使押我來對你說,教你推情察理,快快解放他;不然,就教你去陰司折證也。」刺史聽說,心中悚懼道:「大爺請回,小姪升堂,當就釋放。」行者道:「既如此,燒紙來,我去見閻君回話。」刺史復添香燒紙拜謝。

行者又飛出來看時,東方早已發白。及飛到地靈縣,又見那合縣官卻都在堂上。他思道:「蜢蟲兒說話,被人看見,露出馬腳來不好。」他就半空中改了個大法身,從空裡伸下一隻腳來,把個縣堂屣滿。口中叫道:「眾官聽著:我乃玉帝差來的浪蕩遊神,說你這府監裡屈打了取經的佛子,驚動三界諸神不安,教我傳說,趁早放他;若有差池,教我再來一腳,先踢死合府縣官,後屣死四境居民,把城池都踏為灰燼。」概縣官吏人等慌得一齊跪倒,磕頭禮拜道:「上聖請回。我們如今進府,稟上府尊,即教放出。千萬莫動腳,驚諕死下官。」行者才收了法身,仍變做個蜢蟲兒,從監房瓦縫兒飛入,依舊鑽在轄床中間睡著。

卻說那刺史升堂,才擡出投文牌去,早有寇梁兄弟抱牌跪門叫喊。刺史著令進來。二人將解狀遞上。刺史見了,發怒道:「你昨日遞了失狀,就與你拿了賊來,你又領了贓去,怎麼今日又來遞解狀?」二人滴淚道:「老爺,昨夜小的父親顯魂道:『唐朝聖僧,原將賊徒拿住,奪獲財物,放了賊去,好意將財物送還我家報恩,怎麼反將他當賊,拿在獄中受苦?獄中土地、城隍不安,報了閻王,閻王差鬼使押解我來教你赴府再告,釋放唐僧,庶免災咎;不然,老幼皆亡。』因此,特來遞個解詞。望老爺方便方便。」刺史聽他說了這話,卻暗想道:「他那父親乃是熱屍,新鬼顯魂,報應猶可;我伯父死去五六年了,卻怎麼今夜也來顯魂,教我審放?看起來必是冤枉。」

正忖度間,只見那地靈縣知縣等官急急跑上堂,亂道:「老大人,不好了,不好了,適才玉帝差浪蕩遊神下界,教你快放獄中好人。昨日拿的那些和尚,不是強盜,都是取經的佛子。若少遲延,就要踢殺我等官員,還要把城池連百姓都踏為灰燼。」刺史又大驚失色,即叫刑房吏火速寫牌提出。當時開了監門提出。八戒愁道:「今日又不知怎的打哩。」行者笑道:「管你一下兒也不敢打,老孫俱已幹辦停當。上堂切不可下跪,他還要下來請我們上坐。卻等我問他要行李、要馬匹,少了一些兒,等我打他你看。」

說不了,已至堂口。那刺史、知縣並府縣大小官員一見,都下來迎接道:「聖僧昨日來時,一則接上司忙迫,二則又見了所獲之贓,未及細問端的。」唐僧合掌躬身,又將前情細陳了一遍。眾官滿口認稱,都道:「錯了,錯了。莫怪,莫怪。」又問獄中可曾有甚疏失。行者近前努目睜看,厲聲高叫道:「我的白馬是堂上人得了,行李是獄中人得了,快快還我。今日卻該我拷較你們了:枉拿平人做賊,你們該個甚罪?」府縣官見他作惡,無一個不怕,即便叫收馬的牽馬來,收行李的取行李來,一一交付明白。你看他三人一個個逞兇,眾官只以寇家遮飾。三藏勸解了道:「徒弟,是也不得明白。我們且到寇家去,一則吊問,二來與他對證對證,看是何人見我做賊?」行者道:「說得是。等老孫把那死的叫起來,看是哪個打他?」

沙僧就在府堂上把唐僧撮上馬,吆吆喝喝,一擁而出。那些府縣多官,也一一俱到寇家。諕得那寇梁兄弟在門前不住的磕頭,接進廳。只見他孝堂之中,一家兒都在孝幔裡啼哭。行者叫道:「那打誑語栽害平人的媽媽子且莫哭,等老孫叫你老公來,看他說是哪個打死的,羞他一羞。」眾官員只道孫行者說的是笑話。行者道:「列位大人,略陪我師父坐坐。八戒、沙僧,好生保護。等我去了就來。」好大聖,跳出門,望空就起。只見那遍地彩霞籠住宅,一天瑞氣護元神。眾等方才認得是個騰雲駕霧之仙,起死回生之聖,這裡一一焚香禮拜不題。

那大聖一路觔斗雲,直至幽冥地界,徑撞入森羅殿上。慌得那:
\begin{quote}
十代閻君拱手接,五方鬼判叩頭迎。千株劍樹皆攲側,萬疊刀山盡坦平。枉死城中魑魅化,奈河橋下鬼超生。正是那神光一照如天赦,黑暗陰司處處明。
\end{quote}

十閻王接下大聖,相見了,問及何來何幹。行者道:「銅臺府地靈縣齋僧的寇洪之鬼,是哪個收了?快點查來與我。」十閻王道:「寇洪善士,也不曾有鬼使勾他,他自家到此,遇著地藏王的金衣童子,他引見地藏也。」行者即別了,徑至翠雲宮見地藏王菩薩。菩薩與他禮畢,具言前事。菩薩喜道:「寇洪陽壽,止該卦數命終,不染床蓆,棄世而去。我因他齋僧,是個善士,收他做個掌善緣簿子的案長。既大聖來取,我再延他陽壽一紀,教他跟大聖去。」金衣童子遂領出寇洪。寇洪見了行者,聲聲叫道:「老師,老師,救我一救。」行者道:「你被強盜踢死,此乃陰司地藏王菩薩之處。我老孫特來取你到陽世間對明此事。既蒙菩薩放回,又延你陽壽一紀,待十二年之後,你再來也。」那員外頂禮不盡。

行者謝辭了菩薩,將他吹化為氣,掉於衣袖之間,同去幽府,復返陽間。駕雲頭,到了寇家,即喚八戒捎開材蓋,把他魂靈兒推付本身。須臾間,透出氣來活了。那員外爬出材來,對唐僧四眾磕頭道:「師父,師父,寇洪死於非命,蒙師父至陰司救活,乃再造之恩。」言謝不已。及回頭,見各官羅列,即又磕頭道:「列位老爹都如何在舍?」那刺史道:「你兒子始初遞失狀,坐名告了聖僧,我即差人捕獲。不期聖僧路遇殺劫你家之賊,奪取財物,送還你家。是我下人誤捉,未得詳審,當送監禁。今夜被你顯魂,我先伯亦來家訴告,縣中又蒙浪蕩遊神下界,一時就有這許多顯應,所以放出聖僧。聖僧卻又去救活你也。」那員外跪道:「老爹,其實枉了這四位聖僧。那夜有三十多名強盜,明火執杖,劫去家私。是我難捨,向賊理說,不期被他一腳,撩陰踢死。與這四位何干?」叫過妻子來,「是誰人踢死,你等輒敢妄告?請老爹定罪。」當時一家老小只是磕頭。刺史寬恩,免其罪過。寇洪教安排筵宴,酬謝府縣厚恩。個個未坐回衙。至次日,再掛齋僧牌,又款留三藏。三藏決不肯住。卻又請親友,辦旌幢,如前送行而去。咦!這正是:
\begin{quote}
地闢能存凶惡事,天高不負善心人。
逍遙穩步如來徑,只到靈山極樂門。
\end{quote}

畢竟不知見佛何如,且聽下回分解。
