
\chapter{心猿遭火敗 木母被魔擒}

\begin{quote}
善惡一時忘念,榮枯都不關心。
晦明隱現任浮沉。隨分饑餐渴飲。
神靜湛然常寂,昏冥便有魔侵。
五行蹬蹭破禪林。風動必然寒凜。
\end{quote}

卻說那孫大聖引八戒別了沙僧,跳過枯松澗,徑來到那怪石崖前。果見有一座洞府,真個也景致非凡。但見:
\begin{quote}
迴鑾古道幽還靜,風月也聽玄鶴弄。
白雲透出滿川光,流水過橋仙意興。
猿嘯鳥啼花木奇,藤蘿石蹬芝蘭勝。
蒼搖崖壑散煙霞,翠染松篁招彩鳳。
遠列巔峰似插屏,山朝澗繞真仙洞。
崑崙地脈發來龍,有分有緣方受用。
\end{quote}

將近行到門前,見有一座石碣,上鐫八個大字,乃是「號山枯松澗火雲洞」。那壁廂一群小妖,在那裡掄槍舞劍的跳風頑耍。

孫大聖厲聲高叫道:「那小的們,趁早去報與洞主知道,教他送出我唐僧師父來,免你這一洞精靈的性命;牙迸半個『不』字,我就掀翻了你的山場,屣平了你的洞府。」那些小妖,聞此言,慌忙急轉身,各歸洞裡,關了兩扇石門,到裡邊來報:「大王,禍事了。」

卻說那怪自把三藏拿到洞中,選剝了衣服,四馬攢蹄綑在後院裡,著小妖打乾淨水刷洗,要上籠蒸吃哩。忽聽得報聲禍事,且不刷洗,便來前庭上問:「有何禍事?」小妖道:「有個毛臉雷公嘴的和尚,帶一個長嘴大耳的和尚,在門前要甚麼唐僧師父哩。但若牙迸半個『不』字,就要掀翻山場,屣平洞府。」魔王微微冷笑道:「這是孫行者與豬八戒,他卻也會尋哩。我拿他師父,自半山中到此,有百五十里,卻怎麼就尋上門來?」教小的們把管車的推出車去。那一班幾個小妖,推出五輛小車兒來,開了前門。

八戒望見道:「哥哥,這妖精想是怕我們,推出車子,往那廂搬哩。」行者道:「不是,且看他放在那裡。」只見那小妖將車子按金、木、水、火、土安下,著五個看著,五個進去通報。那魔王問:「停當了?」答應:「停當了。」教取過槍來。有那一夥管兵器的小妖,著兩個擡出一桿丈八長的火尖槍,遞與妖王。妖王掄槍拽步,也無甚麼盔甲,只是腰間束一條錦繡戰裙,赤著腳,走出門前。行者與八戒擡頭觀看,但見那怪物:
\begin{quote}
面如傅粉三分白,唇若塗朱一表才。
鬢挽青雲欺靛染,眉分新月似刀裁。
戰裙巧繡盤龍鳳,形比哪吒更富胎。
雙手綽槍威凜冽,祥光護體出門來。
哏聲響若春雷吼,暴眼明如掣電乖。
要識此魔真姓氏,名揚千古喚紅孩。
\end{quote}

那紅孩兒怪出得門來,高叫道:「是甚麼人在我這裡吆喝!」行者近前笑道:「我賢姪莫弄虛頭。你今早在山路傍高吊在松樹梢頭,是那般一個瘦怯怯的黃病孩兒,哄了我師父。我倒好意馱著你,你就弄風兒把我師父攝將來。你如今又弄這個樣子,我豈不認得你?趁早送出我師父,不要白了面皮,失了親情,恐你令尊知道,怪我老孫以長欺幼,不象模樣。」那怪聞言,心中大怒,咄的一聲喝道:「那潑猴頭!我與你有甚親情?你在這裡滿口胡柴,綽甚聲經兒?那個是你賢姪?」行者道:「哥哥,是你也不曉得。當年我與你令尊做弟兄時,你還不知在那裡哩。」那怪道:「這猴子一發胡說!你是那裡人,我是那裡人,怎麼得與我父親做弟兄?」行者道:「你是不知。我乃五百年前大鬧天宮的齊天大聖孫悟空是也。我當初未鬧天宮時,遍遊海角天涯,四大部洲,無方不到,那時節專慕豪傑。你令尊叫做牛魔王,稱為平天大聖,與我老孫結為七弟兄,讓他做了大哥;還有個蛟魔王,稱為覆海大聖,做了二哥;又有個大鵬魔王,稱為混天大聖,做了三哥;又有個獅㾩王,稱為移山大聖,做了四哥;又有個獼猴王,稱為通風大聖,做了五哥;又有個狨王,稱為驅神大聖,做了六哥;惟有老孫身小,稱為齊天大聖,排行第七。我老弟兄們那時節耍子時,還不曾生你哩。」

那怪物聞言,那裡肯信,舉起火尖槍就刺。行者正是那會家不忙,又使了一個身法,閃過槍頭,掄起鐵棒,罵道:「你這小畜生,不識高低,看棍!」那妖精也使身法,讓過鐵棒道:「潑猢猻,不達時務,看槍!」他兩個也不論親情,一齊變臉,各使神通,跳在雲端裡,好殺:
\begin{quote}
行者名聲大,魔王手段強。一個橫舉金箍棒,一個直挺火尖槍。吐霧遮三界,噴雲照四方。一天殺氣兇聲吼,日月星辰不見光。語言無遜讓,情意兩乖張。那一個欺心失禮儀,這一個變臉沒綱常。棒架威風長,槍來野性狂。一個是混元真大聖,一個是正果善財郎。二人努力爭強勝,只為唐僧拜法王。
\end{quote}

那妖魔與孫大聖戰經二十合,不分勝敗。豬八戒在傍邊看得明白:妖精雖不敗陣,卻只是遮攔隔架,全無攻殺之能;行者縱不贏他,棒法精強,來往只在那妖精頭上,不離了左右。八戒暗想道:「不好啊,行者溜撒,一時間丟個破綻,哄那妖魔鑽進來,一鐵棒打倒,就沒了我的功勞。」你看他抖擻精神,舉著九齒鈀在空裡,望妖精劈頭就築。那怪見了心驚,急拖槍敗下陣來。行者喝教:「八戒,趕上,趕上。」

二人趕到他洞門前,只見妖精一隻手舉著火尖槍,站在那中間一輛小車兒上;一隻手捏著拳頭,往自家鼻子上搥了兩拳。八戒笑道:「這廝放賴不羞。你好道搥破鼻子,淌出些血來,搽紅了臉,往那裡告我們去耶?」那妖魔搥了兩拳,念個咒語,口裡噴出火來,鼻子裡濃煙迸出,閘閘眼,火焰齊生,那五輛車子上火光湧出。連噴了幾口,只見那紅焰焰大火燒空,把一座火雲洞被那煙火迷漫,真個是熯天熾地。八戒慌了道:「哥哥,不停當,這一鑽在火裡,莫想得活。把老豬弄做個燒熟的,加上香料,盡他受用哩。快走,快走。」說聲走,他也不顧行者,跑過澗去了。

這行者神通廣大,捏著避火訣,撞入火中,尋那妖怪。那妖怪見行者來,又吐上幾口,那火比前更勝。好火:
\begin{quote}
炎炎烈烈盈空燎,赫赫威威遍地紅。卻似火輪飛上下,猶如炭屑舞西東。這火不是燧人鑽木,又不是老子炮丹,非天火,非野火,乃是妖魔修煉成真三昧火。五輛車兒合五行,五行生化火煎成。肝木能生心火旺,心火致令脾土平。脾土生金金化水,水能生木徹通靈。生生化化皆因火,火遍長空萬物榮。妖邪久悟呼三昧,永鎮西方第一名。
\end{quote}

行者被他煙火飛騰,不能尋怪,看不見他洞門前路徑,抽身跳出火中。那妖精在門首看得明白,他見行者走了,卻才收了火具,帥群妖,轉於洞內,閉了石門,以為得勝,著小的排宴奏樂,歡笑不題。

卻說行者跳過枯松澗,按下雲頭,只聽得八戒與沙僧朗朗的在松間講話。行者上前喝八戒道:「你這獃子,全無人氣。你就懼怕妖火,敗走逃生,卻把老孫丟下。早是我有些南北哩。」八戒笑道:「哥啊,你被那妖精說著了,果然不達時務。古人云:『識得時務者,呼為俊傑。』那妖精不與你親,你強要認親;既與你賭鬥,放出那般無情的火來,又不走,還要與他戀戰哩。」行者道:「那怪物的手段比我何如?」八戒道:「不濟。」「槍法比我何如?」八戒道:「也不濟。老豬見他撐持不住,卻來助你一鈀,不期他不識耍,就敗下陣來,沒天理,就放火了。」行者道:「正是你不該來。我再與他鬥幾合,我取巧兒撈他一棒,卻不是好?」

他兩個只管論那妖精的手段,講那妖精的火毒。沙和尚倚著松根,笑得騃了。行者看見道:「兄弟,你笑怎麼?你好道有甚手段,擒得那妖魔,破得那火陣?這樁事,也是大家有益的事。常言道:『眾毛攢毬。』你若拿得妖魔,救了師父,也是你的一件大功績。」沙僧道:「我也沒甚手段,也不能降妖。我笑你兩個都著了忙也。」行者道:「我怎麼著忙?」沙僧道:「那妖精手段不如你,槍法不如你,只是多了些火勢,故不能取勝。若依小弟說,以相生相剋拿他,有甚難處?」行者聞言,呵呵笑道:「兄弟說得有理,果然我們著忙了,忘了這事。若以相生相剋之理論之,須是以水剋火。卻往那裡尋些水來,潑滅這妖火,可不救了師父?」沙僧道:「正是這般,不必遲疑。」行者道:「你兩個只在此間,莫與他索戰,待老孫去東洋大海求借龍兵,將些水來,潑息妖火,捉這潑怪。」八戒道:「哥哥放心前去,我等理會得。」

好大聖,縱雲離此地,頃刻到東洋,卻也無心看玩海景,使個逼水法,分開波浪。正行時,見一個巡海夜叉相撞,看見是孫大聖,急回到水晶宮裡,報知那老龍王。敖廣即率龍子、龍孫、蝦兵、蟹卒一齊出門迎接,請裡面坐。坐定,禮畢,告茶。行者道:「不勞茶,有一事相煩:我因師父唐僧往西天拜佛取經,經過號山枯松澗火雲洞,有個紅孩兒妖精,號聖嬰大王,把我師父拿了去。是老孫尋到洞邊,與他交戰,他卻放出火來。我們禁不得他,想著水能剋火,特來問你求些水去,與我下場大雨,潑滅了那火,救唐僧一難。」那龍王道:「大聖差了,若要求取雨水,不該來問我。」行者道:「你是四海龍王,主司雨澤,不來問你,卻去問誰?」龍王道:「我雖司雨,不敢擅專。須得玉帝旨意,吩咐在那地方,要幾尺幾寸,甚麼時辰起住,還要三官舉筆,太乙移文,會令了雷公、電母、風伯、雲童。俗語云:『龍無雲而不行』哩。」行者道:「我也不用著風雲雷電,只是要些雨水滅火。」龍王道:「大聖不用風雲雷電,但我一人也不能助力。著舍弟們同助大聖一功如何?」行者道:「令弟何在?」龍王道:「南海龍王敖欽、北海龍王敖閏、西海龍王敖順。」行者笑道:「我若再遊過三海,不如上界去求玉帝旨意了。」龍王道:「不消大聖去,只我這裡撞動鐵鼓、金鐘,他自頃刻而至。」行者聞其言道:「老龍王,快撞鐘鼓。」

須臾間,三海龍王擁至,問:「大哥,有何事命弟等?」敖廣道:「孫大聖在這裡借雨助力降妖。」三弟即引進見畢,行者備言借水之事。眾神個個歡從,即點起:
\begin{quote}
鯊魚驍勇為前部,鱯痴口大作先鋒。
鯉元帥翻波跳浪,鯁提督吐霧噴風。
鯖太尉東方打哨,鮊都司西路催征。
紅眼馬郎南面舞,黑甲將軍北下衝。
鯕把總中軍掌號,五方兵處處英雄。
縱橫機巧黿樞密,妙算玄微龜相公。
有謀有智鼉丞相,多變多能鱉總戎。
橫行蟹士掄長劍,直跳蝦婆扯硬弓。
鮎外郎查明文簿,點龍兵出離波中。
\end{quote}

有詩為證。詩曰:
\begin{quote}
四海龍王喜助功,齊天大聖請相從。
只因三藏途中難,借水前來滅火紅。
\end{quote}

那行者領著龍兵,不多時,早到號山枯松澗上。行者道:「敖氏昆玉,有煩遠踄。此間乃妖魔之處,汝等且停於空中,不要出頭露面。讓老孫與他賭鬥,若贏了他,不須列位捉拿;若輸與他,也不用列位助陣。只是他但放火時,可聽我呼喚,一齊噴雨。」龍王俱如號令。

行者卻按雲頭,入松林裡,見了八戒、沙僧,叫聲:「兄弟。」八戒道:「哥哥來得快啞。可曾請得龍王來?」行者道:「俱來了。你兩個切須仔細,只怕雨大,莫濕了行李。待老孫與他打去。」沙僧道:「師兄放心前去,我等俱理會得了。」

行者跳過澗,到了門首,叫聲:「開門!」那些小妖又去報道:「孫行者又來了。」紅孩仰面笑道:「那猴子想是火中不曾燒了他,故此又來。這一來切莫饒他,斷然燒個皮焦肉爛才罷。」急縱身,挺著長槍,教:「小的們,推出火車子來。」他出門前,對行者道:「你又來怎的?」行者道:「還我師父來。」那怪道:「你這猴頭,忒不通變。那唐僧與你做得師父,也與我做得按酒,你還思量要他哩,莫想,莫想。」行者聞言,十分惱怒,掣金箍棒,劈頭就打;那妖精使火尖槍,急架相迎。這一場賭鬥,比前不同。好殺:
\begin{quote}
怒發潑妖魔,惱急猴王將。這一個專救取經僧,那一個要吃唐三藏。心變沒親情,情疏無義讓。這個恨不得捉住活剝皮,那個恨不得拿來生蘸醬。真個忒英雄,果然多猛壯。棒來槍架賭輸贏,槍去棒迎爭下上。舉手相掄二十回,兩家本事一般樣。
\end{quote}

那妖王與行者戰經二十回合,見得不能取勝,虛幌一槍,急抽身,捏著拳頭,又將鼻子搥了兩下,卻就噴出火來,那門前車子上煙火迸起,口眼中赤焰飛騰。孫大聖回頭叫道:「龍王何在?」那龍王兄弟帥眾水族,望妖精火光裡噴下雨來。好雨!真個是:
\begin{quote}
瀟瀟灑灑,密密沉沉。瀟瀟灑灑,如天邊墜落星辰;密密沉沉,似海口倒懸浪滾。起初時如拳大小,次後來甕潑盆傾。滿地澆流鴨頂綠,高山洗出佛頭青。溝壑水飛千丈玉,澗泉波漲萬條銀。三叉路口看看滿,九曲溪中漸漸平。這個是唐僧有難神龍助,扳倒天河往下傾。
\end{quote}

那雨淙淙大小,莫能止息那妖精的火勢。原來龍王私雨,只好潑得凡火,妖精的三昧真火如何潑得?好一似火上澆油,越潑越灼。

大聖道:「等我捻著訣,鑽入火中。」掄鐵棒,尋妖要打。那妖見他來到,將一口煙劈臉噴來。行者急回頭,煼得眼花雀亂,忍不住淚落如雨。原來這大聖不怕火,只怕煙。當年因大鬧天宮時,被老君放在八卦爐中鍛過一番,他幸在那巽位安身,不曾燒壞。只是風攪得煙來,把他煼做火眼金睛,故至今只是怕煙。那妖又噴一口,行者當不得,縱雲頭走了。那妖王卻又收了火具,回歸洞府。

這大聖一身煙火,炮燥難禁,徑投於澗水內救火。怎知被冷水一逼,弄得火氣攻心,三魂出舍。可憐氣塞胸堂喉舌冷,魂飛魄散喪殘生。慌得那四海龍王在半空裡收了雨澤,高聲大叫:「天蓬元帥、捲簾將軍,休在林中藏隱,且尋你師兄出來。」

八戒與沙僧聽得呼他聖號,急忙解了馬,挑著擔,奔出林來,也不顧泥濘,順澗邊找尋。只見那上溜頭翻波滾浪,急流中淌下一個人來。沙僧見了,連衣跳下水中,抱上岸來,卻是孫大聖身軀。噫!你看他蜷跼四肢伸不得,渾身上下冷如冰。沙和尚滿眼垂淚道:「師兄,可惜了你,億萬年不老長生客,如今化作個中途短命人。」八戒笑道:「兄弟莫哭。這猴子佯推死,嚇我們哩。你摸他摸,胸前還有一點熱氣沒有?」沙僧道:「渾身都冷了,就有一點兒熱氣,怎的就得回生?」八戒道:「他有七十二般變化,就有七十二條性命。你扯著腳,等我擺佈他。」真個那沙僧扯著腳,八戒扶著頭,把他拽個直,推上腳來,盤膝坐定。八戒將兩手搓熱,仵住他的七竅,使一個按摩禪法。原來那行者被冷水逼了,氣阻丹田,不能出聲。卻幸得八戒按摸揉擦,須臾間,氣透三關,轉明堂,沖開孔竅,叫了一聲:「師父啊!」沙僧道:「哥啊,你生為師父,死也還在口裡。且甦醒,我們在這裡哩。」行者睜開眼道:「兄弟們在這裡?老孫吃了虧也。」八戒笑道:「你才子發昏的,若不是老豬救你啊,已此了帳了,還不謝我哩。」

行者卻才起身,仰面道:「敖氏弟兄何在?」那四海龍王在半空中答應道:「小龍在此伺候。」行者道:「累你遠勞,不曾成得功果,且請回去,改日再謝。」龍王帥水族,泱泱而回,不在話下。

沙僧攙著行者,一同到松林之下坐定。少時間,卻定神順氣,止不住淚滴腮邊。又叫:「師父啊!
\begin{quote}
憶昔當年出大唐,巖前救我脫災殃。
三山六水遭魔障,萬苦千辛割寸腸。
托缽朝餐隨厚薄,參禪暮宿或林莊。
一心指望成功果,今日安知痛受傷?」
\end{quote}

沙僧道:「哥哥,且休煩惱。我們早安計策,去那裡請兵助力,搭救師父耶。」行者道:「那裡請救麼?」沙僧道:「當初菩薩吩咐,著我等保護唐僧,他曾許我們:叫天天應,叫地地應。那裡請救去?」行者道:「想老孫大鬧天宮時,那些神兵都禁不得我。這妖精神通不小,須是比老孫手段大些的才降得他哩。天神不濟,地煞不能,若要拿此妖魔,須是去請觀音菩薩才好。奈何我皮肉酸麻,腰膝疼痛,駕不起觔斗雲,怎生請得?」八戒道:「有甚話吩咐,等我去請。」行者笑道:「也罷,你是去得。若見了菩薩,切休仰視,只可低頭禮拜。等他問時,你卻將地名、妖名說與他,再請救師父之事。他若肯來,定取擒了怪物。」八戒聞言,即便駕了雲霧,向南而去。

卻說那個妖王在洞裡歡喜道:「小的們,孫行者吃了虧去了。這一陣雖不得他死,好道也發個大昏。咦!只怕他又請救兵來也。快開門,等我去看他請誰。」

眾妖開了門,妖精就跑在空裡觀看,只見八戒往南去了。妖精想著南邊再無他處,斷然是請觀音菩薩。急按下雲,叫:「小的們,把我那皮袋尋出來。多時不用,只恐口繩不牢,與我換上一條,放在二門之下。等我去把八戒賺將回來,裝於袋內,蒸得稀爛,犒勞你們。」原來那妖精有一個如意的皮袋。眾小妖拿出來,換了口繩,安於洞門內不題。

卻說那妖王久居於此,俱是熟遊之地,他曉得那條路上南海去近,那條路去遠。他從那近路上,一駕雲頭,趕過了八戒,端坐在壁巖之上,變作一個假觀世音模樣,等候著八戒。

那獃子正縱雲行處,忽然望見菩薩。他那裡識得真假?這才是見像作佛。獃子停雲下拜道:「菩薩,弟子豬悟能叩頭。」妖精道:「你不保唐僧去取經,卻見我有何事幹?」八戒道:「弟子因與師父行至中途,遇著號山枯松澗火雲洞有個紅孩兒妖精,他把我師父攝了去。是弟子與師兄等尋上他門,與他交戰。他原來會放火,頭一陣,不曾得贏。第二陣,請龍王助雨,也不能滅火。師兄被他燒壞了,不能行動,著弟子來請菩薩。萬望垂慈,救我師父一難。」妖精道:「那火雲洞洞主,不是個傷生的,一定是你們衝撞了他也。」八戒道:「我不曾衝撞他,是師兄悟空衝撞他的。他變作一個小孩兒,吊在樹上,試我師父。師父甚有善心,教我解下來,著師兄馱他一程。是師兄摜了他一摜,他就弄風兒,把師父攝去了。」妖精道:「你起來,跟我進那洞裡見洞主,與你說個人情,你陪一個禮,把你師父討出來罷。」八戒道:「菩薩呀,若肯還我師父,就磕他一個頭也罷。」妖王道:「你跟來。」

那獃子不知好歹,就跟著他,徑回舊路,卻不向南洋海,隨赴火雲門,頃刻間到了門首。妖精進去道:「你休疑忌。他是我的故人,你進來。」獃子只得舉步入門。眾妖一齊吶喊,將八戒捉倒,裝於袋內,束緊了口繩,高吊在馱梁之上。妖精現了本像,坐在當中道:「豬八戒,你有甚麼手段,就敢保唐僧取經?就敢請菩薩降我?你大睜著兩個眼,還不認得我是聖嬰大王哩。如今拿你,吊得三五日,蒸熟了賞賜小妖,權為案酒。」八戒聽言,在裡面罵道:「潑怪物!十分無禮。若論你百計千方,騙了我吃,管教你一個個遭腫頭天瘟。」獃子罵了又罵,嚷了又嚷,不題。

卻說孫大聖與沙僧正坐,只見一陣腥風,刮面而過,他就打了一個噴嚏道:「不好,不好!這陣風凶多吉少,想是豬八戒走錯路也。」沙僧道:「他錯了路,不會問人?」行者道:「想必撞見妖精了。」沙僧道:「撞見妖精,他不會跑回?」行者道:「不停當。你坐在這裡看守,等我跑過澗去打聽打聽。」沙僧道:「師兄腰疼,只恐又著他手,等小弟去罷。」行者道:「你不濟事,還讓我去。」

好行者,咬著牙,忍著疼,捻著鐵棒,走過澗,到那火雲洞前,叫聲:「潑怪!」那把門的小妖又急入裡報:「孫行者又在門首叫哩。」那妖王傳令叫拿。那夥小妖槍刀簇擁,齊聲吶喊,即開門,都道:「拿住,拿住。」行者果然疲倦,不敢相迎,將身鑽在路傍,念個咒語,叫:「變!」即變做一個銷金包袱。小妖看見取了進去,報道:「大王,孫行者怕了,只見說一聲『拿』字,慌得把包袱丟下,走了。」妖王笑道:「那包袱也無甚麼值錢之物,左右是和尚的破褊衫、舊帽子,背進來拆洗做補襯。」一個小妖果將包袱背進,不知是行者變的。行者道:「好了,這個銷金包袱背著了。」那妖精不以為事,丟在門內。

好行者,假中又假,虛裡還虛:即拔一根毫毛,吹口仙氣,變作個包袱一樣;他的真身卻又變作一個蒼蠅兒,釘在門樞上。只聽得八戒在那裡哼哩哼的,聲音不清,卻似一個瘟豬。行者嚶的飛了去尋時,原來他吊在皮袋裡也。行者釘在皮袋上,又聽得他惡言惡語罵妖怪長,妖怪短:「你怎麼假變作個觀音菩薩,哄我回來,吊我在此,還說要吃我?有一日我師兄:
\begin{quote}
大展齊天無量法,滿山潑怪等時擒。
解開皮袋放我出,築你千鈀方趁心。」
\end{quote}

行者聞言,暗笑道:「這獃子雖然在這裡面受悶氣,卻還不倒了旗槍。老孫一定要拿了此怪;若不如此,怎生雪恨。」

正欲設法拯救八戒出來,只聽得妖王叫道:「六健將何在?」時有六個小妖是他知己的精靈,封為健將,都有名字:一個叫做雲裡霧,一個叫做霧裡雲;一個叫做急如火,一個叫做快如風;一個叫做興烘掀,一個叫做掀烘興。六健將上前跪下。妖王道:「你們認得老大王家麼?」六健將道:「認得。」妖王道:「你與我星夜去請老大王來,說我這裡捉唐僧蒸與他吃,壽延千紀。」六怪領命,一個個廝拖廝扯,徑出門去了。行者嚶的一聲,飛下袋來,跟定那六怪,躲離洞中。

畢竟不知怎的請來,且聽下回分解。
