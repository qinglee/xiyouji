
\chapter{法性西來逢女國 心猿定計脫煙花}

話說三藏師徒別了村舍人家,依路西進,不上三四十里,早到西梁國界。唐僧在馬上指道:「悟空,前面城池相近,市井上人語喧嘩,想是西梁女國。汝等須要仔細,謹慎規矩,切休放蕩情懷,紊亂法門教旨。」三人聞言,謹遵嚴命。

言未盡,卻至東關廂街口。那裡人都是長裙短襖,粉面油頭,不分老少,盡是婦女。正在兩街上做買做賣,忽見他四眾來時,一齊都鼓掌呵呵,整容歡笑道:「人種來了,人種來了。」慌得那三藏勒馬難行。須臾間就塞滿街道,惟聞笑語。八戒口裡亂嚷道:「我是個銷豬,我是個銷豬。」行者道:「獃子,莫胡談,拿出舊嘴臉便是。」八戒真個把頭搖上兩搖,豎起一雙蒲扇耳,扭動蓮蓬吊搭唇,發一聲喊,把那些婦女們諕得跌跌爬爬。有詩為證。詩曰:
\begin{quote}
聖僧拜佛到西梁,國內衠陰世少陽。
農士工商皆女輩,漁樵耕牧盡紅妝。
嬌娥滿路呼人種,幼婦盈街接粉郎。
不是悟能施醜相,煙花圍困苦難當!
\end{quote}

遂此眾皆恐懼,不敢上前。一個個都捻手矬腰,搖頭咬指,戰戰兢兢,排塞街傍路下,都看唐僧。孫大聖卻也弄出醜相開路;沙僧也裝𡤫虎維持;八戒採著馬,掬著嘴,擺著耳朵。

一行前進,又見那市井上房屋齊整,鋪面軒昂,一般有賣鹽賣米,酒肆茶房;鼓角樓臺通貨殖,旗亭候館掛簾櫳。師徒們轉彎抹角,忽見有一女官侍立街下,高聲叫道:「遠來的使客,不可擅入城門。請投館驛,註名上簿,待下官執名奏駕,驗引放行。」三藏聞言下馬,觀看那衙門上有一匾,上書「迎陽驛」三字。長老道:「悟空,那村舍人家傳言是實,果有迎陽之驛。」沙僧笑道:「二哥,你卻去照胎泉邊照照,看可有雙影?」八戒道:「莫弄我。我自吃了那盞兒落胎泉水,已此打下胎來了,還照他怎的?」三藏回頭吩咐道:「悟能,謹言,謹言。」遂上前與那女官作禮。

女官引路,請他們都進驛內,正廳坐下,即喚看茶。又見那手下人盡是三綹梳頭,兩截穿衣之類。你看他拿茶的也笑。少頃,茶罷。女官欠身問曰:「使客何來?」行者道:「我等乃東土大唐王駕下欽差上西天拜佛求經者。我師父便是唐王御弟,號曰唐三藏。我乃他大徒弟孫悟空。這兩個是我師弟豬悟能、沙悟淨。一行連馬五口。隨身有通關文牒,乞為照驗放行。」那女官執筆寫罷,下來叩頭道:「老爺恕罪。下官乃迎陽驛驛丞,實不知上邦老爺,知當遠接。」拜畢起身,即令管事的安排飲饌。道:「爺爺們寬坐一時,待下官進城啟奏我王,倒換關文,打發領給,送老爺們西進。」三藏欣然而坐不題。

且說那驛丞整了衣冠,徑入城中五鳳樓前,對黃門官道:「我是迎陽館驛丞,有事見駕。」黃門即時啟奏。降旨傳宣至殿,問曰:「驛丞有何事來奏?」驛丞道:「微臣在驛,接得東土大唐王御弟唐三藏,有三個徒弟,名喚孫悟空、豬悟能、沙悟淨,連馬五口,欲上西天拜佛取經。特來啟奏主公,可許他倒換關文放行?」女王聞奏,滿心歡喜,對眾文武道:「寡人夜來夢見金屏生彩艷,玉鏡展光明,乃是今日之喜兆也。」眾女官擁拜丹墀道:「主公,怎見得是今日之喜兆?」女王道:「東土男人,乃唐朝御弟。我國中自混沌開闢之時,累代帝王,更不曾見個男人至此。幸今唐王御弟下降,想是天賜來的。寡人以一國之富,願招御弟為王,我願為后,與他陰陽配合,生子生孫,永傳帝業,卻不是今日之喜兆也?」眾女官拜舞稱揚,無不歡悅。

驛丞又奏道:「主公之論,乃萬代傳家之好。但只是御弟三徒兇惡,不成相貌。」女王道:「卿見御弟怎生模樣?他徒弟怎生兇醜?」驛丞道:「御弟相貌堂堂,丰姿英俊,誠是天朝上國之男兒,南贍中華之人物。那三徒卻是形容獰惡,相貌如精。」女王道:「既如此,把他徒弟與他領給,倒換關文,打發他往西天,只留下御弟,有何不可?」眾官拜奏道:「主公之言極當,臣等欽此欽遵。但只是匹配之事,無媒不可。自古道:『姻緣配合憑紅葉,月老夫妻繫赤繩。』」女王道:「依卿所奏,就著當駕太師作媒,迎陽驛丞主婚,先去驛中與御弟求親。待他許可,寡人卻擺駕出城迎接。」那太師、驛丞領旨出朝。

卻說三藏師徒們在驛廳上正享齋飯,只見外面人報:「當駕太師與我們本官老姆來了。」三藏道:「太師來卻是何意?」八戒道:「怕是女王請我們也。」行者道:「不是相請,就是說親。」三藏道:「悟空,假如不放,強逼成親,卻怎麼是好?」行者道:「師父只管允他,老孫自有處治。」

說不了,二女官早至,對長老下拜。長老一一還禮道:「貧僧出家人,有何德能,敢勞大人下拜?」那太師見長老相貌軒昂,心中暗喜道:「我國中實有造化,這個男子,卻也做得我王之夫。」二官拜畢起來,侍立左右道:「御弟爺爺,萬千之喜了。」三藏道:「我出家人,喜從何來?」太師躬身道:「此處乃西梁女國,國中自來沒個男子。今幸御弟爺爺降臨,臣奉我王旨意,特來求親。」三藏道:「善哉,善哉!我貧僧隻身來到貴地,又無兒女相隨,止有頑徒三個,不知大人求的是那個親事?」驛丞道:「下官才進朝啟奏,我王十分歡喜道,夜來得一吉夢,夢見金屏生彩艷,玉鏡展光明。御弟乃中華上國男兒,我王願以一國之富,招贅御弟爺爺為夫,坐南面稱孤,我王願為帝后。傳旨著太師作媒,下官主婚,故此特來求這親事也。」三藏聞言,低頭不語。太師道:「大丈夫遇時,不可錯過。似此招贅之事,天下雖有,託國之富,世上實稀。請御弟速允,庶好回奏。」長老越加痴啞。

八戒在傍,掬著碓挺嘴叫道:「太師,你去上復國王:我師父乃久修得道的羅漢,決不愛你託國之富,也不愛你傾國之容。快些兒倒換關文,打發他往西去,留我在此招贅,如何?」太師聞說,膽戰心驚,不敢回話。驛丞道:「你雖是個男身,但只形容醜陋,不中我王之意。」八戒笑道:「你甚不通變。常言道:『粗柳簸箕細柳斗,世上誰見男兒醜?』」行者道:「獃子,勿得胡談,任師父尊意,可行則行,可止則止。莫要擔閣了媒妁工夫。」

三藏道:「悟空,憑你怎麼說好?」行者道:「依老孫說,你在這裡也好。自古道『千里姻緣似線牽』哩,那裡再有這般相應處?」三藏道:「徒弟,我們在這裡貪圖富貴,誰去西天取經?卻不望壞了我大唐之帝主也?」太師道:「御弟在上,微臣不敢隱言。我王旨意,原只教求御弟為親,教你三位徒弟赴了會親筵宴,發付領給,倒換關文,往西天取經去哩。」行者道:「太師說得有理。我等不必作難,情願留下師父,與你主為夫。快換關文,打發我們西去。待取經回來,好到此拜爺娘,討盤纏,回大唐也。」那太師與驛丞對行者作禮道:「多謝老師玉成之恩。」八戒道:「太師,切莫要口裡擺菜碟兒。既然我們許諾,且教你主先安排一席,與我們吃鍾肯酒,如何?」太師道:「有有有,就教擺設筵宴來也。」那驛丞與太師歡天喜地,回奏女主不題。

卻說唐長老一把扯住行者,罵道:「你這猴頭,弄殺我也,怎麼說出這般話來,教我在此招婚,你們西天拜佛?我就死也不敢如此。」行者道:「師父放心,老孫豈不知你性情,但只是到此地,遇此人,不得不將計就計。」三藏道:「怎麼叫做將計就計?」行者道:「你若使住法兒不允他,他便不肯倒換關文,不放我們走路。倘或意惡心毒,喝令多人,割了你肉,做甚麼香袋啊,我等豈有善報?一定要使出降魔蕩怪的神通,你知我們的手腳又重,器械又兇,但動動手兒,這一國的人,盡打殺了。他雖然阻當我等,卻不是怪物妖精,還是一國人身;你又平素是個好善慈悲的人,在路上一靈不損:若打殺無限的平人,你心何忍,誠為不善了也。」三藏聽說,道:「悟空,此論最善。但恐女主招我進去,要行夫婦之禮,我怎肯喪元陽,敗壞了佛家德行?走真精,墜落了本教人身?」行者道:「今日准了親事,他一定以皇帝禮,擺駕出城接你。你更不要推辭,就坐他鳳輦龍車,登寶殿,面南坐下。問女王取出御寶印信來,宣我們兄弟進朝,把通關文牒用了印,再請女王寫個手字花押,僉押了交付與我們。一壁廂教擺筵宴,就當與女王會喜,就與我們送行。待筵宴已畢,再叫排駕,只說送我們三人出城,回來與女王配合。哄得他君臣歡悅,更無阻擋之心,亦不起毒惡之念。卻待送出城外,你下了龍車鳳輦,教沙僧伺候左右,伏侍你騎上白馬;老孫卻使個定身法兒,教他君臣人等皆不能動,我們順大路只管西行。行得一晝夜,我卻念個咒,解了術法,還教他君臣們甦醒回城。一則不傷了他的性命,二來不損了你的元神。這叫做『假親脫網』之計,豈非一舉兩全之美也?」三藏聞言,如醉方醒,似夢初覺,樂以忘憂,稱謝不盡道:「深感賢徒高見。」四眾同心合意,正自商量不題。

卻說那太師與驛丞不等宣詔,直入朝門白玉階前,奏道:「主公佳夢最準,魚水之歡就矣。」女王聞奏,捲珠簾,下龍床,啟櫻唇,露銀齒,笑盈盈嬌聲問曰:「賢卿見御弟,怎麼說來?」太師道:「臣等到驛,拜見御弟畢,即備言求親之事。御弟還有推托之辭,幸虧他大徒弟慨然見允,願留他師父與我王為夫,面南稱帝。只教先倒換關文,打發他三人西去;取得經回,卻到此拜認爺娘,討盤費回大唐也。」女王笑道:「御弟再有何說?」太師奏道:「御弟不言,願配我主。只是他那二徒弟,先要吃席肯酒。」

女王聞言,即傳旨,教光祿寺排宴。一壁廂排大駕,出城迎接夫君。眾女官即欽遵王命,打掃宮殿,鋪設庭臺。一班兒擺宴的,火速安排;一班兒擺駕的,流星整備。你看那西梁國雖是婦女之邦,那鑾輿不亞中華之盛。但見:
\begin{quote}
六龍噴彩,雙鳳生祥。六龍噴彩扶車出,雙鳳生祥駕輦來。馥𩡏異香藹,氤氳瑞氣開。金魚玉佩多官擁,寶髻雲鬟眾女排。鴛鴦掌扇遮鑾駕,翡翠珠簾影鳳釵。笙歌音美,絃管聲諧。一片歡情沖碧漢,無邊喜氣出靈臺。三簷羅蓋搖天宇,五色旌旗映御階。此地自來無合巹,女王今日配男才。
\end{quote}

不多時,大駕出城,早到迎陽館驛。忽有人報三藏師徒道:「駕到了。」三藏聞言,即與三徒整衣出廳迎駕。女王捲簾下輦道:「那一位是唐朝御弟?」太師指道:「那驛門外香案前穿襴衣者便是。」女王閃鳳目,簇蛾眉,仔細觀看,果然一表非凡。你看他:
\begin{quote}
丰姿英偉,相貌軒昂。齒白如銀砌,唇紅口四方。頂平額闊天倉滿,目秀眉清地閣長。兩耳有輪真傑士,一身不俗是才郎。好個妙齡聰俊風流子,堪配西梁窈窕娘。
\end{quote}

女王看到那心歡意美之處,不覺淫情汲汲,愛慾恣恣,展放櫻桃小口,呼道:「大唐御弟,還不來占鳳乘鸞也?」三藏聞言,耳紅面赤,羞答答不敢擡頭。

豬八戒在傍,掬著嘴,餳眼觀看那女王,卻也嬝娜。真個:
\begin{quote}
眉如翠羽,肌似羊脂。臉襯桃花瓣,鬟堆金鳳絲。秋波湛湛妖嬈態,春筍纖纖嬌媚姿。斜紅綃飄彩艷,高簪珠翠顯光輝。說甚麼昭君美貌,果然是賽過西施。柳腰微展鳴金珮,蓮步輕移動玉肢。月裡嫦娥難到此,九天仙子怎如斯。宮妝巧樣非凡類,誠然王母降瑤池。
\end{quote}

那獃子看到好處,忍不住口嘴流涎,心頭撞鹿,一時間骨軟筋麻,好便似雪獅子向火,不覺的都化去也。

只見那女王走近前來,一把扯住三藏,俏語嬌聲,叫道:「御弟哥哥,請上龍車,和我同上金鑾寶殿,匹配夫婦去來。」這長老戰兢兢立站不住,似醉如痴。行者在側教道:「師父不必太謙,請共師娘上輦。快快倒換關文,等我們取經去罷。」長老不敢回言,把行者抹了兩抹,止不住落下淚來。行者道:「師父切莫煩惱,這般富貴,不受用還待怎麼哩?」三藏沒及奈何,只得依從,揩了眼淚,強整歡容,移步近前,與女主:
\begin{quote}
同攜素手,共坐龍車。那女主喜孜孜欲配夫妻,這長老憂惶惶只思拜佛。一個要洞房花燭交鴛侶,一個要西宇靈山見世尊。女帝真情,聖僧假意。女帝真情,指望和諧同到老;聖僧假意,牢藏情意養元神。一個喜見男身,恨不得白晝並頭諧伉儷;一個怕逢女色,只思量即時脫網上雷音。二人和會同登輦,豈料唐僧各有心。
\end{quote}

那些文武官見主公與長老同登鳳輦,並肩而坐,一個個眉花眼笑,撥轉儀從,復入城中。孫大聖才教沙僧挑著行李,牽著白馬,隨大駕後邊同行。豬八戒往前亂跑,先到五鳳樓前,嚷道:「好自在,好現成呀。這個弄不成,這個弄不成,吃了喜酒進親才是。」諕得些執儀從引導的女官,一個個回至駕邊道:「主公,那一個長嘴大耳的,在五鳳樓前嚷道要喜酒吃哩。」女主聞奏,與長老倚香肩,偎並桃腮,開檀口,俏聲叫道:「御弟哥哥,長嘴大耳的是你那個高徒?」三藏道:「是我第二個徒弟。他生得食腸寬大,一生要圖口肥,須是先安排些酒食與他吃了,方可行事。」女主急問:「光祿寺安排筵宴,完否?」女官奏道:「已完,設了葷素兩樣,在東閣上哩。」女王又問:「怎麼兩樣?」女官奏道:「臣恐唐朝御弟與高徒等平素吃齋,故有葷素兩樣。」女王卻又笑吟吟,偎著長老的香腮道:「御弟哥哥,你吃葷吃素?」三藏道:「貧僧吃素,但是未曾戒酒。須得幾杯素酒,與我二徒弟吃些。」

說未了,太師啟奏:「請赴東閣會宴。今宵吉日良辰,就可與御弟爺爺成親。明日天開黃道,請御弟爺爺登寶殿,面南,改年號即位。」女王大喜,即與長老攜手相攙,下了龍車,共入端門裡。但見那:
\begin{quote}
風飄仙樂下樓臺,閶闔中間翠輦來。
鳳闕大開光藹藹,皇宮不閉錦排排。
麒麟殿內爐煙裊,孔雀屏邊房影迴。
亭閣崢嶸如上國,玉堂金馬更奇哉。
\end{quote}

既至東閣之下,又聞得一派笙歌聲韻美,又見兩行紅粉貌嬌嬈。正中堂排設兩般盛宴:左邊上首是素筵,右邊上首是葷筵。下兩路盡是單席。那女王斂袍袖,十指尖尖,奉著玉杯,便來安席。行者近前道:「我師徒都是吃素,先請師父坐了左手素席,轉下三席,分左右,我兄弟們好坐。」太師喜道:「正是,正是。師徒如父子也,不可並肩。」眾女官連忙調了席面。女王一一傳杯,安了他弟兄三位。行者又與唐僧丟個眼色,教師父回禮。三藏下來,卻也擎玉杯,與女王安席。那些文武官朝上拜謝了皇恩,各依品從,分坐兩邊,才住了音樂請酒。

那八戒那管好歹,放開肚子,只情吃起。也不管甚麼玉屑、米飯、蒸餅、糖糕、蘑菇、香蕈、筍芽,木耳、黃花菜、石花菜、紫菜、蔓菁、芋頭、蘿菔、山藥、黃精,一骨辣噇了個罄盡。喝了五七杯酒。口裡嚷道:「看添換來,拿大觥來,再吃幾觥,各人幹事去。」沙僧問道:「好筵席不吃,還要幹甚事?」獃子笑道:「古人云:『造弓的造弓,造箭的造箭。』我們如今招的招,嫁的嫁,取經的還去取經,走路的還去走路,莫只管貪杯誤事。快早兒打發關文。正是將軍不下馬,各自奔前程。」女王聞說,即命取大杯來。近侍官連忙取幾個鸚鵡杯、鸕鶿杓、金叵羅、銀鑿落、玻璃盞、水晶盆、蓬萊碗、琥珀鍾,滿斟玉液,連注瓊漿,果然都各飲一巡。

三藏欠身而起,對女王合掌道:「陛下,多蒙盛設,酒已夠了。請登寶殿,倒換關文,趕天早,送他三人出城罷。」女王依言,攜著長老,散了筵宴,上金鑾寶殿,即讓長老即位。三藏道:「不可,不可。適太師言過,明日天開黃道,貧僧才敢即位稱孤。今日即印關文,打發他去也。」女王依言,仍坐了龍床,即取金交椅一張,放在龍床左手,請唐僧坐了。叫徒弟們拿上通關文牒來。大聖便教沙僧解開包袱,取出關文。大聖將關文雙手捧上。那女王細看一番,上有大唐皇帝寶印九顆,下有寶象國印、烏雞國印、車遲國印。女王看罷,嬌滴滴笑語道:「御弟哥哥又姓陳?」三藏道:「俗家姓陳,法名玄奘。因我唐王聖恩認為御弟,賜姓我為唐也。」女王道:「關文上如何沒有高徒之名?」三藏道:「三個頑徒,不是我唐朝人物。」女王道:「既不是你唐朝人物,為何肯隨你來?」三藏道:「大的個徒弟,祖貫東勝神洲傲來國人氏;第二個乃西牛賀洲烏斯莊人氏;第三個乃流沙河人氏:他三人都因罪犯天條,南海觀世音菩薩解脫他苦,秉善皈依,將功折罪,情願保護我上西天取經。皆是途中收得,故此未註法名在牒。」女王道:「我與你添註法名,好麼?」三藏道:「但憑陛下尊意。」女王即令取墨筆來,濃磨香翰,飽潤香毫,牒文之後,寫上孫悟空、豬悟能、沙悟淨三人名諱。卻才取出御印,端端正正印了;又畫個手字花押。傳將下去。孫大聖接了,教沙僧包裹停當。

那女王又賜出碎金碎銀一盤,下龍床遞與行者道:「你三人將此權為路費,早上西天;待汝等取經回來,寡人還有重謝。」行者道:「我們出家人,不受金銀,途中自有乞化之處。」女王見他不受,又取出綾錦十疋,對行者道:「汝等行色匆匆,裁製不及,將此路上做件衣服遮寒。」行者道:「出家人穿不得綾錦,自有護體布衣。」女王見他不受,教:「取御米三升,在路權為一飯。」八戒聽說個「飯」字,便就接了,捎在包袱之間。行者道:「兄弟,行李見今沉重,且倒有氣力挑米?」八戒笑道:「你那裡知道,米好的是個日消貨,只消一頓飯,就了帳也。」遂此合掌謝恩。

三藏道:「敢煩陛下相同貧僧送他三人出城,待我囑付他們幾句,教他好生西去,我卻回來,與陛下永受榮華,無掛無牽,方可會鸞交鳳友也。」女王不知是計,便傳旨擺駕,與三藏並倚香肌,同登鳳輦,出西城而去。滿城中都盞添淨水,爐降真香:一則看女王鑾駕,二來看御弟男身。沒老沒小,盡是粉容嬌面,綠鬢雲鬟之輩。不多時,大駕出城,到西關之外。

行者、八戒、沙僧同心合意,結束整齊,徑迎著鑾輿,厲聲高叫道:「那女王不必遠送,我等就此拜別。」長老慢下龍車,對女王拱手道:「陛下請回,讓貧僧取經去也。」女王聞言,大驚失色,扯住唐僧道:「御弟哥哥,我願將一國之富,招你為夫,明日高登寶位,即位稱君,我願為君之后,喜筵通皆吃了,如何卻又變卦?」八戒聽說,發起個風來,把嘴亂扭,耳朵亂搖,闖至駕前,嚷道:「我們和尚家和你這粉骷髏做甚夫妻?放我師父走路。」那女王見他那等撒潑弄醜,諕得魂飛魄散,跌入輦駕之中。沙僧卻把三藏搶出人叢,伏侍上馬。

只見那路傍閃出一個女子,喝道:「唐御弟,那裡走?我和你耍風月兒去來。」沙僧罵道:「賊輩無知!」掣寶杖劈頭就打。那女子弄陣旋風,嗚的一聲,把唐僧攝將去了,無影無蹤,不知下落何處。咦!正是:
\begin{quote}
脫得煙花網,又遇風月魔。
\end{quote}

畢竟不知那女子是人是怪,老師父的性命得死得生,且聽下回分解。
