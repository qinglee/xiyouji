
\chapter{四海千山皆拱伏 九幽十類盡除名}

卻說美猴王榮歸故里,自剿了混世魔王,奪了一口大刀。逐日操演武藝,教小猴砍竹為標,削木為刀,治旗幡,打哨子,一進一退,安營下寨,頑耍多時。忽然靜坐處,思想道:「我等在此,恐作耍成真,或驚動人王,或有禽王、獸王認此犯頭,說我們操兵造反,興師來相殺,汝等都是竹竿木刀,如何對敵?須得鋒利劍戟方可。如今奈何?」眾猴聞說,個個驚恐道:「大王所見甚長,只是無處可取。」正說間,轉上四個老猴,兩個是赤尻馬猴,兩個是通背猿猴,走在面前道:「大王,若要治鋒利器械,甚是容易。」悟空道:「怎見容易?」四猴道:「我們這山向東去,有二百里水面,那廂乃傲來國界。那國界中有一王位,滿城中軍民無數,必有金銀銅鐵等匠作。大王若去那裡,或買或造些兵器,教演我等,守護山場,誠所謂保泰長久之機也。」悟空聞說,滿心歡喜道:「汝等在此頑耍,待我去來。」

好猴王,急縱觔斗雲,霎時間過了二百里水面。果見那廂有座城池,六街三市,萬戶千門,來來往往,人都在光天化日之下。悟空心中想道:「這裡定有現成的兵器,我待下去買他幾件,還不如使個神通覓他幾件倒好。」他就捻起訣來,念動咒語,向巽地上吸一口氣,呼的吹將去,便是一陣風,飛沙走石,好驚人也:
\begin{quote}
炮雲起處蕩乾坤,黑霧陰霾大地昏。
江海波翻魚蟹怕,山林樹折虎狼奔。
諸般買賣無商旅,各樣生涯不見人。
殿上君王歸內院,階前文武轉衙門。
千秋寶座都吹倒,五鳳高樓幌動根。
\end{quote}

風起處,驚散了那傲來國君王,三街六巿,都慌得關門閉戶,無人敢走。悟空才按下雲頭,徑闖入朝門裡,直尋到兵器館、武庫中,打開門扇看時,那裡面無數器械:刀、槍、劍、戟、斧、鉞、毛、鐮、鞭、鈀、撾、簡、弓、弩、叉、矛,件件俱備。一見甚喜道:「我一人能拿幾何?還使個分身法搬將去罷。」好猴王,即拔一把毫毛,入口嚼爛,噴將出去,念動咒語,叫聲:「變!」變做千百個小猴,都亂搬亂搶,有力的拿五七件,力小的拿三二件,盡數搬個罄淨。徑踏雲頭,弄個攝法,喚轉狂風,帶領小猴,俱回本處。

卻說那花果山大小猴兒,正在那洞門外頑耍,忽聽得風聲響處,見半空中丫丫叉叉,無邊無岸的猴精,諕得都亂跑亂躲。少時,美猴王按落雲頭,收了雲霧,將身一抖,收了毫毛,將兵器都亂堆在山前,叫道:「小的們,都來領兵器。」眾猴看時,只見悟空獨立在平陽之地,俱跑來叩頭問故。悟空將前使狂風、搬兵器,一應事說了一遍。眾猴稱謝畢,都去搶刀奪劍,撾斧爭槍,扯弓扳弩,吆吆喝喝,耍了一日。

次日,依舊排營。悟空會聚群猴,計有四萬七千餘口。早驚動滿山怪獸,都是些狼、蟲、虎、豹、麖、麂、獐、、狐、狸、獾、狢、獅、象、狻猊、猩猩、熊、鹿、野豕、山牛、羚羊、青兕、狡兔、神獒各樣妖王,共有七十二洞,都來參拜猴王為尊。每年獻貢,四時點卯。也有隨班操演的,也有隨節徵糧的。齊齊整整,把一座花果山造得似鐵桶金城。各路妖王,又有進金鼓、進彩旗、進盔甲的,紛紛攘攘,日逐家習舞興師。

美猴王正喜間,忽對眾說道:「汝等弓弩熟諳,兵器精通,奈我這口刀著實榔槺,不遂我意,奈何?」四老猴上前啟奏道:「大王乃是仙聖,凡兵是不堪用。但不知大王水裡可能去得?」悟空道:「我自聞道之後,有七十二般地煞變化之功,觔斗雲有莫大的神通;善能隱身遯身,起法攝法。上天有路,入地有門;步日月無影,入金石無礙;水不能溺,火不能焚。那些兒去不得?」四猴道:「大王既有此神通,我們這鐵板橋下,水通東海龍宮。大王若肯下去,尋著老龍王,問他要件甚麼兵器,卻不趁心?」悟空聞言,甚喜道:「等我去來。」

好猴王,跳至橋頭,使一個閉水法,捻著訣,撲的鑽入波中,分開水路,徑入東洋海底。正行間,忽見一個巡海的夜叉,擋住問道:「那推水來的,是何神聖?說個明白,好通報迎接。」悟空道:「吾乃花果山天生聖人孫悟空,是你老龍王的緊鄰,為何不識?」那夜叉聽說,急轉水晶宮傳報道:「大王,外面有個花果山天生聖人孫悟空,口稱是大王緊鄰,將到宮也。」東海龍王敖廣即忙起身,與龍子、龍孫、蝦兵、蟹將出宮迎道:「上仙請進,請進。」直至宮裡相見,上坐獻茶畢,問道:「上仙幾時得道?授何仙術?」悟空道:「我自生身之後,出家修行,得一個無生無滅之體。近因教演兒孫,守護山洞,奈何沒件兵器。久聞賢鄰享樂瑤宮貝闕,必有多餘神器,特來告求一件。」龍王見說,不好推辭,即著鱖都司取出一把大桿刀奉上。悟空道:「老孫不會使刀,乞另賜一件。」龍王又著鮊太尉領鱔力士,擡出一桿九股叉來。悟空跳下來,接在手中,使了一路,放下道:「輕,輕,輕,又不趁手。再乞另賜一件。」龍王笑道:「上仙,你不看看,這叉有三千六百斤重哩。」悟空道:「不趁手,不趁手。」龍王心中恐懼,又著鯁提督、鯉總兵擡出一柄畫桿方天戟。那戟有七千二百斤重。悟空見了,跑近前,接在手中,丟幾個架子,撒兩個解數,插在中間道:「也還輕,輕,輕。」老龍王一發害怕道:「上仙,我宮中只有這根戟重,再沒甚麼兵器了。」悟空笑道:「古人云:『愁海龍王沒寶』哩!你再去尋尋看,若有可意的,一一奉價。」龍王道:「委的再無。」

正說處,後面閃過龍婆、龍女道:「大王,觀看此聖,決非小可。我們這海藏中,那一塊天河定底的神珍鐵,這幾日霞光艷艷,瑞氣騰騰,敢莫是該出現,遇此聖也?」龍王道:「那是大禹治水之時,定江海淺深的一個定子,是一塊神鐵,能中何用?」龍婆道:「莫管他用不用,且送與他,憑他怎麼改造,送出宮門便了。」老龍王依言,盡向悟空說了。悟空道:「拿出來我看。」龍王搖手道:「扛不動,擡不動,須上仙親去看看。」悟空道:「在何處?你引我去。」

龍王果引導至海藏中間,忽見金光萬道。龍王指定道:「那放光的便是。」悟空撩衣上前,摸了一把,乃是一根鐵柱子,約有斗來粗,二丈有餘長。他儘力兩手撾過道:「忒粗忒長些,再短細些方可用。」說畢,那寶貝就短了幾尺,細了一圍。悟空又顛一顛道:「再細些更好。」那寶貝真個又細了幾分。悟空十分歡喜,拿出海藏看時,原來兩頭是兩個金箍,中間乃一段烏鐵。緊挨箍有鐫成的一行字,喚做:「如意金箍棒,重一萬三千五百斤。」心中暗喜道:「想必這寶貝如人意。」一邊走,一邊心思口念,手顛著道:「再短細些更妙。」拿出外面,只有二丈長短,碗口粗細。

你看他弄神通,丟開解數,打轉水晶宮裡。諕得老龍王膽戰心驚,小龍子魂飛魄散,龜鱉黿鼉皆縮頸,魚蝦鰲蟹盡藏頭。悟空將寶貝執在手中,坐在水晶宮殿上,對龍王笑道:「多謝賢鄰厚意。」龍王道:「不敢,不敢。」悟空道:「這塊鐵雖然好用,還有一說。」龍王道:「上仙還有甚說?」悟空道:「當時若無此鐵,倒也罷了;如今手中既拿著他,身上更無衣服相趁,奈何?你這裡若有披掛,索性送我一副,一總奉謝。」龍王道:「這個卻是沒有。」悟空道:「一客不煩二主。若沒有,我也定不出此門。」龍王道:「煩上仙再轉一海,或者有之。」悟空又道:「走三家不如坐一家。千萬告求一件。」龍王道:「委的沒有,如有即當奉承。」悟空道:「真個沒有?就和你試試此鐵!」龍王慌了道:「上仙,切莫動手,切莫動手,待我看舍弟處可有,當送一副。」悟空道:「令弟何在?」龍王道:「舍弟乃南海龍王敖欽、北海龍王敖順、西海龍王敖閏是也。」悟空道:「我老孫不去,不去。俗語謂『賒三不敵見二』,只望你隨高就低的送一副便了。」老龍道:「不須上仙去。我這裡有一面鐵鼓、一口金鐘,凡有緊急事,擂得鼓響,撞得鐘鳴,舍弟們就頃刻而至。」悟空道:「既是如此,快些去擂鼓撞鐘。」真個那鼉將便去撞鐘,鱉帥即來擂鼓。

少時,鐘鼓響處,果然驚動那三海龍王,須臾來到,一齊在外面會著。敖欽道:「大哥,有甚緊事,擂鼓撞鐘?」老龍道:「賢弟,不好說。有一個花果山甚麼天生聖人,早間來認我做鄰居。後要求一件兵器,獻鋼叉嫌小,奉畫戟嫌輕;將一塊天河定底神珍鐵,自己拿出手,丟了些解數。如今坐在宮中,又要索甚麼披掛。我處無有,故響鐘鳴鼓,請賢弟來。你們可有甚麼披掛,送他一副,打發出門去罷了。」敖欽聞言,大怒道:「我兄弟們點起兵拿他不是?」老龍道:「莫說拿,莫說拿。那塊鐵,挽著些兒就死,磕著些兒就亡;挨挨兒皮破,擦擦兒觔傷。」西海龍王敖閏說:「二哥不可與他動手。且只湊副披掛與他,打發他出了門,啟表奏上上天,天自誅也。」北海龍王敖順道:「說的是。我這裡有一雙藕絲步雲履哩。」西海龍王敖閏道:「我帶了一副鎖子黃金甲哩。」南海龍王敖欽道:「我有一頂鳳翅紫金冠哩。」老龍大喜,引入水晶宮相見了,以此奉上。悟空將金冠、金甲、雲履都穿戴停當,使動如意棒,一路打出去,對眾龍道:「聒噪,聒噪。」四海龍王甚是不平,一邊商議進表上奏不題。

你看這猴王,分開水道,徑回鐵板橋頭,攛將上去。只見四個老猴領著眾猴,都在橋邊等候。忽然見悟空跳出波外,身上更無一點水濕,金燦燦的走上橋來。諕得眾猴一齊跪下道:「大王好華彩耶!好華彩耶!」悟空滿面春風,高登寶座,將鐵棒豎在當中。那些猴不知好歹,都來拿那寶貝,卻便似蜻蜓撼鐵樹,分毫也不能禁動。一個個咬指伸舌道:「爺爺呀!這般重,虧你怎的拿來也!」悟空近前,舒開手,一把撾起,對眾笑道:「物各有主。這寶貝鎮於海藏中,也不知幾千百年,可可的今歲放光。龍王只認做是塊黑鐵,又喚做天河鎮底神珍。那廝每都扛擡不動,請我親去拿之。那時此寶有二丈多長,斗來粗細。被我撾他一把,意思嫌大,他就小了許多;再教小些,他又小了許多;再教小些,他又小了許多。急對天光看處,上有一行字,乃『如意金箍棒,一萬三千五百斤』。你都站開,等我再叫他變一變著。」他將那寶貝顛在手中,叫:「小!小!小!」即時就小做一個繡花針兒相似,可以揌在耳朵裡面藏下。眾猴駭然,叫道:「大王,還拿出來耍耍。」猴王真個去耳朵裡拿出,托放掌上叫:「大!大!大!」即又大做斗來粗細,二丈長短。他弄到歡喜處,跳上橋,走出洞外,將寶貝揝在手中,使一個法天像地的神通,把腰一躬,叫聲:「長!」他就長的高萬丈,頭如泰山,腰如峻嶺,眼如閃電,口似血盆,牙如劍戟;手中那棒,上抵三十三天,下至十八層地獄。把些虎豹狼蟲、滿山群怪、七十二洞妖王,都諕得磕頭禮拜,戰兢兢魄散魂飛。霎時收了法像,將寶貝還變做個繡花針兒,藏在耳內,復歸洞府。慌得那各洞妖王,都來參賀。

此時遂大開旗鼓,響振銅鑼,廣設珍饈百味,滿斟椰液萄漿,與眾飲宴多時,卻又依前教演。猴王將那四個老猴封為健將,將兩個赤尻馬猴喚做馬、流二元帥,兩個通背猿猴喚做崩、芭二將軍。將那安營下寨、賞罰諸事,都付與四健將維持。

他放下心,日逐騰雲駕霧,遨遊四海,行樂千山。施武藝,遍訪英豪;弄神通,廣交賢友。此時又會了個七弟兄,乃牛魔王、蛟魔王、鵬魔王、獅狔王、獼猴王、狨王,連自家美猴王七個。日逐講文論武,走斝傳觴,絃歌吹舞,朝去暮回,無般兒不樂。把那萬里之遙,只當庭闈之路;所謂點頭徑過三千里,扭腰八百有餘程。

一日,在本洞吩咐四健將安排筵宴,請六王赴飲,殺牛宰馬,祭天享地,著眾怪跳舞歡歌,俱吃得酩酊大醉。送六王出去,卻又賞大小頭目。攲在鐵板橋邊松陰之下,霎時間睡著。四健將領眾圍護,不敢高聲。只見那美猴王睡裡,見兩人拿一張批文,上有「孫悟空」三字,走近身,不容分說,套上繩,就把美猴王的魂靈兒索了去,踉踉蹌蹌,直帶到一座城邊。猴王漸覺酒醒,忽擡頭觀看,那城上有一鐵牌,牌上有三個大字,乃「幽冥界」。美猴王頓然醒悟道:「幽冥界乃閻王所居,何為到此?」那兩人道:「你今陽壽該終,我兩人領批,勾你來也。」猴王聽說,道:「我老孫超出三界之外,不在五行之中,已不伏他管轄,怎麼朦朧,又敢來勾我?」那兩個勾死人,只管扯扯拉拉,定要拖他進去。這猴王惱起性來,耳朵中掣出寶貝,幌一幌,碗來粗細。略舉手,把兩個勾死人打為肉醬。自解其索,丟開手,掄著棒,打入城中。諕得那牛頭鬼東躲西藏,馬面鬼南奔北跑。眾鬼卒奔上森羅殿,報著:「大王,禍事!禍事!外面有一個毛臉雷公打將來了。」

慌得那十代冥王急整衣來看,見他相貌兇惡,即排下班次,應聲高叫道:「上仙留名!上仙留名!」猴王道:「你既認不得我,怎麼差人來勾我?」十王道:「不敢,不敢。想是差人差了。」猴王道:「我本是花果山水簾洞天生聖人孫悟空。你等是甚麼官位?」十王躬身道:「我等是陰間天子十代冥王。」悟空道:「快報名來,免打。」十王道:「我等是秦廣王、初江王、宋帝王、忤官王、閻羅王、平等王、泰山王、都市王、卞城王、轉輪王。」悟空道:「汝等既登王位,乃靈顯感應之類,為何不知好歹?我老孫修了仙道,與天齊壽,超昇三界之外,跳出五行之中,為何著人拘我?」十王道:「上仙息怒。普天下同名同姓者多,敢是那勾死人錯走了也?」悟空道:「胡說!胡說!常言道:『官差吏差,來人不差。』你快取生死簿子來我看!」十王聞言,即請上殿查看。

悟空執著如意棒,徑登森羅殿上,正中間南面坐下。十王即命掌案的判官取出文簿來查。那判官不敢怠慢,便到司房裡捧出五六簿文書並十類簿子。逐一查看:臝蟲、毛蟲、羽蟲、昆蟲、鱗介之屬,俱無他名。又看到猴屬之類,原來這猴似人相,不入人名;似臝蟲,不居國界;似走獸,不伏麒麟管;似飛禽,不受鳳凰轄。另有個簿子,悟空親自檢閱,直到那「魂」字一千三百五十號上,方注著孫悟空名字,乃「天產石猴,該壽三百四十二歲,善終」。悟空道:「我也不記壽數幾何,且只消了名字便罷。取筆過來。」那判官慌忙捧筆,飽掭濃墨。悟空拿過簿子,把猴屬之類,但有名者,一概勾之。捽下簿子道:「了帳,了帳,今番不伏你管了。」一路棒,打出幽冥界。那十王不敢相近,都去翠雲宮,同拜地藏王菩薩,商量啟表,奏聞上天,不在話下。

這猴王打出城中,忽然絆著一個草紇繨,跌了個躘踵,猛的醒來,乃是南柯一夢。才覺伸腰,只聞得四健將與眾猴高叫道:「大王,吃了多少酒,睡這一夜,還不醒來?」悟空道:「睡還小可,我夢見兩個人來此勾我,把我帶到幽冥界城門之外,卻才醒悟。是我顯神通,直嚷到森羅殿,與那十王爭吵,將我們的生死簿子看了,但有我等名號,俱是我勾了,都不伏那廝所轄也。」眾猴磕頭禮謝。自此,山猴多有不老者,以陰司無名故也。

美猴王言畢前事,四健將報知各洞妖王,都來賀喜。不幾日,六個義兄弟又來拜賀,一聞銷名之故,又個個歡喜,每日聚樂不題。

卻表啟那個高天上聖大慈仁者玉皇大天尊玄穹高上帝,一日駕坐金闕雲宮靈霄寶殿,聚集文武仙卿早朝之際,忽有丘弘濟真人啟奏道:「萬歲,通明殿外有東海龍王敖廣進表,聽天尊宣詔。」玉皇傳旨:「著宣來。」敖廣宣至靈霄殿下,禮拜畢,傍有引奏仙童接上表文。玉皇從頭看過。表曰:
\begin{quote}
「水元下界東勝神洲東海小龍臣敖廣啟奏大天聖主玄穹高上帝君:近因花果山生、水簾洞住妖仙孫悟空者,欺虐小龍,強坐水宅,索兵器,施法施威;要披掛,騁兇騁勢。驚傷水族,諕走龜鼉。南海龍戰戰兢兢,西海龍悽悽慘慘,北海龍縮首歸降。臣敖廣舒身下拜,獻神珍之鐵棒,鳳翅之金冠,與那鎖子甲、步雲履,以禮送出。他仍弄武藝,顯神通,但云:『聒噪!聒噪!』果然無敵,甚為難制。臣今啟奏,伏望聖裁。懇乞天兵,收此妖孽,庶使海嶽清寧,下元安泰。奉奏。」
\end{quote}

聖帝覽畢,傳旨:「著龍神回海,朕即遣將擒拿。」老龍王頓首謝去。

下面又有葛仙翁天師啟奏道:「萬歲,有冥司秦廣王賫奉幽冥教主地藏王菩薩表文進上。」傍有傳言玉女接上表文。玉皇亦從頭看過。表曰:
\begin{quote}
「幽冥境界,乃地之陰司。天有神而地有鬼,陰陽輪轉;禽有生而獸有死,反復雌雄。生生化化,孕女成男,此自然之數,不能易也。今有花果山水簾洞天產妖猴孫悟空,逞惡行兇,不服拘喚。弄神通,打絕九幽鬼使;恃勢力,驚傷十代慈王。大鬧森羅,強銷名號。致使猴屬之類無拘,獼猴之畜多壽;寂滅輪迴,各無生死。貧僧具表,冒瀆天威。伏乞調遣神兵,收降此妖,整理陰陽,永安地府。謹奏。」
\end{quote}

玉皇覽畢,傳旨:「著冥君回歸地府,朕即遣將擒拿。」秦廣王亦頓首謝去。

大天尊宣眾文武仙卿,問曰:「這妖猴是幾年產育,何代出身,卻就這般有道?」一言未已,班中閃出千里眼、順風耳道:「這猴乃三百年前天產石猴。當時不以為然,不知這幾年在何方修煉成仙,降龍伏虎,強銷死籍也。」玉帝道:「那路神將下界收伏?」言未已,班中閃出太白長庚星,俯伏啟奏道:「上聖,三界中凡有九竅者,皆可修仙。奈此猴乃天地育成之體,日月孕就之身,他也頂天履地,服露餐霞,今既修成仙道,有降龍伏虎之能,與人何以異哉?臣啟陛下,可念生化之慈恩,降一道招安聖旨,把他宣來上界,授他一個大小官職,與他籍名在籙,拘束此間。若受天命,後再陞賞;若違天命,就此擒拿。一則不動眾勞師,二則收仙有道也。」玉帝聞言甚喜,道:「依卿所奏。」即著文曲星官修詔,著太白金星招安。

金星領了旨,出南天門外,按下祥雲,直至花果山水簾洞,對眾小猴道:「我乃天差天使,有聖旨在此,請你大王上界。快快報知。」洞外小猴一層層傳至洞天深處,道:「大王,外面有一老人,背著一角文書,言是上天差來的天使,有聖旨請你也。」美猴王聽得大喜,道:「我這兩日正思量要上天走走,卻就有天使來請。」叫:「快請進來。」猴王急整衣冠,門外迎接。金星徑入當中,面南立定道:「我是西方太白金星,奉玉帝招安聖旨,下界請你上天,拜受仙籙。」悟空笑道:「多感老星降臨。」教小的們安排筵宴款待。金星道:「聖旨在身,不敢久留。就請大王同往,待榮遷之後,再從容敘也。」悟空道:「承光顧,空退,空退。」即喚四健將,吩咐:「謹慎教演兒孫,待我上天去看看路,卻好帶你們上去同居住也。」四健將領諾。

這猴王與金星縱起雲頭,昇在空霄之上。正是那:
\begin{quote}
高遷上品天仙位,名列雲班寶籙中。
\end{quote}

畢竟不知授個甚麼官爵,且聽下回分解。
