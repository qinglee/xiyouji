
\chapter{拯救駝羅禪性穩 脫離穢污道心清}

話說三藏四眾躲離了小西天,欣然上路。行經個月程途,正是春深花放之時,見了幾處園林皆綠暗,一番風雨又黃昏。三藏勒馬道:「徒弟啊,天色晚矣,往那條路上求宿去?」行者笑道:「師父放心。若是沒有借宿處,我三人都有些本事,叫八戒砍草,沙和尚扳松,老孫會做木匠,就在那路上搭個蓬庵,好道也住得年把,你忙怎的?」八戒道:「哥呀,這個所在豈是住場?滿山多虎豹狼蟲,遍地有魑魅魍魎,白日裡尚且難行,黑夜裡怎生敢宿?」行者道:「獃子,越發不長進了。不是老孫海口,只這條棒子揝在手裡,就是塌下天來,也撐得住。」

師徒們正然講論,忽見一座山莊不遠。行者道:「好了,有宿處了。」長老問:「在何處?」行者指道:「那樹叢裡不是個人家?我們去借宿一宵,明早走路。」長老欣然促馬,至莊門外下馬,只見那柴扉緊閉。長老敲門道:「開門,開門。」裡面有一老者,手拖藜杖,足踏蒲鞋,頭頂烏巾,身穿素服,開了門,便問:「是甚人在此大呼小叫?」三藏合掌當胸,躬身施禮道:「老施主,貧僧乃東土差往西天取經者。適到貴地,天晚,特造尊府借宿一宵,萬望方便方便。」老者道:「和尚,你要西行,卻是去不得啊。此處乃小西天,若到大西天,路途甚遠。且休道前去艱難,只這個地方已此難過。」三藏問:「怎麼難過?」老者用手指道:「我這莊村西去三十餘里,有一條稀柿衕,山名七絕。」三藏道:「何為『七絕』?」老者道:「這山徑過有八百里,滿山盡是柿果。古云:『柿樹有七絕:一,益壽;二,多陰;三,無鳥巢;四,無蟲;五,霜葉可玩;六,嘉實;七,枝葉肥大。』故名七絕山。我這敝處地闊人稀,那深山亙古無人走到。每年家熟爛柿子落在路上,將一條夾石衚衕盡皆填滿,又被雨露雪霜經黴過夏,作成一路污穢,這方人家俗呼為『稀屎衕』。但刮西風,有一股穢氣,就是淘東圊也不似這般惡臭。如今正值春深,東南風大作,所以還不聞見也。」三藏心中煩悶不言。

行者忍不住,高叫道:「你這老兒甚不通便,我等遠來投宿,你就說出這許多話來諕人。十分你家窄逼沒處睡,我等在此樹下蹲一蹲,也就過了此宵,何故這般絮聒?」那老者見了他相貌醜陋,便也擰住口,驚嘬嘬的硬著膽,喝了一聲,用藜杖指定道:「你這廝骨撾臉,磕額頭,塌鼻子,凹頡腮,毛眼毛睛,癆病鬼,不知高低,尖著個嘴,敢來衝撞我老人家?」行者陪笑道:「老官兒,你原來有眼無珠,不識我這癆病鬼哩。相法云:『形容古怪,石中有美玉之藏。』你若以言貌取人,乾淨差了。我雖醜便醜,卻倒有些手段。」老者道:「你是那方人氏?姓甚名誰?有何手段?」行者笑道:「我
\begin{quote}
祖居東勝大神洲,花果山前自幼修。
身拜靈臺方寸祖,學成武藝甚全周:
也能攪海降龍母,善會擔山趕日頭;
縛怪擒魔稱第一,移星換斗鬼神愁。
偷天轉地英名大,我是變化無窮美石猴。」
\end{quote}

老者聞言,回嗔作喜,躬著身,便教:「請,請入寒舍安置。」遂此四眾牽馬挑擔,一齊進去。只見那荊針棘刺,鋪設兩邊。二層門是磚石壘的牆壁,又是荊棘苫蓋。入裡才是三間瓦房。老者便扯椅安坐待茶,又叫辦飯。少頃,移過桌子,擺著許多麵觔、豆腐、芋苗、蘿白、辣芥、蔓菁、香稻米飯、醋燒葵湯,師徒們盡飽一餐。

吃畢,八戒扯過行者,背云:「師兄,這老兒始初不肯留宿,今返設此盛齋,何也?」行者道:「這個能值多少錢?到明日,還要他十果十菜的送我們哩。」八戒道:「不羞,憑你那幾句大話,哄他一頓飯吃了,明日卻要跑路,他又管待送你怎的?」行者道:「不要忙,我自有個處治。」

不多時,漸漸黃昏,老者又叫掌燈。行者躬身問道:「公公高姓?」老者道:「姓李。」行者道:「貴地想就是李家莊了?」老者道:「不是,這裡喚做駝羅莊,共有五百多人家居住。別姓俱多,惟我姓李。」行者道:「李施主,府上有何善意,賜我等盛齋?」那老者起身道:「才聞得你說會拿妖怪,我這裡卻有個妖怪,累你替我們拿拿,自有重謝。」行者就朝上唱個喏道:「承照顧了。」

八戒道:「你看他惹禍,聽見說拿妖怪,就是他外公也不這般親熱,預先就唱個喏。」行者道:「賢弟,你不知,我唱個喏就是下了個定錢,他再不去請別人了。」三藏聞言道:「這猴兒,凡事便要自專。倘或那妖精神通廣大,你拿他不住,可不是我出家人打誑語麼?」行者笑道:「師父莫怪,等我再問了看。」

那老者道:「還問甚?」行者道:「你這貴處,地勢清平,又許多人家居住,更不是偏僻之方,有甚麼妖精敢上你這高門大戶?」老者道:「實不瞞你說,我這裡久矣康寧。只這三年六月間,忽然一陣風起。那時人家甚忙,打麥的在場上,插秧的在田裡,俱著了忙,只說是天變了。誰知風過處,有個妖精,將人家牧放的牛馬吃了,豬羊吃了,見雞鵝囫圇咽,遇男女夾活吞。自從那次,這二年常來傷害。長老啊,你若有手段,拿了妖怪,掃淨此土,我等決然重謝,不敢輕慢。」行者道:「這個卻是難拿。」八戒道:「真是難拿,難拿。我們乃行腳僧,借宿一宵,明日走路,拿甚麼妖精?」老者道:「你原來是騙飯吃的和尚。初見時誇口弄舌,說會換斗移星,降妖縛怪,及說起此事,就推卻難拿。」

行者道:「老兒,妖精好拿,只是你這方人家不齊心,所以難拿。」老者道:「怎見得人心不齊?」行者道:「妖精攪擾了三年,也不知傷害了多少生靈。我想著每家只出銀一兩,五百家可湊五百兩銀子,不拘到那裡,也尋一個法官把妖拿了,卻怎麼就甘受他三年磨折?」老者道:「若論說使錢,好道也羞殺人,我們那家不花費三五兩銀子?前年曾訪著山南裡有個和尚,請他到此拿妖,未曾得勝。」行者道:「那和尚怎的拿來?」老者道:
\begin{quote}
那個僧伽,披領袈裟。先談、《孔雀》,後念《法華》。香焚爐內,手把鈴拿。正然念處,驚動妖邪。風生雲起,徑至莊家。僧和怪鬥,其實堪誇:一遞一拳搗,一遞一把抓。和尚還相應,相應沒頭髮。須臾妖怪勝,徑直返煙霞。原來晒乾疤。我等近前看,光頭打的似個爛西瓜。」
\end{quote}

行者笑道:「這等說,吃了虧也。」老者道:「他只拚得一命,還是我們吃虧:與他買棺木殯葬,又把些銀子與他徒弟。那徒弟心還不歇,至今還要告狀,不得乾淨。」

行者道:「可曾再請甚麼人拿他?」老者道:「舊年又請了一個道士。」行者道:「那道士怎麼拿他?」老者道:「那道士:
\begin{quote}
頭戴金冠,身穿法衣。令牌敲響,符水施為。驅神使將,拘到妖魑。狂風滾滾,黑霧迷迷。即與道士,兩個相持。鬥到天晚,怪返雲霓。乾坤清朗朗,我等眾人齊。出來尋道士,渰死在山溪。撈得上來大家看,卻如一個落湯雞!」
\end{quote}

行者笑道:「這等說,也吃虧了。」老者道:「他也只捨得一命,我們又使夠悶數錢糧。」

行者道:「不打緊,不打緊,等我替你拿他來。」老者道:「你若果有手段拿得他,我請幾個本莊長者與你寫個文書:若得勝,憑你要多少銀子相謝,半分不少;如若有虧,切莫和我等放賴,各聽天命。」行者笑道:「這老兒被人賴怕了。我等不是那樣人,快請長者去。」

那老者滿心歡喜,即命家僮請幾個左鄰、右舍、表弟、姨兄、親家、朋友,共有八九位老者,都來相見,會了唐僧,言及拿妖一事,無不欣然。眾老問:「是那一位高徒去拿?」行者叉手道:「是我小和尚。」眾老悚然道:「不濟,不濟。那妖精神通廣大,身體狼犺。你這個長老瘦瘦小小,還不夠他填牙齒縫哩。」行者笑道:「老官兒,你估不出人來。我小自小,結實,都是『吃了磨刀水的秀氣在內』哩。」眾老見說,只得依從道:「長老,拿住妖精,你要多少謝禮?」行者道:「何必說要甚麼謝禮?俗語云:『說金子晃眼,說銀子傻白,說銅錢腥氣。』我等乃積德的和尚,決不要錢。」眾老道:「既如此說,都是受戒的高僧。既不要錢,豈有空勞之理?我等各家俱以魚田為活,若果降了妖孽,淨了地方,我等每家送你兩畝良田,共湊一千畝,坐落一處,你師徒們在上起蓋寺院,打坐參禪,強似方上雲遊。」行者又笑道:「越不停當。但說要了田,就要養馬當差,納糧辦草,黃昏不得睡,五鼓不得眠,好倒弄殺人也。」眾老道:「諸般不要,卻將何謝?」行者道:「我出家人,但只是一茶一飯,便是謝了。」眾老喜道:「這個容易。但不知你怎麼拿他?」行者道:「他但來,我就拿住他。」眾老道:「那妖大著哩:上拄天,下拄地;來時風,去時霧。你卻怎生近得他?」行者笑道:「若論呼風駕霧的妖精,我把他當孫子罷了;若說身體長大,有那手段打他。」

正講處,只聽得呼呼風響。慌得那八九個老者戰戰兢兢道:「這和尚鹽醬口,說妖精,妖精就來了。」那老李開了腰門,把幾個親戚連唐僧,都叫:「進來,進來,妖怪來了。」諕得那八戒也要進去,沙僧也要進去。行者兩隻手扯住兩個道:「你們忒不循理,出家人怎麼不分內外?站住,不要走,跟我去天井裡,看看是個甚麼妖精?」八戒道:「哥啊,他們都是經過帳的,風響便是妖來。他都去躲,我們又不與他有親,又不相識,又不是交契故人,看他做甚?」原來行者力量大,不容說,一把拉在天井裡站下。那陣風越發大了,好風:
\begin{quote}
倒樹摧林狼虎憂,播江攪海鬼神愁。
掀翻華岳三峰石,提起乾坤四部洲。
村舍人家皆閉戶,滿莊兒女盡藏頭。
黑雲漠漠遮星漢,燈火無光遍地幽。
\end{quote}

慌得那八戒戰戰兢兢,伏之於地,把嘴拱開土,埋在地下,卻如釘了釘一般。沙僧蒙著頭臉,眼也難睜。

行者聞風認怪,一霎時,風頭過處,只見那半空中隱隱的兩盞燈來,即低頭叫道:「兄弟們,風過了,起來看。」那獃子扯出嘴來,抖抖灰土,仰著臉,朝天一望,見有兩盞燈光,忽失聲笑道:「好耍子,好耍子,原來是個有行止的妖精,該和他做朋友。」沙僧道:「這般黑夜,又不曾覿面相逢,怎麼就知好歹?」八戒道:「古人云:『夜行以燭,無燭則止。』你看他打一對燈籠引路,必定是個好的。」沙僧道:「你錯看了,那不是一對燈籠,是妖精的兩隻眼亮。」這獃子就諕矮了三寸,道:「爺爺呀!眼有這般大啊,不知口有多少大哩。」行者道:「賢弟莫怕。你兩個護持著師父,待老孫上去討他個口氣,看他是甚妖精。」八戒道:「哥哥,不要供出我們來。」

好行者,縱身打個唿哨,跳到空中,執鐵棒,厲聲高叫道:「慢來,慢來,有吾在此。」那怪見了,挺住身軀,將一根長槍亂舞。行者執了棍勢,問道:「你是那方妖怪?何處精靈?」那怪更不答應,只是舞槍。行者又問,又不答,只是舞槍。行者暗笑道:「好是耳聾口啞。不要走,看棍。」那怪更不怕,亂舞槍遮攔。在那半空中,一來一往,一上一下,鬥到三更時分,未見勝敗。八戒、沙僧在李家天井裡看得明白。原來那怪只是舞槍遮架,更無半分兒攻殺。行者一條棒不離那怪的頭上。八戒笑道:「沙僧,你在這裡護持,讓老豬去幫打幫打,莫教那猴子獨幹這功,領頭一鍾酒。」

好獃子,就跳起雲頭,趕上就築。那怪物又使一條槍抵住。兩條槍就如飛蛇掣電。八戒誇獎道:「這妖精好槍法!不是山後槍,乃是纏絲槍;也不是馬家槍,卻叫做個軟柄槍。」行者道:「獃子莫胡說。那裡有個甚麼軟柄槍?」八戒道:「你看他使出槍尖來架住我們,不見槍柄,不知收在何處。」行者道:「或者是個軟柄槍;但這怪物還不會說話,想是還未歸人道,陰氣還重。只怕天明時陽氣勝,他必要走。但走時,一定趕上,不可放他。」八戒道:「正是,正是。」

又鬥多時,不覺東方發白。那怪不敢戀戰,回頭就走。行者與八戒一齊趕來,忽聞得污穢之氣逼人,乃是七絕山稀柿衕也。八戒道:「是那家淘毛廁哩?哏!臭氣難聞。」行者侮著鼻子,只叫:「快趕妖精,快趕妖精。」那怪物攛過山去,現了本像,乃是一條紅鱗大蟒。你看他:
\begin{quote}
眼射曉星,鼻噴朝霧。密密牙排鋼劍,彎彎爪曲金鉤。頭戴一條肉角,好便似千千塊瑪瑙攢成;身披一派紅鱗,卻就如萬萬片胭脂砌就。盤地只疑為錦被,飛空錯認作虹霓。歇臥處有腥氣沖天,行動時有赤雲罩體。大不大,兩邊人不見東西;長不長,一座山跨占南北。
\end{quote}

八戒道:「原來是這般一個長蛇。若要吃人啊,一頓也得五百個,還不飽足。」行者道:「那軟柄槍乃是兩條信。我們趕他軟了,從後打出去。」這八戒縱身趕上,將鈀便築。那怪物一頭鑽進窟裡,還有七八尺長尾巴丟在外邊。八戒放下鈀,一把撾住道:「著手,著手。」盡力氣往外亂扯,莫想扯得動一毫。行者笑道:「獃子,放他進去,自有處置,不要這等倒扯蛇。」八戒真個撒了手,那怪縮進去了。八戒怨道:「才不放手時,半截子已是我們的了;是這般縮了,卻怎麼得他出來?這不是叫做沒蛇弄了?」行者道:「這廝身體狼犺,窟穴窄小,斷然轉身不得,一定是個照直攛的,定有個後門出頭。你快去後門外攔住,等我在前門外打。」

那獃子真個一溜煙跑過山去,果見有個孔窟,他就扎定腳。還不曾站穩,不期行者在前門外使棍子往裡一搗,那怪物護疼,徑往後門攛出。八戒未曾防備,被他一尾巴打了一跌,莫能掙挫得起,睡在地下忍疼。行者見窟中無物,搴著棒,跑過來叫趕妖怪。那八戒聽得吆喝,自己害羞,忍著疼,爬起來,使鈀亂撲。行者見了,笑道:「妖怪走了,你還撲甚的了?」八戒道:「老豬在此打草驚蛇哩。」行者道:「活獃子,快趕上。」

二人趕過澗去,見那怪盤做一團,豎起頭來,張開巨口,要吞八戒。八戒慌得往後便退。這行者反迎上前,被他一口吞之。八戒搥胸跌腳,大叫道:「哥耶,傾了你也。」行者在妖精肚裡支著鐵棒道:「八戒莫愁,我叫他搭個橋兒你看。」那怪物躬起腰來,就似一道路東虹。八戒道:「雖是像橋,只是沒人敢走。」行者道:「我再叫他變做個船兒你看。」在肚裡將鐵棒撐著肚皮。那怪物肚皮貼地,翹起頭來,就似一隻贛保船,八戒道:「雖是像船,只是沒有桅篷,不好使風。」行者道:「你讓開路,等我叫他使個風你看。」又在裡面盡著力把鐵棒從脊背上搠將出去,約有五七丈長,就似一根桅杆。那廝忍疼掙命,往前一攛,比使風更快,攛回舊路,下了山,有二十餘里,卻才倒在塵埃,動蕩不得,嗚呼喪矣。八戒隨後趕上來,又舉鈀亂築。行者把那物穿了一個大洞,鑽將出來道:「獃子,他死也死了,你還築他怎的?」八戒道:「哥啊,你不知我老豬一生好打死蛇?」遂此收了兵器,抓著尾巴,倒拉將來。

卻說那駝羅莊上李老兒與眾等對唐僧道:「你那兩個徒弟一夜不回,斷然傾了命也。」三藏道:「決不妨事。我們出去看看。」須臾間,只見行者與八戒拖著一條大蟒,吆吆喝喝前來。眾人卻才歡喜。滿莊上老幼男女,都來跪拜道:「爺爺,正是這個妖精在此傷人。今幸老爺施法,斬怪除邪,我輩庶各得安生也。」

眾家都是感激,東請西邀,各各酬謝。師徒們被留住五七日,苦辭無奈,方肯放行。又各家見他不要錢物,都辦些乾糧果品,騎騾壓馬,花紅彩旗,盡來餞行。此處五百人家,到有七八百人相送。

一路上喜喜歡歡,不時到了七絕山稀柿衕口。三藏聞得那般惡穢,又見路道填塞,道:「悟空,似此怎生過得?」行者侮著鼻子道:「這個卻難也。」三藏見行者說難,便就眼中垂淚。李老兒與眾上前道:「老爺勿得心焦。我等送到此處,都已約定意思了:令高徒與我們降了妖精,除了一莊禍害,我們各辦虔心,另開一條好路,送老爺過去。」行者笑道:「你這老兒,俱言之欠當。你初然說這山徑過有八百里,你等又不是大禹的神兵,那裡會開山鑿路?若要我師父過去,還得我們著力,你們都成不得。」三藏下馬,道:「悟空,怎生著力麼?」行者笑道:「眼下就要過山,卻也是難;若說再開條路,卻又難也。須是還從舊衚衕過去,只恐無人管飯。」李老兒道:「長老說那裡話,憑你四位擔擱多少時,我等俱養得起,怎麼說無人管飯。」行者道:「既如此,你們去辦得兩石米的乾飯,再做些蒸餅、饝饝來。等我那長嘴和尚吃飽了,變了大豬,拱開舊路,我師父騎在馬上,我等扶持著,管情過去了。」

八戒聞言道:「哥哥,你們都要圖個乾淨,怎麼獨教老豬出臭?」三藏道:「悟能,你果有本事拱開衚衕,領我過山,註你這場頭功。」八戒笑道:「師父在上,列位施主們都在此,休笑話。我老豬本來有三十六般變化,若說變輕巧華麗飛騰之物,委實不能;若說變山,變樹,變石塊,變土墩,變賴象、科豬、水牛、駱駝,真個全會。只是身體變得大,肚腸越發大。須是吃得飽了,才好幹事。」眾人道:「有東西,有東西,我們都帶得有乾糧、果品、燒餅、饝饝在此,原要開山相送的,且都拿出來,憑你受用。待變化了,行動之時,我們再著人回去做飯送來。」八戒滿心歡喜,脫了皂直裰,丟了九齒鈀,對眾道:「休笑話,看老豬幹這場臭功。」

好獃子,捻著訣,搖身一變,果然變做一個大豬。真個是:
\begin{quote}
嘴長毛短半脂臕,自幼山中食藥苗。
黑面環睛如日月,圓頭大耳似芭蕉。
修成堅骨同天壽,煉就粗皮比鐵牢。
齆齆鼻音呱詀叫,喳喳喉響噴喁哮。
白蹄四隻高千尺,劍鬣長身百丈饒。
從見人間肥豕彘,未觀今日老豬魈。
唐僧等眾齊稱讚,羨美天蓬法力高。
\end{quote}

孫行者見八戒變得如此,即命那些相送人等快將乾糧等物推攢一處,叫八戒受用。那獃子不分生熟,一澇食之,卻上前拱路。行者叫沙僧脫了腳,好生挑擔;請師父穩坐雕鞍。他也脫了䩺鞋,吩咐眾人回去:「若有情,快早送些飯來與我師弟接力。」那些人有七八百相送隨行,多一半有騾馬的,飛星回莊做飯;還有三百人步行的,立於山下遙望他行。原來此莊至山有三十餘里,待回取飯來又三十餘里,往回擔擱約有百里之遙,他師徒們已此去得遠了。眾人不捨,催趲騾馬,進衚衕,連夜趕至,次日方才趕上。叫道:「取經的老爺,慢行,慢行,我等送飯來也。」長老聞言,謝之不盡道:「真是善信之人。」叫八戒住了,再吃些飯食壯神。那獃子拱了兩日,正在饑餓之際,那許多人何止有七八石飯食,他也不論米飯、麵飯,收積來一澇用之,飽餐一頓,卻又上前拱路。三藏與行者、沙僧謝了眾人,分手兩別。正是:
\begin{quote}
駝羅莊客回家去,八戒開山過衕來。
三藏心誠神力擁,悟空法顯怪魔衰。
千年稀柿今朝淨,七絕衚衕此日開。
六慾塵情皆剪絕,平安無阻拜蓮臺。
\end{quote}

這去不知還有多少路程,還遇甚麼妖怪,且聽下回分解。
