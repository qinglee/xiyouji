
\chapter{心猿妒木母 魔主計吞禪}

話說那國王早朝文武多官俱執表章啟奏道:「主公,望赦臣等失儀之罪。」國王道:「眾卿禮貌如常,有何失儀?」眾卿道:「主公啊,不知何故,臣等一夜把頭髮都沒了。」國王執了這沒頭髮之表,下龍床對群臣道:「果然不知何故,朕宮中大小人等,一夜也盡沒了頭髮。」君臣們都各汪汪滴淚道:「從此後,再不敢殺戮和尚也。」王復上龍位,眾官各立本班。王又道:「有事出班來奏,無事捲簾散朝。」只見那武班中閃出巡城總兵官,文班中走出東城兵馬使,當階叩頭道:「臣蒙聖旨巡城,夜來獲得賊臟一櫃、白馬一匹。微臣不敢擅專,請旨定奪。」國王大喜道:「連櫃取來。」

二臣即退至本衙,點起齊整軍士,將櫃擡出。三藏在內,魂不附體道:「徒弟們,這一到國王前,如何理說?」行者笑道:「莫嚷,我已打點停當了,開櫃時,他就拜我們為師哩。只教八戒不要爭競長短。」八戒道:「但只免殺,就是無量之福,還敢爭競哩。」說不了,擡至朝外,入五鳳樓,放在丹墀之下。

二臣請國王開看,國王即命打開。方揭了蓋,豬八戒就忍不住往外一跳,諕得那多官膽戰,口不能言。又見孫行者攙出唐僧,沙和尚搬出行李。八戒見總兵官牽著馬,走上前,咄的一聲道:「馬是我的,拿過來。」嚇得那官兒翻跟頭,跌倒在地。四眾俱立在階中。那國王看見是四個和尚,忙下龍床,宣召三宮妃后,下金鑾寶殿,同群臣拜問道:「長老何來?」三藏道:「是東土大唐駕下差往西方天竺國大雷音寺拜活佛取真經的。」國王道:「老師遠來,為何在這櫃裡安歇?」三藏道:「貧僧知陛下有願心殺和尚,不敢明投上國,扮俗人,夜至寶方飯店裡借宿。因怕人識破原身,故此在櫃中安歇。不幸被賊偷出,被總兵捉獲擡來。今得見陛下龍顏,所謂撥雲見日。望陛下赦放貧僧,海深恩便也。」國王道:「老師是天朝上國高僧,朕失迎迓。朕常年有願殺僧者,曾因僧謗了朕,朕許天願,要殺一萬和尚做圓滿。不期今夜歸依,教朕等為僧。如今君臣后妃,髮都剃落了,望老師勿吝高賢,願為門下。」八戒聽言,呵呵大笑道:「既要拜為門徒,有何贄見之禮?」國王道:「師若肯從,願將國中財寶獻上。」行者道:「莫說財寶,我和尚是有道之僧。你只把關文倒換了,送我們出城,保你皇圖永固,福壽長臻。」那國王聽說,即著光祿寺大排筵宴。君臣合同,拜歸於一。即時倒換關文,請師父改號。行者道:「陛下『法國』之名甚好,但只『滅』字不通。自經我過,可改號『欽法國』,管教你海晏河清千代勝,風調雨順萬方安。」國王謝了恩。擺整朝鑾駕,送唐僧四眾出城西去。君臣們乘善歸真不題。

卻說長老辭別了欽法國王,在馬上欣然道:「悟空,此一法甚善,大有功也。」沙僧道:「哥啊,是那裡尋這許多整容匠,連夜剃這許多頭?」行者把那施變化弄神通的事說了一遍。師徒們都笑不合口。

正歡喜處,忽見一座高山阻路。唐僧勒馬道:「徒弟們,你看這面前山勢崔巍,切須仔細。」行者笑道:「放心,放心,保你無事。」三藏道:「休言無事。我看那山峰挺立,遠遠的有些兇氣,暴雲飛出,漸覺驚惶,滿身麻木,神思不安。」行者笑道:「你把烏巢禪師的《多心經》早已忘了?」三藏道:「我記得。」行者道:「你雖記得,這有四句頌子,你卻忘了哩。」三藏道:「那四句?」行者道:
\begin{quote}
「佛在靈山莫遠求,靈山只在汝心頭。
人人有個靈山塔,好向靈山塔下修。」
\end{quote}

三藏道:「徒弟,我豈不知?若依此四句,千經萬典,也只是修心。」行者道:「不消說了。心淨孤明獨照,心存萬境皆清。差錯些兒成惰懈,千年萬載不成功。但要一片志誠,雷音只在眼下。似你這般恐懼驚惶,神思不安,大道遠矣,雷音亦遠矣。且莫胡疑,隨我去。」那長老聞言,心神頓爽,萬慮皆休。

四眾一同前進。不幾步到於山上。舉目看時:
\begin{quote}
那山真好山,細看色班班。頂上雲飄蕩,崖前樹影寒。飛禽淅瀝,走獸兇頑。林內松千榦,巒頭竹幾竿。吼叫是蒼狼奪食,咆哮是餓虎爭餐。野猿長嘯尋鮮果,麋鹿攀花上翠嵐。風灑灑,水潺潺,時聞幽鳥語間關。幾處藤蘿牽又扯,滿溪瑤草雜香蘭。磷磷怪石,削削峰岩。狐狢成群走,猴猿作隊頑。行客正愁多險峻,奈何古道又灣還。
\end{quote}

師徒們怯怯驚驚,正行之時,只聽得呼呼一陣風起。三藏害怕道:「風起了。」行者道:「春有和風,夏有薰風,秋有金風,冬有朔風。四時皆有風,風起怕怎的?」三藏道:「這風來得甚急,決然不是天風。」行者道:「自古來,風從地起,雲自山出,怎麼得個天風?」

說不了,又見一陣霧起。那霧真個是:
\begin{quote}
漠漠連天暗,濛濛匝地昏。
日色全無影,鳥聲無處聞。
宛然如混沌,彷彿似飛塵。
不見山頭樹,那逢採藥人。
\end{quote}

三藏一發心驚道:「悟空,風還未定,如何又這般霧起?」行者道:「且莫忙,請師父下馬,你兄弟二個在此保守,等我去看看是何吉凶。」

好大聖,把腰一躬,就到半空。用手搭在眉上,圓睜火眼,向下觀之,果見那懸巖邊坐著一個妖精。你看他怎生模樣:
\begin{quote}
炳炳文斑多采豔,昂昂雄勢甚抖擻。
獠牙出口如鋼鑽,利爪藏蹄似玉鉤。
金眼圓睛禽獸怕,銀鬚倒豎鬼神愁。
張狂哮吼施威猛,噯霧噴風運智謀。
\end{quote}

又見那左右手下有三四十個小妖擺列,他在那裡逼法的噴風噯霧。行者暗笑道:「我師父也有些兒先兆,他說不是天風,果然此風是個妖精在這裡弄喧兒哩。若老孫使鐵棒往下就打,這叫做『搗蒜打』,打便打死了,只是壞了老孫的名頭。」那行者一生豪傑,再不曉得暗算計人。他道:「我且回去,照顧豬八戒照顧,教他來先與這妖精見一仗。若是八戒有本事,打倒這妖,算了造化;若無手段,被這妖拿去,等我再去救他,才好出名。他又平日做作,有些躲懶,不肯出頭,卻只是有些口緊,好吃東西。等我哄他一哄,看他怎麼說?」

即時落下雲頭,到三藏前。三藏問道:「悟空,風霧處吉凶何如?」行者道:「這會子明淨了,沒甚風霧。」三藏道:「正是,覺到退下些去了。」行者笑道:「師父,我常時間還看得好,這番卻看錯了。我只說風霧之中恐有妖怪,原來不是。」三藏道:「是甚麼?」行者道:「前面不遠,乃是一莊村。村上人家好善,蒸的白米乾飯、白麵饝饝齋僧哩。這些霧,想是那些人家蒸籠之氣,也是積善之應。」八戒聽說,認了真實,扯過行者,悄悄的道:「哥哥,你先吃了他的齋來的?」行者道:「吃不多兒,因那菜蔬太鹹了些,不喜多吃。」八戒道:「啐!憑他怎麼鹹,我也盡肚吃他一飽。十分作渴,便回來吃水。」行者道:「你要吃麼?」八戒道:「正是,我肚裡有些饑了,先要去吃些兒,不知如何?」行者道:「兄弟莫題。古書云:『父在,子不得自專。』師父又在此,誰敢先去?」八戒笑道:「你若不言語,我就去了。」行者道:「我不言語,看你怎麼得去?」那獃子吃嘴的見識偏有,走上前,唱個大喏道:「師父,適才師兄說,前村裡有人家齋僧。你看這馬,有些要打攪人家,便要草要料,卻不費事?幸如今風霧明淨,你們且略坐坐,等我去尋些嫩草兒,先喂喂馬,然後再往那家子化齋去罷。」唐僧歡喜道:「好啊,你今日卻怎肯這等勤謹?快去快來。」

那獃子暗暗笑著便走。行者趕上扯住道:「兄弟,他那裡齋僧,只齋俊的,不齋醜的。」八戒道:「這等說,又要變化是。」行者道:「正是,你變變兒去。」好獃子,他也有三十六般變化,走到山凹裡,捻著訣,念動咒語,搖身一變,變做個矮瘦和尚。手裡敲個木魚,口裡哼阿哼的,又不會念經,只哼的是「上大人」。

卻說那怪物收風斂霧,號令群妖,在於大路口上,擺開一個圈子陣,專等行客。這獃子晦氣,不多時,撞到當中,被群妖圍住,這個扯住衣服,那個扯著絲絛,推推擁擁,一齊下手。八戒道:「不要扯,等我一家家吃將來。」群妖道:「和尚,你要吃甚的?」八戒道:「你們這裡齋僧,我來吃齋的。」群妖道:「你想這裡齋僧,不知我這裡專要吃僧。我們都是山中得道的妖仙,專要把你們和尚拿到家裡,上蒸籠蒸熟吃哩。你倒還想來吃齋。」八戒聞言,心中害怕,才報怨行者道:「這個弼馬溫,其實憊𪬯。他哄我說是這村裡齋僧,這裡那得村莊人家?那裡齋甚麼僧?卻原來是此妖精。」那獃子被他扯急了,即便現出原身,腰間掣釘鈀,一頓亂築,築退那些小妖。

小妖急跑去報與老妖道:「大王,禍事了。」老怪道:「有甚禍事?」小妖道:「山前來了一個和尚,且是生得乾淨。我說拿家來蒸他吃,若吃不了,留些兒防天陰。不想他會變化。」老妖道:「變化甚的模樣?」小妖道:「那裡成個人相?長嘴大耳朵,背後又有鬃。雙手掄一根釘鈀,沒頭沒臉的亂築,諕得我們跑回來報大王也。」老怪道:「莫怕,等我去看。」掄著一條鐵杵,走近前看時,見那獃子果然醜惡。他生得:
\begin{quote}
碓嘴初長三尺零,獠牙觜出賽銀釘。
一雙圓眼光如電,兩耳搧風唿唿聲。
腦後鬃長排鐵箭,渾身皮糙癩還青。
手中使件蹊蹺物,九齒釘鈀個個驚。
\end{quote}

妖精硬著膽喝道:「你是那裡來的?叫甚名字?快早說來,饒你性命。」八戒笑道:「我的兒,你是也不認得你豬祖宗哩。上前來,說與你聽:
\begin{quote}
巨口獠牙神力大,玉皇陞我天蓬帥。
掌管天河八萬兵,天宮快樂多自在。
只因酒醉戲宮娥,那時就把英雄賣。
一嘴拱倒斗牛宮,吃了王母靈芝菜。
玉皇親打二千鎚,把吾貶下三天界。
教吾立志養元神,下方卻又為妖怪。
正在高莊喜結親,命低撞著孫兄在。
金箍棒下受他降,低頭才把沙門拜。
背馬挑包做夯工,前生少了唐僧債。
鐵腳天蓬本姓豬,法名喚作豬八戒。」
\end{quote}

那妖精聞言,喝道:「你原來是唐僧的徒弟。我一向聞得唐僧的肉好吃,正要拿你哩,你卻撞得來,我肯饒你?不要走,看杵。」八戒道:「孽畜,你原來是個染博士出身。」妖精道:「我怎麼是染博士?」八戒道:「不是染博士,怎麼會使棒槌?」那怪那容分說,近前亂打。他兩個在山凹裡,這一場好殺:
\begin{quote}
九齒釘鈀,一條鐵杵。鈀丟解數滾狂風,杵運機謀飛驟雨。一個是無名惡怪阻山程,一個是有罪天蓬扶性主。性正何愁怪與魔,山高不得金生土。那個杵架猶如蟒出潭,這個鈀來卻似龍離浦。喊聲咤吒振山川,吆喝雄威驚地府。兩個英雄各逞能,捨身卻把神通賭。
\end{quote}

八戒長起威風,與妖精廝鬥,那怪喝令小妖把八戒一齊圍住不題。

卻說行者在唐僧背後,忽失聲冷笑。沙僧道:「哥哥冷笑,何也?」行者道:「豬八戒真個獃呀,聽見說齋僧,就被我哄去了。這早晚還不見回來:若是一頓鈀打退妖精,你看他得勝而回,爭嚷功果;若戰他不過,被他拿去,卻是我的晦氣,背前面後,不知罵了多少弼馬溫哩。悟淨,你休言語,等我去看看。」

好大聖,他也不使長老知道,悄悄的腦後拔了一根毫毛,吹口仙氣,叫:「變!」即變做本身模樣,陪著沙僧,隨著長老。他的真身出個神,跳在空中觀看,但見那獃子被怪圍繞,釘鈀勢亂,漸漸的難敵。行者忍不住,按落雲頭,厲聲高叫道:「八戒不要忙,老孫來了。」那獃子聽得是行者聲音,仗著勢,愈長威風,一頓鈀,向前亂築。那妖精抵敵不住,道:「這和尚先前不濟,這會子怎麼又發起狠來?」八戒道:「我的兒,不可欺負我,我家裡人來也。」一發向前,沒頭沒臉築去。那妖精委架不住,領群妖敗陣去了。行者見妖精敗去,他就不曾近前,撥轉雲頭,徑回本處,把毫毛一抖,收上身來。長老的肉眼凡胎,那裡認得。

不一時,獃子得勝,也自轉來,累得那粘涎鼻涕,白沫生生。氣呼呼的走將來,叫聲:「師父。」長老見了,驚訝道:「八戒,你去打馬草的,怎麼這般狼狽回來?想是山上人家有人看護,不容你打草麼?」獃子放下鈀,搥胸跌腳道:「師父,莫要問,說起來就活活羞殺人。」長老道:「為甚麼羞來?」八戒道:「師兄捉弄我。他先頭說風霧裡不是妖精,沒甚兇兆,是一莊村人家好善,蒸白米乾飯、白麵饝饝齋僧的。我就當真,想著肚內饑了,先去乞些兒,假倚打草為名;豈知若干妖怪,把我圍了,苦戰了這一會,若不是師兄的哭喪棒相助,我也莫想得脫羅網回來也。」行者在傍笑道:「這獃子胡說。你若做了賊,就攀上一牢人。是我在這裡看著師父,何曾側離?」長老道:「是啊,悟空不曾離我。」那獃子跳著嚷道:「師父,你不曉得,他有替身。」長老道:「悟空,端的可有怪麼?」

行者瞞不過,躬身笑道:「是有個把小妖兒,他不敢惹我們。——八戒,你過來,一發照顧你照顧。我們既保師父,走過險峻山路,就似行軍的一般。」八戒道:「行軍便怎的?」行者道:「你做個開路將軍,在前剖路。那妖精不來便罷,若來時,你與他賭鬥,打倒妖精,算你的功果。」八戒量著那妖精手段與他差不多,卻說:「我就死在他手內也罷,等我先走。」行者笑道:「這獃子先說晦氣話,怎麼得長進?」八戒道:「哥哥,你知道:公子登筵,不醉即飽;壯士臨陣,不死帶傷。先說句錯話兒,後便有威風。」行者歡喜,即忙背了馬,請師父騎上,沙僧挑著行李,相隨八戒,一路入山不題。

卻說那妖精帥幾個敗殘的小妖徑回本洞,高坐在那石崖上,默默無言。洞中還有許多看家的小妖,都上前問道:「大王常時出去,喜喜歡歡回來,今日如何煩惱?」老妖道:「小的們,我往常出洞巡山,不管那裡的人與獸,定撈幾個來家,養贍汝等。今日造化低,撞見一個對頭。」小妖問:「是那個對頭?」老妖道:「是一個和尚,乃東土唐僧取經的徒弟,名喚豬八戒。我被他一頓釘鈀,把我築得敗下陣來。好惱啊!我這一向,常聞得人說,唐僧乃十世修行的羅漢,有人吃他一塊肉,可以延壽長生。不期他今日到我山裡,正好拿住他蒸吃,不知他手下有這等徒弟。」

說不了,班部叢中閃上一個小妖,對老妖哽哽咽咽哭了三聲,又嘻嘻哈哈的笑了三聲。老妖喝道:「你又哭又笑,何也?」小妖跪下道:「大王才說要吃唐僧,唐僧的肉不中吃。」老妖道:「人都說吃他一塊肉可以長生不老,與天同壽,怎麼說他不中吃?」小妖道:「若是中吃,也到不得這裡,別處妖精也都吃了。他手下有三個徒弟哩。」老妖道:「你知那三個?」小妖道:「他大徒弟是孫行者,三徒弟是沙和尚,這個是他二徒弟豬八戒。」老妖道:「沙和尚比豬八戒如何?」小妖道:「也差不多兒。」「那個孫行者比他如何?」小妖吐舌道:「不敢說。那孫行者神通廣大,變化多端。他五百年前曾大鬧天宮,上方二十八宿、九曜星官、十二元辰、五卿四相、東西星斗、南北二神、五嶽四瀆、普天神將,也不曾惹得他過,你怎敢要吃唐僧?」老妖道:「你怎麼知得他這等詳細?」小妖道:「我當初在獅駝嶺獅駝洞與那大王居住,那大王不知好歹,要吃唐僧,被孫行者使一條金箍棒,打進門來,可憐就打得犯了骨牌名,都『斷么絕六』。還虧我有些見識,從後門走了,來到此處,蒙大王收留。故此知他手段。」老妖聽言,大驚失色。這正是「大將軍怕讖語」。他聞得自家人這等說,安得不驚。

正都在悚懼之際,又一個小妖上前道:「大王莫惱,莫怕。常言道:『事從緩來。』若是要吃唐僧,等我定個計策拿他。」老妖道:「你有何計?」小妖道:「我有個分瓣梅花計。」老妖道:「怎麼叫做『分瓣梅花計』?」小妖道:「如今把洞口大小群妖點將起來,千中選百,百中選十,十中只選三個。須是有能幹,會變化的,都變做大王的模樣,頂大王之盔,貫大王之甲,執大王之杵,三處埋伏。先著一個戰豬八戒,再著一個戰孫行者,再著一個戰沙和尚:捨著三個小妖,調開他弟兄三個。大王卻在半空伸下拿雲手,去捉這唐僧,就如探囊取物,就如魚水盆內捻蒼蠅,有何難哉?」老妖聞此言,滿心歡喜道:「此計絕妙,絕妙!這一去,拿不得唐僧便罷;若是拿了唐僧,決不輕你,就封你做個前部先鋒。」小妖叩頭謝恩,叫點妖怪。即將洞中大小妖精點起,果然選出三個有能的小妖,俱變做老妖,各執鐵杵,埋伏等待唐僧不題。

卻說這唐長老無慮無憂,相隨八戒上大路。行夠多時,只見那路傍邊撲落的一聲響喨,跳出一個小妖,奔向前邊,要捉長老。孫行者叫道:「八戒,妖精來了,何不動手?」那獃子不認真假,掣釘鈀趕上亂築。那妖精使鐵杵就架相迎。他兩個一往一來的在山坡下正然賭鬥,又見那草科裡響一聲,又跳出個怪來,就奔唐僧。行者道:「師父,不好了,八戒的眼拙,放那妖精來拿你,且等老孫打他去。」急掣棒迎上前喝道:「那裡去?看棒。」那妖精更不打話,舉杵來迎他。兩個在草坡下一撞一衝,正相持處,又聽得山背後呼的風響,又跳出個妖精來,徑奔唐僧。沙僧見了,大驚道:「師父,大哥與二哥的眼都花了,把妖精放將來拿你了。你坐在馬上,等老沙拿他去。」這和尚也不分好歹,即掣杖,對面擋住那妖精鐵杵,恨苦相持,吆吆喝喝,亂嚷亂鬥,漸漸的調遠。那老怪在半空中見唐僧獨坐馬上,伸下五爪鋼鉤,把唐僧一把撾住。那師父丟了馬,脫了鐙,被妖精一陣風徑攝去了。可憐!這正是:
\begin{quote}
禪性遭魔難正果,江流又遇苦災星。
\end{quote}

老妖按下風頭,把唐僧拿到洞裡,叫:「先鋒。」那定計的小妖上前跪倒,口中道:「不敢,不敢。」老妖道:「何出此言?大將軍一言既出,如白染皂。當時說拿不得唐僧便罷,拿了唐僧,封你為前部先鋒。今日你果妙計成功,豈可失信於你?你可把唐僧拿來,著小的們挑水刷鍋,搬柴燒火,把他蒸一蒸。我和你都吃他一塊肉,以圖延壽長生也。」先鋒道:「大王,且不可吃。」老怪道:「既拿來,怎麼不可吃?」先鋒道:「大王吃了他不打緊,豬八戒也做得人情,沙和尚也做得人情,但恐孫行者那主子刮毒。他若曉得是我們吃了,他也不來和我們廝打,他只把那金箍棒往山腰裡一搠,搠個窟窿,連山都掬倒了,我們安身之處也無之矣。」老怪道:「先鋒,憑你有何高見?」先鋒道:「依著我,把唐僧送在後園,綁在樹上,兩三日不要與他飯吃:一則圖他裡面乾淨;二則等他三人不來門前尋找,打聽得他們回去了,我們卻把他拿出來,自自在在的受用,卻不是好?」老怪笑道:「正是,正是,先鋒說得有理。」一聲號令,把唐僧拿入後園,一條繩綁在樹上,眾小妖都去前面聽候。

你看那長老苦捱著繩纏索綁,緊縛牢拴,止不住腮邊流淚,叫道:「徒弟呀,你們在那山中擒怪,甚路裡趕妖?我被潑魔捉來,此處受災,何日相會?痛殺我也!」正自兩淚交流,只見對面樹上有人叫道:「長老,你也進來了?」長老正了性道:「你是何人?」那人道:「我是本山中的樵子,被那山主前日拿來,綁在此間,今已三日,算計要吃我哩。」長老滴淚道:「樵夫啊,你死只是一身,無甚掛礙,我卻死得不甚乾淨。」樵子道:「長老,你是個出家人,上無父母,下無妻子,死便死了,有甚麼不乾淨?」長老道:「我本是東土往西天取經去的,奉唐朝太宗皇帝御旨拜活佛,取真經,要超度那幽冥無主的孤魂。今若喪了性命,可不盼殺那君王,孤負那臣子?那枉死城中,無限的冤魂,卻不大失所望,永世不得超生,一場功果,盡化作風塵,這卻怎麼得乾淨也?」樵子聞言,眼中墮淚道:「長老,你死也只如此,我死又更傷情。我自幼失父,與母鰥居,更無家業,止靠著打柴為生。老母今年八十三歲,只我一人奉養。倘若身喪,誰與他埋屍送老?苦哉,苦哉!痛殺我也。」長老聞言,放聲大哭道:「可憐,可憐!山人尚有思親意,空教貧僧會念經。事君事親,皆同一理;你為親恩,我為君恩。」正是那:
\begin{quote}
流淚眼觀流淚眼,斷腸人送斷腸人。
\end{quote}

且不言三藏身遭困苦。卻說孫行者在草坡下戰退小妖,急回來路傍邊,不見了師父,止存白馬、行囊。慌得他牽馬挑擔,向山頭找尋。咦!正是那:
\begin{quote}
有難的江流專遇難,降魔的大聖亦遭魔。
\end{quote}

畢竟不知尋找師父下落如何,且聽下回分解。
