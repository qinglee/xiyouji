
\chapter{群魔欺本性 一體拜真如}

且不言唐長老困苦。卻說那三個魔頭齊心竭力,與大聖兄弟三人,在城東半山內努力爭持。這一場,正是那鐵刷帚刷銅鍋——家家挺硬。好殺:
\begin{quote}
六般體相六般兵,六樣形骸六樣情。六惡六根緣六慾,六門六道賭輸贏。三十六宮春自在,六六形色恨有名。這一個金箍棒,千般解數;那一個方天戟,百樣崢嶸。八戒釘鈀兇更猛,二怪長槍俊又能。小沙僧寶杖非凡,有心打死;老魔頭鋼刀快利,舉手無情。這三個是護衛真僧無敵將,那三個是亂法欺君潑野精。起初猶可,向後彌兇。六枚都使昇空法,雲端裡面各翻騰。一時間吐霧噴雲天地暗,哮哮吼吼只聞聲。
\end{quote}

他六個鬥夠多時,漸漸天晚。卻又是風霧漫漫,霎時間就黑暗了。原來八戒耳大,蓋著眼皮,越發昏濛;手腳慢,又遮架不住:拖著鈀,敗陣就走。被老魔舉刀砍去,幾乎傷命。幸躲過頭腦,被口刀削斷幾根毛,趕上張開口咬著領頭,拿入城中,丟與小怪,綑在金鑾殿。老妖又駕雲,起在半空助力。沙和尚見事不諧,虛幌著寶杖,顧本身回頭便走。被二怪捽開鼻子,響一聲,連手捲住,拿到城裡,也叫小妖綑在殿下。卻又騰空去叫拿行者。

行者見兩個兄弟遭擒,他自家獨力難撐,正是:好手不敵雙拳,雙拳難敵四手。他喊一聲,把棍子隔開三個妖魔的兵器,縱觔斗駕雲走了。三怪見行者駕觔斗時,即抖抖身,現了本像,搧開兩翅,趕上大聖。你道他怎能趕上?當時如行者鬧天宮,十萬天兵也拿他不住者,以他會駕觔斗雲,一去有十萬八千里路,所以諸神不能趕上。這妖精搧一翅就有九萬里,兩搧就趕過了。所以被他一把撾住,拿在手中,左右掙挫不得。欲思要走,莫能逃脫。即使變化法遁法,又往來難行:變大些兒,他就放鬆了撾住;變小些兒,他又揝緊了撾住。復拿了徑回城內,放了手,捽下塵埃,吩咐群妖,也照八戒、沙僧綑在一處。那老魔、二魔俱下來迎接,三個魔頭同上寶殿。噫!這一番倒不是綑住行者,分明是與他送行。

此時有二更時候,眾怪一齊相見畢,把唐僧推下殿來。那長老於燈光前,忽見三個徒弟都綑在地下,老師父伏於行者身邊,哭道:「徒弟啊,常時逢難,你卻在外運用神通,到那裡取救降魔;今番你亦遭擒,我貧僧怎麼得命?」八戒、沙僧聽見師父這般苦楚,便也一齊放聲痛哭。行者微微笑道:「師父放心,兄弟莫哭,憑他怎的,決然無傷。等那老魔安靜了,我們走路。」八戒道:「哥啊,又來搗鬼了。麻繩綑住,鬆些兒還著水噴,想你這瘦人兒不覺,我這胖的遭瘟哩。不信,你看兩膊上,入肉已有二寸,如何脫身?」行者笑道:「莫說是麻繩綑的,就是碗粗的棕纜,只也當秋風過耳,何足罕哉?」

師徒們正說處,只聞得那老魔道:「三賢弟有力量,有智謀,果成妙計,拿將唐僧來了。」叫:「小的們,著五個打水,七個刷鍋,十個燒火,二十個擡出鐵籠來,把那四個和尚蒸熟,我兄弟們受用;各散一塊兒與小的們吃,也教他個個長生。」八戒聽見,戰兢兢的道:「哥哥,你聽,那妖精計較要蒸我們吃哩。」行者道:「不要怕,等我看他是雛兒妖精,是把勢妖精。」沙和尚哭道:「哥呀,且不要說寬話。如今已與閻王隔壁哩,且講甚麼『雛兒』、『把勢』?」說不了,又聽得二怪說:「豬八戒不好蒸。」八戒歡喜道:「阿彌陀佛!是那個積陰騭的說我不好蒸?」三怪道:「不好蒸,剝了皮蒸。」八戒慌了,厲聲喊道:「不要剝皮,粗自粗,湯響就爛了。」老怪道:「不好蒸的,安在底下一格。」行者笑道:「八戒莫怕,是雛兒,不是把勢。」沙僧道:「怎麼認得?」行者道:「大凡蒸東西,都從上邊起。不好蒸的,安在上頭一格,多燒把火,圓了氣,就好了;若安在底下,一住了氣,就燒半年也是不得氣上的。他說八戒不好蒸,安在底下,不是雛兒是甚的?」八戒道:「哥啊,依你說,就活活的弄殺人了。他打緊見不上氣,擡開了,把我翻轉過來,再燒起火,弄得我兩邊俱熟,中間不夾生了?」

正講時,又見小妖來報:「湯滾了。」老怪傳令叫擡。眾妖一齊上手,將八戒擡在底下一格,沙僧擡在二格。行者估著來擡他,他就脫身道:「此燈光前好做手腳。」拔下一根毫毛,吹口仙氣,叫:「變!」即變做一個行者,綑了麻繩。將真身出神,跳在半空裡,低頭看著。那群妖那知真假,見人就擡,把個假行者擡在上三格;才將唐僧揪翻倒綑住,擡上第四格。乾柴架起,烈火氣焰騰騰。大聖在雲端裡嗟嘆道:「我那八戒、沙僧,還捱得兩滾;我那師父,只消一滾就爛。若不用法救他,頃刻喪矣。」

好行者,在空中捻著訣,念一聲「唵藍淨法界,乾元亨利貞」的咒語,拘喚得北海龍王早至。只見那雲端裡一朵烏雲,應聲高叫道:「北海小龍敖順叩頭。」行者道:「請起,請起。無事不敢相煩。今與唐師父到此,被毒魔拿住,上鐵籠蒸哩。你去與我護持護持,莫教蒸壞了。」龍王隨即將身變作一陣冷風,吹入鍋下,盤旋圍護,更沒火氣燒鍋,他三人方不損命。

將有三更盡時,只聞得老魔發放道:「手下的,我等用計勞形,拿了唐僧四眾;又因相送辛苦,四晝夜未曾得睡。今已綑在籠裡,料應難脫,汝等用心看守,著十個小妖輪流燒火,讓我們退宮,略略安寢。到五更天色將明,必然爛了,可安排下蒜泥鹽醋,請我們起來,空心受用。」眾妖各各遵命。三個魔頭,卻各轉寢宮而去。

行者在雲端裡明明聽著這等吩咐,卻低下雲頭,不聽見籠裡人聲。他想著:「火氣上騰,必然也熱,他們怎麼不怕,又無言語哼嗔?莫敢是蒸死了?等我近前再聽。」好大聖,踏著雲,搖身一變,變作一個黑蒼蠅兒,釘在鐵籠格外聽時,只聞得八戒在裡面道:「晦氣,晦氣,不知是悶氣蒸,又不知是出氣蒸哩。」沙僧道:「二哥,怎麼叫做『悶氣』、『出氣』?」八戒道:「悶氣蒸是蓋了籠頭,出氣蒸不蓋。」三藏在浮上一層應聲道:「徒弟,不曾蓋。」八戒道:「造化,今夜還不得死,這是出氣蒸了。」行者聽得他三人都說話,未曾傷命,便就飛了去,把個鐵籠蓋輕輕兒蓋上。三藏慌了道:「徒弟,蓋上了。」八戒道:「罷了,這個是悶氣蒸,今夜必是死了。」沙僧與長老嚶嚶的啼哭。八戒道:「且不要哭,這一會燒火的換了班了。」沙僧道:「你怎麼知道?」八戒道:「早先擡上來時,正合我意:我有些兒寒濕氣的病,要他騰騰。這會子反冷氣上來了。咦!燒火的長官,添上些柴便怎的?要了你的哩?」

行者聽見,忍不住暗笑道:「這個夯貨,冷還好捱,若熱就要傷命。再說兩遭,一定走了風了,快早救他。且住,要救他須是要現本相。假如現了,這十個燒火的看見,一齊亂喊,驚動老怪,卻不又費事?等我先送他個法兒。」忽想起:「我當初做大聖時,曾在北天門與護國天王猜枚耍子,贏得他瞌睡蟲兒,還有幾個,送了他罷。」即往腰間順帶裡摸摸,還有十二個。「送他十個,還留兩個做種。」即將蟲兒拋了去,散在十個小妖臉上,鑽入鼻孔,漸漸打盹,都睡倒了。只有一個拿火叉的睡不穩,揉頭搓臉,把鼻子左捏右捏,不住的打噴嚏。行者道:「這廝曉得勾當了,我再與他個雙燈。」又將一個蟲兒拋在他臉上。「兩個蟲兒,左進右出,右出左進,諒有一個安住。」那小妖兩三個大啊欠,把腰伸一伸,丟了火叉,也撲的睡倒,再不翻身。

行者道:「這法兒真是妙而且靈。」即現原身,走近前,叫聲:「師父。」唐僧聽見道:「悟空,救我啊。」沙僧道:「哥哥,你在外面叫哩?」行者道:「我不在外面,好和你們在裡邊受罪?」八戒道:「哥啊,溜撒的溜了,我們都是頂缸的,在此受悶氣哩。」行者笑道:「獃子莫嚷,我來救你。」八戒道:「哥啊,救便要脫根救,莫又要復蒸籠。」行者卻揭開籠頭,解了師父,將假變的毫毛抖了一抖,收上身來;又一層層放了沙僧,放了八戒。那獃子才解了,巴不得就要跑。行者道:「莫忙,莫忙。」卻又念聲咒語,發放了龍神,才對八戒道:「我們這去到西天,還有高山峻嶺。師父沒腳力難行,等我還將馬來。

你看他輕手輕腳,走到金鑾殿下,見那些大小群妖俱睡著了。卻解了韁繩,更不驚動。那馬原是龍馬,若是生人,飛踢兩腳,便嘶幾聲。行者曾養過馬,授弼馬溫之官,又是自家一夥,所以不跳不叫。悄悄的牽來,束緊了肚帶,扣備停當,請師父上馬。長老戰兢兢的騎上,也就要走。行者道:「也且莫忙。我們西去還有國王,須要關文,方才去得;不然,將甚執照?等我還去尋行李來。」唐僧道:「我記得進門時,眾怪將行李放在金殿左手下,擔兒也在那一邊。」行者道:「我曉得了。」即抽身跳在寶殿尋時,忽見光彩飄颻。行者知是行李。怎麼就知?以唐僧的錦襴袈裟上有夜明珠,故此放光。急到前,見擔兒原封未動,連忙拿了去,付與沙僧挑著。

八戒牽著馬,他引了路,徑奔正陽門。只聽得梆鈴亂響,門上有鎖,鎖上貼了封皮。行者道:「這等防守,如何去得?」八戒道:「後門裡去罷。」行者引路,徑奔後門,後宰門外也有梆鈴之聲,門上也有封鎖。「卻怎生是好?我這一番,若不為唐僧是個凡體,我三人不管怎的,也駕雲弄風走了。只為唐僧未超三界外,見在五行中,一身都是父母濁骨,所以不得昇駕,難逃。」八戒道:「哥哥,不消商量,我們到那沒梆鈴、不防衛處,撮著師父爬過牆去罷。」行者笑道:「這個不好。此時無奈,撮他過去;到取經回來,你這獃子口敞,延地裡就對人說,我們是爬牆頭的和尚了。」八戒道:「此時也顧不得行檢,且逃命去罷。」行者也沒奈何,只得依他,到那淨牆邊,算計爬出。

噫!有這般事,也是三藏災星未脫。那三個魔頭在宮中正睡,忽然驚覺,說走了唐僧,一個個披衣忙起,急登寶殿。問曰:「唐僧蒸了幾滾了?」那些燒火的小妖已是有睡魔蟲,都睡著了,就是打也莫想打得一個醒來。其餘沒執事的,驚醒幾個,冒冒失失的答應道:「七、七、七、七滾了。」急跑近鍋邊,只見籠格子亂丟在地下,燒火的還都睡著。慌得又來報道:「大王,走、走、走、走了。」三個魔頭都下殿,近鍋前仔細看時,果見那籠格子亂丟在地下,湯鍋盡冷,火腳俱無。那燒火的俱呼呼鼾睡如泥。慌得眾怪一齊吶喊,都叫:「快拿唐僧!快拿唐僧!」這一片喊聲振起,把些前前後後、大大小小妖精,都驚起來,刀槍簇擁,至正陽門下。見那封鎖不動,梆鈴不絕。問外邊巡夜的道:「唐僧從那裡走了?」俱道:「不曾走出人來。」急趕至後宰門,封鎖、梆鈴,一如前門。復亂搶搶的燈籠火把,熯天通紅,就如白日,卻明明的照見他四眾爬牆哩。老魔趕近,喝聲:「那裡走?」那長老諕得腳軟觔麻,跌下牆來,被老魔拿住。二魔捉了沙僧,三魔擒倒八戒,眾妖搶了行李、白馬,只是走了行者。那八戒口裡嘓嘓噥噥的報怨行者道:「天殺的,我說要救便脫根救,如今卻又復籠蒸了。」

眾魔把唐僧擒至殿上,卻不蒸了。二怪吩咐把八戒綁在殿前簷柱上,三怪吩咐把沙僧綁在殿後簷柱上;惟老魔把唐僧抱住不放。三怪道:「大哥,你抱住他怎的?終不然就活吃?卻也沒些趣味。此物比不得那愚夫俗子,拿了可以當飯;此是上邦稀奇之物,必須待天陰閑暇之時,拿他出來,整製精潔,猜枚行令,細吹細打的吃方可。」老魔笑道:「賢弟之言雖當,但孫行者又要來偷哩。」三魔道:「我這皇宮裡面有一座錦香亭子,亭子內有一個鐵櫃。依著我,把唐僧藏在櫃裡,關了亭子;卻傳出謠言,說唐僧已被我們夾生吃了,令小妖滿城講說。那行者必然來探聽消息,若聽見這話,他必死心塌地而去。待三五日不來攪擾,卻拿出來,慢慢受用。如何?」老怪、二怪俱大喜道:「是是是,兄弟說得有理。」可憐把個唐僧連夜拿將進去,藏在櫃中,閉了亭子;傳出謠言,滿城裡都亂講不題。

卻說行者自夜半顧不得唐僧,駕雲走脫。徑至獅駝洞裡,一路棍,把那萬數小妖盡情勦絕。急回來,東方日出。到城邊,不敢叫戰。正是:單絲不線,孤掌難鳴。他落下雲頭,搖身一變,變作個小妖兒,演入門裡,大街小巷,緝訪消息。滿城裡俱道:唐僧被大王夾生兒連夜吃了。前前後後,都是這等說。行者著實心焦。行至金鑾殿前觀看,那裡邊有許多精靈,都戴著皮金帽子,穿著黃布直身,手拿著紅漆棍,腰掛著象牙牌,一往一來,不住的亂走。行者暗想道:「此必是穿宮的妖怪,就變做這個模樣,進去打聽打聽。」

好大聖,果然變得一般無二,混入金門。正走處,只見八戒綁在殿前柱上哼哩。行者近前,叫聲:「悟能。」那獃子認得聲音,道:「師兄,你來了?救我一救。」行者道:「我救你。你可知師父在那裡?」八戒道:「師父沒了,昨夜被妖精夾生兒吃了。」行者聞言,忽失聲淚似泉湧。八戒道:「哥哥莫哭,我也是聽得小妖亂講,未曾眼見。你休誤了,再去尋問尋問。」這行者卻才收淚,又往裡面找尋。忽見沙僧綁在後簷柱上,即近前摸著他胸脯子叫道:「悟淨。」沙僧也識得聲音,道:「師兄,你變化進來了?救我,救我。」行者道:「救你容易,你可知師父在那裡?」沙僧滴淚道:「哥啊,師父被妖精等不得蒸,就夾生兒吃了。」

大聖聽得兩個言語相同,心如刀攪,淚似水流。急縱身望空跳起,且不救八戒、沙僧,回至城東山上,按落雲頭,放聲大哭,叫道:「師父啊:
\begin{quote}
恨我欺天困網羅,師來救我脫沉痾。
潛心篤志同參佛,努力修身共煉魔。
豈料今朝遭蜇害,不能保你上婆娑。
西方勝境無緣到,氣散魂消怎奈何?」
\end{quote}

行者悽悽慘慘的自思自忖,以心問心道:「這都是我佛如來坐在那極樂之境,沒得事幹,弄了那三藏之經。若果有心勸善,理當送上東土,卻不是個萬古流傳?只是捨不得送去,卻教我等來取。怎知道苦歷千山,今朝到此喪命?罷罷罷,老孫且駕個觔斗雲,去見如來,備言前事。若肯把經與我送上東土,一則傳揚善果,二則了我等心願;若不肯與我,教他把鬆箍兒咒念念,退下這個箍子,交還與他,老孫還歸本洞,稱王道寡,耍子兒去罷。」

好大聖,急翻身,駕起觔斗雲,徑投天竺,那裡消一個時辰,早望見靈山不遠。須臾間,按落雲頭,直至鷲峰之下。忽擡頭,見四大金剛擋住道:「那裡走?」行者施禮道:「有事要見如來。」當頭又有崑崙山金霞嶺不壞尊王永住金剛喝道:「這猢猻甚是粗狂。前者大困牛魔,我等為汝努力,今日面見,全不為禮。有事且待先奏,奉召方行。這裡比南天門不同,教你進去出來,兩邊亂走?咄!還不靠開。」那大聖正是煩惱處,又遭此搶白,氣得哮吼如雷,忍不住大呼小叫,早驚動如來。

如來佛祖正端坐在九品寶蓮臺上,與十八尊輪世的阿羅漢講經,即開口道:「孫悟空來了,汝等出去接待接待。」大眾阿羅遵佛旨,兩路幢幡寶蓋,即出山門應聲道:「孫大聖,如來有旨相喚哩。」那山門口四大金剛卻才閃開路,讓行者前進。眾阿羅引至寶蓮臺下,見如來倒身下拜,兩淚悲啼。如來道:「悟空,有何事這等悲啼?」行者道:「弟子屢蒙教訓之恩,託庇在佛爺爺之門下。自歸正果,保護唐僧,拜為師範,一路上苦不可言。今至獅駝山獅駝洞、獅駝城,有三個毒魔,乃獅王、王、大鵬,把我師父捉將去,連弟子一概遭迍,都綑在蒸籠裡,受湯火之災。幸弟子脫逃,喚龍王救免。是夜偷出師等,不料災星難脫,復又擒回。及至天明,入城打聽,叵耐那魔十分狠毒,萬樣驍勇,把師父連夜夾生吃了,如今骨肉無存。又況師弟悟能、悟淨,見綁在那廂,不久性命亦皆傾矣。弟子沒及奈何,特地到此參拜如來。望大慈悲,將鬆箍咒兒念念,退下我這頭上箍兒,交還如來,放我弟子回花果山寬閑耍子去罷。」說未了,淚如泉湧,悲聲不絕。如來笑道:「悟空少得煩惱。那妖精神通廣大,你勝不得他,所以這等心痛。」行者跪在下面,搥著胸膛道:「不瞞如來說,弟子當年鬧天宮,稱大聖,自為人以來,不曾吃虧,今番卻遭這毒魔之手。」

如來聞言道:「你且休恨,那妖精我認得他。」行者猛然失聲道:「如來,我聽見人講說,那妖精與你有親哩。」如來道:「這個刁猢猻,怎麼個妖精與我有親?」行者笑道:「不與你有親,如何認得?」如來道:「我慧眼觀之,故此認得。那老怪與二怪有主。」叫:「阿儺、迦葉,來,你兩個分頭駕雲去五臺山、峨眉山,宣文殊、普賢來見。」二尊者即奉旨而去。如來道:「這是老魔、二怪之主。但那三怪,說將起來,也是與我有些親處。」行者道:「親是父黨?母黨?」如來道:「自那混沌分時,天開於子,地闢於丑,人生於寅。天地再交合,萬物盡皆生。萬物有走獸飛禽,走獸以麒麟為之長,飛禽以鳳凰為之長。那鳳凰又得交合之氣,育生孔雀、大鵬。孔雀出世之時最惡,能吃人,四十五里路,把人一口吸之。我在雪山頂上,修成丈六金身,早被他也把我吸下肚去。我欲從他便門而出,恐污真身。是我剖開他脊背,跨上靈山。欲傷他命,當被諸佛勸解:傷孔雀如傷我母。故此留他在靈山會上,封他做佛母孔雀大明王菩薩。大鵬是與他一母所生,故此有些親處。」行者聞言笑道:「如來,若這般比論,你還是妖精的外甥哩。」如來道:「那怪須是我去,方可收得。」行者叩頭,啟上如來:「千萬望挪玉一降。」

如來即下蓮臺,同諸佛眾,徑出山門。又見阿儺、迦葉引文殊、普賢來見,二菩薩對佛禮拜。如來道:「菩薩之獸,下山多少時了?」文殊道:「七日了。」如來道:「山中方七日,世上幾千年。不知在那廂傷了多少生靈,快隨我收他去。」二菩薩相隨左右,同眾飛空。只見那:
\begin{quote}
滿天縹緲瑞雲分,我佛慈悲降法門。
明示開天生物理,細言闢地化身文。
面前五百阿羅漢,腦後三千揭諦神。
迦葉阿儺隨左右,普文菩薩殄妖氛。
\end{quote}

大聖有此人情,請得佛祖與眾前來,不多時,早望見城池。行者報道:「如來,那放黑氣的乃是獅駝國也。」如來道:「你先下去,到那城中,與妖精交戰,許敗不許勝。敗上來,我自收他。」

大聖即按雲頭,徑至城上,腳踏著垛兒罵道:「潑孽畜!快出來與老孫交戰。」慌得那城樓上小妖急跳下城中報道:「大王,孫行者在城上叫戰哩。」老妖道:「這猴兒兩三日不來,今朝卻又叫戰,莫不是請了些救兵來耶?」三怪道:「怕他怎的?我們都去看來。」三個魔頭,各持兵器,趕上城來,見了行者,更不打話,舉兵器一齊亂刺;行者掄鐵棒掣手相迎。鬥經七八回合,行者佯輸而走。那妖王喊聲大振,叫道:「那裡走?」大聖觔斗一縱,跳上半空;三個精即駕雲來趕。行者將身一閃,藏在佛爺爺金光影裡,全然不見。只見那過去、未來、見在的三尊佛像與五百阿羅漢、三千揭諦神,佈散左右,把那三個妖王圍住,水泄不通。老魔慌了手腳,叫道:「兄弟,不好了,那猴子真是個地裡鬼,那裡請得個主人公來也。」三魔道:「大哥休得悚懼,我們一齊上前,使槍刀搠倒如來,奪他那雷音寶剎。」這魔頭不識起倒,真個舉刀上前亂砍。卻被文殊、普賢念動真言,喝道:「這孽畜還不皈正,更待怎生?」諕得老怪、二怪不敢撐持,丟了兵器,打個滾,現了本相。二菩薩將蓮花臺拋在那怪的脊背上,飛身跨坐,二怪遂泯耳皈依。

二菩薩既收了青獅、白,只有那第三個妖魔不伏。騰開翅,丟了方天戟,扶搖直上,掄利爪要叼捉猴王。原來大聖藏在光中,他怎敢近。如來情知此意,即閃金光,把那鵲巢貫頂之頭迎風一幌,變做鮮紅的一塊血肉。妖精掄利爪叼他一下。被佛爺把手往上一指,那妖翅膊上就了筋,飛不去,只在佛頂上不能遠遁,現了本相,乃是一個大鵬金翅鵰。即開口對佛應聲叫道:「如來,你怎麼使大法力困住我也?」如來道:「你在此處多生孽障,跟我去,有進益之功。」妖精道:「你那裡持齋把素,極貧極苦;我這裡吃人肉,受用無窮。你若餓壞了我,你有罪愆。」如來道:「我管四大部洲,無數眾生瞻仰,凡做好事,我教他先祭汝口。」那大鵬欲脫難脫,要走怎走,是以沒奈何,只得皈依。

行者方才轉出,向如來叩頭道:「佛爺,你今收了妖精,除了大害,只是沒了我師父也。」大鵬咬著牙恨道:「潑猴頭!尋這等狠人困我。你那老和尚幾曾吃他?如今在那錦香亭鐵櫃裡不是?」行者聞言,忙叩頭謝了佛祖。佛祖不敢鬆放了大鵬,也只教他在光焰上做個護法,引眾回雲,徑歸寶剎。

行者卻按落雲頭,直入城裡,那城裡一個小妖兒也沒有了。正是:蛇無頭而不行,鳥無翅而不飛。他見佛祖收了妖王,各自逃生而去。行者才解救了八戒、沙僧,尋著行李、馬匹,與他二人說:「師父不曾吃,都跟我來。」引他兩個徑入內院,找著錦香亭,打開門看,內有一個鐵櫃,只聽得三藏有啼哭之聲。沙僧使降妖杖打開鐵鎖,揭開櫃蓋,叫聲:「師父。」三藏見了,放聲大哭道:「徒弟啊,怎生降得妖魔?如何得到此尋著我也?」行者把上項事從頭至尾,細說了一遍。三藏感謝不盡。師徒們在那宮殿裡尋了些米糧,安排些茶飯,飽吃一餐,收拾出城,找大路投西而去。正是:
\begin{quote}
真經必得真人取,意嚷心勞總是虛。
\end{quote}

畢竟這一去,不知幾時得面如來,且聽下回分解。
