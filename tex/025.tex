
\chapter{鎮元仙趕捉取經僧 孫行者大鬧五莊觀}

卻說他兄弟三眾到了殿上,對師父道:「飯將熟了,叫我們怎的?」三藏道:「徒弟,不是問飯。他這觀裡有甚麼人參果,似孩子一般的東西,你們是那一個偷他的吃了?」八戒道:「我老實,不曉得,不曾見。」清風道:「笑的就是他,笑的就是他。」行者喝道:「我老孫生的是這個笑容兒,莫成為你不見了甚麼果子,就不容我笑?」三藏道:「徒弟息怒。我們是出家人,休打誑語,莫吃昧心食。果然吃了他的,陪他個禮罷,何苦這般抵賴?」行者見師父說得有理,他就實說道:「師父,不干我事。是八戒隔壁聽見那兩個道童吃甚麼人參果,他想一個兒嘗新,著老孫去打了三個,我兄弟各人吃了一個。如今吃也吃了,待要怎麼?」明月道:「偷了我四個,這和尚還說不是賊哩。」八戒道:「阿彌陀佛!既是偷了四個,怎麼只拿出三個來分,預先就打起一個偏手?」那獃子倒轉胡嚷。

二仙童問得是實,越加毀罵。就恨得個大聖鋼牙咬響,火眼睜圓,把條金箍棒揝了又揝,忍了又忍道:「這童子只說當面打人也罷,受他些氣兒。送他個絕後計,教他大家都吃不成。」好行者,把腦後的毫毛拔了一根,吹口仙氣,叫:「變!」變做個假行者,跟定唐僧,陪著悟能、悟淨,忍受著道童嚷罵。他的真身出一個神,縱雲頭,跳將起去,徑到人參園裡,掣金箍棒,往樹上乒乓一下,又使個推山移嶺的神力,把樹一推推倒。可憐葉落枒開根出土,道人斷絕草還丹。那大聖推倒樹,在枝兒上尋果子,那裡得有半個。原來這寶貝遇金而落,他的棒兩頭是金裹的,況鐵又是五金之類,所以敲著就振下來;既下來,又遇土而入。因此上邊再沒一個果子。他道:「好,好,好!大家散火。」他收了鐵棒,徑往前來,把毫毛一抖,收上身來。那些人肉眼凡胎,看不明白。

卻說那仙童罵夠多時,清風道:「明月,這些和尚也受得氣哩,我們就像罵雞一般,罵了這半會,通沒個招聲,想必他不曾偷吃。倘或樹高葉密,數得不明,不要枉罵了他,我和你再去查查。」明月道:「也說得是。」他兩個果又到園中,只見那樹倒枒開,果無葉落。諕得清風腳軟跌根頭,明月腰酥打骸垢,那兩個魂飛魄散。有詩為證。詩曰:
\begin{quote}
三藏西臨萬壽山,悟空斷送草還丹。
枒開葉落仙根露,明月清風心膽寒。
\end{quote}

他兩個倒在塵埃,語言顛倒,只叫:「怎的好?怎的好?害了我五莊觀裡的丹頭,斷絕我仙家的苗裔,師父來家,我兩個怎的回話?」明月道:「師兄莫嚷,我們且整了衣冠,莫要驚張了這幾個和尚。這個沒有別人,定是那個毛臉雷公嘴的那廝,他來出神弄法,壞了我們的寶貝。若是與他分說,那廝畢竟抵賴,定要與他相爭;爭起來,就要交手相打,你想我們兩個怎麼敵得過他四個?且不如去哄他一哄,只說果子不少,我們錯數了,轉與他陪個不是。他們的飯已熟了,等他吃飯時,再貼他些兒小菜。他一家拿著一個碗,你卻站在門左,我卻站在門右,撲的把門關倒,把鎖鎖住,將這幾層門都鎖了,不要放他,待師父來家,憑他怎的處置。他又是師父的故人,饒了他,也是師父的人情;不饒他,我們也拿住個賊在,庶幾可以免我等之罪。」清風聞言道:「有理,有理。」

他兩個強打精神,勉生歡喜,從後園中徑來殿上,對唐僧控背躬身道:「師父,適間言語粗俗,多有衝撞,莫怪,莫怪。」三藏問道:「怎麼說?」清風道:「果子不少,只因樹葉高密,不曾看得明白。才然又去查查,還是原數。」那八戒就趁腳兒蹺道:「你這個童兒,年幼不知事體,就來亂罵,白口咀咒,枉賴了我們也,不當人子。」行者心上明白,口裡不言,心中暗想道:「是謊,是謊。果子已是了了帳,怎的說這般話?想必有起死回生之法。」三藏道:「既如此,盛將飯來,我們吃了去罷。」

那八戒便去盛飯,沙僧安放棹椅。二童忙取小菜,卻是些醬瓜、醬茄、糟蘿蔔、醋豆角、醃窩蕖、綽芥菜,共排了七八碟兒,與師徒們吃飯;又提一壺好茶,兩個茶鍾,伺候左右。那師徒四眾卻才拿起碗來,這童兒一邊一個,撲的把門關上,插上一把兩錤銅鎖。八戒笑道:「這童子差了,你這裡風俗不好,卻怎的關了門裡吃飯?」明月道:「正是,正是,好歹吃了飯兒開門。」清風罵道:「我把你這個害饞勞、偷嘴的禿賊!你偷吃了我的仙果,已該一個擅食田園瓜果之罪;卻又把我的仙樹推倒,壞了我五莊觀裡仙根,你還要說嘴哩。若能夠到得西方參佛面,只除是轉背搖車再托生。」三藏聞言,丟下飯碗,把塊石頭放在心上。那童子將那前山門、二山門,通都上了鎖。卻又來正殿門首,惡語惡言,賊前賊後,只罵到天色將晚,才去吃飯。飯畢,歸房去了。

唐僧埋怨行者道:「你這個猴頭,番番撞禍。你偷吃了他的果子,就受他些氣兒,讓他罵幾句便也罷了,怎麼又推倒他的樹?若論這般情由,告起狀來,就是你老子做官,也說不通。」行者道:「師父莫鬧,那童兒都睡去了,只等他睡著了,我們連夜起身。」沙僧道:「哥啊,幾層門都上了鎖,閉得甚緊,如何走麼?」行者笑道:「莫管,莫管,老孫自有法兒。」八戒道:「愁你沒有法兒哩,你一個變,甚麼蟲蛭兒,瞞格子眼裡就飛將出去。只苦了我們不會變的,便在此頂缸受罪哩。」唐僧道:「他若幹出這個勾當,不同你我出去啊,我就念起舊話經兒,他卻怎生消受?」八戒聞言,又愁又笑道:「師父,你說的那裡話?我只聽得佛教中有卷《楞嚴經》、《法華經》、《孔雀經》、《觀音經》、《金剛經》,不曾聽見個甚那『舊話兒經』啊。」行者道:「兄弟,你不知道。我頂上戴的這個箍兒,是觀音菩薩賜與我師父的,師父哄我戴了,就如生根的一般,莫想拿得下來,叫做緊箍兒咒,又叫做緊箍兒經。他『舊話兒經』,即此是也。但若念動,我就頭疼,故有這個法兒難我。師父,你莫念,我決不負你,管情大家一齊出去。」

說話後,都已天昏,不覺東方月上。行者道:「此時萬籟無聲,冰輪明顯,正好走了去罷。」八戒道:「哥啊,不要搗鬼,門俱鎖閉,往那裡走?」行者道:「你看手段。」把金箍棒捻在手中,使一個「解鎖法」,往門上一指,只聽得突蹡的一聲響,幾層門雙鐄俱落,唿喇的開了門扇。八戒笑道:「好本事,就是叫小爐兒匠使掭子,便也不像這等爽利。」行者道:「這個門兒有甚稀罕,就是南天門,指一指也開了。」卻請師父出了門,上了馬,八戒挑著擔,沙僧攏著馬,徑投西路而去。行者道:「你們且慢行,等老孫去照顧那兩個童兒睡一個月。」三藏道:「徒弟,不可傷他性命;不然,又一個得財傷人的罪了。」行者道:「我曉得。」行者復進去,來到那童兒睡的房門外。他腰裡有帶的瞌睡蟲兒,原來在東天門與增長天王猜枚耍子贏的。他摸出兩個來,瞞窗眼兒彈將進去,徑奔到那童子臉上,鼾鼾沉睡,再莫想得醒。他才拽開雲步,趕上唐僧,順大路一直西奔。

這一夜馬不停蹄,行到天曉。三藏道:「這個猴頭弄殺我也,你因為嘴,帶累我一夜無眠。」行者道:「不要只管埋怨。天色明了,你且在這路旁邊樹林中將就歇歇,養養精神再走。」那長老只得下馬,倚松根權作禪床坐下;沙僧歇了擔子打盹;八戒枕著石睡覺。孫大聖偏有心腸,你看他跳樹扳枝頑耍。四眾歇息不題。

卻說那大仙自元始宮散會,領眾小仙出離兜率,徑下瑤天,墜祥雲,早來到萬壽山五莊觀門首。看時,只見觀門大開,地上乾淨。大仙道:「清風、明月,卻也中用。常時節日高三丈,腰也不伸;今日我們不在,他倒肯起早,開門掃地。」眾小仙俱悅。行至殿上,香火全無,人蹤俱寂,那裡有明月、清風。眾仙道:「他兩個想是因我們不在,拐了東西走了。」大仙道:「豈有此理!修仙的人,敢有這般壞心的事?想是昨晚忘卻關門,就去睡了,今早還未醒哩。」眾仙到他房門首看處,真個關著房門,鼾鼾沉睡;任外邊打門亂叫,那裡叫得醒來。眾仙撬開門板,著手扯下床來,也只是不醒。大仙笑道:「好仙童啊,成仙的人,神滿再不思睡,卻怎麼這般困倦?莫不是有人做弄了他也?快取水來。」一童急取水半盞遞與大仙。大仙念動咒語,噀一口水,噴在臉上,隨即解了睡魔。

二人方醒,忽睜睛,抹抹臉,擡頭觀看,認得是仙師與世同君和仙兄等眾。慌得那清風頓首,明月叩頭道:「師父啊,你的故人原是東來的和尚,一夥強盜,十分兇狠。」大仙笑道:「莫驚恐,慢慢的說來。」清風道:「師父啊,當日別後不久,果有個東土唐僧,一行有四個和尚,連馬五口。弟子不敢違了師命,問及來因,將人參果取了兩個奉上。那長老俗眼愚心,不識我們仙家的寶貝。他說是三朝未滿的孩童,再三不吃。是弟子各吃了一個。不期他那手下有三個徒弟,有一個姓孫的,名悟空行者,先偷四個果子吃了。是弟子們向伊理說,實實的言語了幾句。他卻不容,暗自裡弄了個出神的手段。苦啊!」二童子說到此處,止不住腮邊淚落。眾仙道:「那和尚打你來?」明月道:「不曾打,只是把我們人參樹打倒了。」大仙聞言,更不惱怒,道:「莫哭,莫哭。你不知那姓孫的也是個太乙散仙,也曾大鬧天宮,神通廣大。既然打倒了寶樹,你可認得那些和尚?」清風道:「都認得。」大仙道:「既認得,都跟我來。——眾徒弟們,都收拾下刑具,等我回來打他。」眾仙領命。

大仙與明月、清風縱起祥光,來趕三藏,頃刻間就有千里之遙。大仙在雲端裡向西觀看,不見唐僧。及轉頭向東看時,倒多趕了九百餘里。原來那長老一夜馬不停蹄,只行了一百二十里路;大仙的雲頭,一縱趕過了九百餘里。仙童道:「師父,那路旁樹下坐的是唐僧。」大仙道:「我已見了。你兩個回去安排下繩索,等我自家拿他。」清風、明月先回不題。

那大仙按落雲頭,搖身一變,變作個行腳全真。你道他怎生打扮:
\begin{quote}
穿一領百衲袍,繫一條呂公絛。手搖麈尾,漁鼓輕敲。三耳草鞋登腳下,九陽巾子把頭包。飄飄風滿袖,口唱月兒高。
\end{quote}

徑直來到樹下,對唐僧高叫道:「長老,貧道起手了。」那長老忙忙答禮道:「失瞻,失瞻。」大仙問:「長老是那方來的?為何在途中打坐?」三藏道:「貧僧乃東土大唐差往西天取經者,路過此間,權為一歇。」大仙佯訝道:「長老東來,可曾在荒山經過?」長老道:「不知仙官是何寶山?」大仙道:「萬壽山五莊觀,便是貧道棲止處。」

行者聞言,他心中有物的人,忙答道:「不曾,不曾,我們是打上路來的。」那大仙指定笑道:「我把你這個潑猴!你瞞誰哩?你倒在我觀裡,把我人參果樹打倒,你連夜走在此間,還不招認,遮飾甚麼?不要走,趁早去還我樹來。」那行者聞言,心中惱怒,掣鐵棒,不容分說,望大仙劈頭就打。大仙側身躲過,踏祥光,徑到空中。行者也騰雲,急趕上去。大仙在半空現了本相,你看他怎生打扮:
\begin{quote}
頭戴紫金冠,無憂鶴氅穿。履鞋登足下,絲帶束腰間。體如童子貌,面似美人顏。三鬚飄頷下,鴉翎疊鬢邊。相迎行者無兵器,止將玉麈手中撚。
\end{quote}

那行者沒高沒低的,棍子亂打。大仙把玉麈左遮右擋,奈了他兩三回合。使一個「袖裡乾坤」的手段,在雲端裡把袍袖迎風輕輕的一展,刷地前來,把四僧連馬一袖子籠住。八戒道:「不好了,我們都裝在䌋縺裡了。」行者道:「獃子,不是䌋縺,我們被他籠在衣袖中哩。」八戒道:「這個不打緊,等我一頓釘鈀,築他個窟窿,脫將下去,只說他不小心,籠不牢,吊的了罷。」那獃子使鈀亂築,那裡築得動:手捻著雖然是個軟的,築起來就比鐵還硬。

那大仙轉祥雲,徑落五莊觀坐下,叫徒弟拿繩來。眾小仙一一伺候。你看他從袖子裡卻像撮傀儡一般,把唐僧拿出,縛在正殿簷柱上。又拿出他三個,每一根柱上綁了一個。將馬也拿出拴在庭下,與他些草料。行李拋在廊下。又道:「徒弟,這和尚是出家人,不可用刀槍,不可加鈇鉞。且與我取出皮鞭來,打他一頓,與我人參果出氣。」眾仙即忙取出一條鞭,——不是甚麼牛皮、羊皮、麂皮、犢皮的,原來是龍皮做的七星鞭,著水浸在那裡。令一個有力量的小仙,把鞭執定道:「師父,先打那個?」大仙道:「唐三藏做大不尊,先打他。」

行者聞言,心中暗道:「我那老和尚不禁打,假若一頓鞭打壞了啊,卻不是我造的孽?」他忍不住,開言道:「先生差了。偷果子是我,吃果子是我,推倒樹也是我,怎麼不先打我,打他做甚?」大仙笑道:「這潑猴倒言語膂烈。這等便先打他。」小仙問:「打多少?」大仙道:「照依果數,打三十鞭。」那小仙掄鞭就打。行者恐仙家法大,睜圓眼瞅定,看他打那裡。原來打腿,行者就把腰扭一扭,叫聲:「變!」變作兩條熟鐵腿,看他怎麼打。那小仙一下一下的打了三十,天早向午了。大仙又吩咐道:「還該打三藏訓教不嚴,縱放頑徒撒潑。」那仙又掄鞭來打。行者道:「先生又差了。偷果子時,我師父不知,他在殿上與你二童講話,是我兄弟們做的勾當。縱是有教訓不嚴之罪,我為弟子的也當替打,再打我罷。」大仙道:「這潑猴,雖是狡猾奸頑,卻倒也有些孝意。既這等,還打他罷。」小仙又打了三十。行者低頭看看,兩隻腿似明鏡一般,通打亮了,更不知些疼癢。此時天色將晚,大仙道:「且把鞭子浸在水裡,待明朝再拷打他。」小仙且收鞭去浸,各各歸房。晚齋已畢,盡皆安寢不題。

那長老淚眼雙垂,怨他三個徒弟道:「你等闖出禍來,卻帶累我在此受罪,這是怎的起?」行者道:「且休報怨,打便先打我,你又不曾吃打,倒轉嗟呀怎的?」唐僧道:「雖然不曾打,卻也綁得身上疼哩。」沙僧道:「師父,還有陪綁的在這裡哩。」行者道:「都不要嚷,再停會兒走路。」八戒道:「哥哥又弄虛頭了。這裡麻繩噴水,緊緊的綁著,還比關在殿上,被你使解鎖法搠開門走哩。」行者道:「不是誇口說,那怕他三股的麻繩噴上了水,就是碗粗的棕纜,也只好當秋風。」

正話處,早已萬籟無聲,正是天街人靜。好行者,把身子小一小,脫下索來道:「師父去啞。」沙僧慌了道:「哥哥,也救我們一救。」行者道:「悄言,悄言。」他卻解了三藏,放下八戒、沙僧,整束了褊衫,扣背了馬匹,廊下拿了行李,一齊出了觀門。又教八戒:「你去把那崖邊柳樹伐四顆來。」八戒道:「要他怎的?」行者道:「有用處,快快取來。」那獃子有些夯力,走了去,一嘴一顆,就拱了四顆,一抱抱來。行者將枝梢折了,教兄弟二人復進去,將原繩照舊綁在柱上。那大聖念動咒語,咬破舌尖,將血噴在樹上,叫:「變!」一根變作長老,一根變作自身,那兩根變作沙僧、八戒;都變得容貌一般,相貌皆同,問他也就說話,叫名也就答應。他兩個卻才放開步,趕上師父。這一夜依舊馬不停蹄,躲離了五莊觀。

只是到天明,那長老在馬上搖樁打盹。行者見了,叫道:「師父不濟,出家人怎的這般辛苦?我老孫千夜不眠,也不曉得些困倦。且下馬來,莫教走路的人看見笑你,權在山坡下藏風聚氣處歇歇再走。」

不說他師徒在路暫住。且說那大仙天明起來,吃了早齋,出在殿上,教:「拿鞭來,今日卻該打唐三藏了。」那小仙掄著鞭,望唐僧道:「打你哩。」那柳樹也應道:「打麼。」乒乓打了三十。掄過鞭來,對八戒道:「打你哩。」那柳樹也應道:「打麼。」及打沙僧,也應道教打。及打到行者,那行者在路,偶然打個寒噤道:「不好了!」三藏問道:「怎麼說?」行者道:「我將四顆柳樹變作我師徒四眾,我只說他昨日打了我兩頓,今日想不打了,卻又打我的化身,所以我真身打噤。收了法罷。」那行者慌忙念咒收法。

你看那些道童害怕,丟了皮鞭,報道:「師父啊,為頭打的是大唐和尚,這一會打的都是柳樹之根。」大仙聞言,呵呵冷笑,誇不盡道:「孫行者,真是一個好猴王。曾聞他大鬧天宮,佈地網天羅,拿他不住,果有此理。——你走了便也罷,卻怎麼綁些柳樹在此冒名頂替?決莫饒他,趕去來。」

那大仙說聲趕,縱起雲頭,往西一望,只見那和尚挑包策馬,正然走路。大仙低落雲頭,叫聲:「孫行者,往那裡走?還我人參樹來。」八戒聽見道:「罷了,對頭又來了。」行者道:「師父,且把善字兒包起,讓我們使些兇惡,一發結果了他,脫身去罷。」唐僧聞言,戰戰兢兢,未曾答應。沙僧掣寶杖,八戒舉釘鈀,大聖使鐵棒,一齊上前,把大仙圍住在空中,亂打亂築。這場惡鬥,有詩為證。詩曰:
\begin{quote}
悟空不識鎮元仙,與世同君妙更玄。
三件神兵施猛烈,一根麈尾自飄然。
左遮右擋隨來往,後架前迎任轉旋。
夜去朝來難脫體,淹留何日到西天!
\end{quote}

他兄弟三眾各舉神兵,一齊攻打;那大仙只把蠅帚兒演架。那裡有半個時辰,他將袍袖一展,依然將四僧一馬並行李一袖籠去。返雲頭,又到觀裡,眾仙接著。仙師坐於殿上,卻又在袖兒裡一個個搬出:將唐僧綁在階下矮槐樹上;八戒、沙僧各綁在兩邊樹上;將行者捆倒。行者道:「想是調問哩。」不一時,捆綁停當,教把長頭布取十疋來。行者笑道:「八戒,這先生好意思,拿出布來與我們做中袖哩。減省些兒,做個一口中罷了。」那小仙將家機布搬將出來。大仙道:「把唐三藏、豬八戒、沙和尚都使布裹了。」眾仙一齊上前裹了。行者笑道:「好,好,好,夾活兒就大殮了。」須臾,纏裹已畢。又教拿出漆來。眾仙即忙取了些自收自晒的生熟漆,把他三個渾身布裹漆了,渾身俱裹漆,上留著頭臉在外。八戒道:「先生,上頭倒不打緊,只是下面還留孔兒,我們好出恭。」那大仙又教把大鍋擡出來。行者笑道:「八戒,造化,擡出鍋來,想是煮飯我們吃哩。」八戒道:「也罷了,讓我們吃些飯兒,做個飽死的鬼也好看。」眾仙果擡出一口大鍋支在階下。大仙叫架起乾柴,發起烈火,教:「把清油拗上一鍋,燒得滾了,將孫行者下油鑊炸他一煠,與我人參樹報仇。」

行者聞言,暗喜道:「正可老孫之意,這一向不曾洗澡,有些兒皮膚燥癢,好歹燙燙,足感盛情。」頃刻間,那油鍋將滾。大聖卻又留心,恐他仙法難參,油鍋裡難做手腳,急回頭四顧,只見那臺下東邊是一座日規臺,西邊是一個石獅子。行者將身一縱,滾到西邊,咬破舌尖,把石獅子噴了一口,叫聲:「變!」變作他本身模樣,也這般捆作一團。他卻出了元神,起在雲端裡,低頭看著道士。

只見那小仙報道:「師父,油鍋滾透了。」大仙教:「把孫行者擡下去。」四個仙童擡不動,八個來也擡不動,又加四個也擡不動。眾仙道:「這猴子戀土難移,小自小,倒也結實。」卻教二十個小仙扛將起來,往鍋裡一摜,烹的響了一聲,濺起些滾油點子,把那小道士們臉上燙了幾個燎漿大泡。只聽得燒火的小童喊道:「鍋漏了,鍋漏了。」說不了,油已漏得罄盡,鍋底打破,原來是一個石獅子放在裡面。

大仙大怒道:「這個潑猴,著然無禮,教他當面做了手腳。你走了便罷,怎麼又搗了我的灶?這潑猴枉自也拿他不住;就拿住他,也似摶砂弄汞,捉影捕風。罷,罷,罷,饒他去罷。且將唐三藏解下,另換新鍋,把他扎一扎,與人參樹報報仇罷。」那小仙真個動手,拆解布漆。

行者在半空裡聽得明白,他想著:「師父不濟,他若到了油鍋裡,一滾就死,二滾就焦,到三五滾他就弄做個稀爛的和尚了。我還去救他一救。」好大聖,按落雲頭,上前叉手道:「莫要拆壞了布漆,扎我師父,還等我來下油鍋罷。」那大仙驚罵道:「我把你這猢猴!怎麼弄手段搗了我的灶?」行者笑道:「你遇著我就該倒灶,干我甚事?我才自也要領你些油湯油水之愛,但只是大小便急了,若在鍋裡開風,恐怕污了你的熟油,不好調菜吃。如今大小便通乾淨了,才好下鍋。不要扎我師父,還來扎我罷。」那大仙聞言,呵呵冷笑,走出殿來,一把扯住。

畢竟不知有何話說,端的怎麼脫身,且聽下回分解。
