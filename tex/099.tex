
\chapter{九九數完魔滅盡 三三行滿道歸根}

話表八金剛既送唐僧回國不題。那三層門下,有五方揭諦、四值功曹、六丁六甲、護教伽藍,走向觀音菩薩前啟道:「弟子等向蒙菩薩法旨,暗中保護聖僧,今日聖僧行滿,菩薩繳了佛祖金旨,我等望菩薩准繳法旨。」菩薩亦甚喜道:「准繳,准繳。」又問道:「那唐僧四眾,一路上心行何如?」諸神道:「委實心虔志誠,料不能逃菩薩洞察。但只是唐僧受過之苦,真不可言。他一路上歷過的災愆患難,弟子已謹記在此,這就是他災難的簿子。」菩薩從頭看了一遍,上寫著:
\begin{quote}
蒙差揭諦皈依旨,謹記唐僧難數清:
金蟬遭貶第一難。出胎幾殺第二難。
滿月拋江第三難。尋親報冤第四難。
出城逢虎第五難。折從落坑第六難。
雙叉嶺上第七難。兩界山頭第八難。
陡澗換馬第九難。夜被火燒第十難。
失卻袈裟十一難。收降八戒十二難。
黃風怪阻十三難。請求靈吉十四難。
流沙難渡十五難。收得沙僧十六難。
四聖顯化十七難。五莊觀中十八難。
難活人參十九難。貶退心猿二十難。
黑松林失散二十一難。寶象國捎書二十二難。
金鑾殿變虎二十三難。平頂山逢魔二十四難。
蓮花洞高懸二十五難。烏雞國救主二十六難。
被魔化身二十七難。號山逢怪二十八難。
風攝聖僧二十九難。心猿遭害三十難。
請聖降妖三十一難。黑河沉沒三十二難。
搬運車遲三十三難。大賭輸贏三十四難。
祛道興僧三十五難。路逢大水三十六難。
身落天河三十七難。魚籃現身三十八難。
山遇怪三十九難。普天神難伏四十難。
問佛根源四十一難。吃水遭毒四十二難。
西梁國留婚四十三難。琵琶洞受苦四十四難。
再貶心猿四十五難。難辨獼猴四十六難。
路阻火焰山四十七難。求取芭蕉扇四十八難。
收縛魔王四十九難。賽城掃塔五十難。
取寶救僧五十一難。棘林吟詠五十二難。
小雷音遇難五十三難。諸天神遭困五十四難。
稀柿衕穢阻五十五難。朱紫國行醫五十六難。
拯救疲癃五十七難。降妖取后五十八難。
七情迷沒五十九難。多目遭傷六十難。
路阻獅駝六十一難。怪分三色六十二難。
城裡遇災六十三難。請佛收魔六十四難。
比丘救子六十五難。辨認真邪六十六難。
松林救怪六十七難。僧房臥病六十八難。
無底洞遭困六十九難。滅法國難行七十難。
隱霧山遇魔七十一難。鳳仙郡求雨七十二難。
失落兵器七十三難。會慶釘鈀七十四難。
竹節山遭難七十五難。玄英洞受苦七十六難。
趕捉犀牛七十七難。天竺招婚七十八難。
銅臺府監禁七十九難。凌雲渡脫胎八十難。
路經十萬八千里。聖僧歷難簿分明。
\end{quote}

菩薩將難簿目過了一遍,急傳聲道:「佛門中九九歸真。聖僧受過八十難,還少一難,不得完成此數。」即命揭諦:「趕上金剛,還生一難者。」

這揭諦得令,飛雲一駕向東來,一晝夜趕上八大金剛,附耳低言道:「如此如此,謹遵菩薩法旨,不得違誤。」八金剛聞得此言,刷的把風按下,將他四眾連馬與經墜落下地。噫!正是那:
\begin{quote}
九九歸真道行難,堅持篤志立玄關。
必須苦煉邪魔退,定要修持正法還。
莫把經章當容易,聖僧難過許多般。
古來妙合參同契,毫髮差殊不結丹。
\end{quote}

三藏腳踏了凡地,自覺心驚。八戒呵呵大笑道:「好好好,這正是要快得遲。」沙僧道:「好好好,因是我們走快了些兒,教我們在此歇歇哩。」大聖道:「俗語云:『十日灘頭坐,一日行九灘。』」三藏道:「你三個且休鬥嘴,認認方向,看這是甚麼地方?」沙僧轉頭四望道:「是這裡,是這裡。師父,你聽聽水響。」行者道:「水響想是你的祖家了。」八戒道:「他祖家乃流沙河。」沙僧道:「不是,不是,此通天河也。」三藏道:「徒弟啊,仔細看在那岸?」行者縱身跳起,用手搭涼篷,仔細看了,下來道:「師父,此是通天河西岸。」三藏道:「我記起來了。東岸邊原有個陳家莊。那年到此,虧你救了他兒女,深感我們,要造船相送,幸白黿伏渡。我記得西岸上四無人煙,這番如何是好?」八戒道:「只說凡人會作弊,原來這佛面前的金剛也會作弊。他奉佛旨,教送我們東回,怎麼到此半路上就丟下我們?如今豈不進退兩難?怎生過去?」沙僧道:「二哥休報怨。我的師父已得了道,前在凌雲渡已脫了凡胎,今番斷不落水。教師兄同你我都作起攝法,把師父駕過去也。」行者頻頻的暗笑道:「駕不去,駕不去。」你看他怎麼就說個駕不去?若肯使出神通,說破飛昇之奧妙,師徒們就一千個河也過去了。只因心裡明白,知道唐僧九九之數未完,還該有一難,故羈留於此。

師徒們口裡紛紛的講,足下徐徐的行,直至水邊。忽聽得有人叫道:「唐聖僧,唐聖僧,這裡來,這裡來。」四眾皆驚。舉頭觀看,四無人跡,又沒舟船,卻是一個大白賴頭黿在岸邊探著頭叫道:「老師父,我等了你這幾年,卻才回也?」行者笑道:「老黿,向年累你,今歲又得相逢。」三藏與八戒、沙僧都歡喜不盡。行者道:「老黿,你果有接待之心,可上岸來。」那黿即縱身爬上河來。行者叫把馬牽上他身,八戒還蹲在馬尾之後,唐僧站在馬頸左邊,沙僧站在右邊。行者一腳踏著老黿的項,一腳踏著老黿的頭叫道:「老黿,好生走穩著。」那老黿蹬開四足,踏水面如行平地,將他師徒四眾,連馬五口,馱在身上,徑回東岸而來。誠所謂:
\begin{quote}
不二門中法奧玄,諸魔戰退識人天。
本來面目今方見,一體原因始得全。
秉證三乘隨出入,丹成九轉任周旋。
挑包飛杖通休講,幸喜還元遇老黿。
\end{quote}

老黿馱著他們,屣波踏浪,行經多半日,將次天晚,好近東岸,忽然問曰:「老師父,我向年曾央到西方見我佛如來,與我問聲歸著之事,還有多少年壽,果曾問否?」原來那長老自到西天玉真觀沐浴,凌雲渡脫胎,步上靈山,專心拜佛,及參諸佛菩薩聖僧等眾,意念只在取經,他事一毫不理,所以不曾問得老黿年壽,無言可答。卻又不敢欺,打誑語,沉吟半晌,不曾答應。老黿即知不曾替他問了,就將身一幌,唿喇的淬下水去,把他四眾連馬并經,通皆落水。咦!還喜得唐僧脫了胎,成了道;若似前番,已經沉底。又幸白馬是龍,八戒、沙僧會水,行者笑微微顯大神通,把唐僧扶駕出水,登彼東岸。只是經包、衣服、鞍轡俱濕了。

師徒方登岸整理,忽又一陣狂風,天色昏暗,雷閃並作,走石飛沙。但見那:
\begin{quote}
一陣風,乾坤播蕩;一聲雷,振動山川。一個熌,鑽雲飛火;一天霧,大地遮漫。風氣呼號,雷聲激烈。熌掣紅銷,霧迷星月。風鼓的沙塵撲面,雷驚的虎豹藏形。熌晃的飛禽叫噪,霧漫的樹木無蹤。那風攪得個通天河波浪翻騰,那雷振得個通天河魚龍喪膽。那熌照得個通天河徹底光明,那霧蓋得個通天河岸崖昏慘。好風,頹山烈石松篁倒。好雷,驚蟄傷人威勢豪。好熌,流天照野金蛇走。好霧,混混漫空蔽九霄。
\end{quote}

諕得那三藏按住了經包,沙僧壓住了經擔,八戒牽住了白馬;行者卻雙手輪起鐵棒,左右護持。原來那風、霧、雷、熌,乃是些陰魔作號,欲奪所取之經。勞攘了一夜,直到天明,卻才止息。長老一身水衣,戰兢兢的道:「悟空,這是怎的起?」行者氣呼呼的道:「師父,你不知就裡。我等保護你取獲此經,乃是奪天地造化之功,可以與乾坤並久,日月同明,壽享長春,法身不朽。此所以為天地不容,鬼神所忌,欲來暗奪之耳。一則這經是水濕透了;二則是你的正法身壓住,雷不能轟,電不能照,霧不能迷;又是老孫輪著鐵棒,使純陽之性,護持住了;及至天明,陽氣又盛:所以不能奪去。」三藏、八戒、沙僧方才省悟,各謝不盡。

少頃,太陽高照。卻移經於高崖上,開包曬晾。至今彼處曬經之石尚存。他們又將衣鞋都曬在崖傍,立的立,坐的坐,跳的跳。真個是:
\begin{quote}
一體純陽喜向陽,陰魔不敢逞強梁。
須知水勝真經伏,不怕風雷熌霧光。
自此清平歸正覺,從今安泰到仙鄉。
曬經石上留蹤跡,千古無魔到此方。
\end{quote}

他四眾檢看經本,一一曬晾,早見幾個打魚人來到河邊,擡頭看見。內有認得的道:「老師父可是前年過此河往西天取經的?」八戒道:「正是,正是。你是那裡人?怎麼認得我們?」漁人道:「我們是陳家莊上人。」八戒道:「陳家莊離此有多遠?」漁人道:「過此衝南有二十里就是也。」八戒道:「師父,我們把經搬到陳家莊上曬去,他那裡有住有坐,又有得吃,就教他家與我們漿漿衣服,卻不是好?」三藏道:「不去罷,在此曬乾了,就收拾找路回也。」

那幾個漁人行過南衝,恰遇著陳澄,叫道:「二老官,前年在你家替祭兒子的師父回來了。」陳澄道:「你在那裡看見?」漁人回指道:「都在那石上曬經哩。」陳澄隨帶了幾個佃戶,走過衝來望見,跑近前跪下道:「老爺取經回來,功成行滿,怎麼不到舍下,卻在這裡盤弄?快請,快請到舍。」行者道:「等曬乾了經,和你去。」陳澄又問道:「老爺的經典、衣物,如何濕了?」三藏道:「昔年虧白黿馱渡河西,今年又蒙他馱渡河東。已將近岸,被他問昔年託問佛祖壽年之事,我本未曾問得,他遂淬在水內,故此濕了。」又將前後事細說了一遍。那陳澄拜請甚懇,三藏無已,遂收拾經卷。不期石上把《佛本行經》沾住了幾卷,遂將經尾沾破了。所以至今《佛本行經》不全,曬經石上猶有字跡。三藏懊悔道:「是我們怠慢了,不曾看顧得。」行者笑道:「不在此,不在此。蓋天地不全,這經原是全全的,今沾破了,乃是應不全之奧妙也,豈人力所能與耶?」師徒們果收拾畢,同陳澄赴莊。

那莊上人家,一個傳十,十個傳百,百個傳千,若老若幼,都來接看。陳清聞說,就擺香案在門前迎迓,又命鼓樂吹打。少頃到了,迎入。陳清領合家人眷,俱出來拜見,拜謝昔日救兒女之恩。隨命看茶擺齋。三藏自受了佛祖的仙品、仙餚,又脫了凡胎成佛,全不思凡間之食。二老苦勸,沒奈何,略見他意。孫大聖自來不吃煙火食,也道:「夠了。」沙僧也不甚吃。八戒也不似前番,就放下碗。行者道:「獃子也不吃了?」八戒道:「不知怎麼,脾胃一時就弱了。」遂此收了齋筵,卻又問取經之事。三藏又將先至玉真觀沐浴,凌雲渡身輕,及至雷音寺參如來,蒙珍樓賜宴,寶閣傳經,始被二尊者討人事未遂,故傳無字之經,後復拜告如來,始得授一藏之數,并白黿淬水,陰魔暗奪之事,細細陳了一遍,就欲拜別。

那二老舉家如何肯放,且道:「向蒙救拔兒女,深恩莫報。已創建一座院宇,名之曰救生寺,專侍奉香火不絕。」又喚出原替祭之兒女陳關保、一秤金叩謝。復請至寺觀看。三藏卻又將經包兒收在他家堂前,與他念了一卷《寶常經》。後至寺中,只見陳家又設饌在此。還不曾坐下,又一起來請。還不曾舉箸,又一起來請。絡繹不絕,爭不上手。三藏俱不敢辭,略略見意。只見那座寺果蓋得齊整:
\begin{quote}
山門紅粉膩,多賴施主功。一座樓臺從此立,兩廊房宇自今興。朱紅隔扇,七寶玲瓏。香氣飄雲漢,清光滿太空。幾株嫩柏還澆水,數榦喬松未結叢。活水迎前,通天疊疊翻波浪;高崖倚後,山脈重重接地龍。
\end{quote}

三藏看畢,才上高樓。樓上果裝塑著他四眾之像。八戒看見,扯著行者道:「兄長的相兒甚像。」沙僧道:「二哥,你的又像得緊,只是師父的又忒俊了些兒。」三藏道:「卻好,卻好。」遂下樓來。下面前殿後廊,還有擺齋的候請。行者卻問:「向日大王廟兒如何了?」眾老道:「那廟當年拆了。老爺,這寺自建立之後,年年成熟,歲歲豐登,卻是老爺之福庇。」行者笑道:「此天賜耳,與我們何與?但只我們自今去後,保你這一莊上人家子孫繁衍,六畜安生,年年風調雨順,歲歲雨順風調。」眾等卻叩頭拜謝。

只見那前前後後,更有獻果獻齋的無限人家。八戒笑道:「我的蹭蹬。那時節吃得,卻沒人家連請十請。今日吃不得,卻一家不了,又是一家。」饒他氣滿,略動手,又吃夠八九盤素食;縱然胃傷,又吃了二三十個饅頭。已皆盡飽,又有人家相邀。三藏道:「弟子何能,感蒙至愛。望今夕暫停,明早再領。」

時已深夜。三藏守定真經,不敢暫離,就於樓下打坐看守。將及三更,三藏悄悄的叫道:「悟空,這裡人家。識得我們道成事完了。自古道:『真人不露相,露相不真人。』恐為久淹,失了大事。」行者道:「師父說得有理。我們趁此深夜,人家熟睡,寂寂的去了罷。」八戒卻也知覺,沙僧盡自分明,白馬也能會意。遂此起了身,輕輕的擡上馱垛,挑著擔,從廡廊馱出,到於山門,只見門上有鎖。行者又使個解鎖法,開了二門、大門,找路望東而去。只聽得半空中有八大金剛叫道:「逃走的,跟我來。」那長老聞得香風蕩蕩,起在空中。這正是:
\begin{quote}
丹成識得本來面,體健如如拜主人。
\end{quote}

畢竟不知怎生見那唐王,且聽下回分解。
