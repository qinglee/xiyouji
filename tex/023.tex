
\chapter{三藏不忘本 四聖試禪心}

\begin{quote}
奉法西來道路賒,秋風淅淅落霜花。
乖猿牢鎖繩休解,劣馬勤兜鞭莫加。
木母金公原自合,黃婆赤子本無差。
咬開鐵彈真消息,般若波羅到彼家。
\end{quote}

這回書蓋言取經之道,不離了一身務本之道也。卻說他師徒四眾了悟真如,頓開塵鎖,自跳出性海流沙,渾無罣礙,徑投大路西來。歷遍了青山綠水,看不盡野草閑花。真個也光陰迅速,又值九秋,但見了些:
\begin{quote}
楓葉滿山紅,黃花耐晚風。
老蟬吟漸懶,愁蟋思無窮。
荷破青紈扇,橙香金彈叢。
可憐數行雁,點點遠排空。
\end{quote}

正走處,不覺天晚。三藏道:「徒弟,如今天色又晚,卻往那裡安歇?」行者道:「師父說話差了。出家人餐風宿水,臥月眠霜,隨處是家。又問那裡安歇,何也?」豬八戒道:「哥啊,你可知道你走路輕省,那裡管別人累墜?自過了流沙河,這一向爬山過嶺,身挑著重擔,老大難挨也。須是尋個人家,一則化些茶飯,二則養養精神,才是個道理。」行者道:「獃子,你這般言語,似有報怨之心。還像在高老莊倚懶不求福的自在,恐不能也。既是秉正沙門,須是要吃辛受苦,才做得徒弟哩。」八戒道:「哥哥,你看這擔行李多重?」行者道:「兄弟,自從有了你與沙僧,我又不曾挑著,那知多重?」八戒道:「哥啊,你看看數兒麼:
\begin{quote}
四片黃藤蔑,長短八條繩。又要防陰雨,氈包三四層。匾擔還愁滑,兩頭釘上釘。銅鑲鐵打九環杖,篾絲藤纏大斗篷。
\end{quote}

似這般許多行李,難為老豬一個逐日家擔著走,偏你跟師父做徒弟,拿我做長工。」行者笑道:「獃子,你和誰說哩?」八戒道:「哥哥,與你說哩。」行者道:「錯和我說了。老孫只管師父好歹,你與沙僧專管行李、馬匹。但若怠慢了些兒,孤拐上先是一頓粗棍。」八戒道:「哥啊,不要說打,打就是以力欺人。我曉得你的尊性高傲,你是定不肯挑。但師父騎的馬,那般高大肥盛,只馱著老和尚一個,教他帶幾件兒,也是弟兄之情。」

行者道:「你說他是馬哩,他不是凡馬,本是西海龍王敖閏之子,喚名龍馬三太子。只因縱火燒了殿上明珠,被他父親告了忤逆,身犯天條,多虧觀音菩薩救了他的性命。他在那鷹愁陡澗久等師父,又幸得菩薩親臨,卻將他退鱗去角,摘了項下珠,才變做這匹馬,願馱師父往西天拜佛。這個都是各人的功果,你莫攀他。」那沙僧聞言道:「哥哥,真個是龍麼?」行者道:「是龍。」八戒道:「哥啊,我聞得古人云:龍能噴雲噯霧,播土揚沙;有巴山㨝嶺的手段,有翻江攪海的神通。怎麼他今日這等慢慢而走?」行者道:「你要他快走,我教他快走個兒你看。」

好大聖,把金箍棒揝一揝,萬道彩雲生。那馬看見拿棒,恐怕打來,慌得四隻蹄疾如飛電,颼的跑將去了。那師父手軟勒不住,盡他劣性,奔上山崖,才大達辿步走。師父喘息始定,擡頭遠見一簇松陰,內有幾間房舍,著實軒昂,但見:
\begin{quote}
門垂翠柏,宅近青山。幾株松冉冉,數莖竹斑斑。籬邊野菊凝霜艷,橋畔幽蘭映水丹。粉泥牆壁,磚砌圍圜。高堂多壯麗,大廈甚清安。牛羊不見無雞犬,想是秋收農事閑。
\end{quote}

那師父正按轡徐觀,又見悟空兄弟方到。悟淨道:「師父不曾跌下馬來麼?」長老罵道:「悟空這潑猴,他把馬兒驚了,早是我還騎得住哩。」行者陪笑道:「師父莫罵我,都是豬八戒說馬行遲,故此著他快些。」那獃子因趕馬,走急了些兒,喘氣噓噓,口裡唧唧噥噥的鬧道:「罷了,罷了,見自肚,別腰鬆。擔子沉重,挑不上來,又弄我奔奔波波的趕馬。」長老道:「徒弟啊,你且看那壁廂有一座莊院,我們卻好借宿去也。」行者聞言,急擡頭舉目而看,果見那半空中慶雲籠罩,瑞靄遮盈,情知定是佛仙點化,他卻不敢泄漏天機,只道:「好好好,我們借宿去來。」

長老連忙下馬。見一座門樓,乃是垂蓮象鼻,畫棟雕梁。沙僧歇了擔子,八戒牽了馬匹道:「這個人家,是過當的富實之家。」行者就要進去,三藏道:「不可,你我出家人,各自避些嫌疑,切莫擅入。且自等他有人出來,以禮求宿,方可。」八戒拴了馬,斜倚牆根之下;三藏坐在石鼓上;行者、沙僧坐在臺基邊。久無人出,行者性急,跳起身入門裡看處,原來有向南的三間大廳,簾櫳高控。屏門上掛一軸壽山福海的橫披畫;兩邊金漆柱上,貼著一副大紅紙的春聯,上寫著:「絲飄弱柳平橋晚;雪點香梅小院春。」正中間設一張退光黑漆的香几,几上放一個古銅獸爐。上有六張交椅兩山頭掛著四季吊屏。

行者正然偷看處,忽聽得後門內有腳步之聲,走出一個半老不老的婦人來,嬌聲問道:「是甚麼人,擅入我寡婦之門?」慌得個大聖喏喏連聲道:「小僧是東土大唐來的,奉旨向西方拜佛求經。一行四眾,路過寶方,天色已晚,特奔老菩薩檀府,告借一宵。」那婦人笑語相迎道:「長老,那三位在那裡?請來。」行者高聲叫道:「師父,請進來耶。」三藏才與八戒、沙僧牽馬挑擔而入,只見那婦人出廳迎接。八戒餳眼偷看,你道他怎生打扮:
\begin{quote}
穿一件織錦官綠紵絲襖,上罩著淺紅比甲;繫一條結綵鵝黃錦繡裙,下映著高底花鞋。時樣髻皂紗漫,相襯著二色盤龍髮;宮樣牙梳朱翠晃,斜簪著兩股釧金釵。雲鬢半蒼飛鳳翅,耳環雙墜寶珠排。脂粉不施猶自美,風流還似少年才。
\end{quote}

那婦人見了他三眾,更加欣喜,以禮邀入廳房。一一相見禮畢,請各敘坐看茶。那屏風後,忽有一個丫髻垂絲的女童,托著黃金盤、白玉盞,香茶噴暖氣,異果散幽香。那人綽彩袖,春筍纖長;擎玉盞,傳茶上奉。對他們一一拜了。茶畢,又吩咐辦齋。三藏啟手道:「老菩薩,高姓?貴地是甚地名?」婦人道:「此間乃西牛賀洲之地。小婦人娘家姓賈,夫家姓莫。幼年不幸,公姑早亡,與丈夫守承祖業。有家貲萬貫,良田千頃。夫妻們命裡無子,止生了三個女孩兒。前年大不幸,又喪了丈夫。小婦居孀,今歲服滿。空遺下田產家業,再無個眷族親人,只是我娘女們承領。欲嫁他人,又難捨家業。適承長老下降,想是師徒四眾,小婦娘女四人,意欲坐山招夫,四位恰好。不知尊意肯否如何。」三藏聞言,推聾妝啞,瞑目寧心,寂然不答。

那婦人道:「舍下有水田三百餘頃,旱田三百餘頃,山場果木三百餘頃;黃水牛有一千餘隻,騾馬成群,豬羊無數;東南西北,莊堡草場,共有六七十處;家下有八九年用不著的米穀,十來年穿不著的綾羅,一生有使不著的金銀:勝強似那錦帳藏春,說甚麼金釵兩行。你師徒們若肯回心轉意,招贅在寒家,自自在在,享用榮華,卻不強如往西勞碌?」那三藏也只是如痴如蠢,默默無言。

那婦人道:「我是丁亥年三月初三日酉時生,故夫比我年大三歲,我今年四十五歲。大女兒名真真,今年二十歲;次女名愛愛,今年十八歲;三小女名憐憐,今年十六歲:俱不曾許配人家。雖是小婦人醜陋,卻幸小女俱有幾分顏色,女工針指,無所不會。因是先夫無子,即把他們當兒子看養,小時也曾教他讀些儒書,也都曉得些吟詩作對。雖然居住山莊,也不是那十分粗俗之類,料想也配得過列位長老。若肯放開懷抱,長髮留頭,與舍下做個家長,穿綾著錦,勝強如那瓦缽緇衣,芒鞋雲笠。」三藏坐在上面,好便似雷驚的孩子,雨淋的蝦蟆:只是呆呆掙掙,翻白眼兒打仰。

那八戒聞得這般富貴,這般美色,他卻心癢難撓;坐在那椅子上,一似針戳屁股,左扭右扭的,忍耐不住。走上前,扯了師父一把道:「師父,這娘子告誦你話,你怎麼佯佯不睬?好道也做個理會是。」那師父猛擡頭,咄的一聲,喝退了八戒道:「你這個孽畜!我們是個出家人,豈以富貴動心,美色留意,成得個甚麼道理。」

那婦人笑道:「可憐,可憐。出家人有何好處?」三藏道:「女菩薩,你在家人,卻有何好處?」那婦人道:「長老請坐,等我把在家人好處說與你聽。怎見得?有詩為證。詩曰:
\begin{quote}
春裁方勝著新羅,夏換輕紗賞綠荷;
秋有新蒭香糯酒,冬來暖閣醉顏酡。
四時受用般般有,八節珍饈件件多。
襯錦鋪綾花燭夜,強如行腳禮彌陀。」
\end{quote}

三藏道:「女菩薩,你在家人享榮華,受富貴,有可穿,有可吃,兒女團圓,果然是好。但不知我出家的人,也有一段好處。怎見得?有詩為證。詩曰:
\begin{quote}
出家立志本非常,推倒從前恩愛堂。
外物不生閑口舌,身中自有好陰陽。
功完行滿朝金闕,見性明心返故鄉。
勝似在家貪血食,老來墜落臭皮囊。」
\end{quote}

那婦人聞言,大怒道:「這潑和尚無禮!我若不看你東土遠來,就該叱出。我倒是個真心實意,要把家緣招贅汝等,你倒反將言語傷我。你就是受了戒,發了願,永不還俗,好道你手下人,我家也招得一個,你怎麼這般執法?」三藏見他發怒,只得者者謙謙,叫道:「悟空,你在這裡罷。」行者道:「我從小兒不曉得幹那般事,教八戒在這裡罷。」八戒道:「哥啊,不要栽人麼,大家從長計較。」三藏道:「你兩個不肯,便教悟淨在這裡罷。」沙僧道:「你看師父說的話。弟子蒙菩薩勸化,受了戒行,等候師父。自蒙師父收了我,又承教誨,跟著師父還不上兩月,更不曾進得半分功果,怎敢圖此富貴?寧死也要往西天去,決不幹此欺心之事。」

那婦人見他們推辭不肯,急抽身轉進屏風,撲的把腰門關上。師徒們撇在外面,茶飯全無,再沒人出。八戒心中焦燥,埋怨唐僧道:「師父忒不會幹事,把話通說殺了。你好道還活著些腳兒,只含糊答應,哄他些齋飯吃了,今晚落得一宵快活。明日肯與不肯,在乎你我了。似這般關門不出,我們這清灰冷灶,一夜怎過?」

悟淨道:「二哥,你在他家做個女婿罷。」八戒道:「兄弟,不要栽人,從長計較。」行者道:「計較甚的?你要肯,便就教師父與那婦人做個親家,你就做個倒踏門的女婿。他家這等有財有寶,一定倒陪妝奩,整治個會親的筵席,我們也落些受用,你在此間還俗,卻不是兩全其美?」八戒道:「話便也是這等說,卻只是我脫俗又還俗,停妻再娶妻了。」沙僧道:「二哥原來是有嫂子的?」行者道:「你還不知他哩。他本是烏斯藏高老兒莊高太公的女婿,因被老孫降了。他也曾受菩薩戒行,沒及奈何,被我捉他來做個和尚,所以棄了前妻,投師父往西拜佛。他想是離別的久了,又想起那個勾當,卻才聽見這個勾當,斷然又有此心。獃子,你與這家子做了女婿罷,只是多拜老孫幾拜,我不檢舉你就罷了。」那獃子道:「胡說,胡說。大家都有此心,獨拿老豬出醜。常言道:『和尚是色中餓鬼。』那個不要如此?都這們扭扭捏捏的拿班兒,把好事都弄得裂了,致如今茶水不得見面,燈火也無人管。雖熬了這一夜,但那匹馬明日又要馱人,又要走路,再若餓上這一夜,只好剝皮罷了。你們坐著,等老豬去放放馬來。」

那獃子虎急急的解了韁繩,拉出馬去。行者道:「沙僧,你且陪師父坐這裡,等老孫跟他去,看他往那裡放馬。」三藏道:「悟空,你看便去看他,但只不可只管嘲他了。」行者道:「我曉得。」這大聖走出廳房,搖身一變,變作個紅蜻蜓兒,飛出前門,趕上八戒。

那獃子拉著馬,有草處且不教吃草,嗒嗒嗤嗤的趕著馬,轉到後門首去。只見那婦人帶了三個女子,在後門外閑立著,看菊花兒耍子。他娘女們看見八戒來時,三個女兒閃將進去。那婦人佇立門首道:「小長老那裡去?」這獃子丟了韁繩,上前唱個喏,道聲「娘,我來放馬的。」那婦人道:「你師父忒弄精細。在我家招了女婿,卻不強似做掛搭僧,往西蹡路?」八戒笑道:「他們是奉了唐王的旨意,不敢有違君命,不肯幹這件事。剛才都在前廳上栽我,我又有些奈上祝下的,只恐娘嫌我嘴長耳大。」那婦人道:「我也不嫌,只是家下無個家長,招一個倒也罷了,但恐小女兒有些兒嫌醜。」八戒道:「娘,你上覆令愛,不要這等揀漢。想我那唐僧,人才雖俊,其實不中用。我醜自醜,有幾句口號兒。」婦人道:「你怎的說麼?」八戒道:「我:
\begin{quote}
雖然人物醜,勤緊有些功。若言千頃地,不用使牛耕。只消一頓鈀,佈種及時生。沒雨能求雨,無風會喚風。房舍若嫌矮,起上二三層。地下不掃掃一掃,陰溝不通通一通。家長里短諸般事,踢天弄井我皆能。」
\end{quote}

」那婦人道:「既然幹得家事,你再去與你師父商量商量看,不尷尬,便招你罷。」八戒道:「不用商量,他又不是我的生身父母,幹與不幹,都在於我。」婦人道:「也罷,也罷,等我與小女說。」看他閃進去,撲的掩上後門。

八戒也不放馬,將馬拉向前來。怎知孫大聖已一一盡知,他轉翅飛來,現了本相,先見唐僧道:「師父,悟能牽馬來了。」長老道:「馬若不牽,恐怕撒歡走了。」行者笑將起來,把那婦人與八戒說的勾當,從頭說了一遍,三藏也似信不信的。

少時間,見獃子拉將馬來拴下。長老道:「你馬放了?」八戒道:「無甚好草,沒處放馬。」行者道:「沒處放馬,可有處牽馬麼?」獃子聞得此言,情知走了消息,也就垂頭扭頸,努嘴皺眉,半晌不言。又聽得呀的一聲,腰門開了,有兩對紅燈、一副提爐,香雲靄靄,環珮叮叮,那婦人帶著三個女兒,走將出來,叫真真、愛愛、憐憐,拜見那取經的人物。那女子排立廳中,朝上禮拜。果然也生得標致,但見他:
\begin{quote}
一個個蛾眉橫翠,粉面生春。妖嬈傾國色,窈窕動人心。花鈿顯現多嬌態,繡帶飄搖迥絕塵。半含笑處櫻桃綻,緩步行時蘭麝噴。滿頭珠翠,顫巍巍無數寶釵簪;遍體幽香,嬌滴滴有花金縷細。說甚麼楚娃美貌,西子嬌容。真個是九天仙女從天降,月裡嫦娥出廣寒。
\end{quote}

那三藏合掌低頭,孫大聖佯佯不睬,這沙僧轉背回身。你看那豬八戒眼不轉睛,淫心紊亂,色膽縱橫,扭捏出悄語,低聲道:「有勞仙子下降。娘,請姐姐們去耶。」那三個女子轉入屏風,將一對紗燈留下。婦人道:「四位長老可肯留心,著那個配我小女麼?」悟淨道:「我們已商議了,著那個姓豬的招贅門下。」八戒道:「兄弟,不要栽我,還從眾計較。」行者道:「還計較甚麼?你已在後門首說合的停停當當,娘都叫了,又有甚麼計較?師父做個男親家,這婆兒做個女親家,等老孫做個保親,沙僧做個媒人。也不必看通書,今朝是個天恩上吉日,你來拜了師父,進去做了女婿罷。」八戒道:「弄不成,弄不成,那裡好幹這個勾當?」行者道:「獃子,不要者囂,你那口裡娘也不知叫了多少,又是甚麼弄不成。快快的應承,帶攜我們吃些喜酒,也是好處。」他一隻手揪著八戒,一隻手扯住婦人道:「親家母,帶你女婿進去。」那獃子腳兒趄趄的要往那裡走。那婦人即喚童子:「展抹桌椅,鋪排晚齋,管待三位親家。我領姑夫房裡去也。」一壁廂又吩咐庖丁排筵設宴,明晨會親。那幾個童子又領命訖。他三眾吃了齋,急急鋪鋪,都在客座裡安歇不題。

卻說那八戒跟著丈母,行入裡面,一層層也不知多少房舍,磕磕撞撞,盡都是門檻絆腳。獃子道:「娘,慢些兒走,我這裡邊路生,你帶我帶兒。」那婦人道:「這都是倉房、庫房、碾房各房,還不曾到那廚房邊哩。」八戒道:「好大人家。」磕磕撞撞,轉彎抹角,又走了半會,才是內堂房屋。那婦人道:「女婿,你師兄說今朝是天恩上吉日,就教你招進來了。卻只是倉卒間,不曾請得個陰陽,拜堂撒帳。你可朝上拜八拜兒罷。」八戒道:「娘說得是。你請上坐,等我也拜幾拜,就當拜堂,就當謝親,兩當一兒,卻不省事?」他丈母笑道:「也罷,也罷。果然是個省事幹家的女婿。我坐著,你拜麼。」

咦!滿堂中銀燭輝煌,這獃子朝上禮拜。拜畢,道:「娘,你把那個姐姐配我哩?」他丈母道:「正是這些兒疑難:我要把大女兒配你,恐二女怪;要把二女配你,恐三女怪;欲將三女配你,又恐大女怪:所以終疑未定。」八戒道:「娘,既怕相爭,都與我罷,省得鬧鬧吵吵,亂了家法。」他丈母道:「豈有此理!你一人就占我三個女兒不成!」八戒道:「你看娘說的話,那個沒有三房四妾?就再多幾個,你女婿也笑納了。我幼年間,也曾學得個鏖戰之法,管情一個個伏侍得他歡喜。」那婦人道:「不好,不好。我這裡有一方手帕,你頂在頭上,遮了臉,撞個天婚:教我女兒從你跟前走過,你伸開手扯到那個,就把那個配了你罷。」獃子依言,接了手帕,頂在頭上。有詩為證。詩曰:
\begin{quote}
痴愚不識本原由,色劍傷身暗自休。
從來信有周公禮,今日新郎頂蓋頭。
\end{quote}

那獃子頂裹停當,道:「娘,請姐姐們出來麼。」他丈母叫:「真真、愛愛、憐憐,都來撞天婚,配與你女婿。」只聽得環珮響亮,蘭麝馨香,似有仙子來往。那獃子真個伸手去撈人,兩邊亂撲,左也撞不著,右也撞不著。來來往往,不知有多少女子行動,只是莫想撈著一個。東撲抱著柱科,西撲摸著板壁。兩頭跑暈了,立站不穩,只是打跌。前來蹬著門扇,後去擋著磚牆,磕磕撞撞,跌得嘴腫頭青,坐在地下。喘氣呼呼的道:「娘啊,你女兒這等乖滑得緊,撈不著一個,奈何,奈何?」

那婦人與他揭了蓋頭道:「女婿,不是我女兒乖滑,他們大家謙讓,不肯招你。」八戒道:「娘啊,既是他們不肯招我啊,你招了我罷。」那婦人道:「好女婿啞!這等沒大沒小的,連丈母也都要了?我這三個女兒心性最巧,他一人結了一個珍珠篏錦汗衫兒。你若穿得那個的,就教那個招你罷。」八戒道:「好,好,好,把三件兒都拿來我穿了看,若都穿得,就教都招了罷。」那婦人轉進房裡,止取出一件來,遞與八戒。那獃子脫下青錦布直裰,取過衫兒,就穿在身上。還未曾繫上帶子,撲的一蹻,跌倒在地。原來是幾條繩緊緊繃住。那獃子疼痛難禁,這些人早已不見了。

卻說三藏、行者、沙僧一覺睡醒,不覺的東方發白。忽睜睛擡頭觀看,那裡得那大廈高堂,也不是雕梁畫棟,一個個都睡在松柏林中。慌得那長老忙呼行者。沙僧道:「哥哥,罷了,罷了,我們遇著鬼了。」孫大聖心中明白,微微的笑道:「怎麼說?」長老道:「你看我們睡在那裡耶?」行者道:「這松林下落得快活。但不知那獃子在那裡受罪哩。」長老道:「那個受罪?」行者笑道:「昨日這家子娘女們,不知是那裡菩薩,在此顯化我等,想是半夜裡去了,只苦了豬八戒受罪。」三藏聞言,合掌頂禮。又只見那後邊古柏樹上,飄飄蕩蕩的掛著一張簡帖兒。沙僧急去取來與師父看時,卻是八句頌子云:
\begin{quote}
黎山老母不思凡,南海菩薩請下山。
普賢文殊皆是客,化成美女在林間。
聖僧有德還無俗,八戒無禪更有凡。
從此靜心須改過,若生怠慢路途難。
\end{quote}

那長老、行者、沙僧正然唱念此頌,只聽得林深處高聲叫道:「師父啊,繃殺我了,救我一救,下次再不敢了。」三藏道:「悟空,那叫喚的可是悟能麼?」沙僧道:「正是。」行者道:「兄弟,莫睬他,我們去罷。」三藏道:「那獃子雖是心性愚頑,卻只是一味懞直,倒也有些膂力,挑得行李。還看當日菩薩之念,救他隨我們去罷,料他以後再不敢了。」那沙和尚卻捲起鋪蓋,收拾了擔子;孫大聖解韁牽馬,引唐僧入林尋看。咦!這正是:
\begin{quote}
從正修持須謹慎,掃除愛慾自歸真。
\end{quote}

畢竟不知那獃子凶吉如何,且聽下回分解。
