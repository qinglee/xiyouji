
\chapter{平頂山功曹傳信 蓮花洞木母逢災}

話說唐僧復得了孫行者,師徒們一心同體,共詣西方。自寶象國救了宮主,承君臣送出城西。說不盡沿路饑餐渴飲,夜住曉行。卻又值三春景候,那時節:
\begin{quote}
輕風吹柳綠如絲,佳景最堪題。時催鳥語,暖烘花發,遍地芳菲。海棠庭院來雙燕,正是賞春時。紅塵紫陌,綺羅絃管,鬥草傳卮。
\end{quote}

師徒們正行賞間,又見一山擋路。唐僧道:「徒弟們仔細,前遇山高,恐有虎狼阻擋。」行者道:「師父,出家人莫說在家話。你記得那烏巢和尚的《心經》云『心無罣礙:無罣礙,方無恐怖,遠離顛倒夢想』之言?但只是:『掃除心上垢,洗淨耳邊塵。不受苦中苦,難為人上人。』你莫生憂慮,但有老孫,就是塌下天來,可保無事,怕甚麼虎狼?」長老勒回馬道:「我
\begin{quote}
當年奉旨出長安,只憶西來拜佛顏。
舍利國中金像彩,浮屠塔裡玉毫斑。
尋窮天下無名水,歷遍人間不到山。
逐逐煙波重疊疊,幾時能夠此身閑?」
\end{quote}

行者聞說,笑呵呵道:「師要身閑,有何難事?若功成之後,萬緣都罷,諸法皆空。那時節,自然而然,卻不是身閑也?」長老聞言,只得樂以忘憂。放轡催銀濁,兜韁趲玉龍。

師徒們上得山來,十分險峻,真個嵯峨。好山:
\begin{quote}
巍巍峻嶺,削削尖峰。灣環深澗下,孤峻陡崖邊。灣環深澗下,只聽得唿喇喇戲水蟒翻身;孤峻陡崖邊,但見那崒嵂嵂出林虎剪尾。往上看,巒頭突兀透青霄;回眼觀,壑下深沉鄰碧落。上高來,似梯似凳;下低行,如塹如坑。真個是古怪巔峰嶺,果然是連尖削壁崖。巔峰嶺上,採藥人尋思怕走;削壁崖前,打柴夫寸步難行。胡羊野馬亂攛梭,狡兔山牛如佈陣。山高蔽日遮星斗,時逢妖獸與蒼狼。草徑迷漫難進馬,怎得雷音見佛王?
\end{quote}

長老勒馬觀山,正在難行之處,只見那綠莎坡上,佇立著一個樵夫。你道他怎生打扮:
\begin{quote}
頭戴一頂老藍氈笠,身穿一領毛皂衲衣。老藍氈笠,遮煙蓋日果稀奇;毛皂衲衣,樂以忘憂真罕見。手持鋼斧快磨明,刀伐乾柴收束緊。檐頭春色,幽然四序融融;身外閑情,常是三星澹澹。到老只於隨分過,有何榮辱暫關山?
\end{quote}

那樵子:
\begin{quote}
正在坡前伐朽柴,忽逢長老自東來。
停柯住斧出林外,趨步將身上石崖。
\end{quote}

對長老厲聲高叫道:「那西進的長老,暫停片時,我有一言奉告:此山有一夥毒魔狠怪,專吃你東來西去的人哩。」長老聞言,魂飛魄散,戰兢兢坐不穩雕鞍,急回頭,忙呼徒弟道:「你聽那樵夫報道:『此山有毒魔狠怪。』誰敢去細問他一問?」行者道:「師父放心,等老孫去問他一個端的。」

好行者,拽開步,徑上山來,對樵子叫聲「大哥」,道個問訊。樵夫答禮道:「長老啊,你們有甚緣故來此?」行者道:「不瞞大哥說,我們是東土差來西天取經的。那馬上是我的師父,他有些膽小。適蒙見教,說有甚麼毒魔狠怪,故此我來奉問一聲:那魔是幾年之魔,怪是幾年之怪?還是個把勢,還是個雛兒?煩大哥老實說說,我好著山神、土地遞解他起身。」樵子聞言,仰天大笑道:「你原來是個風和尚。」行者道:「我不風啊,這是老實話。」樵子道:「你說是老實,便怎敢說把他遞解起身?」行者道:「你這等長他那威風,胡言亂語的攔路報信,莫不是與他有親?不親必鄰,不鄰必友。」樵子笑道:「你這個風潑和尚,忒沒道理。我倒是好意,特來報與你們,教你們走路時,早晚間防備,你倒轉賴在我身上。且莫說我不曉得妖魔出處,就曉得啊,你敢把他怎麼的遞解?解往何處?」行者道:「若是天魔,解與玉帝;若是土魔,解與土府。西方的歸佛,東方的歸聖;北方的解與真武,南方的解與火德。是蛟精解與海主,是鬼祟解與閻王。各有地頭方向。我老孫到處里人熟,發一張批文,把他連夜解著飛跑。」

那樵子止不住呵呵冷笑道:「你這個風潑和尚,想是在方上雲遊,學了些書符咒水的法術,只可驅邪縛鬼,還不曾撞見這等狠毒的怪哩。」行者道:「怎見他狠毒?」樵子道:「此山徑過有六百里遠近,名喚平頂山。山中有一洞,名喚蓮花洞。洞裡有兩個魔頭,他畫影圖形,要捉和尚;抄名訪姓,要吃唐僧。你若別處來的還好,但犯了一個『唐』字兒,莫想去得,去得。」行者道:「我們正是唐朝來的。」樵子道:「他正要吃你們哩。」行者道:「造化,造化。但不知他怎的樣吃哩?」樵子道:「你要他怎的吃?」行者道:「若是先吃頭,還好耍子;若是先吃腳,就難為了。」樵子道:「先吃頭怎麼說?先吃腳怎麼說?」行者道:「你還不曾經著哩。若是先吃頭,一口將他咬下,我已死了,憑他怎麼煎炒熬煮,我也不知疼痛。若是先吃腳,他啃了孤拐,嚼了腿亭,吃到腰截骨,我還急忙不死,卻不是零零碎碎受苦?此所以難為也。」樵子道:「和尚,他那裡有這許多工夫,只是把你拿住,綑在籠裡,囫圇蒸吃了。」行者笑道:「這個更好,更好。疼倒不忍疼,只是受些悶氣罷了。」樵子道:「和尚不要調嘴。那妖怪隨身有五件寶貝,神通極大極廣。就是擎天的玉柱,架海的金梁,若保得唐朝和尚去,也須要發發昏是。」行者道:「發幾個昏麼?」樵子道:「要發三四個昏是。」行者道:「不打緊,不打緊。我們一年,常發七八百個昏兒,這三四個昏兒易得發,發發兒就過去了。」

好大聖,全然無懼,一心只是要保唐僧。捽脫樵夫,拽步而轉,徑至山坡馬頭前道:「師父,沒甚大事。有便有個把妖精兒,只是這裡人膽小,放他在心上。有我哩,怕他怎的?走路,走路。」長老見說,只得放懷隨行。正行處,早不見了那樵夫。長老道:「那報信的樵子如何就不見了?」八戒道:「我們造化低,撞見日裡鬼了。」行者道:「想是他鑽進林子裡尋柴去了。等我看看來。」

好大聖,睜開火眼金睛,漫山越嶺的望處,卻無蹤跡。忽擡頭往雲端裡一看,看見是日值功曹,他就縱雲趕上,罵了幾聲「毛鬼」,道:「你怎麼有話不來直說,卻那般變化了,演樣老孫?」慌得那功曹施禮道:「大聖,報信來遲,勿罪,勿罪。那怪果然神通廣大,變化多端。只看你騰那乖巧,運動神機,仔細保你師父;假若怠慢了些兒,西天路莫想去得。」

行者聞言,把功曹叱退,切切在心,按雲頭,徑來山上。只見長老與八戒、沙僧簇擁前進。他卻暗想:「我若把功曹的言語實實告訴師父,師父他不濟事,必就哭了;假若不與他實說,夢著頭,帶著他走,常言道:『乍入蘆圩,不知深淺。』倘或被妖魔撈去,卻不又要老孫費心?且等我照顧八戒一照顧,先著他出頭與那怪打一仗看。若是打得過他,就算他一功;若是沒手段,被怪拿去,等老孫再去救他不遲,卻好顯我本事出名。」正自家計較,以心問心道:「只恐八戒躲懶,便不肯出頭,師父又有些護短。等老孫羈勒他羈勒。」

好大聖,你看他弄個虛頭,把眼揉了一揉,揉出些淚來。迎著師父,往前徑走。八戒看見,連忙叫:「沙和尚,歇下擔子,拿出行李來,我兩個分了罷。」沙僧道:「二哥,分怎的?」八戒道:「分了罷,你往流沙河還做妖怪,老豬往高老莊上盼盼渾家。把白馬賣了,買口棺木,與師父送老。大家散火,還往西天去哩?」長老在馬上聽見,道:「這個夯貨,正走路,怎麼又胡說了?」八戒道:「你兒子便胡說。你不看見孫行者那裡哭將來了?他是個鑽天入地,斧砍火燒,下油鍋都不怕的好漢;如今戴了個愁帽,淚汪汪的哭來,必是那山險峻,妖怪兇狠。似我們這樣軟弱的人兒,怎麼去得?」長老道:「你且休胡談,待我問他一聲,看是怎麼說話。」問道:「悟空,有甚話當面計較,你怎麼自家煩惱?這般樣個哭包臉,是虎諕我也?」行者道:「師父啊,剛才那個報信的是日值功曹,他說妖精兇狠,此處難行,果然的山高路峻,不能前進,改日再去罷。」長老聞言,恐惶悚懼,扯住他虎皮裙子道:「徒弟呀,我們三停路已走了停半,因何說退悔之言?」行者道:「我沒個不盡心的,但只恐魔多力弱,行勢孤單。『縱然是塊鐵,下爐能打得幾根釘?』」長老道:「徒弟啊,你也說得是,果然一個人也難。兵書云:『寡不可敵眾。』我這裡還有八戒、沙僧,都是徒弟,憑你調度使用,或為護將幫手,協力同心,掃清山徑,領我過山,卻不都還了正果?」

那行者這一場扭捏,只逗出長老這幾句話來。他搵了淚道:「師父啊,若要過得此山,須是豬八戒依得我兩件事兒,才有三分去得;假若不依我言,替不得我手,半分兒也莫想過去。」八戒道:「師兄,不去就散火罷。不要攀我。」長老道:「徒弟,且問你師兄,看他教你做甚麼?」獃子真個對行者說道:「哥哥,你教我做甚事?」行者道:「第一件是看師父,第二件是去巡山。」八戒道:「看師父是坐,巡山去是走。終不然教我坐一會又走,走一會又坐?兩處怎麼顧盼得來?」行者道:「不是教你兩件齊幹,只是領了一件便罷。」八戒又笑道:「這等也好計較。但不知看師父是怎樣,巡山是怎樣?你先與我講講,等我依個相應些兒的去幹罷。」行者道:「看師父啊,師父去出恭,你伺候;師父要走路,你扶持;師父要吃齋,你化齋。若他餓了些兒,你該打;黃了些兒臉皮,你該打;瘦了些兒形骸,你該打。」八戒慌了道:「這個難,難,難。伺候扶持,通不打緊;就是不離身馱著,也還容易;假若教我去鄉下化齋,他這西方路上,不識我是取經的和尚,只道是那山裡走出來的一個半壯不壯的健豬,夥上許多人,叉鈀掃帚,把老豬圍倒,拿家去宰了,醃著過年,這個卻不就遭瘟了?」行者道:「巡山去罷。」八戒道:「巡山便怎麼樣兒?」行者道:「就入此山,打聽有多少妖怪,是甚麼山,是甚麼洞,我們好過去。」八戒道:「這個小可,老豬去巡山罷。」那獃子就撒起衣裙,挺著釘鈀,雄糾糾,徑入深山;氣昂昂,奔上大路。

行者在傍,忍不住嘻嘻冷笑。長老罵道:「你這個潑猴!兄弟們全無愛憐之意,常懷嫉妒之心。你做出這樣獐智,巧言令色,撮弄他去甚麼巡山,卻又在這裡笑他。」行者道:「不是笑他,我這笑中有味。你看豬八戒這一去,決不巡山,也不敢見妖怪,不知往那裡去躲閃半會,捏出個謊來,哄我們也。」長老道:「你怎麼就曉得他?」行者道:「我估出他是這等,不信,等我跟他去看看,聽他一聽:一則幫副他手段降妖,二來看他可有個誠心拜佛?」長老道:「好,好,好,你卻莫去捉弄他。」行者應諾了,徑直趕上山坡,搖身一變,變作個蟭蟟蟲兒。其實變得輕巧,但見他:
\begin{quote}
翅薄舞風不用力,腰尖細小如針。穿蒲抹草過花陰,疾似流星還甚。眼睛明映映,聲氣渺瘖瘖。昆蟲之類惟他小,亭亭款款機深。幾番閑日歇幽林,一身渾不見,千眼莫能尋。
\end{quote}

嚶的一翅飛將去,趕上八戒,釘在他耳朵後面鬃根底下。那獃子只管走路,怎知道身上有人。行有七八里路,把釘鈀撇下,吊轉頭來,望著唐僧,指手畫腳的罵道:「你罷軟的老和尚,捉掐的弼馬溫,面弱的沙和尚,他都在那裡自在,捉弄我老豬來蹡路。大家取經,都要望成正果,偏是教我來巡甚麼山。哈哈哈,曉得有妖怪,躲著些兒走,還不夠一半,卻教我去尋他,這等晦氣哩。我往那裡睡覺去,睡一覺回去,含含糊糊的答應他,只說是巡了山,就了其帳也。」那獃子一時間僥幸,搴著鈀,又走,只見山凹裡一彎紅草坡。他一頭鑽得進去,使釘鈀撲個地鋪,轂轆的睡下,把腰伸了一伸,道聲:「快活。就是那弼馬溫,也不得像我這般自在。」

原來行者在他耳根後,句句兒聽著哩,忍不住飛將起來,又琢弄他一琢弄。又搖身一變,變作個啄木蟲兒。但見:
\begin{quote}
鐵嘴尖尖紅溜,翠翎艷艷光明。
一雙鋼爪利如釘。腹餒何妨林靜。
最愛枯槎朽爛,偏嫌老樹伶仃。
圜睛決尾性丟靈。辟剝之聲堪聽。
\end{quote}

這蟲鷖不大不小的,上秤稱,只有二三兩重。紅銅嘴,黑鐵腳。刷剌的一翅飛下來。那八戒丟倒頭,正睡著哩,被他照嘴唇上扢揸的一下。

那獃子慌得爬將起來,口裡亂嚷道:「有妖怪,有妖怪,把我戳了一槍去了,嘴上好不疼呀。」伸手摸摸,泱出血來了。他道:「蹭蹬啊,我又沒甚喜事,怎麼嘴上掛了紅耶?」他看著這血手,口裡絮絮叨叨的兩邊亂看,卻不見動靜。道:「無甚妖怪,怎麼戳我一槍麼?」忽擡頭往上看時,原來是個啄木蟲,在半空中飛哩。獃子咬牙罵道:「這個亡人,弼馬溫欺負我罷了,你也來欺負我。我曉得了。他一定不認我是個人,只把我嘴當一段黑朽枯爛的樹,內中生了蟲,尋蟲兒吃的,將我啄了這一下也。等我把嘴揣在懷裡睡罷。」那獃子轂轆的依然睡倒。行者又飛來,著耳根後又啄了一下。獃子慌得爬起來道:「這個亡人,卻打攪得我狠。想必這裡是他的窠巢,生蛋佈雛,怕我占了,故此這般打攪。罷罷罷,不睡他了。」搴著鈀,徑出紅草坡,找路又走。可不喜壞了孫行者,笑倒個美猴王。行者道:「這夯貨大睜著兩個眼,連自家人也認不得。」

好大聖,搖身又一變,還變做個蟭蟟蟲,釘在他耳朵後面,不離他身上。那獃子入深山,又行有四五里,只見山凹中有桌面大的四四方方三塊青石頭。獃子放下鈀,對石頭唱個大喏。行者暗笑道:「這獃子,石頭又不是人,又不會說話,又不會還禮的,唱他喏怎的,可不是個瞎帳?」原來那獃子把石頭當著唐僧、沙僧、行者三人,朝著他演習哩。他道:「我這回去,見了師父,若問有妖怪,就說有妖怪。他問甚麼山,我若說是泥捏的、土做的、錫打的、銅鑄的、麵蒸的、紙糊的、筆畫的,他們見說我獃哩,若講這話,一發說獃了。我只說是石頭山。他問甚麼洞,也只說是石頭洞。他問甚麼門,卻說是釘釘的鐵葉門。他問裡邊有多遠,只說入內有三層。十分再搜尋,問門上釘子多少,只說老豬心忙記不真。此間編造停當,哄那弼馬溫去。」那獃子捏合了,拖著鈀,徑回本路。

怎知行者在耳朵後,一一聽得明白。行者見他回來,即騰兩翅預先回去,現原身,見了師父。師父道:「悟空,你來了,悟能怎不見回?」行者笑道:「他在那裡編謊哩,就待來也。」長老道:「他兩個耳朵蓋著眼,愚拙之人也,他會編甚麼謊?又是你捏合甚麼鬼話賴他哩。」行者道:「師父,你只是這等護短。這是有對問的話。」把他那鑽在草裡睡覺,被啄木蟲叮醒,朝石頭唱喏,編造甚麼石頭山、石頭洞、鐵葉門、有妖精的話,預先說了。

說畢,不多時,那獃子走將來。又怕忘了那謊,低著頭,口裡溫習。被行者喝了一聲道:「獃子,念甚麼哩?」八戒掀起耳朵來看看道:「我到了地頭了?」那獃子上前跪倒。長老攙起道:「徒弟,辛苦啊。」八戒道:「正是。走路的人,爬山的人,第一辛苦了。」長老道:「可有妖怪麼?」八戒道:「有妖怪,有妖怪,一堆妖怪哩!」長老道:「怎麼打發你來?」八戒說:「他叫我做豬祖宗、豬外公,安排些粉湯素食,教我吃了一頓,說道擺旗鼓送我們過山哩。」行者道:「想是在草裡睡著了,說得是夢話。」獃子聞言,就嚇得矮了二寸道:「爺爺呀!我睡他怎麼曉得?」行者上前,一把揪住道:「你過來,等我問你。」獃子又慌了,戰戰兢兢的道:「問便罷了,揪扯怎的?」行者道:「是甚麼山?」八戒道:「是石頭山。」「甚麼洞?」道:「是石頭洞。」「甚麼門?」道:「是釘釘鐵葉門。」「裡邊有多遠?」道:「入內是三層。」行者道:「你不消說了,後半截我記得真,恐師父不信,我替你說了罷。」八戒道:「嘴臉,你又不曾去,你曉得那些兒,要替我說?」行者笑道:「『門上釘子有多少,只說老豬心忙記不真。』可是麼?」那獃子即慌忙跪倒。行者道:「朝著石頭唱喏,當做我三人,對他一問一答,可是麼?又說:『等我編得謊兒停當,哄那弼馬溫去。』可是麼?」那獃子連忙只是磕頭道:「師兄,我去巡山,你莫成跟我去聽的?」

行者罵道:「我把你個囊糠的夯貨!這般要緊的所在,教你去巡山,你卻去睡覺。不是啄木蟲叮你醒來,你還在那裡睡哩。及叮醒,又編這樣大謊,可不誤了大事?你快伸過孤拐來,打五棍記心。」八戒慌了道:「那個哭喪棒重,擦一擦兒皮塌,挽一挽兒筋傷;若打五下,就是死了。」行者道:「你怕打,卻怎麼扯謊?」八戒道:「哥哥呀,只是這一遭兒,以後再不敢了。」行者道:「一遭便打三棍。」八戒道:「爺爺呀!半棍兒也禁不得。」獃子沒計奈何,扯住師父道:「你替我說個方便兒。」長老道:「悟空說你編謊,我還不信,今果如此,其實該打。但如今過山少人使喚,悟空,你且饒他,待過了山。再打罷。」行者道:「古人云:『順父母言情,呼為大孝。』師父說不打,我就且饒你。你再去與他巡山,若再說謊誤事,我定一下也不饒你。」

那獃子只得爬起來又去。你看他奔上大路,疑心生暗鬼,步步只疑是行者變化了跟住他,故見一物,即疑是行者。走有七八里,見一隻老虎從山坡上跑過,他也不怕,舉著釘鈀道:「師兄來聽說謊的?這遭不編了。」又走處,那山風來得甚猛,呼的一聲,把顆枯木刮倒,滾至面前,他又跌腳搥胸的道:「哥啊,這是怎的起?一行說不敢編謊罷了,又變甚麼樹來打人?」又走向前,只見一個白頸老鴉,當頭喳喳的連叫幾聲,他又道:「哥哥,不羞,不羞。我說不編就不編了,只管又變著老鴉怎的?你來聽麼?」原來這一番行者卻不曾跟他去,他那裡卻自驚自怪,亂疑亂猜,故無往而不疑是行者隨他身也。獃子驚疑且不題。

卻說那山叫做平頂山,那洞叫做蓮花洞。洞裡兩妖:一喚金角大王,一喚銀角大王。金角正坐,對銀角說:「兄弟,我們多少時不巡山了?」銀角道:「有半個月了。」金角道:「兄弟,你今日與我去巡巡。」銀角道:「今日巡山怎的?」金角道:「你不知。近聞得東土唐朝差個御弟唐僧往西方拜佛,一行四眾,叫做孫行者、豬八戒、沙和尚,連馬五口。你看他在那處,與我把他拿來。」銀角道:「我們要吃人,那裡不撈幾個。這和尚到得那裡,讓他去罷。」金角道:「你不曉得。我當年出天界,嘗聞得人言:唐僧乃金蟬長老臨凡,十世修行的好人,一點元陽未泄,有人吃他肉,延壽長生哩。」銀角道:「若是吃了他肉就可以延壽長生,我們打甚麼坐,立甚麼功,煉甚麼龍與虎,配甚麼雌與雄?只該吃他去了。等我去拿他來。」金角道:「兄弟,你有些性急,且莫忙著。你若走出門,不管好歹,但是和尚就拿將來,假如不是唐僧,卻也不當人子。我記得他的模樣,曾將他師徒畫了一個影,圖了一個形。你可拿去,但遇著和尚,以此照驗照驗。」又將某人是某名字,一一說了。銀角得了圖像,知道姓名,即出洞,點起三十名小怪,便來山上巡邏。

卻說八戒運拙,正行處,可可的撞見群魔,當面擋住道:「那來的甚麼人?」獃子才擡起頭來,掀著耳朵,看見是些妖魔,他就慌了,心中暗道:「我若說是取經的和尚,他就撈了去。」只是說走路的。小妖回報道:「大王,是走路的。」那三十名小怪,中間有認得的,有不認得的。傍邊有聽著指點說話的道:「大王,這個和尚,像這圖中豬八戒模樣。」叫掛起影神圖來。八戒看見,大驚道:「怪道這些時沒精神哩,原來是他把我的影神傳將來也。」小妖用槍挑著,銀角用手指道:「這騎白馬的是唐僧,這毛臉的是孫行者。」八戒聽見道:「城隍,沒我便也罷了,豬頭三牲,清醮二十四分。」口裡嘮叨,只管許願。那怪又道:「這黑長的是沙和尚,這長嘴大耳的是豬八戒。」獃子聽見說他,慌得把個嘴揣在懷裡藏了。那怪叫:「和尚,伸出嘴來。」八戒道:「胎裡病,伸不出來。」那怪令小妖使鉤子鉤出來。八戒慌得把個嘴伸出道:「小家形罷了,這不是?你要看便就看,鉤怎的?」

那怪認得是八戒,掣出寶刀,上前就砍。這獃子舉釘鈀按住道:「我的兒,休無禮,看鈀!」那怪笑道:「這和尚是半路上出家的。」八戒道:「好兒子,有些靈性。你怎麼就曉得老爺是半路出家的?」那怪道:「你會使這鈀,一定是在人家園圃中築地,把他這鈀偷將來也。」八戒道:「我的兒,你那裡認得老爺這鈀,我不比那築地之鈀。這是:
\begin{quote}
巨齒鑄來如龍爪,滲金妝就似虎形。
若逢對敵寒風灑,但遇相持火燄生。
能替唐僧消障礙,西天路上捉妖精。
輪動煙霞遮日月,使起昏雲暗斗星。
築倒泰山老虎怕,掀翻大海老龍驚。
饒你這妖有手段,一鈀九個血窟窿。」
\end{quote}

那怪聞言,那裡肯讓。使七星劍,丟開解數,與八戒一往一來,在山中賭鬥有二十回合,不分勝負。八戒發起狠來,捨死的相迎。那怪見他捽耳朵,噴粘涎,舞釘鈀,口裡吆吆喝喝的,也盡有些悚懼,即回頭招呼小怪,一齊動手。若是一個打一個,其實還好。他見那些小妖齊上,慌了手腳,遮架不住,敗了陣,回頭就跑。原來是道路不平,未曾細看,忽被蓏蘿藤絆了個踉蹌。掙起來正走,又被一個小妖睡倒在地,扳著他腳跟,撲的又跌了個狗吃屎。被一群趕上按住,抓鬃毛,揪耳朵,扯著腳,拉著尾,扛扛擡擡,擒進洞去。咦!正是:
\begin{quote}
一身魔發難消滅,萬種災生不易除。
\end{quote}

畢竟不知豬八戒性命如何,且聽下回分解。
