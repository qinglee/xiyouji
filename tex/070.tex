
\chapter{妖魔寶放煙沙火 悟空計盜紫金鈴}

卻說那孫行者抖擻神威,持著鐵棒,踏祥光,起在空中,迎面喝道:「你是那裡來的邪魔?待往何方猖獗?」那怪物厲聲高叫道:「吾黨不是別人,乃麒麟山獬豸洞賽太歲大王爺爺部下先鋒,今奉大王令,到此取宮女二名,伏侍金聖娘娘。你是何人,敢來問我?」行者道:「吾乃齊天大聖孫悟空,因保東土唐僧西天拜佛,路過此國,知你這夥邪魔欺主,特展雄才,治國祛邪。正沒處尋你,卻來此送命。」那怪聞言,不知好歹,展長槍就刺行者;行者舉鐵棒劈面相迎。在半空裡這一場好殺:
\begin{quote}
棍是龍宮鎮海珍,槍乃人間轉煉鐵。凡兵怎敢比仙兵,擦著些兒神氣泄。大聖原來太乙仙,妖精本是邪魔孽。鬼祟焉能近正人,一正之時邪就滅。那個弄風播土諕皇王,這個踏霧騰雲遮日月。丟開架子賭輸贏,無能誰敢誇豪傑!還是齊天大聖能,乒乓一棍槍先折。
\end{quote}

那妖精被行者一鐵棒把根槍打做兩截,慌得顧性命,撥轉風頭,徑往西方敗走。

行者且不趕他,按下雲頭,來至避妖樓地穴之外,叫道:「師父,請同陛下出來,怪物已趕去矣。」那唐僧才扶著君王,同出穴外。見滿天清朗,更無妖邪之氣。那皇帝即至酒席前,自己拿壺把盞,滿斟金杯,奉與行者道:「神僧,權謝,權謝。」這行者接杯在手,還未回言,只聽得朝門外有官來報:「西門上火起了。」行者聞說,將金杯連酒望空一撇,噹的一聲響喨,那個金杯落地。君王著了忙,躬身施禮道:「神僧,恕罪,恕罪,是寡人不是了。禮當請上殿拜謝,只因有這方便酒在此,故就奉耳。神僧卻把杯子撇了,卻不是有見怪之意?」行者笑道:「不是這話,不是這話。」少頃間,又有官來報:「好雨呀,才西門上起火,被一場大雨,把火滅了。滿街上流水,盡都是酒氣。」行者又笑道:「陛下,你見我撇杯,疑有見怪之意,非也。那妖敗走西方,我不曾趕他,他就放起火來。這一杯酒,卻是我滅了妖火,救了西城裡外人家,豈有他意?」

國王更十分歡喜加敬。即請三藏四眾,同上寶殿,就有推位讓國之意。行者笑道:「陛下,才那妖精,他稱是賽太歲部下先鋒,來此取宮女的。他如今戰敗而回,定然報與那廝,那廝定要來與我相爭。我恐他一時興師帥眾,未免又驚傷百姓,恐諕陛下,欲去迎他一迎,就在那半空中擒了他,取回聖后。但不知向那方去?這裡到他那山洞有多少遠近?」國王道:「寡人曾差夜不收軍馬到那裡探聽聲息,往來要行五十餘日。坐落南方,約有三千餘里。」行者聞言,叫:「八戒、沙僧護持在此,老孫去來。」國王扯住道:「神僧且從容一日,待安排些乾糧烘炒,與你些盤纏銀兩,選一匹快馬,方才可去。」行者笑道:「陛下說得是巴山轉嶺步行之話。我老孫不瞞你說,似這三千里路,斟酒在鍾不冷,就打個往回。」國王道:「神僧,你不要怪我說,你這尊貌,卻像個猿猴一般,怎生有這等法力會走路也?」行者道:
\begin{quote}
「我身雖是猿猴數,自幼打開生死路。
遍訪明師把道傳,山前修煉無朝暮。
倚天為頂地為爐,兩般藥物團烏兔。
採取陰陽水火交,時間頓把玄關悟。
全仗天罡搬運功,也憑斗柄遷移步。
退爐進火最依時,抽鉛添汞相交顧。
攢簇五行造化生,合和四象分時度。
二氣歸於黃道間,三家會在金丹路。
悟通法律歸四肢,本來觔斗如神助。
一縱縱過太行山,一打打過凌雲渡。
何愁峻嶺幾千重,不怕長江百十數。
只因變化沒遮攔,一打十萬八千路!」
\end{quote}

那國王見說,又驚又喜,笑吟吟捧著一杯御酒遞與行者道:「神僧遠勞,進此一杯引意。」這大聖一心要去降妖,那裡有心吃酒,只叫:「且放下,等我去了回來再飲。」好行者,說聲去,唿哨一聲,寂然不見。那一國君臣,皆驚訝不題。

卻說行者將身一縱,早見一座高山阻住霧角。即按雲頭,立在那巔峰之上,仔細觀看,好山:
\begin{quote}
沖天占地,礙日生雲。沖天處,尖峰矗矗;占地處,遠脈迢迢。礙日的,乃嶺頭松鬱鬱,生雲的,乃崖下石磷磷。松鬱鬱,四時八節常青;石磷磷,萬載千年不改。林中每聽夜猿啼,澗內常聞妖蟒過。山禽聲咽咽,山獸吼呼呼。山獐山鹿,成雙作對紛紛走;山鴉山鵲,打陣攢群密密飛。山草山花看不盡,山桃山果映時新。雖然倚險不堪行,卻是妖仙隱逸處。
\end{quote}

這大聖看看不厭,正欲找尋洞口,只見那山凹裡烘烘火光飛出,霎時間,撲天紅焰,紅焰之中冒出一股惡煙,比火更毒。好煙!但見那:
\begin{quote}
火光迸萬點金燈,火焰飛千條紅虹。那煙不是灶筩煙,不是草木煙,煙卻有五色:青紅白黑黃。燻著南天門外柱,燎著靈霄殿上梁。燒得那窩中走獸連皮爛,林內飛禽羽盡光。但看這煙如此惡,怎入深山伏怪王?
\end{quote}

孫大聖正自恐懼,又見那山中迸出一道沙來。好沙,真個是遮天蔽日!你看:
\begin{quote}
紛紛絯絯遍天涯,鄧鄧渾渾大地遮。
細塵到處迷人目,粗灰滿谷滾芝麻。
採藥仙僮迷失伴,打柴樵子沒尋家。
手中就有明珠現,時間刮得眼生花。
\end{quote}

這行者只顧看玩,不覺沙灰飛入鼻內,癢斯斯的,打了兩個噴嚏。即回頭,伸手在岩下摸了兩個鵝卵石,塞住鼻子。搖身一變,變做一個攢火的鷂子,飛入煙火中間,驀了幾驀,卻就沒了沙灰,煙火也息了。急現本像下來,又看時,只聽得丁丁東東的一個銅鑼聲響。卻道:「我走錯了路也,這裡不是妖精住處。鑼聲似鋪兵之鑼,想是通國的大路,有鋪兵去下文書。且等老孫去問他一問。」

正走處,忽見似個小妖兒,擔著黃旗,背著文書,敲著鑼兒,急走如飛而來。行者笑道:「原來是這廝打鑼。他不知送的是甚麼書信?等我聽他一聽。」好大聖,搖身一變,變做個猛蟲兒,輕輕的飛在他書包之上。只聽得那妖精敲著鑼,緒緒聒聒的自念自誦道:「我家大王忒也心毒。三年前到朱紫國強奪了金聖皇后,一向無緣,未得沾身,只苦了要來的宮女頂缸。兩個來弄殺了,四個來也弄殺了。前年要了,去年又要,今年又要,如今還要。卻撞個對頭來了,那個要宮女的先鋒被個甚麼孫行者打敗了,不發宮女。我大王因此發怒,要與他國爭持,教我去下甚麼戰書。這一去,那國王不戰則可,戰必不利。我大王使煙火飛沙,那國王君臣百姓等,莫想一個得活。那時我等占了他的城池,大王稱帝,我等稱臣。雖然也有個大小官爵,只是天理難容也。」

行者聽了,暗喜道:「妖精也有存心好的。似他後邊這兩句話,說『天理難容』,卻不是個好的?但只說金聖皇后一向無緣,未得沾身,此話卻不解其意。等我問他一問。」嚶的一聲,一翅飛離了妖精,轉向前路,有十數里地,搖身一變,又變做一個道童:
\begin{quote}
頭挽雙丫髻,身穿百衲衣。
手敲魚鼓簡,口唱道情詞。
\end{quote}

轉山坡,迎著小妖,打個起手道:「長官,那裡去?送的是甚麼公文?」那妖物就像認得他的一般,住了鑼槌,笑嘻嘻的還禮道:「我大王差我到朱紫國下戰書的。」行者接口問道:「朱紫國那話兒,可曾與大王配合哩?」小妖道:「自前年攝得來,當時就有一個神仙,送一件五彩仙衣與金聖宮妝新。他自穿了那衣,就渾身上下都生了針刺,我大王摸也不敢摸他一摸。但挽著些兒,手心就痛,不知是甚緣故。自始至今,尚未沾身。早間差先鋒去要宮女伏侍,被一個甚麼孫行者戰敗了。大王奮怒,所以教我去下戰書,明日與他交戰也。」行者道:「怎的大王卻著惱啊?」小妖道:「正在那裡著惱哩。你去與他唱個道情詞兒,解解悶也。」

好行者,拱手抽身就走。那妖依舊敲鑼前行。行者就行起兇來,掣出棒,復轉身,望小妖腦後一下,可憐,就打得頭爛血流漿迸出,皮開頸折命傾之。收了棍子,卻又自悔道:「急了些兒,不曾問他叫做甚麼名字。罷了。」卻去取下他的戰書,藏於袖內。將他黃旗、銅鑼藏在路旁草裡。因扯著腳要往澗下捽時,只聽噹的一聲,腰間露出一個鑲金的牙牌。牌上有字,寫道:「心腹小校一名,有來有去。五短身材,扢撻臉,無鬚。長川懸掛,無牌即假。」行者笑道:「這廝名字叫做有來有去,這一棍子,打得有去無來!」將牙牌解下,帶在腰間。欲要捽下屍骸,卻又思量起煙火之毒。且不敢尋他洞府,即將棍子舉起,著小妖胸前搗了一下,挑在空中,徑回本國,且當報一個頭功。你看他自思自念,唿哨一聲,到了國界。

那八戒在金鑾殿前正護持著王、師,忽回頭看見行者半空中將個妖精挑來,他卻怨道:「噯!不打緊的買賣。早知老豬去拿來,卻不算我一功?」說未畢,行者按落雲頭,將妖精捽在階下。八戒跑上去,就築了一鈀道:「此是老豬之功。」行者道:「是你甚功?」八戒道:「莫賴我,我有證見,你不看一鈀築了九個眼子哩。」行者道:「你看看可有頭沒頭。」八戒笑道:「原來是沒頭的,我道如何築他也不動動兒?」行者道:「師父在那裡?」八戒道:「在殿裡與王敘話哩。」行者道:「你且去請他出來。」八戒急上殿,點點頭。三藏即便起身下殿,迎著行者。行者將一封戰書揣在三藏袖裡道:「師父收下,且莫與國王看見。」

說不了,那國王也下殿,迎著行者道:「神僧長老來了?拿妖之事如何?」行者用手指道:「那階下不是妖精?被老孫打殺了也。」國王見了道:「是便是個妖屍,卻不是賽太歲。賽太歲,寡人親見他兩次,身長丈八,膊闊五停;面似金光,聲如霹靂。那裡是這般鄙矮?」行者笑道:「陛下認得,果然不是。這是一個報事的小妖,撞見老孫,卻先打死,挑回來報功。」國王大喜道:「好好好,該算頭功。寡人這裡常差人去打探,更不曾得個的實。似神僧一出,就捉了一個回來,真神通也。」叫:「看暖酒來,與長老賀功。」

行者道:「吃酒還是小事。我問陛下:金聖宮別時,可曾留下個甚麼表記?你與我些兒。」那國王聽說「表記」二字,卻似刀劍剜心,忍不住失聲淚下,說道:
\begin{quote}
「當年佳節慶朱明,太歲兇妖發喊聲。
強奪御妻為壓寨,寡人獻出為蒼生。
更無會話並離話,那有長亭共短亭?
表記香囊全沒影,至今撇我苦伶仃。」
\end{quote}

行者道:「陛下在邇,何以惱為?那娘娘既無表記,他在宮內可有甚麼心愛之物?與我一件也罷。」國王道:「你要怎的?」行者道:「那妖王實有神通,我見他放煙、放火、放沙,果是難收。縱收了,又恐娘娘見我面生,不肯跟我回國。須是得他平日心愛之物一件,他方信我,我好帶他回來。為此故要帶去。」國王道:「昭陽宮裡梳妝閣上,有一雙黃金寶串,原是金聖宮手上帶戴。只因那日端午要縛五色彩線,故此褪下,不曾戴上。此乃是他心愛之物,如今現收在減妝盒裡。寡人見他遭此離別,更不忍見;一見即如見他玉容,病又重幾分也。」行者道:「且休題這話,且將金串取來。如捨得,都與我拿去;如不捨,只拿一隻去也。」國王遂命玉聖宮取出。取出即遞與國王。國王見了,叫了幾聲「知疼著熱的娘娘」,遂遞與行者。行者接了,套在肐膊上。

好大聖,不吃得功酒,且駕觔斗雲,唿哨一聲,又至麒麟山上。無心玩景,徑尋洞府而去。正行時,只聽得人語喧嚷,即佇立凝睛觀看。原來那獬豸洞口把門的大小頭目,約摸有五百名,在那裡:
\begin{quote}
森森羅列,密密挨排。森森羅列執干戈,映日光明;密密挨排展旌旗,迎風飄閃。虎將熊師能變化,豹頭彪帥弄精神。蒼狼多猛烈,獺象更驍雄。狡兔乖獐掄劍戟,長蛇大蟒挎刀弓。猩猩能解人言語,引陣安營識汛風。
\end{quote}

行者見了,不敢前進,抽身徑轉舊路。你道他抽身怎麼?不是怕他。他卻至那打死小妖之處,尋出黃旗、銅鑼,迎風捏訣,想象騰那,即搖身一變,變做那有來有去的模樣,乒乓敲著鑼,大踏步,一直前來,徑撞至獬豸洞。正欲看看洞景,只聞得猩猩出語道:「有來有去,你回來了?」行者只得答應道:「來了。」猩猩道:「快走,大王爺爺正在剝皮亭上等你回話哩。」

行者聞言,拽開步,敲著鑼,徑入前門裡看處,原來是懸崖削壁,石屋虛堂,左右有琪花瑤草,前後多古柏喬松。不覺又至二門之內,忽擡頭,見一座八窗明亮的亭子,亭子中間有一張戧金的交椅,椅子上端坐著一個魔王,真個生得惡像。但見他:
\begin{quote}
晃晃霞光生頂上,威威殺氣迸胸前。
口外獠牙排利刃,鬢邊焦髮放紅煙。
嘴上髭鬚如插箭,遍體昂毛似疊氈。
眼突銅鈴欺太歲,手持鐵杵若摩天。
\end{quote}

行者見了,公然傲慢那妖精,更不循一些兒禮法:調轉臉,朝著外,只管敲鑼。妖王問道:「你來了?」行者不答。又問:「有來有去,你來了?」也不答應。妖王上前扯住道:「你怎麼到了家還篩鑼,問之又不答,何也?」行者把鑼往地下一摜道:「甚麼『何也』、『何也』?我說我不去,你卻教我去。行到那廂,只見無數的人馬列成陣勢,見了我,就都叫:『拿妖精,拿妖精。』把我推推扯扯,拽拽扛扛,拿進城去。見了那國王,國王便教:『斬了。』幸虧那兩班謀士道:『兩家相爭,不斬來使。』把我饒了。收了戰書,又押出城外,對軍前打了三十順腿,放我來回話。他那裡不久就要來此與你交戰哩。」妖王道:「這等說,是你吃虧了,怪不得問你更不言語。」行者道:「卻不是怎的?只為護疼,所以不曾答應。」妖王道:「那裡有多少人馬?」行者道:「我也諕昏了,又吃他打怕了,那裡會查他人馬數目?只見那裡森森兵器擺列著:
\begin{quote}
弓箭刀槍甲與衣,干戈劍戟並纓旗。剽槍月鏟兜鍪鎧,大斧團牌鐵蒺藜。長悶棍,短窩槌,鋼叉銃鉋及頭盔。打扮得䩺鞋護頂並胖襖,簡鞭神彈與銅鎚。」
\end{quote}

那王聽了笑道:「不打緊,不打緊。似這般兵器,一火皆空。你且去報與金聖娘娘得知,教他莫惱。今早他聽見我發狠,要去戰鬥,他就眼淚汪汪的不乾。你如今去說那裡人馬驍勇,必然勝我,且寬他一時之心。」

行者聞言,十分歡喜道:「正中老孫之意。」你看他偏是路熟,轉過角門,穿過廳堂。那裡邊盡都是高堂大廈,更不似前邊的模樣。直到後邊宮裡,遠見彩門壯麗,乃是金聖娘娘住處。直入裡面看時,有兩班妖狐、妖鹿,一個個都妝成美女之形,侍立左右。正中間坐著那個娘娘,手托著香腮,雙眸滴淚。果然是:
\begin{quote}
玉容嬌嫩,美貌妖嬈。懶梳妝,散鬢堆鴉;怕打扮,釵環不戴。面無粉,冷淡了胭脂;髮無油,蓬鬆了雲鬢。努櫻唇,緊咬銀牙;皺蛾眉,淚淹星眼。一片心,只憶著朱紫君王;一時間,恨不離天羅地網。誠然是:自古紅顏多薄命,懨懨無語對東風。
\end{quote}

行者上前打了個問訊道:「接喏。」那娘娘道:「這潑村怪,十分無狀。想我在那朱紫國中,與王同享榮華之時,那太師、宰相見了,就俯伏塵埃,不敢仰視。這野怪怎麼叫聲『接喏』?是那裡來的這般村潑?」眾侍婢上前道:「太太息怒。他是大王爺爺心腹的小校,喚名有來有去。今早差下戰書的是他。」娘娘聽說,忍怒問曰:「你下戰書,可曾到朱紫國界?」行者道:「我持書直至城裡,到於金鑾殿,面見君王,已討回音來也。」娘娘道:「你面君,君有何言?」行者道:「那君王敵戰之言,與排兵布陣之事,才與大王說了。只是那君王有思想娘娘之意,有一句合心的話兒,特來上稟。奈何左右人眾,不是說處。」

娘娘聞言,喝退兩班狐、鹿。行者掩上宮門,把臉一抹,現了本像,對娘娘道:「你休怕我。我是東土大唐差往大西天天竺國雷音寺見佛求經的和尚。我師父是唐王御弟唐三藏。我是他大徒弟孫悟空。因過你國倒換關文,見你君臣出榜招醫,是我大施三折之肱,把他相思之病治好了,排宴謝我。飲酒之間,說出你被妖攝來。我會降龍伏虎,特請我來捉怪,救你回國。那戰敗先鋒是我,打死小妖也是我。我見他門外兇狂,是我變作有來有去模樣,捨身到此,與你通信。」那娘娘聽說,沉吟不語。行者取出寶串,雙手奉上道:「你若不信,看此物何來?」娘娘一見垂淚,下座拜謝道:「長老,你果是救得我回朝,沒齒不忘大恩。」

行者道:「我且問你,他那放火、放煙、放沙的,是件甚麼寶貝?」娘娘道:「那裡是甚寶貝,乃是三個金鈴。他將頭一個幌一幌,有三百丈火光燒人;第二個幌一幌,有三百丈煙光燻人;第三個幌一幌,有三百丈黃沙迷人。煙火還不打緊,只是黃沙最毒,若鑽入人鼻孔,就傷了性命。」行者道:「利害,利害。我曾經著,打了兩個嚏噴。卻不知他的鈴兒放在何處?」娘娘道:「他那肯放下?只是帶在腰間,行住坐臥,再不離身。」行者道:「你若有意於朱紫國,還要相會國王,把那煩惱憂愁,都且權解,使出個風流喜悅之容,與他敘個夫妻之情,教他把鈴兒與你收貯。待我取便偷了,降了這妖怪,那時節,好帶你回去,重諧鸞鳳,共享安寧也。」那娘娘依言。

這行者還變作心腹小校,開了宮門,喚進左右侍婢。娘娘叫:「有來有去,快往前亭請你大王來,與他說話。」好行者,應了一聲,即至剝皮亭,對妖精道:「大王,聖宮娘娘有請。」妖王歡喜道:「娘娘常時只罵,怎麼今日有請?」行者道:「那娘娘問朱紫國王之事,是我說:『他不要你了,他國中另扶了皇后。』娘娘聽說,故此沒了想頭,方才命我來奉請。」妖王大喜道:「你卻中用。待我剿除了他國,封你為個隨朝的太宰。」

行者順口謝恩,疾與妖王來至後宮門首。那娘娘歡容迎接,就去用手相攙。那妖王喏喏而退道:「不敢,不敢。多承娘娘下愛,我怕手痛,不敢相傍。」娘娘道:「大王請坐,我與你說。」妖王道:「有話但說不妨。」娘娘道:「我蒙大王辱愛,今已三年,未得共枕同衾。也是前世之緣,做了這場夫妻。誰知大王有外我之意,不以夫妻相待。我想著當時在朱紫國為后,外邦凡有進貢之寶,君看畢,一定與后收之。你這裡更無甚麼寶貝,左右穿的是貂裘,吃的是血食,那曾見綾錦金珠?只一味鋪皮蓋毯。或者就有些寶貝,你因外我,也不教我看見,也不與我收著。且如聞得你有三個鈴鐺,想就是件寶貝,你怎麼走也帶著,坐也帶著?你就拿與我收著,待你用時取出,未為不可。此也是做夫妻一場,也有個心腹相託之意。如此不相託付,非外我而何?」妖王大笑陪禮道:「娘娘怪得是,怪得是。寶貝在此,今日就當付你收之。」便即揭衣取寶。行者在旁,眼不轉睛,看著那怪揭起兩三層衣服,貼身帶著三個鈴兒。他解下來,將些綿花塞了口兒,把一塊豹皮作一個包袱兒包了,遞與娘娘道:「物雖微賤,卻要用心收藏,切不可搖幌著他。」娘娘接過手道:「我曉得。安在這妝臺之上,無人搖動。」叫:「小的們,安排酒來,我與大王交歡會喜,飲幾杯兒。」眾侍婢聞言,即鋪排果菜,擺上些獐鹿兔之肉,將椰子酒斟來奉上。那娘娘做出妖嬈之態,哄著精靈。

孫行者在旁取事,但挨挨摸摸,行近妝臺,把三個金鈴輕輕拿過,慢慢移步,溜出宮門,徑離洞府。到了剝皮亭前無人處,展開豹皮幅子看時,中間一個有茶鍾大,兩頭兩個有拳頭大。他不知利害,就把綿花扯了。只聞得噹的一聲響喨,骨都都的迸出煙、火、黃沙,急收不住,滿亭中烘烘火起。諕得那把門精怪一擁撞入後宮,驚動了妖王,慌忙教:「去救火,救火。」出來看時,原來是有來有去拿了金鈴兒哩。妖王上前喝道:「好賤奴,怎麼偷了我的金鈴寶貝,在此胡弄?」叫:「拿來,拿來。」那門前虎將、熊師、豹頭、彪帥、獺象、蒼狼、乖獐、狡兔、長蛇、大蟒、猩猩,帥眾妖一齊攢簇。

那行者慌了手腳,丟了金鈴,現出本像,掣出金箍如意棒,撒開解數,往前亂打。那妖王收了寶貝,傳號令,教:「關了前門。」眾妖聽了,關門的關門,打仗的打仗。那行者難得脫身,收了棒,搖身一變,變作個痴蒼蠅兒,釘在那無火石壁上。眾妖尋不見,報道:「大王,走了賊也,走了賊也。」妖王問:「可曾自門裡走出去?」眾妖都說:「前門緊鎖牢拴在此,不曾走出。」妖王只說:「仔細搜尋。」有的取水潑火,有的仔細搜尋,更無蹤跡。妖王怒道:「是個甚麼賊子?好大膽,變作有來有去的模樣,進來見我回話,又跟在身邊,乘機盜我寶貝。早是不曾拿將出去。若拿出山頭,見了天風,怎生是好?」虎將上前道:「大王的洪福齊天,我等的氣數不盡,故此知覺了。」熊師上前道:「大王,這賊不是別人,定是那戰敗先鋒的那個孫悟空。想必路上遇著有來有去,傷了性命,奪了黃旗、銅鑼、牙牌,變作他的模樣,到此欺騙了大王也。」妖王道:「正是,正是,見得有理。」叫:「小的們,仔細搜求防避,切莫開門放出走了。」這才是個有分教:
\begin{quote}
弄巧翻成拙,作耍卻為真。
\end{quote}

畢竟不知孫行者怎麼脫得妖門,且聽下回分解。
