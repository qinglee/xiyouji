
\chapter{荊棘嶺悟能努力 木仙庵三藏談詩}

話表祭賽國王謝了唐三藏師徒獲寶擒怪之恩,所贈金玉,分毫不受。卻命當駕官照依四位常穿的衣服各做兩套,鞋襪各做兩雙,絛環各做兩條,外備乾糧烘炒,倒換了通關文牒,大排鑾駕,並文武多官、滿城百姓、伏龍寺僧人,大吹大打,送四眾出城。約有二十里,先辭了國王。眾人又送二十里辭回。伏龍寺僧人送有五六十里不回:有的要同上西天,有的要修行伏侍。行者見都不肯回去,遂弄個手段,把毫毛拔了三四十根,吹口仙氣,叫:「變!」都變作斑斕猛虎,攔住前路,哮吼踴躍。眾僧方懼,不敢前進。大聖才引師父策馬而去,少時間去得遠了。眾僧人放聲大哭,都喊:「有恩有義的老爺!我等無緣,不肯度我們也。」

且不說眾僧啼哭。卻說師徒四眾走上大路,卻才收回毫毛,一直西去。正是時序易遷,又早冬殘春至,不暖不寒,正好逍遙行路。忽見一條長嶺,嶺頂上是路。三藏勒馬觀看,那嶺上荊棘丫叉,薜蘿牽繞,雖是有道路的痕跡,左右卻都是荊刺棘針。唐僧叫:「徒弟,這路怎生走得?」行者道:「怎麼走不得?」又道:「徒弟啊,路痕在下,荊棘在上,只除是蛇蟲伏地而遊,方可去了;若你們走,腰也難伸,教我如何乘馬?」八戒道:「不打緊,等我使出鈀柴手來,把釘鈀分開荊棘,莫說騎馬,就擡轎也包你過去。」三藏道:「你雖有力,長遠難熬,卻不知有多少遠近,怎生費得這許多精神?」行者道:「不須商量,等我去看看。」將身一縱,跳在半空看時,一望無際。真個是:
\begin{quote}
匝地遠天,凝煙帶雨。夾道柔茵亂,漫山翠蓋張。密密搓搓初發葉,攀攀扯扯正芬芳。遙望不知何所盡,近觀一似綠雲茫。蒙蒙茸茸,鬱鬱蒼蒼。風聲飄索索,日影映煌煌。那中間有松有柏還有竹,多梅多柳更多桑。薜蘿纏古樹,藤葛繞垂楊。盤團似架,聯絡如床。有處花開真佈錦,無端卉發遠生香。為人誰不遭荊棘,那見西方荊棘長?
\end{quote}

行者看夠多時,將雲頭按下道:「師父,這去處遠哩。」三藏問:「有多少遠?」行者道:「一望無際,似有千里之遙。」三藏大驚道:「怎生是好?」沙僧笑道:「師父莫愁,我們也學燒荒的,放上一把火,燒絕了荊棘過去。」八戒道:「莫亂談。燒荒的須在十來月,草衰木枯,方好引火。如今正是蕃盛之時,怎麼燒得?」行者道:「就是燒得,也怕人了。」三藏道:「這般怎生得度?」八戒笑道:「要得度,還依我。」

好獃子,捻個訣,念個咒語,把腰躬一躬,叫:「長!」就長了有二十丈高下的身軀。把釘鈀幌一幌,教:「變!」就變了有三十丈長短的鈀柄。拽開步,雙手使鈀,將荊棘左右摟開:「請師父跟我來也。」三藏見了甚喜,即策馬緊隨後面;沙僧挑著行李;行者也使鐵棒撥開。這一日未曾住手,行有百十里。將次天晚,見有一塊空闊之處。當路上有一通石碣,上有三個大字,乃「荊棘嶺」;下有兩行十四個小字,乃「荊棘蓬攀八百里,古來有路少人行」。八戒見了,笑道:「等我老豬與他添上兩句:『自今八戒能開破,直透西方路盡平。』」三藏欣然下馬道:「徒弟啊,累了你也。我們就在此住過了今宵,待明日天光再走。」八戒道:「師父莫住,趁此天色晴明,我等有興,連夜摟開路走他娘。」那長老只得相從。

八戒上前努力,師徒們人不住手,馬不停蹄,又行了一日一夜,卻又天色晚矣。那前面蓬蓬結結,又聞得風敲竹韻,颯颯松聲。卻好又有一段空地,中間乃是一座古廟。廟門之外,有松柏凝青,桃梅鬥麗。三藏下馬,與三個徒弟同看。只見:
\begin{quote}
巖前古廟枕寒流,落目荒煙鎖廢丘。
白鶴叢中深歲月,綠蕪臺下自春秋。
竹搖青珮疑聞語,鳥弄餘音似訴愁。
雞犬不通人跡少,閑花野蔓遶牆頭。
\end{quote}

行者看了道:「此地少吉多凶,不宜久坐。」沙僧道:「師兄差疑了。似這杳無人煙之處,又無個怪獸妖禽,怕他怎的?」說不了,忽見一陣陰風,廟門後轉出一個老者,頭戴角巾,身穿淡服,手持拐杖,足踏芒鞋。後跟著一個青臉獠牙、紅鬚赤身鬼使,頭頂著一盤麵餅。跪下道:「大聖,小神乃荊棘嶺土地。知大聖到此,無以接待,特備蒸餅一盤,奉上老師父,各請一餐。此地八百里,更無人家,聊吃些兒充饑。」八戒歡喜,上前舒手,就欲取餅。不知行者端詳已久,喝一聲:「且住,這廝不是好人。休得無禮,你是甚麼土地,來誑老孫?看棍。」那老者見他打來,將身一轉,化作一陣陰風,呼的一聲,把個長老攝將起去,飄飄蕩蕩,不知攝去何所。慌得那大聖沒跟尋處,八戒、沙僧俱相顧失色,白馬亦只自驚吟。三兄弟連馬四口,恍恍忽忽,遠望高張,並無一毫下落,前後找尋不題。

卻說那老者同鬼使,把長老擡到一座煙霞石屋之前,輕輕放下,與他攜手相攙道:「聖僧休怕。我等不是歹人,乃荊棘嶺十八公是也。因風清月霽之宵,特請你來會友談詩,消遣情懷故耳。」那長老卻才定性,睜眼仔細觀看。真個是:
\begin{quote}
漠漠煙雲去所,清清仙境人家。
正好潔身修煉,堪宜種竹栽花。
每見翠巖來鶴,時聞青沼鳴蛙。
更賽天臺丹灶,仍期華岳明霞。
說甚耕雲釣月,此間隱逸堪誇。
坐久幽懷如海,朦朧月上窗紗。
\end{quote}

三藏正自點看,漸覺月明星朗,只聽得人語相談。都道:「十八公請得聖僧來也。」長老擡頭觀看,乃是三個老者:前一個霜姿丰采,第二個綠鬢婆娑,第三個虛心黛色。各各面貌、衣服俱不相同,都來與三藏作禮。長老還了禮,道:「弟子有何德行,敢勞列位仙翁下愛?」十八公笑道:「一向聞知聖僧有道,等待多時,今幸一見。如果不吝珠玉,寬坐敘懷,足見禪機真派。」三藏躬身道:「敢問仙翁尊號?」十八公道:「霜姿者號孤直公,綠鬢者號凌空子,虛心者號拂雲叟,老拙號曰勁節。」三藏道:「四翁尊壽幾何?」孤直公道:
\begin{quote}
「我歲今經千歲古,撐天葉茂四時春。
香枝鬱鬱龍蛇狀,碎影重重霜雪身。
自幼堅剛能耐老,從今正直喜修真。
烏棲鳳宿非凡輩,落落森森遠俗塵。」
\end{quote}

凌空子笑道:
\begin{quote}
「吾年千載傲風霜,高幹靈枝力自剛。
夜靜有聲如雨滴,秋晴蔭影似雲張。
盤根已得長生訣,受命尤宜不老方。
留鶴化龍非俗輩,蒼蒼爽爽近仙鄉。」
\end{quote}

拂雲叟笑道:
\begin{quote}
「歲寒虛度有千秋,老景瀟然清更幽。
不雜囂塵終冷淡,飽經霜雪自風流。
七賢作侶同談道,六逸為朋共唱酬。
戛玉敲金非瑣瑣,天然情性與仙遊。」
\end{quote}

勁節十八公笑道:
\begin{quote}
「我亦千年約有餘,蒼然貞秀自如如。
堪憐雨露生成力,借得乾坤造化機。
萬壑風煙惟我盛,四時灑落讓吾疏。
蓋張翠影留仙客,博弈調琴講道書。」
\end{quote}

三藏稱謝道:「四位仙翁,俱享高壽,但勁節翁又千歲餘矣。高年得道,丰采清奇,得非漢時之『四皓』乎?」四老道:「承過獎,承過獎。吾等非四皓,乃深山之『四操』也。敢問聖僧,妙齡幾何?」三藏合掌躬身答曰:
\begin{quote}
「四十年前出母胎,未產之時命已災。
逃生落水隨波滾,幸遇金山脫本骸。
養性看經無懈怠,誠心拜佛敢俄捱。
今蒙皇上差西去,路遇仙翁下愛來。」
\end{quote}

四老俱稱道:「聖僧自出娘胎,即從佛教,果然是從小修行,真中正有道之上僧也。我等幸接臺顏,敢求大教。望以禪法指教一二,足慰生平。」長老聞言,慨然不懼,即對眾言曰:禪者,靜也;法者,度也。靜中之度,非悟不成。悟者,洗心滌慮,脫俗離塵是也。夫人身難得,中土難生,正法難遇:全此三者,幸莫大焉。至德妙道,渺漠希夷,六根六識,遂可掃除。菩提者,不死不生,無餘無欠,空色包羅,聖凡俱遣。訪真了元始鉗鎚,悟實了牟尼手段。發揮象罔,踏碎涅槃。必須覺中覺了悟中悟,一點靈光全保護。放開烈焰照婆娑,法界縱橫獨顯露。至幽微,更守固,玄關口說誰人度?我本元修大覺禪,有緣有志方記悟。」四老側耳受了,無邊喜悅。一個個稽首皈依,躬身拜謝道:「聖僧乃禪機之悟本也。」

拂雲叟道:「禪雖靜,法雖度,須要性定心誠。縱為大覺真仙,終坐無生之道。我等之玄,又大不同。」三藏云:「道乃非常,體用合一,如何不同?」拂雲叟笑云:「我等生來堅實,體用比爾不同。感天地以生身,蒙雨露而滋色。笑傲風霜,消磨日月。一葉不凋,千枝節操。似這話不叩沖虛,你執持梵語。道也者,本安中國,反來求證西方,空費了草鞋,不知尋個甚麼?石獅子剜了心肝,野狐涎灌徹骨髓。忘本參禪,妄求佛果,都似我荊棘嶺葛藤謎語,蘿蓏渾言。此般君子,怎生接引?這等規模,如何印授?必須要檢點見前面目,靜中自有生涯。沒底竹籃汲水,無根鐵樹生花。靈寶峰頭牢著腳,歸來雅會上龍華。」三藏聞言,叩頭拜謝。十八公用手攙扶,孤直公將身扯起,凌空子打個哈哈道:「拂雲之言,分明漏泄。聖僧請起,不可盡信。我等趁此月明,原不為講論修持,且自吟哦逍遙,放蕩襟懷也。」拂雲叟笑指石屋道:「若要吟哦,且入小庵一茶,何如?」

長老真個欠身,向石屋前觀看。門上有三個大字,乃「木仙庵」。遂此同入,又敘了坐次。忽見那赤身鬼使,捧一盤茯苓膏,將五盞香湯奉上。四老請唐僧先吃,三藏驚疑,不敢便吃。那四老一齊享用,三藏卻才吃了兩塊。各飲香湯收去。三藏留心偷看,只見那裡玲瓏光彩,如月下一般:
\begin{quote}
水自石邊流出,香從花裡飄來。
滿座清虛雅致,全無半點塵埃。
\end{quote}

那長老見此仙境,以為得意,情樂懷開,十分歡喜,忍不住念了一句道:
\begin{quote}
「禪心似月迥無塵。」
\end{quote}

勁節老笑而即聯道:
\begin{quote}
「詩興如天青更新。」
\end{quote}

孤直公道:
\begin{quote}
「好句漫裁摶錦繡。」
\end{quote}

凌空子道:
\begin{quote}
「佳文不點唾奇珍。」
\end{quote}

拂雲叟道:
\begin{quote}
「六朝一洗繁華盡,四始重刪雅頌分。」
\end{quote}

三藏道:「弟子一時失口,胡談幾字,誠所謂『班門弄斧』。適聞列仙之言,清新飄逸,真詩翁也。」勁節老道:「聖僧不必閑敘,出家人全始全終,既有起句,何無結句?望卒成之。」三藏道:「弟子不能,煩十八公結而成篇為妙。」勁節道:「你好心腸,你起的句,如何不肯結果?慳吝珠璣,非道理也。」三藏只得續後二句云:
\begin{quote}
「半枕松風茶未熟,吟懷瀟灑滿腔春。」
\end{quote}

十八公道:「好個『吟懷瀟灑滿腔春』!」孤直公道:「勁節,你深知詩味,所以只管咀嚼。何不再起一篇?」十八公亦慨然不辭道:「我卻是頂針字起:
\begin{quote}
春不榮華冬不枯,雲來霧往只如無。」
\end{quote}

凌空子道:「我亦體前頂針二句:
\begin{quote}
無風搖拽婆娑影,有客欣憐福壽圖。」
\end{quote}

拂雲叟亦頂針道:
\begin{quote}
「圖似西山堅節老,清如南國沒心夫。」
\end{quote}

孤直公亦頂針道:
\begin{quote}
「夫因側葉稱梁棟,臺為橫柯作憲烏。」
\end{quote}

長老聽了,讚嘆不已道:「真是陽春白雪,浩氣沖霄,弟子不才,敢再起兩句。」孤直公道:「聖僧乃有道之士,大養之人也。不必再相聯句,請賜教全篇,庶我等亦好勉強而和。」三藏無已,只得笑吟一律曰:
\begin{quote}
「杖錫西來拜法王,願求妙典遠傳揚。
金芝三秀詩壇瑞,寶樹千花蓮蕊香。
百尺竿頭須進步,十方世界立行藏。
修成玉像莊嚴體,極樂門前是道場。」
\end{quote}

四老聽畢,俱極讚揚。十八公道:「老拙無能,大膽攙越,也勉和一首。」云:
\begin{quote}
「勁節孤高笑木王,靈椿不似我名揚。
山空百丈龍蛇影,泉汲千年琥珀香。
解與乾坤生氣概,喜因風雨化行藏。
衰殘自愧無仙骨,惟有苓膏結壽場。」
\end{quote}

孤直公道:「此詩起句豪雄,聯句有力,但結句自謙太過矣。堪羨!堪羨!老拙也和一首。」云:
\begin{quote}
「霜姿常喜宿禽王,四絕堂前大器揚。
露重珠纓蒙翠蓋,風輕石齒碎寒香。
長廊夜靜吟聲細,古殿秋陰淡影藏。
元日迎春曾獻壽,老來寄傲在山場。」
\end{quote}

凌空子笑而言曰:「好詩,好詩,真個是月脅天心。老拙何能為和?但不可空過,也須扯談幾句。」曰:
\begin{quote}
「梁棟之材近帝王,太清宮外有聲揚。
晴軒恍若來青氣,暗壁尋常度翠香。
壯節凜然千古秀,深根結矣九泉藏。
凌雲勢蓋婆娑影,不在群芳豔麗場。」
\end{quote}

拂雲叟道:「三公之詩,高雅清淡,正是放開錦繡之囊也。我身無力,我腹無才,得三公之教,茅塞頓開。無已,也打油幾句,幸勿哂焉。」詩曰:
\begin{quote}
「淇澳園中樂聖王,渭川千畝任分揚。
翠筠不染湘娥淚,班籜堪傳漢史香。
霜葉自來顏不改,煙梢從此色何藏?
子猷去世知音少,亙古留名翰墨場。」
\end{quote}

三藏道:「眾仙老之詩,真個是吐鳳噴珠,游夏莫贊。厚愛高情,感之極矣。但夜已深沉,三個小徒不知在何處等我。弟子不能久留,敢此告回尋訪,尤無窮之至愛也。望老仙指示歸路。」四老笑道:「聖僧勿慮。我等也是千載奇逢,況天光晴爽,雖夜深卻月明如晝,再寬坐坐,待天曉自當遠送過嶺,高徒一定可相會也。」

正話間,只見石屋之外,有兩個青衣女童,挑一對絳紗燈籠,後引著一個仙女。那仙女撚著一枝杏花,笑吟吟進門相見。那仙女怎生模樣?他生得:
\begin{quote}
青姿妝翡翠,丹臉賽胭脂。星眼光還彩,蛾眉秀又齊。下襯一條五色梅淺紅裙子,上穿一件煙裡火比甲輕衣。弓鞋彎鳳嘴,綾襪錦拖泥。妖嬈嬌似天臺女,不亞當年俏妲姬。
\end{quote}

四老欠身問道:「杏仙何來?」那女子對眾道了萬福,道:「知有佳客在此賡酬,特來相訪,敢求一見。」十八公指著唐僧道:「佳客在此,何勞求見?」三藏躬身,不敢言語。那女子叫:「快獻茶來。」又有兩個黃衣女童捧一個紅漆丹盤,盤內有六個細磁茶盂,盂內設幾品異果,橫擔著匙兒;提一把白鐵嵌黃銅的茶壺,壺內香茶噴鼻。斟了茶,那女子微露春蔥,捧磁盂先奉三藏,次奉四老,然後一盞,自取而陪。

凌空子道:「杏仙為何不坐?」那女子方才去坐。茶畢,欠身問道:「仙翁今宵盛樂,佳句請教一二如何?」拂雲叟道:「我等皆鄙俚之言,惟聖僧真盛唐之作,甚可嘉羨。」那女子道:「如不吝教,乞賜一觀。」四老即以長老前詩後詩並禪法論,宣了一遍。那女子滿面春風,對眾道:「妾身不才,不當獻醜。但聆此佳句,似不可虛,勉強將後詩奉和一律如何?」遂朗吟道:
\begin{quote}
「上蓋留名漢武王,周時孔子立壇揚。
董仙愛我成林積,孫楚曾憐寒食香。
雨潤紅姿嬌且嫩,煙蒸翠色顯還藏。
自知過熟微酸意,落處年年伴麥場。」
\end{quote}

四老聞詩,人人稱賀,都道:「清雅脫塵,句內包含春意。好個『雨潤紅姿嬌且嫩』!『雨潤紅姿嬌且嫩』!」那女子笑而悄答道:「惶恐,惶恐。適聞聖僧之章,誠然錦心繡口。如不吝珠玉,賜教一闋如何?」唐僧不敢答應。那女子漸有見愛之情,挨挨軋軋,漸近坐邊,低聲悄語,呼道:「佳客莫者,趁此良宵,不耍子待要怎的?人生光景,能有幾何?」十八公道:「杏仙盡有仰高之情,聖僧豈可無俯就之意?如不見憐,是不知趣了也。」孤直公道:「聖僧乃有道有名之士,決不苟且行事。如此樣舉措,是我等取罪過了。污人名,壞人德,非遠達也。果是杏仙有意,可教拂雲叟與十八公做媒,我與凌空子保親,成此姻眷,何不美哉?」

三藏聽言,遂變了顏色,跳起來高叫道:「汝等皆是一類怪物,這般誘我。當時只以低行之言,談玄談道可也。如今怎麼以美人局來騙害貧僧?是何道理?」四老見三藏發怒,一個個咬指擔驚,再不復言。那赤身鬼使暴躁如雷道:「這和尚好不識擡舉。我這姐姐那些兒不好?他人材俊雅,玉質嬌姿,不必說那女工針指,只這一段詩材,也配得過你。你怎麼這等推辭?休錯過了。孤直公之言甚當,如果不可苟合,待我再與你主婚。」三藏大驚失色,憑他們怎麼胡談亂講,只是不從。鬼使又道:「你這和尚,我們好言好語,你不聽從。若是我們發起村野之性,還把你攝了去,教你和尚不得做,老婆不得娶,卻不枉為人一世也?」那長老心如金石,堅執不從。暗想道:「我徒弟們不知在那裡尋我哩!」說一聲,止不住眼中墮淚。那女子陪著笑,挨至身邊,翠袖中取出一個蜜合綾汗巾來,與他揩淚道:「佳客勿得煩惱。我與你倚玉偎香,耍子去來。」長老咄的一聲吆喝,跳起身來就走。被那些人扯扯拽拽,嚷到天明。

忽聽得那裡叫聲:「師父,師父,你在那方言語也?」原來那孫大聖與八戒、沙僧牽著馬,挑著擔,一夜不曾住腳,穿荊度棘,東尋西找。卻好半雲半霧的過了八百里荊棘嶺西下,聽得唐僧吆喝,卻就喊了一聲。那長老掙出門來,叫聲:「悟空,我在這裡哩。快來救我,快來救我。」那四老與鬼使,那女子與女童,幌一幌,都不見了。

須臾間,八戒、沙僧俱到邊前道:「師父,你怎麼得到此也?」三藏扯住行者道:「徒弟啊,多累了你們了。昨日晚間見的那個老者,言說土地送齋一事,是你喝聲要打,他就把我擡到此方。他與我攜手相攙,走入門,又見三個老者,來此會我,俱道我做『聖僧』。一個個言談清雅,極善吟詩。我與他賡和相攀,覺有夜半時候,又見一個美貌女子執燈火,也來這裡會我,吟了一首詩,稱我做『佳客』。因見我相貌,欲求配偶,我方省悟。正不從時,又被他做媒的做媒,保親的保親,主婚的主婚,我立誓不肯。正欲掙著要走,與他嚷鬧,不期你們到了。一則天明,二來還是怕你,只才還扯扯拽拽,忽然就不見了。」行者道:「你既與他敘話談詩,就不曾問他個名字?」三藏道:「我曾問他之號:那老者喚做十八公,號勁節;第二個號孤直公;第三個號凌空子;第四個號拂雲叟;那女子,稱他做杏仙。」八戒道:「此物在於何處?才往那方去了?」三藏道:「去向之方,不知何所;但只談詩之處,去此不遠。」

他三人同師父看處,只見一座石崖,崖上有「木仙庵」三字。三藏道:「此間正是。」行者仔細觀之,卻原來是一株大檜樹、一株老柏、一株老松、一株老竹,竹後有一株丹楓。再看崖那邊,還有一株老杏、二株臘梅、二株丹桂。行者笑道:「你可曾看見妖怪?」八戒道:「不曾。」行者道:「你不知就是這幾株樹木在此成精也。」八戒道:「哥哥怎得知成精者是樹?」行者道:「十八公乃松樹,孤直公乃柏樹,凌空子乃檜樹,拂雲叟乃竹竿,赤身鬼乃楓樹,杏仙即杏樹,女童即丹桂即臘梅也。」八戒聞言,不論好歹,一頓釘鈀,三五長嘴,連拱帶築,把兩顆臘梅、丹桂、老杏、楓楊俱揮倒在地,果然那根下俱鮮血淋漓。三藏近前扯住道:「悟能,不可傷了他。他雖成了氣候,卻不曾傷我。我等找路去罷。」行者道:「師父不可惜他,恐日後成了大怪,害人不淺也。」那獃子索性一頓鈀,將松、柏、檜、竹一齊皆築倒,卻才請師父上馬,順大路一齊西行。

畢竟不知前去如何,且聽下回分解。
