
\chapter{陷虎穴金星解厄 雙叉嶺伯欽留僧}

詩曰:
\begin{quote}
大有唐王降敕封,欽差玄奘問禪宗。
堅心磨琢尋龍穴,著意修持上鷲峰。
邊界遠遊多少國,雲山前度萬千重。
自今別駕投西去,秉教迦持悟大空。
\end{quote}

卻說三藏自貞觀十三年九月望前三日,蒙唐王與多官送出長安關外。一二日馬不停蹄,早至法門寺。本寺住持上房長老,帶領眾僧有五百餘人,兩邊羅列,接至裡面,相見獻茶。茶罷進齋,齋後不覺天晚。正是那:
\begin{quote}
影動星河近,月明無點塵。
雁聲鳴遠漢,砧韻響西鄰。
歸鳥棲枯樹,禪僧講梵音。
蒲團一榻上,坐到夜將分。
\end{quote}

眾僧們燈下議論佛門定旨,上西天取經的原由:有的說水遠山高,有的說路多虎豹;有的說峻嶺陡崖難度,有的說毒魔惡怪難降。三藏箝口不言,但以手指自心,點頭幾度。眾僧們莫解其意,合掌請問道:「法師指心點頭者,何也?」三藏答曰:「心生,種種魔生;心滅,種種魔滅。我弟子曾在化生寺對佛說下洪誓大願,不由我不盡此心。這一去,定要到西天,見佛求經,使我們法輪回轉,願聖主皇圖永固。」眾僧聞得此言,人人稱羨,個個宣揚,都叫一聲:「忠心赤膽大闡法師!」誇讚不盡,請師入榻安寐。

早又是竹敲殘月落,雞唱曉雲生。那眾僧起來,收拾茶水、早齋。玄奘遂穿了袈裟,上正殿,佛前禮拜道:「弟子陳玄奘,前往西天取經,但肉眼愚迷,不識活佛真形。今願立誓:路中逢廟燒香,遇佛拜佛,遇塔掃塔。但願我佛慈悲,早現丈六金身,賜真經,留傳東土。」祝罷,回方丈進齋。齋畢,那二從者整頓了鞍馬,促趲行程。三藏出了山門,辭別眾僧。眾僧不忍分別,直送有十里之遙,噙淚而返。三藏遂直西前進。正是那季秋天氣,但見:
\begin{quote}
數村木落蘆花碎,幾樹楓楊紅葉墜。路途煙雨故人稀,黃菊麗,山骨細,水寒荷破人憔悴。白蘋紅蓼霜天雪,落霞孤鶩長空墜。依稀黯淡野雲飛,玄鳥去,賓鴻至,嘹嘹嚦嚦聲宵碎。
\end{quote}

師徒們行了數日,到了鞏州城,早有鞏州合屬官吏人等迎接入城中。安歇一夜,次早出城前去。一路饑餐渴飲,夜住曉行,兩三日,又至河州衛。此乃是大唐的山河邊界。早有鎮邊的總兵與本處僧道,聞得是欽差御弟法師上西方見佛,無不恭敬。接至裡面供給了,著僧綱請往福原寺安歇。本寺僧人,一一參見,安排晚齋。齋畢,吩咐二從者飽喂馬匹,天不明就行。

及雞方鳴,隨喚從者,卻又驚動寺僧,整治茶湯齋供。齋罷,出離邊界。這長老心忙,太起早了。原來此時秋深時節,雞鳴得早,只好有四更天氣。一行三人,連馬四口,迎著清霜,看著明月,行有數十里遠近,見一山嶺,只得撥草尋路,說不盡崎嶇難走,又恐怕錯了路徑。正疑思之間,忽然失足,三人連馬都跌落坑坎之中。三藏心慌,從者膽戰。卻才悚懼,又聞得裡面哮吼高呼,叫:「拿將來!拿將來!」只見狂風滾滾,擁出五六十個妖邪,將三藏、從者揪了上去。這法師戰戰兢兢的偷眼觀看,上面坐的那魔王十分兇惡。真個是:
\begin{quote}
雄威身凜凜,猛氣貌堂堂。電目飛光艷,雷聲振四方。鋸牙舒口外,鑿齒露腮旁。錦繡圍身體,文斑裹脊梁。鋼鬍稀見肉,鉤爪利如霜。東海黃公懼,南山白額王。
\end{quote}

諕得個三藏魂飛魄散,二從者骨軟筋麻。魔王喝令綁了。眾妖一齊將三人用繩索綁縛。正要安排吞食,只聽得外面喧嘩,有人來報:「熊山君與特處士二位來也。」三藏聞言,擡頭觀看,前走的是一條黑漢。你道他是怎生模樣:
\begin{quote}
雄豪多膽量,輕健夯身軀。
涉水惟兇力,跑林逞怒威。
向來符吉夢,今獨露英姿。
綠樹能攀折,知寒善諭時。
准靈惟顯處,故此號山君。
\end{quote}

又見那後邊來的是一條胖漢。你道怎生模樣:
\begin{quote}
嵯峨雙角冠,端肅聳肩背。
性服青衣穩,蹄步多遲滯。
宗名父作牯,原號母稱牸。
能為田者功,因名特處士。
\end{quote}

這兩個搖搖擺擺,走入裡面,慌得那魔王奔出迎接。熊山君道:「寅將軍一向得意,可賀,可賀。」特處士道:「寅將軍丰姿勝常,真可喜,真可喜。」魔王道:「二公連日如何?」山君道:「惟守素耳。」處士道:「惟隨時耳。」三個敘罷,各坐談笑。

只見那從者綁得痛切悲啼。那黑漢道:「此三者何來?」魔王道:「自送上門來者。」處士笑云:「可能待客否?」魔王道:「奉承,奉承。」山君道:「不可盡用,食其二,留其一可也。」魔王領諾,即呼左右,將二從者剖腹剜心,剁碎其屍:將首級與心肝奉獻二客,將四肢自食,其餘骨肉分給各妖。只聽得嘓啅之聲,真似虎啖羊羔,霎時食盡。把一個長老幾乎諕死。這才是初出長安第一場苦難。

正愴慌之間,漸漸的東方發白。那二怪至天曉方散,俱道:「今日厚擾,容日竭誠奉酬。」方一擁而退。

不一時,紅日高昇,三藏昏昏沉沉,也辨不得東西南北。正在那不得命處,忽然見一老叟,手持拄杖而來。走上前,用手一拂,繩索皆斷。對面吹了一口氣,三藏方甦,跪拜於地道:「多謝老公公,搭救貧僧性命。」老叟答禮道:「你起來。你可曾疏失了甚麼東西?」三藏道:「貧僧的從人已是被怪食了。只不知行李、馬匹在於何處?」老叟用杖指道:「那廂不是一匹馬、兩個包袱?」三藏回頭看時,果是他的物件,並不曾失落,心才略放下些。問老叟曰:「老公公,此處是甚所在?公公何由在此?」老叟道:「此是雙叉嶺,乃虎狼巢穴處。你為何墮此?」三藏道:「貧僧雞鳴時,出河州衛界,不料起得早了,冒霜撥露,忽失落此地。見一魔王,兇頑太甚,將貧僧與二從者綁了。又見一條黑漢,稱是熊山君;一條胖漢,稱是特處士:走進來,稱那魔王是寅將軍。他三個把我二從者吃了,天光才散。不想我是那裡有這大緣大分,感得老公公來此救我?」老叟道:「處士者,是個野牛精;山君者,是個熊羆精;寅將軍者,是個老虎精。左右妖邪,盡都是山精樹鬼、怪獸蒼狼。只因你的本性元明,所以吃不得你。你跟我來,引你上路。」

三藏不勝感激,將包袱捎在馬上,牽著韁繩,相隨老叟徑出了坑坎之中,走上大路。卻將馬拴在道旁草頭上,轉身拜謝那公公,那公公遂化作一陣清風,跨一隻硃頂白鶴,騰空而去。只見風飄飄遺下一張簡帖,書上四句頌子。頌子云:
\begin{quote}
吾乃西天太白星,特來搭救汝生靈。
前行自有神徒助,莫為艱難報怨經。
\end{quote}

三藏看了,對天禮拜道:「多謝金星,度脫此難。」拜畢,牽了馬匹,獨自個孤孤悽悽,往前苦進。這嶺上,真個是:
\begin{quote}
寒颯颯雨林風,響潺潺澗下水。香馥馥野花開,密叢叢亂石磊。鬧嚷嚷鹿與猿,一隊隊獐和麂。喧雜雜鳥聲多,靜悄悄人事靡。那長老,戰兢兢心不寧;這馬兒,力怯怯蹄難舉。
\end{quote}

三藏捨身拚命,上了那峻嶺之間。行經半日,更不見個人煙村舍。一則腹中饑了,二則路又不平。正在危急之際,只見前面有兩隻猛虎咆哮,後邊有幾條長蛇盤繞。左有毒蟲,右有怪獸。三藏孤身無策,只得放下身心,聽天所命。又無奈那馬腰軟蹄彎,即便跪下,伏倒在地,打又打不起,牽又牽不動。苦得個法師襯身無地,真個有萬分悽楚,已自分必死,莫可奈何。

卻說他雖有災迍,卻有救應。正在那不得命處,忽然見毒蟲奔走,妖獸飛逃,猛虎潛蹤,長蛇隱跡。三藏擡頭看時,只見一人,手執鋼叉,腰懸弓箭,自那山坡前轉出,果然是一條好漢。你看他:
\begin{quote}
頭上戴一頂艾葉花斑豹皮帽,身上穿一領羊絨織錦叵羅衣,腰間束一條獅蠻帶,腳下屣一對麂皮靴。環眼圓睛如弔客,圈鬚亂擾似河奎。懸一囊毒藥弓矢,拿一桿點鋼大叉。雷聲震破山蟲膽,勇猛驚殘野雉魂。
\end{quote}

三藏見他來得漸近,跪在路傍,合掌高叫道:「大王救命!大王救命!」那條漢到邊前,放下鋼叉,用手攙起道:「長老休怕。我不是歹人,我是這山中的獵戶,姓劉名伯欽,綽號鎮山太保。我才自來,要尋兩隻山蟲食用。不期遇著你,多有衝撞。」三藏道:「貧僧是大唐駕下欽差,往西天拜佛求經的和尚。適間來到此處,遇著些狼虎蛇蟲,四邊圍繞,不能前進。忽見太保來,眾獸皆走,救了貧僧性命,多謝,多謝。」伯欽道:「我在這裡住人,專倚打些狼虎為生,捉些蛇蟲過活,故此眾獸怕我走了。你既是唐朝來的,與我都是鄉里。此間還是大唐的地界,我也是唐朝的百姓,我和你同食皇王的水土,誠然是一國之人。你休怕,跟我來,到我舍下歇馬,明朝我送你上路。」三藏聞言,滿心歡喜,謝了伯欽,牽馬隨行。

過了山坡,又聽得呼呼風響。伯欽道:「長老休走,坐在此間。風響處,是個山貓來了,等我拿他家去管待你。」三藏見說,又膽戰心驚,不敢舉步。那太保執了鋼叉,拽開步,迎將上去。只見一隻斑斕虎,對面撞見,他看見伯欽,急回頭就走。這太保霹靂一聲,咄道:「那業畜那裡走!」那虎見趕得急,轉身掄爪撲來。這太保三股叉舉手迎敵。諕得個三藏軟癱在草地。這和尚自出娘肚皮,那曾見這樣凶險的勾當。太保與那虎在那山坡下,人虎相持,果是一場好鬥。但見:
\begin{quote}
怒氣紛紛,狂風滾滾。怒氣紛紛,太保沖冠多膂力;狂風滾滾,斑彪逞勢噴紅塵。那一個張牙舞爪,這一個轉步回身。三股叉擎天幌日,千花尾擾霧雲飛。這一個當胸亂刺,那一個劈面來吞。閃過的再生人道,撞著的定見閻君。只聽得那斑彪哮吼,太保聲狠。斑彪哮吼,振裂山川驚鳥獸;太保聲狠,喝開天府現星辰。那一個金睛怒出,這一個壯膽生嗔。可愛鎮山劉太保,堪誇據地獸之君。人虎貪生爭勝負,些兒有慢喪三魂。
\end{quote}

他兩個鬥了有一個時辰,只見那虎爪慢腰鬆,被太保舉叉平胸刺倒。可憐啊,鋼叉尖穿透心肝,霎時間血流滿地。揪著耳朵,拖上路來。好男子,氣不連喘,面不改色,對三藏道:「造化,造化。這隻山貓,夠長老食用一日。」三藏誇讚不盡道:「太保真山神也!」伯欽道:「有何本事,敢勞過獎?這個是長老的洪福。去來,趕早兒剝了皮,煮些肉,管待你也。」

他一隻手執著叉,一隻手拖著虎,在前引路。三藏牽著馬,隨後而行。迤行過山坡,忽見一座山莊。那門前真個是:
\begin{quote}
參天古樹,漫路荒籐。萬壑風塵冷,千崖氣象奇。一徑野花香襲體,數竿幽竹綠依依。草門樓,籬笆院,堪描堪畫;石板橋,白土壁,真樂真稀。秋容蕭索,爽氣孤高。道傍黃葉落,嶺上白雲飄。疏林內山禽聒聒,莊門外細犬嘹嘹。
\end{quote}

伯欽到了門首,將死虎擲下,叫:「小的們何在?」只見走出三四個家僮,都是怪形惡相之類,上前拖拖拉拉,把隻虎扛將進去。伯欽吩咐教趕早剝了皮,安排將來待客。復回頭迎接三藏進內,彼此相見,三藏又拜謝伯欽厚恩憐憫救命。伯欽道:「同鄉之人,何勞致謝。」坐定茶罷,有一老嫗領著一個媳婦,對三藏進禮。伯欽道:「此是家母、小妻。」三藏道:「請令堂上坐,貧僧奉拜。」老嫗道:「長老遠客,各請自珍,不勞拜罷。」伯欽道:「母親啊,他是唐王駕下,差往西天見佛求經者。適間在嶺頭上遇著孩兒,孩兒念一國之人,請他來家歇馬,明日送他上路。」老嫗聞言,十分懽喜道:「好,好,好。就是請他,不得這般恰好。明日你父親週忌,就浼長老做些好事,念卷經文,到後日送他去罷。」這劉伯欽雖是一個殺虎手,鎮山的太保,他卻有些孝順之心。聞得母言,就要安排香紙,留住三藏。

說話間,不覺的天色將晚。小的們排開桌凳,拿幾盤爛熟虎肉,熱騰騰的放在上面。伯欽請三藏權用,再另辦飯。三藏合掌當胸道:「善哉!貧僧不瞞太保說,自出娘胎,就做和尚,更不曉得吃葷。」伯欽聞得此說,沉吟了半晌道:「長老,寒家歷代以來,不曉得吃素。就是有些竹筍,採些木耳,尋些乾菜,做些豆腐,也都是獐鹿虎豹的油煎,卻無甚素處。有兩眼鍋灶,也都是油膩透了。這等奈何?反是我請長老的不是。」三藏道:「太保不必多心,請自受用。我貧僧就是三五日不吃飯,也可忍餓,只是不敢破了齋戒。」伯欽道:「倘或餓死,卻如之何?」三藏道:「感得太保天恩,搭救出虎狼叢裡,就是餓死,也強如喂虎。」

伯欽的母親聞說,叫道:「孩兒不要與長老閑講,我自有素物,可以管待。」伯欽道:「素物何來?」母親道:「你莫管我,我自有素的。」叫媳婦將小鍋取下,著火燒了油膩,刷了又刷,洗了又洗,卻仍安在灶上。先燒半鍋滾水,別用。卻又將些山地榆葉子,著水煎作茶湯。然後將些黃粱粟米,煮起飯來。又把些乾菜煮熟。盛了兩碗,拿出來鋪在桌上。老母對著三藏道:「長老請齋。這是老身與兒婦,親自動手整理的些極潔極淨的茶飯。」三藏下來謝了,方才上坐。

那伯欽另設一處,鋪排些沒鹽沒醬的老虎肉、香獐肉、蟒蛇肉、狐狸肉、兔肉,點剁鹿肉乾巴,滿盤滿碗的陪著三藏吃齋。方坐下,心欲舉箸,只見三藏合掌誦經,諕得個伯欽不敢動箸,急起身立在旁邊。三藏念不數句,卻教請齋。伯欽道:「你是個念短頭經的和尚?」三藏道:「此非是經,乃是一卷揭齋之咒。」伯欽道:「你們出家人,偏有許多計較,吃飯便也念誦念誦。」

吃了齋飯,收了盤碗,漸漸天晚。伯欽引著三藏出中宅,到後邊走走。穿過夾道,有一座草亭。推開門,入到裡面。只見那四壁上掛幾張強弓硬弩,插幾壺箭;過梁上搭兩塊血腥的虎皮;牆根頭插著許多槍刀叉棒;正中間設兩張坐器。伯欽請三藏坐坐。三藏見這般兇險腌臢,不敢久坐,遂出了草亭。又往後再行,是一座大園子,卻看不盡那叢叢菊蕊堆黃,樹樹楓楊掛赤。又見呼的一聲,跑出十來隻肥鹿,一大陣黃獐,見了人,呢呢痴痴,更不恐懼。三藏道:「這獐鹿想是太保養家了的?」伯欽道:「似你那長安城中人家,有錢的集財寶,有莊的集聚稻糧。我們這打獵的,只得聚養些野獸,備天陰耳。」他兩個說話閑行,不覺黃昏,復轉前宅安歇。

次早,那合家老小都起來,就整素齋,管待長老,請開啟念經。這長老淨了手,同太保家堂前拈了香,拜了家堂。三藏方敲響木魚,先念了淨口業的真言,又念了淨身心的神咒,然後開《度亡經》一卷。誦畢,伯欽又請寫薦亡疏一道,再開念《金剛經》、《觀音經》。一一朗音高誦。誦畢,吃了午齋,又念《法華經》、《彌陀經》,各誦幾卷,又念一卷《孔雀經》,及談苾蒭洗業的故事,早又天晚。獻過了種種香火,化了眾神紙馬,燒了薦亡文疏。佛事已畢,又各安寢。

卻說那伯欽的父親之靈,超薦得脫沉淪,鬼魂兒早來到自家宅內,托一夢與合宅長幼道:「我在陰司裡苦難難脫,日久不得超生。今幸得聖僧念了經卷,消了我的罪業,閻王差人送我上中華富地,長者人家托生去了。你們可好生謝送長老,不要怠慢,不要怠慢。我去也。」這才是:萬法莊嚴端有意,薦亡離苦出沉淪。那合家兒夢醒,又早太陽東上。伯欽的娘子道:「太保,我今夜夢見公公來家,說他在陰司苦難難脫,日久不得超生。今幸得聖僧念了經卷,消了他的罪業,閻王差人送他上中華富地,長者人家托生去,教我們好生謝那長老,不得怠慢他。說罷,徑出門,徉徜去了。我們叫他不應,留他不住。醒來卻是一夢。」伯欽道:「我也是那等一夢,與你一般。我們起去對母親說去。」他兩口子正欲去說,只見老母叫道:「伯欽孩兒,你來,我與你說話。」二人至前,老母坐在床上道:「兒啊,我今夜得了個喜夢,夢見你父親來家,說多虧了長老超度,已消了罪業,上中華富地,長者家去托生。」夫妻們俱呵呵大笑道:「我與媳婦皆有此夢,正來告稟,不期母親呼喚,也是此夢。」遂叫一家大小起來,安排謝意,替他收拾馬匹,都至前拜謝道:「多謝長老超薦我亡父脫難超生,報答不盡。」三藏道:「貧僧有何能處,敢勞致謝?」

伯欽把三口兒的夢話對三藏陳訴一遍,三藏也喜。早供給了素齋,又具白銀一兩為謝。三藏分文不受。一家兒又懇懇拜央,三藏畢竟分文未受。但道:「是你肯發慈悲送我一程,足感至愛。」伯欽與母妻無奈,急做了些粗麵燒餅乾糧,叫伯欽遠送。三藏歡喜收納。太保領了母命,又喚兩三個家僮,各帶捕獵的器械,同上大路。看不盡那山中野景,嶺上風光。

行經半日,只見對面處有一座大山,真個是高接青霄,崔巍險峻。三藏不一時到了邊前。那太保登此山如行平地,正走到半山之中,伯欽回身,立於路下道:「長老,請自前進,我卻告回。」三藏聞言,滾鞍下馬道:「千萬敢勞太保再送一程。」伯欽道:「長老不知。此山喚做兩界山,東半邊屬我大唐所管,西半邊乃是韃靼的地界。那廂狼虎不伏我降,我卻也不能過界,你自去罷。」三藏心驚,掄開手,牽衣執袂,滴淚難分。正在那叮嚀拜別之際,只聽得山腳下叫喊如雷道:「我師父來也!我師父來也!」諕得個三藏痴呆,伯欽打掙。

畢竟不知是甚人叫喊,且聽下回分解。
