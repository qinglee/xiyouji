
\chapter{黑河妖孽擒僧去 西洋龍子捉鼉回}

卻說那菩薩念了幾遍,卻才住口,那妖精就不疼了。又正性起身看處,頸項裡與手足上都是金箍,勒得疼痛,便就除那箍兒時,莫想褪得動分毫。這寶貝已此是見肉生根,越抹越痛。行者笑道:「我那乖乖,菩薩恐你養不大,與你戴個頸圈鐲頭哩。」那童子聞此言,又生煩惱,就此綽起槍來,望行者亂刺。行者急閃身,立在菩薩後面,叫:「念咒,念咒。」那菩薩將楊柳枝兒蘸了一點甘露,灑將去,叫聲:「合!」只見他丟了槍,一雙手合掌當胸,再也不能開放。至今留了一個觀音扭,即此意也。那童子開不得手,拿不得槍,方知是法力深微,沒奈何,才納頭下拜。

菩薩念動真言,把淨瓶攲倒,將那一海水依然收去,更無半點存留。對行者道:「悟空,這妖精已是降了,卻只是野心不定。等我教他一步一拜,只拜到落伽山,方才收法。你如今快早去洞中,救你師父去來。」行者轉身叩頭道:「有勞菩薩遠涉,弟子當送一程。」菩薩道:「你不消送,恐怕誤了你師父性命。」行者聞言,歡喜叩別。那妖精早歸了正果,五十三參,參拜觀音。

且不題善菩薩收了童子。卻說那沙僧久坐林間,盼望行者不到,將行李捎在馬上,一隻手執著降妖寶杖,一隻手牽著韁繩,出松林向南觀看,只見行者欣喜而來。沙僧迎著道:「哥哥,你怎麼去請菩薩,此時才來?焦殺我也。」行者道:「你還做夢哩,老孫已請了菩薩,降了妖怪。」行者卻將菩薩的法力,備陳了一遍。沙僧十分歡喜道:「救師父去也。」

他兩個才跳過澗去,撞到門前,拴下馬匹。舉兵器齊打入洞裡,剿淨了群妖,解下皮袋,放出八戒來。那獃子謝了行者道:「哥哥,那妖精在那裡?等我去築他幾鈀,出出氣來。」行者道:「且尋師父去。」

三人徑至後邊,只見師父赤條條,綑在院中哭哩。沙僧連忙解繩,行者即取衣服穿上。三人跪在面前道:「師父吃苦了。」三藏謝道:「賢徒啊,多累你等。怎生降得妖魔也?」行者又將請菩薩,收童子之言,備陳一遍。三藏聽得,即忙跪下,朝南禮拜。行者道:「不消謝他,轉是我們與他作福,收了一個童子。」如今說童子拜觀音,五十三參,參參見佛,即此是也。

教沙僧將洞內寶物收了。且尋米糧,安排齋飯,管待了師父。那長老得性命,全虧孫大聖;取真經,只靠美猴精。

師徒們出洞來,攀鞍上馬,找大路,篤志投西。行經一個多月,忽聽得水聲振耳。三藏大驚道:「徒弟呀,又是那裡水聲?」行者笑道:「你這老師父忒也多疑,做不得和尚。我們一同四眾,偏你聽見甚麼水聲。你把那《多心經》又忘了也?」唐僧道:「《多心經》乃浮屠山烏巢禪師口授,共五十四句,二百七十個字。我當時耳傳,至今常念,你知我忘了那句兒?」行者道:「老師父,你忘了『無眼耳鼻舌身意』。我等出家之人,眼不視色,耳不聽聲,鼻不嗅香,舌不嘗味,身不知寒暑,意不存妄想:如此謂之祛褪六賊。你如今為求經,念念在意;怕妖魔,不肯捨身;要齋吃,動舌;喜香甜,觸鼻;聞聲音,驚耳;睹事物,凝眸;招來這六賊紛紛,怎生得西天見佛?」三藏聞言,默然沉慮道:「徒弟啊,我
\begin{quote}
一自當年別聖君,奔波晝夜甚慇懃。
芒鞋踏破山頭霧,竹笠沖開嶺上雲。
夜靜猿啼殊可嘆,月明鳥噪不堪聞。
何時滿足三三行,得取如來妙法文?」
\end{quote}

行者聽畢,忍不住鼓掌大笑道:「這師父原來只是思鄉難息。若要那三三行滿,有何難哉?常言道:『功到自然成』哩。」八戒回頭道:「哥啊,若照依這般魔障凶高,就走上一千年也不得成功。」沙僧道:「二哥,你和我一般,拙口鈍腮,不要惹大哥熱擦。且只捱肩磨擔,終須有日成功也。」

師徒們正話間,腳走不停,馬蹄正疾,見前面有一道黑水滔天,馬不能進。四眾停立岸邊,仔細觀看,但見那:
\begin{quote}
層層濃浪,疊疊渾波,層層濃浪翻烏潦,疊疊渾波捲黑油。近觀不照人身影,遠望難尋樹木形。滾滾一地墨,滔滔千里灰。水沫浮來如積炭,浪花飄起似翻煤。牛羊不飲,鴉鵲難飛。牛羊不飲嫌深黑,鴉鵲難飛怕渺瀰。只是岸上蘆蘋知節令,灘頭花草鬥青奇。湖泊江河天下有,溪源澤洞世間多。人生皆有相逢處,誰見西方黑水河?
\end{quote}

唐僧下馬道:「徒弟,這水怎麼如此渾黑?」八戒道:「是那家潑了靛缸了。」沙僧道:「不然,是誰家洗筆硯哩。」行者道:「你們且休胡猜亂道,且設法保師父過去。」八戒道:「這河若是老豬過去不難:或是駕了雲頭,或是下河負水,不消頓飯時,我就過去了。」沙僧道:「若教我老沙,也只消縱雲屣水,頃刻而過。」行者道:「我等容易,只是師父難哩。」三藏道:「徒弟啊,這河有多少寬麼?」八戒道:「約摸有十來里寬。」三藏道:「你三個計較,著那個馱我過去罷。」行者道:「八戒馱得。」八戒道:「不好馱:若是馱著騰雲,三尺也不能離地。常言道:『背凡人重若丘山。』若是馱著負水,轉連我墜下水去了。」

師徒們在河邊正都商議,只見那上溜頭有一人棹下一隻小船兒來。唐僧喜道:「徒弟,有船來了,叫他渡我們過去。」沙僧厲聲高叫道:「棹船的,來渡人,來渡人。」船上人道:「我不是渡船,如何渡人?」沙僧道:「天上人間,方便第一。你雖不是渡船,我們也不是常來打攪你的。我等是東土欽差取經的佛子,你可方便方便,渡我們過去,謝你。」那人聞言,卻把船兒棹近岸邊,扶著槳道:「師父啊,我這船小,你們人多,怎能全渡?」三藏近前看了,那船兒原來是一段木頭刻的,中間只有一個艙口,只好坐下兩個人。三藏道:「怎生是好?」沙僧道:「這般啊,兩遭兒渡罷。」八戒就使心術,要躲懶討乖,道:「悟淨,你與大哥在這邊看著行李、馬匹,等我保師父先過去,卻再來渡馬。教大哥跳過去罷。」行者點頭道:「你說的是。」

那獃子扶著唐僧,那梢公撐開船,舉棹沖流,一直而去。方才行到中間,只聽得一聲響喨,捲浪翻波,遮天迷日。那陣狂風十分利害!好風:
\begin{quote}
當空一片炮雲起,中溜千層黑浪高。
兩岸飛沙迷日色,四邊樹倒振天號。
翻江攪海龍神怕,播土揚塵花木凋。
呼呼響若春雷吼,陣陣兇如餓虎哮。
蟹鱉魚蝦朝上拜,飛禽走獸失窩巢。
五湖船戶皆遭難,四海人家命不牢。
溪內漁翁難把鉤,河間梢子怎撐篙?
揭瓦翻磚房屋倒,驚天動地泰山搖。
\end{quote}

這陣風,原來就是那棹船人弄的。他本是黑水河中怪物。眼看著那唐僧與豬八戒,連船兒淬在水裡,無影無形,不知攝了那方去也。

這岸上沙僧與行者心慌道:「怎麼好?老師父步步逢災,才脫了魔障,幸得這一路平安,又遇著黑水迍邅。」沙僧道:「莫是翻了船?我們往下溜頭找尋去。」行者道:「不是翻船,若翻船,八戒會水,他必然保師父,負水而出。我才見那個棹船的有些不正氣,想必就是這廝弄風,把師父拖下水去了。」沙僧聞言道:「哥哥何不早說?你看著馬與行李,等我下水找尋去來。」行者道:「這水色不正,恐你不能去。」沙僧道:「這水比我那流沙河如何?去得,去得。」

好和尚,脫了褊衫,紮抹了手腳,掄著降妖寶杖,撲的一聲,分開水路,鑽入波中,大搭步行將進去。正走處,只聽得有人言語。沙僧閃在傍邊,偷睛觀看,那壁廂有一座亭臺,臺門外橫封了八個大字,乃是「衡陽峪黑水河神府」。又聽得那怪物坐在上面道:「一向辛苦,今日方能得物。這和尚乃十世修行的好人,但得吃他一塊肉,便做長生不老人。我為他也等夠多時,今朝卻不負我志。」教:「小的們,快把鐵籠擡出來,將這兩個和尚囫圇蒸熟,具柬去請二舅爺來,與他暖壽。」沙僧聞言,按不住心頭火起,掣寶杖,將門亂打。口中罵道:「那潑物,快送我唐僧師父與八戒師兄出來!」諕得那門內妖邪急跑去報:「禍事了。」老怪問:「甚麼禍事?」小妖道:「外面有一個晦氣色臉的和尚,打著前門罵,要人哩!」

那怪聞言,即喚取披掛。小妖擡出披掛。老妖結束整齊,手提一根竹節鋼鞭,走出門來,真個是兇頑毒像。但見:
\begin{quote}
方面圜睛霞彩亮,捲唇巨口血盆紅。
幾根鐵線稀髯擺,兩鬢朱砂亂髮蓬。
形似顯靈真太歲,貌如發怒狠雷公。
身披鐵甲團花燦,頭戴金盔嵌寶濃。
竹節鋼鞭提手內,行時滾滾拽狂風。
生來本是波中物,脫去原流變化兇。
要問妖邪真姓字,前身喚做小鼉龍。
\end{quote}

那怪喝道:「是甚人在此打我門哩?」沙僧道:「我把你個無知的潑怪!你怎麼弄玄虛,變作梢公,架船將我師父攝來?快早送還,饒你性命。」那怪呵呵笑道:「這和尚不知死活。你師父是我拿了,如今要蒸熟了請人哩。你上來,與我見個雌雄。三合敵得我啊,還你師父;如三合敵不得,連你一發都蒸吃了,休想西天去也。」沙僧聞言大怒,掄寶杖,劈頭就打;那怪舉鋼鞭,急架相迎。兩個在水底下,這場好殺:
\begin{quote}
降妖杖與竹節鞭,二人怒發各爭先。一個是黑水河中千載怪,一個是靈霄殿外舊時仙。那個因貪三藏肉中吃,這個為保唐僧命可憐。都來水底相爭鬥,各要功成兩不然。殺得蝦魚對對搖頭躲,蟹鱉雙雙縮首潛。只聽水府群妖齊擂鼓,門前眾怪亂爭喧。好個沙門真悟淨,單身獨力展威權。躍浪翻波無勝敗,鞭迎杖架兩牽連。算來只為唐和尚,欲取真經拜佛天。
\end{quote}

他二人戰經三十回合,不見高低。沙僧暗想道:「這怪物是我的對手,枉自不能取勝,且引他出去,教師兄打他。」這沙僧虛丟了個架子,拖著寶杖就走。那妖精更不趕來,道:「你去罷,我不與你鬥了,我且具柬帖兒去請客哩。」

沙僧氣呼呼跳出水來,見了行者道:「哥哥,這怪物無禮。」行者問道:「你下去許多時才出來,端的是甚妖邪?可曾尋見師父?」沙僧道:「他這裡邊有一座亭臺,臺門外橫書八個大字,喚做『衡陽峪黑水河神府』。我閃在傍邊,聽他在裡面說話,教小的們刷洗鐵籠,待要把師父與八戒蒸熟了,去請他舅爺來暖壽。是我發起怒來,就去打門。那怪物提一條竹節鋼鞭走出來,與我鬥了這半日,約有三十合,不分勝負。我卻使個佯輸法,要引他出來,著你助陣。那怪物乖得緊,他不來趕我,只要回去具柬請客。我才上來了。」行者道:「不知是個甚麼妖邪?」沙僧道:「那模樣象像個大鱉;不然,便是個鼉龍也。」行者道:「不知那個是他舅爺?」

說不了,只見那下灣裡走出一個老人,遠遠的跪下,叫:「大聖,黑水河河神叩頭。」行者道:「你莫是那棹船的妖邪,又來騙我麼?」那老人磕頭滴淚道:「大聖,我不是妖邪,我是這河內真神。那妖精舊年五月間,從西洋海趁大潮來於此處,就與小神交鬥。奈我年邁身衰,敵他不過,把我坐的那衡陽峪黑水河神府就占奪去住了,又傷了我許多水族。我卻沒奈何,徑往海內告他。原來西海龍王是他的母舅,不准我的狀子,教我讓與他住。我欲啟奏上天,奈何神微職小,不能得見玉帝。今聞得大聖到此,特來參拜投生。萬望大聖與我出力報冤。」行者聞言道:「這等說,西海龍王都該有罪。他如今攝了我師父與師弟,揚言要蒸熟了,去請他舅爺暖壽。我正要拿他,幸得你來報信。這等,河神你陪著沙僧在此看守,等我去海中,先把那龍王捉來,教他擒此怪物。」河神道:「深感大聖大恩。」

行者即駕雲,徑至西洋大海。按觔斗,捻了避水訣,分開波浪。正然走處,撞見一個黑魚精捧著一個渾金的請書匣兒,從下流頭似箭如梭鑽將上來。被行者撲個滿面,掣鐵棒分頂一下,可憐就打得腦漿迸出,腮骨查開,嗗都的一聲,飄出水面。他卻揭開匣兒看處,裡邊有一張簡帖,上寫著:
\begin{quote}
愚甥鼉潔,頓首百拜,啟上二舅爺敖老大人臺下:向承佳惠,感感。今因獲得二物,乃東土僧人,實為世間之罕物,甥不敢自用。因念舅爺聖誕在邇,特設菲筵,預祝千壽。萬望車駕速臨,是荷。」
\end{quote}

行者笑道:「這廝都把供狀先遞與老孫也。」這才袖了帖子,往前再行。早有一個探海的夜叉望見行者,急抽身撞上水晶宮:「報大王:齊天大聖孫爺爺來了。」

那龍王敖順,即領眾水族,出宮迎接道:「大聖,請入小宮少座,獻茶。」行者道:「我還不曾吃你的茶,你倒先吃了我的酒也!」龍王笑道:「大聖一向皈依佛門,不動葷酒,卻幾時請我吃酒來?」行者道:「你便不曾去吃酒,只是惹下一個吃酒的罪名了。」敖順大驚道:「小龍為何有罪?」行者袖中取出簡帖兒,遞與龍王。

龍王見了,魂飛魄散,慌忙跪下,叩頭道:「大聖恕罪。那廝是舍妹第九個兒子。因妹夫錯行了風雨,刻減了雨數,被天曹降旨,著人曹官魏徵丞相夢裡斬了。舍妹無處安身,是小龍帶他到此,恩養成人。前年不幸舍妹疾故,惟他無方居住,我著他在黑水河養性修真,不期他作此惡孽。小龍即差人去擒他來也。」行者道:「你令妹共有幾個賢郎?都在那裡作怪?」龍王道:「舍妹有九個兒子。那八個都是好的:第一個小黃龍,見居淮瀆;第二個小驪龍,見住濟瀆;第三個青背龍,占了江瀆;第四個赤髯龍,鎮守河瀆;第五個徒勞龍,與佛祖司鐘;第六個穩獸龍,與神宮鎮脊;第七個敬仲龍,與玉帝守擎天華表;第八個蜃龍,在大家兄處砥據太岳。此乃第九個鼉龍,因年幼無甚執事,自舊年才著他居黑水河養性,待成名,別遷調用。誰知他不遵吾旨,衝撞大聖也。」

行者聞言,笑道:「你妹妹有幾個妹丈?」敖順道:「只嫁得一個妹丈,乃涇河龍王。向年以此被斬,舍妹孀居於此,前年疾故了。」行者道:「一夫一妻,如何生此幾個雜種?」敖順道:「此正謂『龍生九種,九種各別。』」行者道:「我才心中煩惱,欲將簡帖為證,上奏天庭,問你個通同作怪,搶奪人口之罪。據你所言,是那廝不遵教誨,我且饒你這次:一則是看你昆玉分上;二來只該怪那廝年幼無知,你也不甚知情。你快差人擒來,救我師父,再作區處。」敖順即喚太子摩昂:「快點五百蝦魚壯兵,將小鼉捉來問罪。」一壁廂安排酒席,與大聖陪禮。行者道:「龍王再勿多心,既講開饒了你便罷,又何須辦酒?我今須與你令郎同去:一則老師父遭愆,二則我師弟盼望。」

那老龍苦留不住,又見龍女捧茶來獻。行者立飲他一盞香茶,別了老龍,隨與摩昂領兵,離了西海,早到黑水河中。行者道:「賢太子,好生捉怪,我上岸去也。」摩昂道:「大聖寬心,小龍子將他拿上來先見了大聖,懲治了他罪名,把師父送上來,才敢帶回海內,見我家父。」行者欣然相別,捏了避水訣,跳出波津,徑到了東邊崖上。沙僧與那河神迎著道:「師兄,你去時從空而去,怎麼回來卻自河內而回?」行者把那打死魚精,得簡帖,怪龍王,與太子同領兵來之事,備陳了一遍。沙僧十分歡喜,都立在岸邊,候接師父不題。

卻說那摩昂太子著介士先到他水府門前,報與妖怪道:「西海老龍王太子摩昂來也。」那怪正坐,忽聞摩昂來,心中疑惑道:「我差黑魚精投簡帖拜請二舅爺,這早晚不見回話,怎麼舅爺不來,卻是表兄來耶?」正說間,只見那巡河的小怪又來報:「大王,河內有一枝兵,屯於水府之西,旗號上書著『西海儲君摩昂小帥』。」妖怪道:「這表兄卻也狂妄。想是舅爺不得來,命他來赴宴。既是赴宴,如何又領兵勞士?咳!但恐其間有故。」教:「小的們,將我的披掛鋼鞭伺候,恐一時變暴。待我且出去迎他,看是何如。」眾妖領命,一個個擦掌摩拳準備。

這鼉龍出得門來,真個見一枝海兵劄營在右。只見:
\begin{quote}
征旗飄繡帶,畫戟列明霞。
寶劍凝光彩,長槍纓繞花。
弓彎如月小,箭插似狼牙。
大刀光燦燦,短棍硬沙沙。
鯨鰲並蛤蚌,蟹鱉共魚蝦。
大小齊齊擺,干戈似密麻。
不是元戎令,誰敢亂爬蹅?
\end{quote}

鼉怪見了,徑至那營門前,厲聲高叫:「大表兄,小弟在此拱候,有請。」有一個巡營的螺螺,急至中軍帳:「報千歲殿下:外有鼉龍叫請哩。」太子按一按頂上金盔,束一束腰間寶帶,手提一根三棱簡,拽開步,跑出營去,道:「你來請我怎麼?」鼉龍進禮道:「小弟今早有簡帖拜請舅爺,想是舅爺見棄,著表兄來的。兄長既來赴席,如何又勞師動眾?不入水府,扎營在此,又貫甲提兵,何也?」太子道:「你請舅爺做甚?」妖怪道:「小弟一向蒙恩賜居於此,久別尊顏,未得孝順。昨日捉得一個東土僧人,我聞他是十世修行的元體,人吃了他,可以延壽,欲請舅爺看過,上鐵籠蒸熟,與舅爺暖壽哩。」太子喝道:「你這廝十分懵懂!你道僧人是誰?」妖怪道:「他是唐朝來的僧人,往西天取經的和尚。」太子道:「你只知他是唐僧,不知他手下徒弟利害哩。」妖怪道:「他有一個長嘴的和尚,喚做個豬八戒,我也把他捉住了,要與唐和尚一同蒸吃。還有一個徒弟,喚做沙和尚,乃是一條黑漢子,晦氣色臉,使一根寶杖,昨日在這門外與我討師父,被我帥出河兵,一頓鋼鞭,戰得他敗陣逃生,也不見怎的利害。」

太子道:「原來是你不知。他還有一個大徒弟,是五百年前大鬧天宮上方太乙金仙齊天大聖。如今保護唐僧往西天拜佛求經,是普陀巖大慈大悲觀音菩薩勸善,與他改名,喚做孫悟空行者。你怎麼沒得做,撞出這件禍來?他又在我海內遇著你的差人,奪了請帖,徑入水晶宮,拿捏我父子們有結連妖邪,搶奪人口之罪。你快把唐僧、八戒送上河邊,交還了孫大聖,憑著我與他陪禮,你還好得性命;若有半個『不』字,休想得全生居於此也。」那怪鼉聞此言,心中大怒道:「我與你嫡親的姑表,你倒反護他人。聽你所言,就教把唐僧送出。天地間那裡有這等容易事也?你便怕他,莫成我也怕他?他若有手段,敢來我水府門前與我交戰三合,我才與他師父;若敵不過我,就連他也拿來,一齊蒸熟,也沒甚麼親人,也不去請客,自家關了門,教小的們唱唱舞舞,我坐在上面,自自在在,吃他娘不是。」

太子見說,開口罵道:「這潑邪!果然無狀。且不要教孫大聖與你對敵,你敢與我相持麼?」那怪道:「要做好漢,怕甚麼相持?」教:「取披掛。」呼喚一聲,眾小妖跟隨左右,獻上披掛,捧上鋼鞭。他兩個變了臉,各逞英雄;傳號令,一齊擂鼓。這一場比與沙僧爭鬥,甚是不同。但見那:
\begin{quote}
旌旗照耀,戈戟搖光。這壁廂營盤解散,那壁廂門戶開張。摩昂太子提金簡,鼉怪掄鞭急架償。一聲炮響河兵烈,三棒鑼鳴海士狂。鰕與鰕爭,蟹與蟹鬥。鯨鰲吞赤鯉,鯁鮊起黃鱨。鯊鯔吃鯖魚走,牡蠣擒蟶蛤蚌慌。少揚刺硬如鐵棍,鯤司針利似鋒芒。鱏鯕追白蟮,魲鱠捉烏鯧。一河水怪爭高下,兩處龍兵定弱強。混戰多時波浪滾,摩昂太子賽金剛。喝聲金簡當頭重,拿住妖鼉作怪王。
\end{quote}

這太子將三稜簡閃了一個破綻,那妖精不知是詐,鑽將進來,被他使個解數,把妖精右臂只一簡,打了個躘踵。趕上前,又一拍腳,跌倒在地。眾海兵一擁上前,揪翻住,將繩子背綁了雙手,將鐵索穿了琵琶骨,拿上岸來。押至孫行者面前道:「大聖,小龍子捉住妖鼉,請大聖定奪。」

行者與沙僧見了道:「你這廝不遵旨令。你舅爺原著你在此居住,教你養性存身,待你名成之日,別有遷用。你怎麼強占水神之宅,倚勢行兇,欺心誑上,弄玄虛,騙我師父、師弟?我待要打你這一棒,奈何老孫這棒子甚重,略打打兒就了了性命。你將我師父安在何處哩?」那怪叩頭不住道:「大聖,小鼉不知大聖大名。卻才逆了表兄,騁強背理,被表兄把我拿住。今見大聖,幸蒙大聖不殺之恩,感謝不盡。你師父還綑在那水府之間,望大聖解了我的鐵索,放了我手,等我到河中送他出來。」摩昂在傍道:「大聖,這廝是個逆怪,他極奸詐,若放了他,恐生惡念。」沙和尚道:「我認得他那裡,等我尋師父去。」

他兩個跳入水中,徑至水府門前。那裡門扇大開,更無一個小卒。直入亭臺裡面,見唐僧、八戒赤條條都綑在那裡。沙僧即忙解了師父,河神亦隨解了八戒,一家背著一個,出水面,徑至岸邊。豬八戒見那妖精鎖綁在側,急掣鈀上前就築,口裡罵道:「潑邪畜!你如今不吃我了?」行者扯住道:「兄弟,且饒他死罪罷,看敖順賢父子之情。」摩昂進禮道:「大聖,小龍子不敢久停。既然救得你師父,我帶這廝去見家父;雖大聖饒了他死罪,家父決不饒他活罪,定有發落處置,仍回覆大聖謝罪。」行者道:「既如此,你領他去罷。多多拜上令尊,尚容面謝。」那太子押著那妖潑,投水中,帥領海兵,徑轉西洋大海不題。

卻說那黑水河神謝了行者道:「多蒙大聖復得水府之恩。」唐僧道:「徒弟啊,如今還在東岸,如何渡此河也?」河神道:「老爺勿慮,且請上馬,小神開路,引老爺過河。」那師父才騎了白馬,八戒採著韁繩,沙和尚挑了行李,孫行者扶持左右。只見河神作起阻水的法術,將上流擋住,須臾,下流撤乾,開出一條大路。師徒們行過西邊,謝了河神,登崖上路。這正是:
\begin{quote}
禪僧有救朝西域,徹地無波過黑河。
\end{quote}

畢竟不知怎生得拜佛求經,且聽下回分解。
