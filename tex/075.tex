
\chapter{心猿鑽透陰陽體 魔王還歸大道真}

卻說孫大聖進於洞口,兩邊觀看。只見:
\begin{quote}
骷髏若嶺,骸骨如林。人頭髮屣成氈片,人皮肉爛作泥塵。人筋纏在樹上,乾焦晃亮如銀。真個是尸山血海,果然腥臭難聞。東邊小妖,將活人拿了剮肉;西下潑魔,把人肉鮮煮鮮烹。若非美猴王如此英雄膽,第二個凡夫也進不得他門。
\end{quote}

不多時,行入二層門裡看時,呀!這裡卻比外面不同:清奇幽雅,秀麗寬平;左右有瑤草仙花,前後有喬松翠竹。又行七八里遠近,才到三層門。閃著身,偷著眼看處,那上面高坐三個老妖,十分獰惡。中間的那個生得:
\begin{quote}
鑿牙鋸齒,圓頭方面。聲吼若雷,眼光如電。仰鼻朝天,赤眉飄焰。但行處,百獸心慌;若坐下,群魔膽戰。這一個是獸中王青毛獅子怪。
\end{quote}

左手下那個生得:
\begin{quote}
鳳目金睛,黃牙粗腿。長鼻銀毛,看頭似尾。圓額皺眉,身軀磊磊。細聲如窈窕佳人,玉面似牛頭惡鬼。這一個是藏齒修身多年的黃牙老象。
\end{quote}

右手下那一個生得:
\begin{quote}
金翅鯤頭,星睛豹眼。振北圖南,剛強勇敢。變生翱翔,鷃笑龍慘。摶風翮百鳥藏頭,舒利爪諸禽喪膽。這個是雲程九萬的大鵬鵰。
\end{quote}

那兩下列著有百十大小頭目,一個個全裝披掛,介冑整齊,威風凜凜,殺氣騰騰。

行者見了,心中歡喜,一些兒不怕,大踏步徑直進門,把梆鈴卸下,朝上叫聲:「大王。」三個老魔笑呵呵問道:「小鑽風,你來了?」行者應聲道:「來了。」「你去巡山,打聽孫行者的下落何如?」行者道:「大王在上,我也不敢說起。」老魔道:「怎麼不敢說?」行者道:「我奉大王命,敲著梆鈴,正然走處,猛擡頭,只看見一個人,蹲在那裡磨杠子,還像個開路神,若站將起來,足有十數丈長短。他就著那澗崖石上,抄一把水,磨一磨,口裡又念一聲,說他那杠子到此還不曾顯個神通,他要磨明,就來打大王。我因此知他是孫行者,特來報知。」

那老魔聞此言,渾身是汗,諕得戰呵呵的道:「兄弟,我說莫惹唐僧。他徒弟神通廣大,預先作了準備,磨棍打我們,卻怎生是好?」教:「小的們,把洞外大小俱叫進來,關了門,讓他過去罷。」那頭目中有知道的報:「大王,門外小妖已都散了。」老魔道:「怎麼都散了?想是聞得風聲不好也。快早關門,快早關門。」眾妖乒乓把前後門盡皆牢拴緊閉。

行者自心驚道:「這一關了門,他再問我家長裡短的事,我對不來,卻不弄走了風,被他拿住?且再諕他一諕,教他開著門,好跑。」又上前道:「大王,他還說得不好。」老魔道:「他又說甚麼?」行者道:「他說拿大大王剝皮,二大王剮骨,三大王抽筋。你們若關了門,不出去啊,他會變化,一時變了個蒼蠅兒,自門縫裡飛進,把我們都拿出去,卻怎生是好?」老魔道:「兄弟們仔細。我這洞裡,遞年家沒個蒼蠅,但是有蒼蠅進來,就是孫行者。」行者暗笑道:「就變個蒼蠅諕他一諕,好開門。」大聖閃在傍邊,伸手去腦後拔了一根毫毛,吹一口仙氣,叫:「變!」即變做一個金蒼蠅,飛去望老魔劈臉撞了一頭。那老怪慌了道:「兄弟,不停當,那話兒進門來了。」驚得那大小群妖,一個個丫鈀掃帚,都上前亂撲蒼蠅。

這大聖忍不住,赥赥的笑出聲來。乾淨他不宜笑,這一笑笑出原嘴臉來了。卻被那第三個老妖魔跳上前,一把扯住道:「哥哥,險些兒被他瞞了。」老魔道:「賢弟,誰瞞誰?」三怪道:「剛才這個回話的小妖不是小鑽風,他就是孫行者。必定撞見小鑽風,不知是他怎麼打殺了,卻變化來哄我們哩。」行者慌了道:「他認得我了。」即把手摸摸,對老怪道:「我怎麼是孫行者?我是小鑽風,大王錯認了。」老魔笑道:「兄弟,他是小鑽風。他一日三次在面前點卯,我認得他。」又問:「你有牌兒麼?」行者道:「有。」擄著衣服,就拿出牌子。老怪一發認實道:「兄弟,莫屈了他。」三怪道:「哥哥,你不曾看見他?他才子閃著身笑了一聲,我見他就露出個雷公嘴來。見我扯住時,他又變作個這等模樣。」叫:「小的們,拿繩來。」眾頭目即取繩索。

三怪把行者扳翻倒,四馬攢蹄綑住。揭起衣裳看時,足足是個弼馬溫。原來行者有七十二般變化,若是變飛禽、走獸、花木、器皿、昆蟲之類,卻就連身子滾去了;但變人物,卻只是頭臉變了,身子變不過來。果然一身黃毛,兩塊紅股,一條尾巴。老妖看著道:「是孫行者的身子,小鑽風的臉皮。是他了。」教:「小的們,先安排酒來,與你三大王遞個得功之杯。既拿倒了孫行者,唐僧坐定是我們口裡食也。」三怪道:「且不要吃酒。孫行者溜撒,他會逃遁之法,只怕走了。教小的們擡出瓶來,把孫行者裝在瓶裡,我們才好吃酒。」

老魔大笑道:「正是,正是。」即點三十六個小妖,入裡面開了庫房門,擡出瓶來。你說那瓶有多大?只得二尺四寸高。怎麼用得三十六個人擡?那瓶乃陰陽二氣之寶,內有七寶八卦、二十四氣,要三十六人,按天罡之數,才擡得動。不一時,將寶瓶擡出,放在三層門外,展得乾淨,揭開蓋。把行者解了繩索,剝了衣服,就著那瓶中仙氣,颼的一聲,吸入裡面,將蓋子蓋上,貼了封皮。卻去吃酒道:「猴兒今番入我寶瓶之中,再莫想那西方之路。若還能夠拜佛求經,除是轉背搖車,再去投胎奪舍是。」你看那大小群妖,一個個笑呵呵,都去賀功不題。

卻說大聖到了瓶中,被那寶貝將身束得小了,索性變化,蹲在當中。半晌,那邊蔭涼,忽失聲笑道:「這妖精外有虛名,內無實事。怎麼告訴人說這瓶裝了人,一時三刻,化為膿血?若似這般涼快,就住上七八年也無事。」咦!大聖原來不知那寶貝根由:假若裝了人,一年不語,一年蔭涼;但聞得人言,就有火來燒了。大聖未曾說完,只見滿瓶都是火焰。幸得他有本事,坐在中間,捻著避火訣,全然不懼。耐到半個時辰,四週圍鑽出四十條蛇來咬。行者掄開手,抓將過來,盡力氣一揝,揝做八十段。少時間,又有三條火龍出來,把行者上下盤遶,著實難禁。自覺慌張無措道:「別事好處,這三條火龍難為。再過一會不出,弄得火氣攻心,怎了?」他想道:「我把身子長一長,券破罷。」好大聖,捻著訣,念聲咒,叫:「長!」即長了丈數高下。那瓶緊靠著身,也就長起去。他把身子往下一小,那瓶兒也就小下來了。行者心驚道:「難難難。怎麼我長他也長,我小他也小?如之奈何?」說不了,孤拐上有些疼痛。急伸手摸摸,卻被火燒軟了。自己心焦道:「怎麼好?孤拐燒軟了,弄做個殘疾之人了。」忍不住吊下淚來。這正是:
\begin{quote}
遭魔遇苦懷三藏,著難臨危慮聖僧。
\end{quote}

道:「師父啊!當年皈正,蒙觀音菩薩勸善,脫離天災。我與你苦歷諸山,收殄多怪,降八戒,得沙僧,千辛萬苦,指望同證西方,共成正果。何期今日遭此毒魔,老孫誤入於此,傾了性命,撇你在半山之中,不能前進。想是我昔日名高,故有今朝之難。」

正此悽愴,忽想起:「菩薩當年在蛇盤山曾賜我三根救命毫毛,不知有無,且等我尋一尋看。」即伸手渾身摸了一把,只見腦後有三根毫毛,十分挺硬。忽喜道:「身上毛都如彼軟熟,只此三根如此硬槍,必然是救我命的。」即便咬著牙,忍著疼,拔下毛,吹口仙氣,叫:「變!」一根即變作金鋼鑽,一根變作竹片,一根變作綿繩。扳張篾片弓兒,牽著那鑽,照瓶底下颼颼的一頓鑽,鑽成一個眼孔,透進光亮。喜道:「造化,造化,卻好出去也。」才變化出身,那瓶復蔭涼了。怎麼就涼?原來被他鑽了,把陰陽之氣泄了,故此遂涼。

好大聖,收了毫毛,將身一小,就變做個蟭蟟蟲兒,十分輕巧,細如鬚髮,長似眉毛,自孔中鑽出且還不走,徑飛在老魔頭上釘著。那老魔正飲酒,猛然放下杯兒道:「三弟,孫行者這回化了麼?」三魔笑道:「還到此時哩?」老魔教傳令擡上瓶來。那下面三十六個小妖即便擡瓶,瓶就輕了許多。慌得眾小妖報道:「大王,瓶輕了。」老魔喝道:「胡說,寶貝乃陰陽二氣之全功,如何輕了?」內中有一個勉強的小妖,把瓶提上來道:「你看這不輕了?」老魔揭蓋看時,只見裡面透亮,忍不住失聲叫道:「這瓶裡空者控也。」大聖在他頭上,也忍不住道一聲「我的兒啊,颼者走也。」眾怪聽見道:「走了,走了。」即傳令:「關門,關門。」

那行者將身一抖,收了剝去的衣服,現本相,跳出洞外,回頭罵道:「妖精不要無禮。瓶子鑽破,裝不得人了,只好拿來出恭。」喜喜歡歡,嚷嚷鬧鬧,踏著雲頭,徑轉唐僧處。那長老正在那裡撮土為香,望空禱祝。行者且停雲頭,聽他禱祝甚的。那長老合掌朝天道:
\begin{quote}
「祈請雲霞眾位仙,六丁六甲與諸天。
願保賢徒孫行者,神通廣大法無邊。」
\end{quote}

大聖聽得這般言語,更加努力,收斂雲光,近前叫道:「師父,我來了。」長老攙住道:「悟空,勞碌。你遠探高山,許久不回,我甚憂慮。端的這山中有何吉凶?」行者笑道:「師父,才這一去,一則是東土眾生有緣有分,二來是師父功德無量無邊,三也虧弟子法力。」將前項妝鑽風、陷瓶裡及脫身之事,細陳了一遍。「今得見尊師之面,實為兩世之人也。」長老感謝不盡道:「你這番不曾與妖精賭鬥麼?」行者道:「不曾。」長老道:「這等保不得我過山了?」行者是個好勝的人,叫喊道:「我怎麼保你過山不得?」長老道:「不曾與他見個勝負,只這般含糊,我怎敢前進。」大聖笑道:「師父,你也忒不通變。常言道:『單絲不線,孤掌難鳴。』那魔三個,小妖千萬,教老孫一人怎生與他賭鬥?」長老道:「寡不敵眾,是你一人也難處。八戒、沙僧他也都有本事,教他們都去,與你協力同心,掃淨山路,保我過去罷。」行者沉吟道:「師言最當。著沙僧保護你,著八戒跟我去罷。」那獃子慌了道:「哥哥沒眼色。我又粗夯,無甚本事,走路扛風,跟你何益?」行者道:「兄弟,你雖無甚本事,好道也是個人。俗云:『放屁添風。』你也可壯我些膽氣。」八戒道:「也罷,也罷,望你帶挈帶挈。但只急溜處,莫捉弄我。」長老道:「八戒在意,我與沙僧在此。」

那獃子抖擻神威,與行者縱著狂風,駕著雲霧,跳上高山,即至洞口。早見那洞門緊閉,四顧無人。行者上前,執鐵棒,厲聲高叫道:「妖怪開門!快出來與老孫打耶。」那洞裡小妖報入。老魔心驚膽戰道:「幾年都說猴兒狠,話不虛傳果是真。」二老怪在傍問道:「哥哥怎麼說?」老魔道:「那行者早間變小鑽風混進來,我等不能相識,幸三賢弟認得,把他裝在瓶裡,他弄本事,鑽破瓶兒,卻又攝去衣服走了。如今在外叫戰,誰敢與他打個頭仗?」更無一人答應。又問,又無人答,都是那裝聾推啞。老魔發怒道:「我等在西方大路上忝著個醜名,今日孫行者這般藐視,若不出去與他見陣,也低了名頭。等我捨了這老性命去與他戰上三合。三合戰得過,唐僧還是我們口裡食;戰不過,那時關了門,讓他過去罷。」遂取披掛結束了,開門前走。

行者與八戒在門傍觀看,真是好一個怪物:
\begin{quote}
鐵額銅頭戴寶盔,盔纓飄舞甚光輝。
輝輝掣電雙睛亮,亮亮鋪霞兩鬢飛。
勾爪如銀尖且利,鋸牙似鑿密還齊。
身披金甲無絲縫,腰束龍絛有見機。
手執鋼刀明晃晃,英雄威武世間稀。
一聲吆喝如雷震,問道「敲門者是誰」?
\end{quote}

大聖轉身道:「是你孫老爺齊天大聖也。」老魔笑道:「你是孫行者?大膽潑猴,我不惹你,你卻為何在此叫戰?」行者道:「『有風方起浪,無潮水自平。』你不惹我,我好尋你?只因你狐群狗黨,結為一夥,算計吃我師父,所以來此施為。」老魔道:「你這等雄糾糾的嚷上我門,莫不是要打麼?」行者道:「正是。」老魔道:「你休猖獗。我若調出妖兵,擺開陣勢,搖旗擂鼓,與你交戰,顯得我是坐家虎,欺負你了。我只與你一個對一個,不許幫丁。」行者聞言,叫:「豬八戒走過,看他把老孫怎的?」那獃子真個閃在一邊。老魔道:「你過來,先與我做個樁兒,讓我盡力氣著光頭砍上三刀,就讓你唐僧過去;假若禁不得,快送你唐僧來,與我做一頓下飯。」行者聞言笑道:「妖怪,你洞裡若有紙筆,取出來,與你立個合同。自今日起,就砍到明年,我也不與你當真。」

那老魔抖擻威風,丁字步站定,雙手舉刀,望大聖劈頂就砍。這大聖把頭往上一迎,只聞扢扠一聲響,頭皮兒紅也不紅。那老魔大驚道:「這猴子好個硬頭兒!」大聖笑道:「你不知。老孫是:
\begin{quote}
生就銅頭鐵腦蓋,天地乾坤世上無。
斧砍鎚敲不得碎,幼年曾入老君爐。
四斗星官監臨造,二十八宿用工夫。
水浸幾番不得壞,周圍扢搭板筋鋪。
唐僧還恐不堅固,預先又上紫金箍。」
\end{quote}

老魔道:「猴兒不要說嘴,看我這二刀來,決不容你性命。」行者道:「左右也只這般砍罷了。」老魔道:「猴兒,你不知這刀:
\begin{quote}
金火爐中造,神功百煉熬。
鋒刃依三略,剛強按六韜。
卻似蒼蠅尾,猶如白蟒腰。
入山雲蕩蕩,下海浪滔滔。
琢磨無遍數,煎熬幾百遭。
深山古洞放,上陣有功勞。
攙著你這和尚天靈蓋,一削就是兩個瓢。」
\end{quote}

大聖笑道:「這妖精沒眼色,把老孫認做個瓢頭哩。也罷,誤砍誤讓,教你再砍一刀看怎麼。」

那老魔舉刀又砍,大聖把頭迎一迎,乒乓的劈做兩半個。大聖就地打個滾,變做兩個身子。那妖一見慌了,手按下鋼刀。豬八戒遠遠望見,笑道:「老魔好砍兩刀的,卻不是四個人了?」老魔指定行者道:「聞你能使分身法,怎麼把這法兒拿出在我面前使?」大聖道:「何為分身法?」老魔道:「為甚麼先砍你一刀不動,如今砍你一刀,就是兩個人?」大聖笑道:「妖怪,你切莫害怕。砍上一萬刀,還你二萬個人。」老魔道:「你這猴兒,你只會分身,不會收身。你若有本事收做一個,打我一棍去罷。」大聖道:「不許說謊,你要砍三刀,只砍了我兩刀。教我打一棍,若打了棍半,就不姓孫。」老魔道:「正是,正是。」

好大聖,就把身摟上來,打個滾,依然一個身子,掣棒劈頭就打。那老魔舉刀架住道:「潑猴無禮,甚麼樣個哭喪棒,敢上門打人?」大聖喝道:「你若問我這條棍,天上地下都有名聲。」老魔道:「怎見名聲?」他道:
\begin{quote}
「棒是九轉鑌鐵煉,老君親手爐中煅。
禹王求得號神珍,四海八河為定驗。
中間星斗暗鋪陳,兩頭箝裹黃金片。
花紋密佈鬼神驚,上造龍紋與鳳篆。
名號靈陽棒一條,深藏海藏人難見。
成形變化要飛騰,飄颻五色霞光現。
老孫得道取歸山,無窮變化多經驗。
時間要大甕來粗,或小些微如鐵線。
粗如南岳細如針,長短隨吾心意變。
輕輕舉動彩雲生,亮亮飛騰如閃電。
攸攸冷氣逼人寒,條條殺霧空中現。
降龍伏虎謹隨身,天涯海角都遊遍。
曾將此棍鬧天宮,威風打散蟠桃宴。
天王賭鬥未曾贏,哪吒對敵難交戰。
棍打諸神沒躲藏,天兵十萬都逃竄。
雷霆眾將護靈霄,飛身打上通明殿。
掌朝天使盡皆忙,護駕仙卿俱攪亂。
舉棒掀翻北斗宮,回首振開南極院。
金闕天皇見棍兇,特請如來與我見。
兵家勝敗自如然,困苦災危無可辨。
整整挨排五百年,虧了南海菩薩勸。
大唐有個出家僧,對天發下洪誓願。
枉死城中度鬼魂,靈山會上求經卷。
西方一路有妖魔,行動甚是不方便。
已知鐵棒世無雙,央我途中為侶伴。
邪魔湯著赴幽冥,肉化紅塵骨化麵。
處處妖精棒下亡,論萬成千無打算。
上方擊壞斗牛宮,下方壓損森羅殿。
天將曾將九曜追,地府打傷催命判。
半空丟下振山川,勝如太歲新華劍。
全憑此棍保唐僧,天下妖魔都打遍。」
\end{quote}

那魔聞言,戰兢兢捨著性命,舉刀就砍;猴王笑吟吟使鐵棒前迎。他兩個先時在洞前撐持,然後跳起去,都在半空裡廝殺。這一場好殺:
\begin{quote}
天河定底神珍棒,棒名如意世間高。誇稱手段魔頭惱,大捍刀擎法力豪。門外爭持還可近,空中賭鬥怎相饒。一個隨心更面目,一個立地長身腰。殺得滿天雲氣重,遍野霧飄颻。那一個幾番立意吃三藏,這一個廣施法力保唐朝。都因佛祖傳經典,邪正分明恨苦交。
\end{quote}

那老魔與大聖鬥經二十餘合,不分輸贏。

原來八戒在底下見他兩個戰到好處,忍不住掣鈀架風,跳將起去,望妖魔劈臉就築。那魔慌了。不知八戒是個虎頭性子,冒冒失失的諕人。他只道嘴長耳大,手硬鈀兇,敗了陣,丟了刀,回頭就走。大聖喝道:「趕上,趕上。」這獃子仗著威風,舉著釘鈀,即忙趕下怪去。老魔見他趕的相近,在坡前立定,迎著風頭,幌一幌現了原身,張開大口,就要來吞八戒。八戒害怕,急抽身往草裡一鑽,也管不得荊針棘刺,也顧不得刮破頭疼,戰兢兢的在草裡聽著梆聲。隨後行者趕到,那怪也張口來吞,卻中了他的機關,收了鐵棒,迎將上去,被老魔一口吞之。諕得個獃子在草裡囊囊咄咄的埋怨道:「這個弼馬溫,不識進退。那怪來吃你,你如何不走,反去迎他!這一口吞在肚中,今日還是個和尚,明日就是個大恭也。」那魔得勝而去,這獃子才鑽出草來,溜回舊路。

卻說三藏在那山坡下正與沙僧盼望,只見八戒喘呵呵的跑來。三藏大驚道:「八戒,你怎麼這等狼狽?悟空如何不見?」獃子哭哭啼啼道:「師兄被妖精一口吞下肚去了。」三藏聽言,諕倒在地,半晌間跌腳拳胸道:「徒弟呀,只說你善會降妖,領我西天見佛,怎知今日死於此怪之手。苦哉,苦哉!我弟子同眾的功勞,如今都化作塵土矣。」那師父十分苦痛。你看那獃子,他也不來勸解師父,卻叫:「沙和尚,你拿將行李來,我兩個分了罷。」沙僧道:「二哥,分怎的?」八戒道:「分開了,各人散火:你往流沙河,還去吃人;我往高老莊,看看我渾家。將白馬賣了,與師父買個壽器送終。」長老氣呼呼的聞得此言,叫皇天,放聲大哭。且不題。

卻說那老魔吞了行者,以為得計,徑回本洞,眾妖迎問出戰之功。老魔道:「拿了一個來了。」二魔喜道:「哥哥拿的是誰?」老魔道:「是孫行者。」二魔道:「拿在何處?」老魔道:「被我一口吞在腹中哩。」第三個魔頭大驚道:「大哥啊,我就不曾吩咐你,孫行者不中吃。」那大聖在肚裡道:「忒中吃,又禁饑,再不得餓。」慌得那小妖道:「大王,不好了,孫行者在你肚裡說話哩。」老魔道:「怕他說話,有本事吃了他,沒本事擺佈他不成?你們快去燒些鹽白湯,等我灌下肚去,把他噦出來,慢慢的煎了吃酒。」小妖真個沖了半盆鹽湯。老怪一飲而乾,洼著口,著實一嘔;那大聖在肚裡生了根,動也不動。卻又攔著喉嚨,往外又吐,吐得頭暈眼花,黃膽都破了;行者越發不動。老魔喘息了,叫聲:「孫行者,你不出來?」行者道:「早哩,正好不出來哩!」老魔道:「你怎麼不出?」行者道:「你這妖精甚不通變。我自做和尚,十分淡薄,如今秋涼,我還穿個單直裰。這肚裡倒暖,又不透風,等我住過冬才好出來。」

眾妖聽說,都道:「大王,孫行者要在你肚裡過冬哩。」老魔道:「他要過冬,我就打起禪來,使個搬運法,一冬不吃飯,就餓殺那弼馬溫。」大聖道:「我兒子,你不知事。老孫保唐僧取經,從廣裡過,帶了個摺疊鍋兒,進來煮雜碎吃。將你這裡邊的肝、腸、肚、肺,細細兒受用,還夠盤纏到清明哩。」那二魔大驚道:「哥啊,這猴子他幹得出來。」三魔道:「哥啊,吃了雜碎也罷,不知在那裡支鍋?」行者道:「三叉骨上好支鍋。」三魔道:「不好了,假若支起鍋,燒動火煙,煼到鼻孔裡,打嚏噴麼?」行者笑道:「沒事,等老孫把金箍棒往頂門裡一搠,搠個窟窿:一則當天窗,二來當煙洞。」

老魔聽說,雖說不怕,卻也心驚,只得硬著膽叫:「兄弟們,莫怕。把我那藥酒拿來,等我吃幾鍾下去,把猴兒藥殺了罷。」行者暗笑道:「老孫五百年前大鬧天宮時,吃老君丹、玉皇酒、王母桃及鳳髓龍肝,那樣東西我不曾吃過?是甚麼藥酒,敢來藥我?」那小妖真個將藥酒篩了兩壺,滿滿斟了一鍾,遞與老魔。老魔接在手中,大聖在肚裡就聞得酒香,道:「不要與他吃。」好大聖,把頭一扭,變做個喇叭口子,張在他喉嚨之下。那怪嘓的嚥下,被行者嘓的接吃了。第二鍾嚥下,被行者嘓的又接吃了。一連吃了七八鍾,都是他接吃了。老魔放下鍾道:「不吃了。這酒常時吃兩鍾,腹中如火;卻才吃了七八鍾,臉上紅也不紅!」

原來這大聖吃不多酒,接了他七八鍾吃了,在肚裡撒起酒風來:不住的支架子、跌四平,、踢飛腳、抓住肝花打鞦韆、豎蜻蜓、翻根頭、亂舞。那怪物疼痛難禁,倒在地下。

畢竟不知死活如何,且聽下回分解。
