
\chapter{唐三藏路阻火焰山 孫行者一調芭蕉扇}

\begin{quote}
若干種性本來同,海納無窮。千思萬慮終成妄,般般色色和融。有日功完行滿,圓明法性高隆。休教差別走西東,緊鎖牢䪊。收來安放丹爐內,煉得金烏一樣紅。朗朗輝輝嬌艷,任教出入乘龍。
\end{quote}

話表三藏遵菩薩教旨,收了行者,與八戒、沙僧剪斷二心,鎖䪊猿馬,同心戮力,趕奔西天。說不盡光陰似箭,日月如梭。歷過了夏月炎天,卻又值三秋霜景。但見那:
\begin{quote}
薄雲斷絕西風緊,鶴鳴遠岫霜林錦。光景正蒼涼,山長水更長。征鴻來北塞,玄鳥歸南陌。客路怯孤單,衲衣容易寒。
\end{quote}

師徒四眾進前行處,漸覺熱氣蒸人。三藏勒馬道:「如今正是秋天,卻怎返有熱氣?」八戒道:「原來不知。西方路上有個斯哈哩國,乃日落之處,俗呼為天盡頭。若到申酉時,國王差人上城,擂鼓吹角,混雜海沸之聲。日乃太陽真火,落於西海之間,如火淬水,接聲滾沸。若無鼓角之聲混耳,即振殺城中小兒。此地熱氣蒸人,想必到日落之處也。」大聖聽說,忍不住笑道:「獃子莫亂談。若論斯哈哩國,正好早哩。似師父朝三暮二的這等擔閣,就從小至老,老了又小,老小三生,也還不到。」八戒道:「哥啊!據你說,不是日落之處,為何這等酷熱?」沙僧道:「想是天時不正,秋行夏令故也。」他三個正都爭講,只見那路傍有座莊院,乃是紅瓦蓋的房舍,紅磚砌的垣牆,紅油門扇,紅漆板榻,一片都是紅的。三藏下馬道:「悟空,你去那人家問個消息,看那炎熱之故何也?」

大聖收了金箍棒,整肅衣裳,扭捏作個斯文氣象,綽下大路,徑至門前觀看。那門裡忽然走出一個老者,但見他:
\begin{quote}
穿一領黃不黃紅不紅的葛布深衣,戴一頂青不青皂不皂的篾絲涼帽。手中拄一根彎不彎直不直暴節竹杖,足下踏一雙新不新舊不舊靸䩺鞋。面似紅銅,鬚如白練。兩道壽眉遮碧眼,一張咍口露金牙。
\end{quote}

那老者猛擡頭,看見行者,吃了一驚,拄著竹杖,喝道:「你是那裡來的怪人?在我這門首何幹?」行者答禮道:「老施主休怕我,我不是甚麼怪人。貧僧是東土大唐欽差上西方求經者。師徒四人,適至寶方,見天氣蒸熱,一則不解其故,二來不知地名,特拜問指教一二。」那老者卻才放心,笑云:「長老勿罪。我老漢一時眼花,不識尊顏。」行者道:「不敢。」老者又問:「令師在那條路上?」行者道:「那南首大路上立的不是?」老者教:「請來,請來。」行者歡喜,把手一招,三藏即同八戒、沙僧,牽白馬,挑行李近前,都對老者作禮。

老者見三藏丰姿標致,八戒、沙僧相貌奇稀,又驚又喜。只得請入裡坐,教小的們看茶,一壁廂辦飯。三藏聞言,起身稱謝道:「敢問公公:貴處遇秋,何返炎熱?」老者道:「敝地喚做火焰山,無春無秋,四季皆熱。」三藏道:「火焰山卻在那邊?可阻西去之路?」老者道:「西方卻去不得。那山離此有六十里遠,正是西方必由之路,卻有八百里火焰,四週圍寸草不生。若過得山,就是銅腦蓋,鐵身軀,也要化成汁哩。」三藏聞言,大驚失色,不敢再問。

只見門外一個少年男子,推一輛紅車兒,住在門傍,叫聲:「賣糕。」大聖拔根毫毛,變個銅錢,問那人買糕。那人接了錢,不論好歹,揭開車兒上衣裹,熱氣騰騰,拿出一塊糕,遞與行者。行者托在手中,好似火裡燒的灼炭,煤爐內的紅釘。你看他左手倒在右手,右手換在左手,只道:「熱熱熱,難吃難吃!」那男子笑道:「怕熱,莫來這裡,這裡是這等熱。」行者道:「你這漢子好不明理。常言道:『不冷不熱,五穀不結。』他這等熱得很,你這糕粉自何而來?」那人道:「若知糕粉米,敬求鐵扇仙。」行者道:「鐵扇仙怎的?」那人道:「鐵扇仙有柄芭蕉扇,求得來,一扇息火,二扇生風,三扇下雨。我們就布種,及時收割,故得五穀養生;不然,誠寸草不能生也。」

行者聞言,急抽身走入裡面,將糕遞與三藏道:「師父放心,且莫隔年焦著。吃了糕,我與你說。」長老接糕在手,向本宅老者道:「公公請糕。」老者道:「我家的茶飯未奉,敢吃你糕?」行者笑道:「老人家,茶飯倒不必賜,我問你:鐵扇仙在那裡住?」老者道:「你問他怎的?」行者道:「適才那賣糕人說,此仙有柄芭蕉扇。求將來,一扇息火,二扇生風,三扇下雨,你這方布種收割,才得五穀養生。我欲尋他討來搧息火燄山過去,且使這方依時收種,得安生也。」老者道:「固有此說,你們卻無禮物,恐那聖賢不肯來也。」三藏道:「他要甚禮物?」老者道:「我這裡人家,十年拜求一度。四豬四羊、花紅表裡、異香時果、雞鵝美酒,沐浴虔誠,拜到那仙山,請他出洞,至此施為。」行者道:「那山坐落何處?喚甚地名?有幾多里數?等我問他要扇子去。」老者道:「那山在西南方,名喚翠雲山。山中有一仙洞,名喚芭蕉洞。我這裡眾信人等去拜仙山,往回要走一月,計有一千四百五六十里。」行者笑道:「不打緊,就去就來。」那老者道:「且住,吃些茶飯,辦些乾糧,須得兩人做伴。那路上沒有人家,又多狼虎,非一日可到,莫當耍子。」行者笑道:「不用,不用。我去也。」說一聲,忽然不見。那老者慌張道:「爺爺呀!原來是騰雲駕霧的神人也。」

且不說這家子供奉唐僧加倍。卻說那行者霎時徑到翠雲山,按住祥光,正自找尋洞口,只聞得丁丁之聲,乃是山林內一個樵夫伐木。行者即趨步至前,又聞得他道:
\begin{quote}
「雲際依依認舊林,斷崖荒草路難尋。
西山望見朝來雨,南澗歸時渡處深。」
\end{quote}

行者近前作禮道:「樵哥,問訊了。」那樵子撇了柯斧,答禮道:「長老何往?」行者道:「敢問樵哥,這可是翠雲山?」樵子道:「正是。」行者道:「有個鐵扇仙的芭蕉洞,在何處?」樵子笑道:「這芭蕉洞雖有,卻無個鐵扇仙,只有個鐵扇公主,又名羅剎女。」行者道:「人言他有一柄芭蕉扇,能熄得火焰山,敢是他麼?」樵子道:「正是,正是。這聖賢有這件寶貝,善能熄火,保護那方人家,故此稱為鐵扇仙。我這裡人家用不著他,只知他叫做羅剎女,乃大力牛魔王妻也。」

行者聞言,大驚失色,心中暗想道:「又是冤家了。當年伏了紅孩兒,說是這廝養的。前在那解陽山破兒洞遇他叔子,尚且不肯與水,要作報仇之意;今又遇他父母,怎生借得這扇子耶?」樵子見行者沉思默慮,嗟嘆不已,便笑道:「長老,你出家人,有何憂疑?這條小路兒向東去,不尚五六里,就是芭蕉洞,休得心焦。」行者道:「不瞞樵哥說,我是東土唐朝差往西天求經的唐僧大徒弟,前年在火雲洞,曾與羅剎之子紅孩兒有些言語,但恐羅剎懷仇不與,故生憂疑。」樵子道:「大丈夫鑒貌辨色,只以求扇為名,莫認往時之溲話,管情借得。」行者聞言,深深唱個大喏道:「謝樵哥教誨,我去也。」

遂別了樵夫,徑至芭蕉洞口。但見那兩扇門緊閉牢關,洞外風光秀麗。好去處!正是那:
\begin{quote}
山以石為骨,石作土之精。煙霞含宿潤,苔蘚助新青。嵯峨勢聳欺蓬島,幽靜花香若海瀛。幾樹喬松棲野鶴,數株衰柳語山鶯。誠然是千年古跡,萬載仙蹤。碧梧鳴彩鳳,活水隱蒼龍。曲逕蓽蘿垂掛,石梯藤葛攀籠。猿嘯翠巖忻月上,鳥啼高樹喜晴空。兩林竹蔭涼如雨,一逕花濃沒繡絨。時見白雲來遠岫,略無定體漫隨風。
\end{quote}

行者上前叫:「牛大哥,開門,開門。」呀的一聲,洞門開了,裡邊走出一個毛兒女,手中提著花籃,肩上擔著鋤子。真個是一身藍縷無妝飾,滿面精神有道心。行者上前迎著合掌道:「女童,累你轉報公主一聲:我本是取經的和尚,在西方路上,難過火焰山,特來拜借芭蕉扇一用。」那毛女道:「你是那寺裡和尚?叫甚名字?我好與你通報。」行者道:「我是東土來的,叫做孫悟空和尚。」

那毛女即便回身,轉於洞內,對羅剎跪下道:「奶奶,洞門外有個東土來的孫悟空和尚,要見奶奶,拜求芭蕉扇,過火焰山一用。」那羅剎聽見「孫悟空」三字,便似撮鹽入火,火上澆油,骨都都紅生臉上,惡狠狠怒發心頭。口中罵道:「這潑猴!今日來了。」叫:「丫鬟,取披掛,拿兵器來。」隨即取了披掛,拿兩口青鋒寶劍,整束出來。行者在洞外閃過,偷看怎生打扮。只見他:
\begin{quote}
頭裹團花手帕,身穿納錦雲袍。
腰間雙束虎觔絛。微露繡裙偏綃。
鳳嘴弓鞋三寸,龍鬚膝褲金銷。
手提寶劍怒聲高。兇比月婆容貌。
\end{quote}

那羅剎出門,高叫道:「孫悟空何在?」行者上前,躬身施禮道:「嫂嫂,老孫在此奉揖。」羅剎咄的一聲道:「誰是你的嫂嫂?那個要你奉揖?」行者道:「尊府牛魔王,當初曾與老孫結義,乃七兄弟之親。今聞公主是牛大哥令正,安得不以嫂嫂稱之?」羅剎道:「你這潑猴!既有兄弟之親,如何坑陷我子?」行者佯問道:「令郎是誰?」羅剎道:「我兒是號山枯松澗火雲洞聖嬰大王紅孩兒。被你傾了,我們正沒處尋你報仇,你今上門納命,我肯饒你?」行者滿臉陪笑道:「嫂嫂原來不察理,錯怪了老孫。你令郎因是捉了師父,要蒸要煮,幸虧了觀音菩薩收他去,救出我師。他如今現在菩薩處做善財童子,實受了菩薩正果,不生不滅,不垢不淨,與天地同壽,日月同庚。你倒不謝老孫保命之恩,返怪老孫,是何道理?」羅剎道:「你這個巧嘴的潑猴!我那兒雖不傷命,再怎生得到我的跟前,幾時能見一面?」行者笑道:「嫂嫂要見令郎,有何難處?你且把扇子借我,搧息了火,送我師父過去,我就到南海菩薩處請他來見你,就送扇子還你,有何不可?那時節,你看他可曾損傷一毫?如有些須之傷,你也怪得有理;如比舊時標致,還當謝我。」羅剎道:「潑猴!少要饒舌,伸過頭來,等我砍上幾劍:若受得疼痛,就借扇子與你;若忍耐不得,教你早見閻君。」行者叉手向前,笑道:「嫂嫂切莫多言。老孫伸著光頭,任尊意砍上多少,但沒氣力便罷。是必借扇子用用。」那羅剎不容分說,雙手掄劍,照行者頭上乒乒乓乓,砍有十數下,這行者全不認真。羅剎害怕,回頭要走。行者道:「嫂嫂那裡去?快借我使使。」那羅剎道:「我的寶貝原不輕借。」行者道:「既不肯借,吃你老叔一棒。」

好猴王,一隻手扯住,一隻手去耳內掣出棒來,幌一幌,有碗來粗細。那羅剎掙脫手,舉劍來迎。行者隨又掄棒便打。兩個在翠雲山前,不論親情,卻只講仇隙。這一場好殺:
\begin{quote}
裙釵本是修成怪,為子懷仇恨潑猴。行者雖然生狠怒,因師路阻讓娥流。先言拜借芭蕉扇,不展驍雄耐性柔。羅剎無知掄劍砍,猴王有意說親由。女流怎與男兒鬥,到底男剛壓女流。這個金箍鐵棒多兇猛,那個霜刃青鋒甚緊稠。劈面打,照頭丟,恨苦相持不罷休。左擋右遮施武藝,前迎後架騁奇謀。卻才鬥到沉酣處,不覺西方墜日頭。羅剎忙將真扇子,一搧揮動鬼神愁。
\end{quote}

那羅剎女與行者相持到晚,見行者棒重,卻又解數周密,料鬥他不過,即便取出芭蕉扇,幌一幌,一扇陰風,把行者搧得無影無形,莫想收留得住。這羅剎得勝回歸。

那大聖飄飄蕩蕩,左沉不能落地,右墜不得存身。就如旋風翻敗葉,流水淌殘花。滾了一夜,直至天明,方才落在一座山上,雙手抱住一塊峰石。定性良久,仔細觀看,卻才認得是小須彌山。大聖長嘆一聲道:「好利害婦人!怎麼就把老孫送到這裡來了?我當年曾記得在此處告求靈吉菩薩降黃風怪救我師父。那黃風嶺至此直南上有三千餘里,今在西路轉來,乃東南方隅,不知有幾萬里。等我下去問靈吉菩薩一個消息,好回舊路。」

正躊躇間,又聽得鐘聲響亮,急下山坡,徑至禪院。那門前道人認得行者的形容,即入裡面報道:「前年來請菩薩去降黃風怪的那個毛臉大聖又來了。」菩薩知是悟空,連忙下寶座相迎,入內施禮道:「恭喜,取經來耶?」悟空答道:「正好未到,早哩,早哩。」靈吉道:「既未曾得到雷音,何以回顧荒山?」行者道:「自上年蒙盛情降了黃風怪,一路上不知歷過多少苦楚。今到火焰山,不能前進,詢問土人,說有個鐵扇仙,芭蕉扇搧得火滅,老孫特去尋訪。原來那仙是牛魔王的妻,紅孩兒的母。他說我把他兒子做了觀音菩薩的童子,不得常見,恨我為仇,不肯借扇,與我爭鬥。他見我的棒重難撐,遂將扇子把我一搧,搧得我悠悠蕩蕩,直至於此,方才落住。故此輕造禪院,問個歸路。此處到火焰山,不知有多少里數?」靈吉笑道:「那婦人喚名羅剎女,又叫做鐵扇公主。他的那芭蕉扇本是崑崙山後,自混沌開闢以來,天地產成的一個靈寶,乃太陰之精葉,故能滅火氣。假若搧著人,要飄八萬四千里,方息陰風。我這山到火焰山,只有五萬餘里,此還是大聖有留雲之能,故止住了;若是凡人,正好不得住也。」行者道:「利害,利害!我師父卻怎生得度那方?」靈吉道:「大聖放心。此一來,也是唐僧的緣法,合教大聖成功。」行者道:「怎見成功?」靈吉道:「我當年受如來教旨,賜我一粒定風丹、一柄飛龍杖。飛龍杖已降了風魔。這定風丹尚未曾見用,如今送了大聖,管教那廝搧你不動,你卻要了扇子,搧息火,卻不就立此功也?」行者低頭作禮,感謝不盡。那菩薩即於衣袖中取出一個錦袋兒,將那一粒定風丹,與行者安在衣領裡邊,將針線緊緊縫了。送行者出門道:「不及留款。往西北上去,就是羅剎的山場也。」

行者辭了靈吉,駕觔斗雲,徑返翠雲山,頃刻而至。使鐵棒打著洞門叫道:「開門,開門!老孫來借扇子使使哩。」慌得那門裡女童即忙來報:「奶奶,借扇子的又來了。」羅剎聞言,心中悚懼道:「這潑猴真有本事。我的寶貝搧著人,要去八萬四千里,方能停止;他怎麼才吹去,就回來也?這番等我一連搧他兩三扇,教他找不著歸路。」急縱身,結束整齊,雙手提劍,走出門來道:「孫行者,你不怕我,又來尋死?」行者笑道:「嫂嫂勿得慳吝,是必借我使使,保得唐僧過山,就送還你。我是個志誠有餘的君子,不是那借物不還的小人。」羅剎又罵道:「潑猢猻!好沒道理,沒分曉。奪子之仇,尚未報得;借扇之意,豈得如心?你不要走,吃我老娘一劍。」大聖公然不懼,使鐵棒劈手相迎。他兩個往往來來,戰經五七回合,羅剎女手軟難掄,孫行者身強善敵。他見事勢不諧,即取扇子,望行者搧了一扇,行者巍然不動。行者收了鐵棒,笑吟吟的道:「這番不比那番,任你怎麼搧來,老孫若動一動,就不算漢子。」那羅剎又搧兩扇。果然不動。羅剎慌了,急收寶貝轉回,走入洞裡,將門緊緊關上。

行者見他閉了門,卻就弄個手段,拆開衣領,把定風丹噙在口中。搖身一變,變作一個蟭蟟蟲兒,從他門隙處鑽進。只見羅剎叫道:「渴了,渴了,快拿茶來。」近侍女童,即將香茶一壺,沙沙的滿斟一碗,沖起茶沫漕漕。行者見了歡喜,嚶的一翅,飛在茶沫之下。那羅剎渴極,接過茶,兩三氣都吃了。行者已到他肚腹之內,現原身,厲聲高叫道:「嫂嫂,借扇子我使使。」羅剎大驚失色,叫:「小的們,關了前門否?」俱說:「關了。」他又說:「既關了門,孫行者如何在家裡叫喚?」女童道:「在你身上叫哩。」羅剎道:「孫行者,你在那裡弄術哩?」行者道:「老孫一生不會弄術,都是些真手段,實本事,已在尊嫂尊腹之內耍子,已見其肺肝矣。我知你也饑渴了,我先送你個坐碗兒解渴。」卻就把腳往下一登。那羅剎小腹之中疼痛難禁,坐於地下叫苦。行者道:「嫂嫂休得推辭,我再送你個點心充饑。」又把頭往上一頂。那羅剎心痛難禁,只在地上打滾,疼得他面黃唇白,只叫:「孫叔叔饒命。」

行者卻才收了手腳道:「你才認得叔叔麼?我看牛大哥情上,且饒你性命。快將扇子拿來我使使。」羅剎道:「叔叔,有扇,有扇,你出來拿了去。」行者道:「拿扇子我看了出來。」羅剎即叫女童拿一柄芭蕉扇,執在傍邊。行者探到喉嚨之上見了道:「嫂嫂,我既饒你性命,不在腰肋之下搠個窟窿出來,還自口出。你把口張三張兒。」那羅剎果張開口。行者還作個蟭蟟蟲,先飛出來,丁在芭蕉扇上。那羅剎不知,連張三次,叫:「叔叔出來罷。」行者化原身,拿了扇子,叫道:「我在此間不是?謝借了,謝借了。」拽開步,往前便走。小的們連忙開了門,放他出洞。

這大聖撥轉雲頭,徑回東路,霎時按落雲頭,立在紅磚壁下。八戒見了,歡喜道:「師父,師兄來了,來了。」三藏即與本莊老者同沙僧出門接著,同至舍內。把芭蕉扇靠在旁邊道:「老官兒,可是這個扇子?」老者道:「正是,正是。」唐僧喜道:「賢徒有莫大之功。求此寶貝,甚勞苦了。」行者道:「勞苦倒也不說。那鐵扇仙,你道是誰?那廝原來是牛魔王的妻,紅孩兒的母,名喚羅剎女,又喚鐵扇公主。我尋到洞外借扇,他就與我講起仇隙,把我砍了幾劍。是我使棒嚇他,他就把扇子搧了我一下,飄飄蕩蕩,直刮到小須彌山。幸見靈吉菩薩,送了我一粒定風丹,指與歸路。復至翠雲山,又見羅剎女。羅剎女又使扇子,搧我不動,他就回洞。是老孫變作一個蟭蟟蟲,飛入洞去。那廝正討茶吃,是我又鑽在茶沫之下,到他肚裡,做起手腳。他疼痛難禁,不住口的叫我做叔叔饒命,情願將扇借與我。我卻饒了他,拿將扇來。待過了火焰山,仍送還他。」三藏聞言,感謝不盡。

師徒們俱拜辭老者,一路西來。約行有四十里遠近,漸漸酷熱蒸人。沙僧只叫:「腳底烙得慌。」八戒又道:「爪子燙得痛。」馬比尋常又快,只因他熱難停,十分躁進。行者道:「師父且請下馬,兄弟們莫走。等我搧息了火,待風雨之後,地土冷些,再過山去。」行者果舉扇,徑至火邊,盡力一搧,那山上火光烘烘騰起;再一扇,更著百倍;又一扇,那火足有千丈之高,漸漸燒著身體。行者急回,已將兩股毫毛燒淨。徑跑至唐僧面前叫:「快回去,快回去。火來了,火來了。」

那師父爬上馬,與八戒、沙僧,復東來有二十餘里,方才歇下,道:「悟空,如何了呀?」行者丟下扇子道:「不停當,不停當,被那廝哄了。」三藏聽說,愁促眉尖,悶添心上,止不住兩淚交流,只道:「怎生是好?」八戒道:「哥哥,你急急忙忙叫回去是怎麼說?」行者道:「我將扇子搧了一下,火光烘烘;第二扇,火氣愈盛;第三扇,火頭飛有千丈之高。若是跑得不快,把毫毛都燒盡矣。」八戒笑道:「你常說雷打不傷,火燒不損,如今何又怕火?」行者道:「你這獃子,全不知事。那時節用心防備,故此不傷;今日只為搧息火光,不曾捻避火訣,又未使護身法,所以把兩股毫毛燒了。」沙僧道:「似這般火盛,無路通西,怎生是好?」八戒道:「只揀無火處走便罷。」三藏道:「那方無火?」八戒道:「東方、南方、北方俱無火。」又問:「那方有經?」八戒道:「西方有經。」三藏道:「我只欲往有經處去哩。」沙僧道:「有經處有火,無火處無經,誠是進退兩難。」

師徒們正自胡談亂講,只聽得有人叫道:「大聖不須煩惱,且來吃些齋飯再議。」四眾回看時,見一老人,身披飄風氅,頭頂偃月冠,手持龍頭杖,足踏鐵靿靴。後帶著一個雕嘴魚腮鬼,鬼頭上頂著一個銅盆,盆內有些蒸餅糕糜、黃糧米飯。在於西路下躬身道:「我本是火焰山土地,知大聖保護聖僧,不能前進,特獻一齋。」行者道:「吃齋小可,這火光幾時滅得,讓我師父過去?」土地道:「要滅火光,須求羅剎女借芭蕉扇。」行者去路旁拾起扇子道:「這不是?那火光越搧越著,何也?」土地看了,笑道:「此扇不是真的,被他哄了。」行者道:「如何方得真的?」那土地又控背躬身,微微笑道:
\begin{quote}
若還要借真蕉扇,須是尋求大力王。
\end{quote}

畢竟不知大力王有甚緣故,且聽下回分解。
