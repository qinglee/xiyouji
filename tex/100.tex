
\chapter{徑回東土 五聖成真}

且不言他四眾脫身,隨金剛駕風而起。卻說陳家莊救生寺內多人,天曉起來,仍治果餚來獻,至樓下,不見了唐僧。這個也來問,那個也來尋,俱慌慌張張,莫知所措,叫苦連天的道:「清清把個活佛放去了。」一會家無計,將辦來的品物,俱擡在樓上,祭祀燒紙。以後每年四大祭,二十四小祭。還有那告病的,保安的,求親許願、求財求子的,無時無日,不來燒香祭賽。真個是金爐不斷千年火,玉盞常明萬載燈。不題。

卻說八大金剛使第二陣香風,把他四眾,不一日送至東土,漸漸望見長安。原來那太宗自貞觀十三年九月望前三日送唐僧出城,至十六年,即差工部官在西安關外起建了望經樓接經。太宗年年親至其地。恰好那一日出駕復到樓上,忽見正西方滿天瑞靄,陣陣香風。金剛停在空中叫道:「聖僧,此間乃長安城了。我們不好下去,這裡人伶俐,恐泄漏吾像。孫大聖三位也不消去。汝自去傳了經與汝主,即便回來,我在霄漢中等你,與你一同繳旨。」大聖道:「尊者之言雖當,但吾師如何挑得經擔?如何牽得這馬?須得我等同去一送。煩你在空少等,諒不敢誤。」金剛道:「前日觀音菩薩啟過如來,往來只在八日,方完藏數。今已經四日有餘,只怕八戒貪圖富貴,誤了期限。」八戒笑道:「師父成佛,我也望成佛,豈有貪圖之理?潑大粗人?都在此等我,待交了經,就來與你回向也。」獃子挑著擔,沙僧牽著馬,行者領著聖僧,都按下雲頭,落於望經樓邊。

太宗同多官一齊見了,即下樓相迎道:「御弟來也?」唐僧即倒身下拜。太宗攙起,又問:「此三者何人?」唐僧道:「是途中收的徒弟。」太宗大喜,即命侍官:「將朕御車馬扣背,請御弟上馬,同朕回朝。」唐僧謝了恩,騎上馬。大聖輪金箍棒緊隨,八戒、沙僧俱扶馬、挑擔,隨駕後共入長安。真個是:
\begin{quote}
當年清宴樂昇平,文武安然顯俊英。
水陸場中僧演法,金鑾殿上主差卿。
關文敕賜唐三藏,經卷原因配五行。
苦煉兇魔種種滅,功成今喜上朝京。
\end{quote}

唐僧四眾隨駕入朝,滿城中無一不知是取經人來了。

卻說那長安唐僧舊住的洪福寺大小僧人,看見幾株松樹一顆顆頭俱向東,驚訝道:「怪哉!怪哉!今夜未曾刮風,如何這樹頭都扭過來了?」內有三藏的舊徒道:「快拿衣服來,取經的老師父來了。」眾僧問道:「你何以知之?」舊徒曰:「當年師父去時,曾有言道:『我去之後,或三五年,或六七年,但看松樹枝頭若是東向,我即回矣。』我師父佛口聖言,故此知之。」急披衣而出。至西街時,早已有人傳播說:「取經的人適才方到,萬歲爺爺接入城來了。」眾僧聽說,又急急跑來,卻就遇著。一見大駕,不敢近前,隨後跟至朝門之外。

唐僧下馬,同眾進朝。唐僧將龍馬與經擔,同行者、八戒、沙僧,站在玉階之下。太宗傳宣御弟上殿,賜坐。唐僧又謝恩坐了,教把經卷擡來。行者等取出,近侍官傳上。太宗又問:「多少經數?怎生取來?」三藏道:「臣僧到了靈山,參見佛祖,蒙差阿儺、伽葉二尊者先引至珍樓內賜齋,次到寶閣內傳經。那尊者需索人事,因未曾備得,不曾送他,他遂以經與了。當謝佛祖之恩,東行,忽被妖風搶了經去。幸小徒有些神通趕奪,卻俱拋擲散漫。因展看,皆是無字空本。臣等著驚,復去拜告懇求。佛祖道:此經成就之時,有比丘聖僧將下山與舍衛國趙長者家看誦了一遍,保祐他家生者安全,亡者超脫,止討了他三斗三升米粒黃金,意思還嫌賣賤了,後來子孫沒錢使用。我等知二尊者需索人事,佛祖明知,只得將欽賜紫金缽盂送他,方傳了有字真經。此經有三十五部,各部中檢了幾卷傳來,共計五千零四十八卷。此數蓋合一藏也。」

太宗更喜,教:「光祿寺設宴在東閣酬謝。」忽見他三徒立在階下,容貌異常,便問:「高徒果外國人耶?」長老俯伏道:「大徒弟姓孫,法名悟空,臣又呼他為孫行者。他出身原是東勝神洲傲來國花果山水簾洞人氏。因五百年前大鬧天宮,被佛祖困壓在西番兩界山石匣之內,蒙觀音菩薩勸善,情願皈依。是臣到彼救出,保護甚虧此徒。二徒弟姓豬,法名悟能,臣又呼他為豬八戒。他出身原是福陵山雲棧洞人氏。因在烏斯藏高老莊上作怪,亦蒙菩薩勸善,虧行者收之。一路上挑擔有力,涉水有功。三徒弟姓沙,法名悟淨,臣又呼他為沙和尚。他出身原是流沙河作怪者,也蒙菩薩勸善,秉教沙門。那匹馬不是主公所賜者。」太宗道:「毛片相同,如何不是?」三藏道:「臣到蛇盤山鷹愁澗涉水,原馬被此馬吞之。虧行者請菩薩問此馬來歷,原是西海龍王之子,因有罪,也蒙菩薩救解,教他與臣作腳力。當時變作原馬,毛片相同。幸虧他登山越嶺,跋涉崎嶇,去時騎坐,來時馱經,亦甚賴其力也。」

太宗聞言,稱讚不已。又問:「遠涉西方,端的路程多少?」三藏道:「總記菩薩之言,有十萬八千里之遠。途中未曾記數,只知經過了一十四遍寒暑。日日山,日日嶺。遇林不小,遇水寬洪。還經幾座國王,俱有照驗的印信。」叫:「徒弟,將通關文牒取上來,對主公繳納。」當時遞上。太宗看了,乃貞觀一十三年九月望前三日給。太宗笑道:「久勞遠涉,今已貞觀二十七年矣。」牒文上有寶象國印、烏雞國印、車遲國印、西梁女國印、祭賽國印、朱紫國印、比丘國印、滅法國印,又有鳳仙郡印、玉華州印、金平府印。太宗覽畢,收了。

早有當駕官請宴,即下殿攜手而行。又問:「高徒能禮貌乎?」三藏道:「小徒俱是山村曠野之妖身,未諳中華聖朝之禮數,萬望主公赦罪。」太宗笑道:「不罪他,不罪他。都同請東閣赴宴去也。」三藏又謝了恩,招呼他三眾,都到閣內觀看。果是中華大國,比尋常不同。你看那:
\begin{quote}
門懸綵繡,地襯紅氈。異香馥郁,奇品新鮮。琥珀杯,玻璃盞,鑲金點翠;黃金盤,白玉碗,嵌錦花纏。爛煮蔓菁,糖澆香芋。蘑菇甜美,海菜清奇。幾次添來薑辣筍,數番辦上蜜調葵。麵觔椿樹葉,木耳豆腐皮。石花仙菜,蕨粉乾薇。花椒煮萊菔,芥末拌瓜絲。幾盤素品還猶可,數種奇稀果奪魁。核桃柿餅,龍眼荔枝。宣州繭栗山東棗,江南銀杏兔頭梨。榛松蓮肉葡萄大,榧子瓜仁菱米齊。橄欖林檎,蘋婆沙果。慈菰嫩藕,脆李楊梅。無般不備,無件不齊。還有些蒸酥蜜食兼嘉饌,更有那美酒香茶與異奇。說不盡百味珍饈真上品,果然是中華大國異西夷。
\end{quote}

師徒四眾與文武多官,俱侍列左右。太宗皇帝仍正坐當中。歌舞吹彈,整齊嚴肅,遂盡樂一日。正是:
\begin{quote}
君王嘉會賽唐虞,取得真經福有餘。
千古流傳千古盛,佛光普照帝王居。
\end{quote}

當日天晚,謝恩宴散。太宗回宮,多官回宅。唐僧等歸於洪福寺,只見寺僧磕頭迎接。方進山門,眾僧報道:「師父,這樹頭兒今早俱忽然向東。我們記得師父之言,遂出城來接,果然到了。」長老喜之不勝,遂入方丈。此時八戒也不嚷茶飯,也不弄喧頭;行者、沙僧,個個穩重。只因道果完成,自然安靜。當晚睡了。

次早,太宗升朝,對群臣言曰:「朕思御弟之功,至深至大,無以為酬。一夜無寐,口占幾句俚談,權表謝意,但未曾寫出。」叫:「中書官來,朕念與你,你一一寫之。」其文蓋云:
\begin{quote}
嘗聞二儀有象,顯覆載以含生;四序無形,潛寒暑以化物。是以窺天鑑地,庸愚皆識其端;明陰洞陽,賢哲罕窮其數。然天地包乎陰陽而易識者,以其有象也;陰陽處乎天地而難窮者,以其無形也。故知象顯可證,雖愚不惑;形潛莫睹,在智猶迷。況乎佛道沖虛,乘幽空寂。宏濟萬品,典御十方。舉威靈而無上,抑神力而無下。大之則彌於宇宙,細之則攝於毫釐。無滅無生,歷千仞而亙古;若潛若顯,運百福而長今。妙道凝玄,遵導莫知其際;法流湛寂,挹挹莫測其源。故知蠢蠢凡愚,區區庸鄙,投其旨趣,能無疑惑者哉?然大教之興,基乎西土。騰漢庭而皎夢,照東域而流慈。古者卜形卜跡之時,言未馳而成化。當常見常隱之世,民仰德而知遵。及乎晦影歸真,遷移越世,金容掩色,不鏡三千之光;麗像開圖,空端四八之相。於是微言廣被,拯禽類於三途;遺訓遐宣,導群生於十地。佛有經,能分大小之乘;更有法,傳訛邪正之術。我僧玄奘法師者,法門之領袖也。幼懷真敏,早悟三空之功;長契神清,先包四忍之行。松風水月,未足比其清華;仙露明珠,詎能方其朗潤?故以智通無累,神測未形。超六塵而迥出,使千古而傳芳。凝心內境,悲正潛靈;栖慮玄門,多門訛謬。思欲分條,是以翹心淨土,策杖孤征。積雪晨飛,途間失地;驚沙夕起,空外迷天。萬里山川,撥煙霞而進步;百重寒暑,歷霜雨而前蹤。誠重勞輕,求深欲達。周遊西宇,十有四年。窮歷異邦,詢求正教。雙林八水,味道餐風;鹿苑鷲峰,瞻奇仰異。承至言於先聖,受真教於上賢。探賾妙門,精窮奧業。三乘六律之道,馳驟於心田;一藏百篋之文,波濤於海口。爰自所歷之國無涯,求取之經有數。總得大乘要文凡三十五部,計五千四十八卷,譯布中華,宣揚勝業。引慈雲於西極,注法雨於東陲。聖教缺而復全,蒼生罪而還福。溫火宅之乾焰,共拔幽途;朗金水之混波,同臻彼岸。是知惡因業墜,善以緣昇。昇墜之端,惟人自作。譬之桂生高嶺,凌雲方得泫其華;蓮出綠波,飛塵不能染其葉。非蓮性自潔而桂質本貞,由所負者高,則微物不能累;所憑者淨,則濁類不能沾。夫以卉木無知,猶資善而成善,矧以人倫有識,寧不緣慶而成慶哉?方冀茲經傳佈,並日月而無窮;景福遐敷,與乾坤而永大也歟!
\end{quote}

寫畢,即召聖僧。此時長老已在朝門外候謝,聞宣急入,行俯伏之禮。太宗傳請上殿,將文字遞與。長老覽遍,復下謝恩,奏道:「主公文辭高古,理趣淵微。但不知是何名目?」太宗道:「朕夜口占,答謝御弟之意,名曰《聖教序》,不知好否?」長老叩頭,稱謝不已。太宗又曰:「朕才愧珪璋,言慚金石。至於內典,尤所未聞。口占敘文,誠為鄙拙。穢翰墨於金簡,標瓦礫於珠林。循躬省慮,靦面恧心。甚不足稱,虛勞致謝。」

當時多官齊賀,頂禮聖教御文,遍傳內外。太宗道:「御弟將真經演誦一番,何如?」長老道:「主公,若演真經,須尋佛地。寶殿非可誦之處。」太宗甚喜,即問當駕官:「長安城中,有那座寺院潔淨?」班中閃上大學士蕭瑀奏道:「城中有一雁塔寺潔淨。」太宗即令多官:「把真經各虔捧幾卷,同朕到雁塔寺,請御弟談經去來。」

多官遂各各捧著,隨太宗駕幸寺中,搭起高臺,鋪設齊整。長老仍命八戒、沙僧牽龍馬,理行囊;行者在我左右。」又向太宗道:「主公欲將真經傳流天下,須當謄錄副本,方可佈散。原本還當珍藏,不可輕褻。」太宗又笑道:「御弟之言甚當,甚當。」隨召翰林院及中書科各官謄寫真經。又建一寺在城之東,名曰謄黃寺。

長老捧幾卷登臺,方欲諷誦,忽聞得香風繚繞,半空中有八大金剛現身,高叫道:「誦經的放下經卷,跟我回西去也。」這底下行者三人連白馬,平地而起;長老亦將經卷丟下,也從臺上起於九霄,相隨騰空而去。慌得那太宗與多官望空下拜。這正是:
\begin{quote}
聖僧努力取經編,西宇周流十四年。
苦歷程途遭患難,多經山水受迍邅。
功完八九還加九,行滿三千及大千。
大覺妙文回上國,至今東土永留傳。
\end{quote}

太宗與多官拜畢,即選高僧,就於雁塔寺裡,修建水陸大會,看誦大藏真經,超脫幽冥孽鬼,普施善慶。將謄錄過經文,傳播天下不題。

卻說八大金剛駕香風,引著長老四眾,連馬五口,復轉靈山。連去連來,適在八日之內。此時靈山諸神,都在佛前聽講。八金剛引他師徒進去,對如來道:「弟子前奉金旨,駕送聖僧等已到唐國,將經交納,今特繳旨。」遂叫唐僧等近前受職。如來道:「聖僧,汝前世原是我之二徒,名喚金蟬子。因為汝不聽說法,輕慢我之大教,故貶汝之真靈,轉生東土。今喜皈依,秉我迦持,又乘吾教,取去真經,甚有功果,加陞大職正果,汝為旃檀功德佛。——孫悟空,汝因大鬧天宮,吾以甚深法力,壓在五行山下,幸天災滿足,歸於釋教;。且喜汝隱惡揚善,在途中煉魔降怪有功,全終全始,加陞大職正果,汝為鬥戰勝佛。——豬悟能,汝本天河水神天蓬元帥,為汝蟠桃會上酗酒戲了仙娥,貶汝下界投胎,身如畜類。幸汝記愛人身,在福陵山雲棧洞造孽,喜歸大教,入我沙門,保聖僧在路,卻又有頑心,色情未泯。因汝挑擔有功,加陞汝職正果,做淨壇使者。」八戒口中嚷道:「他們都成佛,如何把我做個淨壇使者?」如來道:「因汝口壯身慵,食腸寬大。蓋天下四大部洲,瞻仰吾教者甚多,凡諸佛事,教汝淨壇,乃是個有受用的品級,如何不好?——沙悟淨,汝本是捲簾大將。先因蟠桃會上打碎玻璃盞,貶汝下界,汝落於流沙河,傷生吃人造孽。幸皈吾教,誠敬迦持,保護聖僧,登山牽馬有功,加陞大職正果,為金身羅漢。」又叫那白馬:「汝本是西洋大海廣晉龍王之子,因汝違逆父命,犯了不孝之罪。幸得皈身皈法,皈我沙門,每日家虧你馱負聖僧來西,又虧你馱負聖經去東,亦有功者,加陞汝職正果,為八部天龍馬。」

長老四眾,俱各叩頭謝恩。馬亦謝恩訖。仍命揭諦引了馬,下靈山後崖化龍池邊,將馬推入池中。須臾間,那馬打個展身,即退了毛皮,換了頭角,渾身上長起金鱗,腮頷下生出銀鬚,一身瑞氣,四爪祥雲,飛出化龍池,盤繞在山門裡擎天華表柱上。諸佛讚揚如來的大法。

孫行者卻又對唐僧道:「師父,此時我已成佛,與你一般,莫成還戴金箍兒,你還念甚麼緊箍咒掯勒我?趁早兒念個鬆箍兒咒,脫下來,打得粉碎,切莫叫那甚麼菩薩再去捉弄他人。」唐僧道:「當時只為你難管,故以此法制之。今已成佛,自然去矣,豈有還在你頭上之理?你試摸摸看。」行者舉手去摸一摸,果然無之。

此時旃檀佛、鬥戰佛、淨壇使者、金身羅漢俱正果了本位,天龍馬亦自歸真。有詩為證。詩曰:
\begin{quote}
一體真如轉落塵,合和四相復修身。
五行論色空還寂,百怪虛名總莫論。
正果旃檀歸大覺,完成品職脫沉淪。
經傳天下恩光闊,五聖高居不二門。
\end{quote}

五聖果位之時,諸眾佛祖、菩薩、聖僧、羅漢、揭諦、比丘、優婆夷塞、各山各洞神仙、大神、丁甲、功曹、伽藍、土地,一切得道的師仙,始初俱來聽講,至此各歸方位。你看那:
\begin{quote}
靈鷲峰頭聚霞彩,極樂世界集祥雲。金龍穩臥,玉虎安然。烏兔任隨來往,龜蛇憑汝盤旋。丹鳳青鸞情爽爽,玄猿白鹿意怡怡。八節奇花,四時仙果。喬松古檜,翠柏修篁。五色梅時開時結,萬年桃時熟時新。千果千花爭秀,一天瑞靄紛紜。
\end{quote}

大眾合掌皈依,都念:
\begin{quote}
「南無燃燈上古佛。南無藥師琉璃光王佛。
南無釋迦牟尼佛。南無過去未來現在佛。
南無清淨喜佛。南無毘盧尸佛。
南無寶幢王佛。南無彌勒尊佛。
南無阿彌陀佛。南無無量壽佛。
南無接引歸真佛。南無金剛不壞佛。
南無寶光佛。南無龍尊王佛。
南無精進善佛。南無寶月光佛。
南無現無愚佛。南無婆留那佛。
南無那羅延佛。南無功德華佛。
南無才功德佛。南無善遊步佛。
南無旃檀光佛。南無摩尼幢佛。
南無慧炬照佛。南無海德光明佛。
南無大慈光佛。南無慈力王佛。
南無賢善首佛。南無廣莊嚴佛。
南無金華光佛。南無才光明佛。
南無智慧勝佛。南無世靜光佛。
南無日月光佛。南無日月珠光佛。
南無慧幢勝王佛。南無妙音聲佛。
南無常光幢佛。南無觀世燈佛。
南無法勝王佛。南無須彌光佛。
南無大慧力王佛。南無金海光佛。
南無大通光佛。南無才光佛。
南無旃檀功德佛。南無鬥戰勝佛。
南無觀世音菩薩。南無大勢至菩薩。
南無文殊菩薩。南無普賢菩薩。
南無清淨大海眾菩薩。南無蓮池海會佛菩薩。
南無西天極樂諸菩薩。南無三千揭諦大菩薩。
南無五百阿羅大菩薩。南無比丘夷塞尼菩薩。
南無無邊無量法菩薩。南無金剛大士聖菩薩。
南無淨壇使者菩薩。南無八寶金身羅漢菩薩。
南無八部天龍廣力菩薩。如是等一切世界諸佛:
願以此功德,莊嚴佛淨土。
上報四重恩,下濟三途苦。
若有見聞者,悉發菩提心。
同生極樂國,盡報此一身。
十方三世一切佛,諸尊菩薩摩訶薩,摩訶般若波羅密。」
\end{quote}

《西遊記》至此終。
