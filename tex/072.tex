
\chapter{盤絲洞七情迷本 濯垢泉八戒忘形}

話表三藏別了朱紫國王,整頓鞍馬西進。行夠多少山原,歷盡無窮水道,不覺的秋去冬殘,又值春光明媚。師徒們正在路踏青玩景,忽見一座庵林。三藏滾鞍下馬,站立大道之傍。行者問道:「師父,這條路平坦無邪,因何不走?」八戒道:「師兄好不通情。師父在馬上坐得困了,也讓他下來關關風是。」三藏道:「不是關風,我看那裡是個人家,意欲自去化些齋吃。」行者笑道:「你看師父說的是那裡話,你要吃齋,我自去化。俗語云:『一日為師,終身為父。』豈有為弟子者高坐,教師父去化齋之理?」三藏道:「不是這等說。平日間一望無邊無際,你們沒遠沒近的去化齋。今日人家逼近,可以叫應,也讓我去化一個來。」八戒道:「師父沒主張。常言道:『三人出外,小的兒苦。』你況是個父輩,我等俱是弟子。古書云:『有事弟子服其勞。』等我老豬去。」三藏道:「徒弟啊,今日天氣晴明,與那風雨之時不同。那時節,汝等必定遠去。此個人家,等我去,有齋無齋,可以就回走路。」沙僧在傍笑道:「師兄,不必多講,師父的心性如此,不必違拗。若惱了他,就化將齋來,他也不吃。」

八戒依言,即取出缽盂,與他換了衣帽。拽開步,直至那莊前觀看,卻也好座住場。但見:
\begin{quote}
石橋高聳,古樹森齊。石橋高聳,潺潺流水接長溪;古樹森齊,聒聒幽禽鳴遠岱。橋那邊有數椽茅屋,清清雅雅若仙庵;又有那一座蓬窗,白白明明欺道院。窗前忽見四佳人,都在那裡刺鳳描鸞做針線。
\end{quote}

長老見那人家沒個男兒,只有四個女子,不敢進去,將身立定,閃在喬林之下。只見那女子一個個:
\begin{quote}
閨心堅似石,蘭性喜如春。
嬌臉紅霞襯,朱唇絳脂勻。
蛾眉橫月小,蟬鬢疊雲新。
若到花間立,遊蜂錯認真。
\end{quote}

少停有半個時辰,一發靜悄悄,雞犬無聲。自家思慮道:「我若沒本事化頓齋飯,也惹那徒弟笑我:敢道為師的化不出齋來,為徒的怎能去拜佛?」

長老沒計奈何,也帶了幾分不是,趨步上橋。又走了幾步,只見那茅屋裡面有一座木香亭子,亭子下又有三個女子在那裡踢氣毬哩。你看那三個女子,比那四個又生得不同。但見那:
\begin{quote}
飄揚翠袖,搖拽緗裙。飄揚翠袖,低籠著玉筍纖纖;搖拽緗裙,半露出金蓮窄窄。形容體勢十分全,動靜腳跟千樣屣。拿頭過論有高低,張泛送來真又楷。轉身踢個出牆花,退步翻成大過海。輕接一團泥,單槍急對拐。明珠上佛頭,實捏來尖。窄磚偏會拿,臥魚將腳抷。平腰折膝蹲,扭頂翹跟屣。扳凳能喧泛,披肩甚脫灑。絞襠任往來,鎖項隨搖擺。踢的是黃河水倒流,金魚灘上買。那個錯認是頭兒,這個轉身就打拐。端然捧上臁,周正尖來捽。提跟潠草鞋,倒插回頭採。退步泛肩妝,鉤兒只一歹。販簍下來長,便把奪門揣。踢到美心時,佳人齊喝采。一個個汗流粉膩透羅裳,興懶情疏方叫海。
\end{quote}

言不盡,又有詩為證。詩曰:
\begin{quote}
蹴踘當場三月天,仙風吹下素嬋娟。
汗沾粉面花含露,塵染蛾眉柳帶煙。
翠袖低垂籠玉筍,緗裙斜拽露金蓮。
幾回踢罷嬌無力,雲鬢蓬鬆寶髻偏。
\end{quote}

三藏看得時辰久了,只得走上橋頭,應聲高叫道:「女菩薩,貧僧這裡隨緣佈施些兒齋吃。」那些女子聽見,一個個喜喜歡歡拋了針線,撇了氣毬,都笑笑吟吟的接出門來道:「長老,失迎了。今到荒莊,決不敢攔路齋僧,請裡面坐。」三藏聞言,心中暗道:「善哉,善哉!西方正是佛地,女流尚且注意齋僧,男子豈不虔心向佛?」

長老向前問訊了,相隨眾女入茅屋。過木香亭看處,呀!原來那裡邊沒甚房廊。只見那:
\begin{quote}
巒頭高聳,地脈遙長。巒頭高聳接雲煙,地脈遙長通海岳。門近石橋,九曲九灣流水顧;園栽桃李,千株千顆斗穠華。藤薜掛懸三五樹,芝蘭香散萬千花。遠觀洞府欺蓬島,近睹山林壓太華。正是妖仙尋隱處,更無鄰舍獨成家。
\end{quote}

有一女子上前,把石頭門推開兩扇,請唐僧裡面坐。那長老只得進去。忽擡頭看時,鋪設的都是石桌、石凳,冷氣陰陰。長老心驚,暗自思忖道:「這去處少吉多凶,斷然不善。」眾女子喜笑吟吟,都道:「長老請坐。」長老沒奈何,只得坐了。少時間,打個冷禁。眾女子問道:「長老是何寶山?化甚麼緣?還是修橋補路,建寺禮塔,還是造佛印經?請緣簿出來看看。」長老道:「我不是化緣的和尚。」女子道:「既不化緣,到此何幹?」長老道:「我是東土大唐差去西天大雷音求經者。適過寶方,腹間饑餒,特造檀府,募化一齋,貧僧就行也。」眾女子道:「好好好。常言道:『遠來的和尚好看經。』妹妹們。不可怠慢,快辦齋來。」

此時有三個女子陪著,言來語去,論說些因緣。那四個到廚中撩衣斂袖,炊火刷鍋。你道他安排的是些甚麼東西?原來是人油炒煉,人肉煎熬:熬得黑糊充作麵觔樣子,剜的人腦煎作豆腐塊片。兩盤兒捧到石桌上放下,對長老道:「請了。倉卒間,不曾備得好齋,且將就吃些充腹。後面還有添換來也。」那長老聞了一聞,見那腥膻,不敢開口,欠身合掌道:「女菩薩,貧僧是胎裡素。」眾女子笑道:「長老,此是素的。」長老道:「阿彌陀佛!若像這等素的啊,我和尚吃了,莫想見得世尊,取得經卷。」眾女子道:「長老,你出家人,切莫揀人佈施。」長老道:「怎敢,怎敢。我和尚奉大唐旨意,一路西來,微生不損,見苦就救;遇穀粒手拈入口,逢絲縷聯綴遮身。怎敢揀主佈施?」眾女子笑道:「長老雖不揀人佈施,卻只有些上門怪人。莫嫌粗淡,吃些兒罷。」長老道:「實是不敢吃,恐破了戒。望菩薩養生不若放生,放我和尚出去罷。」

那長老掙著要走,那女子攔住門,怎麼肯放,俱道:「上門的買賣,倒不好做。『放了屁兒,卻使手掩。』你往那裡去?」他一個個都會些武藝,手腳又活,把長老扯住,順手牽羊,撲的摜倒在地。眾人按住,將繩子綑了,懸梁高吊。這吊有個名色,叫做「仙人指路」。原來是一隻手向前,牽絲吊起;一隻手攔腰綑住,將繩吊起;兩隻腳向後,一條繩吊起:三條繩把長老吊在梁上,卻是脊背朝上,肚皮朝下。那長老忍著疼,噙著淚,心中暗恨道:「我和尚這等命苦,只說是好人家化頓齋吃,豈知道落了火坑。徒弟啊,速來救我,還得見面;但遲兩個時辰,我命休矣。」

那長老雖然苦惱,卻還留心看著那些女子。那些女子把他吊得停當,便去脫剝衣服。長老心驚,暗自忖道:「這一脫了衣服,是要打我的情了。或者夾生兒吃我的情也有哩。」原來那女子們只解了上身羅衫,露出肚腹,各顯神通:一個個腰眼中冒出絲繩,有鴨蛋粗細,骨都都的,迸玉飛銀時下把莊門瞞了不題。

卻說那行者、八戒、沙僧都在大道之傍,他二人都放馬看擔,惟行者是個頑皮,他且跳樹攀枝,摘葉尋果。忽回頭,只見一片光亮,慌得跳下樹來,吆喝道:「不好,不好,師父造化低了。」行者用手指道:「你看那莊院如何?」八戒、沙僧共目視之,那一片如雪又亮如雪,似銀又光似銀。八戒道:「罷了,罷了,師父遇著妖精了,我們快去救他也。」行者道:「賢弟莫嚷。你都不見怎的,等老孫去來。」沙僧道:「哥哥仔細。」行者道:「我自有處。」

好大聖,束一束虎皮裙,掣出金箍棒,拽開腳,兩三步跑到前邊,看見那絲繩纏了有千百層厚,穿穿道道,卻似經緯之勢。用手按了一按,有些粘軟沾人。行者更不知是甚麼東西。他即舉棒道:「這一棒,莫說是幾千層,就有幾萬層,也打斷了。」正欲打,又停住手道:「若是硬的便可打斷,這個軟的,只好打匾罷了。假如驚了他,纏住老孫,反為不美。等我且問他一問再打。」

你道他問誰?即捻一個訣,念一個咒,拘得個土地老兒在廟裡似推磨的一般亂轉。土地婆兒道:「老兒,你轉怎的?好道是羊兒風發了。」土地道:「你不知,你不知。有一個齊天大聖來了,我不曾接他,他那裡拘我哩。」婆兒道:「你去見他便了,卻如何在這裡打轉?」土地道:「若去見他,他那棍子好不重,他管你好歹就打哩。」婆兒道:「他見你這等老了,那裡就打你?」土地道:「他一生好吃沒錢酒,偏打老年人。」兩口兒講一會,沒奈何,只得走出去,戰兢兢的跪在路傍,叫道:「大聖,當境土地叩頭。」行者道:「你且起來,不要假忙。我且不打你,寄下在那裡。我問你,此間是甚地方?」土地道:「大聖從那廂來?」行者道:「我自東土往西來的。」土地道:「大聖東來,可曾在那山嶺上?」行者道:「正在那山嶺上。我們行李、馬匹還歇在那嶺上不是!」土地道:「那叫做盤絲嶺。嶺下有洞,叫做盤絲洞。洞裡有七個妖精。」行者道:「是男怪,是女怪?」土地道:「是女怪。」行者道:「他有多大神通?」土地道:「小神力薄威短,不知他有多大手段。只知那正南上,離此有三里之遙,有一座濯垢泉,乃天生的熱水,原是上方七仙姑的浴池。自妖精到此居住,占了他的濯垢泉,仙姑更不曾與他爭競,平白地就讓與他了。我見天仙不惹妖魔怪,必定精靈有大能。」行者道:「占了此泉何幹?」土地道:「這怪占了浴池,一日三遭,出來洗澡。如今巳時已過,午時將來啞。」行者聽言道:「土地,你且回去,等我自家拿他罷。」那土地老兒磕了一個頭,戰兢兢的回本廟去了。

這大聖獨顯神通,搖身一變,變作個麻蒼蠅兒,釘在路傍草梢上等待。須臾間,只聽得呼呼吸吸之聲,猶如蠶食葉,卻似海生潮。只好有半盞茶時,絲繩皆盡,依然現出莊村,還像當初模樣。又聽得呀的一聲,柴扉響處,裡邊笑語諠譁,走出七個女子。行者在暗中細看,見他一個個攜手相攙,挨肩執袂,有說有笑的走過橋來,果是標致。但見:
\begin{quote}
比玉香尤勝,如花語更真。柳眉橫遠岫,檀口破櫻唇。釵頭翹翡翠,金蓮閃絳裙。卻似嫦娥臨下界,仙子落凡塵。
\end{quote}

行者笑道:「怪不得我師父要來化齋,原來是這一般好物。這七個美人兒,假若留住我師父,要吃也不夠一頓吃,要用也不夠兩日用;要動手輪流,一擺佈就是死了。且等我去聽他一聽,看他怎的算計。」

好大聖,嚶的一聲,飛在那前面走的女子雲髻上釘住。才過橋來,後邊的走向前來呼道:「姐姐,我們洗了澡,來蒸那胖和尚吃去。」行者暗笑道:「這怪物好沒算計,煮還省些柴,怎麼轉要蒸了吃?」那些女子採花鬥草向南來,不多時到了浴池。但見一座門牆,十分壯麗,遍地野花香豔豔,滿傍蘭蕙密森森。後面一個女子走上前,唿哨的一聲,把兩扇門兒推開,那中間果有一塘熱水。這水:
\begin{quote}
自開闢以來,太陽星原貞有十,後被羿善開弓,射落九烏墜地,止存金烏一星,乃太陽之真火也。天地有九處湯泉,俱是眾烏所化。那九陽泉,乃香冷泉、伴山泉、溫泉、東合泉、潢山泉、孝安泉、廣汾泉、湯泉——此泉乃濯垢泉。
\end{quote}

有詩為證。詩曰:
\begin{quote}
一氣無冬夏,三秋永注春。
炎波如鼎沸,雪浪似湯新。
分溜滋禾稼,停流蕩俗塵。
涓涓珠淚泛,滾滾玉生津。
潤滑原非釀,清平還自溫。
瑞祥本地秀,造化乃天真。
佳人洗處冰肌滑,滌蕩塵煩玉體新。
\end{quote}

那浴池約有五丈餘闊,十丈多長,內有四尺深淺,但見水清徹底。底下水一似滾珠泛玉,骨都都冒將上來,四面有六七個孔竅通流。流去二三里之遙,淌到田裡,還是溫水。池上又有三間亭子。亭子中近後壁放著一張八隻腳的板凳。兩山頭放著兩個彩漆的衣架。行者暗中喜嚶嚶的,一翅飛在那衣架頭上釘住。

那些女子見水又清又熱,便要洗浴,即一齊脫了衣服,搭在衣架上,一齊下去。被行者看見:
\begin{quote}
褪放紐扣兒,解開羅帶結。
酥胸白似銀,玉體渾如雪。
肘膊賽冰鋪,香肩疑粉捏。
肚皮軟又綿,脊背光還潔。
膝腕半圍團,金蓮三寸窄。
中間一段情,露出風流穴。
\end{quote}

那女子都跳下水去,一個個躍浪翻波,負水頑耍。行者道:「我若打他啊,只消把這棍子往池中一攪,就叫做滾湯潑老鼠——一窩兒都是死。可憐,可憐!打便打死他,只是低了老孫的名頭。常言道:『男不與女鬥。我這般一個漢子,打殺這幾個丫頭,著實不濟。不要打他,只送他一個絕後計,教他動不得身,多少是好?」好大聖,捏著訣,念個咒,搖身一變,變作一個餓老鷹。但見:
\begin{quote}
毛猶霜雪,眼若明星。妖狐見處魂皆喪,狡兔逢時膽盡驚。鋼爪鋒芒快,雄姿猛氣橫。會使老拳供口腹,不辭親手逐飛騰。萬里寒空隨上下,穿雲撿物任他行。
\end{quote}

呼的一翅,飛向前,掄開利爪,把他那衣架上搭的七套衣服,盡情叼去,徑轉嶺頭,現出本相,來見八戒、沙僧。

你看那獃子迎著笑道:「師父原來是典當鋪裡拿了去的。」沙僧道:「怎見得?」八戒道:「你不見師兄把他些衣服都搶將來也?」行者放下道:「此乃妖精穿的衣服。」八戒道:「怎麼就有這許多?」行者道:「七套。」八戒道:「如何剝得這般容易,又剝得乾淨?」行者道:「那曾用剝。原來此處喚做盤絲嶺,那莊村喚做盤絲洞。洞中有七個女怪,把我師父拿住,吊在洞裡,都向濯垢泉去洗浴。那泉卻是天地產成的,一塘子熱水。他都算計著洗了澡,要把師父蒸吃。是我跟到那裡,見他脫了衣服下水,我要打他,恐怕污了棍子,又怕低了名頭,是以不曾動棍,只變做一個餓老鷹,叼了他的衣服。他都忍辱含羞,不敢出頭,蹲在水中哩。我等快去解下師父走路罷。」八戒笑道:「師兄,你凡幹事,只要留根。既見妖精,如何不打殺他,卻就去解師父?他如今縱然藏羞不出,到晚間必定出來。他家裡還有舊衣服,穿上一套,來趕我們。縱然不趕,他久住在此,我們取了經,還從那條路回去。常言道:『寧少路邊錢,莫少路邊拳。』那時節,他攔住了吵鬧,卻不是個仇人也?」行者道:「憑你如何主張?」八戒道:「依我,先打殺了妖精,再去解放師父:此乃斬草除根之計。」行者道:「我是不打他,你要打,你去打他。」

八戒抖擻精神,歡天喜地,舉著釘鈀,拽開步,徑直跑到那裡。忽的推開門看時,只見那七個女子蹲在水裡,口中亂罵那鷹哩,道:「這個匾毛畜生,貓嚼頭的亡人,把我們衣服都叼去了,教我們怎的動手?」八戒忍不住笑道:「女菩薩,在這裡洗澡哩?也攜帶我和尚洗洗,何如?」那怪見了,作怒道:「你這和尚,十分無禮。我們是在家的女流,你是個出家的男子。古書云:『七年男女不同席。』你好和我們同塘洗澡?」八戒道:「天氣炎熱,沒奈何,將就容我洗洗兒罷,那裡調甚麼書擔兒,同席不同席?」獃子不容說,丟了釘鈀,脫了皂錦直裰,撲的跳下水來。那怪心中煩惱,一齊上前要打。不知八戒水勢極熟,到水裡搖身一變,變做一個鮎魚精。那怪就都摸魚,趕上拿他不住:東邊摸,忽的又漬了西去;西邊摸,忽的又漬了東去。滑扢虀的,只在那腿襠裡亂鑽。原來那水有攙胸之深,水上盤了一會,又盤在水底,都盤倒了,喘噓噓的,精神倦怠。

八戒卻才跳將上來,現了本相,穿了直裰,執著釘鈀,喝道:「我是那個?你把我當鮎魚精哩。」那怪見了,心驚膽戰,對八戒道:「你先來是個和尚,到水裡變作鮎魚,及拿你不住,卻又這般打扮,你端的是從何到此?是必留名。」八戒道:「這夥潑怪當真的不認得我。我是東土大唐取經的唐長老之徒弟,乃天蓬元帥悟能八戒是也。你把我師父吊在洞裡,算計要蒸他受用。我的師父,又好蒸吃?快早伸過頭來,各築一鈀,教你斷根。」那些妖聞此言,魂飛魄散,就在水中跪拜道:「望老爺方便方便!我等有眼無珠,誤捉了你師父,雖然吊在那裡,不曾敢加刑受苦。望慈悲饒了我的性命,情願貼些盤費,送你師父往西天去也。」八戒搖手道:「莫說這話。俗語說得好:『曾著賣糖君子哄,到今不信口甜人。』是便築一鈀,各人走路。」

獃子一味粗夯,顯手段,那有憐香惜玉之心,舉著鈀,不分好歹,趕上前亂築。那怪慌了手腳,那裡顧甚麼羞恥,只是性命要緊,隨用手侮著羞處,跳出水來,都跑在亭子裡站立,作出法來:臍孔中骨都都冒出絲繩,瞞天搭了個大絲篷,把八戒罩在當中。那獃子忽擡頭,不見天日,即抽身往外便走,那裡舉得腳步。原來放了絆腳索,滿地都是絲繩,動動腳,跌個躘踵;左邊去,一個面磕地;右邊去,一個倒栽蔥;急轉身,又跌了個嘴搵地;忙爬起,又跌了個豎蜻蜓。也不知跌了多少跟頭,把個獃子跌得身麻腳軟,頭暈眼花,爬也爬不動,只睡在地下呻吟。

那怪物卻將他困住,也不打他,也不傷他,一個個跳出門來,將絲篷遮住天光,各回本洞。到了石橋上站下,念動真言,霎時間,把絲篷收了,赤條條的跑入洞裡,侮著那話,從唐僧面前笑嘻嘻的跑過去。走入石房,取幾件舊衣穿了,徑至後門口立定,叫:「孩兒們何在?」原來那妖精一個有一個兒子,卻不是他養的,都是他結拜的乾兒子。有名叫做蜜、螞、蠦、班、蜢、蜡、蜻:蜜是蜜蜂,螞是螞蜂,蠦是蠦蜂,班是班毛,蜢是牛蜢,蜡是抹蜡,蜻是蜻蜓。原來那妖精幔天結網,擄住這七般蟲蛭,卻要吃他。古云:「禽有禽言,獸有獸語。」當時這些蟲哀告饒命,願拜為母。遂此春採百花供怪物,夏尋諸卉孝妖精。忽聞一聲呼喚,都到面前,問:「母親有何使令?」眾怪道:「兒啊,早間我們錯惹了唐朝來的和尚,才然被他徒弟攔在池裡,出了多少醜,幾乎喪了性命。汝等努力,快出門前去退他一退。如得勝後,可到你舅舅家來會我。」那些怪既得逃生,往他師兄處,孽嘴生災不題。你看這些蟲蛭,一個個摩拳擦掌,出來迎敵。

卻說八戒跌得昏頭昏腦,猛擡頭,見絲篷絲索俱無,他才一步一探,爬將起來,忍著疼,找回原路。見了行者,用手扯住道:「哥哥,我的頭可腫,臉可青麼?」行者道:「你怎的來?」八戒道:「我被那廝將絲繩罩住,放了絆腳索,不知跌了多少跟頭,跌得我腰駝背折,寸步難移。卻才絲篷索子俱空,方得了性命回來也。」沙僧見了道:「罷了,罷了,你闖下禍來也,那怪一定往洞裡去傷害師父。我等快去救他。」

行者聞言,急拽步便走;八戒牽著馬。急急來到莊前,但見那石橋上有七個小妖兒擋住道:「慢來,慢來,吾等在此。」行者看了道:「好笑,乾淨都是些小人兒。長的也只有二尺五六寸,不滿三尺;重的也只有八九斤,不滿十斤。」喝道:「你是誰?」那怪道:「我乃七仙姑的兒子。你把我母親欺辱了,還敢無知,打上我門。不要走,仔細。」好怪物,一個個亂打將來。八戒本是跌惱了的性子,又見那夥蟲蛭小巧,就發狠舉鈀來築。那些怪見獃子兇猛,一個個現了本像,飛將起去,叫聲:「變!」須臾間,一個變十個,十個變百個,百個變千個,千個變萬個,個個都變成無窮之數。只見:
\begin{quote}
滿天飛抹蜡,遍地舞蜻蜓。
蜜螞追頭額,蠦蜂扎眼睛。
班毛前後咬,牛蜢上下叮。
撲面漫漫黑,翛翛神鬼驚。
\end{quote}

八戒慌了道:「哥啊,只說經好取,西方路上,蟲兒也欺負人哩。」行者道:「兄弟,不要怕,快上前打。」八戒道:「撲頭撲臉,渾身上下,都叮有十數層厚,卻怎麼打?」行者道:「沒事,沒事,我自有手段。」沙僧道:「哥啊,有甚手段,快使出來罷,一會子光頭上都叮腫了。」

好大聖,拔了一把毫毛,嚼得粉碎,噴將出去,即變做些黃、麻、𪀚、白、鵰、魚、鷂。八戒道:「師兄,又打甚麼市語,黃啊、麻啊哩?」行者道:「你不知。黃是黃鷹,麻是麻鷹,𪀚是𪀚鷹,白是白鷹,鵰是鵰鷹,魚是魚鷹,鷂是鷂鷹。那妖精的兒子是七樣蟲,我的毫毛是七樣鷹。」鷹最能嗛蟲,一嘴一個,爪打翅敲,須臾,打得罄盡,滿空無跡,地積尺餘。

三兄弟方才闖過橋去,徑入洞裡,只見老師父吊在那裡哼哼的哭哩。八戒近前道:「師父,你是要來這裡吊了耍子,不知作成我跌了多少跟頭哩。」沙僧道:「且解下師父再說。」行者即將繩索挑斷,放下師父。問道:「妖精那裡去了?」唐僧道:「那七個都赤條條的往後邊叫兒子去了。」行者道:「兄弟們,跟我來尋去。」

三人各持兵器,往後園裡尋處,不見蹤跡。都到那桃李樹上尋遍不見。八戒道:「去了,去了。」沙僧道:「不必尋他,等我扶師父去也。」弟兄們復來前面,請唐僧上馬。八戒道:「你們扶師父走著,等老豬一頓鈀築倒他這房子,教他來時沒處安身。」行者笑道:「築還費力,不若尋些柴來,與他個斷根罷。」好獃子,尋了些朽松、破竹、乾柳、枯藤,點上一把火,烘烘的都燒得乾淨。師徒卻才放心前來。

咦!畢竟這去,不知那怪的吉凶如何,且聽下回分解。
