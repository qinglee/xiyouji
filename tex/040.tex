
\chapter{嬰兒戲化禪心亂 猿馬刀歸木母空}

卻說那孫大聖兄弟三人按下雲頭。徑至朝內,只見那君臣儲后,幾班兒拜接謝恩。行者將菩薩降魔收怪的那一節,陳訴與他君臣聽了,一個個頂禮不盡。正都在賀喜之間,又聽得黃門官來奏:「主公,外面又有四個和尚來也。」八戒慌了道:「哥哥,莫是妖精弄法,假捏文殊菩薩,哄了我等,卻又變作和尚,來與我們鬥智哩?」行者道:「豈有此理?」即命宣進來看。

眾文武傳令,著他進來。行者看時,原來是那寶林寺僧人,捧著那沖天冠、碧玉帶、赭黃袍、無憂履進得來也。行者大喜道:「來得好,來得好。」且教道人過來,摘下包巾,戴上沖天冠;脫了布衣,穿上赭黃袍;解了絛子,繫上碧玉帶;褪了僧鞋,登上無憂履;教太子拿出白玉珪來,與他執在手裡:早請上殿稱孤。正是自古道:「朝廷不可一日無君。」那皇帝那裡肯坐,哭啼啼,跪在階心道:「我已死三年,今蒙師父救我回生,怎麼又敢妄自稱尊;請那一位師父為君,我情願領妻子城外為民足矣。」那三藏那裡肯受,一心只是要拜佛求經。又請行者,行者笑道:「不瞞列位說,老孫若肯做皇帝,天下萬國九州皇帝都做遍了。只是我們做慣了和尚,是這般懶散。若做了皇帝,就要留頭長髮,黃昏不睡,五鼓不眠;聽有邊報,心神不安;見有災荒,憂愁無奈。我們怎麼弄得慣?你還做你的皇帝,我還做我的和尚,修功行去也。」那國王苦讓不過,只得上了寶殿,南面稱孤,大赦天下,封贈了寶林寺僧人回去。卻才開東閣,筵宴唐僧。一壁廂傳旨宣召丹青,寫下唐師徒四位喜容,供養在金鑾殿上。

那師徒們安了邦國,不肯久停,欲辭王駕投西。那皇帝與三宮妃后、太子、諸臣,將鎮國的寶貝、金銀緞帛,獻與師父酬恩。那三藏分毫不受,只是倒換關文,催悟空等背馬早行。那國王甚不過意,擺整朝鑾駕請唐僧上坐,著兩班文武引導,他與三宮妃后並太子一家兒,捧轂推輪,送出城廓,卻才下龍輦,與眾相別。國王道:「師父啊,到西天經回之日,是必還到寡人界內一顧。」三藏道:「弟子領命。」那皇帝閣淚汪汪,遂與眾臣回去了。

那唐僧一行四僧,上了羊腸大路,一心裡專拜靈山。正值秋盡冬初時節,但見:
\begin{quote}
霜凋紅葉林林瘦,雨熟黃粱處處盈。
日暖嶺梅開曉色,風搖山竹動寒聲。
\end{quote}

師徒們離了烏雞國,夜住曉行,將半月有餘,忽又見一座高山,真個是摩天礙日。三藏馬上心驚,急兜韁忙呼行者。行者道:「師父有何吩咐?」三藏道:「你看前面又有大山峻嶺,須要仔細隄防,恐一時又有邪物來侵我也。」行者笑道:「只管走路,莫再多心,老孫自有防護。」那長老只得寬懷,加鞭策馬,奔至山巖,果然也十分險峻。但見得:
\begin{quote}
高不高,頂上接青霄;深不深,澗中如地府。山前常見骨都都白雲,扢騰騰黑霧。紅梅翠竹,綠柏青松。山後有千萬丈挾魂靈臺,臺後有古古怪怪藏魔洞,洞中有叮叮噹噹滴水泉,泉下有彎彎曲曲流水澗。又見那跳天搠地獻果猿,丫丫叉叉帶角鹿,呢呢痴痴看人獐。至晚巴山尋穴虎,待曉翻波出水龍。登得洞門唿喇的響,驚得飛禽撲魯的起。看那林中走獸鞠律律的行,見此一夥禽和獸,嚇得人心扢磴磴驚。堂倒洞堂堂倒洞,洞當當倒洞當仙。青石染成千塊玉,碧紗籠罩萬堆煙。
\end{quote}

師徒們正當悚懼,又只見那山凹裡有一朵紅雲,直冒到九霄空內,結聚了一團火氣。行者大驚,走近前,把唐僧搊著腳,推下馬來,叫:「兄弟們,不要走了,妖怪來矣。」慌得個八戒急掣釘鈀,沙僧忙掄寶杖,把唐僧圍護在當中。

話分兩頭。卻說紅光裡,真是個妖精。他數年前聞得人講:東土唐僧往西天取經,乃是金蟬長老轉生,十世修行的好人,有人吃他一塊肉,延生長壽,與天地同休。他朝朝在山間等候,不期今日到了。他在那半空裡正然觀看,只見三個徒弟把唐僧圍護在馬上,各各準備。這精靈誇讚不盡道:「好和尚!我才看著一個白面胖和尚騎了馬,真是那唐朝聖僧,卻怎麼被三個醜和尚護持住了?一個個伸拳斂袖,各執兵器,似乎要與人打的一般。噫!不知是那個有眼力的,想應認得我了。似此模樣,莫想得那唐僧的肉吃。」沉吟半晌,以心問心的自家商量道:「若要倚勢而擒,莫能得近;或者以善迷他,卻到得手。但哄得他心迷惑,待我在善內生機,斷然拿了。且下去戲他一戲。」

好妖怪,即散紅光,按雲頭落下。去那山坡裡,搖身一變,變作七歲頑童,赤條條的身上無衣,將麻繩綑了手足,高吊在那松樹梢頭,口口聲聲只叫:「救人,救人!」

卻說那孫大聖忽擡頭再看處,只見那紅雲散盡,火氣全無。便叫:「師父,請上馬走路。」唐僧道:「你說妖怪來了,怎麼又敢走路?」行者道:「我才然間見一朵紅雲從地而起,到空中結做一團火氣,斷然是妖精。這一會紅雲散了,想是個過路的妖精,不敢傷人。我們去耶。」八戒笑道:「師兄說話最巧,妖精又有個甚麼過路的?」行者道:「你那裡知道。若是那山那洞的魔王設宴,邀請那諸山各洞之精赴會,卻就有東西南北四路的精靈都來赴會。故此他只有心赴會,無意傷人。此乃過路之妖精也。」

三藏聞言,也似信不信的,只得攀鞍在馬,順路奔山前進。正行時,只聽得叫聲:「救人!」長老大驚道:「徒弟呀,這半山中,是那裡甚麼人叫?」行者上前道:「師父只管走路,莫纏甚麼人轎、騾轎、明轎、睡轎。這所在,就有轎,也沒個人擡你。」唐僧道:「不是扛擡之轎,乃是叫喚之叫。」行者笑道:「我曉得,莫管閑事,且走路。」

三藏依言,策馬又進。行不上一里之遙,又聽得叫聲:「救人」!長老道:「徒弟,這個叫聲不是鬼魅妖邪。若是鬼魅妖邪,但有出聲,無有回聲。你聽他叫一聲,又叫一聲,想必是個有難之人。我們可去救他一救。」行者道:「師父,今日且把這慈悲心略收起收起,待過了此山,再發慈悲罷。這去處兇多吉少。你知道那倚草附木之說,是物可以成精。諸般還可,只有一般蟒蛇,但修得年遠日深,成了精魅,善能知人小名兒。他若在草科裡,或山凹中,叫人一聲,人不答應還可;若答應一聲,他就把人元神綽去,當夜跟來,斷然傷人性命。且走,且走。古人云:『脫得去,謝神明。』切不可聽他。」長老只得依他,又加鞭催馬而去。

行者心中暗想:「這潑怪不知在那裡,只管叫阿叫的。等我老孫送他一個『卯酉星法』,教他兩不見面。」好大聖,叫沙和尚前來:「攏著馬,慢慢走著,讓老孫解解手。」你看他讓唐僧先行幾步,卻念個咒語,使個移山縮地之法,把金箍棒往後一指,他師徒過此峰頭,往前走了,卻把那怪物撇下。

他再拽開步,趕上唐僧,一路奔山。只見那三藏又聽得那山背後叫聲:「救人!」長老道:「徒弟呀,那有難的人大沒緣法,不曾得遇著我們,我們走過他了。你聽他在山後叫哩。」八戒道:「在便還在山前,只是如今風轉了也。」行者道:「管他甚麼轉風不轉風,且走路。」因此,遂都無言語,恨不得一步踏過此山,不題話下。

卻說那妖精在山坡裡連叫了三四聲,更無人到。他心中思量道:「我等唐僧在此,望見他離不上三里,卻怎麼這半晌還不到?想是抄下路去了。」他抖一抖身軀,脫了繩索,又縱紅光,上空再看。不覺孫大聖仰面回觀,識得是妖怪,又把唐僧撮著腳推下馬來道:「兄弟們,仔細,仔細,那妖精又來也。」慌得那八戒、沙僧各持鈀、棍,將唐僧又圍護在中間。

那精靈見了,在半空中稱羨不已道:「好和尚!我才見那白面和尚坐在馬上,卻怎麼又被他三人藏了?這一去見面方知。先把那有眼力的弄倒了,方才捉得唐僧;不然啊,徒費心機難獲物,枉勞情興總成空。」卻又按下雲頭,恰似前番變化,高吊在松樹梢頭等候。這番卻不上半里之地。

卻說那孫大聖擡頭再看,只見那紅雲又散,復請師父上馬前行。三藏道:「你說妖精又來,如何又請走路?」行者道:「這還是個過路的妖精,不敢惹我們。」長老又懷怒道:「這個潑猴,十分弄我。正當有妖魔處,卻說無事;似這般清平之所,卻又恐嚇我,不時的嚷道有甚麼妖精。虛多實少,不管輕重,將我搊著腳,捽下馬來,如今卻解說甚麼過路的妖精。假若跌傷了我,卻也過意不去,這等這等」行者道:「師父莫怪,若是跌傷了你的手足,卻還好醫治;若是被妖精撈了去,卻何處跟尋?」三藏大怒,哏哏的,要念緊箍兒咒。卻是沙僧苦勸,只得上馬又行。

還未曾坐得穩,只聽又叫:「師父救人啊!」長老擡頭看時,原來是個小孩童,赤條條的吊在樹上。兜住韁,便罵行者道:「這潑猴多大憊𪬯,全無有一些兒善良之意,心心只是要撒潑行兇哩!我那般說叫喚的是個人聲,他就千言萬語,只嚷是妖怪。你看那樹上吊的不是個人麼?」大聖見師父怪下來了,卻又覿面看見模樣,一則做不得手腳,二來又怕念緊箍兒咒,低著頭,再也不敢回言,讓唐僧到了樹下。那長老將鞭梢指著問道:「你是那家孩兒?因有甚事,吊在此間?說與我,好救你。」噫!分明他是個精靈,變化得這等,那師父卻是個肉眼凡胎,不能相識。

那妖魔見他下問,越弄虛頭,眼中噙淚,叫道:「師父呀,山西去有一條枯松澗,澗那邊有一莊村,我是那裡人家。我祖公公姓紅,只因廣積金銀,家私巨萬,混名喚做紅百萬。年老歸世已久,家產遺與我父。近來人事奢侈,家私漸廢,改名喚做紅十萬。專一結交四路豪傑,將金銀借放,希圖利息。怎知那無籍之人,設騙了去啊,本利無歸。我父發了洪誓,分文不借。那借金銀人,身貧無計,結成兇黨,明火執杖,白日殺上我門,將我財帛盡情劫擄;把我父親殺了;見我母親有些顏色,拐將去做甚麼壓寨夫人。那時節,我母親捨不得我,把我抱在懷裡,哭哀哀,戰兢兢,跟隨賊寇。不期到此山中,又要殺我。多虧母親哀告,免教我刀下身亡,卻將繩子吊我在樹上,只教凍餓而死。那些賊將我母親不知掠往那裡去了。我在此已吊三日三夜,更沒一個人來行走。不知那世裡修積,今生得遇老師父。若肯捨大慈悲,救我一命回家,就典身賣命,也酬謝師恩。即使黃沙蓋面,更不敢忘也。」

三藏聞言,認了真實,就教八戒解放繩索,救他下來。那獃子也不識人,便要上前動手。行者在傍,忍不住喝了一聲道:「那潑物!有認得你的在這裡哩,莫要只管架空搗鬼,說謊哄人。你既家私被劫,父被賊傷,母被人擄,救你去交與誰人?你將何物與我作謝?這謊脫節了耶。」那怪聞言,心中害怕,就知大聖是個能人,暗將他放在心上。卻又戰戰兢兢,滴淚而言曰:「師父,雖然我父母空亡,家財盡絕,還有些田產未動,親戚皆存。」行者道:「你有甚麼親戚?」妖怪道:「我外公家在山南,姑娘住居嶺北,澗頭李四是我姨夫,林內紅三是我族伯,還有堂叔、堂兄都住在本莊左右。老師父若肯救我,到了莊上,見了諸親,將老師父拯救之恩,一一對眾言說,典賣些田產,重重酬謝也。」

八戒聽說,扛住行者道:「哥哥,這等一個小孩子家,你只管盤詰他怎的?他說得是強盜,只打劫他些浮財,莫成連房屋田產也劫得去?若與他親戚們說了,我們縱有廣大食腸,也吃不了他十畝田價。救他下來罷。」獃子只是想著吃食,那裡管甚麼好歹,使戒刀挑斷繩索,放下怪來。那怪對唐僧馬下淚汪汪,只情磕頭。長老心慈,便叫:「孩兒,你上馬來,我帶你去。」那怪道:「師父啊,我手腳都吊麻了,腰胯疼痛;一則是鄉下人家,不慣騎馬。」唐僧叫八戒馱著。那妖怪抹了一眼道:「師父,我的皮膚都凍熟了,不敢要這位師父馱。他的嘴長耳大,腦後鬃硬,搠得我慌。」唐僧道:「教沙和尚馱著。」那怪也抹了一眼道:「師父,那些賊來打劫我家時,一個個都搽了花臉,帶假鬍子,拿刀弄杖的。我被他諕怕了,見這位晦氣臉的師父,一發沒了魂了,也不敢要他馱。」唐僧教孫行者馱著。行者呵呵笑道:「我馱,我馱。」那怪物暗自歡喜,順順當當的要行者馱他。

行者把他扯在路傍邊,試了一試,只好有三斤十來兩重。行者笑道:「你這個潑怪物,今日該死了,怎麼在老孫面前搗鬼?我認得你是個那話兒。」妖怪道:「我是好人家兒女,不幸遭此大難,怎麼是個甚麼『那話兒』?」行者道:「你既是好人家兒女,怎麼這等骨頭輕?」妖怪道:「我骨格兒小。」行者道:「你今年幾歲了?」那妖怪道:「我七歲了。」行者笑道:「一歲長一斤,也該七斤,你怎麼不滿四斤重麼?」那怪道:「我小時失乳。」行者說:「也罷,我馱著你,若要尿尿把把,須和我說。」

三藏才與八戒、沙僧前走,行者背著孩兒隨後,一行徑投西去。有詩為證。詩曰:
\begin{quote}
道德高隆魔障高,禪機本靜靜生妖。
心君正直行中道,木母痴屣頑外趫。
意馬不言懷愛慾,黃婆無語自憂焦。
客邪得志空歡喜,畢竟還從正處消。
\end{quote}

孫大聖馱著妖魔,心中埋怨唐僧不知艱苦:「行此險峻山場,空身也難走,卻教老孫馱人。這廝莫說他是妖怪,就是好人,他沒了父母,不知將他馱與何人,倒不如摜殺他罷。」那怪物卻早知覺了,便就使個神通,往四下裡吸了四口氣,吹在行者背上,便覺重有千斤。行者笑道:「我兒啊,你弄重身法壓我老爺哩。」那怪聞言,恐怕大聖傷他,卻就解尸,出了元神,跳將起去,佇立在九霄空裡。這行者背上越重了。猴王發怒,抓過他來,往那路傍邊賴石頭上滑辣的一摜,將屍骸摜得像個肉餅一般。還恐他又無禮,索性將四肢扯下,丟在路兩邊,俱粉碎了。

那物在空中明明看著,忍不住心頭火起道:「這猴和尚十分憊𪬯,就作我是個妖魔,要害你師父,卻還不曾見怎麼下手哩,你怎麼就把我這等傷損?早是我有算計,出神走了;不然,是無故傷生也。若不趁此時拿了唐僧,再讓一番,越教他停留長智。」好怪物,就在半空裡弄了一陣旋風,呼的一聲響喨,走石揚沙,誠然兇狠。好風:
\begin{quote}
淘淘怒捲水雲腥,黑氣騰騰閉日明。
嶺樹連根通拔盡,野梅帶幹悉皆平。
黃沙迷目人難走,怪石傷殘路怎平。
滾滾團團平地暗,遍山禽獸發哮聲。
\end{quote}

刮得那三藏馬上難存,八戒不敢仰視,沙僧低頭掩面。孫大聖情知是怪物弄風,急縱步來趕時,那怪已騁風頭,將唐僧攝去了,無蹤無影,不知攝向何方,無處跟尋。

一時間,風聲暫息,日色光明。行者上前觀看,只見白龍馬戰兢兢發喊聲嘶,行李擔丟在路下,八戒伏於崖下呻吟,沙僧蹲在坡前叫喚。行者喊:「八戒。」那獃子聽見是行者的聲音,卻擡頭看時,狂風已靜,爬起來,扯住行者道:「哥哥,好大風啊!」沙僧卻也上前道:「哥哥,這是一陣旋風。」又問:「師父在那裡?」八戒道:「風來得緊,我們都藏頭遮眼,各自躲風,師父也伏在馬上的。」行者道:「如今卻往那裡去了?」沙僧道:「是個燈草做的,想被一風捲去也。」

行者道:「兄弟們,我等自此就該散了。」八戒道:「正是,趁早散了,各尋頭路,多少是好。那西天路無窮無盡,幾時能到得?」沙僧聞言,打了一個失驚,渾身麻木道:「師兄,你都說的是那裡話?我等因為前生有罪,感蒙觀世音菩薩勸化,與我們摩頂受戒,改換法名,皈依佛果,情願保護唐僧上西方拜佛求經,將功折罪。今日到此,一旦俱休,說出這等各尋頭路的話來,可不違了菩薩的善果,壞了自己的德行,惹人恥笑,說我們有始無終也?」行者道:「兄弟,你說的也是,奈何師父不聽人說。我老孫火眼金睛,認得好歹。才然這風,是那樹上吊的孩兒弄的。我認得他是個妖精,你們不識,那師父也不識,認作是好人家兒女,教我馱著他走。是老孫算計要擺佈他,他就弄個重身法壓我。是我將他摜得粉碎。他想是又使解屍之法,弄陣旋風,把我師父攝去也。因此上怪他每每不聽我說,故我意懶心灰,說各人散了。既是賢弟有此誠意,教老孫進退兩難。——八戒,你端的要怎的處?」八戒道:「我才自失口亂說了幾句,其實也不該散。哥哥,沒及奈何,還信沙弟之言,去尋那妖怪救師父去。」行者卻回嗔作喜道:「兄弟們,還要來結同心,收拾了行李、馬匹,上山找尋怪物,搭救師父去。」

三個人附葛扳藤,尋坡轉澗,行經有五七十里,卻也沒個音信。那山上飛禽走獸全無,老柏喬松常見。孫大聖著實心焦,將身一縱,跳上那巔嶮峰頭,喝一聲叫:「變!」變作三頭六臂,似那大鬧天宮的本像。將金箍棒幌一幌,變作三根金箍棒,劈哩撲辣的,往東打一路,往西打一路,兩邊不住的亂打。八戒見了道:「沙和尚,不好了,師兄是尋不著師父,惱出氣心風來了。」

那行者打了一會,打出一夥窮神來,都披一片、掛一片,褌無襠、褲無口的跪在山前,叫:「大聖,山神、土地來見。」行者道:「怎麼就有許多山神、土地?」眾神叩頭道:「上告大聖:此山喚做六百里鑽頭號山。我等是十里一山神,十里一土地,共該三十名山神、三十名土地。昨日已此聞大聖來了,只因一時會不齊,故此接遲,致令大聖發怒,萬望恕罪。」行者道:「我且饒你罪名。我問你:這山上有多少妖精?」眾神道:「爺爺呀!只有得一個妖精,把我們頭也摩光了,弄得我們少香沒紙,血食全無,一個個衣不充身,食不充口,還吃得有多少妖精哩。」行者道:「這妖精在山前住,是山後住?」眾神道:「他也不在山前山後。這山中有一條澗,叫做枯松澗。澗邊有一座洞,叫做火雲洞。那洞裡有一個魔王,神通廣大,常常的把我們山神、土地拿了去,燒火頂門,黑夜與他提鈴喝號。小妖兒又討甚麼常例錢。」行者道:「汝等乃是陰鬼之仙,有何錢鈔?」眾神道:「正是沒錢與他,只得捉幾個山獐、野鹿,早晚間打點群精;若是沒物相送,就要來拆廟宇,剝衣裳,攪得我等不得安生。萬望大聖與我等剿除此怪,拯救山上生靈。」行者道:「你等既受他節制,常在他洞下,可知他是那裡妖精,叫做甚麼名字?」眾神道:「說起他來,或者大聖也知道。他是牛魔王的兒子,羅剎女養的。他曾在火燄山修行了三百年,煉成三昧真火,卻也神通廣大,牛魔王使他來鎮守號山。乳名叫做紅孩兒,號叫做聖嬰大王。」

行者聞言,滿心歡喜。喝退了土地、山神,卻現了本像,跳下峰頭,對八戒、沙僧道:「兄弟們放心,再不須思念,師父決不傷生,妖精與老孫有親。」八戒笑道:「哥哥莫要說謊。你在東勝神洲,他這裡是西牛賀洲,路程遙遠,隔著萬水千山,海洋也有兩道,怎的與你有親?」行者道:「剛才這夥人都是本境土地、山神,我問他妖怪的原因,他道是牛魔王的兒子,羅剎女養的,名字喚做紅孩兒,號聖嬰大王。想我老孫五百年前大鬧天宮時,遍遊天下名山,尋訪大地豪傑,那牛魔王曾與老孫結七弟兄。一般五六個魔王,止有老孫生得小巧,故此把牛魔王稱為大哥。這妖精是牛魔王的兒子,我與他父親相識,若論將起來,還是他老叔哩,他怎敢害我師父?我們趁早去來。」沙和尚笑道:「哥啊,常言道:『三年不上門,當親也不親』哩。你與他相別五六百年,又不曾往還杯酒,又沒有個節禮相邀,他那裡與你認甚麼親耶?」行者道:「你怎麼這等量人?常言道:『一葉浮萍歸大海,為人何處不相逢。』縱然他不認親,好道也不傷我師父。不望他相留酒席,必定也還我個囫圇唐僧。」

三兄弟各辦虔心,牽著白馬,馬上馱著行李,找大路一直前進。無分晝夜,行了百十里遠近,忽見一松林,林中有一條曲澗,澗下有碧澄澄的活水飛流,那澗梢頭有一座石板橋,通著那廂洞府。行者道:「兄弟,你看那壁廂有石崖磷磷,想必是妖精住處了。我等從眾商議:那個管看守行李、馬匹?那個肯跟我過去降妖?」八戒道:「哥哥,老豬沒甚坐性,我隨你去罷。」行者道:「好,好。」教:「沙僧將馬匹、行李俱潛在樹林深處,小心守護,待我兩個上門去尋師父耶。」那沙僧依命。八戒相隨,與行者各持兵器前來。正是:
\begin{quote}
未煉嬰兒邪火勝,心猿木母共扶持。
\end{quote}

畢竟不知這一去吉凶何如,且聽下回分解。
