
\chapter{心猿識得丹頭 姹女還歸本性}

卻說三藏著妖精送出洞外,沙和尚近前問曰:「師父出來,師兄何在?」八戒道:「他有算計,必定貼換師父出來也。」三藏用手指著妖精道:「你師兄在他肚裡哩。」八戒笑道:「腌臢殺人。在肚裡做甚?出來罷。」行者在裡邊叫道:「張開口,等我出來。」那怪真個把口張開。行者變得小小的,在咽喉之內,正欲出來,又恐他無理來咬,即將鐵棒取出,吹口仙氣,叫:「變!」變作個棗核釘兒,撐住他的上腭子,把身一縱,跳出口外,就把鐵棒順手帶出,把腰一躬,還是原身法像,舉起棒來就打;那妖精也隨手取出兩口寶劍,叮噹架住。兩個在山頭上這場好殺:
\begin{quote}
雙舞劍飛當面架,金箍棒起照頭來。一個是天生猴屬心猿體,一個是地產精靈姹女骸。他兩個,恨沖懷,喜處生仇大會垓。那個要取元陽成配偶,這個要戰純陰結聖胎。棒舉一天寒霧漫,劍迎滿地黑塵篩。因長老,拜如來,恨苦相爭顯大才。水火不投母道損,陰陽難合各分開。兩家鬥罷多時節,地動山搖樹木摧。
\end{quote}

八戒見他們賭鬥,口裡絮絮叨叨,返恨行者,轉身對沙僧道:「兄弟,師兄胡纏。才子在他肚裡,掄起拳來,送他一個滿肚紅,巴開肚皮鑽出來,卻不了帳?怎麼又從他口裡出來,卻與他爭戰,讓他這等猖狂?」沙僧道:「正是。卻也虧了師兄深洞中救出師父,返又與妖精廝戰。且請師父自家坐著,我和你各持兵器,助助大哥,打倒妖精去來。」八戒擺手道:「不不不,他有神通,我們不濟。」沙僧道:「說那裡話,都是大家有益之事,雖說不濟,卻也放屁添風。」

那獃子一時興發,掣了釘鈀,叫聲:「去來。」他兩個不顧師父,一齊駕風趕上,舉釘鈀,使寶杖,望妖精亂打。那妖精戰行者一個已是不能,又見他二人,怎生抵敵,急回頭,抽身就走。行者喝道:「兄弟們趕上。」那妖精見他們趕得緊,即將右腳上花鞋脫下來,吹口仙氣,念個咒語,叫:「變!」即變作本身模樣,使兩口劍舞將來;將身一幌,化一陣清風,徑直回去。這番也只說戰他們不過,顧命而回,豈知又有這般樣事!也是三藏災星未退,他到洞門前牌樓下,卻見唐僧在那裡獨坐,他就近前一把抱住,搶了行李,咬斷韁繩,連人和馬,復又攝將進去不題。

且說八戒閃個空,一鈀把妖精打落地,乃是一隻花鞋。行者看見道:「你這兩個獃子,看著師父罷了,誰要你來幫甚麼功?」八戒道:「沙和尚,如何麼?我說莫來。這猴子好的有些夾腦風,我們替他降了妖怪,返落得他生報怨。」行者道:「在那裡降了妖怪?那妖怪昨日與我戰時,使了一個遺鞋計哄了。你們走了,不知師父如何,我們快去看看。」

三人急回來,果然沒了師父,連行李、白馬一並無蹤。慌得個八戒兩頭亂跑,沙僧前後跟尋,孫大聖亦心焦性燥。正尋覓處,只見那路傍邊斜𢷑著半截兒韁繩。他一把拿起,止不住眼中流淚,放聲叫道:「師父啊,我去時辭別人和馬,回來只見這些繩。」正是那:見鞍思俊馬,滴淚想親人。八戒見他垂淚,不禁仰天大笑。行者罵道:「你這個夯貨,又是要散火哩?」八戒又笑道:「哥啊,不是這話,師父一定又被妖精攝進洞去了。常言道:『事無三不成。』你進洞兩遭了,再進去一遭,管情救出師父來也。」行者揩了眼淚道:「也罷,到此地位,勢不容己,我還進去。你兩個沒了行李、馬匹耽心,卻好生把守洞口。」

好大聖,即轉身跳入裡面,不施變化,就將本身法相。真個是:
\begin{quote}
古怪別腮心裡強,自小為怪神力壯。
高低面賽馬鞍鞽,眼放金光如火亮。
渾身毛硬似鋼針,虎皮裙繫明花響。
上天撞散萬雲飛,下海混起千層浪。
當天倚力打天王,擋退十萬八千將。
官封大聖美猴精,手中慣使金箍棒。
今日西天任顯能,復來洞內扶三藏。
\end{quote}

你看他停住雲光,徑到了妖精宅外。見那門樓門關了,不分好歹,掄鐵棒一下打開,闖將進去。那裡還靜悄悄,全無人跡。東廊下不見唐僧。亭子上桌椅,與各處家火,一件也無。原來他的洞裡週圍有三百餘里,妖精窠穴甚多。前番攝唐僧在此,被行者尋著;今番攝了,又怕行者來尋,當時搬了,不知去向。惱得這行者跌腳搥胸,放聲高叫道:「師父啊,你是個晦氣轉成的唐三藏,災殃鑄就的取經僧。噫!這條路且是走熟了,如何不在?卻教老孫那裡尋找也?」正自吆喝爆燥之間,忽聞得一陣香風撲鼻,他回了性道:「這香煙是從後面飄出,想是在後頭哩。」拽開步,提著鐵棒,走將進去看時,也不見動靜。只見有三間倒坐兒,近後壁卻鋪一張龍吞口雕漆供桌,桌上有一個大流金香爐,爐內有香煙馥郁。那上面供養著一個大金字牌,牌上寫著「尊父李天王之位」;略次些兒,寫著「尊兄哪吒三太子位」。行者見了,滿心歡喜,也不去搜妖怪,找唐僧,把鐵棒捻作個繡花針兒,揌在耳朵裡,掄開手,把那牌子並香爐拿將起來,返雲光,徑出門去。至洞口,唏唏哈哈,笑聲不絕。

八戒、沙僧聽見,掣放洞口,迎著行者道:「哥哥這等歡喜,想是救出師父也?」行者笑道:「不消我們救,只問這牌子要人。」八戒道:「哥啊,這牌子不是妖精,又不會說話,怎麼問他要人?」行者放在地下道:「你們看。」沙僧近前看時,上寫著「尊父李天王之位」、「尊兄哪吒三太子位」。沙僧道:「此意何也?」行者道:「這是那妖精家供養的。我闖入他住居之所,見人物俱無,惟有此牌。想是李天王之女,三太子之妹,思凡下界,假扮妖邪,將我師父攝去。不問他要人,卻問誰要?你兩個且在此把守,等老孫執此牌位,徑上天堂玉帝前告個御狀,教天王爺兒們還我師父。」八戒道:「哥啊,常言道:『告人死罪得死罪。』須是理順,方可為之。況御狀又豈是可輕易告的?你且與我說,怎的告他?」行者笑道:「我有主張:我把這牌位、香爐做個證見,另外再備紙狀兒。」八戒道:「狀兒上怎麼寫?你且念念我聽。」行者道:
\begin{quote}
「告狀人孫悟空,年甲在牒,係東土唐朝西天取經僧唐三藏徒弟。告為假妖攝陷人口事:今有托塔天王李靖同男哪吒太子,閨門不謹,走出親女,在下方陷空山無底洞變化妖邪,迷害人命無數。今將吾師攝陷曲邃之所,渺無尋處。若不狀告,切思伊父子不仁,故縱女氏成精害眾。伏乞憐准,行拘至案,收邪救師,明正其罪,深為恩便。有此上告。」
\end{quote}

八戒、沙僧聞其言,十分歡喜道:「哥啊,告的有理,必得上風。切須早來;稍遲恐妖精傷了師父性命。」行者道:「我快,我快,多時飯熟,少時茶滾就回。」

好大聖,執著這牌位、香爐,將身一縱,駕祥雲,直至南天門外。時有把天門的大力天王與護國天王見了行者,一個個都控背躬身,不敢攔阻,讓他進去。直至通明殿下,有張、葛、許、丘四大天師迎面作禮道:「大聖何來?」行者道:「有紙狀兒,要告兩個人哩。」天師吃驚道:「這個賴皮,不知要告那個?」無奈,將他引入靈霄殿下啟奏,蒙旨宣進。行者將牌位、香爐放下,朝上禮畢,將狀子呈上。葛仙翁接了,鋪在御案。玉帝從頭看了,見這等這等,即將原狀批作聖旨,宣西方長庚太白金星領旨,到雲樓宮宣托塔李天王見駕。行者上前奏道:「望天主好生懲治;不然,又別生事端。」玉帝又吩咐:「原告也去。」行者道:「老孫也去?」四天師道:「萬歲已出了旨意,你可同金星去來。」

行者真個隨著金星,縱雲頭,早至雲樓宮。原來是天王住宅,號雲樓宮。金星見宮門首有個童子侍立。那童子認得金星,即入裡報道:「太白金星老爺來了。」天王遂出迎迓。又見金星捧著旨意,即命焚香。及轉身,又見行者跟入,天王即又作怒。你道他作怒為何?當年行者大鬧天宮時,玉帝曾封天王為降魔大元帥,封哪吒太子為三壇海會之神,帥領天兵,收降行者,屢戰不能取勝。還是五百年前敗陣的仇氣,有些惱他,故此作怒。他且忍不住道:「老長庚,你賫得是甚麼旨意?」金星道:「是孫大聖告你的狀子。」那天王本是煩惱,聽見說個「告」字,一發雷霆大怒道:「他告我怎的?」金星道:「告你假妖攝陷人口事。你焚了香,請自家開讀。」那天王氣呼呼的設了香案,望空謝恩。拜畢,展開旨意看了,原來是這般這般,如此如此,恨得他手撲著香案道:「這個猴頭,他也錯告我了。」金星道:「且息怒。現有牌位、香爐在御前作證,說是你親女哩。」天王道:「我止有三個兒子,一個女兒。大小兒名金吒,侍奉如來,做前部護法;二小兒名木叉,在南海隨觀世音做徒弟;三小兒名哪吒,在我身邊,早晚隨朝護駕。一女年方七歲,名貞英,人事尚未省得,如何會做妖精?不信,抱出來你看。這猴頭著實無禮。且莫說我是天上元勛,封受先斬後奏之職,就是下界小民,也不可誣告。律云:『誣告加三等。』」叫手下:「將縛妖索把這猴頭綑了。」那庭下擺列著巨靈神、魚肚將、藥叉雄帥,一擁上前,把行者綑了。金星道:「李天王莫闖禍啊。我在御前同他領旨意來宣你的人,你那索兒頗重,一時綑壞他,閣氣。」天王道:「金星啊,似他這等詐偽告擾,怎該容他?你且坐下,待我取砍妖刀砍了這個猴頭,然後與你見駕回旨。」金星見他取刀,心驚膽戰,對行者道:「你幹事差了。御狀可是輕易告的?你也不訪的實,似這般亂弄,傷其性命,怎生是好?」行者全然不懼,笑吟吟的道:「老官兒放心,一些沒事。老孫的買賣,原是這等做,一定先輸後贏。」

說不了,天王掄過刀來,望行者劈頭就砍。早有那三太子趕上前,將斬妖劍架住,叫道:「父王息怒。」天王大驚失色。

噫!父見子以劍架刀,就當喝退,怎麼返大驚失色?原來天王生此子時,他左手掌上有個「哪」字,右手掌上有個「吒」字,故名哪吒。這太子三朝兒就下海淨身闖禍,踏倒水晶宮,捉住蛟龍要抽筋為絛子。天王知道,恐生後患,欲殺之。哪吒奮怒,將刀在手,割肉還母,剔骨還父,還了父精母血。一點靈魂,徑到西方極樂世界告佛。佛正與眾菩薩講經,只聞得幢幡寶蓋有人叫道:「救命!」佛慧眼一看,知是哪吒之魂,即將碧藕為骨,荷葉為衣,念動起死回生真言,哪吒遂得了性命。運用神力,法降九十六洞妖魔,神通廣大。後來要殺天王,報那剔骨之仇。天王無奈,告求我佛如來。如來以和為尚,賜他一座玲瓏剔透舍利子如意黃金寶塔。那塔上層層有佛,豔豔光明。喚哪吒以佛為父,解釋了冤仇。所以稱為托塔李天王者,此也。今日因閑在家,未曾托著那塔,恐哪吒有報仇之意,故嚇個大驚失色。

卻即回手,向塔座上取了黃金寶塔,托在手間,問哪吒道:「孩兒,你以劍架住我刀,有何話說?」哪吒棄劍叩頭道:「父王,是有女兒在下界哩。」天王道:「孩兒,我只生了你姊妹四個,那裡又有個女兒哩?」哪吒道:「父王忘了?那女兒原是個妖精。三百年前成怪,在靈山偷食了如來的香花寶燭,如來差我父子天兵,將他拿住。拿住時,只該打死,如來吩咐道:『積水養魚終不釣,深山喂鹿望長生。』當時饒了他性命。積此恩念,拜父王為父,拜孩兒為兄,在下方供設牌位,侍奉香火。不期他又成精,陷害唐僧,卻被孫行者搜尋到巢穴之間,將牌位拿來,就做名告了御狀。此是結拜之恩女,非我同胞之親妹也。」天王聞言,悚然驚訝道:「孩兒,我實忘了。他叫做甚麼名字?」太子道:「他有三個名字:他的本身出處,喚做金鼻白毛老鼠精;因偷香花寶燭,改名喚做半截觀音;如今饒他下界,又改了,喚做地湧夫人是也。」

天王卻才省悟,放下寶塔,便親手來解行者。行者就放起刁來道:「那個敢解我?要便連繩兒擡去見駕,老孫的官事才贏。」慌得天王手軟,太子無言,眾家將委委而退。那大聖打滾撒賴,只要天王去見駕。天王無計可施,哀求金星說個方便。金星道:「古人云:『萬事從寬。』你幹事忒緊了些兒,就把他綑住,又要殺他。這猴子是個有名的賴皮,你如今教我怎的處?若論你令郎講起來,雖是恩女,不是親女,卻也晚親義重,不拘怎生折辨,你也有個罪名。」天王道:「老星怎說個方便,就沒罪了。」金星道:「我也要和解你們,卻只是無情可說。」天王道:「你把那奏招安授官銜的事,說說他也罷了。」真個金星上前,將手摸著行者道:「大聖,看我薄面,解了繩好去見駕。」行者道:「老官兒,不用解,我會滾法,一路滾就滾到也。」金星笑道:「你這猴忒恁寡情。我昔日也曾有些恩義兒到你,我這些些事兒,就不依我?」行者道:「你與我有甚恩義?」金星道:「你當年在花果山為怪,伏虎降龍,強消死籍,聚群妖大肆猖狂,上天欲要擒你,也是老身力奏,降旨招安,把你宣上天堂,封你做弼馬溫。你吃了玉帝仙酒,後又招安,也是老身力奏,封你做『齊天大聖』。你又不守本分,偷桃盜酒,竊老君之丹,如此如此,才得個無滅無生。若不是我,你如何得到今日?」行者道:「古人說得好:『死了莫與老頭兒同墓。』乾淨會揭挑人。我也只是做弼馬溫,鬧天宮罷了,再無甚大事。也罷,也罷,看你老人家面皮,還教他自己來解。」天王才敢向前,解了縛,請行者著衣上坐,一一上前施禮。

行者朝了金星道:「老官兒,何如?我說先輸後贏,買賣兒原是這等做。快催他去見駕,莫誤了我的師父。」金星道:「莫忙,弄了這一會,也吃鍾茶兒去。」行者道:「你吃他的茶,受他的私,賣放犯人,輕慢聖旨,你得何罪?」金星道:「不吃茶,不吃茶。連我也賴將起來了。李天王,快走,快走。」天王那裡敢去,怕他沒的說做有的,放起刁來,口裡胡說亂道,怎生與他折辨?沒奈何,又央金星,教說方便。金星道:「我有一句話兒,你可依我?」行者道:「繩綑刀砍之事,我也通看你面,還有甚話?你說,你說。說得好,就依你;說得不好,莫怪。」金星道:「一日官事十日打。你告了御狀,說妖精是天王的女兒,天王說不是,你兩個只管在御前折辨,反復不已。我說天上一日,下界就是一年。這一年之間,那妖精把你師父陷在洞中,莫說成親,若有個喜花下兒子,也生了一個小和尚兒,卻不誤了大事?」行者低頭想道:「是啊,我離八戒、沙僧,只說多時飯熟、少時茶滾就回,今已弄了這半會,卻不遲了?老官兒,既依你說,這旨意如何回繳?」金星道:「教李天王點兵,同你下去降妖,我去回旨。」行者道:「你怎麼樣回?」金星道:「我只說原告脫逃,被告免提。」行者笑道:「好啊,我倒看你面情罷了,你倒說我脫逃。教他點兵在南天門外等我,我即和你回旨繳狀去。」天王害怕道:「他這一去,若有言語,是臣背君也。」行者道:「你把老孫當甚麼樣人?我也是個大丈夫,『一言既出,駟馬難追』,豈又有污言頂你?」天王即謝了行者。行者與金星回旨。天王點起本部天兵,徑出南天門外。

金星與行者回見玉帝道:「陷唐僧者,乃金鼻白毛老鼠成精,假設天王父子牌位。天王知之,已點兵收怪去了,望天尊赦罪。」玉帝已知此情,降天恩免究。行者即返雲光,到南天門外,見天王、太子佈列天兵等候。噫!那些神將風滾滾,霧騰騰,接住大聖,一齊墜下雲頭,早到了陷空山上。

八戒、沙僧眼巴巴正等,只見天兵與行者來了。獃子迎著天王施禮道:「累及,累及。」天王道:「天蓬元帥,你卻不知。只因我父子受他一炷香,致令妖精無理,困了你師父。來遲莫怪。這個山就是陷空山了?但不知他的洞門還向那邊開?」行者道:「我這條路且是走熟了,只是這個洞叫做個無底洞,周圍有三百餘里,妖精窠穴甚多。前番我師父在那兩滴水的門樓裡,今番靜悄悄,鬼影也沒個,不知又搬在何處去也。」天王道:「任他設盡千般計,難脫天羅地網中。到洞門前再作道理。」大家就行。咦!約有十餘里,就到了那大石邊。行者指那缸口大的門兒道:「兀的便是也。」天王道:「『不入虎穴,安得虎子。』誰敢當先?」行者道:「我當先。」三太子道:「我奉旨降妖,我當先。」那獃子便莽撞起來,高聲叫道:「當頭還要我老豬。」天王道:「不須囉噪,但依我分擺:孫大聖和太子同領著兵將下去,我們三人在口上把守,做個裡應外合,教他上天無路,入地無門,才顯些些手段。」眾人都答應了一聲:「是。」

你看那行者和三太子領了兵將,望洞裡只是一溜。駕起雲光,擡頭一望,果然好個洞啊:
\begin{quote}
依舊雙輪日月,照般一望山川。
珠淵玉井暖弢煙。更有許多堪羨。
疊疊朱樓畫閣,嶷嶷赤壁青田。
三春楊柳九秋蓮。兀的洞天罕見。
\end{quote}

頃刻間,停住了雲光,徑到那妖精舊宅。挨門兒搜尋,吆吆喝喝,一重又一重,一處又一處,把那三百里地,草都踏光了,那見個妖精?那見個三藏?都只說:「這孽畜一定是早出了這洞,遠遠去哩。」那曉得他在那東南黑角落上,望下去,另有個小洞。洞裡一重小小門,一間矮矮屋,盆栽了幾種花,簷傍著數竿竹,黑氣氳氳,暗香馥馥。老怪攝了三藏,搬在這裡,逼住成親。只說行者再也找不著,誰知他命合該休。那些小怪在裡面,一個個嚌嚌嘈嘈,挨挨簇簇。中間有個大膽些的,伸起頸來,望洞外略看一看,一頭撞著個天兵,一聲嚷道:「在這裡!」那行者惱起性來,捻著金箍棒,一下闖將進去。那裡邊窄小,窩著一窟妖精。三太子縱起天兵,一齊擁上,一個個那裡去躲?

行者尋著唐僧,和那龍馬,和那行李。那老怪尋思無路,看著哪吒太子,只是磕頭求命。太子道:「這是玉旨來拿你,不當小可。我父子只為受了一炷香,險些兒和尚拖木頭——做出了寺。」啈聲天兵,取下縛妖索,把那些妖精都綑了。老怪也少不得吃場苦楚。返雲光,一齊出洞。行者口裡嘻嘻嗄嗄。天王掣開洞口,迎著行者道:「今番卻見你師父也。」行者道:「多謝了,多謝了。」就引三藏拜謝天王,次及太子。沙僧、八戒只是要碎剮那老精,天王道:「他是奉玉旨拿的,輕易不得,我們還要去回旨哩。」

一邊天王同三太子領著天兵神將,押住妖精,去奏天曹,聽候發落;一邊行者擁著唐僧,沙僧收拾行李,八戒攏馬,請唐僧騎馬,齊上大路。這正是:
\begin{quote}
割斷絲蘿乾金海,打開玉鎖出樊籠。
\end{quote}

畢竟不知前去何如,且聽下回分解。
