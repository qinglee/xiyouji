
\chapter{心猿正處諸緣伏 劈破傍門見月明}

卻說孫行者按落雲頭,對師父備言菩薩借童子,老君收去寶貝之事。三藏稱謝不已,死心塌地辦虔誠,捨命投西,攀鞍上馬,豬八戒挑著行李,沙和尚攏著馬頭,孫行者執了鐵棒,剖開路,徑下高山前進。說不盡那水宿風餐,披霜冒露。

師徒們行罷多時,前又一山阻路。三藏在那馬上高叫:「徒弟啊,你看那裡山勢崔巍,須是要仔細隄防,恐又有魔障侵身也。」行者道:「師父休要胡思亂想,只要定性存神,自然無事。」三藏道:「徒弟呀,西天怎麼這等難行?我記得離了長安城,在路上春盡夏來,秋殘冬至,有四五個年頭,怎麼還不能得到?」行者聞言,呵呵笑道:「早哩,早哩,還不曾出大門哩。」八戒道:「哥哥不要扯謊。人間就有這般大門?」行者道:「兄弟,我們還在堂屋裡轉哩。」沙僧笑道:「師兄,少說大話嚇我。那裡就有這般大堂屋,卻也沒處買這般大過梁啊。」行者道:「兄弟,若依老孫看時,把這青天為屋瓦,日月作窗櫺,四山五岳為梁柱,天地猶如一敞廳。」八戒聽說道:「罷了,罷了,我們只當轉些時回去罷。」行者道:「不必亂談,只管跟著老孫走路。」

好大聖,橫擔了鐵棒,領定了唐僧,剖開山路,一直前進。那師父在馬上遙觀,好一座山景。真個是:
\begin{quote}
山頂嵯峨摩斗柄,樹梢彷彿接雲霄。青煙堆裡,時聞得谷口猿啼;亂翠陰中,每聽得松間鶴唳。嘯風山魅立溪間,戲弄樵夫;成器狐狸坐崖畔,驚張獵戶。好山!看那八面崖巍,四圍險峻。古怪喬松盤翠蓋,枯摧老樹掛藤蘿。泉水飛流,寒氣透人毛髮冷;巔峰屹,清風射眼夢魂驚。時聽大蟲哮吼,每聞山鳥時鳴。麂鹿成群穿荊棘,往來跳躍;獐結黨尋野食,前後奔跑。佇立草坡,一望並無客旅;行來深凹,四邊俱有豺狼。應非佛祖修行處,盡是飛禽走獸場。
\end{quote}

那師父戰戰兢兢,進此深山,心中悽慘,兜住馬,叫聲:「悟空啊!我
\begin{quote}
自從益智登山盟,王不留行送出城。
路上相逢三棱子,途中催趲馬兜鈴。
尋坡轉澗求荊芥,邁嶺登山拜茯苓。
防己一身如竹瀝,茴香何日拜朝廷?」
\end{quote}

孫大聖聞言,呵呵冷笑道:「師父不必罣念,少要心焦,且自放心前進,還你個功到自然成也。」師徒們玩著山景,信步行時,早不覺紅輪西墜。正是:
\begin{quote}
十里長亭無客走,九重天上現星辰。
八河船隻皆收港,七千州縣盡關門。
六宮五府回官宰,四海三江罷釣綸。
兩座樓頭鐘鼓響,一輪明月滿乾坤。
\end{quote}

那長老在馬上遙觀,只見那山凹裡有樓臺疊疊,殿閣重重。三藏道:「徒弟,此時天色已晚,幸得那壁廂有樓閣不遠,想必是庵觀寺院,我們都到那裡借宿一宵,明日再行罷。」行者道:「師父說得是。不要忙,等我且看好歹如何。」那大聖跳在空中,仔細觀看,果然是座山門。但見:
\begin{quote}
八字磚牆泥紅粉,兩邊門上釘金釘。
疊疊樓臺藏嶺畔,層層宮闕隱山中。
萬佛閣對如來殿,朝陽樓應大雄門。
七層塔屯雲宿霧,三尊佛神現光榮。
文殊臺對伽藍舍,彌勒殿靠大慈廳。
看山樓外青光舞,步虛閣上紫雲生。
松關竹院依依綠,方丈禪堂處處清。
雅雅幽幽供樂事,川川道道喜迴迎。
參禪處有禪僧講,演樂房多樂器鳴。
妙高臺上曇花墜,說法壇前貝葉生。
正是那林遮三寶地,山擁梵王宮。
半壁燈煙光閃灼,一行香靄霧朦朧。
\end{quote}

孫大聖按下雲頭,報與三藏道:「師父,果然是一座寺院,卻好借宿,我們去來。」

這長老放開馬,一直前來,徑到了山門之外。行者道:「師父,這一座是甚麼寺?」三藏道:「我的馬蹄才然停住,腳尖還未出鐙,就問我是甚麼寺,好沒分曉。」行者道:「你老人家自幼為僧,須曾講過儒書,方才去演經法,文理皆通,然後受唐王的恩宥。門上有那般大字,如何不認得?」長老罵道:「潑猢猻!說話無知。我才面西催馬,被那太陽影射,奈何門雖有字,又被塵垢朦朧,所以未曾看見。」行者聞言,把腰兒躬一躬,長了二丈餘高,用手展去灰塵,道:「師父,請看。」上有五個大字,乃是「敕建寶林寺」。行者收了法身,道:「師父,這寺裡誰進去借宿?」三藏道:「我進去。你們的嘴臉醜陋,言語粗疏,性剛氣傲,倘或衝撞了本處僧人,不容借宿,反為不美。」行者道:「既如此,請師父進去,不必多言。」

那長老卻丟了錫杖,解下斗篷,整衣合掌,徑入山門。只見兩邊紅漆欄杆裡面,高坐著一對金剛,裝塑的威儀惡醜:
\begin{quote}
一個鐵面鋼鬚似活容,一個燥眉圜眼若玲瓏。左邊的拳頭骨突如生鐵,右邊的手掌崚嶒賽赤銅。金甲連環光燦爛,明盔繡帶映飄風。西方真個多供佛,石鼎中間香火紅。
\end{quote}

三藏見了,點頭長嘆道:「我那東土,若有人也將泥胎塑這等大菩薩,燒香供養啊,我弟子也不去西天去矣。」正嘆息處,又到了二層山門之內。見有四大天王之像,乃是持國、多聞、增長、廣目,按東北西南風調雨順之意。進了二層門裡,又見有喬松四樹,一樹樹翠蓋蓬蓬,卻如傘狀。忽擡頭,乃是大雄寶殿。那長老合掌皈依,舒身下拜。拜罷起來,轉過佛臺,到於後門之下。又見有倒座觀音普度南海之像。那壁上都是良工巧匠裝塑的那些蝦、魚、蟹、鱉,出頭露尾,跳海水波潮耍子。長老又點頭三五度,感嘆萬千聲道:「可憐啊!鱗甲眾生都拜佛,為人何不肯修行?」

正讚嘆間,又見三門裡走出一個道人。那道人忽見三藏相貌稀奇,丰姿非俗,急趨步上前施禮道:「師父那裡來的?」三藏道:「弟子是東土大唐駕下差來,上西天拜佛求經的。今到寶方,天色將晚,告借一宿。」那道人道:「師父莫怪,我做不得主,我是這裡掃地、撞鐘、打勤勞的道人。裡面還有個管家的老師父哩,待我進去稟他一聲。他若留你,我就出來奉請;若不留你,我卻不敢羈遲。」三藏道:「累及你了。」

那道人急到方丈報道:「老爺,外面有個人來了。」那僧官即起身,換了衣服,按一按毘盧帽,披上袈裟,急開門迎接,問道人:「那裡人來?」道人用手指定道:「那正殿後邊不是一個人?」那三藏光著一個頭,穿一領二十五條達摩衣,足下登一雙拖泥帶水的達公鞋,斜倚在那後門首。僧官見了,大怒道:「道人少打!你豈不知我是僧官,但只有城上來的士夫降香,我方出來迎接?這等個和尚,你怎麼多虛少實,報我接他?看他那嘴臉,不是個誠實的,多是雲遊方上僧,今日天晚,想是要來借宿。我們方丈中,豈容他打攪?教他往前廊下蹲罷了,報我怎麼?」抽身轉去。

長老聞言,滿眼垂淚道:「可憐,可憐!這才是人離鄉賤。我弟子從小兒出家,做了和尚,又不曾拜懺吃葷生歹意,看經懷怒壞禪心;又不曾丟瓦拋磚傷佛殿,阿羅臉上剝真金。噫!可憐啊!不知是那世裡觸傷天地,教我今生常遇不良人。——和尚,你不留我們宿便罷了,怎麼又說這等憊𪬯話,教我們在前道廊下去蹲?此話不與行者說還好,若說了,那猴子進來,一頓鐵棒,把孤拐都打斷你的。」長老道:「也罷,也罷。常言道:『人將禮樂為先。』我且進去問他一聲,看他意下如何?」

那師父踏腳跡,跟他進方丈門裡。只見那僧官脫了衣服,氣呼呼的坐在那裡,不知是念經,又不知是與人家寫法事,見那桌案上有些紙劄堆積。唐僧不敢深入,就立於天井裡,躬身高叫道:「老院主,弟子問訊了。」那和尚就有些不耐煩他進裡邊來的意思,半答不答的還了個禮,道:「你是那裡來的?」三藏道:「弟子乃東土大唐駕下差來,上西天拜活佛求經的。經過寶方,天晚,求借一宿,明日不犯天光就行了。萬望老院主方便方便。」那僧官才欠起身來道:「你是那唐三藏麼?」三藏道:「不敢,弟子便是。」僧官道:「你既往西天取經,怎麼路也不會走?」三藏道:「弟子更不曾走貴處的路。」他道:「正西去,只有四五里遠近,有一座三十里店,店上有賣飯的人家,方便好宿。我這裡不便,不好留你們遠來的僧。」三藏合掌道:「院主,古人有云:『庵觀寺院,都是我方上人的館驛,見山門就有三升米分。』你怎麼不留我,卻是何情?」僧官怒聲叫道:「你這遊方的和尚,便是有些油嘴油舌的說話。」三藏道:「何為油嘴油舌?」僧官道:「古人云:『老虎進了城,家家都閉門。雖然不咬人,日前壞了名。』」三藏道:「怎麼『日前壞了名』?」他道:「向年有幾眾行腳僧,來於山門口坐下。是我見他寒薄,一個個衣破鞋無,光頭赤腳,我嘆他那般襤褸,即忙請入方丈,延之上坐,款待了齋飯,又將故衣各借一件與他,就留他住了幾日。怎知他貪圖自在衣食,更不思量起身,就住了七八個年頭。住便也罷,又幹出許多不公的事來。」三藏道:「有甚麼不公的事?」僧官道:「你聽我說:
\begin{quote}
閑時沿牆拋瓦,悶來壁上扳釘。
冷天向火折窗櫺。夏日拖門攔徑。
幡布扯為腳帶,牙香偷換蔓菁。
常將琉璃把油傾。奪碗奪鍋賭勝。」
\end{quote}

三藏聽言,心中暗道:「可憐啊!我弟子可是那等樣沒脊骨的和尚?」欲待要哭,又恐那寺裡的老和尚笑他,但暗暗扯衣揩淚,忍氣吞聲,急走出去,見了三個徒弟。那行者見師父面上含怒,向前問:「師父,寺裡和尚打你來?」唐僧道:「不曾打。」八戒說:「一定打來;不是,怎麼還有些哭包聲?」那行者道:「罵你來?」唐僧道:「也不曾罵。」行者道:「既不曾打,又不曾罵,你這般苦惱怎麼?好道是思鄉哩?」唐僧道:「徒弟,他這裡不方便。」行者笑道:「這裡想是道士?」唐僧怒道:「觀裡才有道士,寺裡只是和尚。」行者道:「你不濟事。但是和尚,即與我們一般。常言道:『既在佛會下,都是有緣人。』你且坐,等我進去看看。」

好行者,按一按頂上金箍,束一束腰間裙子,執著鐵棒,徑到大雄寶殿上,指著那三尊佛像道:「你本是泥塑金裝假像,內裡豈無感應?我老孫保領大唐聖僧往西天拜佛求取真經,今晚特來此處投宿,趁早與我報名;假若不留我等,就一頓棍打碎金身,教你還現本相泥土。」

這大聖正在前邊發狠,搗叉子亂說,只見一個燒晚香的道人點了幾枝香,來佛前爐裡插。被行者咄的一聲,諕了一跌;爬起來看見臉,又是一跌;嚇得滾滾蹡蹡,跑入方丈裡,報道:「老爺,外面有個和尚來了。」那僧官道:「你這夥道人都少打。一行說教他往前廊下去蹲,又報甚麼?再說打二十。」道人說:「老爺,這個和尚比那個和尚不同:生得惡躁,沒脊骨。」僧官道:「怎的模樣?」道人道:「是個圓眼睛,查耳朵,滿面毛,雷公嘴。手執一根棍子,咬牙狠狠的,要尋人打哩。」僧官道:「等我出去看。」

他即開門,只見行者撞進來了。真個生得醜陋:七高八低孤拐臉,兩隻黃眼睛,一個磕額頭,獠牙往外生。就像屬螃蟹的,肉在裡面,骨在外面。那老和尚慌得把方丈門關了。行者趕上,撲的打破門扇,道:「趕早將乾淨房子打掃一千間,老孫睡覺。」僧官躲在房裡,對道人說:「怪他生得醜麼,原來是說大話折作的這般嘴臉。我這裡連方丈、佛殿、鐘鼓樓、兩廊,共總也不上三百間,他卻要一千間睡覺,卻打那裡來?」道人說:「師父,我也是嚇破膽的人了,憑你怎麼答應他罷。」那僧官戰索索的高叫道:「那借宿的長老,我這小荒山不方便,不敢奉留,往別處去宿罷。」

行者將棍子變得盆來粗細,直壁壁的豎在天井裡,道:「和尚,不方便,你就搬出去。」僧官道:「我們從小兒住的寺,師公傳與師父,師父傳與我輩,我輩要遠繼兒孫。他不知是那裡勾當,冒冒失失的,教我們搬哩。」道人說:「老爺,十分不尷尬,搬出去也罷,扛子打進門來了。」僧官道:「你莫胡說,我們老少眾人四五百名和尚,往那裡搬?搬出去,卻也沒處住。」行者聽見道:「和尚,沒處搬,便著一個出來打樣棍。」老和尚叫道人:「你出去與我打個樣棍來。」那道人慌了道:「爺爺呀!那等個大杠子,教我去打樣棍?」老和尚道:「『養軍千日,用軍一朝。』你怎麼不出去?」道人說:「那杠子莫說打來,若倒下來,壓也壓個肉泥。」老和尚道:「也莫要說壓,只道豎在天井裡,夜晚間走路,不記得啊,一頭也撞個大窟窿。」道人說:「師父,你曉得這般重,卻教我出去打甚麼樣棍?」他自家裡面轉鬧起來。

行者聽見道:「是也禁不得,假若就一棍打殺一個,我師父又怪我行兇了。且等我另尋一個甚麼打與你看看。」忽擡頭,只見方丈門外有一個石獅子,卻就舉起棍來,乒乓一下,打得粉亂麻碎。那和尚在窗眼兒裡看見,就嚇得骨軟觔麻,慌忙往床下拱;道人就往鍋門裡鑽,口中不住叫:「爺爺,棍重,棍重,禁不得,方便,方便!」行者道:「和尚,我不打你。我問你:這寺裡有多少和尚?」僧官戰索索的道:「前後是二百八十五房頭,共有五百個有度牒的和尚。」行者道:「你快去把那五百個和尚都點得齊齊整整,穿了長衣服出去,把我那唐朝的師父接進來,就不打你了。」僧官道:「爺爺,若是不打,便擡也擡進來。」行者道:「趁早去。」僧官叫道人:「你莫說嚇破了膽,就是嚇破了心,便也去與我叫這些人來,接唐僧老爺爺來。」

那道人沒奈何,捨了性命,不敢撞門,從後邊狗洞裡鑽將出去,徑到正殿上,東邊打鼓,西邊撞鐘。鐘鼓一齊響處,驚動了兩廊大小僧眾,上殿問道:「這早還不晚哩,撞鐘打鼓做甚?」道人說:「快換衣服,隨老師父排班,出山門外,迎接唐朝來的老爺。」那眾和尚真個齊齊整整,擺班出門迎接。有的披了袈裟;有的著了偏衫;無的穿著個一口鐘直裰;十分窮的,沒有長衣服,就把腰裙接起兩條披在身上。行者看見道:「和尚,你穿的是甚麼衣服?」和尚見他醜惡,道:「爺爺,不要打,等我說。這是我們城中化的布,此間沒有裁縫,是自家做的個一裹窮。」

行者聞言暗笑,押著眾僧,出山門外跪下。那僧官磕頭高叫道:「唐老爺,請方丈裡坐。」八戒看見道:「師父老大不濟事,你進去時,淚汪汪,嘴上掛得油瓶。師兄怎麼就有此獐智,教他們磕頭來接?」三藏道:「你這個獃子,好不曉禮。常言道:『鬼也怕惡人哩。』」唐僧見他們磕頭禮拜,甚是不過意,上前叫:「列位請起。」眾僧叩頭道:「老爺若和你徒弟說聲方便,不動杠子,就跪一個月也罷。」唐僧叫:「悟空,莫要打他。」行者道:「不曾打;若打,這會已打斷了根矣。」那些和尚卻才起身,牽馬的牽馬,挑擔的挑擔,擡著唐僧,馱著八戒,挽著沙僧,一齊都進山門裡去,卻到後面方丈中,依敘坐下。

眾僧卻又禮拜。三藏道:「院主請起,再不必行禮,作踐貧僧,我和你都是佛門弟子。」僧官道:「老爺是上國欽差,小和尚有失迎接。今到荒山,奈何俗眼不識尊儀,與老爺邂逅相逢。動問老爺:一路上是吃素?是吃葷?我們好去辦飯。」三藏道:「吃素。」僧官道:「徒弟,這個爺爺好的吃葷。」行者道:「我們也吃素,都是胎裡素。」那和尚道:「爺爺呀!這等兇漢也吃素?」有一個膽量大的和尚,近前又問:「老爺既然吃素,煮多少米的飯方夠吃?」八戒道:「小家子和尚,問甚麼?一家煮上一石米。」那和尚都慌了,便去刷洗鍋灶,各房中安排茶飯。高掌明燈,調開桌椅,管待唐僧。

師徒們都吃罷了晚齋,眾僧收拾了家火。三藏稱謝道:「老院主,打攪寶山了。」僧官道:「不敢,不敢。怠慢,怠慢。」三藏道:「我師徒卻在這裡安歇?」僧官道:「老爺不要忙,小和尚自有區處。」叫:「道人,那壁廂有幾個人聽使令的?」道人說:「師父,有。」僧官吩咐道:「你們著兩個去安排草料,與唐老爺喂馬。著幾個去前面把那三間禪堂,打掃乾淨鋪設床帳,快請老爺安歇。」

那些道人聽命,各各整頓齊備,卻來請唐老爺安寢。他師徒們牽馬挑擔,出方丈,徑至禪堂門首看處,只見那裡面燈火光明,兩梢間鋪著四張藤屜床。行者見了,喚那辦草料的道人,將草料擡來,放在禪堂裡面,拴下白馬,教道人都出去。三藏坐在中間。燈下,兩班兒立五百個和尚,都伺候著,不敢撤離。三藏欠身道:「列位請回,貧僧好自在安寢也。」眾僧決不敢退。僧官上前,吩咐大眾:「伏侍老爺安置了再回。」三藏道:「即此就是安置了,都就請回。」眾人卻才敢散去訖。

唐僧舉步出門小解,只見明月當天,叫:「徒弟。」行者、八戒、沙僧都出來侍立。因感這月清光皎潔,玉宇深沉,真是一輪高照,大地分明。對月懷歸,口占一首古風長篇。詩云:
\begin{quote}
皓魄當空寶鏡懸,山河搖影十分全。
瓊樓玉宇清光滿,冰鑑銀盤爽氣旋。
萬里此時同皎潔,一年今夜最明鮮。
渾如霜餅離滄海,卻似冰輪掛碧天。
別館寒窗孤客悶,山村野店老翁眠。
乍臨漢苑驚秋鬢,才到秦樓促晚奩。
庾亮有詩傳晉史,袁宏不寐泛江船。
光浮杯面寒無力,清映庭中健有仙。
處處窗軒吟白雪,家家院宇弄冰弦。
今宵靜玩來山寺,何日相同返故園?
\end{quote}

行者聞言,近前答曰:「師父啊,你只知月色光華,心懷故里,更不知月家之意,乃先天法象之規繩也。月至三十日,陽魂之金散盡,陰魄之水盈輪,故純黑而無光,乃曰『晦』。此時與日相交,在晦朔兩日之間,感陽光而有孕。至初三日一陽現,初八日二陽生,魄中魂半,其平如繩,故曰『上弦』。至今十五日,三陽備足,是以團圓,故曰『望』。至十六日一陰生,二十二日二陰生,此時魂中魄半,其平如繩,故曰『下弦』。至三十日三陰備足,亦當『晦』。此乃先天採煉之意。我等若能溫養二八,九九成功,那時節,見佛容易,返故田亦易也。詩曰:
\begin{quote}
前弦之後後弦前,藥味平平氣象全。
採得歸來爐裡煉,志心功果即西天。」
\end{quote}

那長老聽說,一時解悟,明徹真言。滿心歡喜,稱謝了悟空。沙僧在傍笑道:「師兄此言雖當,只說的是弦前屬陽,弦後屬陰,陰中陽半,得水之金;更不道:
\begin{quote}
水火相攙各有緣,全憑土母配如然。
三家同會無爭競,水在長江月在天。」
\end{quote}

長老聞得,亦開茅塞。正是:理明一竅通千竅,說破無生即是仙。

八戒上前扯住長老道:「師父,莫聽亂講,誤了睡覺。這月啊:
\begin{quote}
缺之不久又團圓,似我生來不十全。
吃飯嫌我肚子大,拿碗又說有黏涎。
他都伶俐修來福,我自痴愚積下緣。
我說你取經還滿三塗業,擺尾搖頭直上天。」
\end{quote}

三藏道:「也罷,徒弟們走路辛苦,先去睡下。等我把這卷經來念一念。」行者道:「師父差了。你自幼出家,做了和尚,小時的經文,那本不熟?卻又領了唐王旨意,上西天見佛,求取大乘真典。如今功未完成,佛未得見,經未曾取,你念的是那卷經兒?」三藏道:「我自出長安,朝朝跋涉,日日奔波,小時的經文恐怕生了。幸今夜得閑,等我溫習溫習。」行者道:「既這等說,我們先去睡也。」他三人各往一張藤床上睡下。長老掩上禪堂門,高剔銀缸,鋪開經本,默默看念。正是那:
\begin{quote}
樓頭初鼓人煙靜,野浦漁舟火滅時。
\end{quote}

畢竟不知那長老怎麼樣離寺,且聽下回分解。
