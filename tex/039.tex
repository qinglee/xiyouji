
\chapter{一粒丹砂天上得 三年故主世間生}

話說那孫大聖頭痛難禁,哀告道:「師父莫念,莫念,等我醫罷。」長老問:「怎麼醫?」行者道:「只除過陰司,查勘那個閻王家有他魂靈,請將來救他。」八戒道:「師父莫信他。他原說不用過陰司,陽世間就能醫活,方見手段哩。」那長老信邪風,又念緊箍兒咒。慌得行者滿口招承道:「陽世間醫罷,陽世間醫罷。」八戒道:「莫要住,只管念,只管念。」行者罵道:「你這獃孽畜,攛道師父咒我哩。」八戒笑得打跌道:「哥耶,哥耶,你只曉得捉弄我,不曉得我也捉弄你捉弄。」行者道:「師父莫念,莫念,待老孫陽世間醫罷。」三藏道:「陽世間怎麼醫?」行者道:「我如今一觔斗雲,撞入南天門裡,不進斗牛宮,不入靈霄殿,徑到那三十三天之上,離恨天宮兜率院內,見太上老君,把他九轉還魂丹求得一粒來,管取救活他也。」

三藏聞言,大喜道:「就去,快來。」行者道:「如今有三更時候罷了,投到回來,好天明了。只是這個人睡在這裡,冷淡冷淡,不像個模樣。須得舉哀人看著他哭,便才好哩。」八戒道:「不消講,這猴子一定是要我哭哩。」行者道:「怕你不哭?你若不哭,我也醫不成。」八戒道:「哥哥,你自去,我自哭罷了。」行者道:「哭有幾樣:若乾著口喊,謂之嚎;扭搜出些眼淚兒來,謂之啕;又要哭得有眼淚,又要哭得有心腸,才算著嚎啕痛哭哩。」八戒道:「我且哭個樣子你看看。」他不知那裡扯個紙條,撚作一個紙撚兒,往鼻孔裡通了兩通,打了幾個涕噴,你看他眼淚汪汪,黏涎答答的,哭將起來。口裡不住的絮絮叨叨,數黃道黑,真個像死了人的一般。哭到那傷情之處,唐長老也淚滴心酸。行者笑道:「正是那樣哀痛,再不許住聲。你這獃子哄得我去了,你就不哭。我還聽哩,若是這等哭便罷,若略住住聲兒,定打二十個孤拐。」八戒笑道:「你去,你去。我這一哭動頭,有兩日哭哩。」沙僧見他數落,便去尋幾枝香來燒獻。行者笑道:「好好好,一家兒都有些敬意,老孫才好用功。」

好大聖,此時有半夜時分,別了他師徒三眾,縱觔斗雲,只入南天門裡。果然也不謁靈霄寶殿,不上那斗牛天宮,一路雲光,徑來到三十三天離恨天兜率宮中。才入門,只見那太上老君正坐在那丹房中,與眾仙童執芭蕉扇,搧火煉丹哩。他見行者來時,即吩咐看丹的童兒:「各要仔細,偷丹的賊又來也。」行者作禮笑道:「老官兒,這等沒搭撒,防備我怎的?我如今不幹那樣事了。」老君道:「你那猴子,五百年前大鬧天宮,把我靈丹偷吃無數,著小聖二郎捉拿上界,送在我丹爐煉了四十九日,炭也不知費了多少。你如今幸得脫身,皈依佛果,保唐僧往西天取經。前者在平頂山上降魔,弄刁難,不與我寶貝,今日又來做甚?」行者道:「前日事,老孫更沒稽遲,將你那五件寶貝當時交還,你反疑心怪我?」

老君道:「你不走路,潛入吾宮怎的?」行者道:「自別後,西遇一方,名烏雞國。那國王被一妖精假裝道士,呼風喚雨,陰害了國王,那妖假變國王相貌,現坐金鑾殿上。是我師父夜坐寶林寺看經,那國王鬼魂參拜我師,敦請老孫與他降妖,辨明邪正。正是老孫思無指實,與弟八戒夜入園中,打破花園,尋著埋藏之所,乃是一眼八角琉璃井內。撈上他的屍首,容顏不改。到寺中見了我師,他發慈悲,著老孫醫救,不許去赴陰司裡求索靈魂,只教在陽世間救治。我想著無處回生,特來參謁。萬望道祖垂憐,把九轉還魂丹借得一千丸兒,與我老孫,搭救他也。」老君道:「這猴子胡說,甚麼一千丸二千丸,當飯吃哩?是那裡土塊捘的,這等容易?咄!快去!沒有!」行者笑道:「百十丸兒也罷。」老君道:「也沒有。」行者道:「十來丸也罷。」老君怒道:「這潑猴卻也纏帳,沒有沒有,出去出去。」行者笑道:「真個沒有,我問別處去求罷。」老君喝道:「去去去!」這大聖拽轉步,往前就走。

老君忽的尋思道:「這猴子憊𪬯哩,說去就去,只怕溜進來就偷。」即命仙童叫回來道:「你這猴子,手腳不穩,我把這還魂丹送你一丸罷。」行者道:「老官兒,既然曉得老孫的手段,快把金丹拿出來,與我四六分分,還是你的造化哩;不然,就送你個皮笊籬——一撈個罄盡。」那老祖取過葫蘆來,倒吊過底子,傾出一粒金丹,遞與行者道:「止有此了,拿去,拿去。送你這一粒,醫活那皇帝,只算你的功果罷。」行者接了道:「且休忙,等我嘗嘗看,只怕是假的,莫被他哄了。」撲的往口裡一丟。慌得那老祖上前扯住,一把揪著頂瓜皮,揝著拳頭,罵道:「這潑猴若要咽下去,就直打殺了。」行者笑道:「嘴臉,小家子樣。那個吃你的哩?能值幾個錢?虛多實少的。在這裡不是?」原來那猴子頦下有嗉袋兒,他把那金丹噙在嗉袋裡,被老祖捻著道:「去罷,去罷,再休來此纏繞。」這大聖才謝了老祖,出離了兜率天宮。

你看他千條瑞靄離瑤闕,萬道祥雲降世塵。須臾間,下了南天門,回到東觀,早見那太陽星上。按雲頭,徑至寶林寺山門外,只聽得八戒還哭哩。忽近前叫聲:「師父。」三藏喜道:「悟空來了,可有丹藥?」行者道:「有。」八戒道:「怎麼得沒有?他偷也去偷人家些來。」行者笑道:「兄弟,你過去罷,用不著你了。你揩揩眼淚,別處哭去。」教沙和尚:「取些水來我用。」沙僧急忙往後面井上,有個方便吊桶,即將半缽盂水遞與行者。行者接了水,口中吐出丹來,安在那皇帝唇裡。兩手扳開牙齒,用一口清水,把金丹沖灌下肚。有半個時辰,只聽他肚裡呼呼的亂響,只是身體不能轉移。行者道:「師父,弄我金丹也不能救活,可是掯殺老孫麼?」三藏道:「豈有不活之理?似這般久死之屍,如何吞得水下?此乃金丹之仙力也。自金丹入腹,卻就腸鳴了,腸鳴乃血脈和動,但氣絕不能迴伸。莫說人在井裡浸了三年,就是生鐵也上鏽了。只是元氣盡絕,得個人度他一口氣便好。」那八戒上前就要度氣,三藏一把扯住道:「使不得,還教悟空來。」那師父甚有主張:原來豬八戒自幼兒傷生作孽吃人,是一口濁氣。惟行者從小修持,咬松嚼柏,吃桃果為生,是一口清氣。這大聖上前,把個雷公嘴,噙著那皇帝口唇,呼的一口氣吹入咽喉,度下重樓,轉明堂,徑至丹田,從湧泉倒返泥垣宮。呼的一聲響喨,那君王氣聚神歸,便翻身,掄拳曲足,叫了一聲:「師父。」雙膝跪在塵埃道:「記得昨夜鬼魂拜謁,怎知道今朝天曉返陽神。」三藏慌忙攙起道:「陛下,不干我事,你且謝我徒弟。」行者笑道:「師父說那裡話,常言道:『家無二主。』你受他一拜兒不虧。」

三藏甚不過意,攙起那皇帝來,同入禪堂。又與八戒、行者、沙僧拜見了,方才按座。只見那本寺的僧人整頓了早齋,卻欲來奉獻,忽見那個水衣皇帝,個個驚張,人人疑說。孫行者跳出來道:「那和尚不要這等驚疑。這本是烏雞國王,乃汝之真主也。三年前被怪害了性命,是老孫今夜救活。如今進他城去,要辨明邪正。若有了齋,擺將來,等我們吃了走路。」眾僧即奉獻湯水,與他洗了面,換了衣服。把那皇帝赭黃袍脫了,本寺僧官將兩領布直裰與他穿了;解下藍田帶,將一條黃絲絛子與他繫了;褪下無憂履,與他一雙舊僧鞋撒了。卻才都吃了早齋,扣背馬匹。

行者問:「八戒,你行李有多重?」八戒道:「哥哥,這行李日逐挑著,倒也不知有多重。」行者道:「你把那一擔兒分為兩擔:將一擔兒你挑著,將一擔兒與這皇帝挑。我們趕早進城幹事。」八戒歡喜道:「造化,造化。當時馱他來,不知費了多少力;如今醫活了,原來是個替身。」那獃子就弄玄虛,將行李分開,就問寺中取條匾擔,輕些的自己挑了,重些的教那皇帝挑著。行者笑道:「陛下,著你那般打扮,挑著擔子,跟我們走走,可虧你麼?」那國王慌忙跪下道:「師父,你是我重生父母一般,莫說挑擔,情願執鞭墜鐙,伏侍老爺,同行上西天去也。」行者道:「不要你去西天,我內中有個緣故。你只挑得四十里進城,待捉了妖精,你還做你的皇帝,我們還取我們的經也。」八戒聽言道:「這等說,他只挑四十里路,我老豬還是長工。」行者道:「兄弟,不要胡說,趁早外邊引路。」

真個八戒領那皇帝前行,沙僧伏侍師父上馬,行者隨後。只見那本寺五百僧人齊齊整整,吹打著細樂,都送出山門之外。行者笑道:「和尚們不必遠送,但恐官家有人知覺,泄漏我的事機,反為不美。快回去,快回去。但把那皇帝的衣服冠帶,整頓乾淨,或是今晚明早,送進城來,我討些封贈賞賜謝你。」眾僧依命各回訖。行者放開大步,趕上師父,一直前來。正是:
\begin{quote}
西方有訣好尋真,金木和同卻煉神。
丹母空懷懞懂夢,嬰兒長恨杌樗身。
必須井底求明主,還要天堂拜老君。
悟得色空還本性,誠為佛度有緣人。
\end{quote}

師徒們在路上,那消半日,早望見城池相近。三藏道:「悟空,前面想是烏雞國了。」行者道:「正是,我們快趕進城幹事。」那師徒進得城來,只見街市上人物齊整,風光鬧熱。早又見鳳閣龍樓,十分壯麗。有詩為證。詩曰:
\begin{quote}
海外宮樓如上邦,人間歌舞若前唐。
花迎寶扇紅雲繞,日照鮮袍翠霧光。
孔雀屏開香靄出,珍珠簾捲彩旗張。
太平景像真堪賀,靜列多官沒奏章。
\end{quote}

三藏下馬道:「徒弟啊,我們就此進朝倒換關文,省得又攏那個衙門費事。」行者道:「說得有理。我兄弟們都進去,人多才好說話。」唐僧道:「都進去,莫要撒村,先行了君臣禮,然後再講。」行者道:「行君臣禮,就要下拜哩。」三藏道:「正是,要行五拜三叩頭的大禮。」行者笑道:「師父不濟。若是對他行禮,誠為不智。你且讓我先走到裡邊,自有處置。等他若有言語,讓我對答。我若拜,你們也拜;我若蹲,你們也蹲。」你看那惹禍的猴王,引至朝門,與閣門大使言道:「我等是東土大唐駕下差來,上西天拜佛求經者。今到此倒換關文,煩大人轉達,是謂不誤善果。」那黃門官即入端門,跪下丹墀,啟奏道:「朝門外有五眾僧人,言是東土唐國欽差上西天拜佛求經,今至此倒換關文,不敢擅入,現在門外聽宣。」那魔王即令傳宣。

唐僧卻同入朝門裡面,那回生的國主隨行。正行,忍不住腮邊墮淚,心中暗道:「可憐!我的銅斗兒江山,鐵圍的社稷,誰知被他陰占了。」行者道:「陛下切莫傷感,恐走漏消息。這棍子在我耳朵裡跳哩,如今決要見功,管取打殺妖魔,掃蕩邪物。這江山不久就還歸你也。」那君王不敢違言,只得扯衣揩淚,捨死相從,徑來到金鑾殿下。又見那兩班文武,四百朝官,一個個威嚴端肅,相貌軒昂。

這行者引唐僧站立在白玉階前,挺身不動。那階下眾官無不悚懼道:「這和尚十分愚濁,怎麼見我王便不下拜?亦不開言呼祝?喏也不唱一個?好大膽無禮。」說不了,只聽得那魔王開口問道:「那和尚是那方來的?」行者昂然答道:「我是南贍部洲東土大唐國奉欽差前往西域天竺國大雷音寺拜活佛求真經者。今到此方,不敢空度,特來倒換通關文牒。」那魔王聞說,心中作怒道:「你東土便怎麼?我不在你朝進貢,不與你國相通,你怎麼見吾抗禮,不行參拜?」行者笑道:「我東土古立天朝,久稱上國,汝等乃下土邊邦。自古道:『上邦皇帝,為父為君;下邦皇帝,為臣為子。』你倒未曾接我,且敢爭我不拜?」那魔王大怒,教文武官:「拿下這野和尚去!」說聲叫「拿」,你看那多官一齊踴躍。這行者喝了一聲,用手一指,教:「莫來!」那一指,就使個定身法,眾官俱莫能行動。真個是校尉階前如木偶,將軍殿上似泥人。

那魔王見他定住了文武多官,急縱身,跳下龍床,就要來拿。猴王暗喜道:「好,正合老孫之意。這一來,就是個生鐵鑄的頭,湯著棍子,也打個窟窿。」正動身,不期傍邊轉出一個救命星來。你道是誰,原來是烏雞國王的太子,急上前扯住那魔王的朝服,跪在面前道:「父王息怒。」妖精問:「孩兒怎麼說?」太子道:「啟父王得知:三年前聞得人說,有個東土唐朝駕下欽差聖僧往西天拜佛求經,不期今日才來到我邦。父王尊性威烈,若將這和尚拿去斬首,只恐大唐有日得此消息,必生嗔怒。你想那李世民自稱王位,一統江山,心尚未足,又過海征伐;若知我王害了他御弟聖僧,一定興兵發馬,來與我王爭敵。奈何兵少將微,那時悔之晚矣。父王依兒所奏,且把那四個和尚,問他個來歷分明,先定他一段不參王駕,然後方可問罪。」

這一篇,原來是太子小心,恐怕來傷了唐僧,故意留住妖魔,更不知行者安排著要打。那魔王果信其言,立在龍床前面,大喝一聲道:「那和尚是幾時離了東土?唐王因甚事著你求經?」行者昂然而答道:「我師父乃唐王御弟,號曰三藏。因唐王駕下有一丞相,姓魏名徵,奉天條夢斬涇河老龍。大唐王夢遊陰司地府,復得回生之後,大開水陸道場,普度冤魂孽鬼。因我師父敷演經文,廣運慈悲,忽得南海觀世音菩薩指教來西。我師父大發弘願,情忻意美,報國盡忠,蒙唐王賜與文牒。那時正是大唐貞觀十三年九月望前三日,離了東土。前至兩界山,收了我做大徒弟,姓孫,名悟空行者;又到烏斯國界高家莊,收了二徒弟,姓豬,名悟能八戒;流沙河界,又收了三徒弟,姓沙,名悟淨和尚;前日在敕建寶林寺,又新收個挑擔的行童道人。」魔王聞說,又沒法搜檢那唐僧,弄巧語盤詰行者,怒目問道:「那和尚,你起初時,一個人離東土,又收了四眾,那三僧可讓,這一道難容。那行童斷然是拐來的。他叫做甚麼名字?有度牒是無度牒?拿他上來取供。」諕得那皇帝戰戰兢兢道:「師父啊!我卻怎的供?」孫行者捻他一把道:「你休怕,等我替你供。」

好大聖,趨步上前,對怪物厲聲高叫道:「陛下,這老道是一個瘖啞之人,卻又有些耳聾。只因他年幼間曾走過西天,認得道路。他的一節兒起落根本,我盡知之,望陛下寬恕,待我替他供罷。」魔王道:「趁早實實的替他供來,免得取罪。」行者道:
\begin{quote}
「供罪行童年且邁,痴聾瘖啞家私壞。
祖居原是此間人,五載之前遭破敗。
天無雨,民乾壞,君王黎庶都齋戒。
焚香沐浴告天公,萬里全無雲靉靆。
百姓饑荒若倒懸,鍾南忽降全真怪。
呼風喚雨顯神通,然後暗將他命害。
推下花園天井中,陰侵龍位人難解。
幸吾來,功果大,起死回生無罣礙。
情願皈依作行童,與僧同去朝西界。
假變君王是道人,道人轉是真王代。」
\end{quote}

那魔王在金鑾殿上聞得這一篇言語,諕得他心頭撞小鹿,面上起紅雲。急抽身就要走路,奈何手內無一兵器。轉回頭,只見一個鎮殿將軍,腰挎一口寶刀,被行者使了定身法,如痴如啞,立在那裡。他近前,奪了這寶刀,就駕雲頭望空而去。氣得沙和尚爆躁如雷,豬八戒高聲喊叫,埋怨行者是一個急猴子:「你就慢說些兒,卻不穩住他了?如今他駕雲逃走,卻往何處追尋?」行者笑道:「兄弟們且莫亂嚷。我等叫那太子下來拜父,嬪后出來拜夫,」卻又念個咒語,解了定身法,「教那多官甦醒回來拜君,方知是真實皇帝。教訴前情,才見分曉,我再去尋他。」好大聖,吩咐八戒、沙僧:「好生保護他君臣、父子、嬪后,與我師父。」只聽說聲去,就不見形影。

他原來跳在九霄雲裡,睜眼四望,看那魔王哩。只見那畜果逃了性命,徑往東北上走哩。行者趕得將近,喝道:「那怪物,那裡去?老孫來了也。」那魔王急回頭,掣出寶刀,高叫道:「孫行者,你好憊𪬯。我來占別人的帝位,與你無干,你怎麼來抱不平,泄漏我的機密?」行者呵呵笑道:「我把你這個大膽的潑怪!皇帝又許你做?你既知我是老孫,就該遠遁,怎麼還刁難我師父,要取甚麼供狀?適才那供狀是也不是?你不要走,是好漢吃我老孫這一棒。」那魔側身躲過,掣寶刀劈面相還。他兩個搭上手,這一場好殺,真是:
\begin{quote}
猴王猛,魔王強,刀迎棒架敢相當。
一天雲霧迷三界,只為當朝立帝王。
\end{quote}

他兩個戰經數合,那妖魔抵不住猴王,急回頭復從舊路跳入城裡,闖在白玉階前兩班文武叢中,搖身一變,即變得與唐三藏一般模樣,並攙手,立在階前。這大聖趕上,就欲舉棒來打。那怪道:「徒弟莫打,是我。」急掣棒要打那個唐僧,卻又道:「徒弟莫打,是我。」一樣兩個唐僧,實難辨認:「倘若一棒打殺妖怪變的唐僧,這個也成了功果;假若一棒打殺我的真實師父,卻怎麼好?」只得停手,叫八戒、沙僧問道:「果然那一個是怪,那一個是我的師父?你指與我,我好打他。」八戒道:「你在半空中相打相嚷,我瞥瞥眼就見兩個師父,也不知誰真誰假。」

行者聞言,捻訣念聲咒語,叫那護法諸天、六丁六甲、五方揭諦、四值功曹、一十八位護駕伽藍、當坊土地、本境山神道:「老孫至此降妖,妖魔變作我師父,氣體相同,實難辨認。汝等暗中知會者,請師父上殿,讓我擒魔。」原來那妖怪善騰雲霧,聽得行者言語,急撒手跳上金鑾寶殿。這行者舉起棒望唐僧就打。可憐!若不是喚那幾位神來,這一下,就是二十個唐僧,也打為肉醬!多虧眾神架住鐵棒道:「大聖,那怪會騰雲,先上殿去了。」行者趕上殿,他又跳將下來扯住唐僧,在人叢裡又混了一混,依然難認。

行者心中不快,又見那八戒在傍冷笑,行者大怒道:「你這夯貨怎的?如今有兩個師父,你有得叫,有得應,有得伏侍哩,你這般歡喜得緊!」八戒笑道:「哥啊,說我獃,你比我又獃哩。師父既不認得,何勞費力?你且忍些頭疼,叫我師父念念那話兒,我與沙僧各攙一個聽著。若不會念的,必是妖怪,有何難也?」行者道:「兄弟,虧你也。正是,那話兒只有三人記得。原是我佛如來心苗上所發,傳與觀世音菩薩,菩薩又傳與我師父,便再沒人知道。也罷,師父,念念。」真個那唐僧就念起來。那魔王怎麼知得,口裡胡哼亂哼。八戒道:「這哼的卻是妖怪了。」他放了手,舉鈀就築。那魔王縱身跳起,踏著雲頭便走。

好八戒,喝一聲,也駕雲頭趕上。慌得那沙和尚丟了唐僧,也掣出寶杖來打。唐僧才停了咒語。孫大聖忍著頭疼,揝著鐵棒,趕在空中。呀!這一場,三個狠和尚,圍住一個潑妖魔。那魔王被八戒、沙僧使釘鈀、寶杖左右攻住了。行者笑道:「我要再去,當面打他,他卻有些怕我,只恐他又走了。等我老孫跳高些,與他個搗蒜打,結果了他罷。」

這大聖縱祥光,起在九霄,正欲下個切手,只見那東北上,一朵彩雲裡面,厲聲叫道:「孫悟空,且休下手。」行者回頭看處,原來是文殊菩薩。急收棒,上前施禮道:「菩薩,那裡去?」文殊道:「我來替你收這個妖怪的。」行者謝道:「累煩了。」那菩薩袖中取出照妖鏡,照住了那怪的原身。行者才招呼八戒、沙僧齊來見了菩薩。卻將鏡子裡看處,那魔王生得好不兇惡:
\begin{quote}
眼似琉璃盞,頭若煉炒缸。渾身三伏靛,四爪九秋霜。搭拉兩個耳,一尾掃帚長。青毛生銳氣,紅眼放金光。匾牙排玉板,圓鬚挺硬槍。鏡裡觀真像,原是文殊一個獅猁王。
\end{quote}

行者道:「菩薩,這是你坐下的一個青毛獅子,卻怎麼走將來成精,你就不收服他?」菩薩道:「悟空,他不曾走,他是佛旨差來的。」行者道:「這畜類成精,侵奪帝位,還奉佛旨差來。似老孫保唐僧受苦,就該領幾道敕書。」菩薩道:「你不知道。當初這烏雞國王好善齋僧,佛差我來度他歸西,早證金身羅漢。因是不可原身相見,變做一種凡僧,問他化些齋供。被吾幾句言語相難,他不識我是個好人,把我一條繩綑了,送在那御水河中,浸了我三日三夜。多虧六甲金身救我歸西,奏與如來。如來將此怪令到此處推他下井,浸他三年,以報吾三日水災之恨。『一飲一啄,莫非前定。』今得汝等來此,成了功績。」行者道:「你雖報了甚麼『一飲一啄』的私仇,但那怪物不知害了多少人也。」菩薩道:「也不曾害人。自他到後,這三年間,風調雨順,國泰民安,何害人之有?」行者道:「固然如此,但只三宮娘娘與他同眠同起,點污了他的身體,壞了多少綱常倫理,還叫做不曾害人?」菩薩道:「點污他不得,他是個騸了的獅子。」八戒聞言,走近前,就摸了一把。笑道:「這妖精真個是糟鼻子不吃酒——枉擔其名了。」行者道:「既如此,收了去罷。若不是菩薩親來,決不饒他性命。」

那菩薩卻念個咒,喝道:「畜生,還不皈正,更待何時!」那魔王才現了原身。菩薩放蓮花罩定妖魔,坐在背上,踏祥光辭了行者。咦!
\begin{quote}
徑轉五臺山上去,寶蓮座下聽談經。
\end{quote}

畢竟不知那唐僧師徒怎的出城,且聽下回分解。
