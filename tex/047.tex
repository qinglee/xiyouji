
\chapter{聖僧夜阻通天水 金木垂慈救小童}

卻說那國王倚著龍床,淚如泉湧,只哭到天晚不住。行者上前高呼道:「你怎麼這等昏亂?見放著那道士的屍骸,一個是虎,一個是鹿,那羊力是一個羚羊。不信時,撈上骨頭來看,那裡人有那樣骷髏?他本是成精的山獸,同心到此害你,因見氣數還旺,不敢下手。若再過二年,你氣數衰敗,他就害了你性命,把你江山一股兒盡屬他了。幸我等早來,除妖邪救了你命。你還哭甚?哭甚?急打發關文,送我出去。」國王聞此,方才省悟。那文武多官俱奏道:「死者果然是白鹿、黃虎,油鍋裡果是羊骨。聖僧之言,不可不聽。」國王道:「既是這等,感謝聖僧。今日天晚,」教:「太師,且請聖僧至智淵寺。明日早朝,大開東閣,教光祿寺安排素淨筵宴酬謝。」果送至寺裡安歇。

次日五更時候,國王設朝,聚集多官,傳旨:「快出招僧榜文,四門各路張掛。」一壁廂大排筵宴,擺駕出朝,至智淵寺門外,請了三藏等,共入東閣赴宴,不在話下。

卻說那脫命的和尚聞有招僧榜,個個欣然,都入城來尋孫大聖,交納毫毛謝恩。這長老散了宴,那國王換了關文,同皇后嬪妃、兩班文武,送出朝門。只見那些和尚跪拜道傍,口稱:「齊天大聖爺爺,我等是沙灘上脫命僧人。聞知爺爺掃除妖孽,救拔我等,又蒙我王出榜招僧,特來交納毫毛,叩謝天恩。」行者笑道:「汝等來了幾何?」僧人道:「五百名,半個不少。」行者將身一抖,收了毫毛。對君臣僧俗人說道:「這些和尚,實是老孫放了;車輛是老孫運轉雙關,穿夾脊,捽碎了;那兩個妖道也是老孫打死了。今日滅了妖邪,方知是禪門有道。向後來,再不可胡為亂信。望你把三教歸一:也敬僧,也敬道,也養育人才。我保你江山永固。」國王依言,感謝不盡,遂送唐僧出城去訖。

這一去,只為慇懃經三藏,努力修持光一元。曉行夜住,渴飲飢餐,不覺的春盡夏殘,又是秋光天氣。一日,天色已晚,唐僧勒馬道:「徒弟,今宵何處安身也?」行者道:「師父,出家人莫說那在家人的話。」三藏道:「在家人怎麼?出家人怎麼?」行者道:「在家人,這時候溫床暖被,懷中抱子,腳後蹬妻,自自在在睡覺。我等出家人,那裡能夠?便是要帶月披星,餐風宿水,有路且行,無路方住。」八戒道:「哥哥,你只知其一,不知其二。如今路多嶮峻,我挑著重擔,著實難走,須要尋個去處,好眠一覺,養養精神,明日方好捱擔;不然,卻不累倒我也?」行者道:「趁月光再走一程,到有人家之所再住。」師徒們沒奈何,只得相隨行者往前。

又行不多時,只聽得滔滔浪響。八戒道:「罷了,來到盡頭路了。」沙僧道:「是一股水擋住也。」唐僧道:「卻怎生得渡?」八戒道:「等我試之,看深淺何如。」三藏道:「悟能,你休亂談,水之淺深,如何試得?」八戒道:「尋一個鵝卵石,拋在當中。若是濺起水泡來,是淺;若是骨都都沉下有聲,是深。」行者道:「你去試試看。」那獃子摸了一塊石頭,望水中拋去,只聽得骨都都泛起魚津,沉下水底。他道:「深深深,去不得!」唐僧道:「你雖試得深淺,卻不知有多少寬闊。」八戒道:「這個卻不知,不知。」行者道:「等我看看。」好大聖,縱觔斗雲,跳在空中,定睛觀看,但見那:
\begin{quote}
洋洋光浸月,浩浩影浮天。
靈派吞華岳,長流貫百川。
千層洶浪滾,萬疊峻波顛。
岸口無漁火,沙頭有鷺眠。
茫然渾似海,一望更無邊。
\end{quote}

急收雲頭,按落河邊道:「師父,寬哩,寬哩,去不得!老孫火眼金睛,白日裡常看千里,凶吉曉得是;夜裡也還看三五百里。如今通看不見邊岸,怎定得寬闊之數?」

三藏大驚,口不能言,聲音哽咽道:「徒弟啊,似這等怎了?」沙僧道:「師父莫哭。你看那水邊立的,可不是個人麼?」行者道:「想是扳罾的漁人,等我問他去來。」拿了鐵棒,兩三步,跑到面前看處,呀!不是人,是一面石碑。碑上有三個篆文大字,下邊兩行有十個小字。三個大字乃「通天河」,十個小字乃「徑過八百里,亙古少行人」。行者叫:「師父,你來看看。」三藏看見,滴淚道:「徒弟呀,我當年別了長安,只說西天易走,那知道妖魔阻隔,山水迢遙。」

八戒道:「師父,你且聽,是那裡鼓鈸聲音?想是做齋的人家。我們且去趕些齋飯吃,問個渡口尋舡,明日過去罷。」三藏馬上聽得,果然有鼓鈸之聲:「卻不是道家樂器,足是我僧家舉事。我等去來。」行者在前引馬,一行聞響而來。那裡有甚正路,沒高沒低,漫過沙灘,望見一簇人家住處,約摸有四五百家,卻也都住得好。但見:
\begin{quote}
倚山通路,傍岸臨溪。處處柴扉掩,家家竹院關。沙頭宿鷺夢魂清,柳外啼鳴喉舌冷。短笛無聲,寒砧不韻。紅蓼枝搖月,黃蘆葉鬥風。陌頭村犬吠疏籬,渡口老漁眠釣艇。燈火稀,人煙靜,半空皎月如懸鏡。忽聞一陣白蘋香,卻是西風隔岸送。
\end{quote}

三藏下馬,只見那路頭上有一家兒,門外豎一首幢幡,內裡有燈燭熒煌,香煙馥郁。三藏道:「悟空,此處比那山凹河邊卻是不同。在人間屋簷下,可以遮得冷露,放心穩睡。你都莫來,讓我先到那齋公門首告求。若肯留我,我就招呼汝等;假若不留,你卻休要撒潑。汝等臉嘴醜陋,只恐諕了人,闖出禍來,卻倒無住處矣。」行者道:「說得有理。請師父先去,我們在此守待。」

那長老才摘了斗笠,光著頭,抖抖褊衫,拖著錫杖,徑來到人家門外。見那門半開半掩,三藏不敢擅入。聊站片時,只見裡面走出一個老者,項下掛著數珠,口念阿彌陀佛,徑自來關門。慌得這長老合掌高叫:「老施主,貧僧問訊了。」那老者還禮道:「你這和尚,卻來遲了。」三藏道:「怎麼說?」老者道:「來遲無物了。早來啊,我舍下齋僧,盡飽吃飯,熟米三升,白布一段,銅錢十文。你怎麼這時候才來?」三藏躬身道:「老施主,貧僧不是趕齋的。」老者道:「既不趕齋,來此何幹?」三藏道:「我是東土大唐欽差往西天取經者,今到貴處,天色已晚。聽得府上鼓鈸之聲,特來告借一宿,天明就行也。」那老者搖手道:「和尚,出家人休打誑語。東土大唐,到我這裡,有五萬四千里路。你這等單身,如何來得?」三藏道:「老施主見得最是。但我還有三個小徒,逢山開路,遇水疊橋,保護貧僧,方得到此。」老者道:「既有徒弟,何不同來?」教:「請,請,我舍下有處安歇。」三藏回頭,叫聲:「徒弟,這裡來。」

那行者本來性急,八戒生來粗魯,沙僧卻也莽撞,三個人聽得師父招呼,牽著馬,挑著擔,不問好歹,一陣風,闖將進去。那老者看見,諕得跌倒在地,口裡只說:「是妖怪來了,妖怪來了。」三藏攙起道:「施主莫怕,不是妖怪,是我徒弟。」老者戰兢兢道:「這般好俊師父,怎麼尋這樣醜徒弟。」三藏道:「雖然相貌不中,卻倒會降龍伏虎,捉怪擒妖。」老者似信不信的,扶著唐僧慢走。

卻說那三個兇頑闖入廳房上,拴了馬,丟下行李。那廳中原有幾個和尚念經,八戒掬著長嘴喝道:「那和尚,念的是甚麼經?」那些和尚聽見問了一聲,忽然擡頭:
\begin{quote}
觀看外來人,嘴長耳朵大。
身粗背膊寬,聲響如雷咋。
行者與沙僧,容貌更醜陋。
廳堂幾眾僧,無人不害怕。
闍黎還念經,班首教行罷。
難顧磬和鈴,佛像且丟下。
一齊吹息燈,驚散光乍乍。
跌跌與爬爬,門限何曾跨。
你頭撞我頭,似倒葫蘆架。
清清好道場,翻成大笑話。
\end{quote}

這兄弟三人見那些人跌跌爬爬,鼓著掌哈哈大笑。那些僧越加悚懼,磕頭撞腦,各顧性命,通跑淨了。

三藏攙那老者,走上廳堂,燈火全無,三人嘻嘻哈哈的還笑。唐僧罵道:「這潑物,十分不善。我朝朝教誨,日日叮嚀。古人云:『不教而善,非聖而何?教而後善,非賢而何?教亦不善,非愚而何?』汝等這般撒潑,誠為至下至愚之類。走進門不知高低,諕倒了老施主,驚散了念經僧,把人家好事都攪壞了,卻不是墮罪與我?」說得他們不敢回言。那老者方信是他徒弟,急回頭作禮道:「老爺,沒大事,沒大事。才然關了燈,散了花,佛事將收也。」八戒道:「既是了帳,擺出滿散,酒飯來,我們吃了睡覺。」老者叫:「掌燈來,掌燈來。」家裡人聽得,大驚小怪道:「廳上念經,有許多香燭,如何又教掌燈?」幾個僮僕出來看時,這裡黑洞洞的,即便點火把燈籠,一擁而至。忽擡頭見八戒、沙僧,慌得丟了火把,忽抽身關了中門。往裡嚷道:「妖怪來了!妖怪來了!」

行者拿起火把,點上燈燭,扯過一張交椅,請唐僧坐在上面;他兄弟們坐在兩傍,那老者坐在前面。正敘坐間,只聽得裡面門開處,又走出一個老者,拄著拐杖道:「是甚麼邪魔,黑夜裡來我善門之家?」前面坐的老者,急起身迎到屏門後道:「哥哥莫嚷,不是邪魔,乃東土大唐取經的羅漢。徒弟們相貌雖兇,果然是相惡人善。」那老者方才放下拄杖,與他四位行禮。禮畢,也坐了面前,叫:「看茶來。排齋。」連叫數聲,幾個僮僕戰戰兢兢,不敢攏帳。

八戒忍不住問道:「老者,你這盛价兩邊走怎的?」老者道:「教他們捧齋來侍奉老爺。」八戒道:「幾個人伏侍?」老者道:「八個人。」八戒道:「這八個人伏侍那個?」老者道:「伏侍你四位。」八戒道:「那白面師父只消一個人,毛臉雷公嘴的只消兩個人,那晦氣臉的要八個人,我得二十個人伏侍方夠。」老者道:「這等說,想是你的食腸大些。」八戒道:「也將就看得過。」老者道:「有人,有人。」七大八小,就叫出有三四十人出來。

那和尚與老者一問一答的講話,眾人方才不怕。卻將上面排了一張桌,請唐僧上坐;兩邊擺了三張桌,請他三位坐;前面一張桌,坐了二位老者。先排上素果品菜蔬,然後是麵飯、米飯、閑食、粉湯,排得齊齊整整。唐長老舉起箸來,先念一卷《啟齋經》。那獃子一則有些急吞,二來有些餓了,那裡等唐僧經完,拿過紅漆木碗來,把一碗白米飯撲的丟下口去,就了了。傍邊小的道:「這位老爺忒沒算計,不籠饅頭,怎的把飯籠了,卻不污了衣服?」八戒笑道:「不曾籠,吃了。」小的道:「你不曾舉口,怎麼就吃了?」八戒道:「兒子們便說謊,分明吃了;不信,再吃與你看。」那小的們又端了碗,盛一碗遞與八戒。獃子幌一幌,又丟下口去就了了。眾僮僕見了道:「爺爺呀!你是磨磚砌的喉嚨,著實又光又溜。」那唐僧一卷經還未完,他已五六碗過手了。然後卻才同舉箸,一齊吃齋。獃子不論米飯麵飯、果品閑食,只情一撈,亂噇,口裡還嚷:「添飯,添飯。」漸漸不見來了。行者叫道:「賢弟,少吃些罷,也強似在山凹裡忍餓,將就夠得半飽也好了。」八戒道:「嘴臉。常言道:『齋僧不飽,不如活埋』哩。」行者教:「收了家火,莫睬他。」二老者躬身道:「不瞞老爺說,白日裡倒也不怕,似這大肚子長老,也齋得起百十眾。只是晚了,收了殘齋,只蒸得一石麵飯、五斗米飯與幾桌素食,要請幾個親鄰與眾僧們散福。不期你列位來,諕得眾僧跑了,連親鄰也不曾敢請,盡數都供奉了列位。如不飽,再教蒸去。」八戒道:「再蒸去,再蒸去。」

話畢,收了家火桌席。三藏拱身,謝了齋供,才問:「老施主高姓?」老者道:「姓陳。」三藏合掌道:「這是我貧僧華宗了。」老者道:「老爺也姓陳?」三藏道:「是,俗家也姓陳。請問適才做的甚麼齋事?」八戒笑道:「師父問他怎的,豈不知道?必然是青苗齋、平安齋、了場齋罷了。」老者道:「不是,不是。」三藏又問:「端的為何?」老者道:「是一場預修亡齋。」八戒笑得打跌道:「公公忒沒眼力。我們是扯謊架橋哄人的大王,你怎麼把這謊話哄我?和尚家豈不知齋事?只有個預修寄庫齋、預修填還齋,那裡有個『預修亡齋』的?你家人又不曾有死的,做甚亡齋?」

行者聞言,暗喜道:「這獃子乖了些也。——老公公,你是錯說了。怎麼叫做『預修亡齋』?」那二位欠身道:「你等取經,怎麼不走正路,卻蹡到我這裡來?」行者道:「走的是正路,只見一股水擋住,不能得渡,因聞鼓鈸之聲,特來造府借宿。」老者道:「你們到水邊,可曾見些甚麼?」行者道:「止見一面石碑,上書『通天河』三字,下書『徑過八百里,亙古少人行』十字,再無別物。」老者道:「再往上岸走走,好的離那碑記只有里許,有一座靈感大王廟,你不曾見?」行者道:「未見。請公公說說,何為靈感?」那兩個老者一齊垂淚道:「老爺啊,那大王:
\begin{quote}
感應一方興廟宇,威靈千里祐黎民。
年年莊上施甘雨,歲歲村中落慶雲。」
\end{quote}

行者道:「施甘雨,落慶雲,也是好意思,你卻這等傷情煩惱,何也?」那老者跌腳搥胸,哏了一聲道:「老爺啊,
\begin{quote}
雖則恩多還有怨,縱然慈惠卻傷人。
只因要吃童男女,不是昭彰正直神。」
\end{quote}

行者道:「要吃童男女麼?」老者道:「正是。」行者道:「想必輪到你家了?」老者道:「今年正到舍下。我們這裡有百家人家居住。此處屬車遲國元會縣所管,喚做陳家莊。這大王一年一次祭賽,要一個童男、一個童女、豬羊牲醴供獻他。他一頓吃了,保我們風調雨順;若不祭賽,就來降禍生災。」行者道:「你府上幾位令郎?」老者搥胸道:「可憐,可憐!說甚麼令郎,羞殺我等。這個是我舍弟,名喚陳清。老拙叫做陳澄。我今年六十三歲,他今年五十八歲,兒女上都艱難。我五十歲上還沒兒子,親友們勸我納了一妾,沒奈何,尋下一房,生得一女,今年才交八歲,取名喚做一秤金。」八戒道:「好貴名。怎麼叫做一秤金?」老者道:「我只兒女艱難,修橋補路,建寺立塔,佈施齋僧,有一本帳目,那裡使三兩,那裡使五兩。到生女之年,卻好用過有三十斤黃金。三十斤為一秤,所以喚做一秤金。」行者道:「那個的兒子麼?」老者道:「舍弟有個兒子,也是偏出,今年七歲了,取名喚做陳關保。」行者問:「何取此名?」老者道:「家下供養關聖爺爺,因在關爺之位下求得這個兒子,故名關保。我兄弟二人,年歲百二,止得這兩個人種,不期輪次到我家祭賽,所以不敢不獻。故此父子之情,難割難捨,先與孩兒做個超生道場。故曰『預修亡齋』者,此也。」

三藏聞言,止不住腮邊淚下道:「這正是古人云:『黃梅不落青梅落,老天偏害沒兒人。』」行者笑道:「等我再問他。老公公,你府上有多大家當?」二老道:「頗有些兒:水田有四五十頃,旱田有六七十頃,草場有八九十處;水黃牛有二三百頭,驢馬有三二十匹,豬羊雞鵝無數。舍下也有吃不著的陳糧,穿不了的衣服。家財產業,也盡得數。」行者道:「你這等家業,也虧你省將起來的。」老者道:「怎見我省?」行者道:「既有這家私,怎麼捨得親生兒女祭賽?拚了五十兩銀子,可買一個童男;拚了一百兩銀子,可買一個童女。連絞纏不過二百兩之數,可就留下自己兒女後代,卻不是好?」二老滴淚道:「老爺,你不知道。那大王甚是靈感,常來我們人家行走。」行者道:「他來行走,你們看見他是甚麼嘴臉?有幾多長短?」二老道:「不見其形,只聞得一陣香風,就知是大王爺爺來了,即忙滿斗焚香,老少望風下拜。他把我們這人家匙大碗小之事,他都知道;老幼生時年月,他都記得。只要親生兒女,他方受用。不要說二三百兩沒處買,就是幾千萬兩,也沒處買這般一模一樣同年同月的兒女。」

行者道:「原來這等。也罷,也罷,你且抱你令郎出來,我看看。」那陳清急入裡面,將關保兒抱出廳上,放在燈前。小孩兒那知死活,籠著兩袖果子,跳跳舞舞的吃著耍子。行者見了,默默念聲咒語,搖身一變,變作那關保兒一般模樣。兩個孩兒攙著手,在燈前跳舞。諕得那老者慌忙跪著。唐僧道:「老爺,不當人子,不當人子。」這老者道:「這位老爺才然說話,怎麼就變作我兒一般模樣,叫他一聲,齊應齊走?卻折了我們年壽,請現本相,請現本相。」行者把臉抹了一把,現了本相。那老者跪在面前道:「老爺原來有這樣本事。」行者笑道:「可像你兒子麼?」老者道:「像像像,果然一般嘴臉,一般聲音,一般衣服,一般長短。」行者道:「你還沒細看哩。取秤來稱稱,可與他一般輕重?」老者道:是是是,是一般重。」行者道:「似這等可祭賽得過麼?」老者道:「忒好,忒好,祭得過了。」

行者道:「我今替這個孩兒性命,留下你家香煙後代,我去祭賽那大王去也。」那陳清跪地磕頭道:「老爺果若慈悲替得,我送白銀一千兩,與唐老爺做盤纏往西天去。」行者道:「就不謝謝老孫?」老者道:「你已替祭,沒了你也。」行者道:「怎的得沒了?」老者道:「那大王吃了。」行者道:「他敢吃我?」老者道:「不吃你,好道嫌腥。」行者笑道:「任從天命。吃了我,是我的命短;不吃,是我的造化。我與你祭賽去。」

那陳清只管磕頭相謝,又允送銀五百兩。惟陳澄也不磕頭,也不說謝,只是倚著那屏門痛哭。行者知之,上前扯住道:「老大,你這不允我,不謝我,想是捨不得你女兒麼?」陳澄才跪下道:「是,捨不得。敢蒙老爺盛情,救替了我侄子也夠了。但只是老拙無兒,止此一女,就是我死之後,他也哭得痛切,怎麼捨得?」行者道:「你快去蒸上五斗米的飯,整治些好素菜,與我那長嘴師父吃。教他變作你的女兒,我兄弟同去祭賽。索性行個陰騭,救你兩個兒女性命,如何?」

那八戒聽得此言,心中大驚道:「哥哥,你要弄精神,不管我死活,就要攀扯我。」行者道:「賢弟,常言道:『雞兒不吃無工之食。』你我進門,感承盛齋,你還嚷吃不飽哩,怎麼就不與人家救些患難?」八戒道:「哥啊,變化的事情,我卻不會哩。」行者道:「你也有三十六般變化,怎麼不會?」三藏叫:「悟能,你師兄說得最是,處得甚當。常言:『救人一命,勝造七級浮屠。』一則感謝厚情;二來當積陰德;況涼夜無事,你兄弟耍耍去來。」八戒道:「你看師父說的話,我只會變山,變樹,變石頭,變癩象,變水牛,變大胖漢還可;若變小女兒,有幾分難哩。」

行者道:「老大莫信他,抱出你令愛來看。」那陳澄急入裡邊,抱將一秤金孩兒,到了廳上。一家子妻妾大小,不分老幼內外,都出來磕頭禮拜,只請救孩兒性命。那女兒頭上戴一個八寶垂珠的花翠箍;身上穿一件紅閃黃的紵絲襖,上套著一件官綠緞子棋盤領的披風;腰間繫一條大紅花絹裙;腳下踏一雙蝦蟆頭淺紅紵絲鞋;腿上穿兩隻綃金膝褲兒。也拿著果子吃哩。行者道:「八戒,這就是女孩兒。你快變的像他,我們祭賽去。」八戒道:「哥呀,似這般小巧俊秀,怎變?」行者叫:「快些,莫討打。」八戒慌了道:「哥哥不要打,等我變了看。」

這獃子念動咒語,把頭搖了幾搖,叫:「變!」真個變過頭來,就也像女孩兒面目,只是肚子胖大,郎伉不像。行者笑道:「再變變。」八戒道:「憑你打了罷,變不過來,奈何?」行者道:「莫成是丫頭的頭,和尚的身子?弄的這等不男不女,卻怎生是好?你可佈起罡來。」他就吹他一口仙氣,果然即時把身子變過,與那孩兒一般。便教:「二位老者,帶你寶眷與令郎、令愛進去,不要錯了。一會家,我兄弟躲懶討乖,走進去,轉難識認。你將好果子與他吃,不可教他哭叫,恐大王一時知覺,走了風汛。等我兩人耍子去也。」

好大聖,吩咐沙僧保護唐僧:「我變作陳關保,八戒變作一秤金。」二人俱停當了,卻問:「怎麼供獻?還是綑了去,是綁了去?蒸熟了去,是剁碎了去?」八戒道:「哥哥,莫要弄我,我沒這個手段。」老者道:「不敢,不敢。只是用兩個紅漆丹盤,請二位坐在盤內,放在桌上,著兩個後生擡一張桌子,把你們擡上廟去。」行者道:「好好好,拿盤子出來,我們試試。」那老者即取出兩個丹盤,行者與八戒坐上。四個後生擡起兩張桌子,往天井裡走走兒,又擡回放在堂上。行者歡喜道:「八戒,像這般子走走耍耍,我們也是上臺盤的和尚了。」八戒道:「若是擡了去,還擡回來,兩頭擡到天明,我也不怕。只是擡到廟裡,就要吃哩,這個卻不是耍子!」行者道:「你只看著我,剗著吃我時,你就走了罷。」八戒道:「知他怎麼吃哩?如先吃童男,我便好跑;如先吃童女,我卻如何?」老者道:「常年祭賽時,我這裡有膽大的鑽在廟後,或在供桌底下,看見他先吃童男,後吃童女。」八戒道:「造化,造化。」

兄弟正然談論,只聽得外面鑼鼓喧天,燈火照耀,同莊眾人打開前門,叫:「擡出童男童女來。」這老者哭哭啼啼,那四個後生將他二人擡將出去。

端的不知性命何如,且聽下回分解。
