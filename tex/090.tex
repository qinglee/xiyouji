
\chapter{師獅授受同歸一 盜道纏禪靜九靈}

卻說孫大聖同八戒、沙僧出城頭,覿面相迎,見那夥妖精都是些雜毛獅子:黃獅精在前引領,狻猊獅、摶象獅在左,白澤獅、伏狸獅在右,猱獅、雪獅在後,中間卻是一個九頭獅子。那青臉兒怪執一面錦鏽團花寶幢,緊挨著九頭獅子;刁鑽古怪兒、古怪刁鑽兒打兩面紅旗。齊齊的都佈在坎宮之地。

八戒莽撞,走近前罵道:「偷寶貝的賊怪,你去那裡夥這幾個毛團來此怎的?」黃獅精切齒罵道:「潑狠禿廝!昨日三個敵我一個,我敗回去,讓你為人罷了,你怎麼這般狠惡,燒了我的洞府,損了我的山場,傷了我的眷族?我和你冤仇深如大海,不要走,吃你老爺一鏟。」好八戒,舉鈀就迎。兩個才交手還未見高低,那猱獅精掄一根鐵蒺藜,雪獅精使一條三楞簡,徑來奔打。八戒發一聲喊道:「來得好!」你看這壁廂,沙和尚急掣降妖杖,近前相助。又見那狻猊精、白澤精與摶象、伏狸二精,一擁齊上。這裡孫大聖使金箍棒架住群精。狻猊使悶棍,白澤使銅鎚,摶象使鋼槍,伏狸使鉞斧。那七個獅子精,這三個狠和尚,好殺:
\begin{quote}
棍鎚槍斧三楞簡,蒺藜骨朵四明鏟。
七獅七器甚鋒芒,圍戰三僧齊吶喊。
大聖金箍鐵棒兇,沙僧寶杖人間罕。
八戒顛風騁勢雄,釘鈀晃亮光華慘。
前遮後擋各施功,左架右迎都勇敢。
城頭王子助威風,擂鼓篩鑼齊壯膽。
投來搶去弄神通,殺得昏蒙天地反。
\end{quote}

那一夥妖精齊與大聖三人,戰經半日,不覺天晚。八戒口吐粘涎,看看腳軟,虛幌一鈀,敗下陣去。被那雪獅、猱獅二精喝道:「那裡走?看打。」獃子躲閃不及,被他照脊梁上打了一簡,睡在地下,只叫:「罷了,罷了。」兩個精把八戒採鬃拖尾,扛將去見那九頭獅子,報道:「祖爺,我等拿了一個來也。」

說不了,沙僧、行者也都戰敗,眾妖精一齊趕來。被行者拔一把毫毛,嚼碎噴將去,叫聲:「變!」即變做百十個小行者,圍圍繞繞,將那白澤、狻猊、摶象、伏狸並金毛獅怪圍裹在中。沙僧、行者卻又上前攢打。到晚,拿住狻猊、白澤,走了伏狸、摶象。金毛報知老怪。老怪見失了二獅,吩咐:「把豬八戒綑了,不可傷他性命。待他還我二獅,卻將八戒與他;他若無知,壞了我二獅,即將八戒殺了對命。」當晚群妖安歇城外不題。

卻說孫大聖把兩個獅子精擡近城邊,老王見了,即傳令開門,差二三十個校尉,拿繩趕出門,綁了獅精,扛入城裡。孫大聖收了法毛,同沙僧徑至城樓上,見了唐僧。唐僧道:「這場事甚是利害呀。悟能性命,不知有無?」行者道:「沒事,我們把這兩個妖精拿了,他那裡斷不敢傷。且將二精牢拴緊縛,待明早抵換八戒也。」三個小王子對行者叩頭道:「師父先前賭鬥,只見一身;及後佯輸而回,卻怎麼就有百十位師身?及至拿住妖精,近城來還是一身。此是甚麼法力?」行者笑道:「我身上有八萬四千毫毛,以一化十,以十化百,百千萬億之變化,皆身外身之法也。」那王子一個個頂禮,即時擺上齋來,就在城樓上吃了。各垛口上都要燈籠旗幟,梆鈴鑼鼓,支更傳箭,放炮吶喊。

早又天明。老怪即喚黃獅精定計道:「汝等今日用心拿那行者、沙僧;等我暗自飛空上城,拿他那師父并那老王父子,先轉九曲盤桓洞,待你得勝回報。」黃獅領計,便引猱獅、雪獅、摶象、伏狸,各執兵器到城邊,滾風釀霧的索戰。這裡行者與沙僧跳出城頭,厲聲罵道:「賊潑怪!快將我師弟八戒送還我,饒你性命;不然,都教你粉骨碎屍。」那妖精那容分說,一擁齊來。這大聖弟兄兩個,各運機謀,擋住五個獅子。這殺比昨日又甚不同:
\begin{quote}
呼呼刮地狂風惡,暗暗遮天黑霧濃。走石飛沙神鬼怕,推林倒樹虎狼驚。鋼槍狠狠鉞斧明,蒺藜簡鏟太毒情。恨不得囫圇吞行者,活活潑潑擒住小沙僧。這大聖一條如意棒,卷舒收放甚精靈。沙僧那柄降妖杖,靈霄殿外有名聲。今番幹運神通廣,西域施功掃蕩精。
\end{quote}

這五個雜毛獅子精與行者、沙僧正自殺到好處,那老怪駕著黑雲,徑直騰至城樓上,搖一搖頭,諕得那城上文武大小官員并守城人夫等都滾下城去。被他奔入樓中,張開口,把三藏與老王父子一頓噙出;復至坎宮地下,將八戒也著口噙之。原來他九個頭就有九張口。一口噙著唐僧,一口噙著八戒,一口噙著老王,一口噙著大王子,一口噙著二王子,一口噙著三王子:六口噙著六人,還空了三張口。發聲喊叫道:「我先去也。」這五個小獅精見他祖得勝,一個個愈展雄才。

行者聞得城上人喊嚷,情知中了他計,急喚沙僧仔細。他卻把臂膊上毫毛盡皆拔下,入口嚼爛噴出,變作千百個小行者,一擁攻上。當時拖倒猱獅,活捉了雪獅,拿住了摶象獅,扛翻了伏狸獅,將黃獅打死,烘烘的嚷到州城之下。倒轉走脫了青臉兒與刁鑽古怪、古怪刁鑽兒三怪。那城上官看見,卻又開門,將繩把五個獅精又綑了,擡進城去。還未發落,只見那王妃哭哭啼啼,對行者禮拜道:「神師啊,我殿下父子並你師父,性命休矣。這孤城怎生是好?」大聖收了法毛,對王妃作禮道:「賢后莫愁。只因我拿他七個獅精,那老妖弄攝法,定將我師父與殿下父子攝去,料必無傷。待明日絕早,我兄弟二人去那山中,管情捉住老妖,還你四個王子。」那王妃一簇女眷聞得此言,都對行者下拜道:「願求殿下父子全生,皇圖堅固。」拜畢,一個個含淚還宮。行者吩咐各官:「將打死的黃獅精,剝了皮;六個活獅精,牢牢拴鎖。取些齋飯來,我們吃了睡覺。你們都放心,保你無事。」

至次日,大聖領沙僧駕起祥雲,不多時,到於竹節山頭。按雲頭觀看,好座高山。但見:
\begin{quote}
峰排突兀,嶺峻崎嶇。深澗下潺湲水漱,陡崖前錦鏽花香。回巒重疊,古道灣環。真是鶴來松有伴,果然雲去石無依。玄猿覓果向晴暉,麋鹿尋花歡日暖。青鸞聲淅嚦,黃鳥語綿蠻。春來桃李爭妍,夏至柳槐競茂。秋到黃花佈錦,冬交白雪飛綿。四時八節好風光,不亞瀛洲仙景象。
\end{quote}

他兩個正在山頭上看景,忽見那青臉兒手拿一條短棍,徑跑出崖谷之間。行者喝道:「那裡走,老孫來也。」諕得那小妖一翻一滾的跑下崖谷。他兩個一直追來,又不見蹤跡。向前又轉幾步,卻是一座洞府,兩扇花斑石門,緊緊關閉。門楟上橫嵌著一塊石版,楷鐫了十個大字,乃是「萬靈竹節山,九曲盤桓洞」。

那小妖原來跑進洞去,即把洞門閉了。到中間對老妖道:「爺爺,外面又有兩個和尚來了。」老妖道:「你大王並猱獅、雪獅、摶象、伏狸可曾來?」小妖道:「不見,不見。只是兩個和尚在山峰高處眺望。我看見回頭就跑,他趕將來,我卻閉門來也。」老妖聽說,低頭不語。半晌,忽的吊下淚來,叫聲:「苦啊!我黃獅孫死了,猱獅孫等又盡被和尚捉進城去矣。此恨怎生報得?」八戒綑在傍邊,與王父子、唐僧俱攢在一處,恓恓惶惶受苦。聽見老妖說聲眾孫被和尚捉進城去,暗暗喜道:「師父莫怕,殿下休愁。我師兄已得勝,捉了眾妖,尋到此間救拔吾等也。」說罷,又聽得老妖叫:「小的們,好生在此看守,等我出去拿那兩個和尚進來,一發懲治。」

你看他身無披掛,手不拈兵,大踏步走到前邊,只聞得孫行者吆喝哩。他就大開了洞門,不答話,徑奔行者。行者使鐵棒,當頭支住。沙僧掄寶杖就打。那老妖把頭搖一搖,左右八個頭,一齊張開口,把行者、沙僧輕輕的又啣於洞內。教:「取繩索來。」那刁鑽古怪、古怪刁鑽與青臉兒是昨夜逃生而回者,即拿兩條繩,把他二人著實綑了。老妖問道:「你這潑猴,把我那七個兒孫捉了;我今拿住你和尚四個、王子四個,也足以抵得我兒孫之命。小的們,選荊條柳棍來,且打這猴頭一頓,與我黃獅孫報報冤仇。」那三個小妖各執柳棍,專打行者。行者本是熬煉過的身體,那些些柳棍兒,只好與他拂癢,他那裡做聲,憑他怎麼捶打,略不介意。八戒、唐僧與王子見了,一個個毛骨悚然。少時,打折了柳棍。直打到天晚,也不計其數。沙僧見打得多了,甚不過意道:「我替他打百十下罷。」老妖道:「你且莫忙,明日就打到你了。一個個挨次兒打將來。」八戒著忙道:「後日就打到我老豬也。」打一會,漸漸的天昏了。老妖叫:「小的們,且住,點起燈火來,你們吃些飲食,讓我到錦雲窩略睡睡去。汝三人都是遭過害的,卻用心看守,待明早再打。」三個小妖移過燈來,拿柳棍又打行者腦蓋,就像敲梆子一般,剔剔托,托托剔,緊幾下,慢幾下。夜將深了,卻都盹睡。

行者就使個遁法,將身一小,脫出繩來。抖一抖毫毛,整束了衣服。耳朵內取出棒來,幌一幌,有吊桶粗細,二丈長短,朝著三個小妖道:「你這孽畜,把你老爺就打了許多棍子。老爺還只照舊,老爺也把這棍子略掗你掗,看道如何?」把三個小妖輕輕一掗,就掗做三個肉餅。卻又剔亮了燈,解放沙僧。八戒綑急了,忍不住大聲叫道:「哥哥,我的手腳都綑腫了,倒不來先解放我?」這獃子喊了一聲,卻早驚動老妖。老妖一轂轆爬起來道:「是誰人解放?」那行者聽見,一口吹息燈,也顧不得沙僧等眾,使鐵棒,打破幾重門走了。

那老妖到中堂裡叫:「小的們,怎麼沒了燈光?只莫走了人也?」叫一聲,沒人答應;又叫一聲,又沒人答應。及取燈火來看時,只見地下血淋淋的三塊肉餅,老王父子及唐僧、八戒俱在,只不見了行者、沙僧。點著火,前後趕看,只見沙僧還背貼在廊下站哩。被他一把拿住捽倒,照舊綑了。又找尋行者,但見幾層門盡皆破損,情知是行者打破走了。也不去追趕,將破門補的補,遮的遮,固守家業不題。

卻說孫大聖出了那九曲盤桓洞,跨祥雲,徑轉玉華州。但見那城頭上各方的土地、神祇與城隍之神迎空拜接。行者道:「汝等怎麼今夜才見?」城隍道:「小神等知大聖下降玉華州,因有賢王款留,故不敢見。今知王等遇怪,大聖降魔,特來叩接。」行者正在嗔怪處,又見金頭揭諦、六甲六丁神將押著一尊土地,跪在面前道:「大聖,吾等捉得這個地裡鬼來也。」行者喝道:「汝等不在竹節山護我師父,卻怎麼嚷到這裡?」丁甲神道:「大聖,那妖精自你逃時,復捉住捲簾大將,依然綑了。我等見他法力甚大,卻將竹節山土地押解至此。他知那妖精的根由,乞大聖問他一問,便好處治,以救聖僧、賢王之苦。」行者聽言,甚喜。那土地戰兢兢叩頭道:「那老妖前年下降竹節山。那九曲盤桓洞原是六獅之窩,那六個獅子自得老妖至此,就都拜為祖翁。祖翁乃是個九頭獅子,號為九靈元聖。若得他滅,須去到東極妙巖宮,請他主人公來,方可收伏;他人莫想擒也。」行者聞言,思憶半晌道:「東極妙巖宮,是太乙救苦天尊啊,他坐下正是個九頭獅子。這等說。」便教:「揭諦、金甲,還同土地回去,暗中護祐師父、師弟並州王父子;本處城隍守護城池。」眾神各各遵守去訖。

這大聖縱觔斗雲,連夜前行。約有寅時,到了東天門外,正撞著廣目天王與天丁、力士一行儀從。眾皆停住,拱手迎道:「大聖何往?」行者對眾禮畢,道:「前去妙巖宮走走。」天王道:「西天路不走,卻又東天來做甚?」行者道:「因到玉華州,蒙州王相款,遣三子拜我等弟兄為師,習學武藝,不期遇著一夥獅怪。今訪得妙巖宮太乙救苦天尊乃怪之主人公,欲請他來降怪救師。」天王道:「那廂因你欲為人師,所以惹出這一窩獅子來也。」行者笑道:「正為此,正為此。」眾天丁、力士一個個拱手,讓道而行。大聖進了東天門,不多時,到妙巖宮前。但見:
\begin{quote}
彩雲重疊,紫氣蘢蔥。瓦漾金波焰,門排玉獸崇。花盈雙闕紅霞遶,日映騫林翠霧籠。果然是萬真環拱,千聖興隆。殿閣層層錦,窗軒處處通。蒼龍盤護神光藹,黃道光輝瑞氣濃。這的是青華長樂界,東極妙巖宮。
\end{quote}

那宮門立著一個穿霓帔的仙童,忽見孫大聖,即入宮報道:「爺爺,外面是鬧天宮的齊天大聖來了。」太乙救苦天尊聽得,即喚侍衛眾仙迎接。迎至宮中,只見天尊高坐九色蓮花座上,百億瑞光之中。見了行者,下座來相見。行者朝上施禮。天尊答禮道:「大聖,這幾年不見,前聞得你棄道歸佛,保唐僧西天取經,想是功行完了?」行者道:「功行未完,卻也將近。但如今因保唐僧到玉華州,蒙王子遣三子拜老孫等為師,習學武藝,把我們三件神兵照樣打造,不期夜間被賊偷去。及天明尋找,原是城北豹頭山虎口洞一個金毛獅子成精盜去。老孫用計取出,那精就夥了若干獅精與老孫大鬧。內有一個九頭獅子,神通廣大,將我師父與八戒並王父子四人都啣去,到一竹節山九曲盤桓洞。次日,老孫與沙僧跟尋,亦被啣去。老孫被他綑打無數,幸而弄法走了。他們正在彼處受罪。問及當坊土地,始知天尊是他主人,特來奉請收降解救。」

天尊聞言,即令仙將到獅子房喚出獅奴來問。那獅奴熟睡,被眾將推搖方醒,揪至中廳來見。天尊問道:「獅獸何在?」那奴兒垂淚叩頭,只教:「饒命,饒命。」天尊道:「孫大聖在此,且不打你。你快說為何不謹,走了九頭獅子。」獅奴道:「爺爺,我前日在大千甘露殿中見一瓶酒,不知偷去吃了,不覺沉醉睡著,失於拴鎖,是以走了。」天尊道:「那酒是太上老君送的,喚做輪迴瓊液,你吃了該醉三日不醒。那獅獸今走幾日了?」大聖道:「據土地說,他前年下降,到今二三年矣。」天尊笑道:「是了,是了,天宮裡一日,在凡世就是一年。」叫獅奴道:「你且起來,饒你死罪,跟我與大聖下方去收他來。汝眾仙都回去,不用跟隨。」

天尊遂與大聖、獅奴,踏雲徑至竹節山。只見那五方揭諦、六丁六甲、本山土地都來跪接。行者道:「汝等護祐,可曾傷著我師?」眾神道:「妖精著了惱睡了,更不曾動甚刑罰。」天尊道:「我那元聖兒也是一個久修得道的真靈:他喊一聲,上通三聖,下徹九泉,等閑也便不傷生。孫大聖,你去他門首索戰,引他出來,我好收之。」

行者聽言,果掣棒跳近洞口,高罵道:「潑妖精,還我人來也。潑妖精,還我人來也。」連叫了數聲,那老妖睡著了,無人答應。行者性急起來,掄鐵棒,往內打進,口中不住的喊罵。那老妖方才驚醒,心中大怒,爬起來,喝一聲:「趕戰。」搖搖頭,便張口來啣。行者回頭跳出。妖精趕到外邊,罵道:「賊猴,那裡走?」行者立在高崖上笑道:「你還敢這等大膽無禮,你死活也不知哩,這不是你老爺主公在此?」那妖精趕到崖前,早被天尊念聲咒語,喝道:「元聖兒,我來了。」那妖認得是主人,不敢展掙,四隻腳伏之於地,只是磕頭。傍邊跑過獅奴兒,一把撾住項毛,用拳著項上打夠百十,口裡罵道:「你這畜生,如何偷走,教我受罪?」那獅獸合口無言,不敢搖動。獅奴兒打得手困,方才住了,即將錦韂安在他身上。天尊騎了,喝聲教走。他就縱身駕起彩雲,徑轉妙巖宮去。

大聖望空稱謝了,卻入洞中,先解玉華王,次解唐三藏,次又解了八戒、沙僧並三王子。共搜他洞裡物件,逍逍停停,將眾領出門外。八戒就取了若干枯柴,前後堆上,放起火來,把一個九曲盤桓洞,燒做個烏焦破瓦窰。大聖又發放了眾神,還教土地在此鎮守。卻令八戒、沙僧各各使法,把王父子背馱回州。他攙著唐僧。不多時,到了州城,天色漸晚,當有妃后、官員都來接見了。擺上齋筵,共坐享之。長老師徒還在暴紗亭安歇。王子們入宮各寢。一宵無話。

次日,王又傳旨,大開素宴。合府大小官員,一一謝恩。行者又叫屠子來,把那六個活獅子殺了,共那黃獅子都剝了皮,將肉安排將來受用。殿下十分歡喜,即命殺了:把一個留在本府內外人用;一個與王府長史等官分用;把五個都剁做一二兩重的塊子,差校尉散給州城內外軍民人等,各吃些須:一則嘗嘗滋味,二則押押驚恐。那些家家戶戶,無不瞻仰。

又見那鐵匠人等造成了三般兵器,對行者磕頭道:「爺爺,小的們工都完了。」問道:「各重多少斤兩?」鐵匠道:「金箍棒有千斤,九齒鈀與降妖杖各有八百斤。」行者道:「也罷。」叫請三位王子出來,各人執兵器。三子對老王道:「父王,今日兵器完矣。」老王道:「為此兵器,幾乎傷了我父子之命。」小王子道:「幸蒙神師施法,救出我等,卻又掃蕩妖邪,除了後患。誠所謂海晏河清,太平之世界也。」當時老王父子賞勞了匠作,又至暴紗亭拜謝了師恩。

三藏又教大聖等快傳武藝,莫誤行程。他三人就各掄兵器,在王府院中,一一傳授。不數日,那三個王子盡皆操演精熟,其餘攻退之方,緊慢之法,各有七十二道解數,無不知之。一則那諸王子心堅,二則虧孫大聖先授了神力,此所以那千斤之棒,八百斤之鈀、杖,俱能舉能運;較之初時自家弄的武藝,真天淵也。有詩為證。詩曰:
\begin{quote}
緣因善慶遇神師,習武何期動怪獅。
掃蕩群邪安社稷,皈依一體定邊夷。
九靈數合元陽理,四面精通道果之。
授受心明遺萬古,玉華永樂太平時。
\end{quote}

那王子又大開筵宴,謝了師教。又取出一大盤金銀,用答微情。行者笑道:「快拿進去,快拿進去。我們出家人,要他何用?」八戒在傍道:「金銀實不敢受,奈何我這件衣服被那些獅子精扯拉破了,但與我們換件衣服,足為愛也。」那王子隨命針工,照依色樣,取青錦、紅錦、茶褐錦各數疋,與三位各做了一件。三人欣然領受,各穿了錦布直裰,收拾了行裝起程。只見那城裡城外,若大若小,無一人不稱是羅漢臨凡,活佛下界。鼓樂之聲,旌旗之色,盈街塞道。正是家家戶外焚香火,處處門前獻彩燈。來至許遠方回,他四眾方得離城西去。這一去頓脫群獅,潛心正果。正是:
\begin{quote}
無慮無憂來佛界,誠心誠意上雷音。
\end{quote}

畢竟不知到靈山還有幾多路程,何時行到,且聽下回分解。
