
\chapter{屍魔三戲唐三藏 聖僧恨逐美猴王}

卻說三藏師徒次日天明收拾前進,那鎮元子與行者結為兄弟,兩人情投意合,決不肯放,又安排管待,一連住了五六日。那長老自服了草還丹,真似脫胎換骨,神爽體健。他取經心重,那裡肯淹留,無已,遂行。

師徒別了上路,早見一座高山。三藏道:「徒弟,前面有山險峻,恐馬不能前,大家須仔細仔細。」行者道:「師父放心,我等自然理會。」好猴王,他在馬前橫擔著棒,剖開山路,上了高崖,看不盡:
\begin{quote}
峰巖重疊,澗壑彎環。虎狼成陣走,麂鹿作群行。無數獐鑽簇簇,滿山狐兔聚叢叢。千尺大蟒,萬丈長蛇。大蟒噴愁霧,長蛇吐怪風。道旁荊棘牽漫,嶺上松柟秀麗。薜蘿滿目,芳草連天。影落滄溟北,雲開斗柄南。萬古常含元氣老,千峰巍列日光寒。
\end{quote}

那長老馬上心驚。孫大聖佈施手段,舞著鐵棒,哮吼一聲,諕得那狼蟲顛竄,虎豹奔逃。

師徒們入此山,正行到嵯峨之處,三藏道:「悟空,我這一日,肚中饑了,你去那裡化些齋吃。」行者陪笑道:「師父好不聰明。這等半山之中,前不巴村,後不著店,有錢也沒買處,教往那裡尋齋?」三藏心中不快,口裡罵道:「你這猴子!想你在兩界山,被如來壓在石匣之內,口能言,足不能行,也虧我救你性命,摩頂受戒,做了我的徒弟。怎麼不肯努力,常懷懶惰之心?」行者道:「弟子亦頗慇懃,何嘗懶惰?」三藏道:「你既慇懃,何不化齋我吃?我肚饑怎行?況此地山嵐瘴氣,怎麼得上雷音?」行者道:「師父休怪,少要言語。我知你尊性高傲,十分違慢了你,便要念那話兒咒。你下馬穩坐,等我尋那裡有人家處化齋去。」

行者將身一縱,跳上雲端裡,手搭涼篷,睜眼觀看。可憐西方路甚是寂寞,更無莊堡人家,正是多逢樹木,少見人煙去處。看多時,只見正南上有一座高山,那山向陽處,有一片鮮紅的點子。行者按下雲頭道:「師父,有吃的了。」那長老問甚東西。行者道:「這裡沒人家化飯,那南山有一片紅的,想必是熟透了的山桃,我去摘幾個來你充饑。」三藏喜道:「出家人若有桃子吃,就為上分了。」行者取了缽盂,縱起祥光,你看他觔斗幌幌,冷氣颼颼,須臾間,奔南山摘桃不題。

卻說常言有云:「山高必有怪,嶺峻卻生精。」果然這山上有一個妖精,孫大聖去時,驚動那怪。他在雲端裡踏著陰風,看見長老坐在地下,就不勝歡喜道:「造化,造化。幾年家人都講東土的唐和尚取大乘,他本是金蟬子化身,十世修行的原體,有人吃他一塊肉,長壽長生。真個今日到了。」那妖精上前就要拿他,只見長老左右手下有兩員大將護持,不敢攏身。他說兩員大將是誰?說是八戒、沙僧。八戒、沙僧雖沒甚麼大本事,然八戒是天蓬元帥,沙僧是捲簾大將,他的威氣尚不曾泄,故不敢攏身。妖精說:「等我且戲他戲,看怎麼說。」

好妖精,停下陰風,在那山凹裡搖身一變,變做個月貌花容的女兒,說不盡那眉清目秀,齒白唇紅。左手提著一個青砂罐兒,右手提著一個綠磁瓶兒,從西向東,徑奔唐僧:
\begin{quote}
聖僧歇馬在山巖,忽見裙釵女近前。
翠袖輕搖籠玉筍,湘裙斜拽顯金蓮。
汗流粉面花含露,塵拂蛾眉柳帶煙。
仔細定睛觀看處,看看行至到身邊。
\end{quote}

三藏見了,叫:「八戒、沙僧,悟空才說這裡曠野無人,你看那裡不走出一個人來了?」八戒道:「師父,你與沙僧坐著,等老豬去看看來。」

那獃子放下釘鈀,整整直裰,擺擺搖搖,充作個斯文氣象,一直的靦面相迎。真個是遠看未實,近看分明,那女子生得:
\begin{quote}
冰肌藏玉骨,衫領露酥胸。柳眉積翠黛,杏眼閃銀星。月樣容儀俏,天然性格清。體似燕藏柳,聲如鶯囀林。半放海棠籠曉日,才開芍藥弄春晴。
\end{quote}

那八戒見他生得俊俏,獃子就動了凡心,忍不住胡言亂語,叫道:「女菩薩,往那裡去?手裡提著是甚麼東西?」分明是個妖怪,他卻不能認得。那女子連聲答應道:「長老,我這青罐裡是香米飯,綠瓶裡是炒麵觔。特來此處無他故,因還誓願要齋僧。」八戒聞言,滿心歡喜,急抽身,就跑了個豬顛風,報與三藏道:「師父,『吉人自有天報』,師父餓了,教師兄去化齋,那猴子不知那裡摘桃兒耍子去了。桃子吃多了,也有些嘈人,又有些下墜。你看那不是個齋僧的來了?」唐僧不信道:「你這個夯貨胡纏。我們走了這向,好人也不曾遇著一個,齋僧的從何而來!」八戒道:「師父,這不到了?」

三藏一見,連忙跳起身來,合掌當胸道:「女菩薩,你府上在何處住?是甚人家?有甚願心,來此齋僧?」分明是個妖精,那長老也不認得。那妖精見唐僧問他來歷,他立地就起個虛情,花言巧語,來賺哄道:「師父,此山叫做蛇回獸怕的白虎嶺,正西下面是我家。我父母在堂,看經好善,廣齋方上遠近僧人。只因無子,求神作福,生了奴奴。欲扳門第,配嫁他人,又恐老來無倚,只得將奴招了一個女婿,養老送終。」三藏聞言道:「女菩薩,你語言差了。聖經云:『父母在,不遠遊,遊必有方。』你既有父母在堂,又與你招了女婿,有願心,教你男子還,便也罷,怎麼自家在山行走?又沒個侍兒隨從。這個是不遵婦道了。」那女子笑吟吟,忙陪俏語道:「師父,我丈夫在山北凹裡,帶幾個客子鋤田。這是奴奴煮的午飯,送與那些人吃的。只為五黃六月,無人使喚,父母又年老,所以親身來送。忽遇三位遠來,卻思父母好善,故將此飯齋僧,如不棄嫌,願表芹獻。」三藏道:「善哉!善哉!我有徒弟摘果子去了,就來。我不敢吃,假如我和尚吃了你飯,你丈夫曉得,罵你,卻不罪坐貧僧也?」那女子見唐僧不肯吃,卻又滿面春生道:「師父啊,我父母齋僧,還是小可;我丈夫更是個善人,一生好的是修橋補路,愛老憐貧。但聽見說這飯送與師父吃了,他與我夫妻情上,比尋常更是不同。」三藏也只是不吃。

旁邊卻惱壞了八戒,那獃子努著嘴,口裡埋怨道:「天下和尚也無數,不曾像我這個老和尚罷軟。現成的飯,三分兒倒不吃,只等那猴子來,做四分才吃。」他不容分說,一嘴把個罐子拱倒,就要動口。只見那行者自南山頂上摘了幾個桃子,托著缽盂,一觔斗,點將回來,睜火眼金睛觀看,認得那女子是個妖精,放下缽盂,掣鐵棒,當頭就打。諕得個長老用手扯住道:「悟空,你走將來打誰?」行者道:「師父,你面前這個女子,莫當做個好人,他是個妖精,要來騙你哩。」三藏道:「你這個猴頭,當時倒也有些眼力,今日如何亂道?這女菩薩有此善心,將這飯要齋我等,你怎麼說他是個妖精?」行者笑道:「師父,你那裡認得。老孫在水簾洞裡做妖魔時,若想人肉吃,便是這等:或變金銀,或變莊臺,或變醉人,或變女色。有那等痴心的愛上我,我就迷他到洞裡,盡意隨心,或蒸或煮受用;吃不了,還要曬乾了防天陰哩。師父,我若來遲,你定入他套子,遭他毒手。」那唐僧那裡肯信,只說是個好人。行者道:「師父,我知道你了,你見他那等容貌,必然動了凡心。若果有此意,叫八戒伐幾棵樹來,沙僧尋些草來,我做木匠,就在這裡搭個窩鋪,你與他圓房成事,我們大家散了,卻不是件事業?何必又跋涉,取甚經去?」那長老原是個軟善的人,那裡吃得他這句言語,羞得光頭徹耳通紅。

三藏正在此羞慚,行者又發起性來,掣鐵棒,望妖精劈臉一下。那怪物有些手段,使個「解屍法」,見行者棍子來時,他卻抖擻精神,預先走了,把一個假屍首打死在地下。諕得個長老戰戰兢兢,口中作念道:「這猴著然無禮,屢勸不從,無故傷人性命。」行者道:「師父莫怪,你且來看看這罐子裡是甚東西?」沙僧攙著長老,近前看時,那裡是甚香米飯,卻是一罐子拖尾巴的長蛆;也不是麵觔,卻是幾個青蛙、癩蝦蟆,滿地亂跳。長老才有三分兒信了。怎禁豬八戒氣不忿,在傍漏八分兒唆嘴道:「師父,說起這個女子,他是此間農婦,因為送飯下田,路遇我等,卻怎麼栽他是個妖怪?哥哥的棍重,走將來試手打他一下,不期就打殺了。怕你念甚麼緊箍兒咒,故意的使個障眼法兒,變做這等樣東西,演幌你眼,使不念咒哩。」

三藏自此一言,就是晦氣到了。果然信那獃子攛唆,手中捻訣,口裡念咒。行者就叫:「頭疼,頭疼。莫念,莫念,有話便說。」唐僧道:「有甚話說?出家人時時常要方便,念念不離善心,掃地恐傷螻蟻命,愛惜飛蛾紗罩燈。你怎麼步步行兇,打死這個無故平人,取將經來何用?你回去罷。」行者道:「師父,你教我回那裡去?」唐僧道:「我不要你做徒弟。」行者道:「你不要我做徒弟,只怕你西天路去不成。」唐僧道:「我命在天,該那個妖精蒸了吃,就是煮了,也算不過。終不然,你救得我的大限?你快回去。」行者道:「師父,我回去便也罷了,只是不曾報得你的恩哩。」唐僧道:「我與你有甚恩?」那大聖聞言,連忙跪下叩頭道:「老孫因大鬧天宮,致下了傷身之難,被我佛壓在兩界山。幸觀音菩薩與我受了戒行,幸師父救脫吾身。若不與你同上西天,顯得我知恩不報非君子,萬古千秋作罵名。」原來這唐僧是個慈憫的聖僧,他見行者哀告,卻也回心轉意道:「既如此說,且饒你這一次,再休無禮。如若仍前作惡,這咒語顛倒就念二十遍。」行者道:「三十遍也由你,只是我不打人了。」卻才伏侍唐僧上馬,又將摘來桃子奉上。唐僧在馬上也吃了幾個,權且充饑。

卻說那妖精脫命昇空,原來行者那一棒不曾打殺妖精,妖精出神去了。他在那雲端裡咬牙切齒,暗恨行者道:「幾年只聞得講他手段,今日果然話不虛傳。那唐僧已是不認得我,將要吃飯。若低頭聞一聞兒,我就一把撈住,卻不是我的人了?不期被他走來,弄破我這勾當,又幾乎被他打了一棒。若饒了這個和尚,誠然是勞而無功也,我還下去戲他一戲。」

好妖精,按落陰雲,在那前山坡下搖身一變,變作個老婦人,年滿八旬,手拄著一根彎頭竹杖,一步一聲的哭著走來。八戒見了,大驚道:「師父,不好了,那媽媽兒來尋人了。」唐僧道:「尋甚人?」八戒道:「師兄打殺的定是他女兒,這個定是他娘尋將來了。」行者道:「兄弟莫要胡說,那女子十八歲,這老婦有八十歲,怎麼六十多歲還生產?斷乎是個假的,等老孫去看來。」

好行者,拽開步,走近前觀看,那怪物:
\begin{quote}
假變一婆婆,兩鬢如冰雪。走路慢騰騰,行步虛怯怯。弱體瘦伶仃,臉如枯菜葉。顴骨望上翹,嘴唇往下別。老年不比少年時,滿臉都是荷葉摺。
\end{quote}

行者認得他是妖精,更不理論,舉棒照頭便打。那怪見棍子起時,依然抖擻,又出化了元神,脫真兒去了,把個假屍首又打死在山路傍之下。

唐僧一見,驚下馬來,睡在路傍,更無二話,只是把緊箍兒咒顛倒足足念了二十遍。可憐把個行者頭勒得似個亞腰兒葫蘆,十分疼痛難忍,滾將來哀告道:「師父莫念了,有甚話說了罷。」唐僧道:「有甚話說?出家人耳聽善言,不墮地獄。我這般勸化你,你怎麼只是行兇?把平人打死一個,又打死一個,此是何說?」行者道:「他是妖精。」唐僧道:「這個猴子胡說,就有這許多妖怪?你是個無心向善之輩,有意作惡之人,你去罷。」行者道:「師父又教我去?回去便也回去了,只是一件不相應。」唐僧道:「你有甚麼不相應處?」八戒道:「師父,他要和你分行李哩。跟著你做了這幾年和尚,不成空著手回去?你把那包袱內的甚麼舊褊衫,破帽子、分兩件與他罷。」

行者聞言,氣得暴跳道:「我把你這個尖嘴的夯貨!老孫一向秉教沙門,更無一毫嫉妒之意,貪戀之心,怎麼要分甚麼行李?」唐僧道:「你既不嫉妒貪戀,如何不去?」行者道:「實不瞞師父說,老孫五百年前,居花果山水簾洞大展英雄之際,收降七十二洞邪魔,手下有四萬七千小怪,頭戴的是紫金冠,身穿的是赭黃袍,腰繫的是藍田帶,足踏的是步雲履,手執的是如意金箍棒,著實也曾為人。自從涅槃罪度,削髮秉正沙門,跟你做了徒弟,把這個金箍兒勒在我頭上,若回去,卻也難見故鄉人。師父果若不要我,把那個鬆箍兒咒念一念,退下這個箍子,交付與你,套在別人頭上,我就快活相應了,也是跟你一場。莫不成這些人意兒也沒有了?」唐僧大驚道:「悟空,我當時只是菩薩暗受一卷緊箍兒咒,卻沒有甚麼鬆箍兒咒。」行者道:「若無松箍兒咒,你還帶我去走走罷。」長老又沒奈何道:「你且起來,我再饒你這一次,卻不可再行兇了。」行者道:「再不敢了。再不敢了。」又伏侍師父上馬,剖路前進。

卻說那妖精原來行者第二棍也不曾打殺他。那怪物在半空中誇獎不盡道:「好個猴王,著然有眼,我那般變了去,他也還認得我。這些和尚他去得快,若過此山,西下四十里,就不伏我所管了。若是被別處妖魔撈了去,好道就笑破他人口,使碎自家心。我還下去戲他一戲。」

好妖精,按聳陰風,在山坡下搖身一變,變做一個老公公,真個是:
\begin{quote}
白髮如彭祖,蒼髯賽壽星。
耳中鳴玉磬,眼裡幌金星。
手拄龍頭拐,身穿鶴氅輕。
數珠掐在手,口誦南無經。
\end{quote}

唐僧在馬上見了,心中大喜道:「阿彌陀佛!西方真是福地,那公公路也走不上來,逼法的還念經哩。」八戒道:「師父,你且莫要誇獎,那個是禍的根哩。」唐僧道:「怎麼是禍根?」八戒道:「師兄打殺他的女兒,又打殺他的婆子,這個正是他的老兒尋將來了。我們若撞在他的懷裡啊,師父,你便償命,該個死罪;把老豬為從,問個充軍;沙僧喝令,問個擺站。那師兄使個遁法走了,卻不苦了我們三個頂缸?」

行者聽見道:「這個獃根,這等胡說,可不諕了師父?等老孫再去看看。」他把棍藏在身邊,走上前,迎著怪物,叫聲:「老官兒,往那裡去?怎麼又走路,又念經?」那妖精錯認了定盤星,把孫大聖也當做個等閑的,遂答道:「長老啊,我老漢祖居此地,一生好善齋僧,看經念佛。命裡無兒,止生得一個小女,招了個女婿。今早送飯下田,想是遭逢虎口。老妻先來找尋,也不見回去。全然不知下落,老漢特來尋看。果然是傷殘他命,也沒奈何,將他骸骨收拾回去,安葬塋中。」行者笑道:「我是個做虎的祖宗,你怎麼袖子裡籠了個鬼兒來哄我?你瞞了諸人,瞞不過我,我認得你是個妖精。」那妖精諕得頓口無言。行者掣出棒來,自忖道:「若要不打他,顯得他倒弄個風兒;若要打他,又怕師父念那話兒咒語。」又思量道:「不打殺他,他一時間抄空兒把師父撈了去,卻不又費心勞力去救他?還打的是。就一棍子打殺,師父念起那咒,常言道:『虎毒不吃兒。』憑著我巧言花語,嘴伶舌便,哄他一哄,好道也罷了。」好大聖,念動咒語,叫當坊土地、本處山神道:「這妖精三番來戲弄我師父,這一番卻要打殺他。你與我在半空中作證,不許走了。」眾神聽令,誰敢不從,都在雲端裡照應。那大聖棍起處,打倒妖魔,才斷絕了靈光。

那唐僧在馬上又諕得戰戰兢兢,口不能言。八戒在傍邊又笑道:「好行者,風發了,只行了半日路,倒打死三個人。」唐僧正要念咒,行者急到馬前叫道:「師父莫念,莫念,你且來看看他的模樣。」卻是一堆粉骷髏在那裡。唐僧大驚道:「悟空,這個人才死了,怎麼就化作一堆骷髏?」行者道:「他是個潛靈作怪的僵尸,在此迷人敗本,被我打殺,他就現了本相。他那脊梁上有一行字,叫做『白骨夫人』。」唐僧聞說,倒也信了。怎禁那八戒傍邊唆嘴道:「師父,他的手重棍兇,把人打死,只怕你念那話兒,故意變化這個模樣,掩你的眼目哩。」唐僧果然耳軟,又信了他,隨復念起。行者禁不得疼痛,跪於路傍,只叫:「莫念,莫念,有話快說了罷。」唐僧道:「猴頭,還有甚說話?出家人行善,如春園之草,不見其長,日有所增;行惡之人,如磨刀之石,不見其損,日有所虧。你在這荒郊野外,一連打死三人,還是無人檢舉,沒有對頭;倘到城市之中,人煙湊集之所,你拿了那哭喪棒,一時不知好歹,亂打起人來,撞出大禍,教我怎的脫身?你回去罷。」行者道:「師父錯怪了我也。這廝分明是個妖魔,他實有心害你。我倒打死他,替你除了害,你卻不認得,反信了那獃子讒言冷語,屢次逐我。常言道:『事不過三。』我若不去,真是個下流無恥之徒。我去,我去。去便去了,只是你手下無人。」唐僧發怒道:「這潑猴越發無禮。看起來,只你是人,那悟能、悟淨就不是人?」

那大聖一聞得說他兩個是人,止不住傷情悽慘,對唐僧道聲:「苦啊!你那時節出了長安,有劉伯欽送你上路。到兩界山,救我出來,投拜你為師。我曾穿古洞,入深林,擒魔捉怪,收八戒,得沙僧,吃盡千辛萬苦。今日昧著惺惺使糊塗,只教我回去。這才是:『鳥盡弓藏,兔死狗烹!』罷,罷,罷,但只是多了那緊箍兒咒。」唐僧道:「我再不念了。」行者道:「這個難說。若到那毒魔苦難處不得脫身,八戒、沙僧救不得你,那時節想起我來,忍不住又念誦起來。就是十萬里路,我的頭也是疼的,假如再來見你,不如不作此意。」

唐僧見他言言語語,越添惱怒,滾鞍下馬來,叫沙僧包袱內取出紙筆,即於澗下取水,石上磨墨,寫了一紙貶書,遞於行者道:「猴頭,執此為照,再不要你做徒弟了;如再與你相見,我就墮了阿鼻地獄。」行者連忙接了貶書道:「師父,不消發誓,老孫去罷。」他將書摺了,留在袖內,卻又軟款唐僧道:「師父,我也是跟你一場,又蒙菩薩指教,今日半塗而廢,不曾成得功果,你請坐,受我一拜,我也去得放心。」唐僧轉回身不睬,口裡唧唧噥噥的道:「我是個好和尚,不受你歹人的禮。」大聖見他不睬,又使個身外法,把腦後毫毛拔了三根,吹口仙氣,叫:「變!」即變了三個行者,連本身四個,四面圍住師父下拜。那長老左右躲不脫,好道也受了一拜。

大聖跳起來,把身一抖,收上毫毛,卻又吩咐沙僧道:「賢弟,你是個好人,卻只要留心防著八戒詀言詀語,途中更要仔細。倘一時有妖精拿住師父,你就說老孫是他大徒弟,西方毛怪聞我的手段,不敢傷我師父。」唐僧道:「我是個好和尚,不題你這歹人的名字,你回去罷。」

那大聖見長老三番兩復,不肯轉意回心,沒奈何才去。你看他:
\begin{quote}
噙淚叩頭辭長老,含悲留意囑沙僧。
一頭拭迸坡前草,兩腳蹬翻地上藤。
上天下地如輪轉,跨海飛山第一能。
頃刻之間不見影,霎時疾返舊途程。
\end{quote}

你看他忍氣別了師父,縱觔斗雲,徑回花果山水簾洞去了。獨自個悽悽慘慘,忽聞得水聲聒耳。大聖在那半空裡看時,原來是東洋大海潮發的聲響。一見了,又想起唐僧,止不住腮邊淚墜,停雲住步,良久方去。

畢竟不知此去反復何如,且聽下回分解。
