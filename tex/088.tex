
\chapter{禪到玉華施法會 心猿木母授門人}

話說唐僧喜喜歡歡別了郡侯,在馬上向行者道:「賢徒,這一場善果,真勝似比丘國搭救兒童,皆爾之功也。」沙僧道:「比丘國只救得一千一百一十一個小兒,怎似這場大雨,滂沱浸潤,活夠者萬萬千千性命?弟子也暗自稱讚大師兄的法力通天,慈恩蓋地也。」八戒笑道:「哥的恩也有,善也有,卻只是外施仁義,內包禍心。但與老豬走,就要作踐人。」行者道:「我在那裡作踐你?」八戒道:「也夠了,也夠了。常照顧我綑,照顧我吊,照顧我煮,照顧我蒸。今在鳳仙郡施了恩惠與萬萬之人,就該住上半年,帶挈我吃幾頓自在飽飯,卻只管催促行路。」長老聞言,喝道:「這個獃子,怎麼只思量擄嘴?快走路,再莫鬥口。」八戒不敢言,掬掬嘴,挑著行囊,打著哈哈,師徒們奔上大路。此時光景如梭,又值深秋之候,但見:
\begin{quote}
水痕收,山骨瘦。紅葉紛飛,黃花時候。霜晴覺夜長,月白穿窗透。家家煙火夕陽多,處處湖光寒水溜。白蘋香,紅蓼茂。橘綠橙黃,柳衰穀秀。荒村雁落碎蘆花,野店雞聲收菽豆。
\end{quote}

四眾行夠多時,又見城垣影影。長老舉鞭遙指叫:「悟空,你看那裡又有一座城池,卻不知是甚去處?」行者道:「你我俱未曾到,何以知之?且行至邊前問人。」說不了,忽見樹叢裡走出一個老者,手持竹杖,身著輕衣,足踏一對棕鞋,腰束一條扁帶。慌得唐僧滾鞍下馬,上前道個問訊。那老者扶杖還禮道:「長老那方來的?」唐僧合掌道:「貧僧東土唐朝差往雷音拜佛求經者。今至寶方,遙望城垣,不知是甚去處,特問老施主指教。」那老者聞言,口稱:「有道禪師,我這敝處乃天竺國下郡,地名玉華縣。縣中城主,就是天竺皇帝之宗室,封為玉華王。此王甚賢,專敬僧道,重愛黎民。老禪師若去相見,必有重敬。」三藏謝了,那老者徑穿樹林而去。

三藏才轉身對徒弟備言前事。他三人欣喜,扶師父上馬。三藏道:「沒多路,不須乘馬。」四眾遂步至城邊街道觀看。原來那關廂人家,做買做賣的,人煙湊集,生意亦甚茂盛。觀其聲音相貌,與中華無異。三藏吩咐:「徒弟們謹慎,切不可放肆。」那八戒低了頭,沙僧掩著臉,惟孫行者攙著師父。兩邊人都來爭看,齊聲叫道:「我這裡只有降龍伏虎的高僧,不曾見降豬伏猴的和尚。」八戒忍不住,把嘴一掬道:「你們可曾看見降豬王的和尚?」諕得滿街上人跌跌𧿼𧿼,都往兩邊閃過。行者笑道:「獃子,快藏了嘴,莫裝扮,仔細腳下過橋。」那獃子低著頭,只是笑。過了吊橋,入城門內,又見那大街上酒樓歌館,熱鬧繁華,果然是神州都邑。有詩為證。詩曰:
\begin{quote}
錦城鐵瓮萬年堅,臨水依山色色鮮。
百貨通湖船入市,千家沽酒店垂帘。
樓臺處處人煙廣,巷陌朝朝客賈喧。
不亞長安風景好,雞鳴犬吠亦般般。
\end{quote}

三藏心中暗喜道:「人言西域諸番,更不曾到此。細觀此景,與我大唐何異?所謂極樂世界,誠此之謂也。」又聽得人說白米四錢一石,麻油八釐一斤,真是五穀豐登之處。

行夠多時,方到玉華王府。府門左右有長史府、審理廳、典膳所、待客館。三藏道:「徒弟,此間是府,等我進去朝王,驗牒而行。」八戒道:「師父進去,我們可好在衙門前站立?」三藏道:「你不看這門上是『待客館』三字?!你們都去那裡坐下,看有草料,買些喂馬。我見了王,倘或賜齋,便來喚你等同享。」行者道:「師父放心前去,老孫自當理會。」那沙僧把行李挑至館中。館中有看館的人役,見他們面貌醜陋,也不敢問他,也不敢教他出去,只得讓他坐下不題。

卻說老師父換了衣帽,拿了關文,徑至王府前。早見引禮官迎著問道:「長老何來?」三藏道:「東土大唐差來大雷音拜佛祖求經之僧,今到貴地,欲倒換關文,特來朝參千歲。」引禮官即為傳奏。那王子果然賢達,即傳旨召進。三藏至殿下施禮,王子即請上殿賜坐。三藏將關文獻上,王子看了,見有各國印信手押,也就欣然將寶印了,押了花字,收摺在案。問道:「國師長老,自你那大唐至此,歷遍諸邦,共有幾多路程?」三藏道:「貧僧也未記程途,但先年蒙觀音菩薩在我王御前顯身,曾留了頌子,言西方十萬八千里。貧僧在路,已經過一十四遍寒暑矣。」王子笑道:「十四遍寒暑,即十四年了。想是途中有甚耽擱?」三藏道:「一言難盡。萬蟄千魔,也不知受了多少苦楚,才到得寶方。」那王子十分歡喜,即著典膳官備素齋管待。三藏:「啟上殿下:貧僧有三個小徒在外等候,不敢領齋,但恐遲誤行程。」王子教:「當殿官,快去請長老三位徒弟,進府同齋。」

當殿官隨出外相請,都道:「未曾見,未曾見。」有跟隨的人道:「待客館中坐著三個醜貌和尚,想必是也。」當殿官同眾至館中,即問看館的道:「那個是大唐取經僧的高徒?我主有旨,請吃齋也。」八戒正坐打盹,聽見一個「齋」字,忍不住跳起身來答道:「我們是,我們是。」當殿官一見了,魂飛魄喪,都戰戰的道:「是個豬魈,豬魈。」行者聽見,一把扯住八戒道:「兄弟,放斯文些,莫撒村野。」那眾官見了行者,又道:「是個猴精,猴精。」沙僧拱手道:「列位休得驚恐。我三人都是唐僧的徒弟。」眾官見了,又道:「灶君,灶君。」孫行者即教八戒牽馬,沙僧挑擔,同眾入玉華王府。當殿官先入啟知。

那王子舉目見那等醜惡,卻也心中害怕。三藏合掌道:「千歲放心。頑徒雖是貌醜,卻都心良。」八戒朝上唱個喏道:「貧僧問訊了。」王子愈覺心驚。三藏道:「頑徒都是山野中收來的,不會行禮,萬望赦罪。」王子奈著驚恐,教典膳官請眾僧去暴紗亭吃齋。三藏謝了恩,辭王下殿,同至亭內,埋怨八戒道:「你這夯貨,全不知一毫禮體。索性不開口,便也罷了,怎麼那般粗魯?一句話,足足衝倒泰山。」行者笑道:「還是我不唱喏的好,也省些力氣。」沙僧道:「他唱喏又不等齊,預先就抒著個嘴吆喝。」八戒道:「活淘氣,活淘氣。師父前日教我見人打個問訊兒是禮,今日打問訊,又說不好,教我怎的幹麼。」三藏道:「我教你見了人打個問訊,不曾教你見王子就此歪纏。常言道:『物有幾等物,人有幾等人。』如何不分個貴賤?」正說處,那典膳官帶領人役,調開桌椅,擺上齋來。師徒們卻不言語,各各吃齋。

卻說那王子退殿進宮,宮中有三個小王子,見他面容改色,即問道:「父王今日為何有此驚恐?」王子道:「適才有東土大唐差來拜佛取經的一個和尚倒換關文,卻一表非凡。我留他吃齋,他說有徒弟在府前,我即命請。少時進來,見我不行大禮,打個問訊,我已不快;及擡頭看時,一個個醜似妖魔,心中不覺驚駭,故此面容改色。」原來那三個小王子比眾不同,一個個好武好強,便就伸拳擄袖道:「莫敢是那山裡走來的妖精假裝人像,待我們拿兵器出去看來。」

好王子,大的個拿一條齊眉棍,第二個掄一把九齒鈀,第三個使一根烏油黑棒子,雄糾糾、氣昂昂的走出王府,吆喝道:「甚麼取經的和尚,在那裡?」時有典膳官員人等跪下道:「小王,他們在這暴紗亭吃齋哩。」小王子不分好歹,闖將進去,喝道:「汝等是人是怪?快早說來,饒你性命。」諕得三藏面容失色,丟下飯碗,躬著身道:「貧僧乃唐朝來取經者,人也,非怪也。」小王子道:「你便還像個人,那三個醜的斷然是怪。」八戒只管吃飯不睬。沙僧與行者欠身道:「我等俱是人,面雖醜而心良,身雖夯而性善。汝三個卻是何來,卻恁樣海口輕狂?」傍有典膳等官道:「三位是我王之子小殿下。」八戒丟了碗道:「小殿下,各拿兵器怎麼?莫是要與我們打哩?」

二王子掣開步,雙手舞鈀,便要打八戒。八戒嘻嘻笑道:「你那鈀只好與我這鈀做孫子罷了。」即揭衣,腰間取出鈀來,幌一幌,金光萬道。丟了解數,有瑞氣千條。把個王子諕得手軟筋麻,不敢舞弄。行者見大的個使一條齊眉棍,跳阿跳的,即耳朵裡取出金箍棒來,幌一幌,碗來粗細,有丈二三長短。著地下一搗,搗了有三尺深淺,豎在那裡。笑道:「我把這棍子送你罷。」那王子聽言,即丟了自己棍,去取那棒,雙手盡氣力一拔,莫想得動分毫;再又端一端,搖一搖,就如生根一般。第三個撒起莽性,使烏油棒便來打。被沙僧一手劈開,取出降妖寶杖,撚一撚,艷艷光生,紛紛霞亮。諕得那典膳等官一個個呆呆掙掙,口不能言。三個小王子一齊下拜道:「神師,神師,我等凡人不識,萬望施展一番,我等好拜授也。」行者走近前,輕輕的把棒拿將起來道:「這裡窄狹,不好展手。等我跳在空中,耍一路兒,你們看看。」

好大聖,唿哨一聲,將觔斗一抖,兩隻腳踏著五色祥雲,起在半空,離地約有三百步高下。把金箍棒丟開個「撒花蓋頂」,「黃龍轉身」,一上一下,左旋右轉,起初時人與棒似錦上添花;次後來不見人,只見一天棒滾。八戒在底下喝聲采,也忍不住手腳,厲聲喊道:「等老豬也去耍耍來。」好獃子,駕起風頭,也到半空,丟開鈀,上三下四,左五右六,前七後八,滿身解數,只聽得呼呼風響。正使到熱鬧處,沙僧對長老道:「師父,也等老沙去操演操演。」好和尚,雙著腳一跳,掄著杖,也起在空中,只見那銳氣氤氳,金光縹緲。雙手使降妖杖丟一個「丹鳳朝陽」,「餓虎撲食」,緊迎慢擋,急轉忙攛。弟兄三個大展神通,都在那半空中,一齊揚威耀武。這才是:
\begin{quote}
真禪景象不凡同,大道緣由滿太空。
金木施威盈法界,刀圭展轉合圓通。
神兵精銳隨時顯,丹器花生到處崇。
天竺雖高還戒性,玉華王子總歸中。
\end{quote}

諕得那三個小王子跪在塵埃。暴紗亭大小人員,並王府裡老王子,滿城中軍民男女,僧尼道俗,一應人等,家家念佛磕頭,戶戶拈香禮拜。果然是:
\begin{quote}
見像歸真度眾僧,人間作福享清平。
從今果正菩提路,盡是參禪拜佛人。
\end{quote}

他三個各逞雄才,使了一路,按下祥雲,把兵器收了。到唐僧面前問訊,謝了師恩,各各坐下不題。

那三個小王子急回宮裡,告奏老王道:「父王萬千之喜,今有莫大之功也。適才可曾看見半空中舞弄麼?」老王道:「我才見半空霞彩,就於宮院內同你母親等眾焚香啟拜,更不知是那裡神仙降聚也。」小王子道:「不是那裡神仙,就是那取經僧三個醜徒弟。一個使金箍鐵棒,一個使九齒釘鈀,一個使降妖寶杖,把我三個的兵器比的通沒有分毫。我們教他使一路,他說:『地上窄狹,不好支吾,等我起在空中,使一路你看。』他就各駕雲頭,滿空中祥雲縹緲,瑞氣氤氳。才然落下,都坐在暴紗亭裡。做兒的十分歡喜,欲要拜他為師,學他手段,保護我邦。此誠莫大之功,不知父王以為何如?」老王聞言,信心從願。

當時父子四人不擺駕,不張蓋,步行到暴紗亭。他四眾收拾行李,欲進府謝齋,辭王起行,偶見玉華王父子上亭來,倒身下拜。慌得長老舒身,撲地還禮;行者等閃過傍邊,微微冷笑。眾拜畢,請四眾進府堂上坐。四眾欣然而入。老王起身道:「唐老師父,孤有一事奉求,不知三位高徒可能容否?」三藏道:「但憑千歲吩咐,小徒不敢不從。」老王道:「孤先見列位時,只以為唐朝遠來行腳僧,其實肉眼凡胎,多致輕褻。適見孫師、豬師、沙師起舞在空,方知是仙是佛。孤三個犬子,一生好弄武藝,今謹發虔心,欲拜為門徒,學些武藝。萬望老師開天地之心,普運慈舟,傳度小兒,必以傾城之資奉謝。」行者聞言,忍不住呵呵笑道:「你這殿下,好不會事。我等出家人,巴不得要傳幾個徒弟。你令郎既有從善之心,切不可說起分毫之利;但只以情相處,足為愛也。」

王子聽言,十分歡喜。隨命大排筵宴,就於本府正堂擺列。噫!一聲旨意,即刻俱完。但見那:
\begin{quote}
結綵飄颻,香煙馥郁。戧金桌子掛絞綃,晃人眼目;彩漆椅兒鋪錦繡,添座風光。樹果新鮮,茶湯香噴。三五道閑食清甜,一兩餐饅頭豐潔。蒸酥蜜煎更奇哉,油炸糖澆真美矣。有幾瓶香糯素酒,斟出來,賽過瓊漿;獻幾番陽羨仙茶,捧到手,香欺丹桂。般般品品皆齊備,色色行行盡出奇。
\end{quote}

一壁廂叫承應的歌舞吹彈,撮弄演戲。他師徒們並王父子盡樂一日。

不覺天晚,散了酒席。又叫即於暴紗亭鋪設床幃,請師安宿。待明早竭誠焚香,再拜求傳武藝。眾皆聽從,即備香湯,請師沐浴,眾卻歸寢。此時那:
\begin{quote}
眾鳥高棲萬簌沉,詩人下榻罷哦吟。
銀河光顯天彌亮,野徑荒涼草更深。
砧杵叮咚敲別院,關山杳窵動鄉心。
寒蛩聲朗知人意,嚦嚦床頭破夢魂。
\end{quote}

一宵晚景題過。明早,那老王父子又來相見。這長老昨日相見還是王禮,今日就行師禮。那三個小王子對行者、八戒、沙僧當面叩頭,拜問道:「尊師之兵器,還借出來與弟子們看看。」八戒聞言,欣然取出釘鈀,拋在地下。沙僧將寶杖拋出,倚在牆邊。二王子與三王子跳起去便拿,就如蜻蜓撼石柱,一個個掙得紅頭赤臉,莫想拿動半分毫。大王子見了,叫道:「兄弟,莫費力了。師父的兵器,俱是神兵,不知有多少重哩。」八戒笑道:「我的鈀也沒多重,只有一藏之數,連柄五千零四十八斤。」三王子問沙僧道:「師父寶杖多重?」沙僧笑道:「也是五千零四十八斤。」大王子求行者的金箍棒看。行者去耳朵裡取出一個針兒來,迎風幌一幌,就有碗來粗細,直直的豎立面前。那王父子都皆悚懼,眾官員個個心驚。三個小王子禮拜道:「豬師、沙師之兵,俱隨身帶在衣下,即可取之。孫師為何自耳中取出,見風即長,何也?」行者笑道:「你不知我這棒不是凡間等閑可有者。這棒是:
\begin{quote}
鴻濛初判陶鎔鐵,大禹神人親所設。湖海江河淺共深,曾將此棒知之切。開山治水太平時,流落東洋鎮海闕。日久年深放彩霞,能消能長能光潔。老孫有分取將來,變化無方隨口訣。要大彌於宇宙間,要小卻似針兒節。棒名如意號金箍,天上人間稱一絕。重該一萬三千五百斤,或粗或細能生滅。也曾助我鬧天宮,也曾隨我攻地闕。伏虎降龍處處通,煉魔除怪方方徹。舉頭一指太陽昏,天地鬼神皆膽怯。混沌仙傳到至今,原來不是凡間鐵。」
\end{quote}

那王子聽言,個個頂禮不盡。三個向前重重拜禮,虔心求授。行者道:「你三人不知學那般武藝?」王子道:「願使棍的就學棍,慣使鈀的就學鈀,愛用杖的就學杖。」行者笑道:「教便也容易,只是你等無力量,使不得我們的兵器,恐學之不精,如畫虎不成反類狗也。古人云:『訓教不嚴師之惰,學問無成子之罪。』汝等既有誠心,可去焚香來拜了天地,我先傳你些神力,然後可授武藝。」

三個小王子聞言,滿心歡喜。即便親擡香案,沐手焚香,朝天禮拜。拜畢,請師傳法。行者轉下身來,對唐僧行禮道:「告尊師,恕弟子之罪。自當年在兩界山蒙師父大德救脫弟子,秉教沙門,一向西來,雖不曾重報師恩,卻也曾渡水登山,竭盡心力。今來佛國之鄉,幸遇賢王三子,投拜我等,欲學武藝。彼既為我等之徒弟,即為我師之徒孫也。謹稟過我師,庶好傳授。」三藏十分大喜。八戒、沙僧見行者行禮,也即轉身朝三藏磕頭道:「師父,我等愚魯,拙口鈍腮,不會說話。望師父高坐法位,也讓我兩個各招個徒弟耍耍,也是西方路上之憶念。」三藏俱欣然允之。

行者才教三個王子都在暴紗亭後,靜室之間,畫了罡斗。教三人都俯伏在內,一個個瞑目寧神。這裡卻暗暗念動真言,誦動咒語,將仙氣吹入他三人心腹之中,把元神收歸本舍。傳與口訣,各授得萬千之膂力,運添了火候,卻像個脫胎換骨之法。運遍了子午周天,那三個小王子方才甦醒,一齊爬將起來,抹抹臉,精神抖擻,一個個骨壯筋強:大王子就拿得金箍棒,二王子就掄得九齒鈀,三王子就舉得降妖杖。

老王見了,歡喜不勝。又排素宴,啟謝他師徒四眾。就在筵前各傳各授:學棍的演棍,學鈀的演鈀,學杖的演杖。雖然打幾個轉身,丟幾般解數,此等終是凡夫,有些著力:走一路,便喘氣噓噓,不能耐久。蓋他那兵器都有變化,其進退攻揚,隨消隨長,皆有自然之妙,此等終是凡夫,豈能以遽及也?當日收了筵宴。

次日,三個王子又來稱謝道:「感蒙神師授賜了膂力,縱然掄得師的兵器,只是轉換艱難。意欲命匠依神師兵器式樣,減削斤兩,打造一般,未知師父肯容否?」八戒道:「好好好,說得有理。我們的器械,一則你們使不得,二則我們要護法降魔,正該另造另造。」王子隨即宣召鐵匠,買辦鋼鐵萬斤,就在王府內前院搭廠,支爐鑄造。先一日將鋼鐵煉熟。次日請行者三人將金箍棒、九齒鈀、降妖杖,都取出放在篷廠之間,看樣造作。遂此晝夜不收。

噫!這兵器原是他們隨身之寶,一刻不可離者,各藏在身,自有許多光彩護體。今放在廠院中幾日,那霞光有萬道沖天,瑞氣有千般罩地。其夜有一妖精,離城只有七十里遠近,山喚豹頭山,洞喚虎口洞,夜坐之間,忽見霞光瑞氣,即駕雲而看,見光彩起處是王府之內。他按下雲頭,近前觀看,乃是這三般兵器放光。妖精又喜又愛道:「好寶貝,好寶貝。這是甚人用的,今放在此?也是我的緣法,拿了去呀,拿了去呀。」他愛心一動,弄起威風,將三般兵器一股收之,徑轉本洞。正是那:
\begin{quote}
道不須臾離,可離非道也。神兵盡落空,枉費參修者。
\end{quote}

畢竟不知怎生尋得這兵器,且聽下回分解。
