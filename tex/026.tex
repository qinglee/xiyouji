
\chapter{孫悟空三島求方 觀世音甘泉活樹}

詩曰:
\begin{quote}
處世須存心上刃,修身切記寸邊而。
常言刃字為生意,但要三思戒怒欺。
上士無爭傳亙古,聖人懷德繼當時。
剛強更有剛強輩,究竟終成空與非。
\end{quote}

卻說那鎮元大仙用手攙著行者道:「我也知道你的本事,我也聞得你的英名,只是你今番越理欺心,縱有騰那,脫不得我手。我就和你講到西天,見了你那佛祖,也少不得還我人參果樹。你莫弄神通。」行者笑道:「你這先生,好小家子樣。若要樹活,有甚疑難?早說這話,可不省了一場爭競?」大仙道:「不爭競,我肯善自饒你!」行者道:「你解了我師父,我還你一棵活樹如何?」大仙道:「你若有此神通,醫得樹活,我與你八拜為交,結為兄弟。」行者道:「不打緊,放了他們,老孫管教還你活樹。」

大仙諒他走不脫,即命解放了三藏、八戒、沙僧。沙僧道:「師父啊,不知師兄搗得是甚麼鬼哩。」八戒道:「甚麼鬼,這叫做『當面人情鬼』。樹死了,又可醫得活?他弄個光皮散兒好看,者著求醫治樹,單單了脫身走路,還顧得你和我哩。」三藏道:「他決不敢撒了我們。我們問他那裡求醫去。」遂叫道:「悟空,你怎麼哄了仙長,解放我等?」行者道:「老孫是真言實語,怎麼哄他?」三藏道:「你往何處去求方?」行者道:「古人云:『方從海上來。』我今要上東洋大海,遍遊三島十洲,訪問仙翁聖老,求一個起死回生之法,管教醫得他樹活。」三藏道:「此去幾時可回?」行者道:「只消三日。」三藏道:「既如此,就依你說,與你三日之限。三日裡來便罷;若三日之外不來,我就念那話兒經了。」行者道:「遵命,遵命。」

你看他急整虎皮裙,出門來對大仙道:「先生放心,我就去就來。你卻要好生伏侍我師父,逐日家三茶六飯,不可欠缺;若少了些兒,老孫回來和你算帳,先搗塌你的鍋底。衣服禳了,與他漿洗漿洗。臉兒黃了些兒,我不要;若瘦了些兒,不出門。」那大仙道:「你去,你去,定不教他忍餓。」

好猴王,急縱觔斗雲,別了五莊觀,徑上東洋大海。在半空中,快如掣電,疾如流星,早到蓬萊仙境。按雲頭,往下仔細觀看,真個好去處。有詩為證。詩曰:
\begin{quote}
大地仙鄉列聖曹,蓬萊分合鎮波濤。
瑤臺影蘸天心冷,巨闕光浮海面高。
五色煙霞含玉籟,九霄星月射金鰲。
西池王母常來此,奉祝三仙幾次桃。
\end{quote}

那行者看不盡仙景,徑入蓬萊。正然走處,見白雲洞外,松陰之下,有三個老兒圍碁,觀局者是壽星,對局者是福星、祿星。行者上前叫道:「老弟們,作揖了。」那三星見了,拂退碁枰,回禮道:「大聖何來?」行者道:「特來尋你們耍子。」壽星道:「我聞大聖棄道從釋,脫性命保護唐僧往西天取經,逐日奔波山路,那些兒得閑,卻來耍子?」行者道:「實不瞞列位說,老孫因往西方,行在半路,有些兒阻滯,特來小事欲干,不知肯否?」福星道:「是甚地方?因何阻滯?乞為明示,吾好裁處。」行者道:「因路過萬壽山五莊觀有阻。」三老驚訝道:「五莊觀是鎮元大仙的仙宮,你莫不是把他人參果偷吃了?」行者笑道:「偷吃了能值甚麼?」三老道:「你這猴子,不知好歹。那果子聞一聞,活三百六十歲。吃一個,活四萬七千年。叫做萬壽草還丹。我們的道,不及他多矣。他得之甚易,就可與天齊壽;我們還要養精、煉氣、存神,調和龍虎、捉坎填離,不知費多少工夫。你怎麼說他的能值甚緊?天下只有此種靈根。」行者道:「靈根,靈根,我已弄了他個斷根哩。」三老驚道:「怎的斷根?」

行者道:「我們前日在他觀裡,那大仙不在家,只有兩個小童接待了我師父,卻將兩個人參果奉與我師。我師不認得,只說是三朝未滿的孩童,再三不吃。那童子就拿去吃了,不曾讓得我們。是老孫就去偷了他三個,我兄弟三人吃了。那童子不知高低,賊前賊後的罵個不住。是老孫惱了,把他樹打了一棍,推倒在地,樹上果子全無,枒開葉落,根出枝傷,已枯死了。不想那童子關住我們,又被老孫扭開鎖走了。次日清辰,那先生回家趕來,問答間,語言不和,遂與他賭鬥,被他閃一閃,把袍袖展開,一袖子都籠去了。繩纏索綁,拷問鞭敲,就打了一日。是夜又逃了,他又趕上,依舊籠去。他身無寸鐵,只是把個麈尾遮架,我兄弟這等三般兵器,莫想打得著。他這一番仍舊擺佈,將布裹漆了我師父與兩師弟,卻將我下油鍋。我又做了個脫身本事走了,把他鍋都打破。他見拿我不住,儘有幾分醋我。是我又與他好講,教他放了我師父、師弟,我與他醫樹管活,兩家才得安寧。我想著『方從海上來』,故此特遊仙境,訪三位老弟。有甚醫樹的方兒,傳我一個,急救唐僧脫苦。」

三星聞言,心中也悶道:「你這猴兒,全不識人。那鎮元子乃地仙之祖;我等乃神仙之宗;你雖得了天仙,還是太乙散數,未入真流,你怎麼脫得他手?若是大聖打殺了走獸飛禽、蜾蟲鱗長,只用我黍米之丹,可以救活。那人參果乃仙木之根,如何醫治?沒方,沒方。」那行者見說無方,卻就眉峰雙鎖,額蹙千痕。福星道:「大聖,此處無方,他處或有,怎麼就生煩惱?」行者道:「無方別訪,果然容易,就是遊遍海角天涯,轉透三十六天,亦是小可。只是我那唐長老法嚴量窄,止與了我三日期限;三日以外不到,他就要念那緊箍兒咒哩。」三星笑道:「好,好,好,若不是這個法兒拘束你,你又鑽天了。」壽星道:「大聖放心,不須煩惱。那大仙雖稱上輩,卻也與我等有識。一則久別,不曾拜望;二來是大聖的人情:如今我三人同去望他一望,就與你道達此情,教那唐和尚莫念緊箍兒咒,休說三日五日,只等你求得方來,我們才別。」行者道:「感激,感激。就請三位老弟行行,我去也。」大聖辭別三星不題。

卻說這三星駕起祥光,即往五莊觀而來。那觀中合眾人等,忽聽得長天鶴唳,原來是三老光臨。但見那:
\begin{quote}
盈空藹藹祥光簇,霄漢紛紛香馥郁。
彩霧千條護羽衣,輕雲一朵擎仙足。
青鸞飛,丹鳳䎘,袖引香風滿地撲。
拄杖懸龍喜笑生,皓髯垂玉胸前拂。
童顏歡悅更無憂,壯體雄威多有福。
執星籌,添海屋,腰掛葫蘆並寶籙。
萬紀千旬福壽長,十洲三島隨緣宿。
常來世上送千祥,每向人間增百福。
概乾坤,榮福祿,福壽無疆今喜得。
\end{quote}

三老乘祥謁大仙,福堂和氣皆無極。那仙童看見,即忙報道:「師父,海上三星來了。」鎮元子正與唐僧師弟閑敘,聞報,即降階奉迎。那八戒見了壽星,近前扯住,笑道:「你這肉頭老兒,許久不見,還是這般脫灑,帽兒也不帶個來。」遂把自家一個僧帽,撲的套在他頭上,撲著手呵呵大笑道:「好,好,好,真是『加冠進祿』也。」那壽星將帽子摜了,罵道:「你這個夯貨,老大不知高低。」八戒道:「我不是夯貨,你等真是奴才。」福星道:「你倒是個夯貨,反敢罵人是奴才?」八戒又笑道:「既不是人家奴才,好道叫做『添壽』、『添福』、『添祿』?」那三藏喝退了八戒,急整衣拜了三星。

那三星以晚輩之禮見了大仙,方才敘坐。坐定,祿星道:「我們一向久闊尊顏。有失恭敬,今因孫大聖攪擾仙山,特來相見。」大仙道:「孫行者到蓬萊去的?」壽星道:「是,因為傷了大仙的丹樹,他來我處求方醫治。我輩無方,他又到別處求訪,但恐違了聖僧三日之限,要念緊箍兒咒。我輩一來奉拜,二來討個寬限。」三藏聞言,連聲應道:「不敢念,不敢念。」

正說處,八戒又跑進來,扯住福星,要討果子吃。他去袖裡亂摸,腰裡亂挖,不住的揭他衣服搜檢。三藏笑道:「那八戒是甚麼規矩!」八戒道:「不是沒規矩,此叫做『番番是福』。」三藏又叱令出去。那獃子䠚出門,瞅著福星,眼不轉睛的發狠。福星道:「夯貨,我那裡惱了你來,你這等恨我?」八戒道:「不是恨你,這叫『回頭望福』。」那獃子出得門來,只見一個小童拿了四把茶匙,方去尋鍾取果看茶,被他一把奪過,跑上殿,拿著小磬兒,用手亂敲亂打,兩頭頑耍。大仙道:「這個和尚越發不尊重了。」八戒笑道:「不是不尊重,這叫做『四時吉慶』。」

且不說八戒打諢亂纏。卻表行者縱祥雲離了蓬萊,又早到方丈仙山,這山真好去處。有詩為證。詩曰:
\begin{quote}
方丈巍峨別是天,太元宮府會神仙。
紫臺光照三清路,花木香浮五色煙。
金鳳自多槃蕊闕,玉膏誰逼灌芝田。
碧桃紫李新成熟,又換仙人信萬年。
\end{quote}

那行者按落雲頭,無心玩景。正走處,只聞得香風馥馥,玄鶴聲鳴,那壁廂有個神仙。但見:
\begin{quote}
盈空萬道霞光現,彩霧飄颻光不斷。
丹鳳啣花也更鮮,青鸞飛舞聲嬌艷。
福如東海壽如山,貌似小童身體健。
壺隱洞天不老丹,腰懸與日長生篆。
人間數次降禎祥,世上幾番消厄願。
武帝曾宣加壽齡,瑤池每赴蟠桃宴。
教化眾僧脫俗緣,指開大道明如電。
也曾跨海祝千秋,常去靈山參佛面。
聖號東華大帝君,煙霞第一神仙眷。
\end{quote}

孫行者靦面相迎,叫聲:「帝君,起手了。」那帝君慌忙回禮道:「大聖,失迎。請荒居奉茶。」遂與行者攙手而入。果然是貝闕仙宮,看不盡瑤池瓊閣。方坐待茶,只見翠屏後轉出一個童兒。他怎生打扮:
\begin{quote}
身穿道服飄霞爍,腰束絲絛光錯落。
頭戴綸巾佈斗星,足登芒履遊仙岳。
煉元真,脫本殼,功行成時遂意樂。
識破原流精氣神,主人認得無虛錯。
逃名今喜壽無疆,甲子週天管不著。
轉回廊,登寶閣,天上蟠桃三度摸。
縹緲香雲出翠屏,小仙乃是東方朔。
\end{quote}

行者見了,笑道:「這個小賊在這裡哩。帝君處沒有桃子你偷吃!」東方朔朝上進禮,答道:「老賊,你來這裡怎的?我師父沒有仙丹你偷吃。」

帝君叫道:「曼倩休亂言,看茶來也。」曼倩原是東方朔的道名,他急入裡取茶二杯。飲訖,行者道:「老孫此來,有一事奉干,未知允否?」帝君道:「何事?自當領教。」行者道:「近因保唐僧西行,路過萬壽山五莊觀,因他那小童無狀,是我一時發怒,把他人參果樹推倒,一時阻滯,唐僧不得脫身,特來尊處求賜一方醫治,萬望慨然。」帝君道:「你這猴子,不管一二,到處裡闖禍。那五莊觀鎮元子,聖號與世同君,乃地仙之祖,你怎麼就衝撞了他?他那人參果樹乃草還丹,你偷吃了,尚說有罪;卻又連樹推倒,他肯干休?」行者道:「正是呢。我們走脫了,被他趕上,把我們就當汗巾兒一般,一袖子都籠去了,所以閣氣。沒奈何,許他求方醫治,故此拜求。」帝君道:「我有一粒九轉太乙還丹,但能治世間生靈,卻不能醫樹。樹乃土木之靈,天滋地潤。若是凡間的果木,醫治還可;這萬壽山乃先天福地,五莊觀乃賀洲洞天,人參果又是天開地闢之靈根,如何可治,無方,無方。」

行者道:「既然無方,老孫告別。」帝君仍欲留奉玉液一杯,行者道:「急救事緊,不敢久滯。」遂駕雲復至瀛洲海島,也好去處。有詩為證。詩曰:
\begin{quote}
珠樹玲瓏照紫煙,瀛洲宮闕接諸天。
青山綠水琪花艷,玉液錕鋘鐵石堅。
五色碧雞啼海日,千年丹鳳吸朱煙。
世人罔究壺中景,象外春光億萬年。
\end{quote}

那大聖至瀛洲,只見那丹崖珠樹之下,有幾個皓髮皤髯之輩,童顏鶴鬢之仙,在那裡著棋飲酒,談笑謳歌。真個是:
\begin{quote}
祥雲光滿,瑞靄香浮。彩鸞鳴洞口,玄鶴舞山頭。碧藕水桃為按酒,交梨火棗壽千秋。一個個丹詔無聞,仙符有籍。逍遙隨浪蕩,散淡任清幽。周天甲子難拘管,大地乾坤只自由。獻果玄猿,對對參隨多美愛;啣花白鹿,雙雙拱伏甚綢繆。
\end{quote}

那些老兒正然灑樂。這行者厲聲高叫道:「帶我耍耍兒便怎的?」眾仙見了,急忙趨步相迎。有詩為證。詩曰:
\begin{quote}
人參果樹靈根折,大聖訪仙求妙訣。
繚繞丹霞出寶林,瀛洲九老來相接。
\end{quote}

行者認得是九老,笑道:「老兄弟們自在哩。」九老道:「大聖當年若存正,不鬧天宮,比我們還自在哩。如今好了,聞你歸真向西拜佛,如何得暇至此?」行者將那醫樹求方之事,具陳了一遍。九老也大驚道:「你也忒惹禍,惹禍!我等實是無方。」

行者道:「既是無方,我且奉別。」九老又留他飲瓊漿,食碧藕。行者定不肯坐,止立飲了一杯漿,吃了一塊藕,急急離了瀛洲,徑轉東洋大海。早望見落伽山不遠,遂落下雲頭,直到普陀巖上,見觀音菩薩在紫竹林中與諸天大神、木叉、龍女講經說法。有詩為證。詩曰:
\begin{quote}
海主城高瑞氣濃,更觀奇異事無窮。
須知隱約千般外,盡出希微一品中。
四聖授時成正果,六凡聽後脫樊籠。
少林別有真滋味,花果馨香滿樹紅。
\end{quote}

那菩薩早已看見行者來到,即命守山大神去迎。那大神出林來,叫聲:「孫悟空,那裡去?」行者擡頭喝道:「你這個熊羆,悟空可是你叫的?當初不是老孫饒了你,你已是做了黑風山的屍鬼矣。今日跟了菩薩,受了善果,居此仙山,常聽法教,你叫不得我一聲『老爺』?」那黑熊真個得了正果,在菩薩處鎮守普陀,稱為大神,是也虧了行者。他只得陪笑道:「大聖,古人云:『君子不念舊惡。』只管題他怎的?菩薩著我來迎你哩。」這行者就端肅尊誠,與大神到了紫竹林裡,參拜菩薩。

菩薩道:「悟空,唐僧行到何處也?」行者道:「行到西牛賀洲萬壽山了。」菩薩道:「那萬壽山有座五莊觀,鎮元大仙你曾會他麼?」行者頓首道:「因是在五莊觀,弟子不識鎮元大仙,毀傷了他的人參果樹,衝撞了他,他就困滯了我師父,不得前進。」那菩薩情知,怪道:「你這潑猴不知好歹,他那人參果樹乃天開地闢的靈根。鎮元子乃地仙之祖,我也讓他三分,你怎麼就打傷他樹?」行者再拜道:「弟子實是不知。那一日他不在家,只有兩個仙童候待我等。是豬悟能曉得他有果子,要一個嘗新,弟子委偷了他三個,兄弟們分吃了。那童子知覺,罵我等無已,是弟子發怒,遂將他樹推倒。他次日回來趕上,將我等一袖子籠去,繩綁鞭抽,拷打了一日。我等當夜走脫,又被他趕上,依然籠了。三番兩次,其實難逃。已允了與他醫樹,卻才自海上求方,遍遊三島,眾神仙都沒有本事。弟子因此志心朝禮,特拜告菩薩,伏望慈憫,俯賜一方,以救唐僧早早西去。」菩薩道:「你怎麼不早來見我,卻往島上去尋找?」

行者聞得此言,心中暗喜道:「造化了,造化了,菩薩一定有方也。」他又上前懇求。菩薩道:「我這淨瓶底的甘露水,善治得仙樹靈苗。」行者道:「可曾經驗過麼?」菩薩道:「經驗過的。」行者問:「有何經驗?」菩薩道:「當年太上老君曾與我賭勝:他把我的楊柳枝拔了去,放在煉丹爐裡,炙得焦乾,送來還我。是我拿了插在瓶中,一晝夜,復得青枝綠葉,與舊相同。」行者笑道:「真造化了,真造化了。烘焦了的尚能醫活,況此推倒的,有何難哉?」菩薩吩咐大眾:「看守林中,我去去來。」遂手托淨瓶,白鸚哥前邊巧囀,孫大聖隨後相從。有詩為證。詩曰:
\begin{quote}
玉毫金象世難論,正是慈悲救苦尊。
過去劫逢無垢佛,至今成得有為身。
幾生慾海澄清浪,一片心田絕點塵。
甘露久經真妙法,管教寶樹永長春。
\end{quote}

卻說那觀裡大仙與三老正然清話,忽見孫大聖按落雲頭,叫道:「菩薩來了,快接,快接。」慌得那三星與鎮元子共三藏師徒,一齊迎出寶殿。菩薩才住了祥雲,先與鎮元子陪了話,後與三星作禮,禮畢上坐。那階前,行者引唐僧、八戒、沙僧都拜了。那觀中諸仙也來拜見。行者道:「大仙不必遲疑,趁早兒陳設香案,請菩薩替你治那甚麼果樹去。」大仙躬身謝菩薩道:「小可的勾當,怎麼敢勞菩薩下降?」菩薩道:「唐僧乃我之弟子,孫悟空衝撞了先生,理當賠償寶樹。」三老道:「既如此,不須謙講了,請菩薩都到園中去看看。」

那大仙即命設具香案,打掃後園,請菩薩先行,三老隨後。三藏師徒與本觀眾仙都到園內觀看時,那棵樹倒在地下,土開根現,葉落枝枯。菩薩叫:「悟空,伸手來。」那行者將左手伸開。菩薩將楊柳枝蘸出瓶中甘露,把行者手心裡畫了一道起死回生的符字,教他放在樹根之下,但看水出為度。那行者捏著拳頭,往那樹根底下揣著,須臾,有清泉一汪。菩薩道:「那個水不許犯五行之器,須用玉瓢舀出,扶起樹來,從頭澆下,自然根皮相合,葉長芽生,枝青果出。」行者道:「小道士們,快取玉瓢來。」鎮元子道:「貧道荒山沒有玉瓢,只有玉茶盞、玉酒杯,可用得麼?」菩薩道:「但是玉器,可舀得水的便罷,取將來看。」大仙即命小童子取出有二三十個茶盞、四五十個酒盞,卻將那根下清泉舀出。行者、八戒、沙僧扛起樹來,扶得周正,擁上土,將玉器內甘泉,一甌甌捧與菩薩。菩薩將楊柳枝細細灑上,口中又念著經咒。不多時,灑淨那舀出之水,見那樹果然依舊青綠葉陰森,上有二十三個人參果。清風、明月二童子道:「前日不見了果子時,顛倒只數得二十二個;今日回生,怎麼又多了一個?」行者道:「『日久見人心。』前日老孫只偷了三個,那一個落下地來,土地說這寶遇土而入,八戒只嚷我打了偏手,故走了風信,只纏到如今,才見明白。」菩薩道:「我方才不用五行之器者,知道此物與五行相畏故耳。」

那大仙十分歡喜,急令取金擊子來,把果子敲下十個,請菩薩與三老復回寶殿,一則謝勞,二來做個人參果會。眾小仙遂調開桌椅,鋪設丹盤,請菩薩坐了上面正席,三老左席,唐僧右席,鎮元子前席相陪,各食了一個。有詩為證。詩曰:
\begin{quote}
萬壽山中古洞天,人參一熟九千年。
靈根現出芽枝損,甘露滋生果葉全。
三老喜逢皆舊契,四僧幸遇是前緣。
自今會服人參果,盡是長生不老仙。
\end{quote}

此時菩薩與三老各吃了一個,唐僧始知是仙家寶貝,也吃了一個,悟空三人亦各吃一個,鎮元子陪了一個,本觀仙眾分吃了一個。行者才謝了菩薩回上普陀巖,送三星徑轉蓬萊島。鎮元子卻又安排蔬酒,與行者結為兄弟。這才是不打不成相識,兩家合了一家。師徒四眾,喜喜歡歡,天晚歇了。那長老才是:
\begin{quote}
有緣吃得草還丹,長壽苦捱妖怪難。
\end{quote}

畢竟到明日如何作別,且聽下回分解。
