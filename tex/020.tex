
\chapter{黃風嶺唐僧有難 半山中八戒爭先}

\begin{quote}
法本從心生,還是從心滅。
生滅盡由誰?請君自辨別。
既然皆己心,何用別人說?
只須下苦功,扭出鐵中血。
絨繩著鼻穿,挽定虛空結。
拴在無為樹,不使他顛劣。
莫認賊為子,心法都忘絕。
休教他瞞我,一拳先打徹。
現心亦無心,現法法也輟。
人牛不見時,碧天光皎潔。
秋月一般圓,彼此難分別。
\end{quote}

這一篇偈子,乃是玄奘法師悟徹了《多心經》,打開了門戶,那長老常念常存,一點靈光自透。

且說他三眾在路餐風宿水,帶月披星,早又至夏景炎天。但見那:
\begin{quote}
花盡蝶無情敘,樹高蟬有聲喧。
野蠶成繭火榴妍,沼內新荷出現。
\end{quote}

那日正行時,忽然天晚,又見山路傍邊有一村舍。三藏道:「悟空,你看那日落西山藏火鏡,月升東海現冰輪。幸而道傍有一人家,我們且借宿一宵,明日再走。」八戒道:「說得是,我老豬也有些餓了,且到人家化些齋吃,有力氣,好挑行李。」行者道:「這個戀家鬼,你離了家幾日,就生報怨。」八戒道:「哥啊,比不得你這喝風啊煙的人。我從跟了師父這幾日,長忍半肚饑,你可曉得?」三藏聞之道:「悟能,你若是在家心重啊,不是個出家的了,你還回去罷。」那獃子慌得跪下道:「師父,你莫聽師兄之言,他有些贓埋人。我不曾報怨甚的,他就說我報怨。我是個直腸的痴漢,我說道肚內饑了,好尋個人家化齋,他就罵我是戀家鬼。師父啊,我受了菩薩的戒行,又承師父憐憫,情願要伏侍師父往西天去,誓無退悔。這叫做『恨苦修行』。怎的說不是出家的話?」三藏道:「既是如此,你且起來。」

那獃子縱身跳起,口裡絮絮叨叨的,挑著擔子,只得死心塌地,跟著前來。早到了路傍人家門首。三藏下馬,行者接了韁繩,八戒歇了行李,都佇立綠蔭之下。三藏拄著九環錫杖,按按藤纏篾織斗篷,先奔門前。只見一老者,斜倚竹床之上口裡嚶嚶的念佛。三藏不敢高言,慢慢的叫一聲:「施主,問訊了。」那老者一骨魯跳將起來,忙斂衣襟,出門還禮道:「長老,失迎。你自那方來的?到我寒門何故?」三藏道:「貧僧是東土大唐和尚,奉聖旨,上雷音寺拜佛求經。適至寶方天晚,意投檀府告借一宵,萬祈方便方便。」那老兒擺手搖頭道:「去不得,西天難取經。要取經,往東天去罷。」三藏口中不語,意下沉吟:「菩薩指道西去,怎麼此老說往東行?東邊那得有經?」靦腆難言,半晌不答。

卻說行者素性兇頑,忍不住,上前高叫道:「那老兒,你這們大年紀,全不曉事。我出家人遠來借宿,就把這厭鈍的話虎諕我。十分你家窄狹,沒處睡時,我們在樹底下,好道也坐一夜,不打攪你。」那老者扯住三藏道:「師父,你倒不言語,你那個徒弟,那般拐子臉別頦腮,雷公嘴,紅眼睛,一個癆病魔鬼,怎麼反衝撞我這年老之人?」行者笑道:「你這個老兒,忒也沒眼色。似那俊刮些兒的,叫做中看不中吃。想我老孫雖小,頗結實,皮裹一團筋哩。」那老者道:「你想必有些手段。」行者道:「不敢誇言,也將就看得過。」老者道:「你家居何處?因甚事削髮為僧?」行者道:「老孫祖貫東勝神洲海東傲來國花果山水簾洞居住。自小兒學做妖怪,稱名悟空。憑本事,做了一個齊天大聖。只因不受天錄,大反天宮,惹了一場災愆。如今脫難消災,轉拜沙門,前求正果。保我這唐朝駕下的師父,上西天拜佛走遭,怕甚麼山高路險,水闊波狂?我老孫也捉得怪,降得魔,伏虎擒龍,踢天弄井,都曉得些兒。倘若府上有甚麼丟磚打瓦、鍋叫門開,老孫便能安鎮。」

那老兒聽得這篇言語,哈哈笑道:「原來是個撞頭化緣的熟嘴兒和尚。」行者道:「你兒子便是熟嘴。我這些時,只因跟我師父走路辛苦,還懶說話哩。」那老兒道:「若是你不辛苦,不懶說話,好道活活的聒殺我。你既有這樣手段,西方也還去得,去得。你一行幾眾?請至茅舍裡安宿。」三藏道:「多蒙老施主不叱之恩。我一行三眾。」老者道:「那一眾在那裡?」行者指著道:「這老兒眼花,那綠蔭下站的不是?」老兒果然眼花,忽擡頭細看,一見八戒這般嘴臉,就諕得一步一跌,往屋裡亂跑,只叫:「關門,關門,妖怪來了!」行者趕上扯住道:「老兒莫怕,他不是妖怪,是我師弟。」老者戰兢兢的道:「好好好,一個醜似一個的和尚。」八戒上前道:「老官兒,你若以相貌取人,乾淨差了。我們醜自醜,卻都有用。」

那老者正在門前與三個和尚相講,只見那莊南邊有兩個少年人,帶著一個老媽媽、三四個小男女,斂衣赤腳,插秧而回。他看見一匹白馬、一擔行李,都在他家門首喧嘩,不知是甚來歷,都一擁上前問道:「做甚麼的?」八戒調過頭來,把耳朵擺了幾擺,長嘴伸了一伸,嚇得那些人東倒西歪,亂蹡亂跌。慌得那三藏滿口招呼道:「莫怕,莫怕。我們不是歹人,我們是取經的和尚。」那老兒才出了門,攙著媽媽道:「婆婆起來,少要驚恐。這師父是唐朝來的,只是他徒弟臉嘴醜些,卻也面惡人善。帶男女們家去。」那媽媽才扯著老兒,二少年領著兒女進去。

三藏卻坐在他門樓裡竹床之上,埋怨道:「徒弟呀,你兩個相貌既醜,言語又粗,把這一家兒嚇得七損八傷,都替我身造罪哩。」八戒道:「不瞞師父說,老豬自從跟了你,這些時俊了許多哩。若像往常在高老莊時,把嘴朝前一掬,把耳兩頭一擺,常嚇殺二三十人哩。」行者笑道:「獃子不要亂說,把那醜也收拾起些。」三藏道:「你看悟空說的話,相貌是生成的,你教他怎麼收拾?」行者道:「把那個耙子嘴揣在懷裡,莫拿出來;把那蒲扇耳貼在後面,不要搖動:這就是收拾了。」那八戒真個把嘴揣了,把耳貼了,拱著頭,立於左右。行者將行李拿入門裡,將白馬拴在樁上。

只見那老兒才引個少年,拿一個板盤兒,托三杯清茶來獻。茶罷,又吩咐辦齋。那少年又拿一張有窟窿無漆水的舊桌,端兩條破頭折腳的凳子,放在天井中,請三眾涼處坐下。三藏方問道:「老施主高姓?」老者道:「在下姓王。」「有幾位令嗣?」道:「有兩個小兒,三個小孫。」三藏道:「恭喜,恭喜。」又問:「年壽幾何?」道:「痴長六十一歲。」行者道:「好,好,好,花甲重逢矣。」三藏復問道:「老施主,始初說西天經難取者,何也?」老者道:「經非難取,只是道中艱澀難行。我們這向西去,只有三十里遠近,有一座山,叫做八百里黃風嶺,那山中多有妖怪。故言難取者,此也。若論此位小長老,說有許多手段,卻也去得。」行者道:「不妨,不妨。有了老孫與我這師弟,任他是甚麼妖怪,不敢惹我。」

正說處,又見兒子拿將飯來,擺在桌上,道聲:「請齋。」三藏就合掌諷起齋經。八戒早已吞了一碗。長老的幾句經還未了,那獃子又吃夠三碗。行者道:「這個饢糠的,好道撞著餓鬼了。」那老王倒也知趣,見他吃得快,道:「這個長老,想著實餓了,快添飯來。」那獃子真個食腸大,看他不擡頭,一連就吃有十數碗。三藏、行者俱各吃不上兩碗。獃子不住,便還吃哩。老王道:「倉卒無殽,不敢苦勸,請再進一箸。」三藏、行者俱道:「夠了。」八戒道:「老兒滴答甚麼,誰和你發課,說甚麼五爻六爻?有飯只管添將來就是。」獃子一頓,把他一家子飯都吃得罄盡,還只說才得半飽。卻才收了家火,在那門樓下,安排了竹床板鋪睡下。

次日天曉,行者去背馬,八戒去整擔。老王又教媽媽整治些點心湯水管待,三眾方致謝告行。老者道:「此去倘路間有甚不虞,是必還來茅舍。」行者道:「老兒,莫說哈話。我們出家人不走回頭路。」遂此策馬挑擔西行。

噫!這一去,果無好路朝西域,定有邪魔降大災。三眾前來,不上半日,果逢一座高山,說起來十分險峻。三藏馬到臨崖,斜挑寶觀看,果然那:
\begin{quote}
高的是山,峻的是嶺;陟的是崖,深的是壑;響的是泉,鮮的是花。那山高不高,頂上接青霄;這澗深不深,底中見地府。山前面,有骨都都白雲,屹嶝嶝怪石,說不盡千丈萬丈挾魂崖。崖後有彎彎曲曲藏龍洞,洞中有叮叮噹噹滴水巖。又見些丫丫叉叉帶角鹿,泥泥痴痴看人獐,盤盤曲曲紅鱗蟒,耍耍頑頑白面猿。至晚巴山尋穴虎,帶曉翻波出水龍,登的洞門唿喇喇響。草裡飛禽撲轤轤起,林中走獸掬行。猛然一陣狼蟲過,嚇得人心趷蹬蹬驚。正是那當倒洞當當倒洞,洞當當倒洞當山。青岱染成千丈玉,碧紗籠罩萬堆煙。
\end{quote}

那師父緩促銀驄,孫大聖停雲慢步,豬悟能磨擔徐行。正看那山,忽聞得一陣旋風大作。三藏在馬上心驚,道:「悟空,風起了。」行者道:「風卻怕他怎的?此乃天家四時之氣,有何懼哉?」三藏道:「此風甚惡,比那天風不同。」行者道:「怎見得不比天風?」三藏道:「你看這風:
\begin{quote}
巍巍蕩蕩颯飄飄,渺渺茫茫出碧霄。
過嶺只聞千樹吼,入林但見萬竿搖。
岸邊擺柳連根動,園內吹花帶葉飄。
收網漁舟皆緊纜,落篷客艇盡拋錨。
途半征夫迷失路,山中樵子擔難挑。
仙果林間猴子散,奇花叢內鹿兒逃。
崖前檜柏顆顆倒,澗下松篁葉葉凋。
播土揚塵沙迸迸,翻江攪海浪濤濤。」
\end{quote}

八戒上前一把扯住行者道:「師兄,十分風大,我們且躲一躲兒乾淨。」行者笑道:「兄弟不濟。風大時就躲,倘或親面撞見妖精,怎的是好?」八戒道:「哥啊,你不曾聞得『避色如避仇,避風如避箭』哩?我們躲一躲,也不虧人。」行者道:「且莫言語,等我把這風抓一把來聞一聞看。」八戒笑道:「師兄又扯空頭謊了,風又好抓得過來聞?就是抓得來,便也鑽了去了。」行者道:「兄弟,你不知道老孫有個『抓風』之法。」好大聖,讓過風頭,把那風尾抓過來聞了一聞,有些腥氣。道:「果然不是好風,這風的味道不是虎風,定是怪風,斷乎有些蹊蹺。」

說不了,只見那山坡下剪尾跑蹄,跳出一隻斑斕猛虎。慌得那三藏坐不穩雕鞍,翻根頭跌下白馬,斜倚在路傍,真個是魂飛魄散。八戒丟了行李,掣釘鈀,不讓行者走上前,大喝一聲道:「孽畜,那裡走!」趕將去,劈頭就築。那隻虎直挺挺站將起來,把那前左爪掄起,摳住自家的胸膛,往下一抓,滑剌的一聲,把個皮剝將下來,站立道傍。你看他怎生惡相?咦!那模樣:
\begin{quote}
血津津的赤剝身軀,紅媸媸的彎環腿足。
火燄燄的兩鬢蓬鬆,硬搠搠的雙眉直豎。
白森森的四個鋼牙,光耀耀的一雙金眼。
氣昂昂的努力大哮,雄糾糾的厲聲高喊。
\end{quote}

喊道:「慢來,慢來。吾當不是別人,乃是黃風大王部下的前路先鋒。今奉大王嚴命,在山巡邏,要拿幾個凡夫去做案酒。你是那裡來的和尚,敢擅動兵器傷我?」八戒罵道:「我把你這個孽畜!你是認不得我。我等不是那過路的凡夫,乃東土大唐御弟三藏之弟子,奉旨上西方拜佛求經者。你早早的遠避他方,讓開大路,休驚了我師父,饒你性命;若似前猖獗,鈀舉處,卻不留情。」那妖精那容分說,急近步,丟一個架子,望八戒劈臉來抓;這八戒忙閃過,掄鈀就築。那怪手無兵器,回身就走;八戒隨後趕來;那怪到了山坡下亂石叢中,取出兩口赤銅刀,急掄起,轉身來迎。兩個在這坡前一往一來,一衝一撞的賭鬥。

那孫行者攙起唐僧道:「師父,你莫害怕。且坐住,等老孫去助助八戒,打倒那怪好走。」三藏才坐將起來,戰兢兢的,口裡念著《多心經》不題。

那行者掣了鐵棒,喝聲叫:「拿了!」此時八戒抖擻精神,那怪敗下陣去。行者道:「莫饒他,務要趕上。」他兩個掄起鈀,舉鐵棒,趕下山來。那怪慌了手腳,使個金蟬脫殼計,打個滾,現了原身,依然是一隻猛虎。行者與八戒那裡肯捨,趕著那虎,定要除根。那怪見他趕得至近,卻又摳著胸膛,剝下皮來,苫蓋在那臥虎石上,脫真身,化一陣狂風,徑回路口。忽見著那師父正念《多心經》,被他一把拿住,駕長風攝將去了。可憐那三藏啊,江流註定多磨折,寂滅門中功行難。

那怪把唐僧擒來洞口,按住狂風,對把門的道:「你去報大王說,前路虎先鋒拿了一個和尚,在門外聽令。」那洞主傳令,教拿進來。那虎先鋒腰插著兩口赤銅刀,雙手捧著唐僧,上前跪下道:「大王,小將不才,蒙鈞令差往山上巡邏,忽遇一個和尚,他是東土大唐駕下御弟三藏法師,上西方拜佛求經,被我擒來奉上,聊具一饌。」

那洞主聞得此言,吃了一驚道:「我聞得前者有人傳說:三藏法師乃大唐奉旨意取經的神僧;他手下有一個徒弟,名喚孫行者,神通廣大,智力高強。你怎麼能夠捉得他來?」先鋒道:「他有兩個徒弟:先來的使一柄九齒釘鈀,他生得嘴長耳大;又一個使一根金箍鐵棒,他生得火眼金睛。正趕著小將爭持,被小將使一個金蟬脫殼之計,撤身得空,把這和尚拿來,奉獻大王,聊表一餐之敬。」洞主道:「且莫吃他哩。」先鋒道:「大王,見食不食,呼為劣蹶?」洞主道:「你不曉得。吃了他不打緊,只恐怕他那兩個徒弟上門吵鬧,未為穩便。且把他綁在後園定風樁上,待三五日,他兩個不來攪擾,那時節,一則圖他身子乾淨,二來不動口舌,卻不任我們心意?或煮或蒸,或煎或炒,慢慢的自在受用不遲。」先鋒大喜道:「大王深謀遠慮,說得有理。」教:「小的們,拿了去。」

旁邊擁上七八個綁縛手,將唐僧拿去,好便似鷹拿燕雀,索綁繩纏。這的是苦命江流思行者,遇難神僧想悟能。道聲:「徒弟啊!不知你在那山擒怪,何處降妖,我卻被魔頭拿來,遭此毒害,幾時再得相見?好苦啊!你們若早些兒來,還救得我命;若十分遲了,斷然不能保矣。」一邊嗟嘆,一邊淚落如雨。

卻說那行者、八戒趕那虎下山坡,只見那虎跑倒了,塌伏在崖前。行者舉棒儘力一打,轉震得自己手疼。八戒復築了一鈀,亦將鈀齒迸起。原來是一張虎皮,蓋著一塊臥虎石。行者大驚道:「不好了,不好了,中了他計也!」八戒道:「中他甚計?」行者道:「這個叫做金蟬脫殼計:他將虎皮蓋在此,他卻走了。我們且回去看看師父,莫遭毒手。」兩個急急轉來,早已不見了三藏。行者大叫如雷道:「怎的好?師父已被他擒去了。」八戒即便牽著馬,眼中滴淚道:「天哪,天哪!卻往那裡找尋?」行者擡著頭道:「莫哭,莫哭,一哭就挫了銳氣。橫豎想只在此山,我們尋尋去來。」

他兩個果奔入山中,穿崗越嶺,行夠多時,只見那石崖之下聳出一座洞府。兩人定步觀瞻,果然兇險。但見那:
\begin{quote}
疊障尖峰,迴巒古道。青松翠竹依依,綠柳碧梧冉冉。崖前有怪石雙雙,林內有幽禽對對。澗水遠流沖石壁,山泉細滴漫沙堤。野雲片片,瑤草芊芊。妖狐狡兔亂攛梭,角鹿香獐齊鬥勇。劈崖斜掛萬年籐,深壑半懸千歲柏。奕奕巍巍欺華嶽,落花啼鳥賽天臺。
\end{quote}

行者道:「賢弟,你可將行李歇在藏風山凹之間,撒放馬匹,不要出頭。等老孫去他門首與他賭鬥,必須拿住妖精,方才救得師父。」八戒道:「不消吩咐,請快去。」

行者整一整直裰,束一束虎裙,掣了棒,撞至那門前,只見那門上有六個大字,乃「黃風嶺黃風洞」。卻便丁字腳站定,執著棒,高叫道:「妖怪,趁早兒送我師父出來,省得掀翻了你窩巢,屣平了你住處。」那小怪聞言,一個個害怕,戰兢兢的跑入裡面報道:「大王,禍事了。」那黃風怪正坐間,問:「有何事?」小妖道:「洞門外來了一個雷公嘴毛臉的和尚,手持著一根許大粗的鐵棒,要他師父哩。」那洞主驚張,即喚虎先鋒道:「我教你去巡山,只該拿些山牛、野彘、肥鹿、胡羊,怎麼拿那唐僧來,卻惹他那徒弟來此鬧吵,怎生區處?」先鋒道:「大王放心穩便,高枕勿憂。小將不才,願帶領五十個小校出去,把那甚麼孫行者拿來湊吃。」洞主道:「我這裡除了大小頭目,還有五七百名小校,憑你選擇,領多少去。只要拿住那行者,我們才自自在在吃那和尚一塊肉,情願與你拜為兄弟;但恐拿他不得,反傷了你,那時休得埋怨我也。」虎怪道:「放心,放心。等我去來。」

果然點起五十名精壯小妖,擂鼓搖旗,纏兩口赤銅刀,騰出門來,厲聲高叫道:「你是那裡來的個猴和尚,敢在此間大呼小叫的做甚?」行者罵道:「你這個剝皮的畜生!你弄甚麼脫殼法兒,把我師父攝了,倒轉問我做甚。趁早好好送我師父出來,還饒你這個性命。」虎怪道:「你師父是我拿了,要與我大王做頓下飯。你識起倒,回去罷;不然,拿住你,一齊湊吃,卻不是買一個又饒一個?」行者聞言,心中大怒,扢迸迸鋼牙錯嚙,滴流流火眼睜圓,掣鐵棒喝道:「你多大手段,敢說這等大話?休走,看棍。」那先鋒急持刀接住。這一場果然不善,他兩個各顯威能,好殺:
\begin{quote}
那怪是個真鵝卵,悟空是個鵝卵石。
赤銅刀架美猴王,渾如壘卵來擊石。
鳥鵲怎與鳳凰爭,鵓鴿敢和鷹鷂敵。
那怪噴風灰滿山,悟空吐霧雲迷日。
來往不禁三五回,先鋒腰軟全無力。
轉身敗了要逃生,卻被悟空抵死逼。
\end{quote}

那虎怪抵架不住,回頭就走。他原來在那洞主面前說了嘴,不敢回洞,徑往山坡上逃生。行者那裡肯放,執著棒,只情趕來,呼呼吼吼,喊聲不絕,卻趕到那藏風山凹之間。正擡頭,見八戒在那裡放馬。八戒忽聽見呼呼聲喊,回頭觀看,乃是行者趕敗的虎怪,就丟了馬,舉起鈀,刺斜著頭一築。可憐那先鋒,脫身要跳黃絲網,豈知又遇罩魚人,卻被八戒一鈀,築得九個窟窿鮮血冒,一頭腦髓盡流乾。有詩為證,詩曰:
\begin{quote}
三五年前歸正宗,持齋把素悟真空。
誠心要保唐三藏,初秉沙門立此功。
\end{quote}

那獃子一腳屣住他的脊背,兩手掄鈀又築。行者見了,大喜道:「兄弟,正是這等。他領了幾十個小妖,敢與老孫賭鬥,被我打敗了,他轉不往洞跑,卻跑來這裡尋死。虧你接著,不然又走了。」八戒道:「弄風攝師父去的可是他?」行者道:「正是,正是。」八戒道:「你可曾問他師父的下落麼?」行者道:「這怪把師父拿在洞裡,要與他甚麼鳥大王做下飯。老孫惱了,就與他鬥將這裡來,卻被你送了性命。兄弟啊,這個功勞算你的。你可還守著馬與行李,等我把這死怪拖了去,再到那洞口索戰。須是拿得那老妖,方才救得師父。」八戒道:「哥哥說得有理。你去,你去。若是打敗了這老妖,還趕將這裡來,等老豬截住殺他。」

好行者,一隻手提著鐵棒,一隻手拖著死虎,徑至他洞口。正是:
\begin{quote}
法師有難逢妖怪,情性相和伏亂魔。
\end{quote}

畢竟不知此去可降得妖怪,救得唐僧,且聽下回分解。
