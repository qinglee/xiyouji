
\chapter{木母助威征怪物 金公施法滅妖邪}

話說孫大聖牽著馬,挑著擔,滿山頭尋叫師父,忽見豬八戒氣呼呼的跑將來道:「哥哥,你喊怎的?」行者道:「師父不見了,你可曾看見?」八戒道:「我原來只跟唐僧做和尚的,你又捉弄我,教做甚麼將軍。我捨著命,與那妖精戰了一會,得命回來。師父是你與沙僧看著的,反來問我?」行者道:「兄弟,我不怪你。你不知怎麼眼花了,把妖精放回來拿師父。我去打那妖精,教沙和尚看著師父的,如今連沙和尚也不見了。」八戒笑道:「想是沙和尚帶師父那裡出恭去了。」說不了,只見沙僧來到。行者問道:「沙僧,師父那裡去了?」沙僧道:「你兩個眼都昏了,把妖精放將來拿師父。老沙去打那妖精的,師父自家在馬上坐來。」行者氣得暴跳道:「中他計了,中他計了。」沙僧道:「中他甚麼計?」行者道:「這是分瓣梅花計,把我弟兄們調開,他劈心裡撈了師父去了。天天天,卻怎麼好?」止不住腮邊淚滴。八戒道:「不要哭,一哭就膿包了。橫豎不遠,只在這座山上,我們尋去來。」

三人沒計奈何,只得入山找尋。行了有二十里遠近,只見那懸崖之下,有一座洞府:
\begin{quote}
削峰掩映,怪石嵯峨。奇花瑤草馨香,紅杏碧桃豔麗。崖前古樹,霜皮溜雨四十圍;門外蒼松,黛色參天二千尺。雙雙野鶴,常來洞口舞清風;對對山禽,每向枝頭啼白晝。簇簇黃藤如掛索,行行煙柳似垂金。方塘積水,深穴依山。方塘積水,隱窮鱗未變的蛟龍;深穴依山,住多年吃人的老怪。果然不亞神仙境,真是藏風聚氣巢。
\end{quote}

行者見了,兩三步跳到門前看處,那石門緊閉,門上橫安著一塊石版,石版上有八個大字,乃「隱霧山折岳連環洞」。行者道:「八戒,動手啊。此間乃妖精住處,師父必在他家也。」那獃子仗勢行兇,舉釘鈀盡力築將去,把他那石頭門築了一個大窟窿,叫道:「妖怪,快送出我師父來,免得釘鈀築倒門,一家子都是了帳。」

守門的小妖急急跑入報道:「大王,闖出禍來了。」老怪道:「有甚禍?」小妖道:「門前有人把門打破,嚷道要師父哩!」老怪大驚道:「不知是那個尋將來也?」先鋒道:「莫怕,等我出去看看。」那小妖奔至前門,從那打破的窟窿處,歪著頭,往外張,見是個長嘴大耳朵,即回頭高叫:「大王莫怕他,這是個豬八戒,沒甚本事,不敢無理。他若無理,開了門,拿他進來湊蒸。怕便只怕那毛臉雷公嘴的和尚。」八戒在外邊聽見道:「哥啊,他不怕我,只怕你哩。師父定在他家了,你快上前。」行者罵道:「潑孽畜,你孫外公在這裡,送我師父出來,饒你命罷。」先鋒道:「大王,不好了,孫行者也尋將來了。」老怪報怨道:「都是你定的甚麼『分瓣分瓣』,卻惹得禍事臨門。怎生結果?」先鋒道:「大王放心,且休埋怨。我記得孫行者是個寬洪海量的猴頭,雖則他神通廣大,卻好奉承。我們拿個假人頭出去哄他一哄,奉承他幾句,只說他師父是我們吃了。若還哄得他去了,唐僧還是我們受用;哄不過,再作理會。」老怪道:「那裡得個假人頭?」先鋒道:「等我做一個兒看。」

好妖怪,將一把衠鋼刀斧,把柳樹根砍做個人頭模樣,噴上些人血,糊糊塗塗的,著一個小怪,使漆盤兒拿至門下,叫道:「大聖爺爺,息怒容稟。」孫行者果好奉承,聽見叫聲「大聖爺爺」,便就止住八戒:「且莫動手,看他有甚話說。」拿盤的小怪道:「你師父被我大王拿進洞來,洞裡小妖村頑,不識好歹,這個來吞,那個來啃,抓的抓,咬的咬,把你師父吃了,只剩了一個頭在這裡也。」行者道:「既吃了便罷,只拿出人頭來,我看是真是假。」那小怪從門窟裡拋出那個頭來。豬八戒見了就哭道:「可憐啊!那們個師父進去,弄做這們個師父出來也。」行者道:「獃子,你且認認是真是假,就哭。」八戒道:「不羞,人頭有個真假的?」行者道:「這是個假人頭。」八戒道:「怎認得是假?」行者道:「真人頭拋出來,撲搭不響;假人頭拋得像梆子聲。你不信,等我拋了你聽。」拿起來往石頭上一摜,噹的一聲響亮。沙和尚道:「哥哥,響哩。」行者道:「響便是個假的。我教他現出本相來你看。」急掣金箍棒,撲的一下打破了。八戒看時,乃是個柳樹根。獃子忍不住罵起來道:「我把你這夥毛團!你將我師父藏在洞裡,拿個柳樹根哄你豬祖宗,莫成我師父是柳樹精變的。」

慌得那拿盤的小怪戰兢兢跑去報道:「難難難,難難難。」老妖道:「怎麼有許多難?」小妖道:「豬八戒與沙和尚倒哄過了,孫行者卻是個販古董的——識貨,識貨。他就認得是個假人頭。如今得個真人頭與他,或者他就去了。」老怪道:「怎麼得個真人頭?我們那剝皮亭內有吃不了的人頭選一個來。」眾妖即至亭內揀了個新鮮的頭,教啃淨頭皮,滑塔塔的,還使盤兒拿出,叫:「大聖爺爺,先前委是個假頭。這個真正是唐老爺的頭,我大王留了鎮宅子的,今特獻出來也。」撲通的把個人頭又從門窟裡拋出,血滴滴的亂滾。

孫行者認得是個真人頭,沒奈何就哭;八戒、沙僧也一齊放聲大哭。八戒噙著淚道:「哥哥,且莫哭。天氣不是好天氣,恐一時弄臭了,等我拿將去,乘生氣埋下再哭。」行者道:「也說得是。」那獃子不嫌穢污,把個頭抱在懷裡,跑上山崖向陽處,尋了個藏風聚氣的所在,取釘鈀築了一個坑,把頭埋了,又築起一個墳塚。才叫沙僧:「你與哥哥哭著,等我去尋些甚麼供養供養。」他就走向澗邊,攀幾根大柳枝,拾幾塊鵝卵石。回至墳前,把柳枝兒插在左右,鵝卵石堆在面前。行者問道:「這是怎麼說?」八戒道:「這柳枝權為松柏,與師父遮遮墳頂;這石子權當點心,與師父供養供養。」行者喝道:「夯貨,人已死了,還將石子兒供他。」八戒道:「表表生人意,權為孝道心。」行者道:「且休胡弄。教沙僧在此:一則廬墓,二則看守行李、馬匹。我和你去打破他的洞府,拿住妖魔,碎屍萬段,與師父報仇去來。」沙和尚滴淚道:「大哥言之極當。你兩個著意,我在此處看守。」

好八戒,即脫了皂錦直裰,束一束著體小衣,舉鈀隨著行者,二人努力向前,不容分辨,徑自把他石門打破,喊聲振天,叫道:「還我活唐僧來耶!」那洞裡大小群妖,一個個魂飛魄散,都報怨先鋒的不是。老妖問先鋒道:「這些和尚打進門來,卻怎處治?」先鋒道:「古人說得好:『手插魚籃,避不得腥。』一不做,二不休,左右帥領家兵殺那和尚去來。」老怪聞言,無計可奈,真個傳令,叫:「小的們,各要齊心,將精銳器械跟我去出征。」果然一齊吶喊,殺出洞門。

這大聖與八戒急退幾步,到那山場平處,抵住群妖,喝道:「那個是出名的頭兒?那個是拿我師父的妖怪?」那群妖扎下營盤,將一面錦繡花旗閃一閃,老怪持鐵杵,應聲高呼道:「那潑和尚,你認不得我?我乃南山大王,數百年放蕩於此。你唐僧已是我拿吃了,你敢如何?」行者罵道:「這個大膽的毛團!你能有多少的年紀,敢稱『南山』二字?李老君乃開天闢地之祖,尚坐於太清之右;佛如來是治世之尊,還坐於大鵬之下;孔聖人是儒教之尊,亦僅呼為『夫子』。你這個孽畜,敢稱甚麼『南山大王』,數百年之放蕩。不要走,吃你外公老爺一棒。」那妖精側身閃過,使杵抵住鐵棒,睜圓眼問道:「你這嘴臉像個猴兒模樣,敢將許多言語壓我?你有甚麼手段,在吾門下猖狂?」行者笑道:「我把你個無名的孽畜!是也不知老孫。你站住,硬著膽,且聽我說:
\begin{quote}
祖居東勝大神洲,天地包含幾萬秋。
花果山頭仙石卵,卵開產化我根苗。
生來不比凡胎類,聖體原從日月儔。
本性自修非小可,天姿穎悟大丹頭。
官封大聖居雲府,倚勢行兇鬥斗牛。
十萬神兵難近我,滿天星宿易為收。
名揚宇宙方方曉,智貫乾坤處處留。
今幸皈依從釋教,扶持長老向西遊。
逢山開路無人阻,遇水支橋有怪愁。
林內施威擒虎豹,崖前復手捉貔貅。
東方果正來西域,那個妖邪敢出頭?
孽畜傷師真可恨,管教時下命將休。」
\end{quote}

那怪聞言,又驚又恨,咬著牙,跳近前來,使鐵杵望行者就打。行者輕輕的用棒架住,還要與他講話,那八戒忍不住,掣鈀亂築那怪的先鋒。先鋒帥眾齊來。這一場在山中平地處混戰,真是好殺:
\begin{quote}
東土大邦上國僧,西方極樂取真經。南山大豹噴風霧,路阻深山獨顯能。施巧計,弄乖伶,無知誤捉大唐僧。相逢行者神通廣,更遭八戒有聲名。群妖混戰山平處,塵土紛飛天不清。那陣上小妖呼哮,槍刀亂舉;這壁廂神僧叱喝,鈀棒齊興。大聖英雄無敵手,悟能精壯喜夯來。南禺老怪,部下先鋒,都為唐僧一塊肉,致令捨死又忘生。這兩個因師性命成仇隙,那兩個為要唐僧忒惡情。往來鬥經多半會,沖衝撞撞沒輸贏。
\end{quote}

孫大聖見那些小妖勇猛,連打不退,即使個分身法,把毫毛拔下一把,嚼在口中,噴出去,叫聲:「變!」都變做本身模樣,一個使一條金箍棒,從前邊往裡打進。那一二百個小妖顧前不能顧後,遮左不能遮右,一個個各自逃生,敗走歸洞。這行者與八戒從陣裡往外殺來,可憐那些不識俊的妖精搪著鈀,九孔血出;挽著棒,骨肉如泥。諕得那南山大王滾風生霧,得命逃回。那先鋒不能變化,早被行者一棒打倒,現出本相,乃是個鐵背蒼狼怪。八戒上前扯著腳,翻過來看了道:「這廝從小兒也不知偷了人家多少豬牙子、羊羔兒吃了。」行者將身一抖,收上毫毛道:「獃子,不可遲慢,快趕老怪,討師父的命去來。」八戒回頭,就不見那些小行者,道:「哥哥的法相兒都去了。」行者道:「我已收來也。」八戒道:「妙啊,妙啊!」兩個喜喜歡歡,得勝而回。

卻說那老怪逃了命回洞,吩咐小妖搬石塊,挑土把前門堵了。那些得命的小妖,一個個戰兢兢的,把門都堵了,再不敢出頭。這行者引八戒,趕至門首吆喝,內無人答應。八戒使鈀築時,莫想得動。行者知之,道:「八戒,莫費氣力,他把門已堵了。」八戒道:「堵了門,師仇怎報?」行者道:「且回上墓前,看看沙僧去。」

二人復至本處,見沙僧還哭哩。八戒越發傷悲,丟了鈀,伏在墳上,手撲著土哭道:「苦命的師父啊!遠鄉的師父啊!那裡再得見你耶!」行者道:「兄弟,且莫悲切。這妖精把前門堵了,一定有個後門出入。你兩個只在此間,等我再去尋看。」八戒滴淚道:「哥啊,仔細著,莫連你也撈去了,我們不好哭得:哭一聲師父,哭一聲師兄,就要哭得亂了。」行者道:「沒事,我自有手段。」

好大聖,收了棒,束束裙,拽開步,轉過山坡,忽聽得潺潺水響。且回頭看處,原來是澗中水響,上溜頭沖泄下來。又見澗那邊有座門兒,門左邊有一個出水的暗溝。他道:「不消講!那就是後門了。若要是原嘴臉,恐有小妖開門看見認得,等我變作個水蛇兒過去。且住,變水蛇恐師父的陰靈兒知道,怪我出家人變蛇纏長。變作個小螃蟹兒過去罷。也不好,恐師父怪我出家人腳多。」即變做一個水老鼠,颼的一聲攛過去,從那出水的溝中,鑽至裡面天井中。探著頭兒觀看,只見那向陽處有個小妖,拿些人肉巴子,一塊塊的理著晒哩。行者道:「我的兒啊,那想是師父的肉吃不了,晒乾巴子防天陰的。我要現本相,趕上前,一棍子打殺,顯得我有勇無謀。且再變化進去,尋那老怪,看是何如。」跳出溝,搖身一變,變做個有翅的螞蟻兒。真個是:
\begin{quote}
力微身小號玄駒,日久藏修有翅飛。
閑渡橋邊排陣勢,喜來床下鬥仙機。
善知雨至常封穴,壘積塵多遂作灰。
巧巧輕輕能爽利,幾番不覺過柴扉。
\end{quote}

他展開翅,無聲無影,一直飛入中堂。只見那老怪煩煩惱惱正坐,有一個小妖從後面跳將來報道:「大王,萬千之喜。」老妖道:「喜從何來?」小妖道:「我才在後門外澗頭上探看,忽聽得有人大哭。即上峰頭望望,原來是豬八戒、孫行者、沙和尚在那裡拜墳痛哭。想是把那個人頭認做唐僧的頭葬下,掆作墳墓哭哩。」行者在暗中聽說,心內歡喜道:「若出此言,我師父還藏在那裡,未曾吃哩。等我再去尋尋,看死活如何,再與他說話。」

好大聖,飛在中堂,東張西看,見傍邊有個小門兒,關得甚緊。即從門縫兒裡鑽去看時,原是個大園子,隱隱的聽得悲聲。徑飛入深處,但見一叢大樹,樹底下綁著兩個人,一個正是唐僧。行者見了,心癢難撓,忍不住,現了本相,近前叫聲:「師父。」那長老認得,滴淚道:「悟空,你來了?快救我一救。悟空,悟空。」行者道:「師父莫只管叫名字:面前有人,怕走了風汛。你既有命,我可救得你。那怪只說已將你吃了,拿個假人頭哄我,我們與他恨苦相持。師父放心,且再熬熬兒,等我把那妖精弄倒,方好來解救。」

大聖念聲咒語,卻又搖身還變做個螞蟻兒,復入中堂,丁在正梁之上。只見那些未傷命的小妖簇簇攢攢,紛紛嚷嚷。內中忽跳出一個小妖,告道:「大王,他們見堵了門,攻打不開,死心塌地,捨了唐僧,將假人頭弄做個墳墓。今日哭一日,明日再哭一日,後日復了三,好道回去。打聽得他們散了啊,把唐僧拿出來,碎劖碎剁,把些大料煎了,香噴噴的大家吃一塊兒,也得個延年長壽。」又一個小妖拍著手道:「莫說,莫說,還是蒸了吃的有味。」又一個說:「煮了吃,還省柴。」又一個道:「他本是個稀奇之物,還著些鹽兒醃醃,吃得長久。」行者在那梁中聽見,心中大怒道:「我師父與你有甚毒情,這般算計吃他?」即將毫毛拔了一把,口中嚼碎,輕輕吹出,暗念咒語,都教變做瞌睡蟲兒,往那眾妖臉上拋去。一個個鑽入鼻中,小妖漸漸打盹,不一時都睡倒了。只有那個老妖睡不穩,他兩隻手揉頭搓臉,不住的打涕噴,捏鼻子。行者道:「莫是他曉得了?與他個雙掭燈。」又拔一根毫毛,依母兒做了,拋在他臉上,鑽於鼻孔內。兩個蟲兒,一個從左進,一個從右入。那老妖起來,伸伸腰,打兩個啊欠,呼呼的也睡倒了。

行者暗喜,才跳下來,現出本相。耳朵裡取出棒來,幌一幌,有鴨蛋粗細,噹的一聲,把旁門打破,跑至後園,高叫:「師父!」長老道:「徒弟,快來解解繩兒,綁壞我了。」行者道:「師父不要忙,等我打殺妖精,再來解你。」急抽身跑至中堂,正舉棍要打,又滯住手道:「不好,等解了師父來打。」復至園中,又思量道:「等打了來救。」如此者兩三番,卻才跳跳舞舞的到園裡。長老見了,悲中作喜道:「猴兒,想是看見我不曾傷命,所以歡喜得沒是處,故這等作跳舞也?」行者才至前,將繩解了,挽著師父就走。又聽得對面樹上綁的人叫道:「老爺捨大慈悲,也救我一命。」長老立定身,叫:「悟空,那個人也解他一解。」行者道:「他是甚麼人?」長老道:「他比我先拿進一日。他是個樵子,說有母親年老,甚是思想,倒是個盡孝的,一發連他都救了罷。」

行者依言,也解了繩索,一同帶出後門,𧿼上石崖,過了陡澗。長老謝道:「賢徒,虧你救了他與我命。悟能、悟淨都在何處?」行者道:「他兩個都在那裡哭你哩。你可叫他一聲。」長老果厲聲高叫道:「八戒,八戒。」那獃子哭得昏頭昏腦的,揩揩鼻涕眼淚道:「沙和尚,師父回家來顯魂哩,在那裡叫我們不是?」行者上前,喝了一聲道:「夯貨,顯甚麼魂?這不是師父來了?」那沙僧擡頭見了,忙忙跪在面前道:「師父,你受了多少苦啊!哥哥怎生救得你來也?」行者把上項事說了一遍。

八戒聞言,咬牙恨齒,忍不住舉起鈀把那墳塚一頓築倒,掘出那人頭,一頓築得稀爛。唐僧道:「你築他為何?」八戒道:「師父啊,不知他是那家的亡人,教我朝著他哭。」長老道:「虧他救了我命哩。你兄弟們打上他門,嚷著要我,想是拿他來搪塞;不然啊,就殺了我也。還把他埋一埋,見我們出家人之意。」那獃子聽長老此言,遂將一包稀爛骨肉埋下,也掆起個墳墓。

行者卻笑道:「師父,你請略坐坐,等我剿除去來。」即又跳下石崖,過澗入洞,把那綁唐僧與樵子的繩索拿入中堂,那老妖還睡著了,即將他四馬攢蹄綑倒,使金箍棒掬起來,握在肩上,徑出後門。豬八戒遠遠的望見道:「哥哥好幹這握頭事。再尋一個兒趁頭挑著不好?」行者到跟前放下,八戒舉鈀就築。行者道:「且住,洞裡還有小妖怪未拿哩。」八戒道:「哥啊,有便帶我進去打他。」行者道:「打又費工夫了,不若尋些柴,教他斷根罷。」那樵子聞言,即引八戒去東凹裡尋了些破梢竹、敗葉松、空心柳、斷根藤、黃蒿、老荻、蘆葦、乾桑,挑了若干,送入後門裡。行者點上火,八戒兩耳搧起風。那大聖將身跳上,抖一抖,收了瞌睡蟲的毫毛。那些小妖及醒來,煙火齊著。可憐!莫想有半個得命。連洞府燒得精空。卻回見師父。師父聽見老妖方醒聲喚,便叫:「徒弟,妖精醒了。」八戒上前一鈀,把老怪築死,現出本相,原來是個艾葉花皮豹子精。行者道:「花皮會吃老虎,如今又會變人。這頓打死,才絕了後患也。」長老謝之不盡,攀鞍上馬。那樵子道:「老爺,向西南去不遠,就是舍下。請老爺到舍,見見家母,叩謝老爺活命之恩,送老爺上路。」

長老欣然,遂不騎馬,與樵子並四眾同行,向西南迤𨓦前來,不多路,果見那:
\begin{quote}
石徑重漫苔蘚,柴門蓬絡藤花。
四面山光連接,一林鳥雀諠譁。
密密松篁交翠,紛紛異卉奇葩。
地僻雲深之處,竹籬茅舍人家。
\end{quote}

遠見一個老嫗倚著柴扉,眼淚汪汪的,兒天兒地的痛哭。這樵子看見自家母親,丟了長老,急忙忙先跑到柴扉前,跪下叫道:「母親,兒來也。」老嫗一把抱住道:「兒啊,你這幾日不來家,我只說是山主拿你去,害了性命,是我心疼難忍。你既不曾被害,何以今日才來?你繩擔、柯斧俱在何處?」樵子叩頭道:「母親,兒已被山主拿去,綁在樹上,實是難得性命,幸虧這幾位老爺。這老爺是東土唐朝往西天取經的羅漢。那老爺倒也被山主拿去綁在樹上。他那三位徒弟老爺神通廣大,把山主一頓打死。卻是個艾葉花皮豹子精;概眾小妖,俱盡燒死,卻將那老老爺解下救出,連孩兒都解救出來。此誠天高地厚之恩。不是他們,孩兒也死無疑了。如今山上太平,孩兒徹夜行走,也無事矣。」

那老嫗聽言,一步一拜,拜接長老四眾,都入柴扉茅舍中坐下。娘兒兩個磕頭稱謝不盡,慌慌忙忙的安排些素齋酬謝。八戒道:「樵哥,我知你府上也寒薄,只可將就一飯,切莫費心大擺佈。」樵子道:「不瞞老爺說,我這山間實是寒薄,沒甚麼香蕈、蘑菰、川椒、大料,只是幾品野菜奉獻老爺,權表寸心。」八戒笑道:「聒噪,聒噪。放快些兒就是,我們肚中饑了。」樵子道:「就有,就有。」果然不多時,展抹桌凳,擺將上來,果是幾盤野菜。但見那:
\begin{quote}
嫩焯黃花菜,酸虀白鼓丁。浮薔馬齒莧,江薺雁腸英。燕子不來香且嫩,芽兒拳小脆還青。爛煮馬藍頭,白熝狗腳跡。貓耳朵,野落蓽,灰條熟爛能中吃;剪刀股,牛塘利,倒灌窩螺操帚薺;碎米薺,萵菜薺,幾品青香又滑膩。油炒烏英花,菱科甚可誇。蒲根菜並茭兒菜,四般近水實清華。看麥娘,嬌且佳;破破納,不穿他。苦麻臺下藩籬架。雀兒綿單,猢猻腳跡,油灼灼煎來只好吃。斜蒿青蒿抱娘蒿,燈蛾兒飛上板蕎蕎。羊耳禿,枸杞頭,加上烏藍不用油。幾般野菜一餐飯,樵子虔心為謝酬。
\end{quote}

師徒們飽餐一頓,收拾起程。那樵子不敢久留,請母親出來再拜,再謝。樵子只是磕頭,取了一條棗木棍,結束了衣裙,出門相送。沙僧牽馬,八戒挑擔,行者緊隨左右,長老在馬上拱手道:「樵哥,煩先引路,到大路上相別。」一齊登高下阪,轉澗尋坡。長老在馬上思量道:「徒弟啊,
\begin{quote}
自從別主來西域,遞遞迢迢去路遙。
水水山山災不脫,妖妖怪怪命難逃。
心心只為經三藏,念念仍求上九霄。
碌碌勞勞何日了,幾時行滿轉唐朝?」
\end{quote}

樵子聞言道:「老爺切莫憂思。這條大路,向西方不滿千里,就是天竺國,極樂之鄉也。」長老聞言,翻身下馬道:「有勞遠涉。既是大路,請樵哥回府,多多拜上令堂老安人:適間厚擾盛齋,貧僧無甚相謝,只是早晚誦經,保佑你母子平安,百年長壽。」那樵子喏喏相辭,復回本路。師徒遂一直投西。正是:
\begin{quote}
降怪解冤離苦厄,受恩上路用心行。
\end{quote}

畢竟不知還有幾日得到西天,且聽下回分解。
