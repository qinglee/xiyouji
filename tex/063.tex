
\chapter{二僧蕩怪鬧龍宮 群聖除邪獲寶貝}

卻說祭賽國王與大小公卿見孫大聖與八戒騰雲駕霧,提著兩個小妖飄然而去,一個個朝天禮拜道:「話不虛傳,今日方知有此輩神仙活佛。」又見他遠去無蹤,卻拜謝三藏、沙僧道:「寡人肉眼凡胎,只知高徒有力量,拿住妖賊便了,豈知乃騰雲駕霧之上仙也。」三藏道:「貧僧無些法力,一路上多虧這三個小徒。」沙僧道:「不瞞陛下說,我大師兄乃齊天大聖皈依,他曾大鬧天宮,使一條金箍棒,十萬天兵,無一個對手,只鬧得太上老君害怕,玉皇大帝心驚。我二師兄乃天蓬元帥果正,他也曾掌管天河八萬水兵大眾。惟我弟子無法力,乃捲簾大將受戒。愚弟兄若幹別事無能,若說擒妖縛怪、拿賊捕亡、伏虎降龍、踢天弄井,以至攪海翻江之類,略通一二。這騰雲駕霧、喚雨呼風,與那換斗移星、擔山趕月,特餘事耳,何足道哉!」國王聞說,愈十分加敬,請唐僧上坐,口口稱為「老佛」,將沙僧等皆稱為「菩薩」。滿朝文武欣然,一國黎民頂禮不題。

卻說孫大聖與八戒駕著狂風,把兩個小妖攝到亂石山碧波潭,住定雲頭。將金箍棒吹了一口仙氣,叫:「變!」變作一把戒刀,將一個黑魚怪割了耳朵,鮎魚精割了下唇,撇在水裡,喝道:「快早去對那萬聖龍王報知,說我齊天大聖孫爺爺在此,著他即送祭賽國金光寺塔上的原寶出來,免他一家性命;若迸半個『不』字,我將這潭水攪淨,教他一門兒老幼遭誅。」

那兩個小妖得了命,負痛逃生,拖著鎖索,淬入水內。諕得那些黿鼉龜鱉,蝦蟹魚精,都來圍住問道:「你兩個為何拖繩帶索?」一個掩著耳,搖頭擺尾;一個侮著嘴,跌腳搥胸。都嚷嚷鬧鬧,徑上龍王宮殿:「報大王,禍事了。」那萬聖龍王正與九頭駙馬飲酒,忽見他兩個來,即停杯問何禍事。那兩個即告道:「昨夜巡探,被唐僧、孫行者掃塔捉獲,用鐵索拴鎖。今早見國王,又被那行者與豬八戒抓著我兩個,一個割了耳朵,一個割了嘴唇,拋在水中,著我來報,要索那塔頂寶貝。」遂將前後事細說了一遍。那老龍聽說是孫行者齊天大聖,諕得魂不附體,魄散九霄,戰兢兢對駙馬道:「賢婿啊,別個來還好計較,若果是他,卻不善也。」駙馬笑道:「太岳放心。愚婿自幼學了些武藝,四海之內,也曾會過幾個豪傑,怕他做甚?等我出去與他交戰三合,管取那廝縮首歸降,不敢仰視。」

好妖怪,急縱身披掛了,使一般兵器,叫做月牙鏟,步出宮,分開水道,在水面上叫道:「是甚麼齊天大聖?快上來納命。」行者與八戒立在岸邊,觀看那妖精怎生打扮:
\begin{quote}
戴一頂爛銀盔,光欺白雪;貫一副兜鍪甲,亮敵秋霜。上罩著錦征袍,真個是彩雲籠玉;腰束著犀紋帶,果然像花蟒纏金。手執著月牙鏟,霞飛電掣;腳穿著豬皮靴,水利波分。遠看時一頭一面,近睹處四面皆人:前有眼,後有眼,八方通見;左也口,右也口,九口言論。一聲吆喝長空振,似鶴飛鳴貫九宸。
\end{quote}

他見無人對答,又叫一聲:「那個是齊天大聖?」行者按一按金箍,理一理鐵棒道:「老孫便是。」那怪道:「你家居何處?身出何方?怎生得到祭賽國,與那國王守塔,卻大膽獲我頭目,又敢行兇,上吾寶山索戰?」行者罵道:「你這賊怪,原來不識你孫爺爺哩。你上前,聽我道:
\begin{quote}
老孫祖住花果山,大海之間水簾洞。
自幼修成不壞身,玉皇封我齊天聖。
只因大鬧斗牛宮,天上諸神難取勝。
當請如來展妙高,無邊智慧非凡用。
為翻觔斗賭神通,手化為山壓我重。
整到如今五百年,觀音勸解方逃命。
大唐三藏上西天,遠拜靈山求佛頌。
解脫吾身保護他,煉魔淨怪從修行。
路逢西域祭賽城。屈害僧人三代命。
我等慈悲問舊情,乃因塔上無光映。
吾師掃塔探分明,夜至三更天籟靜。
捉住妖精取實供,他言汝等偷寶貝。
合盤為盜有龍王,公主連名稱萬聖。
血雨澆淋塔上光,將他寶貝偷來用。
殿前供狀更無虛,我奉君言馳此境。
所以相尋索戰爭,不須再問孫爺姓。
快將寶貝獻還他,免汝老少全家命。
敢若無知騁勝強,教你水涸山頹都蹭蹬。」
\end{quote}

那駙馬聞言,微微冷笑道:「你原來是取經的和尚,沒要緊羅織管事。我偷他的寶貝,你取佛的經文,與你何干,卻來廝鬥?」行者道:「這賊怪甚不達理。我雖不受國王的恩惠,不食他的水米,不該與他出力;但是你偷他的寶貝,污他的寶塔,屢年屈苦金光寺僧人,他是我一門同氣,我怎麼不與他出力,辨明冤枉?」駙馬道:「你既如此,想是要行賭鬥。常言道:『武不善作。』但只怕起手處,不得留情,一時間傷了你的性命,誤了你去取經。」

行者大怒,罵道:「這潑賊怪,有甚強能,敢開大口?走上來,吃老爺一棒。」那駙馬更不心慌,把月牙鏟架住鐵棒,就在那亂石山頭,這一場真個好殺:
\begin{quote}
妖魔盜寶塔無光,行者擒妖報國王。小怪逃生回水內,老龍破膽各商量。九頭駙馬施威武,披掛前來展素強。怒發齊天孫大聖,金箍棒起十分剛。那怪物,九個頭顱十八眼,前前後後放毫光;這行者,一雙鐵臂千斤力,藹藹紛紛並瑞祥。鏟似一陽初現月,棒如萬里遍飛霜。他說:「你無干休把不平報。」我道:「你有意偷寶真不良。那潑賊,少輕狂,還他寶貝得安康。」棒迎鏟架爭高下,不見輸贏練戰場。
\end{quote}

他兩個往往來來,鬥經三十餘合,不分勝負。豬八戒立在山前,見他們戰到甜美之處,舉著釘鈀,從妖精背後一築。原來那怪九個頭,轉轉都是眼睛,看得明白。見八戒在背後來時,即使鏟鐏架著釘鈀,鏟頭抵著鐵棒。又耐戰五七合,擋不得前後齊掄,他卻打個滾,騰空跳起,現了本像,乃是一個九頭蟲。觀其形像十分惡,見此身模怕殺人。他生得:
\begin{quote}
毛羽鋪錦,團身結絮。方圓有丈二規模,長短似黿鼉樣致。兩隻腳尖利如鉤,九個頭攢環一處。展開翅極善飛揚,縱大鵬無他力氣;發起聲遠振天涯,比仙鶴還能高唳。眼多閃灼晃金光,氣傲不同凡鳥類。
\end{quote}

豬八戒看見心驚道:「哥啊,我自為人,也不曾見這等個惡物。是甚血氣生此禽獸也?」行者道:「真個罕有,真個罕有。等我趕上打去。」好大聖,急縱祥雲,跳在空中,使鐵棒照頭便打。那怪物大顯身,展翅斜飛,颼的打個轉身,掠到山前,半腰裡又伸出一個頭來,張開口如血盆相似,把八戒一口咬著鬃,半拖半扯,捉下碧波潭水內而去。及至龍宮外,還變作前番模樣,將八戒擲之於地,叫:「小的們何在?」那裡面鯖鮊鯉鱖之魚精,龜鱉黿鼉之介怪,一擁齊來,道聲:「有。」駙馬道:「把這個和尚綁在那裡,與我巡探的小卒報仇。」眾精推推嚷嚷,擡進八戒去時,那老龍王歡喜,迎出道:「賢婿有功,怎生捉他來也?」那駙馬把上項原故說了一遍。老龍即命排酒賀功不題。

卻說孫行者見妖精擒了八戒,心中懼道:「這廝恁般利害。我待回朝見師,恐那國王笑我;待要開言罵戰,曾奈我又單身,況水面之事不慣。且等我變化了進去,看那怪把獃子怎生擺佈。若得便,且偷他出來幹事。」好大聖,捻著訣,搖身一變,還變做一個螃蟹,淬於水內,徑至牌樓之前。原來這條路是他前番襲牛魔王盜金睛獸走熟了的。直至那宮闕之下,橫爬過去,又見那老龍王與九頭蟲合家兒歡喜飲酒。行者不敢相近,爬過東廊之下,見幾個蝦精蟹精紛紛紜紜耍子。行者聽了一會言談,卻就學語學話,問道:「駙馬爺爺拿來的那長嘴和尚,這會死了不曾?」眾精道:「不曾死,縛在那西廊下哼的不是?」

行者聽說,又輕輕的爬過西廊,真個那獃子綁在柱上哼哩。行者近前道:「八戒,認得我麼?」八戒聽得聲音,知是行者,道:「哥哥,怎麼了?反被這廝捉住我也。」行者四顧無人,將拑咬斷索子叫走。那獃子脫了手道:「哥哥,我的兵器被他收了,又奈何?」行者道:「你可知道收在那裡?」八戒道:「當被那怪拿上宮殿去了。」行者道:「你先去牌樓下等我。」八戒逃生,悄悄的溜出。行者復身爬上宮殿觀看。左首下有光彩森森,乃是八戒的釘鈀放光。使個隱身法,將鈀偷出,到牌樓下,叫聲:「八戒,接兵器。」獃子得了鈀,便道:「哥哥,你先走,等老豬打進宮殿。若得勝,就捉住他一家子;若不勝,敗出來,你在這潭岸上救應。」行者大喜,只教仔細。八戒道:「不怕他,水裡本事,我略有些兒。」行者丟了他,負出水面不題。

這八戒束了皂直裰,雙手纏鈀,一聲喊,打將進去。慌得那大小水族奔奔波波,跑上宮殿,吆喝道:「不好了,長嘴和尚掙斷繩返打進來了。」那老龍與九頭蟲並一家子俱措手不及,跳起來,藏藏躲躲。這獃子不顧死活,闖上宮殿,一路鈀,築破門扇,打破桌椅,把些吃酒的家火之類盡皆打碎。有詩為證。詩曰:
\begin{quote}
木母遭逢水怪擒,心猿不捨苦相尋。
暗施巧計偷開鎖,大顯神威怒恨深。
駙馬忙攜公主躲,龍王戰慄絕聲音。
水宮絳闕門窗損,龍子龍孫盡沒魂。
\end{quote}

這一場,被八戒把玳瑁屏打得粉碎,珊瑚樹摜得凋零。

那九頭蟲將公主安藏在內,急取月牙鏟,趕至前宮,喝道:「潑夯豕彘!怎敢欺心驚吾眷族?」八戒罵道:「這賊怪,你焉敢將我捉來?這場不干我事,是你請我來家打的。快拿寶貝還我,回見國王了事;不然,決不饒你一家命也。」那怪那肯容情,咬定牙齒,與八戒交鋒。那老龍才定了神思,領龍子、龍孫各執槍刀,齊來攻取。八戒見事體不諧,虛幌一鈀,撤身便走。那老龍帥眾追來。須臾,攛出水中,都到潭面上翻騰。

卻說孫行者立於潭岸等候,忽見他們追趕八戒,出離水中,就半踏雲霧,掣鐵棒,喝聲:「休走!」只一下,把個老龍頭打得稀爛。可憐血濺潭中紅水泛,屍飄浪上敗鱗浮。諕得那龍子、龍孫各各逃命,九頭駙馬收龍屍,轉宮而去。

行者與八戒且不追襲,回上岸,備言前事。八戒道:「這廝銳氣挫了,被我那一路鈀打進去時,打得落花流水,魂散魄飛。正與那駙馬廝鬥,卻被老龍王趕著,卻虧了你打死。那廝們回去,一定停喪掛孝,決不肯出來。今又天色晚了,卻怎奈何?」行者道:「管甚麼天晚,乘此機會,你還下去攻戰。務必取出寶貝,方可回朝。」那獃子意懶情疏,徉徉推托。行者催逼道:「兄弟不必多疑,還像剛才引出來,等我打他。」

兩人正自商量,只聽得狂風滾滾,慘霧陰陰,忽從東方徑往南去。行者仔細觀看,乃二郎顯聖,領梅山六兄弟,架著鷹犬,挑著狐兔,擡著獐鹿,一個個腰挎彎弓,手持利刃,縱風霧踴躍而來。行者道:「八戒,那是我七聖兄弟,倒好留請他們,與我助戰。若得成功,倒是一場大機會也。」八戒道:「既是兄弟,極該留請。」行者道:「但內有顯聖大哥,我曾受他降伏,不好見他。你去攔住雲頭,叫道:『真君,且略住住,齊天大聖在此進拜。』他若聽見是我,斷然住了。待他安下,我卻好見。」

那獃子急縱雲頭,上山攔住,厲聲高叫道:「真君,且慢車駕,有齊天大聖請見哩。」那爺爺見說,即傳令,就停住六兄弟,與八戒相見畢,問:「齊天大聖何在?」八戒道:「現在山下聽呼喚。」二郎道:「兄弟們,快去請來。」六兄弟乃是康、張、姚、李、郭、直,各各出營叫道:「孫悟空哥哥,大哥有請。」行者上前,對眾作禮,遂同上山。二郎爺爺迎見,攜手相攙,一同相見,道:「大聖,你去脫大難,受戒沙門,刻日功完,高登蓮座,可賀,可賀。」行者道:「不敢。向蒙莫大之恩,未展斯須之報。雖然脫難西行,未知功行何如。今因路遇祭賽國,答救僧災,在此擒妖索寶。偶見兄長車駕,大膽請留一助。未審兄長自何而來,肯見愛否?」二郎笑道:「我因閑暇無事,同眾兄弟採獵而回。幸蒙大聖不棄留會,足感故舊之情。若命挾力降妖,敢不如命。卻不知此地是何怪賊?」六聖道:「大哥忘了?此間是亂石山,山下乃碧波潭萬聖之龍宮也。」二郎驚訝道:「萬聖老龍卻不生事,怎麼敢偷塔寶?」行者道:「他近日招了一個駙馬,乃是九頭蟲成精。他郎丈兩個做賊,將祭賽國下了一場血雨,把金光寺塔頂舍利佛寶偷來。那國王不解其意,苦拿著僧人拷打。是我師父慈悲,夜來掃搭,當被我在塔上拿住兩個小妖,是他差來巡探的。今早押赴朝中,實實供招了。那國王就請我師收降,師命我等到此。先一場戰,被九頭蟲腰裡伸出一個頭來,把八戒啣了去。我卻又變化下水,解了八戒。才然大戰一場,是我把老龍打死,那廝們收屍掛孝去了。我兩個正議索戰,卻見兄長儀仗降臨,故此輕瀆也。」二郎道:「既傷了老龍,正好與他攻擊,使那廝不能措手,卻不連窩巢都滅絕了?」八戒道:「雖是如此,奈天晚何?」二郎道:「兵家云:『征不待時。』何怕天晚?」

康、姚、郭、直道:「大哥莫忙。那廝家眷在此,料無處去。孫二哥也是貴客,豬剛鬣又歸了正果,我們營內有隨帶的酒餚,教小的們取火,就此鋪設:一則與二位賀喜,二來也當敘情。且歡會這一夜,待天明索戰何遲?」二郎大喜道:「賢弟說得極當。」卻命小校安排。行者道:「列位盛情,不敢固卻。但自做和尚,都是齋戒,恐葷素不便。」二郎道:「有素果品,酒也是素的。」眾兄弟在星月光前,幕天席地,舉杯敘舊。

正是寂寞更長,歡娛夜短。早不覺東方發白。那八戒幾鍾酒吃得興抖抖的道:「天將明了,等老豬下水去索戰也。」二郎道:「元帥仔細,只要引他出來,我兄弟們好下手。」八戒笑道:「我曉得,我曉得。」你看他斂衣纏鈀,使分水法,跳將下去,徑至那牌樓下。發聲喊,打入殿內。此時那龍子披了麻,看著龍屍哭;龍孫與那駙馬,在後面收拾棺材哩。這八戒罵上前,手起處,鈀頭著重,把個龍子夾腦連頭,一鈀築了九個窟窿。諕得那龍婆與眾往裡亂跑,哭道:「長嘴和尚又把我兒打死了。」

那駙馬聞言,即使月牙鏟,帶龍孫往外殺來。這八戒舉鈀迎敵,且戰且退,跳出水中。這岸上齊天大聖與七兄弟一擁上前,槍刀亂扎,把個龍孫剁成幾斷肉餅。那駙馬見不停當,在山前打個滾,又現了本像,展開翅,旋繞飛騰。二郎即取金弓,安上銀彈,扯滿弓,往上就打。那怪急鎩翅,掠到邊前,要咬二郎。半腰裡才伸出一個頭來,被那頭細犬攛上去,汪的一口,把頭血淋淋的咬將下來。那怪物負痛逃生,徑投北海而去。八戒便要趕去,行者止住道:「且莫趕他,正是『窮寇勿追』。他被細犬咬了頭,必定是多死少生。等我變做他的模樣,你分開水路,趕我進去,尋那宮主,詐他寶貝來也。」二郎與六聖道:「不趕他倒也罷了,只是遺這種類在世,必為後人之害。」至今有個九頭蟲滴血,是遺種也。

那八戒依言,分開水路。行者變作怪像前走,八戒吆吆喝喝後追。漸漸追至龍宮,只見那萬聖宮主道:「駙馬,怎麼這等慌張?」行者道:「那八戒得勝,把我趕將進來,覺道不能敵他。你快把寶貝好生藏了。」那宮主急忙難識真假,即於後殿裡取出一個渾金匣子來,遞與行者道:「這是佛寶。」又取出一個白玉匣子,也遞與行者道:「這是九葉靈芝。你拿這寶貝藏去,等我與豬八戒鬥上兩三合,擋住他。你將寶貝收好了,再出來與他合戰。」行者將兩個匣兒收在身邊,把臉一抹,現了本像道:「宮主,你看我可是駙馬麼?」宮主慌了,便要搶奪匣子。被八戒跑上去,著背一鈀,築倒在地。

還有一個老龍婆撤身就走,被八戒扯住,舉鈀才築,行者道:「且住,莫打死他,留個活的,好去國內見功。」遂將龍婆提出水面。

行者隨後捧著兩個匣子上岸,對二郎道:「感兄長威力,得了寶貝,掃淨妖賊也。」二郎道:「一則是那國王洪福齊天,二則是賢昆玉神通無量,我何功之有?」兄弟們俱道:「孫二哥既已功成,我們就此告別。」行者感謝不盡,欲留同見國王。諸公不肯,遂帥眾回灌口去訖。

行者捧著匣子,八戒拖著龍婆,半雲半霧,頃刻間到了國內。原來那金光寺解脫的和尚都在城外迎接。忽見他兩個雲霧定時,近前磕頭禮拜,接入城中。那國王與唐僧正在殿上講論。這裡有先走的和尚,仗著膽,入朝門奏道:「萬歲,孫、豬二老爺擒賊獲寶而來也。」那國王聽說,連忙下殿,共唐僧、沙僧迎著,稱謝神功不盡,隨命排筵謝恩。三藏道:「且不須賜飲,著小徒歸了塔中之寶,方可飲宴。」三藏又問行者道:「汝等昨日離國,怎麼今日才來?」行者把那戰駙馬,打龍王,逢真君,敗妖精,及變作詐寶貝之事,細說了一遍。三藏與國王、大小文武,俱喜之不勝。

國王又問:「龍婆能人言語否?」八戒道:「乃是龍王之妻,生了許多龍子、龍孫,豈不知人言?」國王道:「既知人言,快早說前後做賊之事。」龍婆道:「偷佛寶,我全不知,都是我那夫君龍鬼與那駙馬九頭蟲,知你塔上之光乃是佛家舍利子,三年前下了血雨,乘機盜去。」又問:「靈芝草是怎麼偷的?」龍婆道:「只是小女萬聖宮主私入大羅天上靈霄殿前,偷的王母娘娘九葉靈芝草。那舍利子得這草的仙氣溫養著,千年不壞,萬載生光。去地下或田中掃一掃,即有萬道霞光,千條瑞氣。如今被你奪來,弄得我夫死子絕,婿喪女亡,千萬饒了我的命罷。」八戒道:「正不饒你哩。」行者道:「家無全犯。我便饒你,只便要你長遠替我看塔。」龍婆道:「好死不如惡活。但留我命,憑你教做甚麼。」行者叫取鐵索來。當駕官即取鐵索一條,把龍婆琵琶骨穿了。教沙僧:「請國王來看我們安塔去。」

那國王即忙排駕,遂同三藏攜手出朝,並文武多官,隨至金光寺。行者上塔,將舍利子安在第十三層塔頂寶瓶中間,把龍婆鎖在塔心柱上。念動真言,喚出本國土地、城隍與本寺伽藍們,命三日送飲食一餐,與這龍婆度口;少有差訛,即行處斬。眾神暗中領諾。行者卻將芝草把十三層塔層層掃過,安在瓶內,溫養舍利子。這才是整舊如新,霞光萬道,瑞氣千條,依然八方共睹,四國同瞻。下了塔門,國王就謝道:「不是老佛與三位菩薩到此,怎生得明此事也!」

行者道:「陛下,『金光』二字不好,不是久住之物:金乃流動之物,光乃閃灼之氣。貧僧為你勞碌這場,將此寺改作伏龍寺,教你永遠常存。」那國王即命換了字號,懸上新匾,乃是「敕建護國伏龍寺」。一壁廂安排御宴;一壁廂召丹青寫下四眾生形,五鳳樓註了名號。國王擺鑾駕,送唐僧師徒,賜金玉酬答。師徒們堅辭,一毫不受。這真個是:
\begin{quote}
邪怪剪除諸境靜,寶塔回光大地明。
\end{quote}

畢竟不知此去前路如何,且聽下回分解。
