
\chapter{假合真形擒玉兔 真陰歸正會靈元}

卻說那唐僧憂憂愁愁,隨著國王至後宮。只聽得鼓樂喧天,隨聞得異香撲鼻。低著頭,不敢仰視。行者暗裡欣然,丁在那毘盧帽頂上,運神光,睜火眼金睛觀看。又只見那兩班綵女,擺列的似蕊宮仙府,勝強似錦帳春風。真個是:
\begin{quote}
娉婷嬝娜,玉質冰肌。一雙雙嬌欺楚女,一對對美賽西施。雲髻高盤飛彩鳳,蛾眉微顯遠山低。笙簧雅奏,簫鼓頻吹。宮商角徵羽,抑揚高下齊。清歌妙舞常堪愛,錦砌花團色色怡。
\end{quote}

行者見師父全不動念,暗自裡咂嘴誇稱道:「好和尚,好和尚。身居錦繡心無愛,足步瓊瑤意不迷。」

少時,皇后、嬪妃簇擁著公主出鳷鵲宮,一齊迎接,都道聲:「我王萬歲,萬萬歲。」慌的個長老戰戰兢兢,莫知所措。行者早已知識,見那公主頭頂上微露出一點妖氛,卻也不十分兇惡。即忙爬近耳朵叫道:「師父,公主是個假的。」長老道:「是假的,卻如何放他現相?」行者道:「使出法身,就此拿他也。」長老道:「不可,不可,恐驚了主駕。且待君、后退散,再使法力。」

那行者一生性急,那裡容得,大咤一聲,現了本相,趕上前,揪住公主罵道:「好孽畜!你在這裡弄假成真,只在此這等受用,也儘夠了;心尚不足,還要騙我師父,破他的真陽,遂你的淫性哩。」諕得那國王呆呆掙掙,后妃跌跌爬爬,宮娥綵女無一個不東躲西藏,各顧性命。好便似:
\begin{quote}
春風蕩蕩,秋氣瀟瀟。春風蕩蕩過園林,千花擺動;秋氣瀟瀟來禁苑,萬葉飄搖。刮折牡丹攲檻下,吹歪芍藥臥欄邊。沼岸芙蓉亂撼,臺基菊蕊鋪堆。海棠無力倒塵埃,玫瑰有香眠野境。春風吹折芰荷楟,冬雪壓歪梅嫩蕊。石榴花瓣,亂落在內院東西;岸柳枝條,斜垂在皇宮南北。好花風雨一宵狂,無數殘紅鋪地錦。
\end{quote}

三藏一發慌了手腳,戰兢兢抱住國王,只叫:「陛下,莫怕,莫怕,此是我頑徒使法力,辨真假也。」

卻說那妖精見事不諧,掙脫了手,解剝了衣裳,捽捽頭,搖落了釵環首飾。跑到御花園土地廟裡,取出一條碓嘴樣的短棍,急轉身來亂打行者;行者隨即跟來,使鐵棒劈面相迎。他兩個吆吆喝喝,就在花園內鬥起。後卻大顯神通,各駕雲霧,殺在空中。這一場:
\begin{quote}
金箍鐵棒有名聲,碓嘴短棍無人識。一個因取真經到此方,一個為愛奇花來住跡。那怪久知唐聖僧,要求配合元精液。舊年攝去真公主,變作人身欽愛惜。今逢大聖認妖氛,救援活命分虛實。短棍行兇著頂丟,鐵棒施威迎面擊。喧喧嚷嚷兩相持,雲霧滿天遮白日。
\end{quote}

他兩個殺在半空賭鬥,嚇得那滿城中百姓心慌,盡朝裡多官膽怕。長老扶著國王,只叫:「休驚,請勸娘娘與眾等莫怕。你公主是個假作真形的,等我徒弟拿住他,方知好歹也。」那些妃子有膽大的,把那衣服、釵環拿與皇后看了,道:「這是公主穿的戴的,今都丟下,精著身子,與那和尚在天上爭打,必定是個妖邪。」此時國王、后妃人等才正了性,望空仰視不題。

卻說那妖精與大聖鬥經半日,不分勝敗。行者把棒丟起,叫一聲:「變!」就以一變十,以十變百,以百變千,半天裡,好似蛇遊蟒攪,亂打妖邪。妖邪慌了手腳,化道清風,即奔碧空之上逃走。行者念聲咒語,將鐵棒收做一根,縱祥光一直趕來。將近西天門,望見那旌旗閃灼,行者厲聲高叫道:「把天門的,擋住妖精,不要放他走了。」真個那天門上有護國天王帥領著龐、劉、苟、畢四大元帥,各展兵器攔阻。妖邪不能前進,急回頭,使短棍,又與行者相持。

這大聖掄鐵棒,仔細迎著看時,見那短棍兒一頭奘,一頭細,卻似舂碓臼的杵頭模樣,叱咤一聲,喝道:「孽畜!你拿的是甚麼器械,敢與老孫抵敵?快早降伏,免得這一棒打碎你的天靈。」那妖邪咬著牙道:「你也不知我這兵器,聽我道:
\begin{quote}
仙根是段羊脂玉,磨琢成形不計年。
混沌開時吾已得,洪濛判處我當先。
源流非比凡間物,本性生來在上天。
一體金光和四相,五行瑞氣合三元。
隨吾久住蟾宮內,伴我常居桂殿邊。
因為愛花垂世境,故來天竺假嬋娟。
與君共樂無他意,欲配唐僧了宿緣。
你怎欺心破佳偶,死尋趕戰逞兇頑?
這般器械名頭大,在你金箍棒子前。
廣寒宮裡搗藥杵,打人一下命歸泉。」
\end{quote}

行者聞說,呵呵冷笑道:「好孽畜啊!你既住在蟾宮之內,就不知老孫的手段,你還敢在此支吾?快早現相降伏,饒你性命。」那怪道:「我認得你是五百年前大鬧天宮的弼馬溫,理當讓你。但只是破人親事,如殺父母之仇,故此情理不甘,要打你欺天罔上的弼馬溫。」那大聖惱得是「弼馬溫」三字,他聽得此言,心中大怒,舉鐵棒劈面就打;那妖邪掄杵來迎。就於西天門前,發狠相持。這一場:
\begin{quote}
金箍棒,搗藥杵,兩般仙器真堪比。那個為結婚姻降世間,這個因保唐僧到這裡。原來是國王沒正經,愛花引得妖邪喜。致使如今恨苦爭,兩家都把頑心起。一衝一撞賭輸贏,劖語劖言齊鬥嘴。藥杵英雄世罕稀,鐵棒神威還更美。金光湛湛幌天門,彩霧輝輝連地里。來往戰經十數回,妖邪力弱難搪抵。
\end{quote}

那妖精與行者又鬥了十數回,見行者的棒勢緊密,料難取勝,虛丟一杵,將身幌一幌,金光萬道,徑奔正南上敗走。大聖隨後追襲。忽至一座大山,妖精按金光,鑽入山洞,寂然不見。又恐他遯身回國,暗害唐僧,他認了這山的規模,返雲頭徑轉國內。

此時有申時矣。那國王正扯著三藏,戰戰兢兢,只叫:「聖僧救我。」那些嬪妃、皇后也正愴惶,只見大聖自雲端裡落將下來,叫道:「師父,我來也。」三藏道:「悟空立住,不可驚了聖躬。我問你:假公主之事,端的如何?」行者立於鳷鵲宮外,叉手當胸道:「假公主是個妖邪。初時與他打了半日,他戰不過我,化道清風,徑往天門上跑,是我吆喝天神擋住。他現了相,又與我鬥到十數合,又將身化作金光,敗回正南上一座山上。我急追至山,無處尋覓,恐怕他來此害你,特地回顧也。」國王聽說,扯著唐僧問道:「既然假公主是個妖邪,我真公主在於何處?」行者應聲道:「待我拿住假公主,你那真公主自然來也。」那后妃等聞得此言,都解了恐懼,一個個上前拜告道:「望聖僧救得我真公主來,分了明暗,必當重謝。」行者道:「此間不是我們說話處,請陛下與我師出宮上殿,娘娘等各轉回宮,召我師弟八戒、沙僧來保護師父,我卻好去降妖。一則分了內外,二則免我懸掛。謹當辨明,以表我一場心力。」國王依言,感謝不已。遂與唐僧攜手出宮,徑至殿上。眾后妃各各回宮。一壁廂教備素膳,一壁廂召八戒、沙僧。須臾間,二人早至。行者備言前事,教他兩個用心護持。這大聖縱觔斗雲,飛空而去。那殿前多官,一個個望空禮拜不題。

孫大聖徑至正南方那座山上尋找。原來那妖邪敗了陣,到此山,鑽入窩中,將門兒使石塊擋塞,虛怯怯藏隱不出。行者尋一會,不見動靜,心甚焦惱,捻著訣,念動真言,喚出那山中土地、山神審問。少時,二神至了,叩頭道:「不知,不知,知當遠接,萬望恕罪。」行者道:「我且不打你。我問你:這山叫做甚麼名字?此處有多少妖精?從實說來,饒你罪過。」二神告道:「大聖,此山喚做毛穎山。山中只有三處兔穴,亙古至今,沒甚妖精,乃五環之福地也。大聖要尋妖精,還是西天路上去有。」行者道:「老孫到了西天天竺國,那國王有個公主被個妖精攝去,拋在荒野。他就變做公主模樣,戲哄國王,結綵樓,拋繡毬,欲招駙馬。我保唐僧至其樓下,被他有心打著唐僧,欲為配偶,誘取元陽。是我識破,就於宮中現身捉獲。他就脫了人衣、首飾,使一條短棍,喚名搗藥杵,與我鬥了半日,他就化清風而去。被老孫趕至西天門,又鬥有十數合。他料不能勝,復化金光,逃至此處,如何不見?」

二神聽說,即引行者去那三窟中尋找。始於山腳下窟邊看處,亦有幾個草兔兒,也驚得走了。尋至絕頂上窟中看時,只見兩塊大石頭,將窟門擋住。土地道:「此間必是妖邪,趕急鑽進去也。」行者即使鐵棒,捎開石塊。那妖邪果藏在裡面,呼的一聲,就跳將出來,舉藥杵來打行者掄起鐵棒架住。諕得那山神倒退,土地忙奔。那妖邪口裡囔囔突突的罵著山神、土地道:「誰教你引著他往這裡來找尋?」他支支撐撐的抵著鐵棒,且戰且退,奔至空中。

正在危急之際,卻又天色晚了。這行者愈發狠性,下切手,恨不得一棒打殺。忽聽得九霄碧漢之間有人叫道:「大聖,莫動手,莫動手,棍下留情。」行者回頭看時,原來是太陰星君,後帶著姮娥仙子,降彩雲到於當面。慌得行者收了鐵棒,躬身施禮道:「老太陰往那裡去?老孫失迴避了。」太陰道:「與你對敵的這個妖邪,是我廣寒宮搗玄霜仙藥之玉兔也。他私自偷開玉關金鎖,走出宮來,今經一載。我算他目下有傷命之災,特來救他性命。望大聖看老身饒他罷。」行者喏喏連聲,只道:「不敢,不敢。怪道他會使搗藥杵,原來是個玉兔兒。老太陰不知,他攝藏了天竺國王之公主,卻又假合真形,欲破我聖僧師父之元陽,其情其罪,其實何甘?怎麼便可輕恕饒他?」太陰道:「你亦不知,那國王之公主,也不是凡人,原是蟾宮中之素娥。十八年前,他曾把玉兔兒打了一掌,卻就思凡下界,一靈之光,遂投胎於國王正宮皇后之腹,當時得以降生。這玉兔兒懷那一掌之仇,故於舊年私走出宮,拋素娥於荒野。但只是不該欲配唐僧,此罪真不可逭。幸汝留心,識破真假,卻也未曾傷損你師。萬望看我面上,恕他之罪,我收他去也。」行者笑道:「既有這些因果,老孫也不敢抗違。但只是你收了玉兔兒,恐那國王不信,敢煩太陰君同眾仙妹將玉兔兒拿到那廂,對國王明證明證:一則顯老孫之手段,二來說那素娥下降之因由,然後著那國王取素娥公主之身,以見顯報之意也。」

太陰君信其言,用手指定妖邪,喝道:「那孽畜還不歸正同來。」玉兔兒打個滾,現了原身。真個是:
\begin{quote}
缺唇尖齒,長耳稀鬚。團身一塊毛如玉,展足千山蹄若飛。直鼻垂酥,果賽霜華填粉膩;雙睛紅映,猶欺雪上點胭脂。伏在地,白穰穰一堆素練;伸開腰,白鐸鐸一架銀絲。幾番家吸殘清露瑤天曉,搗藥長生玉杵奇。
\end{quote}

那大聖見了,不勝欣喜,踏雲光,向前引導。那太陰君領著眾姮娥仙子,帶著玉兔兒,徑轉天竺國界。此時正黃昏,看看月上。到城邊,聞得譙樓上擂鼓。那國王與唐僧尚在殿內,八戒、沙僧與多官都在階前,方議退朝,只見正南上一片彩霞,光明如晝。眾擡頭看處,又聞得孫大聖厲聲高叫道:「天竺陛下,請出你那皇后、嬪妃看者:這寶幢下乃月宮太陰星君,兩邊的仙妹是月裡嫦娥。這個玉兔兒卻是你家的假公主,今現真相也。」那國王急召皇后、嬪妃與宮娥、綵女等眾朝天禮拜,他和唐僧及多官亦俱望空拜謝。滿城中各家各戶,也無一人不設香案,叩頭念佛。正此觀看處,豬八戒動了慾心,忍不住,跳在空中,把霓裳仙子抱住道:「姐姐,我與你是舊相識,我和你耍子兒去也。」行者上前,揪著八戒,打了兩掌,罵道:「你這個村潑獃子!此是甚麼去處,敢動淫心?」八戒道:「拉閑散悶耍子而已。」那太陰君令轉仙幢,與眾嫦娥收回玉兔,徑上月宮而去。行者把八戒揪落塵埃。

這國王在殿上謝了行者,又問前因道:「多感神僧大法力捉了假公主。朕之真公主,卻在何處所也?」行者道:「你那真公主也不是凡胎,就是月宮裡素娥仙子。因十八年前,他將玉兔兒打了一掌,就思凡下界,投胎在你正宮腹內,生下身來。那玉兔兒懷恨前仇,所以於舊年間偷開玉關金鎖走下來,把素娥攝拋荒野,他卻變形哄你。這段因果,是太陰君親口才與我說的。今日既去其假者,明日請御駕去尋其真者。」國王聞說,又心意慚惶,止不住腮邊流淚道:「孩兒,我自幼登基,雖城門也不曾出去,卻教我那裡去尋你也?」行者笑道:「不須煩惱,你公主現在給孤布金寺裡裝風,今且各散,到天明我還你個真公主便是。」眾官又拜伏奏道:「我王且心寬,這幾位神僧乃騰雲駕霧之佛,必知未來過去之因由,明日煩神僧同去一尋,便知端的。」國王依言,即請至留春亭擺齋安歇。此時已近二更。正是那:
\begin{quote}
銅壺滴漏月華明,金鐸叮噹風送聲。
杜宇正啼春去半,落花無路近三更。
御園寂寞鞦韆影,碧落空浮銀漢橫。
三市六街無客走,一天星斗夜光晴。
\end{quote}

當夜各寢不題。

這一夜,國王退了妖氣,陡長精神,至五更三點,復出臨朝。朝畢,命請唐僧四眾,議尋公主。長老隨至,朝上行禮。大聖三人,一同打個問訊。國王欠身道:「昨所云公主孩兒,敢煩神僧為一尋救。」長老道:「貧僧前日自東來,行至天晚,見一座給孤布金寺,特進求宿,幸那寺僧相待。當晚齋罷,步月閑行,行至布金舊園,觀看基址,忽聞悲聲入耳,詢問其由。本寺一老僧,年已百歲之外,他屏退左右,細細的對我說了一遍道:『悲聲者,乃舊年春深時,那老僧正明性月,忽然一陣風生,見一女子擲之於地,那僧問之,那女子道:「我是天竺國國王公主,因為夜間玩月觀花,被風刮至於此。」』那老僧多知人禮,即將公主鎖在一間僻靜房中。惟恐本寺頑僧污染,只說是妖精,被他鎖住。公主識得此意,日間胡言亂語,討些茶飯吃了;夜深無人處,思量父母悲啼。那老僧也曾來國打聽幾番,見公主在宮無恙,所以不敢聲言舉奏。因見我徒弟有些神通,那老僧千叮萬囑,教貧僧到此查訪。不期他原是蟾宮玉兔為妖,假合真形,變作公主模樣,他卻又有心要破我元陽。幸虧我徒弟施威顯法,認出真假。今已被太陰星收去。賢公主見在布金寺裝風也。」

國王見說此詳細,放聲大哭。早驚動三宮六院,都來問及前因,無一人不痛哭者。良久,國王又問:「布金寺離城多遠?」三藏道:「只有六十里路。」國王遂傳旨:「著東西二宮守殿,掌朝太師衛國。朕同正宮皇后帥多官、四神僧,去寺取公主也。」當時擺駕,一行出朝。

你看那行者就跳在空中,把腰一扭,先到了寺裡。眾僧慌忙跪接道:「老爺去時,與眾步行,今日何從天上下來?」行者笑道:「你那老師在於何處?快叫他出來,排設香案接駕,天竺國王、皇后、多官與我師都來了。」眾僧不解其意,即請出那老僧。老僧見了行者,倒身下拜道:「老爺,公主之事如何?」行者把那假公主拋繡毬,欲配唐僧,並趕捉賭鬥,與太陰星收去玉兔之言,備陳了一遍。那老僧又磕頭拜謝。行者攙起道:「且莫拜,且莫拜。快安排接駕。」眾僧才知後房裡鎖得是個女子,一個個驚驚喜喜,便都設了香案,擺列山門之外,穿了袈裟,撞起鐘鼓等候。

不多時,聖駕早到。果然是:
\begin{quote}
繽紛瑞靄滿天香,一座荒山倏被祥。
虹流千載清河海,電繞長春賽禹湯。
草木沾恩添秀色,野花得潤有餘芳。
古來長者留遺跡,今喜明君降寶堂。
\end{quote}

國王到於山門之外,只見那眾僧齊齊整整,俯伏接拜;又見孫行者立在中間。國王道:「神僧何先到此?」行者笑道:「老孫把腰略扭一扭兒,就到了。你們怎麼就走這半日?」隨後唐僧等俱到。長老引駕,到於後面房邊,那公主還裝風胡說。老僧跪指道:「此房內就是舊年風吹來的公主娘娘。」國王即令開門。隨即打開鐵鎖,開了門。國王與皇后見了公主,認得形容,不顧穢污,近前一把摟抱道:「我的受苦的兒啊!你怎麼遭這等折磨,在此受罪?」真是父母子女相逢,比他人不同,三人抱頭大哭。哭了一會,敘畢離情,即令取香湯,教公主沐浴更衣,上輦回國。

行者又對國王拱手道:「老孫還有一事奉上。」國王答禮道:「神僧有事吩咐,朕即從之。」行者道:「他這山,名為百腳山。近來說有蜈蚣成精,黑夜傷人,往來行旅,甚為不便。我思蜈蚣惟雞可以降伏,可選絕大雄雞千隻,撒放山中,除此毒蟲。就將此山名改換改換,賜文一道敕封,就當謝此僧供養公主之恩也。」國王甚喜,領諾。隨差官進城取雞;又改山名為寶華山。仍著工部辦料重修,賜與封號,喚做「敕建寶華山給孤布金寺」;把那老僧封為「報國僧官」,永遠世襲,賜俸三十六石。僧眾謝了恩,送駕回朝。

公主入宮,各各相見。安排筵宴,與公主釋悶賀喜。后妃母子,復聚首團圞。國王君臣亦共喜,飲宴一宵不題。

次早,國王傳旨,召丹青圖下聖僧四眾喜容,供養在華夷樓上。又請公主新妝重整,出殿謝唐僧四眾救苦之恩。謝畢,唐僧辭王西去。那國王那裡肯放,大設佳宴,一連吃了五六日,著實好了獃子,盡力放開肚量受用。

國王見他們拜佛心重,苦留不住,遂取金銀二百錠、寶貝各一盤奉謝。師徒們一毫不受。教擺鑾駕,請老師父登輦,差官遠送。那后妃并臣民人等俱各叩謝不盡。及至前途,又見眾僧叩送,俱不忍相別。行者見送者不肯回去,無已,捻訣,往巽地上吹口仙氣,一陣暗風,把送的人都迷了眼目,方才得脫身而去。這正是:
\begin{quote}
沐淨恩波歸了性,出離金海悟真空。
\end{quote}

畢竟不知前路如何,且聽下回分解。
