
\chapter{護法設莊留大聖 須彌靈吉定風魔}

卻說那五十個敗殘的小妖拿著些破旗、破鼓,撞入洞裡,報道:「大王,虎先鋒戰不過那毛臉和尚,被他趕下東山坡去了。」老妖聞說,十分煩惱。正低頭不語,默思計策,又有把前門的小妖道:「大王,虎先鋒被那毛臉和尚打殺了,拖在門口罵戰哩。」那老妖聞言,愈加煩惱道:「這廝卻也無知。我倒不曾吃他師父,他轉打殺我家先鋒,可恨!可恨!」叫:「取披掛來。我也只聞得講甚麼孫行者,等我出去,看是個甚麼九頭八尾的和尚,拿他進來,與我虎先鋒對命。」眾小妖急急擡出披掛。老妖結束齊整,綽一杆三股鋼叉,帥群妖跳出本洞。

那大聖停立門外,見那妖走將出來,著實驍勇。看他怎生打扮,但見那:
\begin{quote}
金盔晃日,金甲凝光。盔上纓飄山雉尾,羅袍罩甲淡鵝黃。勒甲絛盤龍耀彩,護心鏡繞眼輝煌。鹿皮靴,槐花染色;錦圍裙,柳葉絨妝。手持三股鋼叉利,不亞當年顯聖郎。
\end{quote}

那老妖出得門來,厲聲高叫道:「那個是孫行者?」這行者腳屣著虎怪的皮囊,手執著如意的鐵棒,答道:「你孫外公在此。送出我師父來。」那怪仔細觀看,見行者身軀鄙猥,面容羸瘦,不滿四尺。笑道:「可憐,可憐。我只道是怎麼樣扳翻不倒的好漢,原來是這般一個骷髏的病鬼。」行者笑道:「你這個兒子,忒沒眼色。你外公雖是小小的,你若肯照頭打一叉柄,就長六尺。」那怪道:「你硬著頭,吃吾一柄。」大聖公然不懼。那怪果打一下來。他把腰躬一躬,足長了六尺,有一丈長短。慌得那妖把鋼叉按住,喝道:「孫行者,你怎麼把這護身的變化法兒,拿來我門前使出?莫弄虛頭,走上來,我與你見見手段。」行者笑道:「兒子啊,常言道:『留情不舉手,舉手不留情。』你外公手兒重重的,只怕你捱不起這一棒。」那怪那容分說,撚轉鋼叉,望行者當胸就刺;這大聖正是會家不忙,忙家不會,理開鐵棒,使一個「烏龍掠地勢」,撥開鋼叉,又照頭便打。他二人在那黃風洞口,這一場好殺:
\begin{quote}
妖王發怒,大聖施威。妖王發怒,要拿行者抵先鋒;大聖施威,欲捉精靈救長老。叉來棒架,棒去叉迎。一個是鎮山都總帥,一個是護法美猴王。初時還在塵埃戰,後來各起在中央。點鋼叉,尖明銳利;如意棒,身黑箍黃。戳著的魂歸冥府,打著的定見閻王。全憑著手疾眼快,必須要力壯身強。兩家捨死忘生戰,不知那個平安那個傷。
\end{quote}

那老妖與大聖鬥經三十回合,不分勝敗。這行者要見功績,使一個「身外身」的手段:把毫毛揪下一把,用口嚼得粉碎,望上一噴,叫聲:「變!」變有百十個行者,都是一樣打扮,各執一根鐵棒,把那怪圍在空中。那怪害怕,也使一般本事:急回頭,望著巽地上,把口張了三張,呼的一口氣吹將出去,忽然間,一陣黃風,從空刮起。好風,真個利害:
\begin{quote}
冷冷颼颼天地變,無影無形黃沙旋。
穿林折嶺倒松梅,播土揚塵崩嶺坫。
黃河浪潑徹底渾,湘江水湧翻波轉。
碧天振動斗牛宮,爭些刮倒森羅殿。
五百羅漢鬧喧天,八大金剛齊嚷亂。
文殊走了青毛獅,普賢白象難尋見。
真武龜蛇失了群,梓橦騾子飄其韂。
行商喊叫告蒼天,梢公拜許諸般願。
煙波性命浪中流,名利殘生隨水辦。
仙山洞府黑攸攸,海島蓬萊昏暗暗。
老君難顧煉丹爐,壽星收了龍鬚扇。
王母正去赴蟠桃,一風吹亂裙腰釧。
二郎迷失灌州城,哪吒難取匣中劍。
天王不見手心塔,魯班吊了金頭鑽。
雷音寶闕倒三層,趙州石橋崩兩斷。
一輪紅日蕩無光,滿天星斗皆昏亂。
南山鳥往北山飛,東湖水向西湖漫。
雌雄拆對不相呼,子母分離難叫喚。
龍王遍海找夜叉,雷公到處尋閃電。
十代閻王覓判官,地府牛頭追馬面。
這風吹倒普陀山,捲起觀音經一卷。
白蓮花卸海邊飛,吹倒菩薩十二院。
盤古至今曾見風,不似這風來不善。
唿喇喇,乾坤險不炸崩開,萬里江山都是顫。
\end{quote}

那妖怪使出這陣狂風,就把孫大聖毫毛變的小行者刮得在那半空中卻似紡車兒一般亂轉,莫想掄得棒,如何攏得身?慌得行者將毫毛一抖,收上身來。獨自個舉著鐵棒,上前來打。又被那怪劈臉噴了一口黃風,把兩隻火眼金睛刮得緊緊閉合,莫能睜開。因此難使鐵棒,遂敗下陣來。那妖收風回洞不題。

卻說豬八戒見那黃風大作,天地無光,牽著馬,守著擔,伏在山凹之間,也不敢睜眼,不敢擡頭,口裡不住的念佛許願;又不知行者勝負何如,師父死活何如。正在那疑思之時,卻早風定天晴。忽擡頭往那洞門前看處,卻也不見兵戈,不聞鑼鼓。獃子又不敢上他門,又沒人看守馬匹、行李,果是進退兩難,愴惶不已。憂慮間,只聽得孫大聖從西邊吆喝而來,他才欠身迎著道:「哥哥,好大風啊!你從那裡走來?」行者擺手道:「利害,利害!我老孫自為人,不曾見這大風。那老妖使一柄三股鋼叉,來與老孫交戰。戰到有三十餘合,是老孫使一個『身外身』的本事。把他圍打,他甚著急,故弄出這陣風來。果是兇惡,刮得我站立不住,收了本事,冒風而逃。——哏,好風!哏,好風!老孫也會呼風,也會喚雨,不曾似這個妖精的風惡。」八戒道:「師兄,那妖精的武藝如何?」行者道:「也看得過,叉法兒倒也齊整,與老孫也戰個手平。卻只是風惡了,難得贏他。」八戒道:「似這般怎生救得師父?」行者道:「救師父且等再處。不知這裡可有眼科先生,且教他把我眼醫治醫治。」八戒道:「你眼怎的來?」行者道:「我被那怪一口風噴將來,吹得我眼珠酸痛,這會子冷淚常流。」八戒道:「哥啊,這半山中,天色又晚,且莫說要甚麼眼科,連宿處也沒有了。」行者道:「要宿處不難,我料著那妖精還不敢傷我師父,我們且找上大路,尋個人家住下,過此一宵,明日天光,再來降妖罷。」八戒道:「正是,正是。」

他卻牽了馬,挑了擔,出山凹,行上路口。此時漸漸黃昏,只聽得路南山坡下有犬吠之聲。二人停身觀看,乃是一家莊院,影影的有燈火光明。他兩個也不管有路無路,漫草而行,直至那家門首。但見:
\begin{quote}
紫芝翳翳,白石蒼蒼。紫芝翳翳多青草,白石蒼蒼半綠苔。數點小螢光灼灼,一林野樹密排排。香蘭馥郁,嫩竹新栽。清泉流曲澗,古柏倚深崖。地僻更無遊客到,門前惟有野花開。
\end{quote}

他兩個不敢擅入,只得叫一聲:「開門,開門!」那裡有一老者,帶幾個年幼的農夫,叉鈀掃帚齊來,問道:「甚麼人?甚麼人?」行者躬身道:「我們是東土大唐聖僧的徒弟。因往西方拜佛求經,路過此山,被黃風大王拿了我師父去了,我們還未曾救得。天色已晚,特來府上告借一宵,萬望方便方便。」那老者答禮道:「失迎,失迎。此間乃雲多人少之處,卻才聞得叫門,恐怕是妖狐、老虎及山中強盜等類,故此小介愚頑,多有衝撞,不知是二位長老。請進,請進。」

他兄弟們牽馬挑擔而入,徑至裡邊,拴馬歇擔,與莊老拜見敘坐。又有蒼頭獻茶。茶罷,捧出幾碗胡麻飯。飯畢,命設鋪就寢。行者道:「不睡還可,敢問善人,貴地可有賣眼藥的?」老者道:「是那位長老害眼?」行者道:「不瞞你老人家說,我們出家人自來無病,從不曉得害眼。」老人道:「既不害眼,如何討藥?」行者道:「我們今日在黃風洞口救我師父,不期被那怪將一口風噴來,吹得我眼珠酸痛,今有些眼淚汪汪,故此要尋眼藥。」那老者道:「善哉,善哉!你這個長老,小小的年紀,怎麼說謊?那黃風大王,風最利害。他那風,比不得甚麼春秋風、松竹風與那東西南北風。」八戒道:「想必是夾腦風、羊耳風、大麻風、偏正頭風?」長者道:「不是,不是。他叫做三昧神風。」行者道:「怎見得?」老者道:「那風能吹天地暗,善刮鬼神愁,裂石崩崖惡,吹人命即休。你們若遇著他那風吹了時,還想得活哩?只除是神仙,方可得無事。」行者道:「果然,果然。我們雖不是神仙,神仙還是我的晚輩。這條命急切難休,卻只是吹得我眼珠酸痛。」那老者道:「既如此說,也是個有來頭的人。我這敝處卻無賣眼藥的。老漢也有些迎風冷淚,曾遇異人,傳了一方,名喚三花九子膏,能治一切風眼。」行者聞言,低頭唱喏道:「願求些兒,點試點試。」那老者應承,即走進去,取出一個瑪瑙石的小罐兒來,拔開塞口,用玉簪兒蘸出少許,與行者點上,教他不得睜開,寧心睡覺,明早就好。點畢,收了石罐,徑領小介們退於裡面。

八戒解包袱,展開鋪蓋,請行者安置。行者閉著眼亂摸。八戒笑道:「先生,你的明杖兒呢?」行者道:「你這個饢糟的獃子,你照顧我做瞎子哩。」那獃子啞啞的暗笑而睡。行者坐在鋪上,轉運神功,直到三更後方才睡下。

不覺又是五更將曉。行者抹抹臉,睜開眼道:「果然好藥,比常更有百分光明。」卻轉頭後邊望望,呀!那裡得甚房舍窗門,但只見些老槐高柳,兄弟們都睡在那綠莎茵上。那八戒醒來道:「哥哥,你嚷怎的?」行者道:「你睜開眼睛看看。」獃子忽擡頭,見沒了人家,慌得一轂轆爬將起來道:「我的馬哩?」行者道:「樹上拴的不是?」「行李呢?」行者道:「你頭邊放的不是?」八戒道:「這家子也憊𪬯,他搬了,怎麼就不叫我們一聲?通得老豬知道,也好與你送些茶果。想是躲門戶的,恐怕里長曉得,卻就連夜搬了。——噫!我們也忒睡得死,怎麼他家拆房子,響也不聽見響響?」行者吸吸的笑道:「獃子,不要亂嚷。你看那樹上是個甚麼紙帖兒?」八戒走上前,用手揭了,原來上面四句頌子云:
\begin{quote}
莊居非是俗人居,護法伽藍點化廬。
妙藥與君醫眼痛,盡心降怪莫躊躇。
\end{quote}

行者道:「這夥強神,自換了龍馬,一向不曾點他,他倒又來弄虛頭。」八戒道:「哥哥莫扯架子,他怎麼伏你點札?」行者道:「兄弟,你還不知哩。這護教伽藍、六丁六甲、五方揭諦、四值功曹奉菩薩的法旨,暗保我師父者。自那日報了名,只為這一向有了你,再不曾用他們,故不曾點札罷了。」八戒道:「哥哥,他既奉法旨暗保師父,所以不能現身明顯,故此點化仙莊。你莫怪他,昨日也虧他與你點眼,又虧他管了我們一頓齋飯,亦可謂盡心矣。你莫怪他,我們且去救師父來。」行者道:「兄弟說得是。此處到那黃風洞口不遠,你且莫動身,只在林子裡看馬守擔。等老孫去洞裡打聽打聽,看師父下落如何,再與他爭戰。」八戒道:「正是這等,討一個死活的實信。假若師父死了,各人好尋頭幹事;若是未死,我們好竭力盡心。」行者道:「莫亂談,我去也。」

他將身一縱,徑到他門首,門尚關著睡覺。行者不叫門,且不驚動妖怪,捻著訣,念個咒語,搖身一變,變做一個花腳蚊蟲,真個小巧。有詩為證。詩曰:
\begin{quote}
擾擾微形利喙,嚶嚶聲細如雷。
蘭房紗帳善通隨,正愛炎天暖氣。
只怕薰煙撲扇,偏憐燈火光輝。
輕輕小小忒鑽疾,飛入妖精洞裡。
\end{quote}

只見那把門的小妖正打鼾睡,行者往他臉上叮了一口,那小妖翻身醒了,道:「我爺啞!好大蚊子,一口就叮了一個大疙疸。」忽睜眼道:「天亮了。」又聽得支的一聲,二門開了。行者嚶嚶的飛將進去,只見那老妖吩咐各門上謹慎,一壁廂收拾兵器:「只怕昨日那陣風不曾刮死孫行者,他今日必定還來,來時定教他一命休矣。」

行者聽說,又飛過那廳堂,徑來後面,但見一層門關得甚緊。行者漫門縫兒鑽將進去,原來是個大空園子,那壁廂定風樁上繩纏索綁著唐僧哩。那師父紛紛淚落,心心只念著悟空、悟能,不知都在何處。行者停翅,叮在他光頭上,叫聲:「師父。」那長老認得他的聲音,道:「悟空啊,想殺我也。你在那裡叫我哩?」行者道:「師父,我在你頭上哩。你莫要心焦,少得煩惱。我們務必拿住妖精,方才救得你的性命。」唐僧道:「徒弟啊,幾時才拿得妖精麼?」行者道:「拿你的那虎怪,已被八戒打死了。只是老妖的風勢利害,料著只在今日,管取拿他。你放心莫哭,我去啞。」

說聲去,嚶嚶的飛到前面。只見那老妖坐在上面,正點札各路頭目。又見那洞前有一個小妖精,把個令字旗磨一磨,撞上廳來報道:「大王,小的巡山,才出門,見一個長嘴大耳朵的和尚坐在林裡,若不是我跑得快些,幾乎被他捉住。卻不見昨日那個毛臉和尚。」老妖道:「孫行者不在,想必是風吹死也;再不便去那裡求救兵去了。」眾妖道:「大王,若果吹殺了他,是我們的造化;只恐吹不死他,他去請些神兵來,卻怎生是好?」老妖道:「怕那甚麼神兵?若還定得我的風勢,只除了靈吉菩薩來是,其餘何足懼也?」

行者在屋梁上,只聽得他這一句言語,不勝歡喜。即抽身飛出,現本相,來至林中,叫聲:「兄弟。」八戒道:「哥,你往那裡去來?剛才一個打令字旗的妖精,被我趕了去也。」行者笑道:「虧你,虧你。老孫變做蚊蟲兒,進他洞去探看師父,原來師父被他綁在定風樁上哭哩。是老孫吩咐,教他莫哭。又飛在屋梁上聽了一聽,只見那拿令字旗的喘噓噓的走進去報道:只是被你趕他,卻不見我。老妖亂猜亂說,說老孫是風吹殺了,又說是請神兵去了。他卻自家供出一個人來,甚妙,甚妙。」八戒道:「他供的是誰?」行者道:「他說怕甚麼神兵,那個能定他的風勢,只除是靈吉菩薩來是。——但不知靈吉住在何處?」

正商議處,只見大路傍走出一個老公公來。你看他怎生模樣:
\begin{quote}
身健不扶拐杖,冰髯雪鬢蓬蓬。
金花耀眼意朦朧,瘦骨衰筋強硬。
屈背低頭緩步,龐眉赤臉如童。
看他容貌是人稱,卻似壽星出洞。
\end{quote}

八戒望見大喜道:「師兄,常言道:『要知山下路,須問去來人。』你上前問他一聲,何如?」真個大聖藏了鐵棒,放下衣襟,上前叫道:「老公公,問訊了。」那老者半答不答的還了個禮道:「你是那裡和尚?這曠野處,有何事幹?」行者道:「我們是取經的聖僧。昨日在此失了師父,特來動問公公一聲:靈吉菩薩在那裡住?」老者道:「靈吉在直南上,到那裡還有三千里路。有一山,呼名小須彌山,山中有個道場,乃是菩薩講經禪院。汝等是取他的經去了?」行者道:「不是取他的經,我有一事煩他,不知從那條路去。」老者用手向南指道:「這條羊腸路就是了。」哄得那孫大聖回頭看路,那公公化作清風,寂然不見。只是路傍留下一張簡帖,上有四句頌子云:
\begin{quote}
上覆齊天大聖聽:老人乃是李長庚。
須彌山有飛龍杖,靈吉當年受佛兵。
\end{quote}

行者執了帖兒,轉身下路。八戒道:「哥啊,我們連日造化低了,這兩日白日裡見鬼。那個化風去的老兒是誰?」行者把帖兒遞與八戒,念了一遍道:「李長庚是那個?」行者道:「是西方太白金星的名號。」八戒慌得望空下拜道:「恩人,恩人,老豬若不虧金星奏准玉帝啊,性命也不知化作甚的了。」行者道:「兄弟,你卻也知感恩。但莫要出頭,只藏在這樹林深處,仔細看守行李、馬匹。等老孫尋須彌山,請菩薩去耶。」八戒道:「曉得,曉得,你只管快快前去。老豬學得個烏龜法,得縮頭時且縮頭。」

孫大聖跳在空中,縱觔斗雲,徑往直南上去,果然速快,他點頭經過三千里,扭腰八百有餘程。須臾,見一座高山,半中間有祥雲出現,瑞藹紛紛。山凹裡果有一座禪院,只聽得鐘磬悠揚,又見那香煙縹緲。大聖直至門前,見一道人,項掛數珠,口中念佛。行者道:「道人作揖。」那道人躬身答禮道:「那裡來的老爺?」行者道:「這可是靈吉菩薩講經處麼?」道人道:「此間正是,有何話說?」行者道:「累煩你老人家與我傳答傳答:我是東土大唐駕下御弟三藏法師的徒弟齊天大聖孫悟空行者,今有一事,要見菩薩。」道人笑道:「老爺字多話多,我不能全記。」行者道:「你只說是唐僧徒弟孫悟空來了。」

道人依言,上講堂傳報。那菩薩即穿袈裟,添香迎接。這大聖才舉步入門,往裡觀看,只見那:
\begin{quote}
滿堂錦繡,一屋威嚴。眾門人齊誦《法華經》,老班首輕敲金鑄磬。佛前供養,盡是仙果仙花;案上安排,皆是素殽素品。輝煌寶燭,條條金燄射虹霓;馥郁真香,道道玉煙飛彩霧。正是那講罷心閑方入定,白雲片片繞松梢。靜收慧劍魔頭絕,般若波羅善會高。
\end{quote}

那菩薩整衣出迓,行者登堂,坐了客位,隨命看茶。行者道:「茶不勞賜,但我師父在黃風山有難,特請菩薩施大法力降怪救師。」菩薩道:「我受了如來法令,在此鎮押黃風怪。如來賜了我一顆定風丹、一柄飛龍寶杖。當時被我拿住,饒了他的性命,放他去隱性歸山,不許傷生造孽。不知他今日欲害令師,有違教令,我之罪也。」那菩薩欲留行者,治齋相敘,行者懇辭,隨取了飛龍杖,與大聖一齊駕雲。

不多時,至黃風山上。菩薩道:「大聖,這妖怪有些怕我,我只在雲端內住定,你下去與他索戰,誘他出來,我好施法力。」行者依言,按落雲頭,不容分說,掣鐵棒把他洞門打破。叫道:「妖怪,還我師父來也!」慌得那把門小妖急忙傳報。那怪道:「這潑猴著實無禮,再不伏善,反打破我門。這一出去,使陣神風,定要把他吹死。」仍前披掛,手綽鋼叉,又走出門來。見了行者,更不打話,撚叉當胸就刺;大聖側身躲過。舉棒對面相還戰不數合,那怪吊回頭,望巽地上,才待要張口呼風,只見那半空裡,靈吉菩薩將飛龍寶杖丟將下來,不知念了些甚麼咒語,卻是一條八爪金龍,撥喇的掄開兩爪,一把抓住妖精,提著頭,兩三捽,捽在山石崖邊,現了本相,卻是一個黃毛貂鼠。

行者趕上,舉棒就打,被菩薩攔住道:「大聖,莫傷他命我還要帶他去見如來。」又對行者道:「他本是靈山腳下的得道老鼠,因為偷了琉璃盞內的清油,燈火昏暗,恐怕金剛拿他,故此走了,卻在此處成精作怪。如來照見了他,不該死罪,故著我轄押,但他傷生造孽,拿上靈山。今又衝撞大聖,陷害唐僧,我拿他去見如來,明正其罪,才算這場功績哩。」行者聞言,卻謝了菩薩。菩薩西歸不題。

卻說豬八戒在那林內,正思量行者,只聽得山坂下叫聲:「悟能兄弟,牽馬挑擔來耶。」那獃子認得是行者聲音,急收拾跑出林外,見了行者道:「哥哥,怎的幹事來?」行者道:「請靈吉菩薩,使一條飛龍杖,拿住妖精,原來是個黃毛貂鼠成精,被他帶去靈山見如來去了。我和你洞裡去救師父。」那獃子才歡歡喜喜。

二人撞入裡面,把那一窩狡兔、妖狐、香獐、角鹿,一頓釘鈀、鐵棒,盡情打死,卻往後園拜救師父。師父出得門來,問道:「你兩人怎生捉得妖精?如何方救得我?」行者將那請靈吉降妖的事情,陳了一遍。師父謝之不盡。他兄弟們把洞中素物,安排些茶飯吃了,方才出門,找大路向西而去。

畢竟不知向後如何,且聽下回分解。
