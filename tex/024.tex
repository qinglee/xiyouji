
\chapter{萬壽山大仙留故友 五莊觀行者竊人參}

卻說那三人穿林入裡,只見那獃子繃在樹上,聲聲叫喊,痛苦難禁。行者上前笑道:「好女婿呀,這早晚還不起來謝親,又不到師父處報喜,還在這裡賣解兒耍子哩。咄!你娘呢?你老婆呢?好個繃巴吊拷的女婿呀。」那獃子見他來搶白著羞,咬著牙,忍著疼,不敢叫喊。沙僧見了,老大不忍,放下行李,上前解了繩索救下。獃子對他們只是磕頭禮拜,其實羞恥難當。有《西江月》為證:
\begin{quote}
色乃傷身之劍,貪之必定遭殃。
佳人二八好容妝。更比夜叉兇壯。
只有一個原本,再無微利添囊。
好將資本謹收藏。堅守休教放蕩。
\end{quote}

那八戒撮土焚香,望空禮拜。行者道:「你可認得那些菩薩麼?」八戒道:「我已此暈倒昏迷,眼花撩亂,那認得是誰?」行者把那簡帖兒遞與八戒。八戒見了是頌子,更加慚愧。沙僧笑道:「二哥有這般好處哩,感得四位菩薩來與你做親。」八戒道:「兄弟再莫題起,不當人子了。從今後,再也不敢妄為。就是累折骨頭,也只是摩肩壓擔,隨師父西域去也。」三藏道:「既如此說才是。」

行者遂領師父上了大路。行夠多時,忽見有高山擋路,三藏勒馬停鞭道:「徒弟,前面一山,必須仔細,恐有妖魔作耗,侵害吾黨。」行者道:「馬前但有吾等三人,怕甚妖魔?」因此,長老安心前進。只見那座山,真是好山:
\begin{quote}
高山峻極,大勢崢嶸。根接崑崙脈,頂摩霄漢中。白鶴每來棲檜柏,玄猿時復掛藤蘿。日映晴林,疊疊千條紅霧繞;風生陰壑,飄飄萬道彩雲飛。幽鳥亂啼青竹裡,錦雞齊鬥野花間。只見那千年峰、五福峰、芙蓉峰,巍巍凜凜放毫光;萬歲石、虎牙石、三天石,突突磷磷生瑞氣。崖前草秀,嶺上梅香。荊棘密森森,芝蘭清淡淡。深林鷹鳳聚千禽,古洞麒麟轄萬獸。澗水有情,曲曲灣灣多遶顧;峰巒不斷,重重疊疊自週迴。又見那綠的槐、斑的竹、青的松,依依千載鬥穠華;白的李、紅的桃、翠的柳,灼灼三春爭艷麗。龍吟虎嘯,鶴舞猿啼。麋鹿從花出,青鸞對日鳴。乃是仙山真福地,蓬萊閬苑只如然。又見些花開花謝山頭景,雲去雲來嶺上峰。
\end{quote}

三藏在馬上歡喜道:「徒弟,我一向西來,經歷許多山水,都是那嵯峨險峻之處,更不似此山好景,果然的幽趣非常。若是相近雷音不遠路,我們好整肅端嚴見世尊。」行者笑道:「早哩,早哩,正好不得到哩。」沙僧道:「師兄,我們到雷音有多少遠?」行者道:「十萬八千里,十停中還不曾走了一停哩。」八戒道:「哥啊,要走幾年才得到?」行者道:「這些路,若論二位賢弟,便十來日也可到;若論我走,一日也好走五十遭,還見日色;若論師父走,莫想,莫想。」唐僧道:「悟空,你說得幾時方可到?」行者道:「你自小時走到老,老了再小,老小千番也還難;只要你見性志誠,念念回首處,即是靈山。」沙僧道:「師兄,此間雖不是雷音,觀此景致,必有個好人居止。」行者道:「此言卻當。這裡決無邪祟,一定是個聖僧、仙輩之鄉,我們遊玩慢行。」不題。

卻說這座山名喚萬壽山。山中有一座觀,名喚五莊觀。觀裡有一尊仙,道號鎮元子,混名與世同君。那觀裡出一般異寶,乃是混沌初分,鴻濛始判,天地未開之際,產成這顆靈根。蓋天下四大部洲,惟西牛賀洲五莊觀出此,喚名草還丹,又名人參果。三千年一開花,三千年一結果,再三千年才得熟,短頭一萬年方得吃。似這萬年,只結得三十個果子。果子的模樣,就如三朝未滿的小孩相似,四肢俱全,五官咸備。人若有緣,得那果子聞了一聞,就活三百六十歲;吃一個,就活四萬七千年。

當日鎮元大仙得元始天尊的簡帖,邀他到上清天彌羅宮中聽講「混元道果」。大仙門下出的散仙,也不計其數,見如今還有四十八個徒弟,都是得道的全真。當日帶領四十六個上界去聽講,留下兩個絕小的看家:一個喚做清風,一個喚做明月。清風只有一千三百二十歲,明月才交一千二百歲。鎮元子吩咐二童道:「不可違了大天尊的簡帖,要往彌羅宮聽講,你兩個在家仔細。不日有一個故人從此經過,卻莫怠慢了他。可將我人參果打兩個與他吃,權表舊日之情。」二童道:「師父的故人是誰?望說與弟子,好接待。」大仙道:「他是東土大唐駕下的聖僧,道號三藏,今往西天拜佛求經的和尚。」二童笑道:「孔子云:『道不同,不相為謀。』我等是太乙玄門,怎麼與那和尚做甚相識?」大仙道:「你那裡得知。那和尚乃金蟬子轉生,西方聖老如來佛第二個徒弟。五百年前,我與他在蘭盆會上相識。他曾親手傳茶,佛子敬我,故此是為故人也。」

二仙童聞言,謹遵師命。那大仙臨行,又叮嚀囑咐道:「我那果子有數,只許與他兩個,不得多費。」清風道:「開園時,大眾共吃了兩個,還有二十八個在樹,不敢多費。」大仙道:「唐三藏雖是故人,須要防備他手下人囉唣,不可驚動他知。」二童領命訖,那大仙承眾徒弟飛昇,竟朝天界。

卻說唐僧四眾在山遊玩,忽擡頭,見那松篁一簇,樓閣數層。唐僧道:「悟空,你看那裡是甚麼去處?」行者看了道:「那所在不是觀宇,定是寺院。我們走動些,到那廂方知端的。」不一時,來於門首觀看,見那:
\begin{quote}
松坡冷淡,竹徑清幽。往來白鶴送浮雲,上下猿猴時獻果。那門前池寬樹影長,石裂苔花破。宮殿森羅紫極高,樓臺縹緲丹霞墮。真個是福地靈區,蓬萊雲洞。清虛人事少,寂靜道心生。青鳥每傳王母信,紫鸞常寄老君經。看不盡那巍巍道德之風,果然漠漠神仙之宅。
\end{quote}

三藏離鞍下馬,又見那山門左邊有一通碑,碑上有十個大字,乃是「萬壽山福地,五莊觀洞天」。長老道:「徒弟,真個是一座觀宇。」沙僧道:「師父,觀此景鮮明,觀裡必有好人居住。我們進去看看,若行滿東回,此間也是一景。」行者道:「說得好。」遂都一齊進去,又見那二門上有一對春聯:「長生不老神仙府;與天同壽道人家。」行者笑道:「這道士說大話諕人。我老孫五百年前大鬧天宮時,在那太上老君門首,也不曾見有此話說。」八戒道:「且莫管他,進去,進去,或者這道士有些德行,未可知也。」

及至二層門裡,只見那裡面急急忙忙,走出兩個小童兒來。看他怎生打扮:
\begin{quote}
骨清神爽容顏麗,頂結丫髻短髮鬅。
道服自然襟繞霧,羽衣偏是袖飄風。
環絛緊束龍頭結,芒履輕纏蠶口絨。
丰采異常非俗輩,正是那清風明月二仙童。
\end{quote}

那童子控背躬身,出來迎接道:「老師父,失迎,請坐。」長老歡喜,遂與二童子上了正殿觀看。原來是向南的五間大殿,都是上明下暗的雕花格子。那仙童推開格子,請唐僧入殿處,只見那壁中間掛著五彩裝成的「天地」二大字,設一張朱紅雕漆的香几,几上有一副黃金爐瓶,爐邊有方便整香。

唐僧上前,以左手撚香注爐,三匝禮拜。拜畢,回頭道:「仙童,你五莊觀真是西方仙界。何不供養三清、四帝、羅天諸宰,只將『天地』二字侍奉香火?」童子笑道:「不瞞老師說,這兩個字,上頭的,禮上還當;下邊的,還受不得我們的香火,是家師父諂佞出來的。」三藏道:「何為諂佞?」童子道:「三清是家師的朋友,四帝是家師的故人;九曜是家師的晚輩,元辰是家師的下賓。」那行者聞言,就笑得打跌。八戒道:「哥啊,你笑怎的?」行者道:「只講老孫會搗鬼,原來這道童會綑風。」三藏道:「令師何在?」童子道:「家師元始天尊降簡,請到上清天彌羅宮聽講『混元道果』去了,不在家。」行者聞言,忍不住喝了一聲道:「這個臊道童,人也不認得,你在那個面前搗鬼,扯甚麼空心架子?那彌羅宮有誰是太乙天仙?請你這潑牛蹄子去講甚麼?」

三藏見他發怒,恐怕那童子回言,鬥起禍來,便道:「悟空,且休爭競,我們既進來就出去,顯得沒了方情。常言道:『鷺鷥不吃鷺鷥肉。』他師既是不在,攪亂他做甚?你去山門前放馬,沙僧看守行李,教八戒解包袱,取些米糧,借他鍋灶,做頓飯吃,待臨行,送他幾文柴錢,便罷了。各依執事,讓我在此歇息歇息,飯畢就行。」他三人果各依執事而去。

那明月、清風暗自誇稱不盡道:「好和尚,真個是西方愛聖臨凡,真元不昧。師父命我們接待唐僧,將人參果與他吃,以表故舊之情;又教防著他手下人囉唣。果然那三個嘴臉兇頑,性情粗糙。幸得就把他們調開了;若在邊前,卻不與他人參果見面?」清風道:「兄弟,還不知那和尚可是師父的故人。問他一問看,莫要錯了。」二童子又上前道:「啟問老師可是大唐往西天取經的唐三藏?」長老回禮道:「貧僧就是。仙童為何知我賤名?」童子道:「我師臨行,曾吩咐教弟子遠接。不期車駕來促,有失迎迓。老師請坐,待弟子辦茶來奉。」三藏道:「不敢。」那明月急轉本房,取一杯香茶,獻與長老。茶畢,清風道:「兄弟,不可違了師命,我和你去取果子來。」

二童別了三藏,同到房中,一個拿了金擊子,一個拿了丹盤,又多將綠帕墊著盤底,徑到人參園內。那清風爬上樹去,使金擊子敲果。明月在樹下,以丹盤等接。須臾,敲下兩個果來,接在盤中,徑至前殿奉獻道:「唐師父,我五莊觀土僻山荒,無物可奉,土儀素果二枚,權為解渴。」那長老見了,戰戰兢兢,遠離三尺道:「善哉!善哉!今歲倒也年豐時稔,怎麼這觀裡作荒吃人?這個是三朝未滿的孩童,如何與我解渴?」清風暗道:「這和尚在那口舌場中,是非海裡,弄得眼肉胎凡,不識我仙家異寶。」明月上前道:「老師,此物叫做人參果,吃一個兒不妨。」三藏道:「胡說,胡說。他那父母懷胎,不知受了多少苦楚,方生下來。未及三日,怎麼就把他拿來當果子?」清風道:「實是樹上結的。」長老道:「亂談,亂談。樹上又會結出人來?拿過去,不當人子。」

那兩個童兒見千推萬阻不吃,只得拿著盤子,轉回本房。那果子卻也蹺蹊,久放不得;若放多時,即僵了,不中吃。二人到於房中,一家一個,坐在床邊上,只情吃起。

噫!原來有這般事哩。他那道房,與那廚房緊緊的間壁,這邊悄悄的言語,那邊即便聽見。八戒正在廚房裡做飯,先前聽見說取金擊子,拿丹盤,他已在心;又聽見他說唐僧不認得是人參果,即拿在房裡自吃。口裡忍不住流涎道:「怎得一個兒嘗新?」自家身子又狼犺,不能夠得動,只等行者來,與他計較。他在那鍋門前更無心燒火,不時的伸頭探腦,出來觀看。不多時,見行者牽將馬來,拴在槐樹上,徑往後走。那獃子用手亂招道:「這裡來,這裡來。」行者轉身,到於廚房門首,道:「獃子,你嚷甚的?想是飯不夠吃,且讓老和尚吃飽,我們前邊大人家,再化吃去罷。」八戒道:「你進來,不是飯少。這觀裡有一件寶貝,你可曉得?」行者道:「甚麼寶貝?」八戒笑道:「說與你,你不曾見;拿與你,你不認得。」行者道:「這獃子笑話我老孫。老孫五百年前,因訪仙道時,也曾雲遊在海角天涯,那般兒不曾見?」八戒道:「哥啊,人參果你曾見麼?」行者驚道:「這個真不曾見。但只常聞得人說,人參果乃是草還丹,人吃了極能延壽。如今那裡有得?」八戒道:「他這裡有。那童子拿兩個與師父吃,那老和尚不認得,道是三朝未滿的孩兒,不曾敢吃。那童子老大憊𪬯,師父既不吃,便該讓我們,他就瞞著我們,才自在這隔壁房裡,一家一個,嘓啅嘓啅的吃了出去。就急得我口裡水泱。怎麼得一個兒嘗新?我想你有些溜撒,去他那園子裡偷幾個來嘗嘗,如何?」行者道:「這個容易,老孫去,手到擒來。」急抽身,往前就走,八戒一把扯住道:「哥啊,我聽得他在這房裡說,要拿甚麼金擊子去打哩。須是幹得停當,不可走露風聲。」行者道:「我曉得,我曉得。」

那大聖使一個隱身法,閃進道房看時,原來那兩個道童吃了果子,上殿與唐僧說話,不在房裡。行者四下裡觀看,看有甚麼金擊子,但只見窗櫺上掛著一條赤金,有二尺長短,有指頭粗細;底下是一個蒜疙疸的頭子;上邊有眼,系著一根綠絨繩兒。他道:「想必就是此物叫做金擊子。」他卻取下來,出了道房,徑入後邊去,推開兩扇門,擡頭觀看,呀!卻是一座花園!但見:
\begin{quote}
朱欄寶檻,曲砌峰山。奇花與麗日爭妍,翠竹共青天鬥碧。流杯亭外,一灣綠柳似拖煙;賞月臺前,數簇喬松如潑靛。紅拂拂,錦巢榴;綠依依,繡墩草;青茸茸,碧砂蘭;攸蕩蕩,臨溪水。丹桂映金井梧桐,錦槐傍朱欄玉砌。有或紅或白千葉桃,有或香或黃九秋菊。荼架,映著牡丹亭;木槿臺,相連芍藥圃。看不盡傲霜君子竹,欺雪大夫松。更有那鶴莊鹿宅,方沼圓池;泉流碎玉,地萼堆金。朔風觸綻梅花白,春來點破海棠紅。誠所謂人間第一仙景,西方魁首花叢。
\end{quote}

那行者觀看不盡,又見一層門,推開看處,卻是一座菜園:
\begin{quote}
佈種四時蔬菜,菠芹莙薘薑苔。
筍瓜瓠茭白,蔥蒜芫荽韭薤。
窩蕖童蒿苦藚,葫蘆茄子須栽。
蔓菁蘿蔔羊頭埋,紅莧青菘紫芥。
\end{quote}

行者笑道:「他也是個自種自吃的道士。」

走過菜園,又見一層門。推開看處,呀!只見那正中間有根大樹,真個是青枝馥郁,綠葉陰森,那葉兒卻似芭蕉模樣,直上去有千尺餘高,根下有七八丈圍圓。那行者倚在樹下,往上一看,只見向南的枝上露出一個人參果,真個像孩兒一般。原來尾間上是個扢蒂,看他丁在枝頭,手腳亂動,點頭幌腦,風過處似乎有聲。行者歡喜不盡,暗自誇稱道:「好東西呀!果然罕見,果然罕見!」他倚著樹,颼的一聲,攛將上去。那猴子原來第一會爬樹偷果子。他把金擊子敲了一下,那果子撲的落將下來。他也隨跳下來跟尋,寂然不見;四下裡草中找尋,更無蹤跡。

行者道:「蹺蹊,蹺蹊。想是有腳的會走,就走也跳不出牆去。我知道了,想是花園中土地不許老孫偷他果子,他收了去也。」他就捻著訣,念一口「唵」字咒,拘得那花園土地前來,對行者施禮道:「大聖呼喚小神,有何吩咐?」行者道:「你不知老孫是蓋天下有名的賊頭,我當年偷蟠桃、盜御酒、竊靈丹,也不曾有人敢與我分用。怎麼今日偷他一個果子,你就抽了我的頭分去了?這果子是樹上結的,空中過鳥也該有分,老孫就吃他一個,有何大害?怎麼剛打下來,你就撈了去?」土地道:「大聖錯怪了小神也。這寶貝乃是地仙之物,小神是個鬼仙,怎麼敢拿去?就是聞也無福聞聞。」

行者道:「你既不曾拿去,如何打下來就不見了?」土地道:「大聖只知這寶貝延壽,更不知他的出處哩。」行者道:「有甚出處?」土地道:「這寶貝三千年一開花,三千年一結果,再三千年方得成熟。短頭一萬年,只結得三十個。有緣的,聞一聞,就活三百六十歲;吃一個,就活四萬七千年。卻是只與五行相畏。」行者道:「怎麼與五行相畏?」土地道:「這果子遇金而落,遇木而枯,遇水而化,遇火而焦,遇土而入。敲時必用金器,方得下來。打下來,卻將盤兒用絲帕襯墊方可。若受些木器,就枯了,就吃也不得延壽。吃他須用磁器,清水化開食用。遇火即焦而無用。遇土而入者,大聖方才打落地上,他即鑽下土去了。這個土有四萬七千年,就是鋼鑽鑽他也鑽不動些須,比生鐵也還硬三四分,人若吃了,所以長生。大聖不信時,可把這地下打打兒看。」行者即掣金箍棒築了一下,響一聲,迸起棒來,土上更無痕跡。行者道:「果然,果然。我這棍打石頭如粉碎,撞生鐵也有痕,怎麼這一下打不傷些兒?這等說,我卻錯怪了你了,你回去罷。」那土地即回本廟去訖。

大聖卻有算計:爬上樹,一隻手使擊子,一隻手將錦布直裰的襟兒扯起來做個兜子等住,他卻串枝分葉,敲了三個果,兜在襟中。跳下樹,一直前來,徑到廚房裡去。那八戒笑道:「哥哥,可有麼?」行者道:「這不是?老孫的手到擒來。這個果子,也莫背了沙僧,可叫他一聲。」八戒即招手叫道:「悟淨,你來。」那沙僧搬下行李,跑進廚房道:「哥哥,叫我怎的?」行者放開衣兜道:「兄弟,你看這個是甚的東西?」沙僧見了道:「是人參果。」行者道:「好啊!你倒認得,你曾在那裡吃過的?」沙僧道:「小弟雖不曾吃,但舊時做捲簾大將,扶侍鸞輿赴蟠桃宴,嘗見海外諸仙將此果與王母上壽。見便曾見,卻未曾吃。哥哥,可與我些兒嘗嘗?」行者道:「不消講,兄弟們一家一個。」

他三人將三個果各各受用。那八戒食腸大,口又大,一則是聽見童子吃時,便覺饞蟲拱動,卻才見了果子,拿過來,張開口,轂轆的囫圇吞嚥下肚。卻白著眼胡賴,向行者、沙僧道:「你兩個吃的是甚麼?」沙僧道:「人參果。」八戒道:「甚麼味道?」行者道:「悟淨,不要睬他。——你倒先吃了,又來問誰?」八戒道:「哥哥,吃的忙了些,不像你們細嚼細嚥,嘗出些滋味。我也不知有核無核,就吞下去了。哥啊,為人為徹。已經調動我這饞蟲,再去弄個兒來,老豬細細的吃吃。」行者道:「兄弟,你好不知止足。這個東西,比不得那米食麵食,撞著儘飽。像這一萬年只結得三十個,我們吃他這一個,也是大有緣法,不等小可。罷罷罷,夠了。」他欠起身來,把一個金擊子,瞞窗眼兒,丟進他道房裡,竟不睬他。

那獃子只管絮絮叨叨的唧噥。不期那兩個道童復進房來取茶去獻,只聽得八戒還嚷甚麼「人參果吃得不快活,再得一個兒吃吃才好。」清風聽見,心疑道:「明月,你聽那長嘴和尚講:『人參果還要個吃吃。』師父別時叮嚀,教防他手下人囉唣,莫敢是他偷了我們寶貝麼?」明月回頭道:「哥耶,不好了,不好了,金擊子如何落在地下?我們去園裡看看來。」

他兩個急急忙忙的走去,只見花園開了。清風道:「這門是我關的,如何開了?」又急轉過花園,只見菜園門也開了。忙入人參園裡,倚在樹下,望上查數,顛倒來往,只得二十二個。明月道:「你可會算帳?」清風道:「我會,你說將來。」明月道:「果子原是三十個,師父開園,分吃了兩個,還有二十八個;適才打兩個與唐僧吃,還有二十六個;如今止剩得二十二個,卻不少了四個?不消講,不消講,定是那夥惡人偷了,我們只罵唐僧去來。」

兩個出了園門,徑來殿上,指著唐僧,禿前禿後,穢語污言,不絕口的亂罵;賊頭鼠腦,臭短臊長,沒好氣的胡嚷。唐僧聽不過道:「仙童啊,你鬧的是甚麼?消停些兒,有話慢說不妨,不要胡說散道的。」清風說:「你的耳聾?我是蠻話,你不省得?你偷吃了人參果,怎麼不容我說?」唐僧道:「人參果怎麼模樣?」明月道:「才拿來與你吃,你說像孩童的不是?」唐僧道:「阿彌陀佛!那東西一見,我就心驚膽戰,還敢偷他吃哩?就是害了饞痞,也不敢幹這賊事。不要錯怪了人。」清風道:「你雖不曾吃,還有手下人要偷吃的哩。」三藏道:「這等也說得是,你且莫嚷,等我問他們看。果若是偷了,教他賠你。」明月道:「賠呀!就有錢那裡去買?」三藏道:「縱有錢沒處買啊,常言道:『仁義值千金。』教他賠你個禮,便罷了。也還不知是他不是他哩。」明月道:「怎的不是他?他那裡分不均,還在那裡嚷哩。」三藏叫聲:「徒弟,且都來。」沙僧聽見道:「不好了,決撒了。老師父叫我們,小道童胡廝罵,不是舊話兒走了風,卻是甚的?」行者道:「活羞殺人。這個不過是飲食之類,若說出來,就是我們偷嘴了,只是莫認。」八戒道:「正是,正是,昧了罷。」他三人只得出了廚房,走上殿去。

畢竟不知怎麼與他抵賴,且聽下回分解。
