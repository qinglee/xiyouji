
\chapter{三僧大戰青龍山 四星挾捉犀牛怪}

卻說孫大聖挾同二弟滾著風,駕著雲,向東北艮地上,頃刻至青龍山玄英洞口,按落雲頭,八戒就欲築門。行者道:「且消停,待我進去看看師父生死如何,再好與他爭持。」沙僧道:「這門閉緊,如何得進?」行者道:「我自有法力。」

好大聖,收了棒,捻著訣,念聲咒語,叫:「變!」即變做個火焰蟲兒,真個也疾伶。你看他:
\begin{quote}
展翅星流光燦,古云腐草為螢。
神通變化不可輕。自有徘徊之性。
飛近石門懸看,傍邊瑕縫穿風。
將身一縱到幽庭。打探妖魔動靜。
\end{quote}

他自飛入,只見幾隻牛橫敧直倒,一個個呼吼如雷,盡皆睡熟了。至中廳裡面,全無消息。四下門戶通關,不知那三個妖精睡在何處。才轉過廳房,向後又照,只聞得啼泣之聲,乃是唐僧鎖在後房簷柱上哭哩。行者暗暗聽他哭甚,只見他哭道:
\begin{quote}
「一別長安十數年,登山涉水苦熬煎。
幸來西域逢佳節,喜到金平遇上元。
不識燈中假佛像,皆因命裡有災愆。
賢徒追襲施威武,但願英雄展大權。」
\end{quote}

行者聞言,滿心歡喜,展開翅,飛近師前。唐僧揩淚道:「呀!西方景象不同,此時正月,蟄蟲始振,為何就有螢飛?」行者忍不住,叫聲:「師父,我來了。」唐僧喜道:「悟空,我說正月間怎得螢火?原來是你。」行者即現了本相道:「師父啊,為你不識真假,誤了多少路程,費了多少心力。我一行說不是好人,你就下拜,卻被這怪侮暗燈光,盜取酥合香油,連你都攝將來了。我當吩咐八戒、沙僧回寺看守,我即聞風追至此間,不識地名。幸遇四值功曹傳報,說此山名青龍山玄英洞。我日間與此怪鬥至天晚方回,與師弟輩細道此情,卻就不曾睡,同他兩個來此。我恐夜深不便交戰,又不知師父下落,所以變化進來,打聽打聽。」唐僧喜道:「八戒、沙僧如今在外邊哩?」行者道:「在外邊。方才老孫看時,妖精都睡著。我且解了鎖,搠開門,帶你出去罷。」唐僧點頭稱謝。

行者使個解鎖法,用手一抹,那鎖早自開了。領著師父往前正走,忽聽得妖王在中廳內房裡叫道:「小的們,緊閉門戶,小心火燭。這會怎麼不叫更巡邏,梆鈴都不響了?」原來那夥小妖征戰一日,俱辛辛苦苦睡著,聽見叫喚,卻才醒了,梆鈴響處,有幾個執器械的敲著鑼,從後而走,可可的撞著他師徒兩個。眾小妖一齊喊道:「好和尚啊,扭開鎖往那裡去?」行者不容分說,掣出棒幌一幌,碗來粗細,就打,棒起處,打死兩個。其餘的丟了器械,近中廳,打著門叫:「大王,不好了,不好了,毛臉和尚在家裡打殺人了。」那三怪聽見,一轂轆爬將起來,只教「拿住,拿住。」諕得個唐僧手軟腳軟。行者也不顧師父,一路棒,滾向前來。眾小妖遮架不住,被他放倒三兩個,推倒兩三個。打開幾層門,徑自出來,叫道:「兄弟們何在?」八戒、沙僧正舉著鈀、杖等待,道:「哥哥,如何了?」行者將變化入裡解放師父,正走,被妖驚覺,顧不得師父,打出來的事,講說一遍不題。

那妖王把唐僧捉住,依然使鐵索鎖了。執著刀,掄著斧,燈火齊明,問道:「你這廝怎樣開鎖?那猴子如何得進?快早供來,饒你之命;不然,就一刀兩段。」慌得那唐僧戰戰兢兢的跪道:「大王爺爺,我徒弟孫悟空,他會七十二般變化。才變個火焰蟲兒,飛進來救我。不期大王知覺,被小大王等撞見。是我徒弟不知好歹,打傷兩個,眾皆喊叫,舉兵著火,他遂顧不得我,走出去了。」三個妖王呵呵大笑道:「早是驚覺,未曾走了。」叫小的們把前後門緊緊關閉,亦不諠譁。

沙僧道:「閉門不諠譁,想是暗弄我師父。我們動手耶。」行者道:「說的是,快早打門。」那獃子賣弄神通,舉鈀盡力築去,把那石門築得粉碎,卻又厲聲喊罵道:「偷油的賊怪!快送吾師出來也。」諕得那門內小妖滾將進去,報道:「大王,不好了,不好了,前門被和尚打破了。」三個妖王十分煩惱道:「這廝著實無禮。」即命取披掛結束了,各持兵器,帥小妖出門迎敵。此時約有三更時候,半天中月明如晝。走出來,更不打話,便就掄兵。這裡行者抵住鉞斧,八戒敵住大刀,沙僧迎住大棍。這場好殺:
\begin{quote}
僧三眾,棍杖鈀,三個妖魔膽氣加。鉞斧鋼刀藤紇䌋,只聞風響並塵沙。初交幾合噴愁霧,次後飛騰散彩霞。釘鈀解數隨身滾,鐵棒英豪更可誇。降妖寶杖人間少,妖怪頑心不讓他。鉞斧口明尖鐏利,藤條節懞一身花。大刀晃亮如門扇,和尚神通偏賽他。這壁廂因師性命發狠打,那壁廂不放唐僧劈臉撾。斧剁棒迎爭勝負,鈀掄刀砍兩交搽。扢撻藤條降怪杖,翻翻覆覆逞豪華。
\end{quote}

三僧三怪賭鬥多時,不見輸贏。那辟寒大王喊一聲,叫:「小的們上來。」眾精各執兵刃齊來,早把個八戒絆倒在地,被幾個水牛精揪揪扯扯,拖入洞裡綑了。沙僧見沒了八戒,只見那群牛發喊聲。即掣寶杖,望辟塵大王虛丟了架子要走。又被群精一擁而來,拉一個躘踵,急掙不起,也被捉去綑了。行者覺道難為,縱觔斗雲,脫身而去。當時把八戒、沙僧拖至唐僧前。唐僧見了,滿眼垂淚道:「可憐你二人也遭了毒手。悟空何在?」沙僧道:「師兄見捉住我們,他就走了。」唐僧道:「他既走了,必然那裡去求救。但我等不知何日方得脫網。」師徒們悽悽慘慘不題。

卻說行者駕觔斗雲復至慈雲寺,寺僧接著,來問:「唐老爺救得否?」行者道:「難救,難救。那妖精神通廣大,我弟兄三個與他三人鬥了多時,被他呼小妖先捉了八戒,後捉了沙僧,老孫幸走脫了。」眾僧害怕道:「爺爺這般會騰雲駕霧,還捉獲不得,想老師父被傾害也。」行者道:「不妨,不妨。我師父自有伽藍、揭諦、丁甲等神暗中護佑,卻也曾吃過草還丹,料不傷命。只是那妖精有本事,汝等可好看馬匹、行李,等老孫上天去求救兵來。」眾僧膽怯道:「爺爺又能上天?」行者笑道:「天宮原是我的舊家。當年我做齊天大聖,因為亂了蟠桃會,被我佛收降。如今沒奈何,保唐僧取經,將功折罪,一路上輔正除邪。我師父該有此難,汝等卻不知也。」眾僧聽此言,又磕頭禮拜。行者出得門,打個唿哨,即時不見。

好大聖,早至西天門外。忽見太白金星與增長天王、殷、朱、陶、許四大靈官講話。他見行者來,都慌忙施禮道:「大聖那裡去?」行者道:「因保唐僧行至天竺國東界金平府旻天縣,我師被本縣慈雲寺僧留賞元宵。比至金燈橋,有金燈三盞,點燈用酥合香油,價貴白金五萬餘兩,年年有諸佛降祥受用。正看時,果有三尊佛像降臨。我師不識好歹,上橋就拜。我說不是好人,早被他侮暗燈光,連油並我師一風攝去。我隨風追襲,至天曉,到一山,幸四功曹報道:那山名青龍山,山有玄英洞。洞有三怪,名辟寒大王、辟暑大王、辟塵大王。老孫急上門尋討,與他賭鬥一陣,未勝。是我變化入裡,見師父鎖住未傷,隨解了欲出,又被他知覺,我遂走了。後又同八戒、沙僧苦戰,復被他將二人也捉去綑了。老孫因此特啟玉帝,查他來歷,請命將降之。」金星呵呵冷笑道:「大聖既與妖怪相持,豈看不出他的出處?」行者道:「認得,認得,是一夥牛精。只是他大有神通,急不能降也。」金星道:「那是三個犀牛之精。他因有天文之象,累年修悟成真,亦能飛雲步霧。其怪極愛乾淨,常嫌自己影身,每欲下水洗浴。他的名色也多:有兕犀,有雄犀,有牯犀,有斑犀,又有胡冒犀、墮羅犀、通天花文犀。都是一孔三毛二角,行於江海之中,能開水道。似那辟寒、辟暑、辟塵都是角有貴氣,故以此為名而稱大王也。若要拿他,只是四木禽星見面就伏。」行者連忙唱喏問道:「是那四木禽星?煩長庚老一一明示明示。」金星笑道:「此星在斗牛宮外,羅佈乾坤。你去奏聞玉帝,便見分明。」行者拱拱手稱謝,徑入天門裡去。

不一時,到於通明殿下,先見葛、丘、張、許四大天師。天師問道:「何往?」行者道:「近行至金平府地方,因我師寬放禪性,元夜觀燈,遇妖魔攝去。老孫不能收降,特來奏聞玉帝求救。」四天師即領行者至靈霄寶殿啟奏,各各禮畢,備言其事。玉帝傳旨:「教點那路天兵相助?」行者奏道:「老孫才到西天門,遇長庚星說:那怪是犀牛成精,惟四木禽星可以降伏。」玉帝即差許天師同行者去斗牛宮點四木禽星下界收降。

及至宮外,早有二十八宿星辰來接。天師道:「吾奉聖旨,教點四木禽星與孫大聖下界降妖。」傍即閃過角木蛟、斗木獬、奎木狼、井木犴應聲呼道:「孫大聖,點我等何處降妖?」行者笑道:「原來是你。這長庚老兒卻隱匿,我不解其意。早說是二十八宿中的四木,老孫徑來相請,又何必勞煩旨意?」四木道:「大聖說那裡話,我等不奉旨意,誰敢擅離?端的是那方?快早去來。」行者道:「在金平府東北艮地青龍山玄英洞,犀牛成精。」斗木獬、奎木狼、角木蛟道:「若果是犀牛成精,不須我們,只消井宿去罷,他能上山吃虎,下海擒犀。」行者道:「那犀不比望月之犀,乃是修行得道,都有千年之壽者。須得四位同去才好,切勿推調。倘一時一位拿他不住,卻不又費事了?」天師道:「你們說得是甚話?旨意著你四人,豈可不去?趁早飛行。我回旨去也。」那天師遂別行者而去。

四木道:「大聖不必遲疑,你先去索戰,引他出來,我們隨後動手。」行者即近前罵道:「偷油的賊怪!還我師來。」原來那門被八戒築破,幾個小妖弄了幾塊板兒搪住。在裡邊聽得罵詈,急跑進報道:「大王,孫和尚在外面罵哩。!」辟塵兒道:「他敗陣去了,這一日怎麼又來?想是那裡求些救兵來了。」辟寒、辟暑道:「怕他甚麼救兵?快取披掛來。小的們都要用心圍繞,休放他走了。」那夥精不知死活,一個個各執槍刀,搖旗擂鼓,走出洞來,對行者喝道:「你個不怕打的猢猻兒,你又來了。」行者最惱得是這「猢猻」二字,咬牙發狠,舉鐵棒就打。三個妖王調小妖,跑個圈子陣,把行者圈在垓心。那壁廂四木禽星一個個各掄兵刃道:「孽畜!休動手。」那三個妖王看他四星,自然害怕,俱道:「不好了,不好了,他尋將降手兒來了。小的們,各顧性命走耶。」只聽得呼呼吼吼,喘喘呵呵,眾小妖都現了本身,原來是那山牛精、水牛精、黃牛精,滿山亂跑。那三個妖王,也現了本相,放下手來,還是四隻蹄子,就如鐵炮一般,徑往東北上跑。這大聖帥井木犴、角木蛟緊追急趕,略不放鬆。惟有斗木獬、奎木狼在東山凹裡、山頭上、山澗中、山谷內,把些牛精打死的、活捉的,盡皆收淨。卻向玄英洞裡解了唐僧、八戒、沙僧。

沙僧認得是二星,隨同拜謝。因問:「二位如何到此相救?」二星道:「吾等是孫大聖奏玉帝請旨調來收怪救你也。」唐僧又滴淚道:「我悟空徒弟怎麼不見進來?」二星道:「那三個老怪是三隻犀牛,他見吾等,各各顧命,向東北艮方逃遁,孫大聖帥井木犴、角木蛟追趕去了。我二星掃蕩群牛到此,特來解放聖僧。」唐僧復又頓首拜謝,朝天又拜。八戒攙起道:「師父,禮多必詐,不須只管拜了。四星官,一則是玉帝聖旨,二則是師兄人情。今既掃蕩群妖,還不知老妖如何降伏。我們且收拾些細軟東西出來,掀翻此洞,以絕其根,回寺等候師兄罷。」奎木狼道:「天蓬元帥說得有理。你與捲簾大將保護你師回寺安歇,待吾等還去艮方迎敵。」八戒道:「正是,正是。你二位還協同一捉,必須剿盡,方好回旨。」二星官即時追襲。

八戒與沙僧將他洞內細軟寶貝(有許多珊瑚、瑪瑙、珍珠、琥珀、琚、寶貝、美玉、良金)搜出一石,搬在外面。請師父到山崖上坐了,他又進去放起火來,把一座洞燒成灰燼。卻才領唐僧找路回金平慈雲寺去。正是:
\begin{quote}
經云「泰極還生否」,好處逢凶實有之。
愛賞花燈禪性亂,喜遊美景道心漓。
大丹自古宜長守,一失原來到底虧。
緊閉牢拴休曠蕩,須臾懈怠見參差。
\end{quote}

且不言他三眾得命回寺。卻表斗木獬、奎木狼二星官駕雲直向東北艮方趕妖怪來,二人在那半空中尋看不見。直到西洋大海,遠望見孫大聖在海上吆喝。他兩個按落雲頭道:「大聖,妖怪那裡去了?」行者恨道:「你兩個怎麼不來追降?這會子卻冒冒失失的問甚?」斗木獬道:「我見大聖與井、角二星戰敗妖魔追趕,料必擒拿。我二人卻就掃蕩群精,入玄英洞救出你師父、師弟。搜了山,燒了洞,把你師父付托與你二弟,領回府城慈雲寺。多時不見車駕回轉,故又追尋到此也。」行者聞言,方才喜謝道:「如此,卻是有功,多累,多累。但那三個妖魔被我趕到此間,他就鑽下海去。當有井、角二星緊緊追拿,教老孫在岸邊抵擋。你兩個既來,且在岸邊把截,等老孫也再去來。」

好大聖,掄著棒,捻著訣,辟開水逕,直入波濤深處,只見那三個妖魔在水底下與井木犴、角木蛟捨死忘生苦鬥哩。他跳近前喊道:「老孫來也。」那妖精抵住二星官,措手不及,正在危難之處,忽聽得行者叫喊,顧殘生,撥轉頭往海心裡飛跑。原來這怪頭上角極能分水,只聞得花花花,沖開明路。這後邊二星官並孫大聖並力追之。

卻說西海中有個探海的夜叉、巡海的介士,遠見犀牛分開水勢,又認得孫大聖與二天星,即赴水晶宮對龍王慌慌張張報道:「大王,有三隻犀牛,被齊天大聖和二位天星趕來也。」老龍王敖順聽言,即喚太子摩昂:「快點水兵,想是犀牛精辟寒、辟暑、辟塵兒三個惹了孫行者,今既至海,快快拔刀相助。」敖摩昂得令,即忙點兵。頃刻間,龜鱉黿鼉、鯁鮊鱖鯉與蝦兵蟹卒等各執槍刀,一齊吶喊,騰出水晶宮外,擋住犀牛精。犀牛精不能前進,急退後,又有井、角二星並大聖攔阻。慌得他失了群,各各逃生,四散奔走,早把個辟塵兒被老龍王領兵圍住。孫大聖見了心歡,叫道:「消停,消停,捉活的,不要死的。」摩昂聽令,一擁上前,將辟塵兒扳翻在地,用鐵鉤子穿了鼻,攢蹄綑倒。

老龍王又傳號令,教分兵趕那兩個,協助二星官擒拿。即時小龍王帥眾前來,只見井木犴現原身,按住辟寒兒,大口小口的啃著吃哩。摩昂高叫道:「井宿,井宿,莫咬死他,孫大聖要活的,不要死的哩。」連喊數喊,已是被他把頸項咬斷了。

摩昂吩咐蝦兵蟹卒,將個死犀牛擡轉水晶宮,卻又與井木犴向前追趕。只見角木蛟把那辟暑兒倒趕回來,只撞著井宿。摩昂帥龜鱉黿鼉,撒開簸箕陣圍住。那怪只教:「饒命,饒命。」井木犴走近前,一把揪住耳朵,奪了他的刀,叫道:「不殺你,不殺你,拿與孫大聖發落去來。」

當即倒干戈,復至水晶宮外,報道:「都捉來也。」行者見一個斷了頭,血淋淋的,倒在地下。一個被井木犴揪著耳朵,推跪在地。近前仔細看了道:「這頭不是兵刀傷的啊。」摩昂笑道:「不是我喊得緊,連身子都著井星官吃了。」行者道:「既是如此,也罷,取鋸子來,鋸下他的這兩隻角,剝了皮帶去。犀牛肉還留與龍王賢父子享之。」又把辟塵兒穿了鼻,教角木蛟牽著;辟暑兒也穿了鼻,教井木犴牽著:「帶他上金平府見那刺史官,明究其由,問他個積年假佛害民,然後的決。」

眾等遵言,辭龍王父子,都出西海。牽著犀牛,會著奎、斗二星,駕雲霧,徑轉金平府。行者足踏祥光,半空中叫道:「金平府刺史、各佐貳郎官並府城內外軍民人等聽著:吾乃東土大唐差往西天取經的聖僧。你這府縣,每年家供獻金燈,假充諸佛降祥者,即此犀牛之怪。我等過此,因元夜觀燈,見這怪將燈油並我師父攝去,是我請天神收伏。今已掃清山洞,剿盡妖魔,不得為害。以後你府縣再不可供獻金燈,勞民傷財也。」

那慈雲寺裡,八戒、沙僧方保唐僧進得山門,只聽見行者在半空言語,即便撇了師父,丟下擔子,縱風雲起到空中,問行者降妖之事。行者道:「那一隻被井星咬死,已鋸角剝皮帶來;兩隻活拿在此。」八戒道:「這兩個索性推下此城,與官員人等看看,也認得我們是聖是神。左右累四位星官收雲下地,同到府堂,將這怪的決。已此情真罪當,再有甚講?」四星道:「天蓬帥近來知理明律,卻好呀。」八戒道:「因做了這幾年和尚,也略學得些兒。」眾神果推落犀牛,一簇彩雲,降至府堂之上。諕得這府縣官員、城裡城外人等,都家家設香案,戶戶拜天神。

少時間,慈雲寺僧把長老用轎擡進府門,會著行者,口中不離「謝」字道:「有勞上宿星官救出我等。因不見賢徒,懸懸在念,今幸得勝而回。然此怪不知趕向何方才捕獲也?」行者道:「自前日別了尊師,老孫上天查訪,蒙太白金星識得妖魔是犀牛,指示請四木禽星。當時奏聞玉帝,蒙旨差委,直至洞口交戰,妖王走了。又蒙斗、奎二宿救出尊師。老孫與井、角二宿並力追妖,直趕到西洋大海,又虧龍王遣子帥兵相助,所以捕獲到此審究也。」長老讚揚稱謝不已。又見那府縣正官並佐貳首領,都在那裡高燒寶燭,滿斗焚香,朝上禮拜。

少頃間,八戒發起性來,掣出戒刀,將辟塵兒頭一刀砍下,又一刀把辟暑兒頭也砍下。隨即取鋸子鋸下四隻角來。孫大聖更有主張:就教四位星官將此四隻犀角拿上界去,進貢玉帝,回繳聖旨。把自己帶來的二隻,留一隻在府堂鎮庫,以作向後免徵燈油之證;自己帶一隻去,獻靈山佛祖。四星心中大喜。即時拜別大聖,忽駕彩雲回奏而去。

府縣官留住他師徒四眾,大排素宴,遍請鄉官陪奉。一壁廂出給告示,曉諭軍民人等,下年不許點設金燈,永蠲買油大戶之役。一壁廂叫屠子宰剝,犀牛之皮硝熟燻乾,製造鎧甲;把肉普給官員人等。又一壁廂動支枉罰無礙錢糧,買民間空地,起建四星降妖之廟,又為唐僧四眾建立生祠,各各樹碑刻文,用傳千古,以為報謝。

師徒們索性寬懷領受。又被那二百四十家燈油大戶這家酬,那家請,略無虛刻。八戒遂心滿意受用,把洞裡搜來的寶物,每樣各籠些須在袖,以為各家齋筵之賞。住經個月,猶不得起身。長老吩咐:「悟空,將餘剩的寶物盡送慈雲寺僧,以為酬禮。瞞著那些大戶人家,天不明走罷。恐只管貪樂,誤了取經,惹佛祖見罪,又生災厄,深為不便。」行者隨將前件一一處分。

次日五更早起,喚八戒備馬。那獃子吃了自在酒飯,睡得夢夢乍道:「這早備馬怎的?」行者喝道:「師父教走路哩。」獃子抹抹臉道:「又是這長老沒正經。二百四十家大戶都請,才吃了有三十幾頓飽齋,怎麼又弄老豬忍餓?」長老聽言罵道:「饢糟的夯貨,莫胡說,快早起來;再若強嘴,教悟空拿金箍棒打牙。」那獃子聽見說打,慌了手腳道:「師父今番變了:常時疼我愛我,念我蠢夯護我,哥要打時,他又勸解;今日怎麼發狠轉教打麼?」行者道:「師父怪你為嘴,誤了路程。快早收拾行李,備馬,免打!」那獃子真個怕打,跳起來穿了衣服,吆喝沙僧:「快起來,打將來了。」沙僧也隨跳起,各各收拾皆完。長老搖手道:「寂寂悄悄的,不要驚動寺僧。」連忙上馬開了山門,找路而去。這一去,正所謂:
\begin{quote}
暗放玉籠飛彩鳳,私開金鎖走蛟龍。
\end{quote}

畢竟不知天明時,酬謝之家端的如何,且聽下回分解。
